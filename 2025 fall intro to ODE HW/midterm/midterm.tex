\documentclass[a4paper]{article}
%% Formatting %%
\usepackage[margin=3cm]{geometry}
\usepackage{type1cm, titlesec, fancyhdr, titling}
\usepackage{multicol}
\usepackage[dvipsnames]{xcolor}
\usepackage{ulem}
\usepackage{parskip}
\setlength{\parindent}{2em}
\setlength{\headheight}{15pt}
\setlength{\droptitle}{-1.5cm}
\parindent=24pt
%% Math and Symbols %%
\usepackage{amsmath,amsthm,amssymb, mathtools}
\usepackage{yhmath, faktor, dsfont}
\usepackage{academicons, wasysym, marvosym}
\usepackage[scr]{rsfso} 
\usepackage{latexsym, amsmath, amscd, amsmath, amsthm}
\usepackage{amssymb,amsmath,amsthm,graphicx,dsfont}
\usepackage{hyperref}

%% Enhancement %%
\usepackage{graphicx, tabularx}
\usepackage[shortlabels,inline]{enumitem}
%% TikZ %%
\usepackage{tikz-cd}
\usepackage[breakable]{tcolorbox}
\usetikzlibrary{decorations.pathmorphing}
\usetikzlibrary{calc, arrows,matrix}

%% Other packages %%
\usepackage{amsopn}

%% Traditional Chinese %%
\usepackage{CJKutf8}

%% Math environments %%
\newtheoremstyle{mystyle}
  {6pt}{15pt}% 上下間距
  {}%          內文字體
  {}%              縮排
  {\bf}%       標頭字體
  {.}%       標頭後標點
  {1em}% 內文與標頭距離
  {}% Theorem head spec (can be left empty, meaning 'normal')
\theoremstyle{mystyle}	
\newtheorem{theorem}{Theorem}
\newtheorem*{definition}{Definition}
\newtheorem{example}[theorem]{Example}
\newtheorem{exercise}{Exercise}
\newtheorem{solution}{Solution}
\newtheorem{corollary}[theorem]{Corollary}
\newtheorem{property}[theorem]{Property}
\newtheorem{proposition}[theorem]{Proposition}
\newtheorem{lemma}[theorem]{Lemma}
\newtheorem{problem}[theorem]{Problem}
\newtheorem{answer}{Answer}[section]
\newtheorem{fact}[theorem]{fact}
\newtheorem*{remark}{Remark}
\newtheorem*{claim}{Claim}
\newtheorem*{observation}{Observation}

% code environment
\usepackage{listings}
\usepackage{xcolor}
\lstset{
    language=MATLAB,
    basicstyle=\ttfamily,
    keywordstyle=\color{blue},
    commentstyle=\color{green},
    stringstyle=\color{cyan},
    showstringspaces=false
}

\begin{document}
\begin{CJK}{UTF8}{bkai}

\title{%
  \textbf{2025 Fall Introduction to ODE} \\
  \vspace{0.5cm}
  \large 
  Notes for Midterm\\
}
\author{物理三 黃紹凱 B12202004}
\date{\today}

\maketitle

\section{Chapter 1}
\subsection{Reduction of Order}
Given a second-order linear ODE 
\begin{equation}
    \frac{\mathrm{d}^2 y}{\mathrm{d}x^2} + p(x) \frac{\mathrm{d}y}{\mathrm{d}x} + q(x)y = 0,
\end{equation}
with one solution $ y(x) = u_1 (x) $ known, assume the second solution is of the form $ u_2(x) = u_1(x) U(x) $. Find $ U(x) $: 
\begin{equation}
    U (x) = \int^x \mathrm{d}t \, \frac{1}{u_1^2 (t)} \exp \left( -\int^t \mathrm{d}s\, p(s) \right).
\end{equation}

\subsection{Variation of Parameters}
For an initial value problem, the solution has the form 
\begin{equation*}
    y(x) = c_1 u_1 (x) + c_2 u_2 (x) + g (x),
\end{equation*}
where $ u_1 (x), u_2 (x) $ are the complementary functions, and $ g (x) $ is the particular integral. Find the particular integral given the complementary functions: 
\begin{equation}
    y(x) = \int^x \mathrm{d}s \, f(s) \frac{\begin{vmatrix} u_1 (s) & u_2 (s) \\ u_1 (x) & u_2 (x) \end{vmatrix}}{\begin{vmatrix} u_1 (s) & u_2 (s) \\ u_1^{\prime} (s) & u_2^{\prime} (s) \end{vmatrix}} = \int^x \mathrm{d}s\, f(s) \frac{u_1 (s) u_2 (x) - u_2 (s) u_1 (x)}{W(s)}, 
\end{equation}
where $ W(s) $ is the \textbf{Wronskian}. 

Find Wronskian without constructing the solutions: \textbf{Abel's formula}. 
\begin{equation}
    W(x) = W(x_0) \exp \left( -\int_{x_0}^x \mathrm{d}t\, p(t) \right).
\end{equation}

\subsection{Method of Frobenius}
Indicial equation: 
\begin{enumerate}
    \item Roots differ by integer and one of the coefficients is indeterminate when $ c = \alpha $: Obtain both solutions with $ c = \alpha $.
    \item Roots differ by non-integer: Obtain two solutions with $ c = \alpha $ and $ c = \beta $.
    \item Double root: Obtain one solution with $ c = \alpha $, and obtain the second with $ (\partial y / \partial c)\vert_{c = \alpha} $. 
\end{enumerate}

\newpage 

%
\section{Chapter 2}

\subsection{Legendre's Differential Equation}
\begin{equation}
    (1 - x^2) y^{\prime\prime} - 2xy^{\prime} + n(n+1)y = 0, \quad n = 0, 1, 2, \ldots .
\end{equation}
The general solution is 
\begin{equation}
    y(x) = A P_n(x) + B Q_n (x),
\end{equation}
where $ P_n(x) $ is a degree $ n $ polynomial, and $ Q_n(x) $ is an infinite series converging for $ \vert x \vert < 1 $.

\subsection{Generating Function}
\begin{equation}
    g(x,t) = \frac{1}{\sqrt{1 - 2xt + t^2}} = \sum^{\infty}_{n=0} P_n (x) t^n, \quad \vert t \vert < 1.
\end{equation}
Then we have \textbf{Bonnet's formula} and a recurrence relation for derivatives:
\begin{equation}
    (n+1) P_{n+1} (x) - (2n+1) x P_n (x) + n P_{n-1} (x) = 0,
\end{equation}
\begin{equation}
    n P_n (x) = x P^{\prime} _n (x) - P^{\prime}_{n-1} (x).
\end{equation}

\subsection{Rodrigue's Formula}

\textbf{Rodrigue's formula} gives an explicit expression for $ P_n (x) $:
\begin{equation}
    P_n (x) = \frac{1}{2^n n!} \frac{\mathrm{d}^n}{\mathrm{d}x^n} (x^2 - 1)^n.
\end{equation}
From this we have \textbf{Schl\"{a}fli's representation}:
\begin{equation}
    P_n (z) = \frac{1}{2^{n+1} \pi i} \oint_{\mathcal{C}} \mathrm{d}\xi \, \frac{(\xi^2 - 1)^n}{(\xi -z)^{n+1}},
\end{equation}
where $ \mathcal{C} $ is a simple closed contour enclosing $ z $. Then we have \textbf{Laplace's representation}:
\begin{equation}
    P_n (z) = \frac{1}{2\pi} \int^{\pi}_0 \mathrm{d}\theta \, \left( z + \sqrt{z^2 - 1} \cos \theta \right)^n.
\end{equation}

\subsection{Orthogonality}
Legendre polynomials are orthogonal on the interval $ [-1, 1] $:
\begin{equation}
    \int^1_{-1} \mathrm{d}x \, P_n (x) P_m (x) = \frac{2}{2n + 1} \delta_{nm}.
\end{equation}
They form a complete basis for piecewise continuous functions on $ [-1, 1] $, so we have the \textbf{Fourier-Legendre series}:
\begin{equation}
    f(x) = \sum^{\infty}_{n=0} a_n P_n (x), \quad a_n = \frac{2n + 1}{2} \int^1_{-1} \mathrm{d}t \, f(t) P_n (t).
\end{equation}

\subsection{Associated Legendre Equation}
The associated Legendre equation is
\begin{equation}
    (1 - x^2) y^{\prime\prime} - 2xy^{\prime} + \left[ n(n+1) - \frac{m^2}{1-x^2} \right] y = 0, \quad n = 0, 1, 2, \ldots, \quad m = 0, 1, 2, \ldots, n.
\end{equation}
The general solution is
\begin{equation}
    y(x) = A P^m_n (x) + B Q^m_n (x),
\end{equation}
where $ P^m_n(x), Q^m_n(x) $ are the associated Legendre functions, $ P^m_n(x) = 0 $ when $ m>n $, and $ Q^m_n (x) $ is singular at $ x = \pm 1 $. We have 
\begin{equation}
    P^m_n (x) = (1 - x^2)^{m/2} \frac{\mathrm{d}^m}{\mathrm{d}x^m} P_n (x).
\end{equation}

The \textbf{spherical harmonics} that solve the angular part of the spherical Laplace equation are defined as
\begin{equation}
    Y^m_n (\theta, \phi) = (-1)^m \sqrt{\frac{2n + 1}{4\pi} \frac{(n-m)!}{(n+m)!}} P^m_n (\cos \theta) e^{im\phi}, \quad n = 0, 1, 2, \ldots, \quad m = -n, -n+1, \ldots, n.
\end{equation}

\newpage 

%
\section{Chapter 3}
\subsection{The Bessel Equation}
The Bessel equation is
\begin{equation}
    x^2 y^{\prime\prime} + xy^{\prime} + (x^2 - \nu^2) y = 0.
\end{equation}
When $ \nu $ is not a half-integer, the general solution is
\begin{equation}
    y(x) = A J_{\nu} (x) + B J_{-\nu} (x),
\end{equation}
where 
\begin{equation}
    J_{\pm \nu} = \frac{x^{\pm \nu}}{2^{\pm \nu} \Gamma (1 \pm \nu)} \sum_{n=0}^{\infty} (-1)^n \frac{(x/2)^{2n}}{n! (n + 1 \pm \nu)_n}
\end{equation}
and $ (\alpha)_r = \alpha (\alpha + 1) \cdots (\alpha + r - 1) $ is the \textbf{Pochhammer symbol}.

When $ \nu =0 $, the other solution is \textbf{Weber's Bessel function} of order zero: 
\begin{equation}
    Y_0(x) = J_0 (x) \log x - \sum_{n=0}^{\infty} (-1)^n \frac{H_n}{(n!)^2} \left(\frac{x}{2}\right)^{2n}, 
\end{equation}
where $ H_n $ is the $ n $-th \textbf{harmonic number}. When $ \nu $ is not a positive integer, the other solution is
\begin{equation}
    Y_{\nu} (x) = \frac{J_{\nu} (x) \cos (\nu \pi) - J_{-\nu}(x)}{\sin(\nu \pi)}.
\end{equation}

\subsection{Generating Function}
For integer $ n $, the Bessel function is 
\begin{equation}
    J_n (x) = \sum_{k=0}^{\infty} \frac{(-1)^k x^{2k + n}}{2^{2k+n} k! (k+n)!}.
\end{equation}
The generating function is
\begin{equation}
    e^{\frac{x}{2} (t - 1/t)} = \sum_{n=-\infty}^{\infty} J_n (x) t^n.
\end{equation}
We have the alternative integral representation
\begin{equation}
    J_n (x) = \frac{1}{\pi} \int^{\pi}_{0} \mathrm{d}\theta \, \cos \left(n \theta - x \sin \theta \right).
\end{equation}

\subsection{Recurrence Relations}
The Bessel functions and their derivatives satisfy the recurrence relations
\begin{equation}
    \frac{2n}{x} J_n (x) = J_{n-1} (x) + J_{n+1} (x),
\end{equation}
and 
\begin{equation}
    \frac{\mathrm{d}}{\mathrm{d}x} \left(x^{\nu} N_{\nu} (x) \right) = x^{\nu} N_{\nu -1}(x), \quad \frac{\mathrm{d}}{\mathrm{d}x} \left( x^{-\nu} N_{\nu} (x) \right) = - x^{-\nu} N_{\nu + 1} (x).
\end{equation}
where $ N_n $ denotes either $ J_n $ or $ Y_n $.

\subsection{Orthogonality}
The Bessel functions are orthogonal on the interval $ [0, 1] $ with respect to the weight $ x $:
\begin{equation}
    \int^a_0 \mathrm{d}x \, x J_{\nu} (\lambda_{m} x) J_{\nu} (\lambda_{n} x) = \frac{a^2 \delta_{mn}}{2} [J_{\nu + 1} (\lambda_{m})]^2,
\end{equation}
so we have the \textbf{Fourier-Bessel series}:
\begin{equation}
    f(x) = \sum^{\infty}_{n=1} a_n J_{\nu} (\lambda_n x), \quad a_n = \frac{2}{a^2 [J_{\nu + 1} (\lambda_n)]^2} \int^a_0 \mathrm{d}t \, t f(t) J_{\nu} (\lambda_n t).
\end{equation}

\subsection{Solutions Expressible as Bessel Functions}
The modified Bessel equation is
\begin{equation}
    x^2 y^{\prime\prime} + xy^{\prime} - (x^2 + \nu^2) y = 0,
\end{equation}
with solutions
\begin{equation}
    I_{\nu} (x) = i^{-\nu} J_{\nu} (ix), \quad K_{\nu} (x) = \frac{\pi}{2} i^{\nu + 1} H^{(1)}_{\nu} (ix),
\end{equation}
where $ H^{(1)}_{\nu} (x) = J_{\nu} (x) + i Y_{\nu} (x) $ is the \textbf{Hankel function of the first kind}.

The transformation 

\newpage 

% %
% \section{Chapter 4}
% \subsection{Green's Function}

% \subsection{Sturm-Liouville Problem}

% \subsection{Self-Adjoint}

% \newpage 

%
\section{Chapter 6}
\subsection{Integral Transforms}
Integral transforms are of the form
\begin{equation}
    F(s) = \int^{\infty}_0 \mathrm{d}t \, K(s,t) f(t),
\end{equation}
where \( K(s,t) \) is the kernel of the transform. The Fourier transform, Laplace transform, Mellin transform, and Hankel transform are all examples of integral transforms.

\subsection{Laplace Transforms}
Definition of a Laplace transform:
\begin{equation}
    F(s) = \mathcal{L} [f(t)] = \int^{\infty}_0 \mathrm{d}t \, e^{-st} f(t).
\end{equation}

\begin{theorem}[Lerch]
    A function $ f(t) $ is said to be of exponential order on $ [0,\infty) $ if there exist constants $ M, c, T > 0 $ such that
    \begin{equation*}
        \vert f(t) \vert \leq M e^{ct}, \quad t > T.
    \end{equation*}
    If $ f(t) $ is piecewise continuous on every finite interval in $ [0,\infty) $ and of exponential order, then the Laplace transform $ F(s) = \mathcal{L} [f(t)] $ exists for $ s > c $.
\end{theorem}

\begin{theorem}[Existence of Laplace Transform]
    If $ f $ is of exponential order on $ [0,\infty) $, then $ \mathcal{L}[f] \to 0 $ as $ s \to \infty $.
\end{theorem}

A useful Laplace transform identity for calculating integrals is 
\begin{equation}
    \int_0^{\infty} \mathrm{d}x \, f(x) g(x) = \int^{\infty}_0 \mathrm{d}x\,(\mathcal{L}f)(x) (\mathcal{L}^{-1}g) (x) = \int^{\infty}_0 \mathrm{d}x\,(\mathcal{L}^{-1}f)(x) (\mathcal{L}g) (x) 
\end{equation}

\subsection{Convolution Theorem}

\subsection{Inverse Laplace Transform}
The inverse Laplace transform can be found using the \textbf{Bromwich integral}:
\begin{equation}
    g(t) = \mathcal{L}^{-1} [G(s)] = \frac{1}{2\pi i} \int^{\gamma + i\infty}_{\gamma - i\infty} \mathrm{d}s \, e^{st} G(s),
\end{equation}
where $ \gamma $ is the smallest real number such that $ e^{-\gamma t} g(t)$ is bounded as $ t \to  \infty $. 

\end{CJK}
\end{document}