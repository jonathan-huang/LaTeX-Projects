\documentclass[a4paper]{article}
%% Formatting %%
\usepackage[margin=3cm]{geometry}
\usepackage{type1cm, titlesec, fancyhdr, titling}
\usepackage{multicol}
\usepackage[dvipsnames]{xcolor}
\usepackage{ulem}
\usepackage{parskip}
\setlength{\parindent}{2em}
\setlength{\headheight}{15pt}
\setlength{\droptitle}{-1.5cm}
\parindent=24pt
%% Math and Symbols %%
\usepackage{amsmath,amsthm,amssymb, mathtools}
\usepackage{yhmath, faktor, dsfont}
\usepackage{academicons, wasysym, marvosym}
\usepackage[scr]{rsfso} 
\usepackage{latexsym, amsmath, amscd, amsmath, amsthm}
\usepackage{amssymb,amsmath,amsthm,graphicx,dsfont}
\usepackage{hyperref}

%% Enhancement %%
\usepackage{graphicx, tabularx}
\usepackage[shortlabels,inline]{enumitem}
%% TikZ %%
\usepackage{tikz-cd}
\usepackage[breakable]{tcolorbox}
\usetikzlibrary{decorations.pathmorphing}
\usetikzlibrary{calc, arrows,matrix}

%% Other packages %%
\usepackage{amsopn}

%% Traditional Chinese %%
\usepackage{CJKutf8}

%% Math environments %%
\newtheoremstyle{mystyle}
  {6pt}{15pt}% 上下間距
  {}%          內文字體
  {}%              縮排
  {\bf}%       標頭字體
  {.}%       標頭後標點
  {1em}% 內文與標頭距離
  {}% Theorem head spec (can be left empty, meaning 'normal')
\theoremstyle{mystyle}	
\newtheorem{theorem}{Theorem}
\newtheorem*{definition}{Definition}
\newtheorem{example}[theorem]{Example}
\newtheorem{exercise}{Exercise}
\newtheorem{solution}{Solution}
\newtheorem{corollary}[theorem]{Corollary}
\newtheorem{property}[theorem]{Property}
\newtheorem{proposition}[theorem]{Proposition}
\newtheorem{lemma}[theorem]{Lemma}
\newtheorem{problem}[theorem]{Problem}
\newtheorem{answer}{Answer}[section]
\newtheorem{fact}[theorem]{fact}
\newtheorem*{remark}{Remark}
\newtheorem*{claim}{Claim}
\newtheorem*{observation}{Observation}

% code environment
\usepackage{listings}
\usepackage{xcolor}
\lstset{
    language=MATLAB,
    basicstyle=\ttfamily,
    keywordstyle=\color{blue},
    commentstyle=\color{green},
    stringstyle=\color{cyan},
    showstringspaces=false
}

\begin{document}
\begin{CJK}{UTF8}{bkai}

\title{%
  \textbf{2025 Fall Introduction to ODE} \\
  \vspace{0.5cm}
  \large 
  Homework 5 (Due Oct 6 12:00, 2025)\\
}
\author{物理/數學三 黃紹凱 B12202004}
\date{\today}

\maketitle

% Problem 1
\begin{problem}
        Using Laplace transforms, solve the initial value problem
    \begin{align*}
        \frac{dx}{dt}+x+\frac{dy}{dt}=0,\quad \frac{dx}{dt}-x+2\frac{dy}{dt}=e^{-t}, \quad \text{subject to } x(0)=y(0)=1.
    \end{align*}
\end{problem}
\begin{solution}
    ~

    \underline{Steps:}
    \begin{enumerate}
        \item Take the Laplace transform of both equations.
        \item Solve the resulting algebraic equations for \(X(s)\) and \(Y(s)\).
        \item Find the inverse Laplace transforms to get \(x(t)\) and \(y(t)\).
    \end{enumerate}
    
    \underline{Method:}
    Recall that the Laplace transform of derivatives is given by
    \begin{equation}
        \label{eq:laplace_derivative}
        \mathcal{L} [f^{\prime}] = sF(s) - f(0), \quad \mathcal{L} [f^{\prime\prime}] = s^2 F(s) - sf(0) - f^{\prime}(0).
    \end{equation}
    Write the Laplace transform of $ x(t) $ and $ y(t) $ as $ X(s) $, $ Y(s) $, respectively. Then we have
    \begin{align*}
        sX(s) - x(0) + X(s) + sY(s) - y(0) &= 0, \\
        sX(s) - x(0) - X(s) + 2sY(s) - 2y(0) &= \frac{1}{s+1}.
    \end{align*}
    Substitute the initial conditions \( x(0) = y(0) = 1 \) into the equations to get
    \begin{align*}
        (s+1)X(s) + sY(s) &= 2, \\
        (s-1)X(s) + 2sY(s) &= \frac{1}{s+1} + 3,
    \end{align*}
    so 
    \begin{equation*}
        \begin{split}
            X(s) &= \frac{s}{(s+1)(s+3)} = \frac{3}{2(s+3)} - \frac{1}{2(s+1)}, \\
            Y(s) &= \frac{s+6}{s(s+3)} = -\frac{1}{s+3} + \frac{2}{s}.
        \end{split}
    \end{equation*}
    Therefore, the inverse Laplace transforms are given by
    \begin{align*}
        x(t) &= \frac{3}{2} e^{-3t} - \frac{1}{2} e^{-t}, \\
        y(t) &= -e^{-3t} + 2.
    \end{align*}
\end{solution}

% Problem 2
\begin{problem}
    Show that \(\mathcal{L}[tf(t)]=-\frac{dF}{ds}\),
    where \(F(s)=\mathcal{L}[f(t)]\). Hence solve the initial value
    problem
    \begin{align*}
        \frac{d^2x}{dt^2}+2t\frac{dx}{dt}-4x=1, \quad \text{subject to }x(0)=x'(0)=0.
    \end{align*}
\end{problem}

\begin{solution}
    ~

    \underline{Steps:}
    \begin{enumerate}
        \item We prove the desired property with integration by parts.
        \item Transform the differential equation using the Laplace transform into a first-order ODE. 
        \item Solve the first-order ODE for \(X(s)\).
        \item Find the inverse Laplace transform to get \(x(t)\).
    \end{enumerate}
    
    \underline{Method:}
    By definition of the Laplace transform and Leibniz' rule of differentiating under the integral sign, we have
    \begin{equation}
        \label{eq:laplace_tf}
        \begin{split}
            \frac{\mathrm{d}F}{\mathrm{d}s} &= \frac{\mathrm{d}}{\mathrm{d}s} \int^{\infty}_0 \mathrm{d}t \, e^{-st} f(t) \\
            &= \int^{\infty}_0 \mathrm{d}t \, \frac{\partial}{\partial s}\left(e^{-st} f(t)\right) \\
            &= - \int^{\infty}_0 \mathrm{d}t\, e^{-st} tf(t) \\
            &= \mathcal{L} [tf(t)].
        \end{split}
    \end{equation}
    Taking the Laplace transform of both sides of the differential equation using \ref{eq:laplace_derivative} and \ref{eq:laplace_tf}, we have
    \begin{equation*}
        s^2 X(s) - sx(0) - x'(0) - 2 \frac{\mathrm{d}}{\mathrm{d}s}\left(sX(s) - x(0)\right) - 4X(s) = \frac{1}{s},
    \end{equation*}
    \begin{equation*}
        \Longrightarrow s^2 X(s) - 2s X^{\prime} (s) - 6 X(s) = \frac{1}{s}.
    \end{equation*}
    Rearranging gives
    \begin{equation*}
        X^{\prime} (s) + \frac{6 - s^2}{2s} X(s) = - \frac{1}{2s^2} .
    \end{equation*}
    This is a first-order linear ODE. The integrating factor is given by
    \begin{equation*}
        \mu (s) = e^{\int \frac{6-s^2}{2s} \mathrm{d}s} = s^3 e^{-\sqrt{s /4}}.
    \end{equation*}
    Therefore, we have
    \begin{equation*}
        X(s) \mu (s) = - \frac{1}{2} \int^s_0 \mathrm{d}s^{\prime} \, \frac{1}{(s')^{2}} (s^{\prime})^3 e^{-(s')^2/4} = e^{-(s^{\prime})^{2} /4} - 1,
    \end{equation*}
    where we used a substitution \( u = (s')^2 / 4 \) to evaluate the integral. Thus, we have
    \begin{equation*}
        X(s) = \frac{e^{-s^2/4} - 1}{s^3 e^{-s^2/4}} = \frac{1 - e^{s^2/4}}{s^3}.
    \end{equation*}
    The inverse Laplace transform can be found by noting that
    \begin{equation*}
        x(t) = \mathcal{L}^{-1} [X(s)] = \mathcal{L}^{-1} \left[\frac{1}{s^3}\right] - \mathcal{L}^{-1} \left[\frac{e^{s^2/4}}{s^3}\right].
    \end{equation*}
    The first term is given by
    \begin{equation*}
        \mathcal{L}^{-1} \left[\frac{1}{s^3}\right] = \frac{t^2}{2},
    \end{equation*}
    while the second term tends to infinity as \( s \to \infty \) and thus does not have a well-defined inverse Laplace transform. Therefore, the solution to the initial value problem is simply 
    \begin{equation*}
        x(t) = \frac{t^2}{2}.
    \end{equation*}
\end{solution}

% Problem 3
\begin{problem}
    Find the inverse Laplace transform of
    \begin{equation*}
        F(s)=\frac{1}{(s-1)(s^2+1)},
    \end{equation*}
    by (a) expressing \(F(s)\) as partial fractions and inverting the
    constituent parts, and (b) using the convolution theorem.\\
\end{problem}

\begin{solution}
    ~

    \underline{Steps:} 
    \begin{enumerate}
        \item Express \(F(s)\) as partial fractions:
        \item Invert the constituent parts using known Laplace transforms.
        \item Use the convolution theorem to directly find the inverse Laplace transform.
    \end{enumerate}
    \underline{Method:}
    \begin{enumerate}[(a)]
        \item We can write 
        \[
        F(s) = \frac{A}{s-1} + \frac{Bs+C}{s^2+1} = \frac{(A+B)s^2 + (C-B)s + (A-C)}{(s-1)(s^2+1)}.
        \]
        and solve for $ A, B, C $. This gives 
        \[ 
        A = \frac{1}{2}, \quad B = C = -\frac{1}{2}.
        \]
        Therefore, we have
        \[
        F(s) = \frac{1/2}{s-1} - \frac{1/2 s}{s^2+1} - \frac{1/2}{s^2+1}.
        \]
        Apply the inverse Laplace transform to each term to get 
        \[
        \mathcal{L}^{-1}[F(s)] = \frac{1}{2} e^t - \frac{1}{2} \sin t - \frac{1}{2} \cos t = \frac{1}{2} \left( e^t - \sin t - \cos t \right).
        \]
        \item Next, we use the convolution theorem to directly find the inverse Laplace transform. The convolution theorem states that given three functions $ f(t), g(t), h(t) $ and their respective Laplace transforms $ F(s), G(s), H(s) $ satisfying \( F(s) = G(s)H(s) \), then
        \[
        \mathcal{L}^{-1}[F(s)] = \int_0^t g(t-\tau)h(\tau)d\tau.
        \]
        Equivalently, we can write it with the convolution operation:
         \[
        f(t) = \left(g \ast h\right)(t).
        \]
        Let \( G(s) = (s-1)^{-1} \) and \( H(s) = (s^2+1)^{-1} \). Then we have $ g(t) = e^t $ and $ h(t) = \sin(t) $. Therefore, we can write via the convolution theorem that
        \[ 
        f(t) = \int_0^t g(t-\tau)h(\tau)d\tau = \int_0^t \mathrm{d}\tau \, e^{t-\tau} \sin(\tau).
        \]
        The integral can be evaluated using integration by parts, yielding
        \[
        f(t) = \int_0^t e^{t-\tau} \sin(\tau) d\tau = - e^{-\tau}\left(\sin \tau + \cos \tau \right) \bigg|_0^t - f(t),
        \]
        so 
        \[
        f(t) = \frac{1}{2} \left( e^t - \sin t - \cos t \right).
        \]
    \end{enumerate}
\end{solution}

\end{CJK}
\end{document}