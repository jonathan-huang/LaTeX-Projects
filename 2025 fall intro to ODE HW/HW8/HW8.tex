\documentclass[a4paper]{article}
%% Formatting %%
\usepackage[margin=3cm]{geometry}
\usepackage{type1cm, titlesec, fancyhdr, titling}
\usepackage{multicol}
\usepackage[dvipsnames]{xcolor}
\usepackage{ulem}
\usepackage{parskip}
\setlength{\parindent}{2em}
\setlength{\headheight}{15pt}
\setlength{\droptitle}{-1.5cm}
\parindent=24pt
%% Math and Symbols %%
\usepackage{amsmath,amsthm,amssymb, mathtools}
\usepackage{yhmath, faktor, dsfont}
\usepackage{academicons, wasysym, marvosym}
\usepackage[scr]{rsfso} 
\usepackage{latexsym, amsmath, amscd, amsmath, amsthm}
\usepackage{amssymb,amsmath,amsthm,graphicx,dsfont}
\usepackage{hyperref}

%% Enhancement %%
\usepackage{graphicx, tabularx}
\usepackage[shortlabels,inline]{enumitem}
%% TikZ %%
\usepackage{tikz-cd}
\usepackage[breakable]{tcolorbox}
\usetikzlibrary{decorations.pathmorphing}
\usetikzlibrary{calc, arrows,matrix}

%% Other packages %%
\usepackage{amsopn}

%% Traditional Chinese %%
\usepackage{CJKutf8}

%% Math environments %%
\newtheoremstyle{mystyle}
  {6pt}{15pt}% 上下間距
  {}%          內文字體
  {}%              縮排
  {\bf}%       標頭字體
  {.}%       標頭後標點
  {1em}% 內文與標頭距離
  {}% Theorem head spec (can be left empty, meaning 'normal')
\theoremstyle{mystyle}	
\newtheorem{theorem}{Theorem}
\newtheorem*{definition}{Definition}
\newtheorem{example}[theorem]{Example}
\newtheorem{exercise}{Exercise}
\newtheorem{solution}{Solution}
\newtheorem{corollary}[theorem]{Corollary}
\newtheorem{property}[theorem]{Property}
\newtheorem{proposition}[theorem]{Proposition}
\newtheorem{lemma}[theorem]{Lemma}
\newtheorem{problem}[theorem]{Problem}
\newtheorem{answer}{Answer}[section]
\newtheorem{fact}[theorem]{fact}
\newtheorem*{remark}{Remark}
\newtheorem*{claim}{Claim}
\newtheorem*{observation}{Observation}

% code environment
\usepackage{listings}
\usepackage{xcolor}
\lstset{
    language=MATLAB,
    basicstyle=\ttfamily,
    keywordstyle=\color{blue},
    commentstyle=\color{green},
    stringstyle=\color{cyan},
    showstringspaces=false
}

\begin{document}
\begin{CJK}{UTF8}{bkai}

\title{%
  \textbf{2025 Fall Introduction to ODE} \\
  \vspace{0.5cm}
  \large 
  Homework 8 (Due November 10 12:00, 2025)\\
}
\author{物理三 黃紹凱 B12202004}
\date{\today}

\maketitle

% Exercise 1
\begin{exercise}
    Suppose \(a,b,c\) are nonnegative continuous functions on \([0,\infty)\), \(u\) is a nonnegative bounded continuous solution of the inequality
    \begin{align*}
        u(t) \leq a(t)+\int_0^tb(t-s)u(s)\,ds+\int_0^{\infty}c(s)u(t+s)\,ds, \quad t \geq 0,
    \end{align*}
    and \(a(t) \to 0\), \(b(t) \to 0\) as \(t \to \infty\), \(\int_0^{\infty}b(s)\,ds <\infty\), \(\int_0^{\infty}c(s)\,ds<\infty\). Prove that \(u(t)\to0\) as \(t \to \infty\) if
    \begin{align*}
        \int_0^{\infty}b(s)\,ds+\int_0^{\infty}c(s)\,ds<1.
    \end{align*}    
\end{exercise}

\begin{solution}
    ~

    \underline{Steps:}
    \begin{enumerate}
        \item Choose a sequence that approaches the limsup of \(u(t)\) as \(t \to \infty\).
        \item Split the integral appropriately and estimate the terms. 
        \item Combine everything and show that the limsup must be zero.
    \end{enumerate}
    
    \underline{Method:}
    ~ 

    \begin{enumerate}
        \item Let $ \{t_n\}_{n \in \mathbb{N}} $ be a sequence such that $ t_n \to \infty $ as $ n \to \infty $ and $ u(t_n) \to L = \limsup_{t \to \infty} u(t) $. By a change of variables, we have 
        \[
            u(t) \leq a(t) + \int_0^t \mathrm{d}r\, b(r) u(t-r) + \int_0^{\infty} \mathrm{d}s\, c(s) u(t+s).
        \]
        \item Since $ b(t) \geq 0 $ and $ \int^{\infty}_0 \mathrm{d}s\, b(s)<\infty $, we have $ \lim_{R \to \infty} \int_R^{\infty} \mathrm{d}s\, b(s) = 0 $. This holds similarly for $ c(t) $, so there exist $ R_b, R_c > 0 $ such that 
        \[
            \int_{R_b}^{\infty} \mathrm{d}s\, b(s) < \varepsilon, \quad \int_{R_c}^{\infty} \mathrm{d}s\, c(s) < \varepsilon.
        \]
        By the definition of $ L $, there exists $ T_\varepsilon > 0 $ such that $ u(t) \leq L + \varepsilon $ for all $ t \geq T_\varepsilon $. Let's take $ T = T_\varepsilon + \max \{R_b, R_c\} $, and $ n $ sufficiently large such that $ t_n \geq T $.
        \item Now we will estimate each term at $ t = t_n $: First, since $ a(t) \to 0 $ as $ t \to \infty $, there exists $ N_1 \in \mathbb{N} $ such that for all $ n \geq N_1 $, we have $ a(t_n) < \varepsilon $. Next, we split the first integral:
        \[
            \int_0^{t_n} \mathrm{d}r\, b(r) u(t_n - r) = \int_0^{R_b} \mathrm{d}r\, b(r) u(t_n - r) + \int_{R_b}^{t_n} \mathrm{d}r\, b(r) u(t_n - r). 
        \]
        The first term is bounded by $ (L + \varepsilon) \int_0^{R_b} \mathrm{d}r\, b(r) $, and the second term is bounded by $ M \int_{R_b}^{\infty} \mathrm{d}r\, b(r) < M \varepsilon $, where $ M = \sup_{t \geq 0} u(t) $. Similarly, we split the second integral: 
        \[
            \int_0^{\infty} \mathrm{d}s\, c(s) u(t_n + s) = \int_0^{R_c} \mathrm{d}s\, c(s) u(t_n + s) + \int_{R_c}^{\infty} \mathrm{d}s\, c(s) u(t_n + s).
        \]
        The first term is bounded by $ (L + \varepsilon) \int_0^{R_c} \mathrm{d}s\, c(s) $, and the second term is bounded by $ M \int_{R_c}^{\infty} \mathrm{d}s\, c(s) < M \varepsilon $. Combining everything, we have for all $ n $ sufficiently large,
        \begin{align*}
            u(t_n) &\leq a(t_n) + \int_0^{t_n} \mathrm{d}r\, b(r) u(t_n - r) + \int_0^{\infty} \mathrm{d}s\, c(s) u(t_n + s) \\
            &= \varepsilon + (L + \varepsilon) \int_0^{R_b} \mathrm{d}r\, b(r) + M \varepsilon + (L + \varepsilon) \int_0^{R_c} \mathrm{d}s\, c(s) + M \varepsilon \\
            &= (L + \varepsilon) \left( \int_0^{R_b} \mathrm{d}r\, b(r) + \int_0^{R_c} \mathrm{d}s\, c(s) \right) + \varepsilon (1 + 2M).
        \end{align*}
        Rearranging gives
        \[
            L \leq \frac{1 + B + C + 2M}{1- (B+C)} \varepsilon,
        \]
        where $ B = \int_0^{\infty} \mathrm{d}r\, b(r) $ and $ C = \int_0^{\infty} \mathrm{d}s\, c(s) $. Since $ \varepsilon > 0 $ is arbitrary, we conclude that $ L \leq \limsup_{t \to \infty} u(t) = 0 $. Since $ u(t) \geq 0 $, we have $ \lim_{t \to \infty} u(t) = 0 $.
    \end{enumerate}
\end{solution}

% Exercise 2
\begin{exercise}
    For any real matrix \(D\), \(\det D \neq 0\), show there is a real matrix \(B\) such that \(e^B=D^2\). If \(C\) is a real matrix in Lemma 7.1 and there is a real matrix \(B\) such that \(e^B=C\), must there exist a real matrix \(D\) such that \(C=D^2\)\,?
\end{exercise}

\begin{solution}
     ~

    \underline{Steps:}
    \begin{enumerate}
        \item Write down the real Jordan form of $ D $. 
        \item Prove the existence of logarithm of $ D^2 $, and hence the existence of $ B $ such that $ e^B = D^2 $.
        \item Discuss whether there must exist a real matrix \(D\) such that \(C=D^2\).
    \end{enumerate}
    
    \underline{Method:}
    ~ 

    \begin{enumerate}
        \item Let $ D \in M_{n}(\mathbb{R}) $ be a real invertible matrix. Let $ \lambda $ be an eigenvalue of $ D $ with eigenvector $ v $, so $ Dv = \lambda v $. Then, we have
        \[
            D^2 v = D (\lambda v) = \lambda D v = \lambda^2 v.
        \]
        Therefore, the eigenvalues of $ D^2 $ are $ \lambda_i^2 > 0$, where $  \lambda_i $ are the eigenvalues of $ D $ and are nonzero since $ \det D \ne 0 $. By the \textbf{Real Jordan Form Theorem}, there is a matrix $ P \in \operatorname{GL}_n (\mathbb{R}) $ such that
        \[
            D = P J P^{-1},
        \]
        where $ J $ is block-diagonal with blocks of two types: 
        \begin{enumerate}[(1)]
            \item Real eigenvalue $ \lambda $: real Jordan blocks of the form
            \[
                J = \begin{pmatrix}
                    \lambda & 1 & & & 0 \\
                    & \lambda & 1 & & \\
                    & & \ddots & \ddots & \\
                    & & & \lambda & 1 \\
                    0 & & & & \lambda \\
                \end{pmatrix}.
            \]
            \item Complex eigenvalue pair $ \alpha \pm i \beta $, $ \beta > 0 $: a block $ K \in M_{2m \times 2m}(\mathbb{R}) $ of the form 
            \[
                K = \begin{pmatrix}
                    C & I_2 & & & 0 \\
                    & C & I_2 & & 0 \\
                    & & \ddots & \ddots & \\
                    & & & C & I_2 \\
                    0 & & & & C \\
                \end{pmatrix}, 
            \]
            where 
            \[
                I_2 = \begin{pmatrix}
                    1 & 0 \\
                    0 & 1
                \end{pmatrix}, \quad 
                C = \begin{pmatrix}
                    \alpha & \beta \\
                    -\beta & \alpha
                \end{pmatrix} \in M_{2 \times 2}(\mathbb{R}).
            \]
        \end{enumerate}

        For the real eigenvalues, replace each Jordan block $ J(\lambda_k) $ corresponding to eigenvalue $ \lambda_k^2 $ by a Jordan block $ \tilde{J}(\lambda_k) $ with eigenvalue $ \log(\lambda_k^2) $, which is real since $ \lambda_k^2 > 0 $. For complex eigenvalue pairs, let's define the standard representation of complex matrices: let $ M = X + iY \in M_n (\mathbb{C}) $ be a complex matrix with $ X, Y \in M_n (\mathbb{R}) $, then its standard representation is given by 
        \[
            \Phi (M) = \begin{pmatrix}
                X & -Y \\
                Y & X
            \end{pmatrix} \in M_{2n} (\mathbb{R}).
        \]
        Then $ \Phi: M_n (\mathbb{C}) \to M_{2n} (\mathbb{R}) $ is an injective homomorphism, and $ \Phi (e^M) = e^{\Phi (M)} $. Then, notice that $ K_k = \Phi (J(\mu_k)) $, where $ J(\mu_k) $ is the Jordan block corresponding to the complex eigenvalue $ \mu_k = \alpha_k + i \beta_k $ of $ D^2 $. Construct the logarithm of $ J(\mu_k) \in M_m (\mathbb{C}) $ as 
        \[
            L_c (\mu_k) = \log (\mu_k^2) I + \log (I + N), \quad N = \frac{J(\mu_k^2) - \mu_k^2 I}{\mu_k^2} \text{ is nilpotent},  
        \]
        where 
        \[
            \log (I + N) = \sum_{j=1}^{m-1} \frac{(-1)^{j+1}}{j} N^j.
        \]
        Let $ \tilde{J}(\mu_k) = \Phi (L_c (\mu_k)) \in M_{2m} (\mathbb{R}) $ be the real logarithm, where $ \exp \left(\tilde{J}(\mu)\right) = e^{\Phi (L_c)} = \Phi (e^{L_c}) = \Phi (J(\mu_k)) = K_k $. Assemble all the $ \tilde{J}_k = \tilde{J} (\nu_k) $, where $ \nu = \lambda $ or $ \mu $, into a real block-diagonal matrix $ J_B $, then $ J^2 = e^{J_B} $, and we have
        \[
            D^2 = P J^2 P^{-1} = P e^{J_B} P^{-1} = e^{P J_B P^{-1}} \equiv e^B. 
        \] 
        Thus, such a real matrix $ B $ exists.

        \item First we begin with a lemma from the textbook:
        \begin{lemma}[Hale, Lemma 7.1]
            If $ C $ is an $ n \times n $ matrix with $ \det C \ne 0 $, then there is a matrix $ B $ such that $ e^B = C $.
        \end{lemma}
        That is, any invertible matrix has a logarithm. We will show that any such matrix has a square root as well. Since there is a matrix $ B $ such that $ e^B = C $, consider the matrix $ D = e^{B/2} $, then
        \[
            D^2 = e^{B/2} e^{B/2} = e^B = C
        \]
        since $ \frac{B}{2} $ commutes with itself. Thus, such a real matrix $ D $ exists.
    \end{enumerate}
\end{solution}

% Exercise 3
\begin{exercise}
    Let \(X=X(t) \in \mathbb{R}^{n \times n}\) be the fundamental matrix solution of \(E\, \dot{\vec{x}} =H(t) \vec{x}\), where \(E\) is given by
    \begin{align*}
        E=
        \begin{pmatrix}
        0 & I_k\\
        -I_k &0
        \end{pmatrix}
    \end{align*}
    and \(H=H(t)\) is symmetric and periodic with period T. Suppose \(H\) is a nonconstant matrix, and let \(\mathbb{H}=\{X(t):t \in \mathbb{R}\}\). Must \(\mathbb{H}\) be a subgroup of the symplectic group \(G:=\{M \in \mathbb{R}^{2k \times 2k}:M'EM=E\}\)\,? Justify your answer. Here, \(M'\) denotes the transpose of the
    matrix \(M\).
\end{exercise}

\begin{solution}
    ~

    \underline{Steps:}
    \begin{enumerate}
        \item Show that $ X(t) $ is symplectic for all $ t \in \mathbb{R} $, and hence $ \mathbb{H} \subseteq G $.
        \item Show that $ \mathbb{H} $ is not closed under matrix multiplication with a counterexample. Hence, $ \mathbb{H} $ is not a subgroup of $ G $.
    \end{enumerate}
    
    \underline{Method:}
    ~ 

    \begin{enumerate}
        \item Let $ X(t) $ be the fundamental matrix solution of the system $ E \dot{\vec{x}} = H(t) \vec{x} $. We will show that $ X(t) $ is symplectic for all $ t \in \mathbb{R} $. Consider the matrix $ Y(t) = X(t)' E X(t) $. Differentiating $ Y(t) $ with respect to $ t $, we have
        \[
            \frac{dY}{dt} = \dot{X}(t)' E X(t) + X(t)' E \dot{X}(t).
        \]
        Taking transpose of $ \dot{X}(t) = E^{-1} H(t) X(t) $, we have $ (E\dot{X})^{\prime} = (H(t) X)^{\prime} $. UUsing $ E^{\prime} = -E $, $ H^{\prime} = H $, we get $ \dot{X}^{\prime} E = -X^{\prime} H $. Substituting back gives $ \dot{Y} = 0 $. At $ t=0 $, $ X(t) = I $, so $ X^{\prime}(t) E X(t) = Y(t) = Y(0) = I $. Therefore, $ X(t) $ is symplectic for all $ t \in \mathbb{R} $, and $ \mathbb{H} \subseteq G $.
        \item To show that $ \mathbb{H} $ is not a subgroup of $ G $, we will show that matrix multiplication is not closed in $ \mathbb{H} $ by constructing a counterexample with $ k=1 $. Let
        \[
            E = \begin{pmatrix}
                0 & 1 \\
                -1 & 0
            \end{pmatrix}
        \]
        and $ H(t) = \sin t I_2 $, where $ I_2 $ is the $ 2 \times 2 $ identity matrix. Then $ H(t) $ is non-constant in time, symmetric, and periodic with period $ 2\pi $. The system $ E \dot{x} = H(t) x $ can be written as 
        \[
            \begin{dcases}
                \dot{x_2} &= \sin t x_1, \\
                \dot{x_1} &= -\sin t x_2, 
            \end{dcases}
        \]
        which is easily solved by the change of variable $ z = x_1 + i x_2 $. The solution is
        \[
            z(t) = z(0) \exp \left(\int_0^t \sin s \, ds\right) = z(0) e^{i(1 - \cos t)},
        \]
        and the fundamental matrix solution is a rotational matrix given by
        \[
            X(t) = \begin{pmatrix}
                \cos \theta (t) & -\sin \theta (t) \\
                \sin \theta (t) & \cos \theta (t)
            \end{pmatrix}, \text{where } \theta(t) = 1 - \cos t. 
        \]
        As can be seen, $ X(t) $ is symplectic, and since $ \cos t \in [-1,1] $, $ \theta \in [0,2] $, our set $ \mathbb{H} $ is given by 
        \[
            \mathbb{H} = \left\{  X(t) \mid t \in \mathbb{R} \right\} = \left\{  \begin{pmatrix}
                \cos \phi & -\sin \phi \\
                \sin \phi & \cos \phi
            \end{pmatrix} \mid \phi \in [0,2] \right\}.
        \]
        Now, pick $ t_1, t_2 $ such that $ \phi_1 = \theta(t_1) $, $ \phi_2 = \theta (t_2) $ are equal to $ \frac{3}{2} $. Then, by property of rotation matrices, we have
        \[
            X(t_1) X(t_2) = \begin{pmatrix}
                \cos 3 & -\sin 3 \\
                \sin 3 & \cos 3
            \end{pmatrix}, 
        \]
        where none of $ 3 + 2n\pi $ lies inside $ [0,2] $. Therefore, $ X(t_1) X(t_2) \notin \mathbb{H} $, and $ \mathbb{H} $ is not closed under matrix multiplication. Thus, $ \mathbb{H} $ is not a subgroup of $ G $.
    \end{enumerate}
\end{solution}

\end{CJK}
\end{document}