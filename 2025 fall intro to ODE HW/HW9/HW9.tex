\documentclass[a4paper]{article}
%% Formatting %%
\usepackage[margin=3cm]{geometry}
\usepackage{type1cm, titlesec, fancyhdr, titling}
\usepackage{multicol}
\usepackage[dvipsnames]{xcolor}
\usepackage{ulem}
\usepackage{parskip}
\setlength{\parindent}{2em}
\setlength{\headheight}{15pt}
\setlength{\droptitle}{-1.5cm}
\parindent=24pt
%% Math and Symbols %%
\usepackage{amsmath,amsthm,amssymb, mathtools}
\usepackage{yhmath, faktor, dsfont}
\usepackage{academicons, wasysym, marvosym}
\usepackage[scr]{rsfso} 
\usepackage{latexsym, amsmath, amscd, amsmath, amsthm}
\usepackage{amssymb,amsmath,amsthm,graphicx,dsfont}
\usepackage{hyperref}

%% Enhancement %%
\usepackage{graphicx, tabularx}
\usepackage[shortlabels,inline]{enumitem}
%% TikZ %%
\usepackage{tikz-cd}
\usepackage[breakable]{tcolorbox}
\usetikzlibrary{decorations.pathmorphing}
\usetikzlibrary{calc, arrows,matrix}

%% Other packages %%
\usepackage{amsopn}

%% Traditional Chinese %%
\usepackage{CJKutf8}

%% Math environments %%
\newtheoremstyle{mystyle}
  {6pt}{15pt}% 上下間距
  {}%          內文字體
  {}%              縮排
  {\bf}%       標頭字體
  {.}%       標頭後標點
  {1em}% 內文與標頭距離
  {}% Theorem head spec (can be left empty, meaning 'normal')
\theoremstyle{mystyle}	
\newtheorem{theorem}{Theorem}
\newtheorem*{definition}{Definition}
\newtheorem{example}[theorem]{Example}
\newtheorem{exercise}{Exercise}
\newtheorem{solution}{Solution}
\newtheorem{corollary}[theorem]{Corollary}
\newtheorem{property}[theorem]{Property}
\newtheorem{proposition}[theorem]{Proposition}
\newtheorem{lemma}[theorem]{Lemma}
\newtheorem{problem}[theorem]{Problem}
\newtheorem{answer}{Answer}[section]
\newtheorem{fact}[theorem]{fact}
\newtheorem*{remark}{Remark}
\newtheorem*{claim}{Claim}
\newtheorem*{observation}{Observation}

% code environment
\usepackage{listings}
\usepackage{xcolor}
\lstset{
    language=MATLAB,
    basicstyle=\ttfamily,
    keywordstyle=\color{blue},
    commentstyle=\color{green},
    stringstyle=\color{cyan},
    showstringspaces=false
}

\begin{document}
\begin{CJK}{UTF8}{bkai}

\title{%
  \textbf{2025 Fall Introduction to ODE} \\
  \vspace{0.5cm}
  \large 
  Homework 9 (Due November 17 12:00, 2025)\\
}
\author{物理三 黃紹凱 B12202004}
\date{\today}

\maketitle

% Exercise 1
\begin{exercise}
    Consider the IVP:
    \[
        \frac{dy}{dt}=t+y, \quad y(0)=1.
    \]
    Perform the first three successive iterations (starting with \(y_0(t)=1\)) to approximate the solution on the interval \(|t| \leq 1\). Then, identify the pattern or the exact solution if possible.
\end{exercise}

\begin{solution}
    ~

    \underline{Steps:}
    \begin{enumerate}
        \item Compute the first three successive iterations.
        \item Identify the pattern and show that the exact solution solves the initial value problem.
    \end{enumerate}
    
    \underline{Method:}
    ~ 

    \begin{enumerate}
        \item Transform the ODE into an integral equation by integrating both sides. This gives
        \[
            y(t) = y(0) + \int_0^t \mathrm{d}s\, (s + y(s)) = 1 + \frac{1}{2}t^2 + \int_0^t \mathrm{d}s\, y(s) .
        \]
        The first term in the successive iteration is $ y_0 = y(0) = 1 $. Then we have 
        \[
            y_1 (t) = 1 + \frac{1}{2}t^2 + \int_0^t \mathrm{d}s\, 1 = 1 + t + \frac{1}{2}t^2,
        \]
        \[
            y_2 (t) = 1 + \frac{1}{2}t^2 + \int_0^t \mathrm{d}s\, \left(1 + s + \frac{1}{2}s^2\right) = 1 + t + t^2 + \frac{1}{6}t^3,
        \]
        \[
            y_3 (t) = 1 + \frac{1}{2}t^2 + \int_0^t \mathrm{d}s\, \left(1 + s + s^2 + \frac{1}{6}s^3\right) = 1 + t + t^2 + \frac{1}{3}t^3 + \frac{1}{24}t^4.
        \]
        \item The pattern suggests that the \(n\)-th iteration is given by
        \[
            y_n (t) = 2 \left(1 + t + \frac{1}{2!}t^2 + \frac{1}{3!} t^3 + \cdots + \frac{1}{(n+1)!} t^{n+1 }\right) - 1 - t \xrightarrow[]{n \to \infty} y(t) = 2e^t - 1 - t.
        \]
        Since $ t + y $ and $ \frac{\partial}{\partial y} (t + y) = 1 $ are continuous, Theorem 8.1 guarantees convergence. Differentiating $ y(t) $, we get $ \frac{\mathrm{d}y}{\mathrm{d}t} = 2e^t - 1 = t + y(t) $. Moreover, $ y(0) = 1 $. Hence, the exact solution is \[y(t) = 2e^t - 1 - t.\] 
    \end{enumerate}
\end{solution}

\newpage
% Exercise 2
\begin{exercise}
    Consider the IVP:
    \[
        \frac{dy}{dt}=y^2+1, \quad y(0)=0.
    \]
    \begin{enumerate}[(a)]
        \item Show that the solution exists locally using the existence theorem.
        \item Demonstrate that the solution blows up in finite time (i.e., no global solution on all \(t \geq 0\)). Estimate the blowup time.
    \end{enumerate}
\end{exercise}

\begin{solution}
    ~

    \underline{Steps:}
    \begin{enumerate}[(a)]
        \item Show that the conditions of the local existence theorem are satisfied.
        \item Demonstrate finite blowup time by solving the ODE explicitly and finding the time at which the solution becomes unbounded.
    \end{enumerate}
    
    \underline{Method:}
    ~ 

    \begin{enumerate}[(a)]
        \item Recall the existence theorem from the textbook: 
        \begin{theorem}[King Theorem 8.1, Picard-Lindel\"{o}f]
            ~ 

            If $ f $ and $ \partial f / \partial y \in C^0 (R) $, where $ R = \left\{ (y,t) \middle| \vert y - y_0 \vert \leq b,\; \vert t - t_0 \vert \leq a \right\} $, then the successive approximations $ y_k (t) $ converge on $ I $ to a solution of the differentiatal equation $ \mathrm{d}y/\mathrm{d}t = f(t,y) $ that satisfies the initial condition $ y(t_0) = y_0 $. 
        \end{theorem}
        Since $ f(t,y) = y^2 + 1 $ is a polynomial in $ y $, it continuously differentiable in $ y $. Thus, $ f $ and $ \partial f / \partial y $ are in $ C^0 (R) $ for some region $ R \subseteq \mathbb{R}^2 $. By the Existence Theorem, there exists $ \delta >0 $ such that a solution exists locally on $ (-\delta , \delta) $,  and can be found by successive iterations.
        \item We can solve the ODE explicitly by separating variables:
        \[
            \frac{\mathrm{d}y}{y^2 + 1} = \mathrm{d}t \implies \tan^{-1} y = t + C.
        \]
        Applying the initial condition $ y(0) = 0 $, we get $ C = 0 $. Thus, the solution is
        \[
            y(t) = \tan t,
        \]
        where $ \lim_{t \to \frac{\pi}{2}^-} y(t) = +\infty $. Therefore, the blowup time is $ T^* = \frac{\pi}{2} $, and for $ t\geq 0 $ a solution exists only on $ [0, \frac{\pi}{2}) $. 
    \end{enumerate}
\end{solution}

\newpage
% Exercise 3
\begin{exercise}
    How could successive approximations to the solution of \(y'=3y^{\frac{2}{3}}\) fail to converge to a solution?
\end{exercise}

\begin{solution}
    ~

    \underline{Steps:}
    \begin{enumerate}
        \item Show that the function \(f(y) = 3y^{\frac{2}{3}}\) does not satisfy the Lipschitz condition near \(y=0\), so conditions of the Local Existence Theorem are not met.
        \item Explain how this failure leads to successive approximations potentially failing to converge.
    \end{enumerate}
    
    \underline{Method:}
    ~ 

    \begin{enumerate}
        \item The derivative \(f^{\prime} (y) = 2y^{-\frac{1}{3}}\) of $ f $  becomes unbounded as \(y \to 0\), and thus $f$ does not satisfy the Lipschitz condition around $ y(0) = 0 $. Since the hypotheses of the standard local existence theorem (Picard-Lindel\"{o}f) are not satisfied, convergence of successive approximations is not guaranteed.
        \item Let's solve the IVP explicitly by separating variables:
        \[
            y^{-2/3} \mathrm{d}y = 3 \mathrm{d}t \;\implies\; 3y^{1/3} = 3t + C.
        \]
        Imposing the initial condition $ y(0) = 0 $, we get one solution given by \(y(t) = t^3\), while the trivial solution \(y(t) = 0\) also satisfies the ODE and the initial condition. In fact, using this result, we can construct infinitely many solutions of the form 
        \[
            y(t) = \begin{dcases}
                0, & 0 \leq t \leq a, \\
                (t - a)^3, & t > a,
            \end{dcases}
        \]  
        for arbitrary $ a \geq 0 $. To show successive approximation does not converge to a unique solution, consider the following initial functions: Let $ y_0 = 0 $ be the trivial solution, then successive approximations yield the trivial solution $ y_n = 0 $ for all $ n $, and $ y_n \to y(t) = 0 $. On the other hand, if we choose $ y_0 = t^3 $, which satisfies $ y(0) = 0 $. then successive approximations will give
        \[
            y_1 (t) = \int_0^t \mathrm{d}s\, y_0(s) = \int_0^t \mathrm{d}s\, 3s^2 = t^3,
        \]
        \[
            \vdots 
        \]
        \[
            y_{n} (t) = \int_0^t \mathrm{d}s\, y_{n-1} (s) = \int_0^t \mathrm{d}s\, 3 s^2 = t^3 ,
        \]
        converging to the non-trivial solution \(y(t) = t^3\). Thus, depending on the choice of initial function, successive approximations can converge to different solutions or fail to converge altogether.
    \end{enumerate}
\end{solution}

\end{CJK}
\end{document}