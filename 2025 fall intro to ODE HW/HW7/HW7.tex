\documentclass[a4paper]{article}
%% Formatting %%
\usepackage[margin=3cm]{geometry}
\usepackage{type1cm, titlesec, fancyhdr, titling}
\usepackage{multicol}
\usepackage[dvipsnames]{xcolor}
\usepackage{ulem}
\usepackage{parskip}
\setlength{\parindent}{2em}
\setlength{\headheight}{15pt}
\setlength{\droptitle}{-1.5cm}
\parindent=24pt
%% Math and Symbols %%
\usepackage{amsmath,amsthm,amssymb, mathtools}
\usepackage{yhmath, faktor, dsfont}
\usepackage{academicons, wasysym, marvosym}
\usepackage[scr]{rsfso} 
\usepackage{latexsym, amsmath, amscd, amsmath, amsthm}
\usepackage{amssymb,amsmath,amsthm,graphicx,dsfont}
\usepackage{hyperref}

%% Enhancement %%
\usepackage{graphicx, tabularx}
\usepackage[shortlabels,inline]{enumitem}
%% TikZ %%
\usepackage{tikz-cd}
\usepackage[breakable]{tcolorbox}
\usetikzlibrary{decorations.pathmorphing}
\usetikzlibrary{calc, arrows,matrix}

%% Other packages %%
\usepackage{amsopn}

%% Traditional Chinese %%
\usepackage{CJKutf8}

%% Math environments %%
\newtheoremstyle{mystyle}
  {6pt}{15pt}% 上下間距
  {}%          內文字體
  {}%              縮排
  {\bf}%       標頭字體
  {.}%       標頭後標點
  {1em}% 內文與標頭距離
  {}% Theorem head spec (can be left empty, meaning 'normal')
\theoremstyle{mystyle}	
\newtheorem{theorem}{Theorem}
\newtheorem*{definition}{Definition}
\newtheorem{example}[theorem]{Example}
\newtheorem{exercise}{Exercise}
\newtheorem{solution}{Solution}
\newtheorem{corollary}[theorem]{Corollary}
\newtheorem{property}[theorem]{Property}
\newtheorem{proposition}[theorem]{Proposition}
\newtheorem{lemma}[theorem]{Lemma}
\newtheorem{problem}[theorem]{Problem}
\newtheorem{answer}{Answer}[section]
\newtheorem{fact}[theorem]{fact}
\newtheorem*{remark}{Remark}
\newtheorem*{claim}{Claim}
\newtheorem*{observation}{Observation}

% code environment
\usepackage{listings}
\usepackage{xcolor}
\lstset{
    language=MATLAB,
    basicstyle=\ttfamily,
    keywordstyle=\color{blue},
    commentstyle=\color{green},
    stringstyle=\color{cyan},
    showstringspaces=false
}

\begin{document}
\begin{CJK}{UTF8}{bkai}

\title{%
  \textbf{2025 Fall Introduction to ODE} \\
  \vspace{0.5cm}
  \large 
  Homework 7 (Due Nov 3 12:00, 2025)\\
}
\author{物理/數學三 黃紹凱 B12202004}
\date{\today}

\maketitle

% Exercise 1
\begin{exercise}
    Let \(A(t)\) and \(B(t)\) be defined as
    \begin{align*}
        A(t)=\begin{pmatrix}
        -a & 0\\
        0 & \sin{\log{t}+\cos{\log{t}-2a}}
        \end{pmatrix}, \quad 
        B(t)=\begin{pmatrix}
        0 & 0\\
        e^{-at} & 0
        \end{pmatrix}, \quad t \geq 0,
    \end{align*}
    where \(1<2a<1+e^{-\pi}\). Is the system \(\dot x=[A(t)+B(t)]x\)
    unstable? Prove or disprove your answer.
\end{exercise}
    
\begin{solution}
    ~

    \underline{Steps:}
    \begin{enumerate}
        \item State the definition for a solution to be \emph{(Lyapunov) unstable}.
        \item Analyze the system \(\dot x = [A(t) + B(t)]x\) and find its solution.
        \item Find a lower bound for the solution for some specific initial condition and time sequence.
        \item Show the sequence grows without bound, and conclude the zero solution is not Lyapunov stable.
    \end{enumerate}
    
    \underline{Method:}
    ~ 

    \begin{enumerate}
        \item A system is said to be unstable if there exists an \(\varepsilon > 0\) such that for any \(\delta > 0\), there exists an initial condition \(x(t_0)\) with \(|x(t_0)| < \delta\) and a time \(t > t_0\) such that \(|x(t)| > \varepsilon\). 
        \item The matrix \(A(t) + B(t)\) is lower-triangular, so we may directly solve for $ x_1 (t) $:  
        \begin{equation*}
            \dot x = \left[A(t) + B(t)\right] x = \begin{pmatrix}
                -a & 0 \\
                e^{-at} & \sin(\log t) + \cos(\log t) - 2a
            \end{pmatrix}
            \begin{pmatrix}
                x_1 \\
                x_2
            \end{pmatrix}.
        \end{equation*}
        First, we have \( \dot x_1 = -a x_1 \), which gives the solution $ x_1 (t) = x_1(0) e^{-at} $. Then substitute into the second equation:
        \[
            \dot x_2 = x_1(0) e^{-at} + \left[\sin(\log t) + \cos(\log t) - 2a\right] x_2. 
        \]  
        \[
            \Longrightarrow\; \dot x_2 - \left(\sin \log t + \cos \log t - 2a\right) x_2 = x_1(0) e^{-at}.
        \]
        This is a first-order linear ODE, so we can use the integrating factor method. The integrating factor is given by
        \[
            \mu (t) = \exp \left(-\int \left[\sin(\log t) + \cos(\log t) - 2a\right] \mathrm{d}t\right) = e^{2at} e^{\sin(\log t)}, 
        \]
        where we used $ \frac{\mathrm{d}}{\mathrm{d}t} \left(t \sin \log t\right) = \sin \log t + \cos \log t $. Then, we have 
        \[
            x_2 (t) = \frac{1}{\mu (t)} \int_{0}^t \mathrm{d}s\, \left(x_1(0) e^{-s \sin \log s} \right) = x_1 (0) e^{t \left(\sin \log t - 2a\right)} \int _0^t \mathrm{d}s\, e^{-s \sin \log s}. 
        \]
        Since we only have to find one solution, we set $ x_2(0) = 0 $ for our following discussion. 

        Let $ t_n = e^{2\pi n + \frac{\pi}{2}}$, then $ \sin \log t_n = \sin \left(2\pi n + \frac{\pi}{2}\right) = 1$. Similarly, let $ \tilde{t}_n = t_n e^{-\pi} = e^{2\pi n - \frac{\pi}{2}} $, so that $ \sin \log \tilde{t}_n = \sin \left(2\pi n - \frac{\pi}{2}\right) = -1$. For each $ n \in \mathbb{N} $ and $ \xi \in (0,1) $, by continuity of snie function, there exists $ \delta > 0 $ such that $ \sin x \leq -1 + \xi $ whenever $ \xi \in \left[2\pi n - \frac{\pi}{2} - \delta , 2 \pi n - \frac{\pi}{2} + \delta \right] $. Therefore, $ \sin \log s \leq -1 + \xi $ whenever $ \tilde{t}_n e^{-\delta} \leq s \leq \tilde{t}_n e^\delta $.

        \item Let $ S_n = \left[e^{\tilde{t}_n + \delta}, e^{\tilde{t}_n - \delta}\right] $, then we have $ \sin \log s \leq -1 + \xi $ for all $ s \in S_n $. Thus, 
        \[
            e^{-s \sin \log s} \geq e^{s (1 - \xi)}, \quad s \in S_n. 
        \]
        \[
            \Longrightarrow\; \int_0^t \mathrm{d}s\, e^{-s \sin \log s} \geq \int_{S_n} \mathrm{d}s\, e^{s (1- \xi)} \geq \tilde{t}_n \left(e^\delta - e^{-\delta}\right) e^{(1-\xi)\tilde{t}_n e^{-\delta}} \geq 0, \quad t \geq e^{\tilde{t}_n + \delta}.
        \]
        Evaluate $ x_2 (t) $ at $ t_n $ gives 
        \begin{align*}
            x_2 (t_n) &\geq  x_1(0) e^{t_n (1-2a)} \tilde{t}_n \left(e^\delta - e^{-\delta}\right) e^{(1-\xi)\tilde{t}_n e^{-\delta}} \\
            &= x_1(0) \left(e^\delta - e^{-\delta}\right) t_n e^{-\pi} e^{t_n \left[(1-2a) + (1-\xi)e^{-\pi} e^{-\delta}\right]}. 
        \end{align*}

        \item We have $ 1 < 2a < 1 + e^{-\pi} $, so $ 1 - 2a + e^{-\pi} > 0 $. Consider the function $ f(\xi , \delta) = (1-2a) + (1-\xi)e^{-\pi} e^{-\delta} $, by assumption $ f(0,0) > 0 $. Since $ f $ is continuous, there exists a disk of radius $ C $ about $ (0,0) $ such that $ f\left(\overline{\xi}, \overline{\delta} \right) > 0 $ for all $ \overline{\xi}, \overline{\delta} $ in the disk. Thus, we can choose $ \xi \in (0, \overline{\xi} ) $ and $ \delta = \min \{\delta^{\prime} , \overline{\delta} \} $, where $ \delta ^{\prime} $ is the bound given earlier by the continuity of sine. Then $ t_n \left[(1-2a) + (1-\xi ) e^{-\pi } e^{-\delta}\right] > 0 $, and $ x_2 (t)  \to \infty $ as $ n \to \infty $. Moreover, since $ x_1(0) $ is bounded, the norm $ \lVert x(t) \rVert \to \infty $ as $ t \to \infty $.    
    \end{enumerate}
\end{solution}

% Exercise 2
\begin{exercise}
    Consider the ODE system
    \begin{align}
        \label{eq:ODE}
        \dot x=A(t)x+f(t,x),
    \end{align}
    where \(A(t) \in \mathbb{R}^{n \times n}\) and \(f:\mathbb{R}^{n+1} \to \mathbb{R}^n\) is continous and satisfies \(|f(t,x)| \leq C(t)|x|\), for \(t \in \mathbb{R}\) and \(x \in \mathbb{R}^n\). Here, \(C(t)\) is a continous function satisfying
    \begin{align*}
        \int_{t}^{t+1}C(s)\,ds \leq \gamma, \quad t \geq \beta,
    \end{align*}
    for some constant \(\gamma=\gamma(\beta)>0\). Suppose the ODE
    system \(\dot x=A(t)x\) is uniformly asymptotically stable with
    respect to the zero solution. Prove that there is a constant \(r>0\)
    such that the zero solution of \ref{eq:ODE} is uniformly
    asymptotically stable if \(r>\gamma\).
\end{exercise}

\begin{solution}
    ~

    \underline{Steps:}
    \begin{enumerate}
        \item State the definition of being \emph{uniformly asymptotically stable with respect to the zero solution}.
        \item Show the Duhamel property. 
        \item Bound the solution $ x(t) $ using Gronwall's inequality to prove the claim.
    \end{enumerate}
    
    \underline{Method:}
    ~

    \begin{enumerate}
        \item A system is said to be \emph{uniformly asymptotically stable with respect to the zero solution} if for every \( \varepsilon > 0 \), there exists a \( \delta > 0 \) such that for any initial condition \( |x(t_0)| < \delta \), the solution \( x(t) \) satisfies \( |x(t)| < \varepsilon \) for all \( t \geq t_0 \).
        \item We first prove a proposition.
        \begin{proposition}[Duhamel's property]
            Let \( x(t) \) be the solution to the non-homogeneous system \(\dot x = A(t)x + f(t,x)\), $ t \geq t_0 $. Then, the solution can be expressed as
            \[
                x(t) = \Phi (t, t_0)x(t_0) + \int_{t_0}^{t} \mathrm{d}s\, \Phi(t,s)f(s,x(s)),
            \]
            where $ \Phi (t, t_0) = X(t) X(t_0)^{-1} $ is the state transition matrix of $ \dot x = A(t) x $. 
        \end{proposition}
        \begin{proof}
            Let $ y(t) = \Phi (t_0, t) x(t) $. Then we have $ \Phi (t_0, t) X(t) = X(t_0) $, so 
            \[
                \left(\partial_t \Phi (t_0, t)\right) X(t) + \Phi (t_0, t) \left(\partial_t X(t)\right) = 0. 
            \]
            \[
                \Longrightarrow\; \partial_t \Phi (t_0, t) = - \Phi (t_0, t) \left(\partial_t X(t)\right) X(t)^{-1} = -\Phi (t_0, t) A(t).
            \]
            Thus, we have
            \[
                \dot y (t) = \Phi (t_0, t) \left(\dot x - A(t) x\right) = \Phi (t_0, t) f(t, x(t)).
            \]
            Integrating from \( t_0 \) to \( t \), we get 
            \[
                y(t) = y(t_0) + \int_{t_0}^{t} \mathrm{d}s\, \Phi(t,s)f(s,x(s)) = \Phi (t_0, t) x(t_0) - x(t_0) = \int_{t_0}^{t} \mathrm{d}s\, \Phi(t_0 ,s)f(s,x(s)).
            \]
            \[
                \Longrightarrow\; x(t) = \Phi (t, t_0) \left[x(t_0) + \int_{t_0}^{t} \mathrm{d}s\, \Phi(t_0 ,s)f(s,x(s))\right] = \Phi (t, t_0)x(t_0) + \int_{t_0}^{t} \mathrm{d}s\, \Phi(t,s)f(s,x(s)), 
            \]
            since $ \Phi (t, t_0) \Phi (t_0, s) = \Phi (t, s) $ by the semigroup property.
        \end{proof}
        \item For a linear time0varying system, uniform asymptotic stability of the zero solution is equivalent to uniform exponential stability. Thus, there exist positive constants \( K \) and \( r \) such that the state transition matrix \( \Phi(t, t_0) \) satisfies
        \[
            \|\Phi(t, t_0)\| \leq K e^{-\alpha (t - t_0)}, \quad t \geq t_0.
        \]
        Using the Duhamel property, we have
        \begin{align*}
            \lVert x(t) \rVert &\leq \lVert \Phi (t, t_0) x(t_0) \rVert + \left\lVert \int_{t_0}^{t} \mathrm{d}s\, \Phi(t,s)f(s,x(s)) \right\rVert \\
            &\leq K e^{-r (t - t_0)} \lVert x(t_0) \rVert + \int_{t_0}^{t} \mathrm{d}s\, K e^{-r (t - s)} C(s) \lVert x(s) \rVert.
        \end{align*}
        Let $ u(t) = e^{rt} \lVert x(t) \rVert $, then 
        \[
            \lVert u(t) \rVert \leq K \left[e^{rt_0} \lVert x(t_0) \rVert + \int_{t_0}^t \mathrm{d}s\, C(s) e^{rs} \lVert x(s) \rVert \right] = K \left[u(t_0) + \int_{t_0}^t \mathrm{d}s\, C(s) u(s) \right].
        \] 
        Then 
        \[
            u(t) \leq K u(t_0) + K \int_{t_0}^t \mathrm{d}s\, C(s) u(s) \leq K u(t_0) \exp \left(\int_{t_0}^t \mathrm{d}s\, C(s) \right).
        \]
        We can bound the term in the exponential using the assumption on \( C(t) \):
        \begin{align*}
            \lVert x(t) \rVert &\leq K e^{-r(t-t_0)} \lVert x(t_0) \rVert \exp \left(\int_{t_0}^t \mathrm{d}s\, C(s) \right) \leq K e^{-r(t-t_0)} e^{\gamma (t - t_0 + 1)} \lVert x(t_0) \rVert \\
            &= K e^{\gamma} e^{-(r - \gamma)(t - t_0)} \lVert x(t_0) \rVert.
        \end{align*}
        Since \( r > \gamma \), we have \( \lVert x(t) \rVert \to 0 \) as \( t \to \infty \). More precisely, for any $ \varepsilon > 0 $, let $ \delta = \frac{1}{K} e^{-\gamma} \varepsilon $, then  
        \[
            \lVert x(t) \rVert = K e^\gamma e^{- (r - \gamma)(t - t_0)} \lVert x(t_0) \rVert < K e^\gamma \delta = \varepsilon
        \]
        whenever $ \lVert x(t_0) \rVert < \delta $. Thus, the zero solution of the system is uniformly asymptotically stable.
    \end{enumerate}
\end{solution}

\end{CJK}
\end{document}