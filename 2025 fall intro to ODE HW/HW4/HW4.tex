\documentclass[a4paper]{article}
%% Formatting %%
\usepackage[margin=3cm]{geometry}
\usepackage{type1cm, titlesec, fancyhdr, titling}
\usepackage{multicol}
\usepackage[dvipsnames]{xcolor}
\usepackage{ulem}
\usepackage{parskip}
\setlength{\parindent}{2em}
\setlength{\headheight}{15pt}
\setlength{\droptitle}{-1.5cm}
\parindent=24pt
%% Math and Symbols %%
\usepackage{amsmath,amsthm,amssymb, mathtools}
\usepackage{yhmath, faktor, dsfont}
\usepackage{academicons, wasysym, marvosym}
\usepackage[scr]{rsfso} 
\usepackage{latexsym, amsmath, amscd, amsmath, amsthm}
\usepackage{amssymb,amsmath,amsthm,graphicx,dsfont}
\usepackage{hyperref}

%% Enhancement %%
\usepackage{graphicx, tabularx}
\usepackage[shortlabels,inline]{enumitem}
%% TikZ %%
\usepackage{tikz-cd}
\usepackage[breakable]{tcolorbox}
\usetikzlibrary{decorations.pathmorphing}
\usetikzlibrary{calc, arrows,matrix}

%% Other packages %%
\usepackage{amsopn}

%% Traditional Chinese %%
\usepackage{CJKutf8}

%% Math environments %%
\newtheoremstyle{mystyle}
  {6pt}{15pt}% 上下間距
  {}%          內文字體
  {}%              縮排
  {\bf}%       標頭字體
  {.}%       標頭後標點
  {1em}% 內文與標頭距離
  {}% Theorem head spec (can be left empty, meaning 'normal')
\theoremstyle{mystyle}	
\newtheorem{theorem}{Theorem}
\newtheorem*{definition}{Definition}
\newtheorem{example}[theorem]{Example}
\newtheorem{exercise}{Exercise}
\newtheorem{solution}{Solution}
\newtheorem{corollary}[theorem]{Corollary}
\newtheorem{property}[theorem]{Property}
\newtheorem{proposition}[theorem]{Proposition}
\newtheorem{lemma}[theorem]{Lemma}
\newtheorem{problem}[theorem]{Problem}
\newtheorem{answer}{Answer}[section]
\newtheorem{fact}[theorem]{fact}
\newtheorem*{remark}{Remark}
\newtheorem*{claim}{Claim}
\newtheorem*{observation}{Observation}

% code environment
\usepackage{listings}
\usepackage{xcolor}
\lstset{
    language=MATLAB,
    basicstyle=\ttfamily,
    keywordstyle=\color{blue},
    commentstyle=\color{green},
    stringstyle=\color{cyan},
    showstringspaces=false
}

\begin{document}
\begin{CJK}{UTF8}{bkai}

\title{%
  \textbf{2025 Fall Introduction to ODE} \\
  \vspace{0.5cm}
  \large 
  Homework 4 (Due Sep 29 12:00, 2025)\\
}
\author{物理/數學三 黃紹凱 B12202004}
\date{\today}

\maketitle

\begin{problem}
    Comment on the difficulties that you face when trying to construct the Green's function for the boundary value problem
    \begin{equation}
        y^{\prime\prime} (x)+y(x)=f(x) \quad \text{subject to} \quad y(a)=y^{\prime} (b)=0.
    \end{equation}
\end{problem}

\begin{solution}
    ~

    \underline{Steps:}
    \begin{enumerate}
        \item Construct the general solution to the homogeneous equation $ y^{\prime\prime} + y = 0 $. 
        \item Solve for the Green's function $ G(x, \xi) $ using the boundary conditions and the jump condition at $ x = \xi $.
        \item Write the solution to the inhomogeneous equation as
        \begin{equation}
            y(x) = \int_{a}^{b} G(x, \xi) f(\xi) \, \mathrm{d} \xi,
        \end{equation}
        and discuss the problem found.
    \end{enumerate}
    
    \underline{Method:}
    First let's pick the standard fundamental solutions to the homogeneous case $ y^{\prime\prime} + y = 0 $ such that they satisfy the boundary conditions. We have: 
    \begin{equation}
        \begin{split}
            y_1(x) &= \sin (x-a), \quad y_1(a) = 0, \\
            y_2(x) &= \cos (x-b), \quad y_2^{\prime} (b) = 0.
        \end{split}
    \end{equation} 
    The Wronskian is a constant given by 
    \begin{equation}
        W = y_1 y_2^{\prime} - y_2 y_1^{\prime} = -\cos (a-b),
    \end{equation}
    and the Green's function is given by
    \begin{equation}
        G(x,t) = \begin{cases}
            \left[y_1(x) y_2(\xi)\right] / W, & a \leq x < \xi \leq b, \\
            \left[y_1(\xi) y_2(x)\right] / W, & a \leq \xi < x \leq b,
        \end{cases}
    \end{equation}
    whenever $ W(x) $ is nonzero. However, if $ a - b = n \pi /2 $, $ n \in \mathbb{Z} $, then $ W = 0 $ and the Green's function cannot be constructed. We can interpret this result by notiving that the boundary conditions are not independent when $ a-b = n \pi /2 $, since $ W = 0 $ at these points. In this case, the boundary value problem may not have a solution for arbitrary $ f(x) $. 
\end{solution}

% Problem 2
\begin{problem}
    Write the generalized Legendre equation,
    \begin{equation}
        (1-x^2)\frac{d^2y}{dx^2}-2x\frac{dy}{dx}+\left\{n(n+1)-\frac{m^2}{1-x^2}\right\}y=0,    
    \end{equation}
    as a Sturm-Liouville equation.
\end{problem}

\begin{solution}
    ~
    \underline{Steps:}
    \begin{enumerate}
        \item Rewrite the equation in the standard form of a Sturm-Liouville problem.
        \item Identify the functions $ p(x), q(x), r(x) $ and the eigenvalue $ \lambda $.
        \item Discuss the boundary conditions at the endpoints $ x = \pm 1 $.
    \end{enumerate}
    
    \underline{Method:}
    Sturm-Liouville equations are of the form
    \begin{equation}
        \frac{d}{dx}\left[p(x)\frac{dy}{dx}\right] + q(x) y(x) = - \lambda r(x) y(x) = 0,
    \end{equation}
    where $ \lambda $ is the eigenvalue which depends on the boundary conditions. Furthermore, the functions $ p(x), q(x), r(x) $ are real-valued and continuous on the closed interval $ [a,b] $, $ p(x) $ is differentiable, and $ p(x) > 0, r(x) > 0 $ on the open interval $ (a,b) $.

    From the textbook, the endpoint $x = a$ is a \textbf{singular endpoint} if $a = -\infty$ or if $a < \infty $ but the above conditions do not hold on the closed interval $[a, c]$ for some $c \in (a, b)$. Similar definitions hold for the other endpoint, $x = b$. Hence $ \pm 1 $ are singular endpoints of the generalized Legendre equation. Therefore, we apply Friedrichs boundary conditions. 

    Then we have 
    \begin{equation}
        \frac{\mathrm{d}}{\mathrm{d}x} \left[(1-x^{2}) \frac{\mathrm{d}y(x)}{\mathrm{d}x}\right] - \left(\frac{m^{2}}{1-x^{2}}\right)y(x) = - n(n+1) y(x) ,
    \end{equation}
    which is a Sturm-Liouville equation with eigenvalue $ \lambda = n(n+1) $ and functions
    \begin{equation}
        p(x) = 1 - x^{2}, \quad q(x) = -\frac{m^{2}}{1-x^{2}}, \quad r(x) = 1
    \end{equation}
    satisfying the conditions above on the interval $(-1, 1)$. Since the Sturm-Liouville operator is singular at $ \pm 1 $, we assume Friedrichs boundary conditions following the description in [King \& Billingham \& Otto]. That is, we require that the solution $ y(x, \lambda) $ satisfies
    \begin{equation}
        \left\vert y(x, \lambda) \right\vert \leq A \text{ for } x\in (-1,0] \text{ and } \left\vert y(x, \lambda) \right\vert \leq B \text{ for } x\in [0,1),
    \end{equation}
    for some $ A, B \in \mathbb{R}_{\geq 0} $.
\end{solution}

% Problem 3
\begin{problem}
    Show that
    \begin{equation}
        -(xy^{\prime} (x))^{\prime} = \lambda xy(x),
    \end{equation}
    is self-adjoint on the interval \((0,1)\), with \(x=0\) a singular
    endpoint and \(x=1\) a regular endpoint with the condition \(y(1)=0\).
\end{problem}

\begin{solution}
    ~

    \underline{Steps:}
    \begin{enumerate}
        \item Rewrite the equation in the standard form of a Sturm-Liouville problem.
        \item Use Lagrange's identity to show that $ L $ is self-adjoint
    \end{enumerate}
    
    \underline{Method:}
    A linear operator is said to be self-adjoint if it satisfies
    \begin{equation}
        \langle L u, v \rangle = \langle u, L v \rangle,
    \end{equation}
    where the inner product is defined as
    \begin{equation}
        \langle u, v \rangle = \int_{0}^{1} \mathrm{d} x \, u(x) v(x).
    \end{equation}
    Expanding the left-hand side, we can write the differential equation in terms of a linear operator $ L $: 
    \begin{equation}
        x \frac{\mathrm{d}^{2}y}{\mathrm{d}x^{2}} + \frac{\mathrm{d}y}{\mathrm{d}x} +\lambda x y(x) \equiv Ly = 0.
    \end{equation}
    Let $ u, v $ be two functions satisfying the boundary conditions. Then, by Lemma 4.1 (Lagrange's identity) in [King \& Billingham \& Otto], let 
    \begin{equation}
        L =  x \frac{\mathrm{d}^{2}}{\mathrm{d}x^{2}} + \frac{\mathrm{d}}{\mathrm{d}x} +\lambda x
    \end{equation}
    be a linear differential operator on (0,1), and $ u, v \in C^{2}(0,1) $, then 
    \begin{equation}
        u(Lv) - v(Lu) = \left[p \left(uv^{\prime} - u^{\prime} v\right)\right],
    \end{equation} 
    thus 
    \begin{equation}
        \langle L u, v \rangle - \langle u, L v \rangle = \left[ x \left( u(x) v^{\prime}(x) - u^{\prime}(x) v(x) \right) \right]_{0}^{1} = 0.
    \end{equation}
    where the terms at $ x=1 $ vanish due to the boundary condition $ y(1) = 0 $, and the terms at $ x=0 $ vanish by some additional regularity condition on $ y(x) $ at the singular endpoint. One sufficient and natural choice would be 
    \begin{equation}
        \lim_{x \to 0^+} x y^{\prime} (x) = 0. 
    \end{equation}
    Therefore, we have shown that 
    \begin{equation}
        \langle L u, v \rangle = \langle u, L v \rangle,
    \end{equation}
    and  $ L $ is self-adjoint.
\end{solution}

\end{CJK}
\end{document}