\documentclass[a4paper]{article}
%% Formatting %%
\usepackage[margin=3cm]{geometry}
\usepackage{type1cm, titlesec, fancyhdr, titling}
\usepackage{multicol}
\usepackage[dvipsnames]{xcolor}
\usepackage{ulem}
\usepackage{parskip}
\setlength{\parindent}{2em}
\setlength{\headheight}{15pt}
\setlength{\droptitle}{-1.5cm}
\parindent=24pt
%% Math and Symbols %%
\usepackage{amsmath,amsthm,amssymb, mathtools}
\usepackage{yhmath, faktor, dsfont}
\usepackage{academicons, wasysym, marvosym}
\usepackage[scr]{rsfso} 
\usepackage{latexsym, amsmath, amscd, amsmath, amsthm}
\usepackage{amssymb,amsmath,amsthm,graphicx,dsfont}
\usepackage{hyperref}

%% Enhancement %%
\usepackage{graphicx, tabularx}
\usepackage[shortlabels,inline]{enumitem}
%% TikZ %%
\usepackage{tikz-cd}
\usepackage[breakable]{tcolorbox}
\usetikzlibrary{decorations.pathmorphing}
\usetikzlibrary{calc, arrows,matrix}

%% Other packages %%
\usepackage{amsopn}

%% Traditional Chinese %%
\usepackage{CJKutf8}

%% Math environments %%
\newtheoremstyle{mystyle}
  {6pt}{15pt}% 上下間距
  {}%          內文字體
  {}%              縮排
  {\bf}%       標頭字體
  {.}%       標頭後標點
  {1em}% 內文與標頭距離
  {}% Theorem head spec (can be left empty, meaning 'normal')
\theoremstyle{mystyle}	
\newtheorem{theorem}{Theorem}
\newtheorem{definition}{Definition}
\newtheorem{example}[theorem]{Example}
\newtheorem{exercise}{Exercise}
\newtheorem{solution}{Solution}
\newtheorem{corollary}[theorem]{Corollary}
\newtheorem{property}[theorem]{Property}
\newtheorem{proposition}[theorem]{Proposition}
\newtheorem{lemma}{Lemma}
\newtheorem{problem}[theorem]{Problem}
\newtheorem{answer}{Answer}[section]
\newtheorem{fact}[theorem]{fact}
\newtheorem*{claim}{Claim}
\newtheorem*{observation}{Observation}

\newenvironment{exerciseManual}[1]{%
  \renewcommand{\theexercise}{#1}%
  \begin{exercise}%
  \addtocounter{exercise}{-1}%
}{%
  \end{exercise}%
}

\newenvironment{solutionManual}[1]{%
  \renewcommand{\thesolution}{#1}%
  \begin{solution}%
  \addtocounter{solution}{-1}%
}{%
  \end{solution}%
}

\theoremstyle{remark}
\newtheorem*{remark}{Remark}

\newcommand{\bvec}[1]{\mathbf{#1}} % vector

\begin{document}
\begin{CJK}{UTF8}{bkai}

\title{%
  \textbf{2025 Fall Introduction to Geometry} \\
  \vspace{0.5cm}
  \Large Solutions to Exercises in Do Carmo \\
}
\author{黃紹凱 B12202004}
\date{\today}

\maketitle

\section{Chapter 1.1}
\section{Chapter 1.2}

% 1.2.5
\begin{exerciseManual}{1.2.5}
    Let $\alpha \colon I \to \mathbb{R}^3$ be a parametrized curve, with $\alpha'(t) \neq 0$ for all $t \in I$. Show that $|\alpha(t)|$ is a nonzero constant if and only if $\alpha(t)$ is orthogonal to $\alpha'(t)$ for all $t \in I$.
\end{exerciseManual}

\begin{solutionManual}{1.2.5} 
    Suppose $|\alpha(t)| = c \neq 0$ for all $t \in I$. Then,
    \[
        \frac{\mathrm{d}}{\mathrm{d}t}\vert \alpha(t) \vert^{2} = 2 \alpha (t) \cdot \alpha^{\prime} (t) = \frac{\mathrm{d}}{\mathrm{d}t} c^{2} = 0. 
    \]
    Thus $ \alpha (t) \cdot \alpha^{\prime} (t) = 0 $, and $ \alpha (t) $ and $ \alpha^{\prime} (t) $ are orthogonal. Conversely, suppose $ \alpha (t) $ and $ \alpha^{\prime} (t) $ are orthogonal for all $ t \in I $, so $ \alpha (t) \cdot \alpha^{\prime} (t) = 0 $. Then, we have
    \[
        \frac{\mathrm{d}}{\mathrm{d}t}\vert \alpha(t) \vert^{2} = 2 \alpha (t) \cdot \alpha^{\prime} (t) = 0.
    \]
    Thus $ \vert \alpha (t) \vert $ is a constant.
\end{solutionManual}

\section{Chapter 1.3}

\begin{definition}[regular curve]
    A parametrized curve $ \alpha : I \to \mathbb{R}^3 $ is said to be \emph{regular} if $ \alpha^{\prime} (t) \neq 0 $ for all $ t \in I $.
\end{definition}

% 1.3.2
\begin{exerciseManual}{1.3.2}
    A circular disk of radius 1 in the plane $xy$ rolls without slipping along the $x$-axis. The figure described by a point on the circumference of the disk is called a \textbf{cycloid} (Figure 1-7).
    \begin{enumerate}[label=\textbf{\alph*.}]
        \item Obtain a parametrized curve $\alpha \colon \mathbb{R} \to \mathbb{R}^2$ the trace of which is the cycloid, and determine its singular points.
        \item Compute the arc length of the cycloid corresponding to a complete rotation of the disk.
    \end{enumerate}
\end{exerciseManual}

\begin{solutionManual}{1.3.2}
    ~
    \begin{enumerate}[label=\textbf{\alph*.}]
        \item Let \( \alpha(t) = (x(t), y(t)) \) be the parametrized curve of the cycloid. As the disk rolls without slipping and the radius of the disk is 1, the distance traveled along the $ x $-axis is $ t $ and the $ y $-coordinate is given by the height of the point on the circumference. Therefore, we have:
        \begin{equation}
            \begin{split}
                x(t) &= t - \sin(t), \\
                y(t) &= 1 - \cos(t).
            \end{split}
        \end{equation}
        The singular points occur when $ \alpha ^{\prime} (t) = 0 $. This is equivalent to 
        \begin{equation}
            \begin{split}
                x^{\prime} (t) &= 1 - \cos(t) = 0, \\
                y^{\prime} (t) &= \sin(t) = 0.
            \end{split}
        \end{equation} 
        Hence the singular points are at $ t = 2n \pi $ for all $ n \in \mathbb{Z} $.  
        \item The arc length of the cycloid for a complete rotation is given by integrating over $ [0, 2\pi ] $. 
        \begin{equation}
            \begin{split}
                L &= \int_0^{2\pi} \mathrm{d} t\, |\alpha'(t)| = \int_0^{2\pi} \mathrm{d} t\, \sqrt{(1 - \cos(t))^2 + (\sin(t))^2} \\
                &= \int_0^{2\pi} \mathrm{d} t\, \sqrt{2 - 2\cos(t)} = 8.
            \end{split}
        \end{equation}
    \end{enumerate}
\end{solutionManual}

% 1.3.4
\begin{exerciseManual}{1.3.4}
    Let \( \alpha: (0, \pi) \to \mathbb{R}^2 \) be given by
    \[
    \alpha(t) = \left( \sin t, \, \cos t + \log \tan \frac{t}{2} \right),
    \]
    where \( t \) is the angle that the \( y \)-axis makes with the vector \( \alpha(t) \). The trace of \( \alpha \) is called the \textit{tractrix} (see Fig. 1-9). Show that:
    \begin{enumerate}[label=\textbf{\alph*.}]
        \item \( \alpha \) is a differentiable parametrized curve, regular except at \( t = \pi/2 \).
        \item The length of the segment of the tangent of the tractrix between the point of tangency and the \( y \)-axis is constantly equal to \( 1 \).
    \end{enumerate}
\end{exerciseManual}

\begin{solutionManual}{1.3.4}
    Recall that a \textbf{regular}  curve is a smooth, parametrized curve with a non-vanishing derivative.
    \begin{enumerate}[label=\textbf{\alph*.}]
        \item First we shall compute the derivative of $ \alpha (t) $ as 
        \begin{equation}
            \begin{split}
                \alpha^{\prime} (t) &= \left( \cos t, \, -\sin t + \frac{1}{\sin t} \right) \\
                &= \left(\cos t, \cot t \cos t\right)
            \end{split}
        \end{equation}
        Since $ \alpha^{\prime} (t) $ is continuous on \( (0, \pi) \) and \( \alpha^{\prime} (t) \neq 0 \) for all \( t \in (0, \pi) \setminus \{ \pi/2 \} \), \( \alpha(t) \) is a differentiable parametrized curve, regular except at \( t = \pi/2 \).
        \item The equation of the tangent line at \( \alpha(t) \) is given by
        \begin{equation}
            y - y_{0} (t) = \cot t \left( x - x_{0} (t) \right),
        \end{equation}
        where $ y_{0} = \cos t + \log \tan \frac{t}{2}$ and $ x_{0} = \sin t $. Setting \( x = 0 \) to find the intersection with the \( y \)-axis, we have
        \begin{equation}    
            \begin{split}
                \Delta y &\equiv y - y_{0} (t) = -\cot t \sin t = - \cos t, \\
                \Delta x &\equiv x - x_{0} (t) = -\sin t.
            \end{split}
        \end{equation}
        Then the distance is $ \sqrt{(\Delta y)^{2} + (\Delta x)^{2} } = 1 $.
    \end{enumerate}
\end{solutionManual}

% 1.3.7
\begin{exerciseManual}{1.3.7}
    A map \( \alpha: I \to \mathbb{R}^3 \) is called a curve of class \( C^k \) if each of the coordinate functions in the expression \( \alpha(t) = (x(t), y(t), z(t)) \) has continuous derivatives up to order \( k \). If \( \alpha \) is merely continuous, we say that \( \alpha \) is of class \( C^0 \). A curve \( \alpha \) is called simple if the map \( \alpha \) is one-to-one. Thus, the curve in Example 3 of Sec. 1-2 is not simple.

    Let \( \alpha: I \to \mathbb{R}^3 \) be a simple curve of class \( C^0 \). We say that \( \alpha \) has a weak tangent at \( t = t_0 \in I \) if the line determined by \( \alpha(t_0 + h) \) and \( \alpha(t_0) \) has a limit position when \( h \to 0 \). We say that \( \alpha \) has a strong tangent at \( t = t_0 \) if the line determined by \( \alpha(t_0 + h) \) and \( \alpha(t_0 + k) \) has a limit position when \( h, k \to 0 \). Show that
    \begin{enumerate}
        \item[a.] \( \alpha(t) = (t^3, t^2), t \in \mathbb{R}, \) has a weak tangent but not a strong tangent at \( t = 0 \).
        \item[b.] If \( \alpha: I \to \mathbb{R}^3 \) is of class \( C^1 \) and regular at \( t = t_0 \), then it has a strong tangent at \( t = t_0 \).
        \item[c.] The curve given by
        \[
        \alpha(t) =
        \begin{cases}
            (t^2, t^2), & t \geq 0, \\
            (t^2, -t^2), & t \leq 0,
        \end{cases}
        \]
        is of class \( C^1 \) but not of class \( C^2 \). Draw a sketch of the curve and its tangent vectors.
    \end{enumerate}
\end{exerciseManual}

\begin{solutionManual}{1.3.7}
    
\end{solutionManual}

% 1.3.8
\begin{exercise}
    Let $\alpha:I\rightarrow\mathbb{R}^3$ be a differentiable curve and let $[a, b]\subseteq I$ be a closed interval. For every partition
    \[
        a=t_0<t_1<\cdots <t_n =b
    \]
    of $[a, b]$, consider the sum $\sum_{i=1}^n |\alpha(t_i)-\alpha(t_{i-1})|=l(\alpha, P)$, where $P$ stands for the given partition. The norm $|P|$ of a partition $P$ is define as 
    $$
        |P|=\max (t_{i}-t_{i-1}), \, i=1,\dots, n
    $$
    Geometrically, $l(\alpha, P)$ is the length of the polygon inscribed in $\alpha([a, b])$ with the vertices in $\alpha(t_i)$. The point of the exercise is to show that the arc length of $\alpha([a, b])$ is, in some sense, a limit of the length of the inscribed polygons. 
    Prove that given $\epsilon>0$ there exists $\delta>0$ such that if $|P|<\delta$ then 
    \[
        \left|\int_a^b |\alpha'(t)| \mathrm{d} t - l(\alpha, P)\right|<\epsilon
    \]
\end{exercise}

\begin{solution}
    Since $ \alpha (t) $ is differentiable on the closed interval $ [a, b] $, $ \alpha^{\prime} (t) $ is continuous. Thus, for any $ \epsilon^{\prime} > 0 $ there exists $ \delta^{\prime} > 0 $ such that $ \alpha^{\prime} (t_2) - \alpha^{\prime} (t_1) < \epsilon $ whenever $ \vert t_2 - t_1 \vert < \delta^{\prime} $.
    For a partition $ P $, let $ \epsilon^{\prime} > 0 $. The integral can be bounded as:
    \begin{equation}
        \begin{split}
            \left| \int_a^b \mathrm{d} t \, |\alpha'(t)| - l(\alpha, P) \right| &= \left| \int_a^b \mathrm{d} t\, |\alpha'(t)| - \sum_{i=1}^n |\alpha(t_i) - \alpha(t_{i-1})| \right| \\
            &\leq \sum_{i=1}^n \left| \int_{t_{i-1}}^{t_i}\mathrm{d} t |\alpha'(t)| - |\alpha(t_i) - \alpha(t_{i-1})| \right| \\
            &\leq \sum_{i=1}^n \int_{t_{i-1}}^{t_i} \mathrm{d} t\, \left| |\alpha'(t)| - \frac{|\alpha(t_i) - \alpha(t_{i-1})|}{t_i - t_{i-1}} \right| \\
            &\leq \sum_{i=1}^{n} \int^{t_i}_{t_{i-1}} \mathrm{d}t\, \left\vert \alpha^{\prime} (t) - \alpha^{\prime} (\xi) \right\vert \\
            &< n (b-a) \epsilon^{\prime} . 
        \end{split}
    \end{equation}
    whenever $ t - \xi < \vert P \vert < \operatorname{min}_{i \in \{1, \dots, n\}} \left(\delta^{\prime}_{i}\right) $. We have used the Mean Value Theorem to obtain $ \xi $. Now, let $ \epsilon^{\prime} = \epsilon / n(b-a) $, $ \delta = \delta^{\prime} $, then for any partition $ P $ with $ |P| < \delta $, we have
    \begin{equation}
        \left| \int_a^b \mathrm{d} t \, |\alpha'(t)| - l(\alpha, P) \right| < \epsilon.
    \end{equation}
\end{solution}

\subsection{Chapter 1.4}

% 1.4.11
\begin{exercise}
    ~
    \begin{enumerate}[label=\textbf{\alph*.}]
        \item Show that the volume $V$ of a parallelepiped generated by three linearly independent vectors $u, v, w\in \mathbb{R}^3$ is given by $V=|(u\land v)\cdot w|$, and introduce an oriented volume in $\mathbb{R}^3$.
        \item Prove that 
        \[
            V^2=\begin{vmatrix}
                u \cdot u & u \cdot v & u \cdot w\\
                v \cdot u &v \cdot v& v \cdot w\\
                w \cdot u & w \cdot v& w \cdot w
            \end{vmatrix} \tag{1}
        \]
    \end{enumerate}
\end{exercise}

\begin{solution}
    ~
    \begin{enumerate}[label=\textbf{\alph*.}]
        \item By definition, the volume of the parallelepiped is given by the area of the base times the height. The area of the base formed by $ u $ and $ v $ is given by $ |u \land v| $, and the height is given by the projection of $ w $ onto the normal vector of the base, which is $ \frac{(u \land v)}{|u \land v|} $. Therefore, the volume $ V $ is given by
        \begin{equation}
            V = |u \land v| \cdot \left| w \cdot \frac{(u \land v)}{|u \land v|} \right| = |(u \land v) \cdot w|.
        \end{equation}
        The oriented volume can be introduced as $ V = (u \land v) \cdot w $. If the vectors $ u, v, w $ (in order) form a right-handed system, the oriented volume is positive; otherwise, it is negative.
        \item Recall that the vector product $ u \land v \in \mathbb{R}$ is the unique vector where $ (u \land v) \cdot w = \operatorname{det} (u, v, w)$. By (a), the volume of the parallelepiped is given by
        \begin{equation}
            V = |(u \land v) \cdot w|.
        \end{equation} 
        Then,
        \begin{equation}
            \begin{split}
                V^2 &= ((u \land v) \cdot w) ((u \land v) \cdot w) \\
                &= \operatorname{det} (u, v, w)^{2} \\
                &= \begin{vmatrix}
                    u_1 & u_2 & u_3 \\
                    v_1 & v_2 & v_3 \\
                    w_1 & w_2 & w_3
                \end{vmatrix}
                \begin{vmatrix}
                    u_1 & u_2 & u_3 \\
                    v_1 & v_2 & v_3 \\
                    w_1 & w_2 & w_3
                \end{vmatrix} \\
                &= \left\vert
                \begin{pmatrix}
                    u_1 & u_2 & u_3 \\
                    v_1 & v_2 & v_3 \\
                    w_1 & w_2 & w_3
                \end{pmatrix}
                \begin{pmatrix}
                    u_1 & v_1 & w_1 \\
                    u_2 & v_2 & w_2 \\
                    u_3 & v_3 & w_3
                \end{pmatrix}
                \right\vert \\
                &= \begin{vmatrix}
                    u \cdot u & u \cdot v & u \cdot w\\
                    v \cdot u &v \cdot v& v \cdot w\\
                    w \cdot u & w \cdot v& w \cdot w
                \end{vmatrix}.
            \end{split}
        \end{equation}
    \end{enumerate}
\end{solution}

\subsection{Chapter 1.5}

\begin{definition}[curvature]
    \label{def:curvature}
    Let $ \alpha : I \to \mathbb{R}^3 $ be a curve parametrized by arc length. The number $ \vert \alpha^{\prime\prime} \vert = k(s) $ is called the \emph{curvature} of $ \alpha $ at $ s \in I $. 
\end{definition}

\begin{definition}[torsion]
    \label{def:torsion}
    Let $ \alpha : I \to \mathbb{R}^3 $ be a curve parametrized by arc length such that $ k(s) \neq 0 $, $ s \in I $. The number $ \tau (s) $ defined by $ b^{\prime} (s) = \tau (s) n(s) $ is called the \emph{torsion} of $ \alpha $ at $ s $. 
\end{definition}

\begin{theorem}[Fundamental Theorem of the Local Theory of Curves]
    Given differentiable functions $ k(s) > 0 $ and $ \tau (s) $, $ s\in I $, there exists a regular parametrized curve $ \alpha : T \to \mathbb{R}^3 $ such that $ s $ is the arc length, $ k(s) $ is the curvature, and $ \tau (s) $ is the torsion of $ \alpha $. Moreover, $ \alpha $ is unique up to rigid motion, i.e. for any other curve $ \overline{\alpha} $ satisfying the same conditions, there exists an orthogonal linear map $ \rho $, $ \operatorname{det} \rho > 0 $, and a translation $ t \in \mathbb{R}^3 $ such that $ \overline{\alpha} = \rho \circ \alpha + t $.
\end{theorem}

\begin{remark}
    The requirements on $ k(s) > 0 $ are required. Otherwise, consider two arc length parametrized curves on $ I = [0,1] $ given by 
    \begin{align*}
        \alpha (s) &= \begin{dcases}
            (\cos s, \sin s, 0), &\quad 0 \leq s \leq \frac{1}{2}, \\
            (\cos \frac{1}{2}, \sin \frac{1}{2}, 0) + (s- \frac{1}{2})(1,0,0), &\quad \frac{1}{2} < s \leq 1.
        \end{dcases} \\
        \overline{\alpha} (s) &= \begin{dcases}
            (\cos s, \sin s, 0), &\quad 0 \leq s \leq \frac{1}{2}, \\
            (\cos \frac{1}{2}, \sin \frac{1}{2}, 0) + (s- \frac{1}{2})(0,0,1), &\quad \frac{1}{2} < s \leq 1.
        \end{dcases}
    \end{align*}
    Both curves have curvature $ k(s) = 1 $ for $ s \in [0, 1/2) $ and $ k(s) = 0 $ for $ s \in (1/2, 1] $, and torsion $ \tau (s) = 0 $ for all $ s \in I $. However, there is no rigid motion mapping $ \alpha $ to $ \overline{\alpha} $ as the ray segments point in different directions.
\end{remark}
% 1.5.1
\begin{exercise}
    Given the parametrized curve (helix)
    \[
    \alpha(s) = \left(a\cos\frac{s}{c},\, a\sin\frac{s}{c},\, b\frac{s}{c}\right), \quad s \in \mathbb{R},
    \]
    where \(c^2 = a^2 + b^2\).

    \begin{enumerate}[label=\textbf{\alph*.}]
        \item Show that the parameter \(s\) is the arc length.
        \item Determine the curvature and the torsion of \(\alpha\).
        \item Determine the osculating plane of \(\alpha\).
        \item Show that the lines containing \(n(s)\) and passing through \(\alpha(s)\) meet the \(z\)-axis under a constant angle equal to \(\pi/2\).
        \item Show that the tangent lines to \(\alpha\) make a constant angle with the \(z\)-axis.
    \end{enumerate}
\end{exercise}

\begin{solution}
    ~
    \begin{enumerate}[label=\textbf{\alph*.}]
        \item The parameter $ s $ is the arc length since its derivative is unity:  
        \[
            \alpha^{\prime} (s) = \left(- \frac{a}{c} \sin \frac{s}{c}, \frac{a}{c} \cos \frac{s}{c}, \frac{b}{c}\right) \;\Longrightarrow\; \vert \alpha^{\prime} (s) \vert = \left(\frac{a}{c}\right)^2 + \left(\frac{b}{c}\right)^2 = 1. 
        \]
        \item The curvature $ k(s) $ is given by $ \alpha^{\prime\prime} (s) = k(s)n(s) $, so $ k(s) = \vert \alpha^{\prime\prime} (s) \vert $. Therefore, 
        \[
            k(s) = \left\vert \left(- \frac{a}{c^2} \cos \frac{s}{c}, - \frac{a}{c^2} \sin \frac{s}{c}, 0\right) \right\vert = \frac{a}{c^2}.
        \]
        The torsion $ \tau (s) $ is given by $ b^{\prime} (s) = \tau (s) n(s) $. Let's compute $ t(s) = \alpha^{\prime} (s) $, 
        \[
            n(s) = \frac{1}{k(s)} \alpha^{\prime\prime} (s) = \left(- \cos \frac{s}{c}, -\sin \frac{s}{c}, 0 \right),
        \] 
        \[
            b(s) = t(s) \wedge n(s) = \left(\frac{b}{c} \sin \frac{s}{c}, - \frac{b}{c} \cos \frac{s}{c}, \frac{a}{c}\right) \;\Longrightarrow\; b^{\prime} (s) = \left(\frac{b}{c^2} \cos \frac{s}{c}, \frac{b}{c^2} \sin \frac{s}{c}, 0\right).
        \]
        Comparing coefficients, we get $ \tau (s) = - b / c^2 $.  
        \item The osculating plane is the plane spanned by $ t(s) $ and $ n(s) $. So it is defined by the normal vector
        \[
            b(s) = t(s) \wedge n(s) = \left(\frac{b}{c} \sin \frac{s}{c}, - \frac{b}{c} \cos \frac{s}{c}, \frac{a}{c}\right).
        \]
        \item The line containing $ n(s) $ and passing through $ \alpha (s) $ is given by 
        \[
            \alpha (s) + \lambda n(s), \quad \lambda \in \mathbb{R}, 
        \]
        with direction vector $ n(s) $ such that $ n(s) \cdot (0,0,1) = 0 $, so it is always perpendicular to the $ z $-axis. 
        \item By the above computation, 
        \[
            t(s) = \left(- \frac{a}{c} \sin \frac{s}{c}, \frac{a}{c} \cos \frac{s}{c}, \frac{b}{c}\right).
        \]
        Let $ \theta $ be the angle between $ t(s) $ and the $ z $-axis. Then $ \cos \theta = t(s) \cdot (0,0,1) = \frac{b}{c} $.
    \end{enumerate}
\end{solution}

% 1.5.2
\begin{exercise}[\textbf{*}]
    Show that the torsion \(\tau\) of \(\alpha\) is given by
    \[
    \tau(s) = -\frac{\alpha'(s) \wedge \alpha''(s) \cdot \alpha'''(s)}{|k(s)|^2}.
    \]
\end{exercise}

\begin{solution}
    The torsion $ \tau (s) $ is defined as 
    \begin{equation}
        \label{def:torsion}
        b^{\prime} (s) = \tau (s) n(s).
    \end{equation}
    Given $ \alpha (s) $, we have $  = \alpha^{\prime} $, $ \alpha^{\prime\prime} = k n $, and $ \alpha^{\prime\prime\prime} = k^{\prime} n + k n^{\prime} = k^{\prime} n - k^2 t - k \tau b $ by the Frenet-Serret formulas. Let's compute the wedge product $ \alpha^{\prime} \wedge \alpha^{\prime\prime} = t \wedge k n = k b $. Thus, the triple product is 
    \[
        (\alpha ^{\prime} \wedge \alpha^{\prime\prime}) \cdot \alpha^{\prime\prime\prime} = k b \cdot (k^{\prime} n - k^2 t - k \tau b) = - k^2 \tau,
    \]
    and we have 
    \[
        \tau (s) = - \frac{(\alpha^{\prime} (s) \wedge \alpha^{\prime\prime} (s)) \cdot \alpha^{\prime\prime\prime} (s)}{k(s)^2}.
    \]
\end{solution}

% 1.5.3
\begin{exercise}
    Assume that \(\alpha(I) \subset \mathbb{R}^2\) (i.e., \(\alpha\) is a plane curve) and give \(k\) a sign as in the text.  
    Transport the vectors \(t(s)\) parallel to themselves in such a way that the origins of \(t(s)\) agree with the origin of \(\mathbb{R}^2\); the end points of \(t(s)\) then describe a parametrized curve \(s \mapsto t(s)\) called the \emph{indicatrix of tangents} of \(\alpha\).  
    Let \(\theta(s)\) be the angle from \(e_1\) to \(t(s)\) in the orientation of \(\mathbb{R}^2\).  
    Prove (a) and (b) (notice that we are assuming that \(k \neq 0\)).

    \begin{enumerate}[label=\textbf{(\alph*)}]
        \item The indicatrix of tangents is a regular parametrized curve.
        \item \(\displaystyle \frac{dt}{ds} = \left(\frac{d\theta}{ds}\right)n,\) that is, \(k = \frac{d\theta}{ds}.\)
    \end{enumerate}
\end{exercise}

\begin{solution}
    
\end{solution}

% 1.5.4
\begin{exercise}[\textbf{*}]
    Assume that all normals of a parametrized curve pass through a fixed point. Prove that the trace of the curve is contained in a circle.
\end{exercise}

\begin{solution}
    Let $ \alpha (s) $ be an arc length parametrization of the curve. Without loss of generality, assume the fixed point to be the origin. The normal at $ \alpha (s) $ passes through the origin, so it is $ \alpha (s) = \lambda (s) n(s) $ for some $ \lambda (s) $. Then,
    \[
        \frac{\mathrm{d}}{\mathrm{d}s} \vert \alpha (s) \vert^{2} = 2 \alpha (s) \cdot \alpha^{\prime} (s) = 2 \lambda (s) n(s) \cdot \alpha^{\prime} (s) = 0,
    \]
    so $ \vert \alpha (s) \vert $ is constant. We may set it to $ R > 0 $, so the trace is contained in a \textit{sphere} of radius $ R $ centered at the origin. Next, notice that
    \[
        \alpha^{\prime} = t = \lambda^{\prime} n + \lambda n^{\prime} = \lambda n - \lambda k t - \lambda \tau b \;\Longrightarrow\; (1 +\lambda k) t - \lambda^{\prime} n + \lambda \tau b = 0. 
    \]
    Since $ \{t, n, b\} $ forms an orthonormal basis for $ \mathbb{R}^3 $, we have $ 1 + k \lambda = \lambda^{\prime} = \lambda \tau = 0 $. Hence, $ \lambda \neq 0 $ is a constant, $ k = -1 / \lambda $, and $ \tau = 0 $. Therefore, $ \alpha $ is planar with constant curvature and magnitude, and hence a circle.  
\end{solution}

% 1.5.5
\begin{exercise}
    A regular parametrized curve \(\alpha\) has the property that all its tangent lines pass through a fixed point.

    \begin{enumerate}[label=\textbf{\alph*.}]
        \item Prove that the trace of \(\alpha\) is a (segment of a) straight line.
        \item Does the conclusion in part (a) still hold if \(\alpha\) is not regular?
    \end{enumerate}
\end{exercise}

\begin{solution}
    ~
    \begin{enumerate}[label=\textbf{\alph*.}]
        \item Let $ \alpha (s) $ be an arc length regular parametrization of the curve. Without loss of generality, assume the fixed point to be the origin. The tangent at $ \alpha (s) $ passes through the origin, so it satisfies $ \alpha (s) = \lambda (s) t(s) = \lambda (s) \alpha^{\prime} $. Then,
        \[
            \alpha^{\prime} (s) = \lambda^{\prime} (s) \alpha^{\prime} (s) + \lambda (s) \alpha^{\prime\prime} (s) \;\Longrightarrow\; (1 - \lambda^{\prime} (s)) \alpha^{\prime} (s) - \lambda (s) \alpha^{\prime\prime} (s) = 0 .
        \]
        Since $ \alpha(s) $ is parametrized by arc length, 
        \[ \alpha^{\prime} (s) \cdot \alpha^{\prime\prime} (s) = \frac{1}{2} \frac{\mathrm{d}}{\mathrm{d}s}\vert \alpha^{\prime} (s) \vert^2 = 0. \]
        Therefore, $ \alpha^{\prime} (s) $ and $ \alpha^{\prime\prime} (s) $ are linearly independent, and $ \lambda (s) = 0 $, $ \lambda^{\prime} (s) = 1 $ whenever $ \alpha^{\prime\prime} \neq 0 $. However, consider the set $ S = \{ s \in \mathbb{R} \vert \alpha^{\prime\prime} (s) \neq 0 \} $, which is open since $ \alpha^{\prime\prime} (s) $ is continuous and $ S = \left(\alpha^{\prime\prime}\right)^{-1} (\mathbb{R} \setminus \{0\}) $ is the inverse image of an open set. Then $ S $ contains intervals if it is non-empty, so $ \lambda (s) = 0 $ and $ \lambda^{\prime} (s) = 1 $ cannot hold for all $ s \in S $. Thus, $ S = \varnothing $, and $ \alpha^{\prime\prime} (s) = 0 $ for all $ s $. Therefore, the trace of $ \alpha (s) $ is a segment of a straight line.
        \item No, since if $ \alpha (s) $ is not regular, then $ \alpha^{\prime} (s) $ may vanish for some $ s $, at which we cannot assume $ \alpha^{\prime} (s) $ and $ \alpha^{\prime\prime} (s) $ are linearly independent.
    \end{enumerate}
\end{solution}

% 1.5.6
\begin{exercise}
    A \emph{translation} by a vector \(v\) in \(\mathbb{R}^3\) is the map \(A: \mathbb{R}^3 \to \mathbb{R}^3\) given by  
    \(A(p) = p + v, \, p \in \mathbb{R}^3.\)  
    A linear map \(\rho: \mathbb{R}^3 \to \mathbb{R}^3\) is an \emph{orthogonal transformation} when  
    \(\rho u \cdot \rho v = u \cdot v\) for all vectors \(u, v \in \mathbb{R}^3.\)  
    A \emph{rigid motion} in \(\mathbb{R}^3\) is the result of composing a translation with an orthogonal transformation with positive determinant (this last condition is included because we expect rigid motions to preserve orientation).

    \begin{enumerate}[label=\textbf{\alph*.}]
        \item Demonstrate that the norm of a vector and the angle \(\theta\) between two vectors, \(0 \le \theta \le \pi\), are invariant under orthogonal transformations with positive determinant.
        \item Show that the vector product of two vectors is invariant under orthogonal transformations with positive determinant.  
        Is the assertion still true if we drop the condition on the determinant?
        \item Show that the arc length, the curvature, and the torsion of a parametrized curve are (whenever defined) invariant under rigid motions.
    \end{enumerate}
\end{exercise}

\begin{solution}
    ~
    \begin{enumerate}[label=\textbf{\alph*.}]
        \item Since an orthogonal transformation preserves the inner product, it also preserves $ \sqrt{u \cdot u} $ and $ \cos^{-1} \frac{u \cdot v}{\vert u \vert \vert v \vert} $ for all vectors $ u, v \in \mathbb{R}^3 $. 
        \item Let $ u, v, w \in \mathbb{R}^3 $ be arbitrary vectors. Then consider the inner product of $ \rho u \wedge \rho v $ with $ w $:
        \begin{equation*}
            \begin{split}
                \left(\rho u \wedge \rho v \right) \cdot w &= \operatorname{det} (\rho u, \rho v, w) \\
                &= \operatorname{det} (\rho) \operatorname{det} (u, v, \rho^{-1} w) \\
                &= \operatorname{det}(\rho) \left((u \wedge v) \cdot \rho^{-1} w\right) \\
                &= \operatorname{det}(\rho) \left(\rho (u \wedge v) \cdot w\right),
            \end{split}
        \end{equation*}
        for all $ w \in \mathbb{R}^3 $. Hence, $ \rho u \cdot \rho v = \operatorname{det}(\rho) \rho (u \wedge v) = \rho (u \wedge v) $ whenever $ \operatorname{det}(\rho) = 1 $. Therefore, the assertion is false when $ \operatorname{det}(\rho) \neq 1 $. 
        \item Let $ R = A \circ \rho : \mathbb{R}^3 \to \mathbb{R}^3 $ be a rigid motion. Then, for all parametrized curves $ \alpha (t) $, we have $ \tilde{\alpha} = R \alpha = \rho \alpha + p $. The arc length is then given by
        \[
            \tilde{s}(t) = \int \mathrm{d}\tau \, \vert \tilde{\alpha}^{\prime} (\tau) \vert = \int \mathrm{d}\tau \, \vert \rho \alpha^{\prime} (\tau) \vert = \int \mathrm{d}\tau \, \vert \alpha^{\prime} (\tau) \vert = s(t),
        \]
        since $ \rho $ preserves the norm. Now, we use the arc length parametrization. The curvature is given by
        \[
            \tilde{k}(\tilde{s}) = \vert \tilde{\alpha}^{\prime\prime} (\tilde{s}) \vert = \vert \rho \alpha^{\prime\prime} (s) \vert = \vert \alpha^{\prime\prime} (s) \vert = k(s),
        \]
        and the torsion is given by
        \begin{equation*}
            \begin{split}
                \tilde{\tau} (\tilde{s}) &= - \frac{(\tilde{\alpha}^{\prime} (\tilde{s}) \wedge \tilde{\alpha}^{\prime\prime} (\tilde{s})) \cdot \tilde{\alpha}^{\prime\prime\prime} (\tilde{s})}{\tilde{k}(\tilde{s})^2} = - \frac{(\rho \alpha^{\prime} (s) \wedge \rho \alpha^{\prime\prime} (s)) \cdot \rho \alpha^{\prime\prime\prime} (s)}{k(s)^2} \\
                &= - \frac{\rho (\alpha^{\prime} (s) \wedge \alpha^{\prime\prime} (s)) \cdot \rho \alpha^{\prime\prime\prime} (s)}{k(s)^2} = - \frac{(\alpha^{\prime} (s) \wedge \alpha^{\prime\prime} (s)) \cdot \alpha^{\prime\prime\prime} (s)}{k(s)^2} = \tau (s),
            \end{split}
        \end{equation*}
        since $ \rho $ preserves the vector and the inner product.
    \end{enumerate}    
\end{solution}

% 1.5.7
\begin{exercise}[\textbf{*}]
    Let $\alpha:I\to\mathbb{R}^2$ be a regular parametrized plane curve (arbitrary parameter), and define $n=n(t)$ and $k=k(t)$ as in Remark~1. Assume that $k(t)\neq 0$, $t\in I$. In this situation, the curve
    \begin{equation}
        \label{equ:evolute}
        \beta(t)=\alpha(t)+\frac{1}{k(t)}\,n(t),\qquad t\in I,
    \end{equation}
    is called the \emph{evolute} of $\alpha$ (Fig.~1--17).
    \begin{enumerate}[label=\textbf{\alph*.}]
        \item Show that the tangent at $t$ of the evolute of $\alpha$ is the normal to $\alpha$ at $t$.
        \item Consider the normal lines of $\alpha$ at two neighboring points $t_1,t_2$, $t_1\neq t_2$. Let $t_1$ approach $t_2$ and show that the intersection points of the normals converge to a point on the trace of the evolute of $\alpha$.
    \end{enumerate}
\end{exercise}

\begin{solution} 
    ~
    \begin{enumerate}[label=\textbf{\alph*.}]
        \item Let $ \beta $ be the evolute. By the chain rule, we have 
        \begin{equation*}
            n^{\prime} (t) = \frac{\mathrm{d}n}{\mathrm{d}s} \frac{\mathrm{d}s}{t} = -k(t) \frac{\alpha^{\prime} (t)}{\vert \alpha ^{\prime} (t) \vert} \vert \alpha^{\prime} (t) \vert = - k(t)\, \alpha^{\prime} (t).
        \end{equation*}
        By direct differentiation of $ \beta $, we get 
        \begin{equation*}
            \beta^{\prime} (t) = \alpha^{\prime} (t) + \frac{- k(t)^{2} \, \alpha^{\prime} (t) - n(t)\, k(t)}{k(t)^{2}} = - \frac{k^{\prime} (t)}{k(t)^{2}}n(t). 
        \end{equation*}
        Hence, the tangent at $ t $ of $ \beta $ is precisely $ n(t) $. 
        \item Let the normal be given by $ n(t) = (a(t), b(t)) $, then $ a^{\prime} (t) \neq 0 $ or $ b^{\prime} (t) \neq 0 $ for all $ t $ since $ \alpha $ is regular. Take some $ t_{2} \in I $, assume without loss of generality that $ a^{\prime} (t_{2}) \neq 0 $. For $ t \in J = (t_2 - \delta, t_2 + \delta) $, we have 
        \begin{equation*}
            \vert a^{\prime} (t_2) \vert - \left\vert \frac{a_{t_2} - a_{t}}{t_2 - t} \right\vert \leq \left\vert a^{\prime} (t_2) \frac{a_{t_2} - a_{t}}{t_2 - t} \right\vert < \frac{1}{2} \vert a^{\prime} (t_2) \vert, 
        \end{equation*}
        and 
        \begin{equation*}
            \left\vert \frac{a(t_2) - a(t)}{t_2 - t} \right\vert > \frac{\vert a^{\prime} (t_2) \vert }{2} > 0, 
        \end{equation*}
        hence $ a(t) \neq a(t_2) $ for any $ t $ in a neighborhood of $ t_2 $. Therefore, if we fix $ t_1 \in J $, $ t_1 \neq t_2 $, then the normal lines $ N_1, N_2 $ of $ \alpha $ at $ t_1, t_2 $ will have a unique intersection. $ L_1, L_2 $ are well-defined given that $ n(t) \neq 0 $ for all $ t \in I $. Let $ h \in \mathbb{R}^{2} $ be the intersection point, then 
        \begin{equation*}
            h = \alpha (t_1) + p_1 n(t_1) = \alpha (t_2) + p_2 n(t_2),
        \end{equation*}  
        where $ p_1, p_2 \in I $ are constants. We shall show that as $ t_1 \to t_2 $, $ p_2 \to 1 / k (t_2) $. The area spanned by $ n(t_1) $ and $ \alpha (t_1) $ is 
        \begin{equation*}
            \operatorname{det} (\alpha (t_1), n(t_1)) = \operatorname{det} (\alpha (t_2), n(t_1)) + p_1\operatorname{det} (n (t_2), n(t_1)),
        \end{equation*}
        then 
        \begin{equation*}
            p_2 = \frac{\operatorname{det}(\alpha (t_1) - \alpha (t_2), \; n(t_1))}{\operatorname{det}(n (t_2) , \; n(t_1)) }. 
        \end{equation*}
        Taking the limit $ t_1 \to t_2 $ gives, by L'Hôpital's rule,
        \begin{equation*}
            \begin{split}
                \lim_{t_1 \to t_2} p_2 &= \frac{\operatorname{det}(\alpha^{\prime} (t_2), n(t_2))}{\operatorname{det}(n(t_2), n^{\prime} (t_2))} = \frac{1}{k(t_2)} \\
                &= \lim_{t_1 \to t_2} \frac{\operatorname{det}(\alpha^{\prime} (t_1), \; n(t_1)) - \operatorname{det} (\alpha (t_1) - \alpha (t_2), \; -k(t_1) \, \alpha^{\prime} (t_1))}{\operatorname{det}(n(t_2), - k(t_1)\, \alpha^{\prime} (t_1))} \\
                &= \lim_{t_1 \to t_2} \frac{\vert \alpha^{\prime} (t_1) \vert }{k(t_1) \, \vert  \vert } + \lim_{t_1 \to t_2} \frac{\operatorname{det}(k(t_1) \, \alpha^{\prime} (t_1), \; \alpha(t_1) - \alpha (t_2))}{k(t_1) \, \vert \alpha^{\prime} (t_1) \vert } \\
                &= \frac{1}{k(t_2)}. 
            \end{split}
        \end{equation*}
        Therefore, 
        \begin{equation*}
            \lim_{t_1 \to t_2} h = \alpha (t_2) + \frac{1}{k(t_2)} n(t_2) = \beta (t_2), 
        \end{equation*}
        which is a point on the evolute of $ \alpha $.
    \end{enumerate}
\end{solution}

% 1.5.8
\begin{exercise}
    The trace of the parametrized curve (arbitrary parameter)
    \begin{equation}
        \alpha(t)=(t,\cosh t),\qquad t\in\mathbb{R},
    \end{equation}
    is called the \emph{catenary}.
    \begin{enumerate}[label=\textbf{\alph*.}]
    \item Show that the signed curvature (cf.\ Remark~1) of the catenary is
    \begin{equation}
        k(t) = \frac{1}{\cosh^2 t}.
    \end{equation}
    \item Show that the evolute (cf.\ Exercise~7) of the catenary is
    \begin{equation}
        \beta(t)=\bigl(t-\sinh t\,\cosh t,\; 2\cosh t\bigr).
    \end{equation}
    \end{enumerate}
\end{exercise}

\begin{solution}
    ~

    To keep the notation unambiguous, we will denote the (unit) tangent vector by $T$. Recall that $ n(t) = T^{\prime} (t) / \vert T^{\prime} (t) \vert $, by remark 1, the signed curvature is given by
    \begin{equation}
        k(t) \, n(t) = \frac{dT}{ds} = \frac{dT/dt}{ds/dt} = \frac{T'(t)}{|\alpha'(t)|}.
    \end{equation}
    Plugging in the expression for $ n(t) $ simplifies it to 
    \begin{equation}
        \label{equ:curvature}
        k(t) = \frac{|T'(t)|}{|\alpha'(t)|}.
    \end{equation}

    \begin{enumerate}[label=\textbf{\alph*.}]
        \item We have $ \alpha^{\prime} (t) = (1, \sinh t) $, $ |\alpha^{\prime} (t)| = \sqrt{1+\sinh^2 t} = \cosh t $. Then $ T(t) = \alpha^{\prime} (t) / \vert \alpha^{\prime} (t) \vert = \operatorname{sech} t (1, \sinh t) $ and 
        \begin{equation*}
            T^{\prime} (t) = \operatorname{sech}^{2} t \left(- \sinh t, 1\right), 
        \end{equation*}
        \begin{equation*}
            \vert T^{\prime} (t) \vert = \operatorname{sech}^{2} t \sqrt{\sinh^2 t + 1} = \operatorname{sech} t,
        \end{equation*}
        By equation \eqref{equ:curvature}, we have
        \begin{equation}
            k(t) = \frac{\operatorname{sech} t}{\cosh t} = \operatorname{sech}^2 t = \frac{1}{\cosh^2 t}.
        \end{equation}
        \item By definition in Exercise 7, the evolute is given by
        \begin{equation}
            \begin{split}
                \beta(t) &= \alpha(t) + \frac{1}{k(t)} n(t) \\
                &= (t, \cosh t) + \cosh^2 t \, \operatorname{sech} t (-\sinh t, 1) \\
                &= (t - \sinh t \cosh t, 2 \cosh t).
            \end{split}
        \end{equation}
    \end{enumerate}
\end{solution}

% 1.5.9
\begin{exercise}
    Given a differentiable function $k(s)$, $s\in I$, show that the parametrized plane curve having $k(s)=k$ as curvature is given by
    \begin{equation}
        \alpha(s)=\left(\int \mathrm{d}s \, \cos\theta(s)+a,\;\int \mathrm{d}s \,\sin\theta(s)+b\right),
    \end{equation}
    where
    \begin{equation}
        \theta(s)=\int \mathrm{d}s \, k(s) +\varphi,
    \end{equation}
    and that the curve is determined up to a translation of the vector $(a,b)$ and a rotation of the angle $\varphi$.
\end{exercise}

\begin{solution}
    Let $ \alpha (s) $ be as given, we have 
    \begin{equation}
        \alpha'(s) = \left( \cos \theta(s), \sin \theta(s) \right) = \left( \cos \left( \int k(s)\,ds+\varphi \right), \sin \left( \int k(s)\,ds+\varphi \right) \right), 
    \end{equation}
    and
    \begin{equation}
        \alpha^{\prime \prime} (s) = k(s) \left( -\sin \theta(s), \cos \theta(s) \right),
    \end{equation}
    hence $ \vert \alpha^{\prime \prime} (s) \vert = k(s) $. By the definition of translation, the curve is determined up to a translation of the vector $ (a,b) $, so suppose $ a = b = 0 $. Now suppose we rotate the curve by an angle $ \varphi $ counterclockwise, then the new curve $ \tilde{\alpha} (s) $ is given by
    \begin{equation*}
        \begin{split}
            \tilde{\alpha} (s) &= \begin{pmatrix}
            \cos \varphi & -\sin \varphi \\
            \sin \varphi & \cos \varphi
            \end{pmatrix} \alpha (s) \\
            &= \begin{pmatrix}
                \cos \varphi \int \mathrm{d}s\, \cos \theta (s) -\sin \varphi \int \mathrm{d}s\, \sin \theta (s) \\
                \sin \varphi \int \mathrm{d}s\, \cos \theta (s) + \cos \varphi \int \mathrm{d}s\, \sin \theta (s)
            \end{pmatrix} \\ 
            &= \begin{pmatrix}
                \int \mathrm{d}s\, \cos (\theta (s) + \varphi) \\
                \int \mathrm{d}s\, \sin (\theta (s) + \varphi)
            \end{pmatrix}.
        \end{split}
    \end{equation*}
    Thus, the curve is determined up to an arbitrary rotation of the angle $ \varphi $.

    \begin{remark}
        This exercises shows how to construct a curve with any given curvature functions $ k(s) $, up to a translation and rotation. This is a special case of the \textbf{Fundamental Theorem of the Local Theory of Curves}.
    \end{remark}
\end{solution}

% 1.5.10
\begin{exercise}
    Consider the map
    \begin{equation}
        \alpha(t) = 
        \begin{cases} 
            (t, 0, e^{-1/t^{2}}), & t > 0 \\
            (t, e^{-1/t^{2}}, 0), & t < 0 \\
            (0, 0, 0), & t = 0
        \end{cases}
    \end{equation}
    \begin{enumerate}[label=\textbf{\alph*.}]
        \item Prove that $\alpha$ is a differentiable curve.
        \item Prove that $\alpha$ is regular for all $t$ and that the curvature $k(t) \neq 0$, for $t \neq 0$, $t \neq \pm\sqrt{2/3}$, and $k(0) = 0$.
        \item Show that the limit of the osculating planes as $t \to 0, t > 0$, is the plane $y = 0$ but that the limit of the osculating planes as $t \to 0, t < 0$, is the plane $z = 0$ (this implies that the normal vector is discontinuous at $t = 0$ and shows why we excluded points where $k = 0$).
        \item Show that $\tau$ can be defined so that $\tau \equiv 0$, even though $\alpha$ is not a plane curve.
    \end{enumerate}
\end{exercise}

\begin{solution}
    ~
    \begin{enumerate}[(a)]
        \item The curve $ \alpha $ is differentiable if $ \alpha^{\prime} $ exists everywhere. For $ t>0 $ and $ t<0 $ it is made of elementary functions, so it is differentiable. At $ t=0 $, the x coordinate is differentiable, so consider the z coordinateo only. 
        \begin{lemma}
            \label{lem:differentiable}
            The map
            \begin{equation}
                f(x) = \begin{dcases}
                    e^{-1/x^{2}}, & x > 0;\\
                    0, & x \leq 0.
                \end{dcases}
            \end{equation} 
            is differentiable at $ x = 0 $ and $ f^{(n)}(0) = 0 $.
        \end{lemma}
        \begin{proof}
            Let $ f(x) = e^{-1/x^{2}} $, notice that 
            \begin{equation}
                f(x) \leq n! x^{2n} \quad \text{for all } n.
            \end{equation}
            Thus, for $ n=1 $ we have $ f^{\prime} (0) = \lim_{x \to 0} f(x)/x = 0 $ by the squeeze theorem. Assume that $ f^{(k)}(0) = 0 $ for all $ k < n $. By induction we know that $ f^{(k)} $ is of the form $ f^{(m)} (x) = f(x) \sum_{r=1}^{N}a_{r} x^{-r} $ for $ x>0 $, so choosing some $ n $ large enough such that 
            \begin{equation*}
                f^{(k+1)}(x) \leq n! x^{2n} \sum_{r=1}^{N}a_{r} x^{-r} \leq Cx^m
            \end{equation*} 
            for some constant $ C $, we have $ f $ is $ (k+1) $ times differentiable and $ f^{(k+1)}(0) = 0 $. By induction we are done. 
        \end{proof}
        By Lemma (\ref{lem:differentiable}), $ \alpha $ is differentiable. 
        \item The curve has derivative 
        \begin{equation*}
            \alpha^{\prime} = 
            \begin{dcases}
                \left(1, 0, \frac{2}{t^{3}} e^{-1/t^{2}}\right), & t > 0,\\
                \left(1, \frac{2}{t^{3}} e^{-1/t^{2}}, 0\right), & t < 0,\\
                (1, 0, 0), & t = 0.
            \end{dcases}
        \end{equation*}
        Since $ e^{-1/t^{2}} $ is always positive, $ \alpha^{\prime}(t) \neq 0 $ for all $ t $, so $ \alpha $ is regular. Next, we compute the curvature $ k(t) $. 
        \begin{lemma}
            \label{lem:curvature}
            For a regular curve $ \alpha(t) $, the curvature is given by
            \begin{equation}
                \label{eq:Tprime}
                k(t) = \frac{|\alpha^{\prime}(t) \wedge \alpha^{\prime\prime}(t)|}{|\alpha^{\prime}(t)|^3}.
            \end{equation}
        \end{lemma}
        \begin{proof}
            Let $ \alpha \colon I \to \mathbb{R}^3 $ be a regular curve. Then, we have $ T^{\prime} (t(s)) = k(t(s)) N(t(s)) $, where $ t(s) $ is the reparametrization by arc length. Then $ \lvert T^{\prime} (t(s)) \rvert = k(t(s)) \lvert N(t(s)) \rvert = k(t(s)) $. The left hand side is $ \mathrm{d}T / \mathrm{d}s = \left( \mathrm{d}T/\mathrm{d}t \right)\left(\mathrm{d}t / \mathrm{d}s\right) = \left( \mathrm{d}T/\mathrm{d}t \right) / \lvert \alpha^{\prime} (t) \rvert $. Moreover, 
            \begin{equation}
                \frac{\mathrm{d}T}{\mathrm{d}t} = \frac{\lvert \alpha^{\prime} \rvert^{2} \alpha^{\prime\prime} - \left(\alpha^{\prime} \cdot \alpha^{\prime\prime} \right)\alpha^{\prime} }{\lvert \alpha^{\prime} \rvert^{3} } = \frac{\alpha^{\prime} \wedge (\alpha^{\prime\prime} \wedge \alpha^{\prime} )}{\lvert \alpha^{\prime} \rvert^{3}}.  
            \end{equation} 
            Since $ \alpha^{\prime} \perp \alpha^{\prime\prime} \wedge \alpha^{\prime} $, 
            \[
                k(t(s)) = \lvert T^{\prime}(t(s)) \rvert = \frac{|\alpha^{\prime}(t) \wedge \alpha^{\prime\prime}(t)|}{|\alpha^{\prime}(t)|^3}.
            \]
        \end{proof}
        We have $ \alpha^{\prime} (t) $ given above, and 
        \begin{equation*}
            \begin{split}
                \alpha^{\prime\prime} &= 
                \begin{dcases}
                    \left(0, 0, \left(\frac{4}{t^{6}} - \frac{6}{t^{4}}\right) e^{-1/t^{2}}\right), & t > 0,\\
                    \left(0, \left(\frac{4}{t^{6}} - \frac{6}{t^{4}}\right) e^{-1/t^{2}}, 0\right), & t < 0,\\
                    (0, 0, 0), & t = 0.
                \end{dcases} \\
                \alpha^{\prime} \wedge \alpha^{\prime\prime} &= 
                \begin{dcases}
                    \left(0, - \left(\frac{4}{t^{6}} - \frac{6}{t^{4}}\right) e^{-1/t^{2}}, 0\right), & t > 0,\\
                    \left(0, 0, \left(\frac{4}{t^{6}} - \frac{6}{t^{4}}\right) e^{-1/t^{2}}\right), & t < 0,\\
                    (0, 0, 0), & t = 0.
                \end{dcases}
            \end{split}
        \end{equation*}
        Using Lemma \ref{lem:curvature}, we have
        \begin{equation}
            k(t) = 
            \begin{dcases}
                \left| \left(\frac{4}{t^{6}} - \frac{6}{t^{4}}\right) e^{-1/t^{2}} \right| \left.\middle/\right. \left(1 + \frac{4}{t^{6}} e^{-2/t^{2}}\right)^{3/2}, & t \neq 0,\\
                0, & t = 0.
            \end{dcases}
        \end{equation}
        From above we know $ k(t) = 0$ when and only when  $ t = 0 $ and $ t = \pm \sqrt{2/3} $.
        \item The osculating plane is determined by the normal vector $ N(t) $ and the tangent vector $ T(t) $. By equation (\ref{eq:Tprime}) and the definition $ \mathrm{d}T(t(s)) / \mathrm{d}s = k(t(s)) N(t(s))$, the normal vector is
        \begin{equation}
            \begin{split}
                N(t) &= \frac{1}{k(t)} \frac{\mathrm{d}T(t(s))}{\mathrm{d}s} \\
                &= \frac{\alpha^{\prime}(t) \wedge (\alpha^{\prime\prime} (t) \wedge \alpha^{\prime}(t) )}{\lvert \alpha^{\prime} (t) \rvert^{4}} \cdot \frac{|\alpha^{\prime}(t)|^3}{|\alpha^{\prime}(t) \wedge \alpha^{\prime\prime}(t)|} \\
                &= \frac{\alpha^{\prime}(t) \wedge (\alpha^{\prime\prime} (t) \wedge \alpha^{\prime}(t) )}{\lvert \alpha^{\prime}(t) \rvert |\alpha^{\prime}(t) \wedge \alpha^{\prime\prime}(t)|}.
            \end{split}
        \end{equation}
        For $ t > 0 $, we have
        \begin{equation*}
            N(t) = \left(1 + \frac{4}{t^6}e^{-1/t^{2}}\right)^{-1/2} \left(-\frac{2}{t^3}e^{-1/t^{2}}, 0, 1\right)
        \end{equation*}
        and
        \begin{equation*}
            T(t) = \left(1 + \frac{4}{t^6}e^{-1/t^{2}}\right)^{-1/2} \left(1, 0, \frac{2}{t^3}e^{-1/t^{2}}\right),
        \end{equation*} 
        hence $ N_P = \lim_{t \to 0^+} T(t)\wedge N(t) = (0, 0, 1) \wedge (1, 0, 0) = (0,1,0) $. Furthermore, $ \lim_{t \to 0^+} \alpha (t) = (0,0,0)$, so the osculating plane is $ y=0 $. 

        On the other hand, for $ t<0 $, we have 
        \begin{equation*}
            N(t) = \left(1 + \frac{4}{t^6}e^{-1/t^{2}}\right)^{-1/2} \left(-\frac{2}{t^3}e^{-1/t^{2}}, 1, 0\right)
        \end{equation*}
        and
        \begin{equation*}
            T(t) = \left(1 + \frac{4}{t^6}e^{-1/t^{2}}\right)^{-1/2} \left(1, \frac{2}{t^3}e^{-1/t^{2}}, 0\right),
        \end{equation*}
        hence $ N_P = \lim_{t \to 0^-} T(t) \wedge N(t) = (0, 1, 0) \wedge (1, 0, 0) = (0,0,-1) $. Furthermore, $ \lim_{t \to 0^-} \alpha (t) = (0,0,0)$, so the osculating plane is $ z=0 $. Notice that $ N(t) $ is discontinuous at $ t=0 $, thus undefined there. 
        \item Since $ k(0) = k(\pm \sqrt{2/3} ) = 0 $, $ N(0) $ and $ N(\pm\sqrt{2/3} ) $ are not well-defined. Therefore, we can define $ \tau $ to be zero at these points. For $ t \neq 0, \pm \sqrt{2/3}$, we have 
        \begin{equation*}
            B(t) = T(t) \wedge N(t) = 
            \begin{dcases}
                -(0,1,0), & t>0, \\
                (0,0,1), & t<0. \\
            \end{dcases}
        \end{equation*} 
        The binormal vector $ B(t) $ is constant on $ I \backslash \{0\} $, so $ B^{\prime} (s) = B^{\prime} (t) \cdot \vert \alpha^{\prime} (t) \vert^{-1} = 0 = \tau (t(s)) N(t(s)) $. Hence we can choose $ \tau (t) \equiv 0 $ for $ t \in I \backslash \{0, \pm \sqrt{2/3} \} $. This is an example of \textbf{a curve with identically zero torsion that is not a plane curve}.  
    \end{enumerate}
\end{solution}

% 1.5.11
\begin{exercise}
    One often gives a plane curve in polar coordinates by $\rho=\rho(\theta)$, $a\leq \theta \leq b$.
    \begin{enumerate}[label=\textbf{\alph*.}]
        \item Show that the arc length is
        \begin{equation}
            \int_a^b \mathrm{d}\theta \, \sqrt{\rho^2+(\rho')^2},
        \end{equation}
        where the prime denotes the derivative relative to $\theta$.
        \item Show that the curvature is
        \begin{equation}
            k(\theta)=\frac{2(\rho')^2-\rho\rho''+\rho^2}{\bigl((\rho')^2+\rho^2\bigr)^{3/2}}.
        \end{equation}
    \end{enumerate}
\end{exercise}

\begin{solution}
    ~ 

    \begin{enumerate}[label=\textbf{\alph*.}]
        \item Calculate the curve vector in Cartesian coordinates:
        \begin{equation*}
            \alpha(\theta) = (\rho(\theta) \cos \theta, \; \rho(\theta) \sin \theta),
        \end{equation*}
        Then 
        \begin{equation*}
            \alpha'(\theta) = (\rho'(\theta) \cos \theta - \rho(\theta) \sin \theta, \; \rho'(\theta) \sin \theta + \rho(\theta) \cos \theta),
        \end{equation*}
        and computing the norm gives
        \begin{equation*}
            \vert \alpha^{\prime} (\theta) \vert = \sqrt{(\rho'(\theta))^2 + \rho^2(\theta)}. 
        \end{equation*}
        The arclength is defined to be  
        \begin{equation}
            s(a,b) = \int_a^b \mathrm{d}\theta \, |\alpha'(\theta)| = \int_a^b \mathrm{d}\theta \, \sqrt{\rho^2 + (\rho')^2} .
        \end{equation}
        \item The unit tangent is 
        \begin{equation*}
            T(\theta) = \frac{\alpha'(\theta)}{|\alpha'(\theta)|} = \frac{1}{\sqrt{(\rho')^2 + \rho^2}} (\rho'(\theta) \cos \theta - \rho(\theta) \sin \theta, \; \rho'(\theta) \sin \theta + \rho(\theta) \cos \theta).
        \end{equation*}
        Then we calculate $ T'(\theta) $ and its magnitude, where prime denotes derivative with respect to $ \theta $. After some cumbersome algebra, we get 
        \begin{equation*}
            T'(\theta) = \frac{1}{((\rho')^2 + \rho^2)^{3/2}} \left( (2(\rho')^2 - \rho \rho'' + \rho^2) (-\sin \theta, \; \cos \theta) \right),
        \end{equation*}
        By equation \eqref{equ:curvature}, we have
        \begin{equation}
            k(\theta) = \frac{|T'(\theta)|}{|\alpha'(\theta)|} = \frac{2(\rho')^2 - \rho \rho'' + \rho^2}{((\rho')^2 + \rho^2)^{3/2}}.
        \end{equation}
    \end{enumerate}
\end{solution}

% 1.5.12
\begin{exercise}
    Let \(\alpha: I \to \mathbb{R}^3\) be a regular parametrized curve (not necessarily by arc length) and let \(\beta: J \to \mathbb{R}^3\) be a reparametrization of \(\alpha(I)\) by the arc length \(s = s(t)\), measured from \(t_0 \in I\) (see Remark~2).  
    Let \(t = t(s)\) be the inverse function of \(s\) and set \(d\alpha/dt = \alpha'\), \(d^2\alpha/dt^2 = \alpha''\), etc.  
    Prove that

    \begin{enumerate}[label=\textbf{\alph*.}]
        \item \(\displaystyle \frac{dt}{ds} = \frac{1}{|\alpha'|}, \quad \frac{d^2t}{ds^2} = -\frac{\alpha' \cdot \alpha''}{|\alpha'|^4}.\)
        \item The curvature of \(\alpha\) at \(t \in I\) is  
        \[
        k(t) = \frac{|\alpha' \wedge \alpha''|}{|\alpha'|^3}.
        \]
        \item The torsion of \(\alpha\) at \(t \in I\) is  
        \[
        \tau(t) = -\frac{(\alpha' \wedge \alpha'') \cdot \alpha'''}{|\alpha' \wedge \alpha''|^2}.
        \]
        \item If \(\alpha: I \to \mathbb{R}^2\) is a plane curve \(\alpha(t) = (x(t), y(t))\), the signed curvature (see Remark~1) of \(\alpha\) at \(t\) is  
        \[
        k(t) = \frac{x'y'' - x''y'}{\left((x')^2 + (y')^2\right)^{3/2}}.
        \]
    \end{enumerate}
\end{exercise}

\begin{solution}
    ~
    \begin{enumerate}[label=\textbf{\alph*.}]
        \item By the definition of arc length, we have
        \[
            s(t) = \int _{t_0}^t \mathrm{d}u \, |\alpha'(u)| \;\Longrightarrow\; \frac{ds}{dt} = |\alpha'(t)| \;\Longrightarrow\; \frac{dt}{ds} = \frac{1}{|\alpha'|}.
        \]
        Differentiating again gives
        \[
            \frac{\mathrm{d}^2 t}{\mathrm{d}s^2} = \frac{1}{|\alpha^{\prime} |} \frac{\mathrm{d}}{\mathrm{d}t} \left( \frac{1}{|\alpha^{\prime}|} \right) = - \frac{\alpha^{\prime} \cdot \alpha^{\prime\prime}}{|\alpha^{\prime}|^4}.
        \]
        \item For a space curve, we have $ k(s) = \vert \alpha^{\prime\prime} (s) \vert $ in the arc length parametrization. By the chain rule, so $ k(s(t)) = \vert \alpha^{\prime\prime} (s(t)) \vert $. By the chain rule, we have
        \[
            \alpha^{\prime} = \frac{\mathrm{d}}{\mathrm{d}t}\alpha (s(t))
        \]
        \item *
    \end{enumerate}
\end{solution}

% 1.5.13
\begin{exercise}[\textbf{*}]
    Assume that \(\tau(s) \neq 0\) and \(k'(s) \neq 0\) for all \(s \in I\).  
    Show that a necessary and sufficient condition for \(\alpha(I)\) to lie on a sphere is that
    \[
    R^2 + (R')^2 T^2 = \text{const.},
    \]
    where \(R = 1/k\), \(T = 1/\tau\), and \(R'\) is the derivative of \(R\) with respect to \(s\).
\end{exercise}

\begin{solution}
    Suppose $ \alpha $ lies on a sphere of radius $ r $ centered at $ 0 $, then $ \vert \alpha \vert = R $. Differentiating three times gives the following equations
    \[
        \alpha \cdot \alpha^{\prime} = 0,
    \]
    \[
        \alpha^{\prime} \cdot \alpha^{\prime} + \alpha \cdot \alpha^{\prime\prime} = 0 \;\Longrightarrow\; \alpha \cdot \alpha^{\prime\prime} = -1 \quad \quad (*),
    \]  
    \[
      \alpha^{\prime} \cdot \alpha^{\prime\prime} + \alpha \cdot \alpha^{\prime\prime\prime} = 0 \;\Longrightarrow\; \alpha \cdot \alpha^{\prime\prime\prime} = 0  \quad \quad (^{\ast} ),
    \]
    where we suppressed $ s $ and used $ \alpha^{\prime} \cdot \alpha^{\prime\prime} = 0 $. Let's write down the Frenet equations: 
    \begin{equation}
        \label{eq:frenet}
        t^{\prime} = k n, \quad n^{\prime} = - k t - \tau b, \quad b^{\prime} = \tau n.
    \end{equation}
    By $ (*) $, we have $ k \alpha \cdot n = -1 $, so $ \alpha \cdot n = - 1 / k $.
    By $ (**) $ and $ \alpha^{\prime\prime\prime} = k^{\prime} n + k n^{\prime} $, we have the relation $ k^{\prime} \alpha \cdot n + k \alpha \cdot n^{\prime} = 0 $. Substitute the Frenet equations \eqref{eq:frenet} into it gives 
    \[
        k^{\prime} \left(-\frac{1}{k}\right) + k \alpha \cdot (-k t + \tau b) = - \frac{k^{\prime}}{k} + k \tau \alpha \cdot b = 0.  
    \]  
    Now we have $ \alpha \cdot n = - 1 / k $, $ \alpha \cdot t = 0 $, and $ \alpha \cdot b = \frac{k^{\prime}}{\tau k^2} $, so we can write $ \alpha $ in the Frenet frame $ \{t, n, b\} $ as 
    \[
        \alpha = - \frac{1}{k} n + \frac{k^{\prime}}{\tau k^2} b,
    \]
    hence 
    \[
        \vert \alpha \vert^2 = \frac{1}{k^2} + \frac{(k^{\prime})^2}{\tau^2 k^4} = R^2 + (R^{\prime})^2 T^2, \quad k = \frac{1}{R}, \; \tau = \frac{1}{T}. 
    \]
    Conversely, suppose $ R^2 + (R^{\prime})^2 T^2 = \text{const} $, where $ R = 1 / k $ and $ T = 1 / \tau $. Motivated by the Frenet frame formula for $ \alpha $, consider the quantity 
    \[
        \beta = \alpha + \frac{1}{k} n - \frac{k^{\prime}}{\tau k^2} b,
    \]  
    then 
    \begin{equation*}
        \begin{split}
            \frac{\mathrm{d}\beta}{\mathrm{d}s} &= t + \frac{(-k t - \tau b) k - n k^{\prime}}{k^2} - \frac{\mathrm{d}}{\mathrm{d}s} \left(\frac{k^{\prime}}{\tau k^2}\right) b - \frac{k^{\prime}}{\tau k^2} \tau n \\
            &= \frac{\tau b}{k} - \frac{\mathrm{d}}{\mathrm{d}s} \left(\frac{k^{\prime}}{\tau k^2}\right) b \\
            &= \frac{k^2 \tau b}{k^{\prime}} \left[ \frac{k^{\prime}}{k^3} - \frac{k^{\prime}}{\tau k^2} \frac{\mathrm{d}}{\mathrm{d}s} \left(\frac{k^{\prime}}{\tau k^2}\right) \right] \\
            &= \frac{k^2 \tau b}{2 k^{\prime}} \frac{\mathrm{d}}{\mathrm{d}s}\left[\frac{1}{k^2} + \left(\frac{k^{\prime}}{\tau k^2}\right)^2 \right] = 0.
        \end{split}
    \end{equation*}
    Therefore, $ \beta (s) = \beta (0) $ is a constant vector, and we have 
    \[
        \vert \alpha - \beta (0) \vert = \sqrt{\frac{1}{k^2} + \frac{(k^{\prime})^2}{\tau^2 k^4}} = \sqrt{R^2 + (R^{\prime})^2 T^2} = \text{const},
    \]
    and hence $ \alpha $ lies on a sphere centered about $ \beta (0) $.
\end{solution}

% 1.5.14
\begin{exercise}[\textbf{*}]
    Let $\alpha:(a,b)\to\mathbb{R}^2$ be a regular parametrized plane curve. Assume that there exists $t_0$, $a<t_0<b$, such that the distance $\lvert \alpha(t)\rvert$ from the origin to the trace of $\alpha$ will be a maximum at $t_0$. Prove that the curvature $k$ of $\alpha$ at $t_0$ satisfies
    \begin{equation*}
        \lvert k(t_0)\rvert \geq \frac{1}{\lvert \alpha(t_0)\rvert}.
    \end{equation*}
\end{exercise}

\begin{solution}
    Notice that $ f(t) = \vert \alpha (t) \vert $ is nonnegative, so $ f^{2}(t) = \alpha (t) \cdot \alpha (t) $ also attains a maximum at $ t_0 $. Then
    \begin{equation*}
        \frac{d}{dt} f^{2} (t) \Big|_{t=t_0} = 2 \alpha(t_0) \cdot \alpha^{\prime} (t_0) = 0,
    \end{equation*}
    differentiating again gives
    \begin{equation*}
        \frac{d^2}{dt^2} f^{2} (t) \Big|_{t=t_0} =  \alpha^{\prime} (t_0) \cdot \alpha^{\prime} (t_0) + \alpha(t_0) \cdot \alpha^{\prime\prime} (t_0) \leq 0,
    \end{equation*}
    since $ f(t) $ attains a maximum at $ t_0 $. We also have $ \alpha^{\prime} (t_0) \cdot \alpha^{\prime} (t_0) = 1 $ since it is a parametrization by arclength, and $ \alpha^{\prime\prime} (t_0) = k(t_0) n(t_0) $. Then let $ \theta $ be the angle between $ \alpha(t_0) $ and $ \alpha^{\prime\prime} $, we have
    \begin{equation*}
        k(t_0) n(t_0) \alpha(t_0) = \vert k(t_0) \vert \vert n(t_0) \vert \vert \alpha(t_0) \vert \cos \theta \leq -1.
    \end{equation*}
    Notice that $ \vert n(t_0) \vert = 1 $ and $ \cos \theta < 0 $, we have
    \begin{equation*}
        k(t_0) \geq \frac{1}{\vert \alpha(t_0) \cos \theta \vert} \geq \frac{1}{\vert \alpha (t_0) \vert }.
    \end{equation*} 
\end{solution}

% 1.5.15
\begin{exercise}[\textbf{*}]
    Show that the knowledge of the vector function \( b = b(s) \) (binormal vector) of a curve \(\alpha\), with nonzero torsion everywhere, determines the curvature \(k(s)\) and the absolute value of the torsion \(\tau(s)\) of \(\alpha\).
\end{exercise}

\begin{solution}
    By the Frenet equations, we have $ b^{\prime} = \tau n $, so for an arc length parametrized curve, $ \vert b^{\prime} \vert = \vert \tau \vert $. Next, differentiate to get 
    \[
        b^{\prime\prime} = \tau^{\prime} n + \tau n^{\prime} = \tau^{\prime} n - \tau k t - \tau^2 b \;\Longrightarrow\; \tau b^{\prime\prime} = \tau \tau^{\prime} n - \tau^2 k t - \tau^3 b .
    \]
    From $ b^{\prime} = \tau n $, we have $ b^{\prime} \tau^{\prime} = \tau \tau^{\prime} n $, so 
    \[
        \tau b^{\prime\prime} = b^{\prime} \tau^{\prime} - \tau^2 k t - \tau^3 b \;\Longrightarrow\; t = \frac{b^{\prime} \tau^{\prime} - \tau^3 b - \tau b^{\prime\prime}}{\tau^2 k}. 
    \]
    Take the norm on both sides yields 
    \[
        k = \frac{\vert \tau^3 b - \tau^{\prime} b^{\prime} + \tau b^{\prime\prime} \vert}{\tau^2} = \frac{\left\vert \vert b^{\prime} \vert^4 - (b^{\prime} \cdot b^{\prime\prime}) b^{\prime} + \vert b^{\prime} \vert^2 b^{\prime\prime} \right\vert}{\vert b^{\prime} \vert^3},
    \]
    where we assumed $ \tau = \vert b^{\prime} \vert $ without loss of generality as the formula is invariant under $ \tau \to - \tau $, and hence $ \tau^{\prime} = (b^{\prime} \cdot b^{\prime\prime}) / \vert b^{\prime} \vert $. Therefore, 
    \begin{equation}
        \vert \tau \vert = \vert b^{\prime} \vert, \quad k = \frac{\left\vert \vert b^{\prime} \vert^4 - (b^{\prime} \cdot b^{\prime\prime}) b^{\prime} + \vert b^{\prime} \vert^2 b^{\prime\prime} \right\vert}{\vert b^{\prime} \vert^3}. 
    \end{equation}
\end{solution}

% 1.5.16
\begin{exercise}[\textbf{*}]
    Show that the knowledge of the vector function \( n = n(s) \) (normal vector) of a curve \(\alpha\), with nonzero torsion everywhere, determines the curvature \(k(s)\) and the torsion \(\tau(s)\) of \(\alpha\).
\end{exercise}

\begin{solution}
    The normal $ n $ is determined by $ \alpha^{\prime\prime} = k n $, and $ n^{\prime} = -k t - \tau b $ by the second Frenet equation. Following the hint, we shall show that
    \begin{equation}
        \label{eq:curvature_torsion_ n}
        \frac{(n \wedge n^{\prime}) \cdot n^{\prime\prime}}{\vert n^{\prime} \vert^2} = \dfrac{\dfrac{\mathrm{d}}{\mathrm{d}s} \left(\dfrac{k}{\tau}\right)}{\left(\dfrac{k}{\tau}\right)^2 + 1}.
    \end{equation}
    Let $ t = \alpha^{\prime} $, $ b = t \wedge n = \alpha^{\prime} \wedge n $ in the Frenet equation, then 
    \[
        n^{\prime} = -k \alpha^{\prime} - \tau b \;\Longrightarrow\; \vert n^{\prime} \vert^2 = k^2 + \tau^2, 
    \]
    \[
        n \wedge n^{\prime} = n \wedge (-k \alpha^{\prime} - \tau b) = - \tau t + k b,
    \] 
    since $ n \wedge b = \alpha^{\prime} = t $. Next, differentiate $ n^{\prime} $ to get
    \[
        n^{\prime\prime} = - \left[k^{\prime} t + (k^2 + \tau ^2)n + \tau ^{\prime} b\right]
    \]
    and
    \[
        (n\wedge n^{\prime}) \cdot n^{\prime\prime} = (- \tau t + k b) \cdot \left[- k^{\prime} t - (k^2 + \tau^2) n - \tau^{\prime} b \right] = \tau k^{\prime} - k \tau^{\prime} = \tau^2 \left(\frac{k}{\tau}\right)^{\prime}. 
    \]
    Therefore, we have
    \[
        \frac{(n \wedge n^{\prime}) \cdot n^{\prime\prime}}{\vert n^{\prime} \vert^2} = \dfrac{\dfrac{\mathrm{d}}{\mathrm{d}s} \left(\dfrac{k}{\tau}\right)}{\left(\dfrac{k}{\tau}\right)^2 + 1} \equiv a(s) \;\Longrightarrow\; \tan^{-1} \left(\frac{k}{\tau}\right) = \int \mathrm{d}s\, a(s). 
    \]
    Hence, we have, up to a constant $ C $ that can only be determined by initial conditions,
    \[
        \frac{k}{\tau} = \tan \left[ \int \mathrm{d}s\, \frac{(n (s) \wedge n^{\prime} (s)) \cdot n^{\prime\prime} (s)}{\vert n^{\prime} (s) \vert^2} + C \right], \quad \tau^2 + k^2 = \vert n^{\prime} (s) \vert^2.
    \]
    \begin{remark}
        The problem is ill-posed. Consider the counterexample: let
        \[
            \beta (t) = \left(a \cos s, a \sin s, bs \right), \quad s \in \mathbb{R}
        \]
        with $ a^2 + b^2 = 1 $, $ a, b > 0 $ be a helix. For all values of $ a, b $, we have 
        \[
            \beta^{\prime\prime} (s) = -a (\cos s, \sin s, 0) \;\Longrightarrow\; n(s) = - (\cos s, \sin s, 0),
        \]
        and in general we have $ k = a $ and $ \tau = -b $ through direct calculation. Taking $ (a,b) = (1 / \sqrt{2}, 1 / \sqrt{2}) $ and $ (a,b) = (1/2, \sqrt{3} / 2 ) $ gives two different curves with the same normal vector function $ n(s) $, non-vanishing torsion, and different curvature and torsion.
    \end{remark}
\end{solution}

% 1.5.17
\begin{exercise}
     In general, a curve $\alpha$ is called a \emph{helix} if the tangent lines of $\alpha$ make a constant angle with a fixed direction. Assume that $\tau(s) \neq 0$, $s \in I$, and prove that:
    \begin{enumerate}
        \item[*\textbf{a.}] $\alpha$ is a helix if and only if $\frac{k}{\tau} = \text{const}$.
        \item[*\textbf{b.}] $\alpha$ is a helix if and only if the lines containing $n(s)$ and passing through $\alpha(s)$ are parallel to a fixed plane.
        \item[*\textbf{c.}] $\alpha$ is a helix if and only if the lines containing $b(s)$ and passing through $\alpha(s)$ make a constant angle with a fixed direction.
        \item[\textbf{d.}] The curve
        \begin{equation}
            \alpha(s) = \left( \frac{a}{c} \int \sin \theta(s) \, ds, \frac{a}{c} \int \cos \theta(s) \, ds \right)
        \end{equation}
        where $c^2 = a^2 + b^2$, is a helix, and that $\frac{k}{\tau} = \frac{a}{b}$.
    \end{enumerate}
\end{exercise}

\begin{solution}
    ~
    \begin{enumerate}[(a)]
        \item Suppose there exists a vector $ v \in \mathbb{R}^3 $ such that $ v \cdot t(s) = C $ for some constant $ C $. Then 
        \begin{equation*}
            \frac{\mathrm{d}t}{\mathrm{d}s} \cdot v = k(s) n(s) \cdot v = 0,
        \end{equation*}
        so $ n(s) \cdot v = 0 $. Differentiating again gives 
        \begin{equation*}
            \frac{\mathrm{d}n}{\mathrm{d}s} \cdot v = -k(s) t(s) \cdot v + \tau (s) b(s) \cdot v = -k(s) C + \tau (s) b(s) \cdot v = 0.
        \end{equation*}
        Since $ \tau (s) \neq 0 $, we have  
        \[ 
            C k(s) / \tau (s) = \left( b(s) \cdot v \right) = \left(t(s) \wedge n(s)\right) \cdot v = \left(v \wedge t(s)\right) \cdot n(s) .
        \]
        Since $ t(s), v \perp n(s) $, the triple product is equal to $ \vert n(s) \vert \vert t(s) \vert \vert v \vert \sin (C) = \vert v \vert \sin C$. Therefore, $ k(s) / \tau (s) $ is a constant. Conversely, if $ k(s) / \tau (s) = C^{\prime} $ for some constant $ C^{\prime} $, then we can take $ v = t(s) + C^{\prime} b(s) $, which is a constant vector since
        \begin{equation*}
            \frac{\mathrm{d}v}{\mathrm{d}s} = k(s) n(s) + C^{\prime} \left( -\tau (s) n(s) \right) = 0.
        \end{equation*} 
        Then
        \begin{equation*}
            \frac{\mathrm{d}t}{\mathrm{d}s}\cdot v = 0.
        \end{equation*} 
        \item Suppose $ \alpha (s) $ is a helix, then there exists a vector $ v \in \mathbb{R}^3 $ such that $ v \cdot t(s) = C $ for some constant $ C $. Let $ L $ be a line containing $ n(s) $ and passing through $ \alpha(s) $. Then $ n(s) \cdot v = 0 $ by result in part (a), so $ L \perp v $, hence parallel to the plane with normal vector $ v $. Conversely, for any point $ s\in I $, suppose the line $ L $ containing $ n(s) $ and passing $ \alpha (s) $ is parallel to the plane $ P $ with normal vector $ v \in \mathbb{R}^{3} $. Then $ n(s) \cdot v = 0 $, and 
        \begin{equation*}
            \frac{\mathrm{d}T}{\mathrm{d}s} \cdot v = k(s) n(s) \cdot v = 0.
        \end{equation*}
        Hence $ \mathrm{d}T / \mathrm{d}s = \mathrm{d} (T\cdot v) / \mathrm{d}s = 0 $, and $ T(s) \cdot v = C^{\prime} $ for some constant $ C^{\prime} $, and $ \alpha (s) $ is a helix.
        \item By definition of helix, there exists a vector $ v \in \mathbb{R}^3 $ such that $ v \cdot t(s) = C $ for some constant $ C $. By (b), all the lines containing $ n(s) $ and passing through $ \alpha (s) $ are parallel to the plane with some fixed normal vector $ u \in \mathbb{R}^{3} $, so $ n(s) \cdot u = 0 $. Consider $ b \cdot (u \wedge v) = (t(s) \wedge n(s)) \cdot (u \wedge v) = (t(s) \cdot u)(n(s) \cdot v) - (t(s) \cdot v)(n(s) \cdot u) = 0 $, since $ n(s) \cdot v = 0 $ from (a). Conversely, suppose there exists a vector $ v \in \mathbb{R}^3 $ such that $ b(s) \cdot v = C $ for some constant $ C $. Then $ \left(t(s) \wedge n(s)\right) \cdot v = C $,
        \begin{equation*}
            \frac{\mathrm{d}b}{\mathrm{d}s} \cdot v = -\tau (s) n(s) \cdot v = 0,
        \end{equation*} 
        and by $ \tau (s) \neq 0 $ we have $ n(s) \cdot v = 0 $. Finally, 
        \begin{equation*}
            \frac{\mathrm{d}}{\mathrm{d}s}\left(t(s) \cdot v\right) = k(s) n(s) \cdot v = 0,
        \end{equation*}
        therefore, $ \alpha (s) $ is a helix. 
        \item With $ s $ suppressed in the expressions, derivatives of $ \alpha $ are 
        \begin{equation*}
            \begin{split}
                \alpha^{\prime} &= \left( \frac{a}{c} \sin \theta(s), \frac{a}{c} \cos \theta(s), \frac{b}{c} \right),\\
                \alpha^{\prime\prime} &= \left( \frac{a}{c} \theta^{\prime}(s) \cos \theta(s), -\frac{a}{c} \theta^{\prime}(s) \sin \theta(s), 0 \right),\\
                \alpha^{\prime\prime\prime} &= \left( \frac{a}{c} \left( \theta^{\prime\prime}(s) \cos \theta(s) - (\theta^{\prime}(s))^2 \sin \theta(s) \right), -\frac{a}{c} \left( \theta^{\prime\prime}(s) \sin \theta(s) + (\theta^{\prime}(s))^2 \cos \theta(s) \right), 0 \right).
            \end{split}
        \end{equation*}
        The curvature is $ k(s) = \vert \alpha^{\prime} (s) \vert = \frac{a}{c} \theta^{\prime} $. The torsion is given by the formula 
        \begin{equation*}
            \tau (s) = - \frac{(\alpha^{\prime}(s) \wedge \alpha^{\prime\prime}(s)) \cdot \alpha^{\prime\prime\prime}(s)}{k(s)^2}
        \end{equation*}  
        by [Do Carmo] Exercise 1.5.2. Direct calculation gives 
        \begin{equation*}
            (\alpha^{\prime} \wedge \alpha^{\prime\prime}) \cdot \alpha^{\prime\prime\prime} = \left( \frac{ab}{c^2} \theta^{\prime}(s) \sin \theta(s), -\frac{ab}{c^2} \theta^{\prime}(s) \cos \theta(s), -\frac{a^2}{c^2} (\theta^{\prime}(s))^2 \right) = \frac{a^{2} b}{c^{3}}(\theta^{\prime})^{3} ,
        \end{equation*}
        so 
        \begin{equation*}
            \tau (s) = \frac{b}{c} \theta^{\prime}(s) = \frac{b}{a} k(s).
        \end{equation*}
    \end{enumerate}
\end{solution}

\newpage

\subsection{Chapter 1.6}
% 1.6.1
\begin{exercise}[\textbf{*}]
    Let $\alpha: I \to \mathbb{R}^3$ be a curve parametrized by arc length with curvature $k(s) \neq 0$, $s \in I$. Let $P$ be a plane satisfying both of the following conditions:

    \begin{enumerate}
        \item $P$ contains the tangent line at $s$.
        \item Given any neighborhood $J \subset I$ of $s$, there exist points of $\alpha(J)$ in both sides of $P$.
    \end{enumerate}

    Prove that $P$ is the osculating plane of $\alpha$ at $s$.
\end{exercise}

\begin{solution}
    Let $ n $ be the normal vector of plane $ P $, then condition 1 implies that $ n_P \perp t(s) $, as $ t(s) \in P $. To show the desired result, we will show that $ n(s) \perp n_P $. Consider $ f(s) = t(s) \cdot n_P = 0 $, differentiating both sides gives $ f^{\prime}(s) = t(s) \cdot n_P^{\prime} = k(s) n(s) \cdot n_P = 0 $, so $ n(s) \perp n_P $. Thus, the binormal vector $ b(s) \parallel n_P$. Furthermore, by condition 2 we can take some interval $ J = \left(s-\frac{1}{m}, s+\frac{1}{m}\right) \subseteq I $, then there exists $ s_1^{(m)} \in \left(s-\frac{1}{m}, s\right) $ and $ s_2^{(m)} \in \left(s, s+\frac{1}{m}\right) $ such that $ \alpha(s_1^{(m)}) $ and $ \alpha(s_2^{(m)}) $ are in different sides of plane $ P $. This holds for all $ m \in \mathbb{N} $, so as $ m \to \infty $, $ p \equiv \alpha (s) = \lim_{m \to \infty} \alpha (s_1^{(m)}) $ lies on the left side of $ P $, and $ p \equiv \alpha (s) = \lim_{m \to \infty} \alpha (s_2^{(m)}) $ lies on the right side of $ P $, hence $ p = \alpha (s) \in P $. Since $ P $ contains $ \alpha (s) $ and has $ b(s) $ as a normal vector, $ P $ is the osculating plane of $ \alpha $ at $ s $. 
\end{solution}

% 1.6.2
\begin{exercise}
    Let $\alpha: I \to \mathbb{R}^3$ be a curve parametrized by arc length, with curvature $k(s) \neq 0$, $s \in I$. Show that

    \begin{enumerate}
        \item[*\textbf{a.}] The osculating plane at $s$ is the limit position of the plane passing through $\alpha(s)$, $\alpha(s + h_1)$, $\alpha(s + h_2)$ when $h_1, h_2 \to 0$.
        \item[\textbf{b.}] The limit position of the circle passing through $\alpha(s)$, $\alpha(s + h_1)$, $\alpha(s + h_2)$ when $h_1, h_2 \to 0$ is a circle in the osculating plane at $s$, the center of which is on the line that contains $n(s)$ and the radius of which is the radius of curvature $1/k(s)$; this circle is called the \emph{osculating circle} at $s$.
    \end{enumerate}
\end{exercise}

\begin{solution}
    ~
    \begin{enumerate}[(a)]
        \item Since the plane, which we will call $ P $, by construction passes through $ \alpha (s) $, we are left to show that the normal vector $ n_P $ of $ P $ converges to $ b(s) $ in the limit $ h_1, h_2 \to 0 $. We have 
        \begin{equation*}
            \begin{split}
                n_P &= \frac{\left(\alpha (s + h_1) - \alpha (s) \right) \wedge \left(\alpha (s + h_2) - \alpha (s)\right)}{\left\vert \left(\alpha (s + h_1) - \alpha (s) \right) \wedge \left(\alpha (s + h_2) - \alpha (s)\right) \right\vert } \\
                &= \frac{\left( h_1 \alpha^{\prime} (s) + O(h_1^2) \right) \wedge \left( h_2 \alpha^{\prime} (s) + O(h_2^2) \right)}{\left\vert \left( h_1 \alpha^{\prime} (s) + O(h_1^2) \right) \wedge \left( h_2 \alpha^{\prime} (s) + O(h_2^2) \right) \right\vert } \\
                &= \left(\frac{ \alpha^{\prime} (s) \wedge \alpha^{\prime\prime} (s)}{\left\vert \alpha^{\prime} (s) \wedge \alpha^{\prime\prime} (s) \right\vert } + O(h_1) + O(h_2) \right),
            \end{split}
        \end{equation*}
        hence 
        \begin{equation*}
            \lim_{h_1, h_2 \to 0} n_P = \frac{ \alpha^{\prime} (s) \wedge \alpha^{\prime\prime} (s)}{\left\vert \alpha^{\prime} (s) \wedge \alpha^{\prime\prime} (s) \right\vert } .
        \end{equation*}
        Then the binormal vector is parallel to $ N_P $ since  
        \begin{equation*}
            b(s) = t(s) \wedge n(s) = \alpha^{\prime} (s) \wedge \alpha^{\prime\prime} (s) / \vert \alpha^{\prime\prime} (s) \vert \parallel n_P.
        \end{equation*}
        \item Without loss of generality, shift the origin to $ s $ so that $ \alpha (s), \alpha (s+h_1), \alpha (s+h_2) $ become $ \alpha (0), \alpha (h_1), \alpha (h_2) $, respectively. Let $ (x_0,y_0,z_0) $ be the center of the circle passing through $ \alpha (0) $, $ \alpha (h_1) $, and $ \alpha (h_2) $, then the equation of the circle can be written as $ F(s) = (x(s) - x_0)^{2} + (y(s)-y_0)^{2} + (z(s)-z_0)^{2} - r^{2} $. Calculate the derivatives to be 
        \begin{equation*}
            F^{\prime} (s) = 2(x(s)-x_0)x^{\prime}(s) + 2(y(s)-y_0)y^{\prime}(s) + 2(z(s)-z_0)z^{\prime}(s)
        \end{equation*} 
        and 
        \begin{equation*}
            \begin{split}
                F^{\prime\prime} (s) &= 2(x^{\prime}(s))^{2} + 2(y^{\prime}(s))^{2} + 2(z^{\prime}(s))^{2} \\
                &+ 2(x(s)-x_0)x^{\prime\prime}(s) + 2(y(s)-y_0)y^{\prime\prime}(s) + 2(z(s)-z_0)z^{\prime\prime}(s).
            \end{split}
        \end{equation*}
        Taking the limit as $ s \to 0 $ gives $ F^{\prime} (0) = -2 x_0 $ and $ F^{\prime\prime} (0) = 2-2k(0) y_0 $. Since the plane passes through $ \alpha (0), \alpha (h_1), \alpha (h_2) $, we have $ F(0) = F(h_1) = F(h_2) = 0 $. By the Mean Value Theorem, there exists some $ s_1 \in (0, h_1) $ such that $ F^{\prime} (s_1) = 0 $. As $ h_1 \to 0 $, we have $ s_1 \to 0 $, by continuity of $ F $ we have $ F^{\prime} (s_1) \to 0 $ as $ s_1 \to 0 $ as $ h_1, h_2 \to 0 $. Similarly, suppose $ h_1 < h_2 $, there exists some $ s_2 \in (h_1, h_2) $ such that $ F^{\prime} (s_2) = 0 $. By the Mean Value Theorem, there exists some $ s_3 \in (s_1, s_2) $ such that $ F^{\prime\prime} (s_3) = 0 $. As $ h_1, h_2 \to 0 $, we have $ s_1, s_2 \to 0 $, so by continuity of $ F^{\prime\prime} $, $ F^{\prime\prime} (s_3) \to 0 $ as $ s_3 \to 0 $. Therefore,
        \begin{equation*}
            \lim_{h_1, h_2 \to 0} F^{\prime} (s_1) = F^{\prime} (0) = -2 x_0 = 0 \implies x_0 = 0,  
        \end{equation*}
        and
        \begin{equation*}
            \lim_{h_1, h_2 \to 0} F^{\prime\prime} (s_2) = F^{\prime\prime} (0) = 2-2k(0)y_0 = 0 \implies y_0 = \frac{1}{k(0)}.
        \end{equation*}
        By $ (a) $ we know the circle lies on the osculating plane at $ \alpha (0) $ as $ h_1, h_2 \to 0 $, so $ c \to 0 $. Hence the center of the circle converges to $ (0, 1/k(0), 0) $, which lies on the line containing $ n(0) $, and the radius converges to $ 1/k(0) $.
    \end{enumerate}
\end{solution}

\end{CJK}
\end{document}