\documentclass[a4paper]{article}
%% Formatting %%
\usepackage[margin=3cm]{geometry}
\usepackage{type1cm, titlesec, fancyhdr, titling}
\usepackage{multicol}
\usepackage[dvipsnames]{xcolor}
\usepackage{ulem}
\usepackage{parskip}
\setlength{\parindent}{2em}
\setlength{\headheight}{15pt}
\setlength{\droptitle}{-1.5cm}
\parindent=24pt
%% Math and Symbols %%
\usepackage{amsmath,amsthm,amssymb, mathtools}
\usepackage{yhmath, faktor, dsfont}
\usepackage{academicons, wasysym, marvosym}
\usepackage[scr]{rsfso} 
\usepackage{latexsym, amsmath, amscd, amsmath, amsthm}
\usepackage{amssymb,amsmath,amsthm,graphicx,dsfont}
\usepackage{hyperref}

%% Enhancement %%
\usepackage{graphicx, tabularx}
\usepackage[shortlabels,inline]{enumitem}
%% TikZ %%
\usepackage{tikz-cd}
\usepackage[breakable]{tcolorbox}
\usetikzlibrary{decorations.pathmorphing}
\usetikzlibrary{calc, arrows,matrix}

%% Other packages %%
\usepackage{amsopn}

%% Traditional Chinese %%
\usepackage{CJKutf8}

%% Math environments %%
\newtheoremstyle{mystyle}
  {6pt}{15pt}% 上下間距
  {}%          內文字體
  {}%              縮排
  {\bf}%       標頭字體
  {.}%       標頭後標點
  {1em}% 內文與標頭距離
  {}% Theorem head spec (can be left empty, meaning 'normal')
\theoremstyle{mystyle}	
\newtheorem{theorem}{Theorem}
\newtheorem{definition}{Definition}
\newtheorem{example}[theorem]{Example}
\newtheorem{exercise}{Exercise}
\newtheorem{solution}{Solution}
\newtheorem{corollary}[theorem]{Corollary}
\newtheorem{property}[theorem]{Property}
\newtheorem{proposition}[theorem]{Proposition}
\newtheorem{lemma}{Lemma}
\newtheorem{problem}[theorem]{Problem}
\newtheorem{answer}{Answer}[section]
\newtheorem{fact}[theorem]{fact}
\newtheorem*{claim}{Claim}
\newtheorem*{observation}{Observation}

\newenvironment{exerciseManual}[1]{%
  \renewcommand{\theexercise}{#1}%
  \begin{exercise}%
  \addtocounter{exercise}{-1}%
}{%
  \end{exercise}%
}

\newenvironment{solutionManual}[1]{%
  \renewcommand{\thesolution}{#1}%
  \begin{solution}%
  \addtocounter{solution}{-1}%
}{%
  \end{solution}%
}

\theoremstyle{remark}
\newtheorem*{remark}{Remark}

\newcommand{\bvec}[1]{\mathbf{#1}} % vector

\begin{document}
\begin{CJK}{UTF8}{bkai}

\title{%
  \textbf{2025 Fall Introduction to Geometry} \\
  \vspace{0.5cm}
  \Large Solutions to Exercises in Do Carmo \\
}
\author{黃紹凱 B12202004}
\date{\today}

\maketitle

\section{Chapter 3.2}

\begin{definition}[second fundamental form]
    The quadratic form $ \operatorname{II}_p $, defined in $ T_p(S) $ by $ \operatorname{II}_p(v) = - \langle \mathrm{d}N_p(v), v \rangle $ is called the second fundamental form of $S$ at $p$.
\end{definition}

\begin{definition}[normal curvature]
    Let C be a regular curve in S passing through $ p \in S $, $ k $ the curvature of $ C $ at $ p $, and $ \cos \theta = \langle n, N \rangle $, where $ n $ is the normal vector to $ C $ and $ N $ is the normal vector to $ S $ at $ p $. The number $ k_n = k \cos \theta $ is then called the normal curvature of $ C \subseteq S $ at $ p $.
\end{definition}

\begin{definition}[Do Carmo 3.2.5, line of curvature]
    If a regular connected curve $ C \subseteq S $ is such that for all $ p \in S $ the tangent line of $ C $ is a principal direction at $ p $, then $ C $ is said to be a \emph{line of curvature} of $ S $.
\end{definition}

\begin{definition}[Do Carmo 3.2.9, asymptotic curve]
    Let $ p \in S $. An \emph{asymptotic direction} of $ S $ at $ p $ is a direction in $ T_p(S) $ for which the normal curvature is zero. An \emph{asymptotic curve} of $ S $ is a regular connected curve $ C \subseteq S $ such that for each $ p \in S $ the tangent line of $ C $ at $ p $ is an asymptotic direction.
\end{definition}

\begin{proposition}[Meusnier]
    All curves lying on a surface $ S $ and having at a given point $ p \in S $ the same tangent line have at this point the same normal curvatures.
\end{proposition}

\begin{proposition}[Oline Rodrigues]
    A necessary and sufficient condition for a connected regular curve $ C $ on $ S $ to be a line of curvature of $ S $ is $ N^{\prime} (t) = \lambda (t) \alpha^{\prime} (t) $, for any parametrization $ \alpha (t) $ of $ C $, where $ N (t) = (N \circ \alpha) (t) $ and $ \lambda (t) $ is a differentiable function of $ t $. In this case, $ - \lambda (t) $ is the principal curvature along $ \alpha^{\prime} (t) $. 
\end{proposition}

\begin{definition}[shape operator]
    The linear map $ \mathcal{S} : T_p(S) \to T_p(S) $ defined by $ \mathcal{S} (v) = - \mathrm{d}N_p (v) $ is called the shape operator of $ S $ at $ p $.
\end{definition}

% 3.2.2
\begin{exerciseManual}{3.2.2}
    Show that if a surface is tangent to a plane along a curve, then the points of this curve are either parabolic or planar.
\end{exerciseManual}

\begin{solutionManual}{3.2.2}
    Suppose a surface $S$ is tangent to a plane $\Pi$ along a curve $C$. Let $p \in C$ be an arbitrary point on the curve. Parametrize the curve $C$ by $\alpha: I \to S \cap \Pi$, where $I$ is an open interval containing $0$ and $\alpha(0) = p$. Let $ N: S \to S^2 $ be the Gauss map of $ S $. Since the tangent plane of $ S $ is $ \Pi $ for all $ p \in S $, the unit normal $ N(\alpha (s)) $ is equal to the constant normal $ n $ of $ \Pi $. Thus, 
    \[
        0 = \frac{\mathrm{d}}{\mathrm{d}s} N(\alpha (s)) = \mathrm{d} N_{\alpha (s)}(\alpha'(s)).
    \]
    Therefore, the differential of the Gauss map $ \bvec{d}N_p $ has a nontrivial kernel containing $ \alpha'(0) \neq 0 $ for all $ \alpha(s) \in S $. But $ \bvec{d}N_p: T_p(S) \to T_{N(p)}(S^2) $ is a linear map between finite-dimensional vector spaces, $ \mathrm{d}N_p $ is not invertible, and hence $ \det \left(\mathrm{d}N_p\right) \neq 0 $ for all $ p \in C $. Thus, all points on $ C $ are either parabolic or planar.
\end{solutionManual}

% 3.2.8
\begin{exerciseManual}{3.2.8}
    Describe the region of the unit sphere covered by the image of the Gauss map of the following surfaces:
    \begin{enumerate}[label=\textbf{\alph*.}]
        \item Paraboloid of revolution $z = x^2 + y^2$.
        \item Hyperboloid of revolution $x^2 + y^2 - z^2 = 1$.
        \item Catenoid $x^2 + y^2 = \cosh^2 z$.
    \end{enumerate}
\end{exerciseManual}

\begin{solutionManual}{3.2.8}
    Let's take the natural orientation: upward normal for graphs and outward normal for surfaces of revolution. 

    \begin{enumerate}[label=\textbf{\alph*.}]
        \item Let the graph be $ z = f(x,y) = x^2 + y^2 $, then the normal to the surface is 
        \[
            N = \frac{(-f_x, -f_y, 1)}{\sqrt{f_x^2 + f_y^2 + 1}},
        \]
        where $ f_x = 2x $, $ f_y = 2y $. Since $ (x,y) \in \mathbb{R}^2 $ and the z component $ N^z = 1 / \sqrt{1 + 4(x^2 + y^2)} \in (0,1] $, the Gauss map is the open upper hemisphere of the unit sphere.
        \item As a level set $ F(x,y,z) = x^2 + y^2 - z^2 - 1 $, the (outward) normal vector is
        \[
            N = \frac{\nabla F}{|\nabla F|} = \frac{(2x, 2y, -2z)}{\sqrt{4x^2 + 4y^2 + 4z^2}} = \frac{(x,y,-z)}{\sqrt{x^2 + y^2 + z^2}}.
        \]
        Since $ x^2 + y^2 = z^2 + 1 \geq 1 $, the z component
        \[
            N^z = - \frac{z}{\sqrt{x^2 + y^2 + z^2}} = - \frac{z}{\sqrt{2z^2 + 1}} \in \left(-\frac{1}{\sqrt{2}}, \frac{1}{\sqrt{2}}\right).
        \]
        Thus, the Gauss map covers the open band $ \{p \in S^2 \mid \vert N^z \vert < \frac{1}{\sqrt{2}} \} $. 
        \item Let's write this in the following parametrization: 
        \[
            \bvec{x}(z, \theta) = (\cosh z \cos \theta, \cosh z \sin \theta, z), \quad z \in \mathbb{R}, \theta \in [0, 2\pi).
        \]
        Then, 
        \[
            \bvec{x}_z = (\sinh z \cos \theta, \sinh z \sin \theta, 1), \quad \bvec{x}_\theta = (-\cosh z \sin \theta, \cosh z \cos \theta, 0).
        \]
        The normal vector is given by
        \[
            N = \frac{\bvec{x}_z \times \bvec{x}_\theta}{|\bvec{x}_z \times \bvec{x}_\theta|} = \frac{(-\cosh z \cos \theta, -\cosh z \sin \theta, \sinh z \cosh z)}{\sqrt{\cosh^2 z + \sinh^2 z \cosh^2 z}} = \frac{(-\cos \theta, -\sin \theta, \sinh z)}{\sqrt{1 + \sinh^2 z}}.
        \]
        \[
            \Longrightarrow\; N = (-\operatorname{sech} z \cos \theta, -\operatorname{sech} z \sin \theta, \tanh z).
        \]
        Since $ \theta \in [0, 2\pi) $ and $ N^z = - \tanh z \in (-1,1) $, the spherical image $N(C) = S^2 \setminus \{(0,0,\pm 1)\} $.
    \end{enumerate}
\end{solutionManual}

% 3.2.9
\begin{exerciseManual}{3.2.9}
    ~

    \begin{enumerate}[label=\textbf{\alph*.}]
        \item Prove that the image $N \circ \alpha$ by the Gauss map $N: S \to S^2$ of a parametrized regular curve $\alpha: I \to S$ which contains no planar or parabolic points is a parametrized regular curve on the sphere $S^2$ (called the \emph{spherical image} of $\alpha$).
        \item If $C = \alpha(I)$ is a line of curvature, and $k$ is its curvature at $p$, then
        \[
            k = |k_n k_N|,
        \]
        where $k_n$ is the normal curvature at $p$ along the tangent line of $C$ and $k_N$ is the curvature of the spherical image $N(C) \subset S^2$ at $N(p)$.
    \end{enumerate}
\end{exerciseManual}

\begin{solutionManual}{3.2.9}
    ~
    
    \begin{enumerate}[label=\textbf{\alph*.}]
        \item Suppose $ \alpha: I \to S $ is a parametrized regular curve with no planar or parabolic points. Then, the Gauss map $ N: S \to S^2 $ satisfies $ \det \left(\mathrm{d}N_p\right) \neq 0 $, and $ \mathrm{d}N_p $ is invertible, and hence injective for all $ p \in C $. Since $ \alpha $ is a regular curve, $ \alpha^{\prime} (t) \neq 0 $ for all $ t \in I $, and hence
        \[
            (N \circ \alpha)'(t) = \mathrm{d}N_{\alpha(t)}(\alpha'(t)) \neq 0,
        \]
        which shows that the spherical image $N(C)$ is a regular curve on $S^2$.
        \item Since $ C $ is a line of curvature, the tangent vector $ t = \alpha' (s) $ at $ p = \alpha(s) $ is a principal direction. Hence, $ \mathcal{S} (t) = k_n t $ where $ k_n $ is the normal curvature along $ t $ at $ p $. Let $ N : S \to S^2 $ be the Gauss map of $ S $. Using $ \mathrm{d}N = - \mathcal{S} (t) $, we have
        \[
            \frac{\mathrm{d}}{\mathrm{d}s} N (\alpha (s)) = \mathrm{d}N_{\alpha(s)}(\alpha'(s)) = - \mathcal{S} (t) = - k_n t.
        \]
        Thus, $ \vert N^{\prime} \vert = \vert k_n \vert $, and the tangent vector of the spherical image $ N(C) $ at $ N(p) $ is
        \[
            t_N = \frac{N^{\prime}}{|N^{\prime}|} = \frac{- k_n t}{|k_n|} = - \operatorname{sgn} (k_n) t.
        \]
        Let $ s_N $ be the arc length parameter of the spherical image $ N(C) $. Then,
        \[
            \vert k_N \vert = \left\vert \frac{\mathrm{d} t_N}{\mathrm{d} s_N} \right\vert = \frac{\vert \mathrm{d} t_N / \mathrm{d} s \vert}{\vert \mathrm{d} s_N / \mathrm{d} s \vert} = \frac{\mathrm{d} t_N / \mathrm{d} s}{|N^{\prime}|} = \frac{k}{|k_n|},
        \]
        where we used $ t^{\prime} = kn $ in the last equality. Therefore, $ k = |k_n k_N| $.
    \end{enumerate}
\end{solutionManual}

% 3.2.10
\begin{exerciseManual}{3.2.10}
    Assume that the osculating plane of a line of curvature $C \subset S$, which is nowhere tangent to an asymptotic direction, makes a constant angle with the tangent plane of $S$ along $C$. Prove that $C$ is a plane curve.
\end{exerciseManual}

\begin{solutionManual}{3.2.10}
    Let $ t $, $ n $, $ b $ be the Frenet frame of the curve $ C $. Since the osculating plane makes a constant angle with the tangent plane of $ S $, the unit normal $ N $ of $ S $ along $ C $ satisfies 
    \[
        b \cdot N = \text{const.}
    \]  
    Differentiate both sides with respect to the arc length parameter $ s $ of $ C $ and use Frenet's formula:
    \[
        b^{\prime} \cdot N + b \cdot N^{\prime} = 0 \;\Longrightarrow\; -\tau n \cdot N + b \cdot N^{\prime} = 0.
    \] 
    Next, $ N^{\prime} = - \mathcal{S} (t) $ by the Weingarten formula, where $ \mathcal{S} $ is the shape operator of $ S $. Since $ C $ is a line of curvature, $ t $ is a principal direction of $ S $, and $ \mathcal{S} (t) = k_n t $, where $ k_n $ is the normal curvature of $ S $ along $ C $. Thus, 
    \[
        -\tau n \cdot N - k_n b \cdot t = -\tau k_n / k = 0, 
    \]
    where $ k $ is the curvature of $ C $.  Since $ C $ is nowhere tangent to an asymptotic direction, $ k_n \neq 0 $, so $ \tau = 0 $. This implies $ b^{\prime} = - \tau n = 0 $, so 
    \[
        \frac{\mathrm{d}}{\mathrm{d}s} (b \cdot c) = c b^{\prime} = 0 \;\Longrightarrow\; b = \text{const.}
    \] 
    and hence $ C $ is a plane curve. 
\end{solutionManual}

% 3.2.14
\begin{exerciseManual}{3.2.14*}
    If the surface $S_1$ intersects the surface $S_2$ along the regular curve $C$, then the curvature $k$ of $C$ at $p \in C$ is given by
    \[
    k^2 \sin^2 \theta = \lambda_1^2 + \lambda_2^2 - 2 \lambda_1 \lambda_2 \cos \theta,
    \]
    where $\lambda_1$ and $\lambda_2$ are the normal curvatures at $p$, along the tangent line to $C$, of $S_1$ and $S_2$, respectively, and $\theta$ is the angle made up by the normal vectors of $S_1$ and $S_2$ at $p$.
\end{exerciseManual}

\begin{solutionManual}{3.2.14}
    Suppose $ S_1 $ and $ S_2 $ intersect along the regular curve $ C $. Let $ N_1 $, $ N_2 $ be the unit normals and let $ \lambda_1 $, $ \lambda_2 $ be the normal curvatures along the tangent line to $ C $ of $ S_1 $ and $ S_2 $, respectively. Let $ t $, $ n $, $ b $ be the Frenet frame of the curve $ C $. Since $ C $ lies on $ S_1 $ and $ S_2 $, $ t \perp N_i $, $ i = 1,2 $. Thus, we can write $ N_i = n \cos \phi_i + b \sin \phi_i $ for some $ \phi_i \in [0, \frac{\pi}{2}] $, $ i = 1, 2 $. The normal curvatures are given by
    \[
        \lambda_i = \alpha^{\prime\prime} \cdot N_i = k n \cdot N_i = k \cos \phi_i, \quad i = 1, 2.
    \]
    By definition, the angle $ \theta $ between $ N_1 $ and $ N_2 $ satisfies
    \[
        \cos \theta = N_1 \cdot N_2 = \cos \phi_1 \cos \phi_2 + \sin \phi_1 \sin \phi_2 = \cos (\phi_1 - \phi_2).
    \] 
    By direct computation, we have 
    \begin{align*}
        \lambda_1^2 + \lambda_2^2 - 2 \lambda_1 \lambda_2 \cos \theta &= k^2 (\cos^2 \phi_1 + \cos^2 \phi_2 - 2 \cos \phi_1 \cos \phi_2 \cos (\phi_1 - \phi_2)) \\
        &= k^2 \left( \cos^2 \phi_1 + \cos^2 \phi_2 - 2 \cos \phi_1 \cos \phi_2 ( \cos \phi_1 \cos \phi_2 + \sin \phi_1 \sin \phi_2) \right) \\
        &= k^2 \left(\cos^2 \phi_1 + \cos^2 \phi_2 - \cos^2 \phi_1 (1-\sin^2 \phi_2) \right. \\ 
        &\quad \left. - \cos^2 \phi_2 (1-\sin^2 \phi_1) - 2 \sin \phi_1 \sin \phi_2 \cos \phi_1 \cos \phi_2 \right)\\
        &= k^2 \left(\sin^2 \phi_1 \cos^2 \phi_2 + \sin^2 \phi_2 \cos^2 \phi_1 - 2 \sin \phi_1 \sin \phi_2 \cos \phi_1 \cos \phi_2 \right) \\
        &= k^2 \sin^2 (\phi_1 - \phi_2) = k^2 \sin^2 \theta.
    \end{align*}
\end{solutionManual}

\newpage 

\section{Chapter 3.3}

\begin{proposition}[Gaussian curvature as a ratio of areas]
    Let $ p \in S $ be such that $ K(p) \neq 0 $, and let $ V $ be a neighborhood of $ p $ where $ K $ does not change sign. Then 
    \[
        K(p) = \lim_{A \to 0} \frac{\operatorname{area} (N(A))}{\operatorname{area} (A)},
    \]
    where $ A \subseteq V $ is a region containing $ p $ and $ N(A) \subseteq S^2 $ is its spherical image by the Gauss map $ N: S \to S^2 $. The limit is taken through a sequence $ \{A_n\} $, where there is some $ N \in \mathbb{N} $ such that any ball about $ p $ contains all $ A_n $ for $ n > N $.
\end{proposition}

\begin{remark}
    The curvature of a plane curve $ C $ at $ p $ is given by 
    \[
        k(p) = \lim_{\ell (s) \to 0} \frac{\ell (T(s))}{\ell (s)}, 
    \]
    where $ T(s) $ is the image of $ s $ in the indicatrix of tangents, and $ \ell $ is the length function. Thus, the Gaussian curvature is the analogue for surfaces of the curvature of a plane curve.
\end{remark}

% 3.3.1
\begin{exerciseManual}{3.3.1}
    Show that at the origin $(0,0,0)$ of the hyperboloid $z = axy$ we have 
    \[
    K = -a^2, \qquad H = 0.
    \]
\end{exerciseManual}

\begin{solutionManual}{3.3.1}
    Consider the parametrization $ \bvec{x}(u,v) = (u,v, auv) $ of the hyperboloid $ z = axy $. The first-order partial derivatives are $ \bvec{x}_u = (1,\, 0,\, av) $, $ \bvec{x}_v = (0,\, 1,\, au) $. Thus, we have 
    \[
        E = \langle \bvec{x}_u, \bvec{x}_u \rangle = 1, \quad F = \langle \bvec{x}_u, \bvec{x}_v \rangle = 0, \quad G = \langle \bvec{x}_v, \bvec{x}_v \rangle = 1.
    \]
    The normal vector at the origin is
    \[
        N = \frac{\bvec{x}_u \wedge \bvec{x}_v}{|\bvec{x}_u \wedge \bvec{x}_v|} = \frac{(-av, -au, 1)}{\sqrt{1 + a^2 (u^2 + v^2)}} \implies N(0,0) = (0,0,1).
    \] 
    The second-order partial derivatives are $ \bvec{x}_{uu} = (0,\, 0,\, 0) $, $ \bvec{x}_{uv} = (0,\, 0,\, a) $, $ \bvec{x}_{vv} = (0,\, 0,\, 0) $, so 
    \[
        e = \langle \bvec{x}_{uu}, N \rangle = 0, \quad f = \langle \bvec{x}_{uv}, N \rangle = a, \quad g = \langle \bvec{x}_{vv}, N \rangle = 0.
    \]
    Finally, the Gaussian curvature and the mean curvature at the origin are
    \[
        K = \frac{eg - f^2}{EG - F^2} = \frac{0 - a^2}{1 \cdot 1 - 0} = -a^2, \quad H = \frac{Eg - 2Ff + Ge}{2(EG - F^2)} = \frac{1 \cdot 0 - 0 + 1 \cdot 0}{2(1 \cdot 1 - 0)} = 0.
    \]
\end{solutionManual}

% 3.3.2
\begin{exerciseManual}{3.3.2*}

    \noindent Determine the asymptotic curves and the lines of curvature of the helicoid 
    \[
    x = v \cos u, \quad y = v \sin u, \quad z = cu,
    \]
    and show that its mean curvature is zero.
\end{exerciseManual}

\begin{solutionManual}{3.3.2}
    ~ 

    \begin{observation}[Computing asymptotic directions and lines of curvature]
        For a tangent vector $ w \in T_p(S) $ and parametrization $ \bvec{x}(u,v) $, we can write $ w = \bvec{x}_u \mathrm{d}u + \bvec{x}_v \mathrm{d}v $. Then 
        \[
            k_n (w) = \operatorname{II}_p (w, w) / \operatorname{I}_p (w, w) = \frac{e \mathrm{d}u^2 + 2f \mathrm{d}u \mathrm{d}v + g \mathrm{d}v^2}{E \mathrm{d}u^2 + 2F \mathrm{d}u \mathrm{d}v + G \mathrm{d}v^2}.
        \]
        The asymptotic directions satisfy $ k_n (w) = 0 $, hence
        \[
            \operatorname{II}_p (w,w) = e \,\mathrm{d}u^2 + 2f \,\mathrm{d}u\, \mathrm{d}v + g \,\mathrm{d}v^2 = 0. 
        \]
        The lines of curvature are where $ k_n $ attains extremal values, so let $ \lambda = \mathrm{d}v / \mathrm{d}u $ and solve $ \mathrm{d} k_n / \mathrm{d} \lambda = 0 $: 
        \[
          \frac{\mathrm{d}k_n}{\mathrm{d}\lambda} = \frac{\mathrm{d}}{\mathrm{d}\lambda} \left( \frac{e + 2f \lambda + g \lambda^2}{E + 2F \lambda + G \lambda^2} \right) = 0, 
        \] 
        and hence 
        \[
          (fE - eF) \, \mathrm{d}\, u^2 + (gE - eG) \, \mathrm{d}u \, \mathrm{d}v + (gF - fG) \, \mathrm{d}v^2 = 0.
        \]
    \end{observation} 
    
    Consider the parametrization $ \bvec{x}(u,v) = (v \cos u, v \sin u, cu) $, we have 
    \[
        \bvec{x}_u = (-v \sin u, \, v \cos u, \, c), \quad 
        \bvec{x}_v = (\cos u, \,\sin u, \, 0).
    \]
    \[
        \implies E = \langle \bvec{x}_u, \bvec{x}_u \rangle = v^2 + c^2, \quad F = \langle \bvec{x}_u, \bvec{x}_v \rangle = 0, \quad G = \langle \bvec{x}_v, \bvec{x}_v \rangle = 1.
    \]
    The normal vector is given by
    \[
        N = \frac{\bvec{x}_u \wedge \bvec{x}_v}{|\bvec{x}_u \wedge \bvec{x}_v|} = \frac{(-c \sin u, c \cos u, -v)}{\sqrt{c^2 + v^2}}. 
    \]
    Then, we have $ \bvec{x}_{uu} = (-v \cos u, -v \sin u, 0) $, $ \bvec{x}_{uv} = (- \sin u, \cos u, 0) $, and $ \bvec{x}_{vv} = (0, 0, 0) $. Thus, the coefficients of the second fundamental form are
    \[
        e = \langle \bvec{x}_{uu}, N \rangle = 0, \quad 
        f = \langle \bvec{x}_{uv}, N \rangle = \frac{c}{\sqrt{c^2 + v^2}} , \quad 
        g = \langle \bvec{x}_{vv}, N \rangle = 0.
    \]
    Plug into the formulas in the observation, we have $ 2 f \, \mathrm{d}u \mathrm{d}v \neq 0 $. Since $ f \neq 0 $, it must be that $ \mathrm{d}u = 0 $ or $ \mathrm{d}v = 0 $, and hence the asymptotic curves are $ u = \text{const.} $ and $ v = \text{const.} $. For the lines of curvature, we have
    \[
        (fE - eF) \, \mathrm{d}u^2 + (gE - eG) \, \mathrm{d}u \, \mathrm{d}v + (gF - fG) \, \mathrm{d}v^2 = fE \, \mathrm{d}u^2 - fG \, \mathrm{d}v^2 = 0.
    \]
    Since $ f \neq 0 $, we have $ E \, \mathrm{d}u^2 - G \, \mathrm{d}v^2 = 0 $, or equivalently,
    \[
        (v^2 + c^2) \, \mathrm{d}u^2 - \mathrm{d}v^2 = 0 \implies \mathrm{d}u = \pm \frac{\mathrm{d}v}{\sqrt{v^2 + c^2}}.
    \]  
    Integrating both sides, we obtain the lines of curvature:
    \[
        u = \pm \sinh^{-1} \left( \frac{v}{c} \right) + \text{const.}
    \]
    Finally, the mean curvature is
    \[
        H = \frac{Eg - 2Ff + Ge}{2(EG - F^2)} = \frac{(v^2 + c^2) \cdot 0 - 0 + 1 \cdot 0}{2((v^2 + c^2) \cdot 1 - 0)} = 0.
    \]

    \begin{remark}
        The helicoid is a minimal surface since its mean curvature is zero.
    \end{remark}
\end{solutionManual}

% 3.3.3
\begin{exerciseManual}{3.3.3*}
    Determine the asymptotic curves of the catenoid
    \[
    \mathbf{x}(u,v) = (\cosh v \cos u, \cosh v \sin u, v).
    \]
\end{exerciseManual}

\begin{solutionManual}{3.3.3}
    The first-order partial derivatives are 
    \[
        \bvec{x}_u = (-\cosh v \sin u, \, \cosh v \cos u, \, 0), \quad 
        \bvec{x}_v = (\sinh v \cos u, \, \sinh v \sin u, \, 1).
    \] 
    \[
        \implies E = \langle \bvec{x}_u, \,\bvec{x}_u \rangle = \cosh^2 v, \quad F = \langle \bvec{x}_u, \, \bvec{x}_v \rangle = 0, \quad G = \langle \bvec{x}_v, \, \bvec{x}_v \rangle = \cosh^2 v.
    \]
    The normal vector is given by
    \[
        N = \frac{\bvec{x}_u \wedge \bvec{x}_v}{|\bvec{x}_u \wedge \bvec{x}_v|} = -\frac{(\cos u,\, \sin u, \,\sinh v)}{\sqrt{1 + \sinh^2 v}} = - (\operatorname{sech} v \cos u, \, \operatorname{sech} v \sin u, \, \tanh v). 
    \]
    Then, we have $ \bvec{x}_{uu} = (-\cosh v \cos u, \, -\cosh v \sin u, \, 0) $, $ \bvec{x}_{uv} = (-\sinh v \sin u, \, \sinh v \cos u, \, 0) $, and $ \bvec{x}_{vv} = (\cosh v \cos u, \, \cosh v \sin u, \, 0) $. Thus, the coefficients of the second fundamental form are
    \[
        e = \langle \bvec{x}_{uu},\, N \rangle = 1, \quad 
        f = \langle \bvec{x}_{uv},\, N \rangle = 0 , \quad 
        g = \langle \bvec{x}_{vv},\, N \rangle = - 1.
    \]
    The asymptotic curves satisfy null second fundamental form:
    \[
        \operatorname{II}_p (w,w) = e \,\mathrm{d}u^2 + 2f \,\mathrm{d}u\, \mathrm{d}v + g \,\mathrm{d}v^2 = e \,\mathrm{d}u^2 + g \,\mathrm{d}v^2 = 0.
    \]
    Since $ e = -g \neq 0 $, we have $ \mathrm{d}u^2 = \mathrm{d}v^2 $, or equivalently, $ \mathrm{d}u = \pm \mathrm{d}v $. Integrating both sides, we obtain the asymptotic curves $ u = \pm v + \text{const.} $
\end{solutionManual}

% 3.3.4
\begin{exerciseManual}{3.3.4}
    Determine the asymptotic curves and the lines of curvature of $z = xy$.
\end{exerciseManual}

\begin{solutionManual}{3.3.4}
    The parametrization is given by $ \bvec{x}(u,v) = (u, v, uv) $. Then we compute $ \bvec{x}_u = (1,\, 0,\, v) $, $ \bvec{x}_v = (0,\, 1,\, u) $, $ \bvec{x}_{uu} = (0,\,0,\,0) $, $ \bvec{x}_{uv} = (0,\,0,\,1) $, $ \bvec{x}_{vv} = (0,\,0,\,0) $, and 
    \[
        N = \frac{\bvec{x}_u \wedge \bvec{x}_v}{|\bvec{x}_u \wedge \bvec{x}_v|} = \frac{(-v, -u, 1)}{\sqrt{1 + u^2 + v^2}}.
    \] 
    Hence, we compute the coefficients of the first and second fundamental forms: 
    \[
        E = \langle \bvec{x}_u, \bvec{x}_u \rangle = 1 + v^2, \quad F = \langle \bvec{x}_u, \bvec{x}_v \rangle = uv, \quad G = \langle \bvec{x}_v, \bvec{x}_v \rangle = 1 + u^2,
    \]
    \[
        e = \langle \bvec{x}_{uu}, N \rangle = 0, \quad f = \langle \bvec{x}_{uv}, N \rangle = \frac{1}{\sqrt{1 + u^2 + v^2}}, \quad g = \langle \bvec{x}_{vv}, N \rangle = 0.
    \]
    The asymptotic curves satisfy null second fundamental form:
    \[
        \operatorname{II}_p (w,w) = e \,\mathrm{d}u^2 + 2f \,\mathrm{d}u\, \mathrm{d}v + g \,\mathrm{d}v^2 = 2f \,\mathrm{d}u\, \mathrm{d}v = 0.
    \]
    Since $ f \neq 0 $, it must be that $ \mathrm{d}u = 0 $ or $ \mathrm{d}v = 0 $, and hence the asymptotic curves are $ u = \text{const.} $ and $ v = \text{const.} $, corrsponding to the $ y $ and $ x $ axes, respectively. The lines of curvature satisfy
    \[
        (fE - eF) \, \mathrm{d}u^2 + (gE - eG) \, \mathrm{d}u \, \mathrm{d}v + (gF - fG) \, \mathrm{d}v^2 = fE \, \mathrm{d}u^2 - fG \, \mathrm{d}v^2 = 0.
    \]
    Since $ f \neq 0 $, we have $ E \, \mathrm{d}u^2 - G \, \mathrm{d}v^2 = 0 $, or equivalently,
    \[
        (1 + v^2) \, \mathrm{d}u^2 - (1 + u^2) \, \mathrm{d}v^2 = 0 \implies \frac{\mathrm{d}u}{\mathrm{d}v} = \pm \sqrt{\frac{1 + u^2}{1 + v^2}}.
    \]
    Hence, the lines of curvature are given by $ \sinh^{-1} u \pm \sinh^{-1}v = \text{const.} $
\end{solutionManual}

% 3.3.5
\begin{exerciseManual}{3.3.5}\textit{(Enneper's Surface)}

    \noindent Consider the parametrized surface (Enneper’s surface)
    \[
    \mathbf{x}(u,v) = \left(u - \frac{u^3}{3} + uv^2,\; v - \frac{v^3}{3} + vu^2,\; u^2 - v^2 \right)
    \]
    and show that
    \begin{enumerate}[label=\textbf{\alph*.}]
        \item The coefficients of the first fundamental form are
        \[
        E = G = (1 + u^2 + v^2)^2, \quad F = 0.
        \]
        \item The coefficients of the second fundamental form are
        \[
        e = 2, \quad g = -2, \quad f = 0.
        \]
        \item The principal curvatures are
        \[
        k_1 = \frac{2}{(1 + u^2 + v^2)^2}, \qquad 
        k_2 = -\frac{2}{(1 + u^2 + v^2)^2}.
        \]
        \item The lines of curvature are the coordinate curves.
        \item The asymptotic curves are $u+v=\text{const.}$ and $u-v=\text{const.}$
    \end{enumerate}
\end{exerciseManual}

\begin{solutionManual}{3.3.5}
    ~

    \begin{enumerate}[label=\textbf{\alph*.}]
        \item Calculate the first-order partial derivatives: 
        \[
            \bvec{x}_u = \left(1 - u^2 + v^2,\, 2uv,\, 2u\right), \quad \bvec{x}_v = \left(2uv,\, 1 - v^2 + u^2,\, -2v\right).
        \]
        Then the coefficients of the first fundamental form are
        \begin{align*}
            E &= \langle \bvec{x}_u, \bvec{x}_u \rangle = (1 - u^2 + v^2)^2 + 4u^2 v^2 + 4u^2 = (1 + u^2 + v^2)^2, \\
            F &= \langle \bvec{x}_u, \bvec{x}_v \rangle = 2uv(1-u^2+v^2) + 2uv(1+u^2-v^2) - 4uv = 0, \\
            G &= \langle \bvec{x}_v, \bvec{x}_v \rangle = 4u^2 v^2 + (1 + u^2 - v^2)^2 + 4v^2 = (1 + u^2 + v^2)^2.
        \end{align*}

        \item Calculate the second-order partial derivatives:
        \[
            \bvec{x}_{uu} = \left(-2u,\, 2v,\, 2\right), \quad \bvec{x}_{uv} = \left(2v,\, 2u,\, 0\right), \quad \bvec{x}_{vv} = \left(2u,\, -2v,\, -2\right).
        \]
        Next, we find the normal vector: 
        \[
            \bvec{x}_u \wedge \bvec{x}_v = \left( -2u (1 + r^2),\, 2v (1 + r^2),\, 1-r^4 \right), \quad \text{where } r^2 = u^2 + v^2,
        \]
        \[
            \left\vert \bvec{x}_u \wedge \bvec{x}_v \right\vert = (1+r^2)^2 .
        \]
        Therefore,    
        \[
            N = \frac{\bvec{x}_u \wedge \bvec{x}_v}{\left\vert \bvec{x}_u \wedge \bvec{x}_v \right\vert} = \frac{1}{(1 + u^2 + v^2)} \left(-2u,\, 2v,\, 1 - u^2 - v^2\right).
        \]
        The coefficients of the second fundamental form are given by the following inner products:
        \begin{align*}
            e &= \langle N, \bvec{x}_{uu} \rangle = \frac{1}{(1 + u^2 + v^2)} \left( 4u^2 + 4v^2 + 2(1 - u^2 - v^2)\right) = 2, \\
            f &= \langle N, \bvec{x}_{uv} \rangle = \frac{1}{(1 + u^2 + v^2)} \left( -4uv + 4uv + 0\right) = 0, \\
            g &= \langle N, \bvec{x}_{vv} \rangle = \frac{1}{(1 + u^2 + v^2)} \left( -4u^2 - 4v^2 - 2(1 - u^2 - v^2) \right) = -2.
        \end{align*}

        \item The shape operator in the $ (u,v) $ basis is given by $ S = \operatorname{I}^{-1} \operatorname{II} $, where  
        \[
            \operatorname{I} = \begin{pmatrix}
                E & F \\
                F & G
            \end{pmatrix} = \begin{pmatrix}
                (1 + u^2 + v^2)^2 & 0 \\
                0 & (1 + u^2 + v^2)^2
            \end{pmatrix},
        \]
        and
        \[
            \operatorname{II} = \begin{pmatrix}
                e & f \\
                f & g
            \end{pmatrix} = \begin{pmatrix}
                2 & 0 \\
                0 & -2
            \end{pmatrix}.
        \]
        Thus,
        \[
            S = \operatorname{I}^{-1} \operatorname{II} = \frac{1}{(1 + u^2 + v^2)^2} \begin{pmatrix}
                1 & 0 \\
                0 & 1
            \end{pmatrix} \begin{pmatrix}
                2 & 0 \\
                0 & -2
            \end{pmatrix} = \frac{1}{(1 + u^2 + v^2)^2} \begin{pmatrix}
                2 & 0 \\
                0 & -2
            \end{pmatrix}.
        \]
        The principal curvatures are the eigenvalues of the shape operator, which are easily seen to be
        \[
            k_1 = \frac{2}{(1+u^2 + v^2)^2}, \quad k_2 = -\frac{2}{(1 + u^2 + v^2)^2}.
        \]
        \item The lines of curvature correspond to the eigenvectors of the shape operator, which are $ \partial_u $ and $ \partial_v $. Since the shape operator is diagonal in the $ (\bvec{x}_u, \bvec{x}_v) $ basis, the lines of curvature are the coordinate curves $ u = \text{const.} $ and $ v = \text{const.} $.
        
        \item For each $ p $ on an asymptotic curve, the normal curvature in the direction of the tangent vector is zero. The normal curvature $ k_n $ in the direction of a unit tangent vector $ \bvec{t} = a \bvec{x}_u + b \bvec{x}_v $ is given by
        \[
            k_n = \langle S(\bvec{t}), \bvec{t} \rangle = \frac{2}{(1 + u^2 + v^2)^2} ((\mathrm{d}u)^2 - (\mathrm{d}v)^2).
        \]
        Setting $ k_n = 0 $ gives $ (\mathrm{d}u)^2 = (\mathrm{d}v)^2 $, which implies $ \mathrm{d}u = \pm \mathrm{d}v $. Therefore, the asymptotic directions correspond to the curves where $ u + v = \text{const.} $ and $ u - v = \text{const.} $

        \begin{remark}
            Since the mean curvature $ H = \frac{k_1 + k_2}{2} = 0 $ everywhere, Enneper's surface is a minimal surface.
        \end{remark}
    \end{enumerate}
\end{solutionManual}

% 3.4.6
\begin{exerciseManual}{3.4.6}\textit{(A Surface with $K \equiv -1$; the Pseudosphere)}
    ~ 

    \begin{enumerate}[label=\textbf{\alph*.}]
        \item[\textbf{*a.}] Determine an equation for the plane curve $C$, which is such that the segment of the tangent line between the point of tangency and some line $r$ in the plane, which does not meet the curve, is constantly equal to $1$ (this curve is called the \emph{tractrix}; see Fig.\ 1--9).
        
        \item[\textbf{b.}] Rotate the tractrix $C$ about the line $r$; determine if the "surface" of revolution thus obtained (the \emph{pseudosphere}; see Fig.\ 3--22) is regular and find a parametrization in a neighborhood of a regular point.
        \item[] 
        \item[\textbf{c.}] Show that the Gaussian curvature of any regular point of the pseudosphere is $-1$.
    \end{enumerate}

    \begin{figure}[htbp]
        \centering
        \includegraphics[width=0.4\textwidth]{3-22.png}
        % \caption{The pseudosphere.}
    \end{figure}
\end{exerciseManual}

\begin{solutionManual}{3.3.6}
    ~

    \begin{enumerate}[label=\textbf{\alph*.}]
        \item Let $ C $ be the curve parametrized by arc length $ s $, i.e., $ \alpha(s) = (x(s), z(s)) $, with $ s \geq 0 $. Assume that the line $ r $ is the $ x $-axis. The tangent vector at $ \alpha(s) $ is given by $ \alpha'(s) = (x'(s), z'(s)) $. The line tangent to $ C $ at $ \alpha(s) $ intersects the $ x $-axis at the point 
        \[
            T(s) = \left( x(s) - \frac{z(s)}{z'(s)} x'(s), 0 \right).
        \]
        The length of the segment between the point of tangency and the intersection point is given by
        \[
            \ell (s) = |\alpha(s) - T(s)| = \sqrt{\left( \frac{z(s)}{z'(s)} x'(s) \right)^2 + z(s)^2} = z(s) \sqrt{1 + \left( \frac{x'(s)}{z'(s)} \right)^2 }.
        \]
        Since $ s $ is the arc length parameter, we have
        \[
            (x'(s))^2 + (z'(s))^2 = 1 \implies 1 + \left( \frac{x'(s)}{z'(s)} \right)^2 = \frac{1}{(z'(s))^2}.
        \]
        Therefore,
        \[
            \ell (s) = z(s) \cdot \frac{1}{|z'(s)|} = -\frac{z(s)}{z'(s)}, \quad \text{ where } z'(s) < 0.
        \]
        Setting $ \ell (s) = 1 $, we have $z^{\prime} (s) = -z(s)$, and hence $ z(s) = z(0) e^{-s} $. By arc length parametrization, we have
        \[
            x(s) = \int_0^s \mathrm{d}t\, x^{\prime} (t) = \int_0^s \mathrm{d}t\, \sqrt{1 - a^2 e^{-2t}}, \quad a \equiv z(0).
        \]
        Thus, the tractrix is given by
        \[
            C: \quad \alpha(s) = \left( \int_{0}^{s} \sqrt{1 - a^2 e^{-2t}} \, dt, \; a e^{-s} \right).
        \]

        \item Rotate the tractrix $ C $ about the $ x $-axis. The parametrization of the pseudosphere is given by
        \[
            \bvec{x}(u,v) = \left( \int_{0}^{u} \sqrt{1 - a^2 e^{-2t}} \, dt, \; a e^{-u} \cos v, \; a e^{-u} \sin v \right), \quad u \geq 0, \; 0 \leq v < 2\pi.
        \]

        \item We will compute the first and second fundamental forms to find the Gaussian curvature. First, we have 
        \[
            \bvec{x}_u = \left( \sqrt{1 - a^2 e^{-2u}}, \; -a e^{-u} \cos v, \; -a e^{-u} \sin v \right), \quad 
            \bvec{x}_v = \left( 0, \; -a e^{-u} \sin v, \; a e^{-u} \cos v \right).
        \]
        Thus, the coefficients of the first fundamental form are
        \[
            E = \langle \bvec{x}_u, \, \bvec{x}_u \rangle = 1, \quad F = \langle \bvec{x}_u, \, \bvec{x}_v \rangle = 0, \quad G = \langle \bvec{x}_v, \, \bvec{x}_v \rangle = a^2 e^{-2u}.
        \]
        Next, we have 
        \[
            \begin{split}
                N &= \frac{\bvec{x}_u \wedge \bvec{x}_v}{|\bvec{x}_u \wedge \bvec{x}_v|} = \frac{\left( - a^2 e^{-2u}, \, - a e^{-u} \cos v \sqrt{1 - a^2 e^{-2u}}, \, - a e^{-u} \sin v \sqrt{1 - a^2 e^{-2u}} \right)}{a e^{-u}} \\
                &= \left( - a e^{-u}, \, - \cos v \sqrt{1 - a^2 e^{-2u}}, \, - \sin v \sqrt{1 - a^2 e^{-2u}} \right), 
            \end{split}
        \]
        and 
        \begin{align*}
            \bvec{x}_{uu} &= \left( \frac{a^2 e^{-2u}}{\sqrt{1 - a^2 e^{-2u}}}, \; a e^{-u} \cos v, \; a e^{-u} \sin v \right), \\
            \bvec{x}_{uv} &= \left( 0, \; a e^{-u} \sin v, \; - a e^{-u} \cos v \right), \\ 
            \bvec{x}_{vv} &= \left( 0, \; - a e^{-u} \cos v, \; - a e^{-u} \sin v \right).
        \end{align*}
        Then, 
        \begin{align*}
            e &= \langle N, \, \bvec{x}_{uu} \rangle = -a e^{-u} \left( \frac{a^2 e^{-2u}}{\sqrt{1 - a^2 e^{-2u}}} + \sqrt{1 - a^2 e^{-2u}}\right) = \frac{-a e^{-u}}{\sqrt{1 - a^2 e^{-2u}}}, \\
            f &= \langle N, \, \bvec{x}_{uv} \rangle = 0, \\
            g &= \langle N, \, \bvec{x}_{vv} \rangle = a e^{-u} \sqrt{1 - a^2 e^{-2u}}.
        \end{align*}
        Finally, the Gaussian curvature is given by
        \[
            K = \frac{eg - f^2}{EG - F^2} = \frac{\left( \frac{-a e^{-u}}{\sqrt{1 - a^2 e^{-2u}}} \right) \left( a e^{-u} \sqrt{1 - a^2 e^{-2u}} \right) - 0}{1 \cdot a^2 e^{-2u} - 0} = -1.
        \]
    \end{enumerate}
\end{solutionManual}

% 3.4.7
\begin{exerciseManual}{3.3.7}\textit{(Surfaces of Revolution with Constant Gaussian Curvature)}
    ~
    
    \noindent A surface of revolution
    \[
        (\varphi(v)\cos u,\; \varphi(v)\sin u,\; \psi(v)), \qquad \varphi(v)\neq 0,
    \]
    is given as a surface of revolution with constant Gaussian curvature $K$. Choose the parameter $v$ such that
    \[
    (\varphi')^{2} + (\psi')^{2} = 1,
    \]
    that is, $v$ is the arc length of the generating curve $(\varphi(v), \psi(v))$. Show that:
    \begin{enumerate}[label=\textbf{\alph*.}]
        \item $\varphi$ satisfies $\varphi'' + K\varphi = 0$ and $\psi$ is given by 
        \[
            \psi(v) = \int \sqrt{1 - (\varphi')^{2}}\, dv,
        \]
        thus $0 < u < 2\pi$, and the domain of $v$ is such that the last integral makes sense.
        \item All surfaces of revolution with constant curvature $K = 1$ which intersect perpendicularly the plane $xOy$ are given by
        \[
            \varphi(v) = C\cos v, \qquad 
            \psi(v) = \int_{0}^{v} \sqrt{1 - C^{2}\sin^{2} t}\, dt,
        \]
        where $C$ is a constant $(C = \varphi(0))$. Determine the domain of $v$ and draw a rough sketch of the profile of the surface in the $xz$-plane for the cases $C=1$, $C>1$, $C<1$. (Observe that $C = 1$ gives a sphere.)
        \item All surfaces of revolution with constant curvature $K = -1$ may be given by one of the following types:
        \begin{enumerate}
            \item $\varphi(v) = C\cosh v$,  
            \[
                \psi(v) = \int_{0}^{v} \sqrt{1 - C^{2}\sinh^{2} t}\, dt;
            \]
            \item $\varphi(v) = C\sinh v$,  
            \[
                \psi(v) = \int_{0}^{v} \sqrt{1 - C^{2}\cosh^{2} t}\, dt;
            \]
            \item $\varphi(v) = e^{v}$,  
            \[
                \psi(v) = \int_{0}^{v} \sqrt{1 - e^{2t}}\, dt.
            \]
        \end{enumerate}
        Determine the domain of $v$ and draw a rough sketch of the profile of the surface in the $xz$-plane.
        \item The surface of type 3 in part (c) is the pseudosphere of Exercise 6.
        \item The only surfaces of revolution with $K \equiv 0$ are the right circular cylinder, the right circular cone, and the plane.
    \end{enumerate}
\end{exerciseManual}

\begin{solutionManual}{3.3.7}
    ~ 

    \begin{enumerate}[label=\textbf{\alph*.}]
        \item The generating curve $ \alpha (v) = \left(\varphi (v), \psi (v)\right) $ is arc length parametrized. Here we follow the steps in Example 4 and comput 
        \[
            \bvec{x}_u = \left( -\varphi (v) \sin u, \, \varphi (v) \cos u, \, 0 \right), \quad 
            \bvec{x}_v = \left( \varphi' (v) \cos u, \, \varphi' (v) \sin u, \, \psi' (v) \right).
        \]
        Then, $ E = \varphi^2 $, $ F = 0 $, and $ G = \left(\varphi^{\prime} \right)^2 + \left(\psi^{\prime} \right)^2 = 1 $. The normal vector is given by
        \[
            N = \frac{\bvec{x}_u \wedge \bvec{x}_v}{|\bvec{x}_u \wedge \bvec{x}_v|} = \left( -\frac{\psi' (v) \cos u}{\varphi (v)}, \, -\frac{\psi' (v) \sin u}{\varphi (v)}, \, \varphi' (v) \right).
        \]
        Then, 
        \begin{align*}
            \bvec{x}_{uu} &= \left( -\varphi (v) \cos u, \, -\varphi (v) \sin u, \, 0 \right), \\ 
            \bvec{x}_{uv} &= \left( -\varphi' (v) \sin u, \, \varphi' (v) \cos u, \, 0 \right), \\ 
            \bvec{x}_{vv} &= \left( \varphi'' (v) \cos u, \, \varphi'' (v) \sin u, \, \psi'' (v) \right).
        \end{align*}
        and 
        \begin{align*}
            e &= \langle N, \, \bvec{x}_{uu} \rangle = \psi' (v), \\
            f &= \langle N, \, \bvec{x}_{uv} \rangle = 0, \\
            g &= \langle N, \, \bvec{x}_{vv} \rangle = \varphi'' (v) \left( -\frac{\psi' (v)}{\varphi (v)} \right) + \psi'' (v) \varphi' (v)
        \end{align*}
        The Gaussian curvature is given by
        \[
            K = \frac{eg - f^2}{EG - F^2} = 
        \]

        \item If the surface intersects perpendicularly the plane $ xOy $, then $ \psi' (0) = 0 $, which implies $ \varphi^{\prime} (0) = \pm 1 $. Without loss of generality, we may take $ \varphi^{\prime} (0) = 1 $. The solution of $ \phi'' + \phi = 0 $ subject to this initial condition is $ \varphi (v) = C \cos v $. Since $ (\varphi')^2 + (\psi')^2 = 1 $, we have $ (\psi')^2 = 1 - C^2 \sin^2 v $, and hence 
        \[
          \varphi (v) = C \cos v, \quad \psi (v) = \int_0^v \mathrm{d}t\, \sqrt{1 - C^2 \sin^2 t}.
        \]
        
        \item 
    \end{enumerate}
\end{solutionManual}

% 3.3.8
\begin{exerciseManual}{3.3.8}\textit{(Contact of Order $\ge 2$ of Surfaces)}
    Two surfaces $S$ and $\bar{S}$, with a common point $p$, have \emph{contact of order $\ge 2$ at $p$} if there exist parametrizations $\mathbf{x}(u,v)$ and $\bar{\mathbf{x}}(u,v)$ in $p$ of $S$ and $\bar{S}$, respectively, such that
    \[
    \mathbf{x}_u = \bar{\mathbf{x}}_u, \quad
    \mathbf{x}_v = \bar{\mathbf{x}}_v, \quad
    \mathbf{x}_{uu} = \bar{\mathbf{x}}_{uu}, \quad
    \mathbf{x}_{uv} = \bar{\mathbf{x}}_{uv}, \quad
    \mathbf{x}_{vv} = \bar{\mathbf{x}}_{vv}.
    \]
    \begin{enumerate}
        \item[\textbf{a.}] Let $S$ and $\bar{S}$ have contact of order $\ge 2$ at $p$; $\mathbf{x}\colon U \to S$ and $\bar{\mathbf{x}}\colon U \to \bar{S}$ be arbitrary parametrizations in $p$ of $S$ and $\bar{S}$ respectively; and $f\colon V \subset \mathbb{R}^3 \to \mathbb{R}$ be a differentiable function in a neighborhood $V$ of $p$ in $\mathbb{R}^3$. Then the partial derivatives of order $\le 2$ of $f \circ \bar{\mathbf{x}}\colon U \to \mathbb{R}$ are zero in $\bar{\mathbf{x}}^{-1}(p)$ if and only if the partial derivatives of order $\le 2$ of $f \circ \mathbf{x}\colon U \to \mathbb{R}$ are zero in $\mathbf{x}^{-1}(p)$.

        \item[\textbf{*b.}] Let $S$ and $\bar{S}$ have contact of order $\ge 2$ at $p$. Let $z = f(x,y)$ and $z = \bar{f}(x,y)$ be the equations, in a neighborhood of $p$, of $S$ and $\bar{S}$, respectively, where the $xy$-plane is the common tangent plane at $p=(0,0)$. Then the function $f(x,y) - \bar{f}(x,y)$ has all partial derivatives of order $\le 2$ equal to zero at $(0,0)$.
        
        \item[\textbf{c.}] Let $p$ be a point in a surface $S \subset \mathbb{R}^3$. Let $Oxyz$ be a Cartesian coordinate system for $\mathbb{R}^3$ such that $O=p$ and the $xy$-plane is the tangent plane of $S$ at $p$. Show that the paraboloid
        \[
        z = \tfrac{1}{2}(x^2 f_{xx} + 2xy f_{xy} + y^2 f_{yy}),
        \]
        obtained by neglecting third- and higher-order terms in the Taylor development around $p=(0,0)$, has contact of order $\ge 2$ at $p$ with $S$ (the surface $(*)$ is called the \emph{osculating paraboloid} of $S$ at $p$).

        \item[\textbf{*d.}] If a paraboloid (the degenerate cases of plane and parabolic cylinder are included) has contact of order $\ge 2$ with a surface $S$ at $p$, then it is the osculating paraboloid of $S$ at $p$.

        \item[\textbf{*e.}] If two surfaces have contact of order $\ge 2$ at $p$, then the osculating paraboloids of $S$ and $\bar{S}$ at $p$ coincide. Conclude that the Gaussian and mean curvatures of $S$ and $\bar{S}$ at $p$ are equal.

        \item[\textbf{*f.}] The notion of contact of order $\ge 2$ is invariant by diffeomorphisms of $\mathbb{R}^3$; that is, if $S$ and $\bar{S}$ have contact of order $\ge 2$ at $p$ and $\varphi\colon \mathbb{R}^3 \to \mathbb{R}^3$ is a diffeomorphism, then $\varphi(S)$ and $\varphi(\bar{S})$ have contact of order $\ge 2$ at $\varphi(p)$.

        \item[\textbf{*g.}] If $S$ and $\bar{S}$ have contact of order $\ge 2$ at $p$, then
        \[
        \lim_{r \to 0} \frac{d}{r^2} = 0,
        \]
        where $d$ is the length of the segment cut by the surfaces in a straight line normal to $T_p(S) = T_p(\bar{S})$, which is at a distance $r$ from $p$.
    \end{enumerate}
\end{exerciseManual}

\begin{solutionManual}{3.3.8}
    ~ 

    \begin{enumerate}[label=\textbf{\alph*.}]
        \item Suppose the partial derivatives of order $\le 2$ of $f \circ \bar{\mathbf{x}}$ are zero in $\bar{\mathbf{x}}^{-1}(p)$. Then, by the chain rule, we have
        \[
            (f \circ \bar{\mathbf{x}})_u = \nabla f \cdot \bar{\mathbf{x}}_u = 0, \quad (f \circ \bar{\mathbf{x}})_v = \nabla f \cdot \bar{\mathbf{x}}_v = 0,
        \]
        \[
            (f \circ \bar{\mathbf{x}})_{uu} = \nabla f \cdot \bar{\mathbf{x}}_{uu} + \bar{\mathbf{x}}_u^T H_f \bar{\mathbf{x}}_u = 0,
        \]
        \[
            (f \circ \bar{\mathbf{x}})_{uv} = \nabla f \cdot \bar{\mathbf{x}}_{uv} + \bar{\mathbf{x}}_u^T H_f \bar{\mathbf{x}}_v = 0,
        \]
        \[
            (f \circ \bar{\mathbf{x}})_{vv} = \nabla f \cdot \bar{\mathbf{x}}_{vv} + \bar{\mathbf{x}}_v^T H_f \bar{\mathbf{x}}_v = 0,
        \]
        where $H_f$ is the Hessian matrix of $f$ at $p$. Since $S$ and $\bar{S}$ have contact of order $\ge 2$ at $p$, in the region $ \bvec{x}^{-1}(p) $ we have $ (f \circ \mathbf{x})_{uu} = \nabla f \cdot \bvec{x}_{uu} + \bvec{x}_u^T H_f \bvec{x}_u = \nabla f \cdot \overline{\bvec{x}}_{uu} + \overline{\bvec{x}}_u^T H_f \overline{\bvec{x}}_u = 0 $. Similarly, $ (f \circ \mathbf{x})_{uv} = (f \circ \mathbf{x})_{vv} = (f \circ \bvec{x})_u = (f \circ \bvec{x})_v = 0 $. The converse follows by symmetry.

        \item Since $ S $, $ \overline{S} $ have $ z=0 $ as the common tangent plane, their graph at $ p=0 $ satisfy $ f(0,0) = \overline{f} (0,0) = 0 $ and $ \nabla f(0,0) = \nabla \overline{f} (0,0) = 0 $. Let's define the function $ F: \mathbb{R}^3 \to \mathbb{R} $, such that $ F(x,y,z) = z - \frac{1}{2}f_{xx}(0,0) x^2 - f_{xy}(0,0)xy - \frac{1}{2}f_{yy}(0,0) y^2 $. Since $ F $ is a polynomial of $ x, y, x $, it is differentiable. The parametrizations $ \bvec{x} $, $ \overline{\bvec{x}} $ for $ S $ and $ \overline{S} $ at $ p $ are given by $ \bvec{x}(x,y) = \left(x, y, f(x,y)\right) $ and $ \overline{\bvec{x}} (x,y) = \left(x, y, \overline{f}(x,y)\right) $, respectively. Then, $ (F \circ \bvec{x}) (x,y) = f(x,y) - \frac{1}{2} f_{xx}(0,0) x^2 - f_{xy} (0,0) xy - \frac{1}{2}f_{yy}(0,0) y^2 $, so all the partial derivatives of order $ \le 2 $ of $ F \circ \bvec{x} $ at $ (0,0) $ are zero. By part \textbf{a.}, all the partial derivatives of order $ \le 2 $ of $ F \circ \overline{\bvec{x}} $ at $ (0,0) $ are also zero. Therefore, 
        \[
            F \circ \overline{\bvec{x}} (x,y) = \overline{f}(x,y) - \frac{1}{2} f_{xx}(0,0) x^2 - f_{xy} (0,0) xy - \frac{1}{2}f_{yy}(0,0) y^2
        \] 
        has all partial derivatives of order $ \le 2 $ vanish at $ p $. Thus, the function $ f(x,y) - \overline{f}(x,y) $ has all partial derivatives of order $ \le 2 $ vanish at $ p $.

        \item In a neighborhood of $ p $, the surface $ S $ can be expressed as the graph of a function $ z = f(x,y) $, where the $ xy $-plane is the tangent plane at $ p $. Since the $ xy $-plane is the tangent plane at $ p $, we have $ f(0,0) = f_x(0,0) = f_y(0,0) = 0 $, so the Taylor expansion of $ f(x,y) $ around $ p $ is given by
        \[
            f(x,y) = \frac{1}{2} \left( f_{xx}(0,0) x^2 + 2 f_{xy}(0,0) xy + f_{yy}(0,0) y^2 \right) + R_3(x,y).
        \]
        Let $ \overline{S} $ be the paraboloid defined by
        \[
            z = g(x,y) = \frac{1}{2} \left( f_{xx}(0,0) x^2 + 2 f_{xy}(0,0) xy + f_{yy}(0,0) y^2 \right).
        \]
        The parametrizations for $ S $ and $ \overline{S} $ at $ p $ are given by $ \bvec{x}(x,y) = (x, y, f(x,y)) $ and $ \overline{\bvec{x}}(x,y) = (x, y, g(x,y)) $, respectively. The second-order partial derivatives of $ f $ and $ g $ at $ p $ are equal, since the remainder term $ R_3(x,y) $ contains only terms of order $ \ge 3 $. Therefore, by definition, $ S $ and $ \overline{S} $ have contact of order $ \ge 2 $ at $ p $.

        \item Suppose a paraboloid $ \overline{S} $ has contact of order $ \ge 2 $ with a surface $ S $ at $ p $. Let the equation of $ S $ in a neighborhood of $ p $ be given by $ z = f(x,y) $, where the $ xy $-plane is the tangent plane at $ p $. The equation of the paraboloid $ \overline{S} $ can be expressed as
        \[
            z = \overline{f}(x,y) = a x^2 + 2b xy + c y^2,
        \]
        for some constants $ a, b, c \in \mathbb{R} $. The second-order Taylor expansion of $ f(x,y) $ around $ p $ is given by
        \[
            f(x,y) = \frac{1}{2} \left( f_{xx}(0,0) x^2 + 2 f_{xy}(0,0) xy + f_{yy}(0,0) y^2 \right).
        \]
        Comparing this with the expression for $ \overline{f}(x,y) $, we find that
        \[
            a = \frac{1}{2} f_{xx}(0,0), \quad b = \frac{1}{2} f_{xy}(0,0), \quad c = \frac{1}{2} f_{yy}(0,0).
        \]
        Thus, the paraboloid $ \overline{S} $ is the osculating paraboloid of $ S $ at $ p $ as defined in \textbf{c.}.

        \item Let $ P $, $ \overline{P} $ be the osculating paraboloids of $ S $ and $ \overline{S} $, respectively. By \textbf{b.}, $ S $, $ \overline{S} $ have contact of order $ \geq 2 $ at $ p $ with $ P $, $ \overline{P} $, respectively. Since $ S $ also has contact of order $ \geq 2 $ with $ \overline{S} $, all the partial derivatvies of order $ \leq 2 $ of $ f $ and $ \overline{f} $ vanish at $ p $, where $ f $, $ \overline{f} $ are the equations in a neighborhood of $ p $, of $ S $ and $ \overline{S} $, respectively. Therefore, 
        \[
            \frac{1}{2} \left(f_{xx}(p)x^2 + 2f_{xy}(p)xy + f_{yy}(p)y^2 \right) = \frac{1}{2} \left(\overline{f}_{xx}(p)x^2 + 2\overline{f}_{xy}(p)xy + \overline{f}_{yy}(p)y^2 \right),
        \] 
        and the osculating paraboloids $ P $ and $ \overline{P} $ coincide. Since the Gaussian and mean curvatures depend only on the partial derivatives of order $ \leq 2 $ of the parametrization at $ p $, the Gaussian and mean curvatures of $ S $ and $ \overline{S} $ at $ p $ are equal.

        \item Suppose $ S $ and $ \overline{S} $ have contact of order $ \geq 2 $ at $ p $. Let $ \varphi: \mathbb{R}^3 \to \mathbb{R}^3 $ be a diffeomorphism. The parametrizations for $ S $ and $ \overline{S} $ at $ p $ are given by $ \bvec{x}(u,v) $ and $ \overline{\bvec{x}}(u,v) $, respectively. The parametrizations for $ \varphi(S) $ and $ \varphi(\overline{S}) $ at $ \varphi(p) $ are given by $ \bvec{y} = \left(\varphi \circ \bvec{x}\right)(u,v) $ and $ \overline{\bvec{y}} = \left( \varphi \circ \overline{\bvec{x}}\right) (u,v) $, respectively. Then, by the chain rule, we have
        \[
            \bvec{y}_u = \mathrm{d}\varphi_{\bvec{x}} \cdot \bvec{x}_u, \quad \bvec{y}_v = \mathrm{d}\varphi_{\bvec{x}} \cdot \bvec{x}_v, \quad \bvec{y}_{uu} = \mathrm{d}^2 \varphi_{\bvec{x}} (\bvec{x}_u, \bvec{x}_u) + \mathrm{d}\varphi_{\bvec{x}} \cdot \bvec{x}_{uu},
        \]
        \[
            \bvec{y}_{uv} = \mathrm{d}^2 \varphi |_{\bvec{x}} (\bvec{x}_u, \bvec{x}_v) + \mathrm{d}\varphi |_{\bvec{x}} \cdot \bvec{x}_{uv}, \quad 
            \bvec{y}_{vv} = \mathrm{d}^2 \varphi |_{\bvec{x}} (\bvec{x}_v, \bvec{x}_v) + \mathrm{d}\varphi |_{\bvec{x}} \cdot \bvec{x}_{vv},
        \]
        and similarly for $ \overline{\bvec{y}} $, where $ \mathrm{d}^2 \phi |_{\bvec{x}} $ is the bilinear differential of $ \phi $ evaluated at $ \bvec{x} $. 
        
        Since $ S $ and $ \overline{S} $ have contact of order $ \geq 2 $ at $ p $, it follows that $ \bvec{y}_u = \overline{\bvec{y}}_u $, $ \bvec{y}_v = \overline{\bvec{y}}_v $, $ \bvec{y}_{uu} = \overline{\bvec{y}}_{uu} $, $ \bvec{y}_{uv} = \overline{\bvec{y}}_{uv} $, and $ \bvec{y}_{vv} = \overline{\bvec{y}}_{vv} $. Thus, $ \varphi(S) $ and $ \varphi(\overline{S}) $ have contact of order $ \geq 2 $ at $ \varphi(p) $.

        \item We may choose a Cartesian coordinate system $ Oxyz $ such that $ O = p $, and $ z=0 $ is the common tangent plane of $ S $ and $ \overline{S} $ at $ p $. Let the equations of $ S $ and $ \overline{S} $ in a neighborhood of $ p $ be given by $ z = f(x,y) $ and $ z = \overline{f}(x,y) $, respectively. Since $ S $ and $ \overline{S} $ have contact of order $ \geq 2 $ at $ p $, by part \textbf{b.}, all the partial derivatives of order $ \leq 2 $ of the function $ G(x,y) \equiv f(x,y) - \overline{f}(x,y) $ vanish at $ p $. Therefore, $ G(0,0) = \nabla G (0,0) = \nabla^2 G(0,0) = 0 $, where $ \nabla^2 G $ is the Hessian matrix of $ G $. Take a point $ q = (x,y,0) \in T_p (S) $ in the tangent plane, a distance $ r = \sqrt{x^2 + y^2} $ from $ p $. The straight line $ L_q $ normal to the tangent plane passing through $ q $ intersects the surfaces $ S $ and $ \overline{S} $ at the points $ (x,y,f(x,y)) $ and $ (x,y,\overline{f}(x,y)) $, respectively, and $ d = \vert f(x,y) - \overline{f} (x,y) \vert = \vert G(x,y) \vert $. 
        
        Define the function $ g(t) = G(tu) $ for a fixed $ u $, where $ u \in \mathbb{R}^2 $ is a unit vector such that $ (x,y) = ru $. Then $ g $ is differentiable, and $ g(0) = g'(0) = g''(0) = 0 $, since all the partial derivatives of order $ \leq 2 $ of $ F $ vanish at $ p $. By Taylor's formula with remainder, we have 
        \[
            g(t) = g(0) + g^{\prime} (0) + \int_0^t \mathrm{d}s\, (t-s) g^{\prime\prime} (s) = \int_0^t \mathrm{d}s\, (t-s) g^{\prime\prime} (s)
        \]
        for all $ t $ in a neighborhood of $ 0 $. Next we will bound $ \vert g \vert $. Since $ F $ is smooth, $ \nabla^2 F $ is continuous, so for all $ \varepsilon > 0 $ there exists $ \delta > 0 $, such that $ \Vert (x,y) \Vert < \delta $ implies $ \Vert \nabla^2 F (x,y) \Vert < 2 \varepsilon $. Hence, for $ t < \delta $, $ \vert g^{\prime\prime} (t) \vert = \vert u^T \nabla^2 F u \vert \leq \vert \nabla^2 F \vert \lVert u^2 \rVert < 2\varepsilon $. Take $ t=r<\delta $, then we have 
        \begin{align*}
            \vert G(ru) \vert &= \vert g(r) \vert = \left\vert \int_0^r \mathrm{d}s\, (r-s) g^{\prime\prime} (s) \right\vert \leq \int_0^r \mathrm{d}s\, (r-s) \vert g^{\prime\prime} (s) \vert \\
            &\leq \int_0^r \mathrm{d}s\, (r-s) 2 \varepsilon r^2 = \varepsilon r^2.
        \end{align*}
        Notice that $ d = G(x,y) = G(ru) $, so for all $ \varepsilon>0 $ there exists $ \delta > 0 $ such that $ \frac{d}{r^2} < \varepsilon $ whenever $ \sqrt{x^2 + y^2} < \delta $. This proves the desired result.
    \end{enumerate}
\end{solutionManual}

% 3.3.13
\begin{exerciseManual}{3.3.13}
    Let $F\colon \mathbb{R}^3 \to \mathbb{R}^3$ be the map (a similarity) defined by $F(p)=c p$, $p \in \mathbb{R}^3$, $c$ a positive constant. Let $S \subset \mathbb{R}^3$ be a regular surface and set $\bar{S} = F(S)$. Show that $\bar{S}$ is a regular surface, and find formulas relating the Gaussian and mean curvatures, $K$ and $H$, of $S$ with the Gaussian and mean curvatures, $\bar{K}$ and $\bar{H}$, of $\bar{S}$.
\end{exerciseManual}

\begin{solutionManual}{3.3.13}
    ~ 

    \begin{enumerate}
        \item Let $ \bvec{x}: U \subseteq \mathbb{R} \to S $ be a local parametrization of $ S $. Let $ \overline{S} = F(S) $, then $ \overline{\bvec{x}} = F \circ \bvec{x}: U \to \overline{S} $ is a local parametrization of $ \overline{S} $. The map $ F $ is smooth, and since $ \mathrm{d}F = c \operatorname{Id} $ is an isomorphism, $ \mathrm{d}\overline{\bvec{x}} = \mathrm{d}F \circ \mathrm{d}\bvec{x} = c \mathrm{d}\bvec{x} $ has rank $ 2 $ whenever $ \mathrm{d}\bvec{x} $ has rank $ 2 $. Thus, $ \overline{\bvec{x}} $ is a homeomorphism onto its image and $ \mathrm{d}\overline{x} $ is injective (hence an immersion). Therefore, $ \overline{S} $ is a regular surface.
        \item For any local parametrization $ \bvec{x} $ and $ \overline{\bvec{x}} $, we have $ \overline{\bvec{x}} = c \bvec{x} $. Thus, 
        \[
            \overline{\bvec{x}}_u = c \bvec{x}_u, \quad \overline{\bvec{x}}_v = c \bvec{x}_v, \quad \overline{\bvec{x}} \wedge \overline{\bvec{x}}_v = c^2 (\bvec{x}_u \wedge \bvec{x}_v).
        \]
        Hence, the normal for $ \overline{S} $ satisfies $ \overline{N} = N $. Write the Weingarten map for $ S $ and $ \overline{S} $ as $ \mathcal{S} $ and $ \overline{\mathcal{S}} $, respectively. By definition, $ \mathrm{d}N = - \mathcal{S} \circ \mathrm{d}\bvec{x} $, so 
        \[
            \mathrm{d}\overline{N} = \mathrm{d}N = - \mathcal{S} \circ \mathrm{d}\bvec{x} = - \mathcal{S} \circ \frac{1}{c}\, \mathrm{d}\overline{\bvec{x}} = - \left(\frac{1}{c} \mathcal{S}\right) \circ \mathrm{d}\overline{\bvec{x}}.
        \]
        Therefore, $ \overline{\mathcal{S}} = \frac{1}{c} \mathcal{S} $, and the principle curvatures satisfy $ \overline{k}_i = \frac{1}{c} k_i $, since they are the eigenvalues of $ \mathcal{S} $. The Gaussian curvature $ K $ and mean curvature $ H $ of $ S $ are then given by
        \begin{align*}
            \overline{K} = \overline{k}_1 \overline{k}_2 = \frac{1}{c^2} k_1 k_2 = \frac{1}{c^2} K, \\
            \overline{H} = \frac{\overline{k}_1 + \overline{k}_2}{2} = \frac{1}{c} \frac{k_1 + k_2}{2} = \frac{1}{c} H.
        \end{align*}
    \end{enumerate}
\end{solutionManual}

% 3.4.15
\begin{exerciseManual}{*3.3.15}
    Give an example of a surface which has an isolated parabolic point $p$ (that is, no other parabolic point is contained in some neighborhood of $p$).
\end{exerciseManual}

\begin{solutionManual}{3.3.15}
    We know that the graph of $ x^4 $ has an isolated point of zero curvature, so it can be used as one direction. Then add something to bend it in the other direction away from the origin, such that the Hessian is not changed, i.e. we add a quartic term. We can construct an example that looks like $ x^4 $ in one direction and $ y^2 $ in the other at the origin: let $ \bvec{x}(u,v) = (u,\,v,\,u^4 + u^2 v^2 + v^2) $. 
    
    \begin{claim}
        The image $ S \subseteq \mathbb{R}^3 $ of $ \bvec{x} $ has a parabolic point at $ (0,\,0,\,0) $, and all the other points are elliptic. We compute the second fundamental form as follows:  
        \[
          \bvec{x}_u = \left(1,\,0,\,4u^3 + 2u v^2\right), \quad \bvec{x}_v = \left(0,\,1,\,2u^2 v + 2v \right), 
        \]
        \[
            \bvec{x}_{uu} = \left(0,\,0,\,12 u^2 + 2v^2\right), \quad \bvec{x}_{uv} = \left(0,\,0,\,4uv\right), \quad \bvec{x}_{vv} = \left(0,\,0,\,2u^2 + 2\right).,
        \]
        The unit normal is given by 
        \[
            N = \frac{\bvec{x}_u \wedge \bvec{x}_v}{\vert \bvec{x}_u \wedge \bvec{x}_v \vert} = \frac{\left(-4 u^3 - 2uv^2,\, - 2u^2 v - 2v, \, 1\right)}{\sqrt{16 u^6 + 20 u^4 v^2 + 12 u^2 v^4 + 4v^2 + 1}}.
        \]
        Let $ A = \left(16 u^6 + 20 u^4 v^2 + 12 u^2 v^4 + 4v^2 + 1\right)^{-1/2} > 0 $, we have  
        \begin{align*}
            e &= A (12u^2 + 2v^2), \quad 
            f = A(4uv), \quad 
            g = A(2u^2 + 2).
        \end{align*}
        Hence, given a tangent direction $ w(u,v) = a(u,v) \bvec{x}_u + b(u,v) \bvec{x}_v \in T_p(S) $, we have 
        \begin{align*}
            \operatorname{II} (w, w) &= e\,a(u,v)^2 + 2f\,a(u,v)\,b(u,v) + g\, b(u,v)^2 \\
            &= 2A \left(6u^2 a^2 + (v a + u b)^2 + (1 - 3 u^2) b^2\right) > 0
        \end{align*}
        if and only if $ u^2 < 1/3 $. Unless $ b=0 $, $ \operatorname{II} (w,w) > 0 $ whenever $ \vert u \vert < 1/\sqrt{3} $, so near the origin there is only one direction in which $ k_n  = \operatorname{II}(w) / \operatorname{I} (w) = 0 $. Take this direction, assume $ a \neq 0 $, then a point $ p $ is parabolic if and only if $ k_n (p) = 0 $, if and only if $ \operatorname{II}_p (w) = 2A a^2 (6 u^2 + v^2) = 0 $, if and only if $ u = v = 0 $. Hence, $ (0,0,0) $ is an isolated parabolic point. 
    \end{claim}
\end{solutionManual}

% 3.4.16
\begin{exerciseManual}{*3.3.16}
    Show that a surface which is compact (i.e., it is bounded and closed in $\mathbb{R}^3$) has an elliptic point.
\end{exerciseManual}

\begin{solutionManual}{3.3.16}
    Recall that an elliptic point is some point $ p $ where $ \det \left( \mathrm{d}N_p \right) < 0 $.
\end{solutionManual}

% 3.4.17
\begin{exerciseManual}{3.3.17}
    Define Gaussian curvature for a nonorientable surface. Can you define mean curvature for a nonorientable surface?
\end{exerciseManual}

\begin{solutionManual}{3.3.17}
  
\end{solutionManual}

% 3.4.18
\begin{exerciseManual}{3.4.18}
    Show that the M\"{o}bius strip of Fig.\ 3--1 can be parametrized by
    \[
        \mathbf{x}(u,v)
        =
        \bigl(
            \,(2 - v \sin \tfrac{u}{2}) \sin u,\;
            (2 - v \sin \tfrac{u}{2}) \cos u,\;
            v \cos \tfrac{u}{2}
        \bigr),
    \]
    and that its Gaussian curvature is
    \[
        K \;=\; -\frac{1}{\left\{\tfrac14 v^{2} + \bigl(2 - v\sin\!\tfrac{u}{2}\bigr)^{2}\right\}^{2}}.
    \]
\end{exerciseManual}

\begin{solutionManual}{3.4.18}

    \begin{figure}[htbp]
        \begin{minipage}{0.5\textwidth}
            \centering
            \includegraphics[width=0.95\textwidth]{2-31.png}
            % \caption*{(a) A M\"{o}bius strip.}
        \end{minipage}%
        \begin{minipage}{0.5\textwidth}
            \centering
            \includegraphics[width=0.95\textwidth]{3-1.png}
            % \caption*{(b) A M\"{o}bius strip with a twist.}
        \end{minipage}
        % \caption{M\"{o}bius strip.}
    \end{figure}

    The M\"{o}bius strip is constructed by twisting the cyclinder segment 
    \[
        \overline{\bvec{x}}(u,v) = \left(2 \sin u, \, 2 \cos u, \, v \right), \quad u \in [0, 2\pi], v \in [-1,1], 
    \]
    by an angle half of the turning angle $ u $. Therefore, we have 
    \[
        \bvec{x}(u,v) = \left((2- v \sin \tfrac{u}{2}) \sin u,\, (2- v \sin \tfrac{u}{2}) \cos u,\, v \cos \tfrac{u}{2}\right). 
    \]
    To compute the Gaussian curvature, let's use , and compute the following: 
    \begin{align*}
        \bvec{x}_u &= \left( (2 - v \sin \tfrac{u}{2} )\cos u - \tfrac{v}{2} \cos \tfrac{u}{2} \sin u , \, - (2 - \sin \tfrac{u}{2}) \sin u - \tfrac{v}{2} \cos \tfrac{u}{2} \cos u , \, - \tfrac{v}{2} \sin \tfrac{u}{2}\right), \\
        \bvec{x}_v &= \left( - \sin \tfrac{u}{2} \sin u,\, \sin \tfrac{u}{2} \sin u,\, \cos \tfrac{u}{2} \right). 
    \end{align*}
    Then, we have 
    \[
        E = \left(2 - v \sin \frac{u}{2}\right)^2 + \left(\frac{v}{2}\right)^2, \quad F = 0, \quad G = 1.
    \]
    Therefore, $ \bvec{x} $ is an orthogonal parametrization. Next, notice that $ \bvec{x}_{vv} = 0 $, and hence $ g = 0 $. Since $ N = (\bvec{x}_u \wedge \bvec{x}_v) / \|\bvec{x}_u \wedge \bvec{x}_v\| $, we have 
    \[
        f = \langle \bvec{x}_{uv}, \, N \rangle = \frac{\det (\bvec{x}_u, \, \bvec{x}_v, \, \bvec{x}_{uv})}{\vert \bvec{x}_u \wedge \bvec{x}_v \vert}. 
    \] 
    Now, let's compute the determinant: we have 
    \[
        \bvec{x}_{uv} = \left( - \tfrac{1}{2} \cos \tfrac{u}{2} \sin u - \sin \tfrac{u}{2} \cos u,\, - \tfrac{1}{2} \cos \tfrac{u}{2} \cos u + \sin \tfrac{u}{2} \sin u,\, - \tfrac{1}{2} \sin \tfrac{u}{2} \right).
    \]
    Thus, we can compute the determinant directly, by expanding along the third column:
    \begin{align*}
        \det (\bvec{x}_u, \, \bvec{x}_v, \, \bvec{x}_{uv}) 
        &= 
        \begin{vmatrix}
            (2 - v \sin \tfrac{u}{2} )\cos u - \tfrac{v}{2} \cos \tfrac{u}{2} \sin u & - \sin \tfrac{u}{2} \sin u & - \tfrac{1}{2} \cos \tfrac{u}{2} \sin u - \sin \tfrac{u}{2} \cos u \\
            - (2 - v \sin \tfrac{u}{2}) \sin u - \tfrac{v}{2} \cos \tfrac{u}{2} \cos u & \sin \tfrac{u}{2} \cos u & - \tfrac{1}{2} \cos \tfrac{u}{2} \cos u + \sin \tfrac{u}{2} \sin u \\
            - \tfrac{v}{2} \sin \tfrac{u}{2} & \cos \tfrac{u}{2} & - \tfrac{1}{2} \sin \tfrac{u}{2}
        \end{vmatrix} \\ 
        &= - \frac{1}{2} \left( (2 - v \sin \tfrac{u}{2})^2 + \left(\tfrac{v}{2}\right)^2 \right) = -\frac{E}{2}.
    \end{align*}
    Moreover, we have $ \vert \bvec{x}_u \wedge \bvec{x}_v \vert = \sqrt{EG - F^2} = \sqrt{E} $, and $ f = - \frac{E}{2} / \sqrt{E} = - \frac{\sqrt{E}}{2} $. Finally, the Gaussian curvature is given by
    \[
        K = \frac{eg - f^2}{EG - F^2} = - \frac{f^2}{EG} = - \frac{1}{4E} = -\frac{1}{\left\{\tfrac14 v^{2} + \bigl(2 - v\sin\!\tfrac{u}{2}\bigr)^{2}\right\}^{2}}.
    \]  
\end{solutionManual}

% 3.3.19
\begin{exerciseManual}{*3.3.19}
    Obtain the asymptotic curves of the one-sheeted hyperboloid
    \[
        x^{2} + y^{2} - z^{2} = 1.
    \]
\end{exerciseManual}

\begin{solutionManual}{3.3.19}
    We first rederive a few important identities. Let $ \bvec{x}(u,v) = (u,\,v,\,f(u,v)) $ be a parametrization of the surface. Then, we have 
    \[
        \bvec{x}_u = (1,\,0,\,f_u), \quad \bvec{x}_v = (0,\,1,\,f_v) \implies N = \frac{(-f_u,\, -f_v,\, 1)}{\sqrt{1 + f_u^2 + f_v^2}}. 
    \]
    Next, we have $ \bvec{x}_{uu} = (0,\,0,\,f_{uu}) $, $ \bvec{x}_{uv} = (0,\,0,\,f_{uv}) $, $ \bvec{x}_{vv} = (0,\,0,\,f_{vv}) $, and  
    \[
        e = \frac{f_{uu}}{\sqrt{1 + f_u^2 + f_v^2}}, \quad f = \frac{f_{uv}}{\sqrt{1 + f_u^2 + f_v^2}}, \quad g = \frac{f_{vv}}{\sqrt{1 + f_u^2 + f_v^2}}.
    \]
    By the Gauss formula, the Gaussian curvature is given by
    \[
        K = \frac{eg - f^2}{EG - F^2} = \frac{f_{uu} f_{vv} - f_{uv}^2}{(1 + f_u^2 + f_v^2)^2}.
    \]
    For the one-sheeted hyperboloid, we have $ z = f(x,y) = \sqrt{x^2 + y^2 - 1} $ on the upper sheet where $ z > 0 $. Compute 
    \[
        f_x = \frac{x}{\sqrt{x^2 + y^2 - 1}}, \quad f_y = \frac{y}{\sqrt{x^2 + y^2 - 1}},
    \]
    \[
        f_{xx} = \frac{y^2 - 1}{(x^2 + y^2 - 1)^{3/2}}, \quad f_{yy} = \frac{x^2 - 1}{(x^2 + y^2 - 1)^{3/2}}, \quad f_{xy} = -\frac{xy}{(x^2 + y^2 - 1)^{3/2}}.
    \]
    Explicitly, the coefficients $ e $, $ f $, $ g $ are given by
    \[
        e = \frac{y^2 - 1}{(x^2 + y^2) \sqrt{x^2 + y^2 - 1}}, \quad f = -\frac{xy}{(x^2 + y^2) \sqrt{x^2 + y^2 - 1}}, \quad g = \frac{x^2 - 1}{(x^2 + y^2) \sqrt{x^2 + y^2 - 1}}.
    \]
    
    One parametrizion of the asymptotic curves satisfy the equation
    \[
        \begin{dcases}
            x &= t \cos \theta - \sin \theta, \\
            y &= t \sin \theta + \cos \theta, \\
            z &= t.
        \end{dcases}
    \]
    \begin{remark}
        We can check that the Gaussian curvature is negative:
        \[
            K = \frac{f_{xx} f_{yy} - f_{xy}^2}{(1 + f_x^2 + f_y^2)^2} = - \frac{1}{(x^2 + y^2 - 1)^2} < 0, 
        \]
        as expected for a hyperbolic surface.
    \end{remark} 
\end{solutionManual}

% 3.3.21
\begin{exerciseManual}{*3.3.21}
    Let $S$ be a surface with orientation $N$. Let $V \subset S$ be an open set in $S$ and let
    $f : V \subset S \to \mathbb{R}$ be any nowhere-zero differentiable function in $V$.
    Let $v_{1}$ and $v_{2}$ be two differentiable (tangent) vector fields in $V$ such that
    at each point of $V$, $v_{1}$ and $v_{2}$ are orthonormal and $v_{1} \wedge v_{2} = N$.
    \begin{enumerate}[label=\textbf{\alph*.}]
        \item Prove that the Gaussian curvature $K$ of $V$ is given by
        \[
            K
            \;=\;
            \frac{\,
                \langle d(fN)(v_{1}) \wedge d(fN)(v_{2}),\, fN\rangle
            \,}{f^{3}}.
        \]
        The virtue of this formula is that by a clever choice of $f$ we can often simplify
        the computation of $K$, as illustrated in part (b).

        \item Apply the above result to show that if $f$ is the restriction of
        \[
            \sqrt{ \frac{x^{2}}{a^{4}} + \frac{y^{2}}{b^{4}} + \frac{z^{2}}{c^{4}} }
        \]
        to the ellipsoid
        \[
            \frac{x^{2}}{a^{2}} + \frac{y^{2}}{b^{2}} + \frac{z^{2}}{c^{2}} = 1,
        \]
        then the Gaussian curvature of the ellipsoid is
        \[
            K = \frac{1}{a^{2}b^{2}c^{2}} \, \frac{1}{f^{4}}.
        \]
    \end{enumerate}
\end{exerciseManual}

\begin{solutionManual}{3.3.21}
    ~ 

    \begin{enumerate}[label=\textbf{\alph*.}]
        \item Since $ v_i $, $ i=1,2 $ are tangent vector fields, we have $ \mathrm{d}(fN)_p (v_i) = v_i (f) N_p + f \mathrm{d}N_p (v_i) $. Then, 
        \[
            \begin{split}
                \mathrm{d}(fN)(v_1) \wedge \mathrm{d}(fN)(v_2) &= \left(v_1 (f) N + f \mathrm{d}N (v_1)\right) \wedge \left( v_2 (f) N + f \mathrm{d}N (v_2)\right) \\
                &= f \left(v_1 (f) \wedge \mathrm{d}N (v_2) - v_2(f) \wedge \mathrm{d}N (v_1)\right) + f^2 \left(\mathrm{d}N (v_1) \wedge \mathrm{d}N (v_2)\right).
            \end{split}
        \]
        Taking the inner product with $ fN $, we have
        \[
            \langle \mathrm{d}(fN)(v_1) \wedge \mathrm{d}(fN)(v_2),\, fN \rangle = f^3 \langle \mathrm{d} N(v_1) \wedge \mathrm{d}N (v_2),\, N \rangle,        
        \]
        by linearity of the determinant. Hence, $ \langle \mathrm{d}(fN)(v_1) \wedge \mathrm{d}(fN)(v_2),\, fN \rangle / f^3 = \langle \mathrm{d}N (v_1) \wedge \mathrm{d} N(v_2),\, N \rangle $ is independent of $ f $. In the basis $ \{v_1, v_2\} $, we may write 
        \[
            \mathrm{d}N_1 = a_{11} v_1 + a_{21} v_2, \quad \mathrm{d}N_2 = a_{12} v_1 + a_{22} v_2,
        \]
        taking the wedge product gives $ \mathrm{d} N (v_1) \wedge \mathrm{d} N (v_2) = \left(a_{11} a_{22} - a_{12} a_{21} \right) \left(v_1 \wedge v_2\right) $, and $ \langle \mathrm{d} N(v_1) \wedge \mathrm{d} N (v_2), N \rangle = \det \left(a_{ij}\right) $ since $ \{v_1, v_2, N\} $ is a positively oriented orthonormal frame. The shape operator $ \mathcal{S} $ satisfies $ \mathcal{S} (v_i) = - \mathrm{d}N (v_i) $, and hence $ S_p = - \mathrm{d}N_p $. The Gaussian curvature is 
        \[
            \begin{split}
                K &= \det \mathcal{S} = \det (-\mathrm{d}N) = \det (\mathrm{d}N) = \det \left(a_{ij}\right) \\
                &= \langle \mathrm{d}N (v_1) \wedge \mathrm{d}N (v_2),\, N \rangle = \frac{\langle \mathrm{d}(fN)(v_1) \wedge \mathrm{d}(fN)(v_2),\, fN \rangle}{f^3}.
            \end{split}
        \]
        \item Given the implicit equation $ F(x,y,z) = \frac{x^2}{a^2} + \frac{y^2}{b^2} + \frac{z^2}{c^2} = 1 $, let $ A = \operatorname{diag} \left(\frac{1}{a^2}, \frac{1}{b^2}, \frac{1}{c^2}\right) $ and $ p = (x,y,z) $. Then, $ F(p) = \langle Ap, p \rangle = 1 $ and $ \nabla F = 2Ap $. The unit normal is given by
        \[
            N = \frac{\nabla F}{\Vert \nabla F \Vert} = \frac{Ap}{\left\vert Ap \right\vert} = \frac{Ap}{f(p)},
        \]
        since $ f(p) = \sqrt{\langle Ap, p \rangle} = \Vert Ap \Vert $. Therefore, $ f(p) N(p) = (fN)(p) = Ap $ is a linear map. Hence, for any $ v \in T_p (S) $, we have $ \mathrm{d}(fN)_p (v) = A v $. Let $ v_1, v_2 $ be an orthonormal basis of $ T_p (S) $, such that $ \{v_1, v_2, N\} $ is a positively oriented orthonormal frame. Then, 
        \[
            \begin{split}
                \langle \mathrm{d}(fN)(v_1) \wedge \mathrm{d}(fN)(v_2),\, fN \rangle &= \langle A v_1 \wedge A v_2,\, Ap \rangle = \det (A) \langle v_1 \wedge v_2,\, p \rangle \\
                &= \det (A) \langle N,\, p \rangle = \det (A) \frac{\langle Ap, p \rangle}{f(p)} = \det (A) \frac{1}{f(p)}.
            \end{split}
        \]
        \[
            \implies K = \frac{\langle \mathrm{d}(fN)(v_1) \wedge \mathrm{d}(fN)(v_2),\, fN \rangle}{f^3} = \frac{\det (A)}{f^4} = \frac{1}{a^2 b^2 c^2} \frac{1}{f^4}.
        \]
        The explicit formula for $ K $ is then 
        \[
            K = \frac{1}{a^2 b^2 c^2} \left(\frac{x^2}{a^4} + \frac{y^2}{b^4} + \frac{z^2}{c^4} \right)^{-2}.
        \]
    \end{enumerate}
    
\end{solutionManual}

% 3.3.24
\begin{exerciseManual}{3.3.24}\textit{(Local Convexity and Curvature)}
    ~

    A surface $S \subset \mathbb{R}^3$ is \emph{locally convex} at a point $p \in S$ if there exists a neighborhood $V \subset S$ of $p$ such that $V$ is contained in one of the closed half-spaces determined by $T_p(S)$ in $\mathbb{R}^3$. If, in addition, $V$ has only one common point with $T_p(S)$, then $S$ is called \emph{strictly locally convex} at $p$.
    \begin{enumerate}
        \item[\textbf{a.}] Prove that $S$ is strictly locally convex at $p$ if the principal curvatures of $S$ at $p$ are nonzero with the same sign (that is, the Gaussian curvature $K(p)$ satisfies $K(p) > 0$).
        
        \item[\textbf{b.}] Prove that if $S$ is locally convex at $p$, then the principal curvatures at $p$ do not have different signs (thus, $K(p) \ge 0$).
        
        \item[\textbf{c.}] To show that $K \ge 0$ does not imply local convexity, consider the surface
        \[
        f(x,y) = x^3(1 + y^2),
        \]
        defined in the open set $U = \{ (x,y) \in \mathbb{R}^2 : y^2 < \tfrac{1}{2} \}$. Show that the Gaussian curvature of this surface is nonnegative on $U$ and yet the surface is not locally convex at $(0,0) \in U$ (a deep theorem, due to R.~Sacksteder, implies that such an example cannot be extended to the entire $\mathbb{R}^2$ if we insist on keeping the curvature nonnegative; cf.\ Remark 3 of Sec.~5-6).

        \item[\textbf{*d.}] The example of part (c) is also very special in the following local sense. Let $p$ be a point in a surface $S$, and assume that there exists a neighborhood $V \subset S$ of $p$ such that the principal curvatures on $V$ do not have different signs (this does not happen in the example of part c). Prove that $S$ is locally convex at $p$.
    \end{enumerate}
\end{exerciseManual}

\begin{solutionManual}{3.3.24}
    ~
    
    \begin{enumerate}[label=\textbf{\alph*.}]
        \item Without loss of generality, assume $ k_1, k_2 > 0 $, since if both are negative, just replace the chosen unit normal by its negative. Let $ \bvec{x}: U \subseteq \mathbb{R}^2 \to S \subseteq \mathbb{R}^3 $ be a local parametrization of $ S $ such that $ \{\bvec{x}_u, \bvec{x}_v\} $ is an \emph{orthonormal basis of principle directions at $ p \in S $}, where $ p = \bvec{x}(0,0) $. Following the definition of Exercise 3.3.22, define the \emph{height function} $ h: U \to \mathbb{R} $ of $ S $ relative to $ T_p(S) $ by
        \[
            h(u,v) = \langle \bvec{x}(u,v) - p, N(p) \rangle,
        \]
        where $ N(p) $ is the unit normal vector $ p $. We compute the derivatives as follows: 
        \begin{align*}
            h(p) &= \langle \bvec{x}(0,0) - p, N(p) \rangle = 0, \\
            h_u(p) &= \langle \bvec{x}_u(0,0), N(p) \rangle = 0, \\
            h_v(p) &= \langle \bvec{x}_v(0,0), N(p) \rangle = 0, \\
            h_{uu}(p) &= \langle \bvec{x}_{uu}(0,0), N(p) \rangle = e(p), \\
            h_{uv}(p) &= \langle \bvec{x}_{uv}(0,0), N(p) \rangle = f(p), \\
            h_{vv}(p) &= \langle \bvec{x}_{vv}(0,0), N(p) \rangle = g(p),
        \end{align*}
        where $ h_{ij}(p) $ are the coefficients of the second fundamental form at $ p $. Since $ \bvec{x}_u (0,0) $ and $ \bvec{x}_v (0,0) $ are principle directions and orthonormal, we have $ e(p) = k_1 $, $ f(p) = 0 $, and $ g(p) = k_2 $. Thus, the Hessian matrix of $ h $ at $ p $ is given by
        \[
            \nabla^2 h (p) = \begin{pmatrix}
                h_{uu}(p) & h_{uv}(p) \\
                h_{uv}(p) & h_{vv}(p)
            \end{pmatrix} = \begin{pmatrix}
                k_1 & 0 \\
                0 & k_2
            \end{pmatrix},
        \]
        and Taylor expansion gives 
        \[
            h(u,v) = \frac{1}{2} \left(k_1 u^2 + k_2 v^2 \right) + o\left(u^2 + v^2\right),
        \]
        Since $ k_1, k_2 > 0 $, the quadratic form $ Q = \frac{1}{2}\left( k_1 u^2 + k_2 v^2 \right) $ associated with $ \nabla^2 h (p) $ is positive definite. Hence, there exists a neighborhood $ W \subset U $ of $ p $ and some $ c > 0 $ such that $ Q(u,v) > c (u^2 + v^2) $ for all $ (u,v) \in W $. Now since 
        \[
            \frac{h(u,v) - Q(u,v)}{u^2 + v^2} \to 0 \quad \text{as } (u,v) \to (0,0),
        \]
        there exists a radius $ \delta>0 $ such that $ \sqrt{u^2 + v^2} < \delta $ implies $ \vert h(u,v) - Q(u,v) \vert < \frac{c}{2} (u^2 + v^2) $. Therefore, for all $ (u,v) \in W $ with $ \sqrt{u^2 + v^2} < \delta $, we have
        \[
            h(u,v) \geq Q(u,v) - \vert h(u,v) - Q(u,v) \vert > c (u^2 + v^2) - \frac{c}{2} (u^2 + v^2) = \frac{c}{2} (u^2 + v^2) > 0, 
        \]
        with $ h(u,v) = 0 $ if and only if $ (u,v) = (0,0) $. Thus, the neighborhood $ V = \bvec{x} (W \cap \{(u,v) : \sqrt{u^2 + v^2} < \delta \}) $ of $ p $ is contained in the half-space $ H^+ = \{ q \in \mathbb{R}^3 \mid \langle q - p, N(p) \rangle \geq 0 \} $, and $ V $ has only one common point with $ T_p(S) $. Therefore, $ S $ is strictly locally convex at $ p $. 

        \item Suppose $ S $ is locally convex at $ p $, so there exists a neighborhood $ V \subset S $ of $ p $ such that $ V $ is contained in one of the closed half-spaces determined by $ T_p(S) $. Define the height function as above, by local convexity we may choose an orientation $ N(p) $ such that $ h(u,v) \geq 0 $ in a neighborhood of $ (0,0) $, and $ h(0,0) = h_u (0,0) = h_v (0,0) = 0 $. Suppose that the principal curvatures at $ p $ have different signs, say $ k_1 > 0 > k_2 $. Then, along the coordinate axes, we have $ h(u,0) = \frac{1}{2}k_1 u^2 > 0 $ for all $ \vert u \vert < \delta_u $, and $ h(0,v) = \frac{1}{2}k_2 v^2 < 0 $ for all $ \vert v \vert < \delta_v $. Hence, in every neighborhood of $ (0,0) $, we can find points such that $ h(u,v) > 0 $ and others such that $ h(u,v) < 0 $, contradicting local convexity. Therefore, the principal curvatures at $ p $ do not have different signs, and hence $ K(p) \geq 0 $.
        
        \item The Gaussian curvature $ K $ of the surface defined by $ z = f(x,y) $ is given by
        \[
            K = \frac{f_{xx} f_{yy} - f_{xy}^2}{(1 + f_x^2 + f_y^2)^2}.
        \]
        Let's compute the necessary partial derivatives of $ f(x,y) = x^3(1 + y^2) $:
        \[
            f_x = 3x^2(1 + y^2), \quad f_y = 2x^3 y, \quad 
            f_{xx} = 6x(1 + y^2), \quad f_{yy} = 2x^3, \quad f_{xy} = 6x^2 y.
        \]
        Then, we have 
        \[
            K = \frac{(6x(1 + y^2))(2x^3) - (6x^2 y)^2}{(1 + (3x^2(1 + y^2))^2 + (2x^3 y)^2)^2} = \frac{12x^4(1 - 2y^2)}{(1 + 9x^4(1 + y^2)^2 + 4x^6 y^2)^2} \geq 0.
        \]
        However, the surface is not locally convex at $ (0,0) $, since for any neighborhood $ V $ of $ (0,0) $, there exist points with both positive and negative $ x $ values, and hence $ z $-coordinates, so $ V $ is not contained in one of the closed half-spaces determined by the tangent plane at $ (0,0) $. 

        \item Suppose $ V \subseteq S $ is a neighborhood of $ p $ such that the principal curvatures on $ V $ do not have different signs. Without loss of generality, assume $ k_1 (q), k_2 (q) \geq 0 $ for all $ q \in V $, since if at some point one of them were positive and later negative, it would have to cross zero alone, producing a point where the two have different signs, which is excluded by definition of $ V $. Follow the steps of \textbf{a.}, we define the height function $ h: U \to \mathbb{R} $ of $ S $ relative to $ T_p(S) $ by $ h(u,v) = \langle \bvec{x}(u,v) - p, N(p) \rangle $. Pick an orthonormal basis of principal directions $ \{\bvec{x}_u, \bvec{x}_v\} $. The Hessian matrix of $ h $ at $ p $ is given, again, by
        \[
            \nabla^2 h (p) = \begin{pmatrix}
                k_1 & 0 \\
                0 & k_2
            \end{pmatrix}.
        \]
        Near $ (0,0) $, we have 
        \[
            h(u,v) = \frac{1}{2} \left( k_1 u^2 + k_2 v^2 \right) + o\left(u^2 + v^2\right),
        \] 
        and the quadratic form $ Q = \frac{1}{2} \left(k_1 u^2 + k_2 v^2\right) $ is positive-definite. Now we consider two cases: 
        \begin{enumerate}
            \item At least one of the principal curvatures at $ p $ is positive, say $ k_1 > 0 $. Then, there exists a neighborhood $ W \subset U $ of $ p $ and some $ c > 0 $ such that $ Q(u,v) > c (u^2 + v^2) $ for all $ (u,v) \in W $. Following the same steps as in \textbf{a.}, we can show local convexity at $ p $.
            \item Both principal curvatures at $ p $ are zero, i.e., $ k_1 = k_2 = 0 $, so $ Q = 0 $. Since the principal curvatures are continuous functions on $ S $, we have $ h(0,0) = 0 $ and $ h(u,v) \geq 0 $ in a neighborhood of $ p $. Therefore, $ S $ is locally convex at $ p $.
        \end{enumerate}
    \end{enumerate}
\end{solutionManual}

\newpage 

\section{Chapter 3.4}

% 3.4.2
\begin{exerciseManual}{3.4.2}
    Prove that the vector field obtained on the torus by parametrizing all its meridians by arc length and taking their tangent vectors (Example 1) is differentiable.
\end{exerciseManual}

\begin{solutionManual}{3.4.2}
    From Do Carmo 3.4 Definition 1, a vector field $ w $ is \emph{differentiable} if, for some parametrization $ \bvec{x}: U \to \mathbb{R}^3 $, the functions $ a(u,v) $ and $ b(u,v) $ given by $ w = a(u,v) \bvec{x}_u + b(u,v) \bvec{x}_v $ are differentiable on $ U $. Parametrize the torus by
    \[
        \bvec{x}(u,v) = \left( (R + r \cos v) \cos u, (R + r \cos v) \sin u, r \sin v \right),
    \]
    where $ R $ is the distance from the center of the tube to the center of the torus, and $ r $ is the radius of the tube. Fix $ \theta = \theta_0 $ and vary $ \phi = \frac{s}{r} $, we have  
    \[
        \alpha_{\theta_0} (s) = \bvec{x}(\theta_0, s/r) = \left( (R + r \cos s/r) \cos \theta_0, (R + r \cos s/r) \sin \theta_0, r \sin s/r \right).
    \]
    Then the vector field obtained by parametrizing the meridians by arc length is given by
    \[
        w(\bvec{x}(\theta_0, s/r)) = \alpha_{\theta_0}'(s) = \left( -\sin s/r \cos \theta_0, -\sin s/r \sin \theta_0, \cos s/r \right).
    \]
    Let $ w(\bvec{x}(\theta, \phi)) = a(\theta, \phi) \bvec{x}_\theta + b(\theta, \phi) \bvec{x}_\phi $, we have 
    \[
        \bvec{x}_\theta = \left( -(R + r \cos \phi) \sin \theta, (R + r \cos \phi) \cos \theta, 0 \right),
    \]
    \[
        \bvec{x}_\phi = \left( -r \sin \phi \cos \theta, -r \sin \phi \sin \theta, r \cos \phi \right).
    \]
    Comparing the coefficients, we get $ a(\theta, \phi) = 0 $, $ b(\theta, \phi) = \frac{1}{r} $. Since they are both differentiable, $ w $ is differentiable.  
\end{solutionManual}

% 3.4.3
\begin{exerciseManual}{3.4.3}
    Prove that a vector field $w$ defined on a regular surface $S \subset \mathbb{R}^3$ is differentiable if and only if it is differentiable as a map $w : S \to \mathbb{R}^3$.
\end{exerciseManual}

\begin{solutionManual}{3.4.3}
    Suppose $ w $ is differentiable as a vector field. Then, there exist a parametrization $ \bvec{x}: U \to S $ such that $ w = a(u,v) \bvec{x}_u + b(u,v) \bvec{x}_v $ for differentiable functions $ a(u,v) $ and $ b(u,v) $. Since $ \bvec{x}_u $ and $ \bvec{x}_v $ are differentiable, $ w \circ \bvec{x} = a(u,v) \bvec{x}_u + b(u,v) \bvec{x}_v $ is differentiable. Thus, $ w $ is differentiable as a map. Conversely, suppose $ w $ is differentiable as a map $ w: S \to \mathbb{R}^3 $. Then, for any parametrization $ \bvec{x}: U \to S $ and each $ (u,v) \in U $, since $ \{\bvec{x}_u, \bvec{x}_v\} $  forms a basis for $ T_p(S) $, there exist scalars $ a(u,v) $ and $ b(u,v) $ such that $ \left(w \circ \bvec{x}\right) (u,v) = a(u,v)\bvec{x}_u + b(u,v)\bvec{x}_v $. Then, we have 
    \[
        \langle w, \bvec{x}_u \rangle = a \langle \bvec{x}_u, \bvec{x}_u \rangle + b \langle \bvec{x}_v, \bvec{x}_u \rangle, \quad \langle w, \bvec{x}_v \rangle = a \langle \bvec{x}_u, \bvec{x}_v \rangle + b \langle \bvec{x}_v, \bvec{x}_v \rangle.
    \]
    Let $ \alpha = \langle w, \bvec{x}_u \rangle $, $ \beta = \langle w, \bvec{x}_v \rangle $, then 
    \[
        \begin{pmatrix}
            \alpha \\ \beta
        \end{pmatrix} = 
        \begin{pmatrix}
            E & F \\ F & G
        \end{pmatrix}
        \begin{pmatrix}
            a \\ b
        \end{pmatrix}
    \]
    Since $ \{\bvec{x}_u, \bvec{x}_v\} $ are linearly independent, $ \det \left(\operatorname{I}\right) = EG - F^2 \neq 0 $, and we have
    \[
        a = \frac{G \alpha - F \beta}{EG - F^2}, \quad b = \frac{-F \alpha + E \beta}{EG - F^2}.
    \]
    Since $ w $, $ \bvec{x}_u $ and $ \bvec{x}_v $ are differentiable, $ \alpha $ and $ \beta $ are differentiable. Also, since $ E $, $ F $ and $ G $ are differentiable, $ a(u,v) $ and $ b(u,v) $ are differentiable. Therefore, $ w $ is differentiable as a vector field.
\end{solutionManual}

% 3.4.6
\begin{exerciseManual}{3.4.6}
    A straight line $r$ meets the $z$ axis and moves in such a way that it makes a constant angle $\alpha \neq 0$ with the $z$ axis and each of its points describes a helix of pitch $c \neq 0$ about the $z$ axis.  
    The figure described by $r$ is the trace of the parametrized surface (see Fig.\ 3--32)
    \[
    x(u,v) = (v \sin\alpha \cos u,\; v \sin\alpha \sin u,\; v \cos\alpha + cu).
    \]
    The map $x$ is easily seen to be a regular parametrized surface.  
    Restrict the parameters $(u,v)$ to an open set $U$ so that $x(U)=S$ is a regular surface.

    \begin{enumerate}[label=\textbf{\alph*.}]
        \item Find the orthogonal family (cf.\ Example 3) to the family of coordinate curves $u=\text{const}$.
        \item Use the curves $u=\text{const}$ and their orthogonal family to obtain an orthogonal parametrization for $S$.  
        Show that in the new parameters $(\tilde{u},\tilde{v})$ the coefficients of the first fundamental form are
        \[
        \tilde{G} = 1, \qquad \tilde{F}=0, \qquad 
        \tilde{E} = \{c^{2} + (\tilde{v} - c\tilde{u}\cos\alpha)^{2}\}\sin^{2}\alpha.
        \]
    \end{enumerate}
\end{exerciseManual}

\begin{figure}[h]
    \centering
    \includegraphics[width=0.6\textwidth]{3.32.png}
\end{figure}

\begin{solutionManual}{3.4.6}
    ~

    \begin{enumerate}[label=\textbf{\alph*.}]
        \item The coordinate curves $ u = \text{const} $ have tangent vectors $ \bvec{x}_v $. Let the curve be given by $ v = v(t) $, $ u = u_0 $. Then, its tangent vector is $ \bvec{x}_u u^{\prime} (t) + \bvec{x}_v v^{\prime} (t) $. Orthogonaity gives $ \langle \bvec{x}_u u^{\prime} + \bvec{x}_v v^{\prime} , \bvec{x}_v \rangle = 0 $, and hence $ F u^{\prime} + G v^{\prime} = 0 $. Let's calculate the coefficients of the first fundamental form:
        \[
            \bvec{x}_u = \left( -v \sin \alpha \sin u, v \sin \alpha \cos u, c \right), \quad \bvec{x}_v = \left( \sin \alpha \cos u, \sin \alpha \sin u, \cos \alpha \right).
        \]
        Thus, we have
        \[
            E = \langle \bvec{x}_u, \bvec{x}_u \rangle = v^2 \sin^2 \alpha + c^2, \quad F = \langle \bvec{x}_u, \bvec{x}_v \rangle = c \cos \alpha, \quad G = \langle \bvec{x}_v, \bvec{x}_v \rangle = 1.
        \]
        Treating $ v(t) $ as a function of $ u $, i.e. $ v(t) = v(t(u)) $, we have
        \[
            \frac{\mathrm{d}v}{\mathrm{d}u} = - \frac{F}{G} = -c \cos \alpha \implies v(u) = -c u \cos \alpha + k. 
        \]
        Thus, the orthogonal family to the curves $ u = \text{const} $ is given by $ c u \cos \alpha + v = k $ in the $ (u,v) $-plane.
        \item We have two transverse families of curves in the $ (u,v) $-plane, given by $ u = \text{const.} $ and $ c u \cos \alpha + v = \text{const.} $. Let's define new parameters $ (\tilde{u}, \tilde{v}) $ by
        \[
            \tilde{u} = u, \quad \tilde{v} = c u \cos \alpha + v.
        \]
        The parametrization in the new parameters is given by $ \tilde{\bvec{x}}(\tilde{u}, \tilde{v}) = \bvec{x}(u,v) = \bvec{x}(\tilde{u}, \tilde{v} - c \tilde{u} \cos \alpha) $. Let's calculate the coefficients of the first fundamental form $ \tilde{E} $, $ \tilde{F} $, $ \tilde{G} $ in the new parameters:
        \[
            \tilde{\bvec{x}}_{\tilde{u}} = \bvec{x}_u u_{\tilde{u}} + \bvec{x}_v v_{\tilde{u}} = \bvec{x}_u - c \cos \alpha \bvec{x}_v,
        \]
        \[
            \tilde{\bvec{x}}_{\tilde{v}} = \bvec{x}_u u_{\tilde{v}} + \bvec{x}_v v_{\tilde{v}} = \bvec{x}_v.
        \]
        Substituting in the values of $ E $, $ F $, and $ G $ calculated in part \textbf{a.}, we have  
        \begin{equation*}
            \begin{split}
                \tilde{E} &= \langle \tilde{\bvec{x}}_{\tilde{u}}, \tilde{\bvec{x}}_{\tilde{u}} \rangle = \langle \bvec{x}_u - c \cos \alpha \bvec{x}_v, \bvec{x}_u - c \cos \alpha \bvec{x}_v \rangle \\
                &= E - 2 c \cos \alpha F + c^2 \cos^2 \alpha G, \\
                &= (v^2 \sin^2 \alpha + c^2) - 2 c^2 \cos^2 \alpha + c^2 \cos^2 \alpha = (v^2 + c^2 \sin^2 \alpha) \sin^2 \alpha \\
                &= \{ c^2 + (\tilde{v} - c \tilde{u} \cos \alpha)^2 \} \sin^2 \alpha. \\
                \tilde{F} &= \langle \tilde{\bvec{x}}_{\tilde{u}}, \tilde{\bvec{x}}_{\tilde{v}} \rangle = \langle \bvec{x}_u - c \cos \alpha \bvec{x}_v, \bvec{x}_v \rangle = F - c \cos \alpha G = 0, \\
                \tilde{G} &= \langle \tilde{\bvec{x}}_{\tilde{v}}, \tilde{\bvec{x}}_{\tilde{v}} \rangle = \langle \bvec{x}_v, \bvec{x}_v \rangle = G = 1.
            \end{split}
        \end{equation*}
    \end{enumerate}
\end{solutionManual}

% 3.4.7
\begin{exerciseManual}{3.4.7}
    Define the derivative $w(f)$ of a differentiable function $f : U \subset S \to \mathbb{R}$ relative to a vector field $w$ in $U$ by
    \[
    w(f)(q) = \left.\frac{d}{dt}(f \circ \alpha)\right|_{t=0}, \qquad q \in U,
    \]
    where $\alpha : I \to S$ is a curve such that $\alpha(0)=q$ and $\alpha'(0)=w(q)$.

    Prove that:
    \begin{enumerate}[label=\textbf{\alph*.}]
        \item $w$ is differentiable in $U$ if and only if $w(f)$ is differentiable for all differentiable $f$ in $U$.
        \item Let $\lambda,\mu$ be real numbers and $g : U \subset S \to \mathbb{R}$ be a differentiable function on $U$; then
        \[
        w(\lambda f + \mu f') = \lambda w(f) + \mu w(f'), \qquad
        w(fg) = w(f)g + f w(g).
        \]
    \end{enumerate}
\end{exerciseManual}

\begin{solutionManual}{3.4.7}
    ~

    \begin{enumerate}[label=\textbf{\alph*.}]
        \item Suppose $ w $ is differentiable in $ U $, then it is differentiable as a map $ w: U \to \mathbb{R}^3 $ by Exercise 3.4.3. For any differentiable function $ f: U \to \mathbb{R} $, let $ \bvec{x}: V \to U $ be a local parametrization of $ U $, and $ (u,v) $ a local coordinate. Then, we have
        \[
            (w \circ \bvec{x})(u,v) = a(u,v) \bvec{x}_u + b(u,v) \bvec{x}_v, 
        \]
    where $ a $, $ b $ are differentiable functions. Fix $ q = \bvec{x}(u,v) \in U $ and a curve $ \alpha = \bvec{x}(u(t),v(t))$ such that $ \alpha(0) = q $, $ \alpha^{\prime}(0) = w(q) $. Let $ \phi (u,v) = (f \circ \bvec{x})(u,v) $, then, we have 
        \[
            w(f)(q) = \frac{\mathrm{d}}{\mathrm{d}t} (f \circ \alpha)(0) = \frac{\mathrm{d}}{\mathrm{d}t} \phi (u(t), v(t)) \bigg|_{t=0} = \phi_u u^{\prime}(0) + \phi_v v^{\prime}(0),
        \] 
        and notice that in the basis $ \{\bvec{x}_u, \bvec{x}_v\} $, $ (u^{\prime} (t), v^{\prime} (t)) = (a(u,v), b(u,v)) $, so 
        \[
            w(f)(q) = \phi_u u^{\prime}(0) + \phi_v v^{\prime}(0) = \phi_u a(u,v) + \phi_v b(u,v) 
        \] 
        is differentiable as a function of $ (u,v) $. Since $ \bvec{x} $ is a local parametrization, $ w(f) $ is differentiable in $ U $. Conversely, let $ \pi_i $ be the standard projection, we have $ f_i = \left. \pi_i \right|_U : U \to \mathbb{R} $. By hypothesis, each $ w(f_i) $ is differentiable. Fix $ q \in U $ and a curve $ \alpha $ such that $ \alpha (0) = q $, $ \alpha^{\prime} (0) = w(q) $. Then 
        \[
            w(f_i)(q) = \frac{\mathrm{d}}{\mathrm{d}t} (f_i \circ \alpha)(0) = \frac{\mathrm{d}}{\mathrm{d}t} (\pi_i \circ \alpha)(0) = \left(w(q)\right)_i, 
        \]
        and 
        \[
            w(q) = \left(w(f_1)(q), w(f_2)(q), w(f_3)(q)\right). 
        \]
        Since each component is differentiable, $ w $ is differentiable as a map $ w: U \to \mathbb{R}^3 $, and hence differentiable as a vector field in $ U $ by Exercise 3.4.3.

        \item Let $ q \in U $, $ \alpha : I \to S $ be a curve such that $ \alpha(0) = q $ and $ \alpha'(0) = w(q) $. Then, we have 
        \begin{equation*}
            \begin{split}
                w(\lambda f + \mu f^{\prime} ) &= \frac{\mathrm{d}}{\mathrm{d}t} \left. \left((\lambda f + \mu f^{\prime} ) \circ \alpha \right) \right|_{t=0} \\
                &= \lambda \frac{\mathrm{d}}{\mathrm{d}t} \left. (f \circ \alpha) \right|_{t=0} + \mu \frac{\mathrm{d}}{\mathrm{d}t} \left. (f^{\prime} \circ \alpha) \right|_{t=0} \\
                &= \lambda w(f) + \mu w(f^{\prime}), 
            \end{split}
        \end{equation*}
        and 
        \begin{equation*}
            \begin{split}
                w (fg) &= \frac{\mathrm{d}}{\mathrm{d}t} \left. \left( (fg) \circ \alpha \right) \right|_{t=0} \\
                &= \frac{\mathrm{d}}{\mathrm{d}t} \left. \left( (f \circ \alpha)(g \circ \alpha) \right) \right|_{t=0} \\
                &= \left. \frac{\mathrm{d}}{\mathrm{d}t} (f \circ \alpha) \right|_{t=0} (g \circ \alpha)(0) + (f \circ \alpha)(0) \left. \frac{\mathrm{d}}{\mathrm{d}t} (g \circ \alpha) \right|_{t=0} \\
                &= w(f) g(q) + f(q) w(g).
            \end{split}
        \end{equation*}
    \end{enumerate}
\end{solutionManual}

% 3.4.8
\begin{exerciseManual}{3.4.8}
    Show that if $w$ is a differentiable vector field on a surface $S$ and $w(p) \neq 0$ for some $p \in S$, then it is possible to parametrize a neighborhood of $p$ by $x(u,v)$ in such a way that $x_{u} = w$.
\end{exerciseManual}

\begin{solutionManual}{3.4.8}
    Let's express $ w $ in a local parametrization $ \bvec{x}: U \to S $ in a neighborhood of $ p = \bvec{x}(0,0) $. Let $ (u,v) $ be a local coordinate, then, by a slight abuse of notation,
    \[
        w(u,v) \equiv (w \circ \bvec{x}) (u,v) = a(u,v) \bvec{x}_u + b(u,v) \bvec{x}_v,
    \]
    where $ a(u,v) $, $ b(u,v) $ are differentiable functions.
    \begin{claim}
        Let $ \bvec{a} (u,v) = \left(a(u,v), b(u,v)\right) $. Suppose $ \mathrm{d}\bvec{a} \neq 0 $, then there exists a neighborhood $ V $ of $ p $ and coordinates $ (\tilde{u}, \tilde{v}) $ such that $ \bvec{a} = a(\tilde{u}, \tilde{v}) $. I.e. $ w = (1,0) $ in the basis $ \{\tilde{\bvec{x}}_u, \tilde{\bvec{x}}_v\} = \{\bvec{x}_{\tilde{u}}, \bvec{x}_{\tilde{v}}\} $.
    \end{claim}
    \begin{proof}
        Let $ (u, v) $ be a local coordinate in a neighborhood of $ p $. Since $ \mathrm{d}\bvec{a} = \bvec{a}_u \mathrm{d}u + \bvec{a}_v \mathrm{d}v $ and $ \mathrm{d}\bvec{a}_p \neq 0 $, at least one of $ \bvec{a}_u (p) $ and $ \bvec{a}_v (p) $ is non-zero. Without loss of generality, suppose $ \bvec{a}_u (p) \neq 0 $. Then, by the Inverse Function Theorem, there exists a neighborhood $ V $ of $ p $ such that the map $ \psi: V \to \mathbb{R}^2 $ defined by $ \psi (u,v) = (a(u,v), v) $ is a diffeomorphism onto its image. Let $ (\tilde{u}, \tilde{v}) = \psi (u,v) $, then we have $ \bvec{a} = a(\tilde{u}, \tilde{v}) $, as desired.
    \end{proof}
    \noindent Let $ \Phi (t, \bvec{x}(0,0)) $ be the solution to the differential equation 
    \[
        \frac{\mathrm{d}y}{\mathrm{d}t} = \bvec{a}(y), \quad y(0) = \bvec{x}(0,0),
    \]
    and let $ \phi (u, v) = \Phi (u, (0,v)) $. By the smooth dependence of solution of an ODE on initial conditions, $ \Phi $, and hence $ \phi $, is differentiable. Then, we have 
    \[
        \frac{\partial}{\partial u} \phi (u,v) = \bvec{a}(\phi (u,v)) = w(\phi (u,v)).
    \]
    Furthermore, since $ \phi (0,v) = \Phi (0, (0, v)) = (0, v) $, we have $ \mathrm{d}\phi_p = 1 $, and hence $ \phi $ is a local parametrization around $ p $. Let $ \tilde{\bvec{x}} (u,v) = \phi (u,v) $, then we have $ \tilde{\bvec{x}}_u = w(\tilde{\bvec{x}}(u,v)) $.

    \begin{remark}
        This is the \textbf{vector-straightening lemma} for surfaces, which is a special case of the more general Frobenius theorem.
    \end{remark}
\end{solutionManual}

% 3.4.9
\begin{exerciseManual}{3.4.9}
    ~

    \begin{enumerate}[label=\textbf{\alph*.}]
        \item Let $A : V \to W$ be a nonsingular linear map of vector spaces $V$ and $W$ of dimension $2$ and endowed with inner products $\langle\, , \, \rangle$ and $(\, , \, )$, respectively. $A$ is a \emph{similitude} if there exists a real number $\lambda \neq 0$ such that 
        \[
        (A v_1, A v_2) = \lambda \langle v_1, v_2 \rangle
        \quad\text{for all } v_1, v_2 \in V.
        \]
        Assume that $A$ is not a similitude and show that there exists a \emph{unique} pair of orthonormal vectors $e_1$ and $e_2$ in $V$ such that $A e_1, A e_2$ are orthogonal in $W$.

        \item Use part \textbf{a.} to prove \emph{Tissot’s theorem}:  
        Let $\varphi : U_1 \subset S_1 \to S_2$ be a diffeomorphism from a neighborhood $U_1$ of a point $p$ of a surface $S_1$ into a surface $S_2$. Assume that the linear map $\mathrm{d}\varphi$ is nowhere a similitude. Then it is possible to parametrize a neighborhood of $p$ in $S_1$ by an orthogonal parametrization $\bvec{x}_1 : U \to S_1$ such that $\varphi \circ \bvec{x}_1 = \bvec{x}_2 : U \to S_2$ is also an orthogonal parametrization in a neighborhood of $\varphi(p) \in S_2$.
    \end{enumerate}
\end{exerciseManual}

\begin{solutionManual}{3.4.9}
    ~ 

    \begin{enumerate}[label=\textbf{\alph*.}]
        \item Suppose there does not exist a real number such that $ (A v_1, A v_2) = \lambda \langle v_1, v_2 \rangle $ for all $ v_1, v_2 \in V $. Let $ \{u_1, u_2\} $ be an orthonormal basis of $ V $, and let $ A u_1 = w_1 $, $ A u_2 = w_2 $. Since $ A $ is not a similitude, we have $ (w_1, w_2) \neq 0 $. Define
        \[
            e_1 = \frac{1}{\sqrt{2(1 - \rho)}} (u_1 - u_2), \quad e_2 = \frac{1}{\sqrt{2(1 + \rho)}} (u_1 + u_2), \quad \text{where }\rho = \frac{(w_1, w_2)}{\Vert w_1\Vert \Vert w_2\Vert}. 
        \]
        Then, we have $ \langle e_1, e_2 \rangle = 0 $, $ \|e_1\| = \|e_2\| = 1 $, and
        \[
            (A e_1, A e_2) = \frac{1}{\sqrt{(1 - \rho)(1 + \rho)}} \left( (w_1, w_1) - (w_2, w_2) \right) = 0.
        \]
        Suppose there exists another pair of orthonormal vectors $ f_1 $, $ f_2 $ such that $ (A f_1, A f_2) = 0 $. Let $ f_1 = \cos \theta e_1 + \sin \theta e_2 $, $ f_2 = - \sin \theta e_1 + \cos \theta e_2 $, then, we have 
        \[
            0 = (A f_1, A f_2) = \cos \theta \sin \theta \left((A e_1, A e_1) - (A e_2, A e_2)\right). 
        \]
        If $ (A e_1, A e_1) = (A e_2, A e_2) $, then for any $ v = a e_1 + b e_2 \in V $, we have
        \[
            \left\vert A v \right\vert^2 = (Av, Av) = (e_1, e_1) (a^2 + b^2) = (e_1, e_1) \Vert v \Vert^2 .
        \]
        By the polarization identity, 
        \[
            (A v_1, A v_2) = \frac{1}{4} \left[ \left\Vert v_1 + v_2 \right\Vert^2 - \left\Vert v_1 - v_2 \right\Vert^2 \right] = (e_1, e_1) \langle v_1, v_2 \rangle,
        \]  
        $ A $ is a similitude, which is a contradiction. Thus, we have $ \cos \theta \sin \theta = 0 $, and hence $ \theta = k \pi/2 $, $ k \in \mathbb{Z} $. Therefore, the pair $ (e_1, e_2) $ is unique up to sign.

        \item Suppose $ \bvec{x}_1: U \subseteq \mathbb{R}^2 \to S_1 $ satisfies $ \langle \bvec{x}_{1u}, \bvec{x}_{1v} \rangle = 0 $. Let $ \bvec{x}_2 = \phi \circ \bvec{x}_1 : U \to S_2 $. Since $ \mathrm{d}\phi $ is not a similitude, by part \textbf{a.}, there exists a unique pair of orthonormal vectors $ e_1, e_2 \in T_p (S_1) $ such that $ \left(\mathrm{d}\phi (p) (e_1), \mathrm{d}\phi (p) (e_2)\right) = 0 $ in $ T_{\phi(p)} (S_2) $. Let $ e_1 = \cos \theta \, \bvec{x}_{1u} + \sin \theta \, \bvec{x}_{1v} $, $ e_2 = - \sin \theta \, \bvec{x}_{1u} + \cos \theta \, \bvec{x}_{1v} $. Let $ \tilde{u} = \cos \theta \, u - \sin \theta \, v $, $ \tilde{v} = \sin \theta \, u + \cos \theta \, v $. Then,
        \[
            \tilde{\bvec{x}}_{1\tilde{u}} = \bvec{x}_{1u} u_{\tilde{u}} + \bvec{x}_{1v} v_{\tilde{u}} = \cos \theta \, \bvec{x}_{1u} + \sin \theta \, \bvec{x}_{1v} = e_1,
        \]
        \[
            \tilde{\bvec{x}}_{1\tilde{v}} = \bvec{x}_{1u} u_{\tilde{v}} + \bvec{x}_{1v} v_{\tilde{v}} = -\sin \theta \, \bvec{x}_{1u} + \cos \theta \, \bvec{x}_{1v} = e_2.
        \]
        Thus, $ \tilde{\bvec{x}}_1 $ is an orthogonal parametrization of $ S_1 $ about $ p $. Let $ \tilde{\bvec{x}}_2 = \phi \circ \tilde{\bvec{x}}_1 $, then
        \[
            \tilde{\bvec{x}}_{2\tilde{u}} = \mathrm{d}\phi (\tilde{\bvec{x}}_1) (\tilde{\bvec{x}}_{1\tilde{u}}) = \mathrm{d}\phi (\tilde{\bvec{x}}_1) (e_1), \quad \tilde{\bvec{x}}_{2\tilde{v}} = \mathrm{d}\phi (\tilde{\bvec{x}}_1) (\tilde{\bvec{x}}_{1\tilde{v}}) = \mathrm{d}\phi (\tilde{\bvec{x}}_1) (e_2).
        \]
        \[
            \implies \left(\tilde{\bvec{x}}_{2\tilde{u}}, \tilde{\bvec{x}}_{2\tilde{v}}\right) = \left( \mathrm{d}\phi (\tilde{\bvec{x}}_1) (e_1), \mathrm{d}\phi (\tilde{\bvec{x}}_1) (e_2) \right) = 0.
        \]
        Thus, $ \tilde{\bvec{x}}_2 $ is an orthogonal parametrization of $ S_2 $ about $ \phi (p) $. 
    \end{enumerate}
\end{solutionManual}

% 3.4.10

\begin{exerciseManual}{3.4.10}
    Let $T$ be the torus of Example 6 of Sec.\ 2--2 and define a map 
    $\varphi : \mathbb{R}^2 \to T$ by
    \[
        \varphi(u,v) = \bigl((r\cos u + a)\cos v,\; (r\cos u + a)\sin v,\; r\sin u\bigr),
    \]
    where $u$ and $v$ are the Cartesian coordinates of $\mathbb{R}^2$. Let $u = at$, $v = bt$ be a straight line in $\mathbb{R}^2$, passing by $(0,0) \in \mathbb{R}^2$, and consider the curve in $T$
    \[
    \alpha(t) = \varphi(at, bt).
    \]
    Prove that:

    \begin{enumerate}[label=\textbf{\alph*.}]
        \item $\varphi$ is a local diffeomorphism.

        \item The curve $\alpha(t)$ is a regular curve; $\alpha(t)$ is a closed curve if and only if $b/a$ is a rational number.

        \item (\emph{Optional})  
        If $b/a$ is irrational, the curve $\alpha(t)$ is dense in $T$; that is, in each neighborhood of a point $p \in T$ there exists a point of $\alpha(t)$.
    \end{enumerate}
\end{exerciseManual}

\begin{solutionManual}{3.4.10}
    ~ 

    \begin{enumerate}
        \item[\textbf{a.}] Since each component $ \phi_1, \phi_2, \phi_3 $ of $ \varphi $ is composed of elementary functions and thus differentiable, $ \varphi $ is differentiable. The mapping is not globally bijective,  but since 
        \[
            J_{\varphi} = 
            \begin{pmatrix}
                -r \sin u \cos v & -(r \cos u + a) \sin v \\
                -r \sin u \sin v & (r \cos u + a) \cos v \\
                r \cos u & 0
            \end{pmatrix} \implies \operatorname{rank} J_{\varphi}(u,v) = 2 \quad \text{ for all } (u,v) \in \mathbb{R}^2,
        \]
        by the Inverse Function Theorem, $ \varphi $ is a local homeomorphism, and hence a local diffeomorphism.

        \item[\textbf{b.}] We have $ \alpha^{\prime} (t) = \varphi_u (at, bt) a + \varphi_v (at, bt) b $. Since $ \{\varphi_u, \varphi_v\} $ are linearly independent, $ \alpha^{\prime} (t) \neq 0 $ for all $ t $ when $ a, b $ are not both zero, and hence $ \alpha (t) $ is a regular curve. Suppose $ \alpha (t) $ is a closed curve, then there exists $ T > 0 $ such that $ \alpha (t + T) = \alpha (t) $ for all $ t $. Then we have $ \varphi (a(t + T), b(t + T)) = \varphi (at, bt) $, and by inspecting $ \phi_3 $, there must exist $ m, n \in \mathbb{Z} $ such that $ a T = 2 m \pi $, $ b T = 2 n \pi $. Thus, we have $ b/a = n/m \in \mathbb{Q} $. Conversely, suppose $ b/a = n/m $, $ m, n \in \mathbb{Z} $. Let $ T = 2 \pi \operatorname{lcm} \left(\frac{m}{a}, \frac{n}{b}\right) $, then we have
        \[
            \alpha (t + T) = \varphi (a(t + T), b(t + T)) = \varphi (at + 2 m^{\prime} \pi, bt + 2 n^{\prime} \pi) = \varphi (at, bt) = \alpha (t), \quad m^{\prime} , n^{\prime} \in \mathbb{Z}.
        \]
        \item[\textbf{*c.}] Suppose $ b/a $ is irrational. Let $ p \in T $, and let $ U $ be a neighborhood of $ p $. Let $ \mathbb{T}^2 = \mathbb{R}^2 / (2 \pi \mathbb{Z})^2 $ be the flat torus, and consider the projection
        \[
            \pi : \mathbb{R}^2 \to \mathbb{T}^2, \quad \pi (u,v) = (u + 2 \pi \mathbb{Z}, v + 2 \pi \mathbb{Z}).
        \]
        The map $ \psi: \mathbb{T}^2 \to T^2 $ defined by $ \psi \left(u + 2 \pi \mathbb{Z}, v + 2 \pi \mathbb{Z}\right) = \varphi (u,v) $ is well-defined, since the components of $ \phi $ are periodic with period $ 2 \pi $ in $ (u,v) $. Therefore, $ \varphi = \psi \circ \pi $ and $ \phi $ factors through $ \pi $. Then, write $ \alpha (t) = \psi (\pi (at, bt)) \in T $. Since $ \phi $ is a diffeomorphism onto its image and $ \psi = \left. \phi \right|_{[0,2 \pi ), [0, 2 \pi)} $, $ \psi $ is a diffeomorphism onto its image, and in particular a homeomorphism. Thus, $ \alpha (t) $ is dense in $ T $ if and only if $ \beta (t) \equiv \pi (at, bt) $ is dense in $ \mathbb{T}^2 $. 
        \[
        \begin{tikzcd}
        \mathbb{R}^2 \arrow[r, "\varphi"] \arrow[d, "\pi"'] & T \\
        \mathbb{T}^2 \arrow[ur, "\psi"'] &
        \end{tikzcd}
        \]
        \begin{lemma}[orbit of an irrational rotation is dense]
            Let $ R_{\theta} : S^1 \to S^1 $ be the rotation defined by $ R_{\theta} (z) = e^{i \theta} z $, where $ \theta / (2 \pi) \in \mathbb{R}\setminus \mathbb{Q} $. Then, for any $ z \in S^1 $, the orbit $ O = \{ R_{\theta}^n (z) : n \in \mathbb{Z} \} $ is dense in $ S^1 $. 
        \end{lemma}
        \begin{proof}
            Suppose $ O $ is not dense in $ S^1 $, so $ C \equiv \operatorname{cl}_{S^1} (O) \subsetneqq S^1 $. Since $ R_\theta $ is continuous and bijective, we have $ R_{\theta} (C) = C $, so $ C $ is closed and invariant, and $ \varnothing \subsetneq S^1 \setminus C $ is open. Therefore, there is some non-empty interval $ I = (a,b) \subseteq S^1 $ such that $ I_n \equiv R_{\theta} (I) \subseteq S^1 $ for all $ n \in \mathbb{N} $. Suppose $ I_n \cap I_m \neq \varnothing $, then there exists $ x \in I $ such that $ x + n \theta \equiv x + m \theta $ (mod $ 2\pi $) for $ m, n \in \mathbb{Z} $. Hence $ (n-m)\theta \in (- |I|, |I|) $ (mod $ 2 \pi $), where $ |I| < 2\pi $. Then, $ (n-m)\theta = 2 k \pi $ for some $ k \in \mathbb{Z} $, which contradicts the irrationality of $ \theta / (2 \pi) $. Thus, $ I_n \cap I_m = \varnothing $ for all $ n \neq m $. Since $ S^1 $ is finite, $ \bigcup_{n=0}^{\infty} I_n \subseteq S^1 $ cannot contain infinitely many disjoint open intervals of finite length, which is a contradiction. Therefore, $ O $ is dense in $ S^1 $.
        \end{proof}

        Define $ \Gamma = \{ \beta (t) \mid t \in \mathbb{R} \} $. Let $ [(u_0, v_0)] \in \mathbb{T}^2 $ be arbitrary and let $ \varepsilon>0 $. Since $ a \neq 0 $, for every $ k \in \mathbb{Z} $ define 
        \[
            t_k = \frac{u_0 + 2 k \pi}{a} \implies u(t_k) \equiv u_0 \;(\text{mod } 2\pi ), \quad v(t_k) = \frac{b}{a} u_0 + 2 k \pi \frac{b}{a} \;(\text{mod } 2\pi ).
        \]
        Let $ \gamma = v_0 - u_0 a / b $ and $ \theta / 2 \pi = b / a $. Then since $ u(t_k) - u_0 = 0 $, the condition 
        \[
            d_{\mathbb{T}^2} (\beta (t_k), [(u_0, v_0)]) \equiv \max \left(\vert u(t) - u_0 \vert, \vert v(t) - v_0 \vert \right) < \varepsilon
        \]
        is satisfied at $ t = t_k $ whenever $ \vert k \theta - \gamma \vert < \varepsilon $ in $ S^1 $. By the lemma, since $ b / a $ is irrational, such a $ k $ exists. Thus, $ \Gamma $ is dense in $ \mathbb{T}^2 $, and hence $ \alpha (t) $ is dense in $ T $.
    \end{enumerate}
\end{solutionManual}

% 3.4.11
\begin{exerciseManual}{*3.4.11}
    Use the local uniqueness of trajectories of a vector field $w$ in $U \subset S$ to prove the following result. Given $p \in U$, there exists a unique trajectory $\alpha : I \to U$ of $w$, with $\alpha(0) = p$, which is \emph{maximal} in the following sense: Any other trajectory $\beta : J \to U$, with $\beta(0) = p$, is the restriction of $\alpha$ to $J$ (i.e., $J \subset I$ and $\alpha|_J = \beta$).
\end{exerciseManual}

\begin{solutionManual}{3.4.11}
    A trajecry $ \alpha: I \to U $ of $ w $ with $ \alpha (0) = p $ satisfies $ \alpha^{\prime} (t) = w (\alpha (t)) $ for all $ t \in I $. Let $ \mathcal{F} $ be the set of all trajectories $ \beta : J_{\beta} \to U $, such that $ \{0\} \subseteq J_{\beta} \subseteq \mathbb{R} $ is open for each $ \beta $. Define $ I = \bigcup_{\beta \in \mathcal{F}} J_{\beta} $. For each $ t \in I $, pick any $ \beta \in \mathcal{F} $ such that $ t \in J_{\beta} $, and define $ \alpha (t) = \beta (t) $ on $ J_{\beta} $. We claim this is the desired maximal trajectory. Suppose there exists another $ \gamma \in \mathcal{F} $ such that $ \alpha |_{J_{\gamma}} = \gamma $ and $ t \in J_{\beta} \cap J_{\gamma} $. Then, the local uniqueness of trajectories implies there exists $ \{0\} \subseteq K \subseteq J_{\beta} \cap J_{\gamma} $ such that $ \beta |_{K} = \gamma |_{K} $. The set $ \{ s \in J_{\beta} \cap J_{\gamma} \mid| \beta (s) = \gamma (s) \} $ is open in $ J_{\beta} \cap J_{\gamma} $ by local uniqueness theorem, and closed in $ J_{\beta} \cap J_{\gamma} $ by continuity, thus it is equal to $ J_{\beta} \cap J_{\gamma} $. Therefore, $ \alpha (t) $ is well-defined. By construction, $ \alpha $ is a trajectory of $ w $ with $ \alpha (0) = p $. Furthermore, for any other trajectory $ \beta : J \to U $ with $ \beta (0) = p $, by definition of $ I $, we have $ J \subseteq I $ and $ \alpha |_{J} = \beta $. Thus, $ \alpha $ is maximal. 
\end{solutionManual}

% 3.4.12
\begin{exerciseManual}{*3.4.12}
    Prove that if $w$ is a differentiable vector field on a compact surface $S$ and $\alpha(t)$ is the maximal trajectory of $w$ with $\alpha(0) = p \in S$, then $\alpha(t)$ is defined for all $t \in \mathbb{R}$.
\end{exerciseManual}

\begin{solutionManual}{3.4.12}
    Since $ S $ is compact, $ \alpha (t) $ is a 
\end{solutionManual}

% 3.4.13
\begin{exerciseManual}{3.4.13}
    Construct a differentiable vector field on an open disk of the plane (which is not compact) such that a maximal trajectory $\alpha(t)$ is not defined for all $t \in \mathbb{R}$. (This shows that the compactness condition of Exercise 12 is essential.)
\end{exerciseManual}

\begin{solutionManual}{3.4.13}
    Let $ D = \{(x,y) \in \mathbb{R}^2 : x^2 + y^2 < 1 \} $ be the open unit disk in $ \mathbb{R}^2 $. Define the vector field $ w: D \to \mathbb{R}^2 $ by $ w(x,y) = (1,0) $, and a trajectory $ \alpha (t) = \left(x(t), y(t)\right) : I \to D $ of $ w $. Then, we have $ x^{\prime} (t) = 1 $, $ y^{\prime} (t) = 0 $ subject to $ x(0) = y(0) = 0 $. Thus, $ x(t) = t $, $ y(t) = 0 $, and $ \alpha (t) = (t, 0) $. The maximal interval $ I $ such that $ \alpha (t) \in D $ is $ (-1, 1) $, which is not equal to $ \mathbb{R} $. 

    \begin{remark}
        The closed disk would seem like a counterexample to the counterexample. However, the closed disk is compact but not a surface, and hence does not contradict the previous exercise.
    \end{remark}
\end{solutionManual}

\end{CJK}
\end{document}