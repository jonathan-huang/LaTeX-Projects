\documentclass[a4paper]{article}
%% Formatting %%
\usepackage[margin=3cm]{geometry}
\usepackage{type1cm, titlesec, fancyhdr, titling}
\usepackage{multicol}
\usepackage[dvipsnames]{xcolor}
\usepackage{ulem}
\usepackage{parskip}
\setlength{\parindent}{2em}
\setlength{\headheight}{15pt}
\setlength{\droptitle}{-1.5cm}
\parindent=24pt
%% Math and Symbols %%
\usepackage{amsmath,amsthm,amssymb, mathtools}
\usepackage{yhmath, faktor, dsfont}
\usepackage{academicons, wasysym, marvosym}
\usepackage[scr]{rsfso} 
\usepackage{latexsym, amsmath, amscd, amsmath, amsthm}
\usepackage{amssymb,amsmath,amsthm,graphicx,dsfont}
\usepackage{hyperref}

%% Enhancement %%
\usepackage{graphicx, tabularx}
\usepackage[shortlabels,inline]{enumitem}
%% TikZ %%
\usepackage{tikz-cd}
\usepackage[breakable]{tcolorbox}
\usetikzlibrary{decorations.pathmorphing}
\usetikzlibrary{calc, arrows,matrix}

%% Other packages %%
\usepackage{amsopn}

%% Traditional Chinese %%
\usepackage{CJKutf8}

%% Math environments %%
\newtheoremstyle{mystyle}
  {6pt}{15pt}% 上下間距
  {}%          內文字體
  {}%              縮排
  {\bf}%       標頭字體
  {.}%       標頭後標點
  {1em}% 內文與標頭距離
  {}% Theorem head spec (can be left empty, meaning 'normal')
\theoremstyle{mystyle}	
\newtheorem{theorem}{Theorem}
\newtheorem{definition}{Definition}
\newtheorem{example}[theorem]{Example}
\newtheorem{exercise}{Exercise}
\newtheorem{solution}{Solution}
\newtheorem{corollary}[theorem]{Corollary}
\newtheorem{property}[theorem]{Property}
\newtheorem{proposition}[theorem]{Proposition}
\newtheorem{lemma}{Lemma}
\newtheorem{problem}[theorem]{Problem}
\newtheorem{answer}{Answer}[section]
\newtheorem{fact}[theorem]{fact}
\newtheorem*{remark}{Remark}
\newtheorem*{claim}{Claim}
\newtheorem*{observation}{Observation}

\newcommand{\bvec}[1]{\mathbf{#1}} % vector

\begin{document}
\begin{CJK}{UTF8}{bkai}

\title{%
  \textbf{2025 Fall Introduction to Geometry} \\
  \vspace{0.5cm}
  \large 
  Homework 6 (Due October 17, 2025)\\
}
\author{物理三 黃紹凱 B12202004}

\maketitle

% Problem 1
\begin{problem}[Do Carmo 2.5.3]
    Obtain the first fundamental form of the sphere in the parametrization given by stereographic projection (cf.\ Exercise 16, Sec.\ 2--2).
\end{problem}

\begin{solution}
    Refer to Exercise 2.2.16, let the sphere be $ S^2 = \{ (x,y,z) \in \mathbb{R}^3 : x^2 + y^2 + (z-1)^2 = 1 \} $. The stereographic projection from the north pole $ N = (0,0,2) $ to the $ xy $-plane is given by
    \begin{equation*}
        \mathbf{x}(u,v) = \left( \frac{4u}{u^2 + v^2 + 4}, \frac{4v}{u^2 + v^2 + 4}, \frac{2(u^2 + v^2)}{u^2 + v^2 + 4} \right).
    \end{equation*}
    We have
    \begin{align*}
        \mathbf{x}_u &= \left( \frac{4(-u^2 + v^2 + 4)}{(u^2 + v^2 + 4)^2}, \frac{-8uv}{(u^2 + v^2 + 4)^2}, \frac{16u}{(u^2 + v^2 + 4)^2} \right), \\
        \mathbf{x}_v &= \left( \frac{-8uv}{(u^2 + v^2 + 4)^2}, \frac{4(u^2 - v^2 + 4)}{(u^2 + v^2 + 4)^2}, \frac{16v}{(u^2 + v^2 + 4)^2} \right).
    \end{align*}
    Thus we have
    \begin{align*}
        E &= \langle \mathbf{x}_u, \mathbf{x}_u \rangle = \frac{16 (-u^2 + v^2 + 4)^2 + 64u^2 v^2 + 256 u^2}{(u^2 + v^2 + 4)^4} = \frac{16}{(u^2 + v^2 + 4)^2}, \\
        F &= \langle \mathbf{x}_u, \mathbf{x}_v \rangle = \frac{-32uv(-u^2 + v^2 + 4) - 32uv (u^2 - v^2 + 4) + 256uv}{(u^2 + v^2 + 4)^4} = 0, \\
        G &= \langle \mathbf{x}_v, \mathbf{x}_v \rangle = \frac{64 u^2 v^2 + 16 (u^2 - v^2 + 4)^2 + 256 v^2}{(u^2 + v^2 + 4)^4} = \frac{16}{(u^2 + v^2 + 4)^2}.
    \end{align*}
    Therefore, the first fundamental form is
    \begin{equation*}
        I_p = \frac{16}{(u^2 + v^2 + 4)^2} \left((u^{\prime})^2 + (v^{\prime})^2\right).
    \end{equation*}
\end{solution}

% Problem 2
\begin{problem}[Do Carmo 2.5.7]
    The coordinate curves of a parametrization $\mathbf{x}(u,v)$ constitute a \emph{Tchebyshef net} if the lengths of the opposite sides of any quadrilateral formed by them are equal. Show that a necessary and sufficient condition for this is
    \[
    \frac{\partial E}{\partial v} \;=\; \frac{\partial G}{\partial u} \;=\; 0.
    \]
\end{problem}

\begin{solution}
    The \emph{coordinate curves} of a parametrization $ \mathbf{x}(u,v) $ are the curves obtained by fixing one of the parameters and varying the other. Suppose we have a quadrilateral formed by the coordinate curves at points $ (u_1, v_1), (u_2, v_1), (u_2, v_2), (u_1, v_2) $. Let $ s (\bvec{x}(u_1, v_1), \bvec{x}(u_2, v_2)) \equiv s ((u_1, v_1), (u_2, v_2)) $ denote the arc length between two points. Then the lengths of the opposite sides are equal if and only if
    \begin{align*}
        s ( (u_1, v_1), (u_2, v_1) ) = s ( (u_1, v_2), (u_2, v_2) ) \;\Longrightarrow\; \int _{u_1}^{u_2} \mathrm{d}u \,\sqrt{E(u,v_1)} &= \int _{u_1}^{u_2} \mathrm{d}u \,\sqrt{E(u,v_2)}, \\
        s ( (u_1, v_1), (u_1, v_2) ) = s ( (u_2, v_1), (u_2, v_2) ) \;\Longrightarrow\; \int _{v_1}^{v_2} \mathrm{d}v \,\sqrt{G(u_1,v)} &= \int _{v_1}^{v_2} \mathrm{d}v \,\sqrt{G(u_2,v)}.
    \end{align*}
    Since $ u_1, u_2, v_1, v_2 $ are arbitrary, we have
    \begin{equation*}
        \sqrt{E(u,v_1)} = \sqrt{E(u,v_2)}, \quad \sqrt{G(u_1,v)} = \sqrt{G(u_2,v)}.
    \end{equation*}
    Therefore, $ E $ is independent of $ v $ and $ G $ is independent of $ u $, giving the desired result:
    \begin{equation*}
        \frac{\partial E}{\partial v} = 0, \quad \frac{\partial G}{\partial u} = 0.
    \end{equation*}
\end{solution}

% Problem 3
\begin{problem}[Do Carmo 2.5.10]
    Let $P=\{(x,y,z)\in\mathbb{R}^{3}\,;\; z=0\}$ be the $xy$-plane and let $\mathbf{x}\colon U\to P$ be a parametrization of $P$ given by
    \[
    \mathbf{x}(\rho,\theta) = (\rho\cos\theta,\; \rho\sin\theta),
    \]
    where
    \[
    U=\{(\rho,\theta)\in\mathbb{R}^{2}\,;\; \rho>0,\; 0<\theta<2\pi\}.
    \]
    Compute the coefficients of the first fundamental form of $P$ in this parametrization.
\end{problem}

\begin{solution}
    We have
    \begin{align}
        \bvec{x}_{\rho} &= (\cos \theta, \sin \theta, 0), \\
        \bvec{x}_{\theta} &= (-\rho \sin \theta, \rho \cos \theta, 0).
    \end{align}
    Thus the coefficients of the first fundamental form are
    \begin{align*}
        E = \langle \bvec{x}_{\rho}, \bvec{x}_{\rho} \rangle &= \cos^2 \theta + \sin^2 \theta = 1, \\
        F = \langle \bvec{x}_{\rho}, \bvec{x}_{\theta} \rangle &= -\rho \cos \theta \sin \theta + \rho \sin \theta \cos \theta = 0, \\
        G = \langle \bvec{x}_{\theta}, \bvec{x}_{\theta} \rangle &= \rho^2 \sin^2 \theta + \rho^2 \cos^2 \theta = \rho^2.
    \end{align*}
\end{solution}

% Problem 4
\begin{problem}[Do Carmo 2.5.15, Orthogonal Families of Curves]
    ~
    \begin{enumerate}[label=\textbf{\alph*.}]
        \item Let $E,F,G$ be the coefficients of the first fundamental form of a regular surface $S$ in the parametrization $\mathbf{x}\colon U\subset \mathbb{R}^{2}\to S$. Let $\varphi(u,v)=\text{const.}$ and $\psi(u,v)=\text{const.}$ be two families of regular curves on $\mathbf{x}(U)\subset S$ (cf.\ Exercise 28, Sec.\ 2--4). Prove that these two families are orthogonal (i.e., whenever two curves of distinct families meet, their tangent lines are orthogonal) if and only if
        \[
        E\,\varphi_{v}\psi_{v}\;-\;F\bigl(\varphi_{u}\psi_{v}+\varphi_{v}\psi_{u}\bigr)\;+\;G\,\varphi_{u}\psi_{u}\;=\;0.
        \]
        \item Apply part~a to show that on the coordinate neighborhood $\mathbf{x}(U)$ of the helicoid of Example~3 the two families of regular curves
        \[
        v\cos u=\text{const.},\quad v\neq 0,\qquad
        \bigl(v^{2}+a^{2}\bigr)\sin^{2}u=\text{const.},\quad v\neq 0,\quad u\neq \pi,
        \]
        are orthogonal.
    \end{enumerate}
\end{problem}

\begin{solution}
    ~
    \begin{enumerate}[label=\textbf{\alph*.}]
        \item Suppose $ \varphi(u,v) = c_1 $ and $ \psi(u,v) = c_2 $ are two families of regular curves on $ \mathbf{x}(U) \subset S $. The curves satisfy
        \begin{equation*}
            \phi_u u^{\prime} + \phi_v v^{\prime} = 0, \quad \psi_u u^{\prime} + \psi_v v^{\prime} = 0.
        \end{equation*}
        So we can choose the tangent vectors of the two families of curves to be
        \begin{equation*}
            t(\varphi) = - \varphi_v \bvec{x}_u + \varphi_u \bvec{x}_v, \quad t(\psi) = - \psi_v \bvec{x}_u + \psi_u \bvec{x}_v.
        \end{equation*}
        The two families are orthogonal if and only if $ \langle t(\varphi), t(\psi) \rangle = 0 $, which is equivalent to
        \begin{align*}
            \langle t(\varphi), t(\psi) \rangle &= \langle - \varphi_v \bvec{x}_u + \varphi_u \bvec{x}_v, - \psi_v \bvec{x}_u + \psi_u \bvec{x}_v \rangle \\
            &= \varphi_v \psi_v \langle \bvec{x}_u, \bvec{x}_u \rangle - \varphi_v \psi_u \langle \bvec{x}_u, \bvec{x}_v \rangle - \varphi_u \psi_v \langle \bvec{x}_v, \bvec{x}_u \rangle + \varphi_u \psi_u \langle \bvec{x}_v, \bvec{x}_v \rangle \\
            &= E \varphi_v \psi_v - F (\varphi_u \psi_v + \varphi_v \psi_u) + G \varphi_u \psi_u = 0.
        \end{align*}
        \item From Example 3, the helicoid is given by the parametrization $ \bvec{x}(u,v) = (v \cos u, v \sin u, a u) $, with the coefficients of the first fundamental form being
        \begin{equation*}
            E(u,v) = v^2 + a^2, \quad F(u,v) = 0, \quad G(u,v) = 1.
        \end{equation*}
        Let $ \phi (u,v) = v \cos u $, $ \psi (u,v) = (v^2 + a^2) \sin^2 u $. Then
        \begin{align*}
            \phi_u &= -v \sin u, & \phi_v &= \cos u, \\
            \psi_u &= 2 (v^2 + a^2) \sin u \cos u, & \psi_v &= 2v \sin^2 u.
        \end{align*}
        Substituting these into equation (\ref{eq:orthogonal_families}) in part (a), we have
        \begin{align*}
            &(v^2 + a^2) \cos u (2v \sin^2 u) - 0 + 1 (-v \sin u)(2 (v^2 + a^2) \sin u \cos u) = 0.
        \end{align*}
        Therefore, the two families of regular curves are orthogonal.
    \end{enumerate}
\end{solution}

\end{CJK}
\end{document}