\documentclass[a4paper]{article}
%% Formatting %%
\usepackage[margin=3cm]{geometry}
\usepackage{type1cm, titlesec, fancyhdr, titling}
\usepackage{multicol}
\usepackage[dvipsnames]{xcolor}
\usepackage{ulem}
\usepackage{parskip}
\setlength{\parindent}{2em}
\setlength{\headheight}{15pt}
\setlength{\droptitle}{-1.5cm}
\parindent=24pt
%% Math and Symbols %%
\usepackage{amsmath,amsthm,amssymb, mathtools}
\usepackage{yhmath, faktor, dsfont}
\usepackage{academicons, wasysym, marvosym}
\usepackage[scr]{rsfso} 
\usepackage{latexsym, amsmath, amscd, amsmath, amsthm}
\usepackage{amssymb,amsmath,amsthm,graphicx,dsfont}
\usepackage{hyperref}

%% Enhancement %%
\usepackage{graphicx, tabularx}
\usepackage[shortlabels,inline]{enumitem}
%% TikZ %%
\usepackage{tikz-cd}
\usepackage[breakable]{tcolorbox}
\usetikzlibrary{decorations.pathmorphing}
\usetikzlibrary{calc, arrows,matrix}

%% Other packages %%
\usepackage{amsopn}

%% Traditional Chinese %%
\usepackage{CJKutf8}

%% Math environments %%
\newtheoremstyle{mystyle}
  {6pt}{15pt}% 上下間距
  {}%          內文字體
  {}%              縮排
  {\bf}%       標頭字體
  {.}%       標頭後標點
  {1em}% 內文與標頭距離
  {}% Theorem head spec (can be left empty, meaning 'normal')
\theoremstyle{mystyle}	
\newtheorem{theorem}{Theorem}
\newtheorem*{definition}{Definition}
\newtheorem{example}[theorem]{Example}
\newtheorem{exercise}{Exercise}
\newtheorem{solution}{Solution}
\newtheorem{corollary}[theorem]{Corollary}
\newtheorem{property}[theorem]{Property}
\newtheorem{proposition}[theorem]{Proposition}
\newtheorem{lemma}[theorem]{Lemma}
\newtheorem{problem}[theorem]{Problem}
\newtheorem{answer}{Answer}[section]
\newtheorem{fact}[theorem]{fact}
\newtheorem*{remark}{Remark}
\newtheorem*{claim}{Claim}
\newtheorem*{observation}{Observation}

\begin{document}
\begin{CJK}{UTF8}{bkai}

\title{%
  \textbf{2025 Fall Introduction to Geometry} \\
  \vspace{0.5cm}
  \large 
  Homework 1 (Due Sep 12, 2025)\\
}
\author{物理/數學三 黃紹凱 B12202004}
\date{\today}

\maketitle

% Problem 1
\begin{problem}{1.2.5}
    Let $\alpha \colon I \to \mathbb{R}^3$ be a parametrized curve, with $\alpha'(t) \neq 0$ for all $t \in I$. Show that $|\alpha(t)|$ is a nonzero constant if and only if $\alpha(t)$ is orthogonal to $\alpha'(t)$ for all $t \in I$.
\end{problem}

\begin{solution} 
    Suppose $|\alpha(t)| = c \neq 0$ for all $t \in I$. Then,
    \[
        \frac{\mathrm{d}}{\mathrm{d}t}\vert \alpha(t) \vert^{2} = 2 \alpha (t) \cdot \alpha^{\prime} (t) = \frac{\mathrm{d}}{\mathrm{d}t} c^{2} = 0. 
    \]
    Thus $ \alpha (t) \cdot \alpha^{\prime} (t) = 0 $, and $ \alpha (t) $ and $ \alpha^{\prime} (t) $ are orthogonal. Conversely, suppose $ \alpha (t) $ and $ \alpha^{\prime} (t) $ are orthogonal for all $ t \in I $, so $ \alpha (t) \cdot \alpha^{\prime} (t) = 0 $. Then, we have
    \[
        \frac{\mathrm{d}}{\mathrm{d}t}\vert \alpha(t) \vert^{2} = 2 \alpha (t) \cdot \alpha^{\prime} (t) = 0.
    \]
    Thus $ \vert \alpha (t) \vert $ is a constant.
\end{solution}

\medskip

% Problem 2
\begin{problem}{1.3.2}
    A circular disk of radius 1 in the plane $xy$ rolls without slipping along the $x$-axis. The figure described by a point on the circumference of the disk is called a \textbf{cycloid} (Figure 1-7).

\begin{enumerate}[a.]
    \item Obtain a parametrized curve $\alpha \colon \mathbb{R} \to \mathbb{R}^2$ the trace of which is the cycloid, and determine its singular points.
    \item Compute the arc length of the cycloid corresponding to a complete rotation of the disk.
\end{enumerate}
\end{problem}

\begin{solution}
    ~
    \begin{enumerate}[a.]
        \item Let \( \alpha(t) = (x(t), y(t)) \) be the parametrized curve of the cycloid. As the disk rolls without slipping and the radius of the disk is 1, the distance traveled along the $ x $-axis is $ t $ and the $ y $-coordinate is given by the height of the point on the circumference. Therefore, we have:
        \begin{equation}
            \begin{split}
                x(t) &= t - \sin(t), \\
                y(t) &= 1 - \cos(t).
            \end{split}
        \end{equation}
        The singular points occur when $ \alpha ^{\prime} (t) = 0 $. This is equivalent to 
        \begin{equation}
            \begin{split}
                x^{\prime} (t) &= 1 - \cos(t) = 0, \\
                y^{\prime} (t) &= \sin(t) = 0.
            \end{split}
        \end{equation} 
        Hence the singular points are at $ t = 2n \pi $ for all $ n \in \mathbb{Z} $.  
        \item The arc length of the cycloid for a complete rotation is given by integrating over $ [0, 2\pi ] $. 
        \begin{equation}
            \begin{split}
                L &= \int_0^{2\pi} \mathrm{d} t\, |\alpha'(t)| = \int_0^{2\pi} \mathrm{d} t\, \sqrt{(1 - \cos(t))^2 + (\sin(t))^2} \\
                &= \int_0^{2\pi} \mathrm{d} t\, \sqrt{2 - 2\cos(t)} = 8.
            \end{split}
        \end{equation}
    \end{enumerate}
\end{solution}

\medskip

% Problem 3
\begin{problem}{1.3.4}
 Let \( \alpha: (0, \pi) \to \mathbb{R}^2 \) be given by
\[
\alpha(t) = \left( \sin t, \, \cos t + \log \tan \frac{t}{2} \right),
\]
where \( t \) is the angle that the \( y \)-axis makes with the vector \( \alpha(t) \). The trace of \( \alpha \) is called the \textit{tractrix} (see Fig. 1-9). Show that:

\begin{enumerate}[a.]
    \item \( \alpha \) is a differentiable parametrized curve, regular except at \( t = \pi/2 \).
    
    \item The length of the segment of the tangent of the tractrix between the point of tangency and the \( y \)-axis is constantly equal to \( 1 \).
\end{enumerate}
\end{problem}

\begin{solution}
    Recall that a \textbf{regular}  curve is a smooth, parametrized curve with a non-vanishing derivative.
    \begin{enumerate}[a.]
        \item First we shall compute the derivative of $ \alpha (t) $ as 
        \begin{equation}
            \begin{split}
                \alpha^{\prime} (t) &= \left( \cos t, \, -\sin t + \frac{1}{\sin t} \right) \\
                &= \left(\cos t, \cot t \cos t\right)
            \end{split}
        \end{equation}
        Since $ \alpha^{\prime} (t) $ is continuous on \( (0, \pi) \) and \( \alpha^{\prime} (t) \neq 0 \) for all \( t \in (0, \pi) \setminus \{ \pi/2 \} \), \( \alpha(t) \) is a differentiable parametrized curve, regular except at \( t = \pi/2 \).
        \item The equation of the tangent line at \( \alpha(t) \) is given by
        \begin{equation}
            y - y_{0} (t) = \cot t \left( x - x_{0} (t) \right),
        \end{equation}
        where $ y_{0} = \cos t + \log \tan \frac{t}{2}$ and $ x_{0} = \sin t $. Setting \( x = 0 \) to find the intersection with the \( y \)-axis, we have
        \begin{equation}    
            \begin{split}
                \Delta y &\equiv y - y_{0} (t) = -\cot t \sin t = - \cos t, \\
                \Delta x &\equiv x - x_{0} (t) = -\sin t.
            \end{split}
        \end{equation}
        Then the distance is $ \sqrt{(\Delta y)^{2} + (\Delta x)^{2} } = 1 $.
    \end{enumerate}
\end{solution}

\medskip

% Problem 4
\begin{problem}{1.3.8}
    Let $\alpha:I\rightarrow\mathbb{R}^3$ be a differentiable curve and let $[a, b]\subseteq I$ be a closed interval. For every partition
    \[
        a=t_0<t_1<\cdots <t_n =b
    \]
    of $[a, b]$, consider the sum $\sum_{i=1}^n |\alpha(t_i)-\alpha(t_{i-1})|=l(\alpha, P)$, where $P$ stands for the given partition. The norm $|P|$ of a partition $P$ is define as 
    $$
        |P|=\max (t_{i}-t_{i-1}), \, i=1,\dots, n
    $$
    Geometrically, $l(\alpha, P)$ is the length of the polygon inscribed in $\alpha([a, b])$ with the vertices in $\alpha(t_i)$. The point of the exercise is to show that the arc length of $\alpha([a, b])$ is, in some sense, a limit of the length of the inscribed polygons. 
    Prove that given $\epsilon>0$ there exists $\delta>0$ such that if $|P|<\delta$ then 
    \[
        \left|\int_a^b |\alpha'(t)| \mathrm{d} t - l(\alpha, P)\right|<\epsilon
    \]
\end{problem}

\begin{solution}
    Since $ \alpha (t) $ is differentiable on the closed interval $ [a, b] $, $ \alpha^{\prime} (t) $ is continuous. Thus, for any $ \epsilon^{\prime} > 0 $ there exists $ \delta^{\prime} > 0 $ such that $ \alpha^{\prime} (t_2) - \alpha^{\prime} (t_1) < \epsilon $ whenever $ \vert t_2 - t_1 \vert < \delta^{\prime} $.
    For a partition $ P $, let $ \epsilon^{\prime} > 0 $. The integral can be bounded as:
    \begin{equation}
        \begin{split}
            \left| \int_a^b \mathrm{d} t \, |\alpha'(t)| - l(\alpha, P) \right| &= \left| \int_a^b \mathrm{d} t\, |\alpha'(t)| - \sum_{i=1}^n |\alpha(t_i) - \alpha(t_{i-1})| \right| \\
            &\leq \sum_{i=1}^n \left| \int_{t_{i-1}}^{t_i}\mathrm{d} t |\alpha'(t)| - |\alpha(t_i) - \alpha(t_{i-1})| \right| \\
            &\leq \sum_{i=1}^n \int_{t_{i-1}}^{t_i} \mathrm{d} t\, \left| |\alpha'(t)| - \frac{|\alpha(t_i) - \alpha(t_{i-1})|}{t_i - t_{i-1}} \right| \\
            &\leq \sum_{i=1}^{n} \int^{t_i}_{t_{i-1}} \mathrm{d}t\, \left\vert \alpha^{\prime} (t) - \alpha^{\prime} (\xi) \right\vert \\
            &< n (b-a) \epsilon^{\prime} . 
        \end{split}
    \end{equation}
    whenever $ t - \xi < \vert P \vert < \operatorname{min}_{i \in \{1, \dots, n\}} \left(\delta^{\prime}_{i}\right) $. We have used the Mean Value Theorem to obtain $ \xi $. Now, let $ \epsilon^{\prime} = \epsilon / n(b-a) $, $ \delta = \delta^{\prime} $, then for any partition $ P $ with $ |P| < \delta $, we have
    \begin{equation}
        \left| \int_a^b \mathrm{d} t \, |\alpha'(t)| - l(\alpha, P) \right| < \epsilon.
    \end{equation}
\end{solution}

\medskip

% Problem 5
\begin{problem}[Exercise 1.4.11]
    \begin{enumerate}[a.]
        \item Show that the volume $V$ of a parallelepiped generated by three linearly independent vectors $u, v, w\in \mathbb{R}^3$ is given by $V=|(u\land v)\cdot w|$, and introduce an oriented volume in $\mathbb{R}^3$.
        \item Prove that 
        \[
            V^2=\begin{vmatrix}
                u \cdot u & u \cdot v & u \cdot w\\
                v \cdot u &v \cdot v& v \cdot w\\
                w \cdot u & w \cdot v& w \cdot w
            \end{vmatrix} \tag{1}
        \]
    \end{enumerate}
\end{problem}

\begin{solution}
    ~
    \begin{enumerate}[a.]
        \item By definition, the volume of the parallelepiped is given by the area of the base times the height. The area of the base formed by $ u $ and $ v $ is given by $ |u \land v| $, and the height is given by the projection of $ w $ onto the normal vector of the base, which is $ \frac{(u \land v)}{|u \land v|} $. Therefore, the volume $ V $ is given by
        \begin{equation}
            V = |u \land v| \cdot \left| w \cdot \frac{(u \land v)}{|u \land v|} \right| = |(u \land v) \cdot w|.
        \end{equation}
        The oriented volume can be introduced as $ V = (u \land v) \cdot w $. If the vectors $ u, v, w $ (in order) form a right-handed system, the oriented volume is positive; otherwise, it is negative.
        \item Recall that the vector product $ u \land v \in \mathbb{R}$ is the unique vector where $ (u \land v) \cdot w = \operatorname{det} (u, v, w)$. By (a), the volume of the parallelepiped is given by
        \begin{equation}
            V = |(u \land v) \cdot w|.
        \end{equation} 
        Then,
        \begin{equation}
            \begin{split}
                V^2 &= ((u \land v) \cdot w) ((u \land v) \cdot w) \\
                &= \operatorname{det} (u, v, w)^{2} \\
                &= \begin{vmatrix}
                    u_1 & u_2 & u_3 \\
                    v_1 & v_2 & v_3 \\
                    w_1 & w_2 & w_3
                \end{vmatrix}
                \begin{vmatrix}
                    u_1 & u_2 & u_3 \\
                    v_1 & v_2 & v_3 \\
                    w_1 & w_2 & w_3
                \end{vmatrix} \\
                &= \left\vert
                \begin{pmatrix}
                    u_1 & u_2 & u_3 \\
                    v_1 & v_2 & v_3 \\
                    w_1 & w_2 & w_3
                \end{pmatrix}
                \begin{pmatrix}
                    u_1 & v_1 & w_1 \\
                    u_2 & v_2 & w_2 \\
                    u_3 & v_3 & w_3
                \end{pmatrix}
                \right\vert \\
                &= \begin{vmatrix}
                    u \cdot u & u \cdot v & u \cdot w\\
                    v \cdot u &v \cdot v& v \cdot w\\
                    w \cdot u & w \cdot v& w \cdot w
                \end{vmatrix}.
            \end{split}
        \end{equation}
    \end{enumerate}
\end{solution}

\end{CJK}
\end{document}