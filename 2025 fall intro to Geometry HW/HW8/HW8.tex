\documentclass[a4paper]{article}
%% Formatting %%
\usepackage[margin=3cm]{geometry}
\usepackage{type1cm, titlesec, fancyhdr, titling}
\usepackage{multicol}
\usepackage[dvipsnames]{xcolor}
\usepackage{ulem}
\usepackage{parskip}
\setlength{\parindent}{2em}
\setlength{\headheight}{15pt}
\setlength{\droptitle}{-1.5cm}
\parindent=24pt
%% Math and Symbols %%
\usepackage{amsmath,amsthm,amssymb, mathtools}
\usepackage{yhmath, faktor, dsfont}
\usepackage{academicons, wasysym, marvosym}
\usepackage[scr]{rsfso} 
\usepackage{latexsym, amsmath, amscd, amsmath, amsthm}
\usepackage{amssymb,amsmath,amsthm,graphicx,dsfont}
\usepackage{hyperref}

%% Enhancement %%
\usepackage{graphicx, tabularx}
\usepackage[shortlabels,inline]{enumitem}
%% TikZ %%
\usepackage{tikz-cd}
\usepackage[breakable]{tcolorbox}
\usetikzlibrary{decorations.pathmorphing}
\usetikzlibrary{calc, arrows,matrix}

%% Other packages %%
\usepackage{amsopn}

%% Traditional Chinese %%
\usepackage{CJKutf8}

%% Math environments %%
\newtheoremstyle{mystyle}
  {6pt}{15pt}% 上下間距
  {}%          內文字體
  {}%              縮排
  {\bf}%       標頭字體
  {.}%       標頭後標點
  {1em}% 內文與標頭距離
  {}% Theorem head spec (can be left empty, meaning 'normal')
\theoremstyle{mystyle}	
\newtheorem{theorem}{Theorem}
\newtheorem{definition}{Definition}
\newtheorem{example}[theorem]{Example}
\newtheorem{exercise}{Exercise}
\newtheorem{solution}{Solution}
\newtheorem{corollary}[theorem]{Corollary}
\newtheorem{property}[theorem]{Property}
\newtheorem{proposition}[theorem]{Proposition}
\newtheorem{lemma}{Lemma}
\newtheorem{problem}[theorem]{Problem}
\newtheorem{answer}{Answer}[section]
\newtheorem{fact}[theorem]{fact}
\newtheorem*{remark}{Remark}
\newtheorem*{claim}{Claim}
\newtheorem*{observation}{Observation}

\newcommand{\bvec}[1]{\mathbf{#1}} % vector

\begin{document}
\begin{CJK}{UTF8}{bkai}

\title{%
  \textbf{2025 Fall Introduction to Geometry} \\
  \vspace{0.5cm}
  \large 
  Homework 8 (Due Nov 14, 2025)\\
}
\author{物理三 黃紹凱 B12202004}

\maketitle


\begin{definition}[Do Carmo 3.2.5, line of curvature]
    If a regular connected curve $ C \subseteq S $ is such that for all $ p \in S $ the tangent line of $ C $ is a principal direction at $ p $, then $ C $ is said to be a \emph{line of curvature} of $ S $.
\end{definition}

\begin{definition}[Do Carmo 3.2.9, asymptotic curve]
    Let $ p \in S $. An \emph{asymptotic direction} of $ S $ at $ p $ is a direction in $ T_p(S) $ for which the normal curvature is zero. An \emph{asymptotic curve} of $ S $ is a regular connected curve $ C \subseteq S $ such that for each $ p \in S $ the tangent line of $ C $ at $ p $ is an asymptotic direction.
\end{definition}

% Exercise 1
\begin{exercise}[Do Carmo 3.3.5, Enneper's Surface]
    Consider the parametrized surface (Enneper’s surface)
    \[
    \mathbf{x}(u,v) = \left(u - \frac{u^3}{3} + uv^2,\; v - \frac{v^3}{3} + vu^2,\; u^2 - v^2 \right)
    \]
    and show that
    \begin{enumerate}[label=\textbf{\alph*.}]
        \item The coefficients of the first fundamental form are
        \[
        E = G = (1 + u^2 + v^2)^2, \quad F = 0.
        \]
        \item The coefficients of the second fundamental form are
        \[
        e = 2, \quad g = -2, \quad f = 0.
        \]
        \item The principal curvatures are
        \[
        k_1 = \frac{2}{(1 + u^2 + v^2)^2}, \qquad 
        k_2 = -\frac{2}{(1 + u^2 + v^2)^2}.
        \]
        \item The lines of curvature are the coordinate curves.
        \item The asymptotic curves are $u+v=\text{const.}$ and $u-v=\text{const.}$
    \end{enumerate}
\end{exercise}

\begin{solution}
    ~

    \begin{enumerate}[label=\textbf{\alph*.}]
        \item Calculate the first-order partial derivatives: 
        \[
            \bvec{x}_u = \left(1 - u^2 + v^2,\, 2uv,\, 2u\right), \quad \bvec{x}_v = \left(2uv,\, 1 - v^2 + u^2,\, -2v\right).
        \]
        Then the coefficients of the first fundamental form are
        \begin{align*}
            E &= \langle \bvec{x}_u, \bvec{x}_u \rangle = (1 - u^2 + v^2)^2 + 4u^2 v^2 + 4u^2 = (1 + u^2 + v^2)^2, \\
            F &= \langle \bvec{x}_u, \bvec{x}_v \rangle = 2uv(1-u^2+v^2) + 2uv(1+u^2-v^2) - 4uv = 0, \\
            G &= \langle \bvec{x}_v, \bvec{x}_v \rangle = 4u^2 v^2 + (1 + u^2 - v^2)^2 + 4v^2 = (1 + u^2 + v^2)^2.
        \end{align*}

        \item Calculate the second-order partial derivatives:
        \[
            \bvec{x}_{uu} = \left(-2u,\, 2v,\, 2\right), \quad \bvec{x}_{uv} = \left(2v,\, 2u,\, 0\right), \quad \bvec{x}_{vv} = \left(2u,\, -2v,\, -2\right).
        \]
        Next, we find the normal vector: 
        \[
            \bvec{x}_u \wedge \bvec{x}_v = \left( -2u (1 + r^2),\, 2v (1 + r^2),\, 1-r^4 \right), \quad \text{where } r^2 = u^2 + v^2,
        \]
        \[
            \left\vert \bvec{x}_u \wedge \bvec{x}_v \right\vert = (1+r^2)^2 .
        \]
        Therefore,    
        \[
            N = \frac{\bvec{x}_u \wedge \bvec{x}_v}{\left\vert \bvec{x}_u \wedge \bvec{x}_v \right\vert} = \frac{1}{(1 + u^2 + v^2)} \left(-2u,\, 2v,\, 1 - u^2 - v^2\right).
        \]
        The coefficients of the second fundamental form are given by the following inner products:
        \begin{align*}
            e &= \langle N, \bvec{x}_{uu} \rangle = \frac{1}{(1 + u^2 + v^2)} \left( 4u^2 + 4v^2 + 2(1 - u^2 - v^2)\right) = 2, \\
            f &= \langle N, \bvec{x}_{uv} \rangle = \frac{1}{(1 + u^2 + v^2)} \left( -4uv + 4uv + 0\right) = 0, \\
            g &= \langle N, \bvec{x}_{vv} \rangle = \frac{1}{(1 + u^2 + v^2)} \left( -4u^2 - 4v^2 - 2(1 - u^2 - v^2) \right) = -2.
        \end{align*}

        \item The shape operator in the $ (u,v) $ basis is given by $ S = \operatorname{I}^{-1} \operatorname{II} $, where  
        \[
            \operatorname{I} = \begin{pmatrix}
                E & F \\
                F & G
            \end{pmatrix} = \begin{pmatrix}
                (1 + u^2 + v^2)^2 & 0 \\
                0 & (1 + u^2 + v^2)^2
            \end{pmatrix},
        \]
        and
        \[
            \operatorname{II} = \begin{pmatrix}
                e & f \\
                f & g
            \end{pmatrix} = \begin{pmatrix}
                2 & 0 \\
                0 & -2
            \end{pmatrix}.
        \]
        Thus,
        \[
            S = \operatorname{I}^{-1} \operatorname{II} = \frac{1}{(1 + u^2 + v^2)^2} \begin{pmatrix}
                1 & 0 \\
                0 & 1
            \end{pmatrix} \begin{pmatrix}
                2 & 0 \\
                0 & -2
            \end{pmatrix} = \frac{1}{(1 + u^2 + v^2)^2} \begin{pmatrix}
                2 & 0 \\
                0 & -2
            \end{pmatrix}.
        \]
        The principal curvatures are the eigenvalues of the shape operator, which are easily seen to be
        \[
            k_1 = \frac{2}{(1+u^2 + v^2)^2}, \quad k_2 = -\frac{2}{(1 + u^2 + v^2)^2}.
        \]
        \item The lines of curvature correspond to the eigenvectors of the shape operator, which are $ \partial_u $ and $ \partial_v $. Since the shape operator is diagonal in the $ (\bvec{x}_u, \bvec{x}_v) $ basis, the lines of curvature are the coordinate curves $ u = \text{const.} $ and $ v = \text{const.} $.
        
        \item For each $ p $ on an asymptotic curve, the normal curvature in the direction of the tangent vector is zero. The normal curvature $ k_n $ in the direction of a unit tangent vector $ \bvec{t} = a \bvec{x}_u + b \bvec{x}_v $ is given by
        \[
            k_n = \langle S(\bvec{t}), \bvec{t} \rangle = \frac{2}{(1 + u^2 + v^2)^2} ((\mathrm{d}u)^2 - (\mathrm{d}v)^2).
        \]
        Setting $ k_n = 0 $ gives $ (\mathrm{d}u)^2 = (\mathrm{d}v)^2 $, which implies $ \mathrm{d}u = \pm \mathrm{d}v $. Therefore, the asymptotic directions correspond to the curves where $ u + v = \text{const.} $ and $ u - v = \text{const.} $

        From Do Carmo, the normal curvature is given by 
        \[
            k_n = k \langle n, N \rangle,
        \]
    \end{enumerate}
\end{solution}

% Exercise 2
\begin{exercise}[Do Carmo 3.3.8, Contact of Order $\ge 2$ of Surfaces]
    Two surfaces $S$ and $\bar{S}$, with a common point $p$, have \emph{contact of order $\ge 2$ at $p$} if there exist parametrizations $\mathbf{x}(u,v)$ and $\bar{\mathbf{x}}(u,v)$ in $p$ of $S$ and $\bar{S}$, respectively, such that
    \[
    \mathbf{x}_u = \bar{\mathbf{x}}_u, \quad
    \mathbf{x}_v = \bar{\mathbf{x}}_v, \quad
    \mathbf{x}_{uu} = \bar{\mathbf{x}}_{uu}, \quad
    \mathbf{x}_{uv} = \bar{\mathbf{x}}_{uv}, \quad
    \mathbf{x}_{vv} = \bar{\mathbf{x}}_{vv}.
    \]
    \begin{enumerate}
        \item[\textbf{a.}] Let $S$ and $\bar{S}$ have contact of order $\ge 2$ at $p$; $\mathbf{x}\colon U \to S$ and $\bar{\mathbf{x}}\colon U \to \bar{S}$ be arbitrary parametrizations in $p$ of $S$ and $\bar{S}$ respectively; and $f\colon V \subset \mathbb{R}^3 \to \mathbb{R}$ be a differentiable function in a neighborhood $V$ of $p$ in $\mathbb{R}^3$. Then the partial derivatives of order $\le 2$ of $f \circ \bar{\mathbf{x}}\colon U \to \mathbb{R}$ are zero in $\bar{\mathbf{x}}^{-1}(p)$ if and only if the partial derivatives of order $\le 2$ of $f \circ \mathbf{x}\colon U \to \mathbb{R}$ are zero in $\mathbf{x}^{-1}(p)$.

        \item[\textbf{*b.}] Let $S$ and $\bar{S}$ have contact of order $\ge 2$ at $p$. Let $z = f(x,y)$ and $z = \bar{f}(x,y)$ be the equations, in a neighborhood of $p$, of $S$ and $\bar{S}$, respectively, where the $xy$-plane is the common tangent plane at $p=(0,0)$. Then the function $f(x,y) - \bar{f}(x,y)$ has all partial derivatives of order $\le 2$ equal to zero at $(0,0)$.
        
        \item[\textbf{c.}] Let $p$ be a point in a surface $S \subset \mathbb{R}^3$. Let $Oxyz$ be a Cartesian coordinate system for $\mathbb{R}^3$ such that $O=p$ and the $xy$-plane is the tangent plane of $S$ at $p$. Show that the paraboloid
        \[
        z = \tfrac{1}{2}(x^2 f_{xx} + 2xy f_{xy} + y^2 f_{yy}),
        \]
        obtained by neglecting third- and higher-order terms in the Taylor development around $p=(0,0)$, has contact of order $\ge 2$ at $p$ with $S$ (the surface $(*)$ is called the \emph{osculating paraboloid} of $S$ at $p$).

        \item[\textbf{*d.}] If a paraboloid (the degenerate cases of plane and parabolic cylinder are included) has contact of order $\ge 2$ with a surface $S$ at $p$, then it is the osculating paraboloid of $S$ at $p$.

        \item[\textbf{*e.}] If two surfaces have contact of order $\ge 2$ at $p$, then the osculating paraboloids of $S$ and $\bar{S}$ at $p$ coincide. Conclude that the Gaussian and mean curvatures of $S$ and $\bar{S}$ at $p$ are equal.

        \item[\textbf{*f.}] The notion of contact of order $\ge 2$ is invariant by diffeomorphisms of $\mathbb{R}^3$; that is, if $S$ and $\bar{S}$ have contact of order $\ge 2$ at $p$ and $\varphi\colon \mathbb{R}^3 \to \mathbb{R}^3$ is a diffeomorphism, then $\varphi(S)$ and $\varphi(\bar{S})$ have contact of order $\ge 2$ at $\varphi(p)$.

        \item[\textbf{*g.}] If $S$ and $\bar{S}$ have contact of order $\ge 2$ at $p$, then
        \[
        \lim_{r \to 0} \frac{d}{r^2} = 0,
        \]
        where $d$ is the length of the segment cut by the surfaces in a straight line normal to $T_p(S) = T_p(\bar{S})$, which is at a distance $r$ from $p$.
    \end{enumerate}
\end{exercise}

\begin{solution}
    ~ 

    \begin{enumerate}[label=\textbf{\alph*.}]
        \item Suppose the partial derivatives of order $\le 2$ of $f \circ \bar{\mathbf{x}}$ are zero in $\bar{\mathbf{x}}^{-1}(p)$. Then, by the chain rule, we have
        \[
            (f \circ \bar{\mathbf{x}})_u = \nabla f \cdot \bar{\mathbf{x}}_u = 0, \quad (f \circ \bar{\mathbf{x}})_v = \nabla f \cdot \bar{\mathbf{x}}_v = 0,
        \]
        \[
            (f \circ \bar{\mathbf{x}})_{uu} = \nabla f \cdot \bar{\mathbf{x}}_{uu} + \bar{\mathbf{x}}_u^T H_f \bar{\mathbf{x}}_u = 0,
        \]
        \[
            (f \circ \bar{\mathbf{x}})_{uv} = \nabla f \cdot \bar{\mathbf{x}}_{uv} + \bar{\mathbf{x}}_u^T H_f \bar{\mathbf{x}}_v = 0,
        \]
        \[
            (f \circ \bar{\mathbf{x}})_{vv} = \nabla f \cdot \bar{\mathbf{x}}_{vv} + \bar{\mathbf{x}}_v^T H_f \bar{\mathbf{x}}_v = 0,
        \]
        where $H_f$ is the Hessian matrix of $f$ at $p$. Since $S$ and $\bar{S}$ have contact of order $\ge 2$ at $p$, in the region $ \bvec{x}^{-1}(p) $ we have $ (f \circ \mathbf{x})_{uu} = \nabla f \cdot \bvec{x}_{uu} + \bvec{x}_u^T H_f \bvec{x}_u = \nabla f \cdot \overline{\bvec{x}}_{uu} + \overline{\bvec{x}}_u^T H_f \overline{\bvec{x}}_u = 0 $. Similarly, $ (f \circ \mathbf{x})_{uv} = (f \circ \mathbf{x})_{vv} = (f \circ \bvec{x})_u = (f \circ \bvec{x})_v = 0 $. The converse follows by symmetry.

        \item Since $ S $, $ \overline{S} $ have $ z=0 $ as the common tangent plane, their graph at $ p=0 $ satisfy $ f(0,0) = \overline{f} (0,0) = 0 $ and $ \nabla f(0,0) = \nabla \overline{f} (0,0) = 0 $. Let's define the function $ F: \mathbb{R}^3 \to \mathbb{R} $, such that $ F(x,y,z) = z - \frac{1}{2}f_{xx}(0,0) x^2 - f_{xy}(0,0)xy - \frac{1}{2}f_{yy}(0,0) y^2 $. Since $ F $ is a polynomial of $ x, y, x $, it is differentiable. The parametrizations $ \bvec{x} $, $ \overline{\bvec{x}} $ for $ S $ and $ \overline{S} $ at $ p $ are given by $ \bvec{x}(x,y) = \left(x, y, f(x,y)\right) $ and $ \overline{\bvec{x}} (x,y) = \left(x, y, \overline{f}(x,y)\right) $, respectively. Then, $ (F \circ \bvec{x}) (x,y) = f(x,y) - \frac{1}{2} f_{xx}(0,0) x^2 - f_{xy} (0,0) xy - \frac{1}{2}f_{yy}(0,0) y^2 $, so all the partial derivatives of order $ \le 2 $ of $ F \circ \bvec{x} $ at $ (0,0) $ are zero. By part \textbf{a.}, all the partial derivatives of order $ \le 2 $ of $ F \circ \overline{\bvec{x}} $ at $ (0,0) $ are also zero. Therefore, 
        \[
            F \circ \overline{\bvec{x}} (x,y) = \overline{f}(x,y) - \frac{1}{2} f_{xx}(0,0) x^2 - f_{xy} (0,0) xy - \frac{1}{2}f_{yy}(0,0) y^2
        \] 
        has all partial derivatives of order $ \le 2 $ vanish at $ p $. Thus, the function $ f(x,y) - \overline{f}(x,y) $ has all partial derivatives of order $ \le 2 $ vanish at $ p $.

        \item In a neighborhood of $ p $, the surface $ S $ can be expressed as the graph of a function $ z = f(x,y) $, where the $ xy $-plane is the tangent plane at $ p $. Since the $ xy $-plane is the tangent plane at $ p $, we have $ f(0,0) = f_x(0,0) = f_y(0,0) = 0 $, so the Taylor expansion of $ f(x,y) $ around $ p $ is given by
        \[
            f(x,y) = \frac{1}{2} \left( f_{xx}(0,0) x^2 + 2 f_{xy}(0,0) xy + f_{yy}(0,0) y^2 \right) + R_3(x,y).
        \]
        Let $ \overline{S} $ be the paraboloid defined by
        \[
            z = g(x,y) = \frac{1}{2} \left( f_{xx}(0,0) x^2 + 2 f_{xy}(0,0) xy + f_{yy}(0,0) y^2 \right).
        \]
        The parametrizations for $ S $ and $ \overline{S} $ at $ p $ are given by $ \bvec{x}(x,y) = (x, y, f(x,y)) $ and $ \overline{\bvec{x}}(x,y) = (x, y, g(x,y)) $, respectively. The second-order partial derivatives of $ f $ and $ g $ at $ p $ are equal, since the remainder term $ R_3(x,y) $ contains only terms of order $ \ge 3 $. Therefore, by definition, $ S $ and $ \overline{S} $ have contact of order $ \ge 2 $ at $ p $.

        \item Suppose a paraboloid $ \overline{S} $ has contact of order $ \ge 2 $ with a surface $ S $ at $ p $. Let the equation of $ S $ in a neighborhood of $ p $ be given by $ z = f(x,y) $, where the $ xy $-plane is the tangent plane at $ p $. The equation of the paraboloid $ \overline{S} $ can be expressed as
        \[
            z = \overline{f}(x,y) = a x^2 + 2b xy + c y^2,
        \]
        for some constants $ a, b, c \in \mathbb{R} $. The second-order Taylor expansion of $ f(x,y) $ around $ p $ is given by
        \[
            f(x,y) = \frac{1}{2} \left( f_{xx}(0,0) x^2 + 2 f_{xy}(0,0) xy + f_{yy}(0,0) y^2 \right).
        \]
        Comparing this with the expression for $ \overline{f}(x,y) $, we find that
        \[
            a = \frac{1}{2} f_{xx}(0,0), \quad b = \frac{1}{2} f_{xy}(0,0), \quad c = \frac{1}{2} f_{yy}(0,0).
        \]
        Thus, the paraboloid $ \overline{S} $ is the osculating paraboloid of $ S $ at $ p $ as defined in \textbf{c.}.

        \item Let $ P $, $ \overline{P} $ be the osculating paraboloids of $ S $ and $ \overline{S} $, respectively. By \textbf{b.}, $ S $, $ \overline{S} $ have contact of order $ \geq 2 $ at $ p $ with $ P $, $ \overline{P} $, respectively. Since $ S $ also has contact of order $ \geq 2 $ with $ \overline{S} $, all the partial derivatvies of order $ \leq 2 $ of $ f $ and $ \overline{f} $ vanish at $ p $, where $ f $, $ \overline{f} $ are the equations in a neighborhood of $ p $, of $ S $ and $ \overline{S} $, respectively. Therefore, 
        \[
            \frac{1}{2} \left(f_{xx}(p)x^2 + 2f_{xy}(p)xy + f_{yy}(p)y^2 \right) = \frac{1}{2} \left(\overline{f}_{xx}(p)x^2 + 2\overline{f}_{xy}(p)xy + \overline{f}_{yy}(p)y^2 \right),
        \] 
        and the osculating paraboloids $ P $ and $ \overline{P} $ coincide. Since the Gaussian and mean curvatures depend only on the partial derivatives of order $ \leq 2 $ of the parametrization at $ p $, the Gaussian and mean curvatures of $ S $ and $ \overline{S} $ at $ p $ are equal.

        \item Suppose $ S $ and $ \overline{S} $ have contact of order $ \geq 2 $ at $ p $. Let $ \varphi: \mathbb{R}^3 \to \mathbb{R}^3 $ be a diffeomorphism. The parametrizations for $ S $ and $ \overline{S} $ at $ p $ are given by $ \bvec{x}(u,v) $ and $ \overline{\bvec{x}}(u,v) $, respectively. The parametrizations for $ \varphi(S) $ and $ \varphi(\overline{S}) $ at $ \varphi(p) $ are given by $ \bvec{y} = \left(\varphi \circ \bvec{x}\right)(u,v) $ and $ \overline{\bvec{y}} = \left( \varphi \circ \overline{\bvec{x}}\right) (u,v) $, respectively. Then, by the chain rule, we have
        \[
            \bvec{y}_u = \mathrm{d}\varphi_{\bvec{x}} \cdot \bvec{x}_u, \quad \bvec{y}_v = \mathrm{d}\varphi_{\bvec{x}} \cdot \bvec{x}_v, \quad \bvec{y}_{uu} = \mathrm{d}^2 \varphi_{\bvec{x}} (\bvec{x}_u, \bvec{x}_u) + \mathrm{d}\varphi_{\bvec{x}} \cdot \bvec{x}_{uu},
        \]
        \[
            \bvec{y}_{uv} = \mathrm{d}^2 \varphi |_{\bvec{x}} (\bvec{x}_u, \bvec{x}_v) + \mathrm{d}\varphi |_{\bvec{x}} \cdot \bvec{x}_{uv}, \quad 
            \bvec{y}_{vv} = \mathrm{d}^2 \varphi |_{\bvec{x}} (\bvec{x}_v, \bvec{x}_v) + \mathrm{d}\varphi |_{\bvec{x}} \cdot \bvec{x}_{vv},
        \]
        and similarly for $ \overline{\bvec{y}} $, where $ \mathrm{d}^2 \phi |_{\bvec{x}} $ is the bilinear differential of $ \phi $ evaluated at $ \bvec{x} $. 
        
        Since $ S $ and $ \overline{S} $ have contact of order $ \geq 2 $ at $ p $, it follows that $ \bvec{y}_u = \overline{\bvec{y}}_u $, $ \bvec{y}_v = \overline{\bvec{y}}_v $, $ \bvec{y}_{uu} = \overline{\bvec{y}}_{uu} $, $ \bvec{y}_{uv} = \overline{\bvec{y}}_{uv} $, and $ \bvec{y}_{vv} = \overline{\bvec{y}}_{vv} $. Thus, $ \varphi(S) $ and $ \varphi(\overline{S}) $ have contact of order $ \geq 2 $ at $ \varphi(p) $.

        \item We may choose a Cartesian coordinate system $ Oxyz $ such that $ O = p $, and $ z=0 $ is the common tangent plane of $ S $ and $ \overline{S} $ at $ p $. Let the equations of $ S $ and $ \overline{S} $ in a neighborhood of $ p $ be given by $ z = f(x,y) $ and $ z = \overline{f}(x,y) $, respectively. Since $ S $ and $ \overline{S} $ have contact of order $ \geq 2 $ at $ p $, by part \textbf{b.}, all the partial derivatives of order $ \leq 2 $ of the function $ G(x,y) \equiv f(x,y) - \overline{f}(x,y) $ vanish at $ p $. Therefore, $ G(0,0) = \nabla G (0,0) = \nabla^2 G(0,0) = 0 $, where $ \nabla^2 G $ is the Hessian matrix of $ G $. Take a point $ q = (x,y,0) \in T_p (S) $ in the tangent plane, a distance $ r = \sqrt{x^2 + y^2} $ from $ p $. The straight line $ L_q $ normal to the tangent plane passing through $ q $ intersects the surfaces $ S $ and $ \overline{S} $ at the points $ (x,y,f(x,y)) $ and $ (x,y,\overline{f}(x,y)) $, respectively, and $ d = \vert f(x,y) - \overline{f} (x,y) \vert = \vert G(x,y) \vert $. 
        
        Define the function $ g(t) = G(tu) $ for a fixed $ u $, where $ u \in \mathbb{R}^2 $ is a unit vector such that $ (x,y) = ru $. Then $ g $ is differentiable, and $ g(0) = g'(0) = g''(0) = 0 $, since all the partial derivatives of order $ \leq 2 $ of $ F $ vanish at $ p $. By Taylor's formula with remainder, we have 
        \[
            g(t) = g(0) + g^{\prime} (0) + \int_0^t \mathrm{d}s\, (t-s) g^{\prime\prime} (s) = \int_0^t \mathrm{d}s\, (t-s) g^{\prime\prime} (s)
        \]
        for all $ t $ in a neighborhood of $ 0 $. Next we will bound $ \vert g \vert $. Since $ F $ is smooth, $ \nabla^2 F $ is continuous, so for all $ \varepsilon > 0 $ there exists $ \delta > 0 $, such that $ \Vert (x,y) \Vert < \delta $ implies $ \Vert \nabla^2 F (x,y) \Vert < 2 \varepsilon $. Hence, for $ t < \delta $, $ \vert g^{\prime\prime} (t) \vert = \vert u^T \nabla^2 F u \vert \leq \vert \nabla^2 F \vert \lVert u^2 \rVert < 2\varepsilon $. Take $ t=r<\delta $, then we have 
        \begin{align*}
            \vert G(ru) \vert &= \vert g(r) \vert = \left\vert \int_0^r \mathrm{d}s\, (r-s) g^{\prime\prime} (s) \right\vert \leq \int_0^r \mathrm{d}s\, (r-s) \vert g^{\prime\prime} (s) \vert \\
            &\leq \int_0^r \mathrm{d}s\, (r-s) 2 \varepsilon r^2 = \varepsilon r^2.
        \end{align*}
        Notice that $ d = G(x,y) = G(ru) $, so for all $ \varepsilon>0 $ there exists $ \delta > 0 $ such that $ \frac{d}{r^2} < \varepsilon $ whenever $ \sqrt{x^2 + y^2} < \delta $. This proves the desired result.
    \end{enumerate}
\end{solution}

% Exercise 3
\begin{exercise}[Do Carmo 3.3.13]
    Let $F\colon \mathbb{R}^3 \to \mathbb{R}^3$ be the map (a similarity) defined by $F(p)=c p$, $p \in \mathbb{R}^3$, $c$ a positive constant. Let $S \subset \mathbb{R}^3$ be a regular surface and set $\bar{S} = F(S)$. Show that $\bar{S}$ is a regular surface, and find formulas relating the Gaussian and mean curvatures, $K$ and $H$, of $S$ with the Gaussian and mean curvatures, $\bar{K}$ and $\bar{H}$, of $\bar{S}$.
\end{exercise}

\begin{solution}
    ~ 

    \begin{enumerate}
        \item Let $ \bvec{x}: U \subseteq \mathbb{R} \to S $ be a local parametrization of $ S $. Let $ \overline{S} = F(S) $, then $ \overline{\bvec{x}} = F \circ \bvec{x}: U \to \overline{S} $ is a local parametrization of $ \overline{S} $. The map $ F $ is smooth, and since $ \mathrm{d}F = c \operatorname{Id} $ is an isomorphism, $ \mathrm{d}\overline{\bvec{x}} = \mathrm{d}F \circ \mathrm{d}\bvec{x} = c \mathrm{d}\bvec{x} $ has rank $ 2 $ whenever $ \mathrm{d}\bvec{x} $ has rank $ 2 $. Thus, $ \overline{\bvec{x}} $ is a homeomorphism onto its image and $ \mathrm{d}\overline{x} $ is injective (hence an immersion). Therefore, $ \overline{S} $ is a regular surface.
        \item For any local parametrization $ \bvec{x} $ and $ \overline{\bvec{x}} $, we have $ \overline{\bvec{x}} = c \bvec{x} $. Thus, 
        \[
            \overline{\bvec{x}}_u = c \bvec{x}_u, \quad \overline{\bvec{x}}_v = c \bvec{x}_v, \quad \overline{\bvec{x}} \wedge \overline{\bvec{x}}_v = c^2 (\bvec{x}_u \wedge \bvec{x}_v).
        \]
        Hence, the normal for $ \overline{S} $ satisfies $ \overline{N} = N $. Write the Weingarten map for $ S $ and $ \overline{S} $ as $ \mathcal{S} $ and $ \overline{\mathcal{S}} $, respectively. By definition, $ \mathrm{d}N = - \mathcal{S} \circ \mathrm{d}\bvec{x} $, so 
        \[
            \mathrm{d}\overline{N} = \mathrm{d}N = - \mathcal{S} \circ \mathrm{d}\bvec{x} = - \mathcal{S} \circ \frac{1}{c}\, \mathrm{d}\overline{\bvec{x}} = - \left(\frac{1}{c} \mathcal{S}\right) \circ \mathrm{d}\overline{\bvec{x}}.
        \]
        Therefore, $ \overline{\mathcal{S}} = \frac{1}{c} \mathcal{S} $, and the principle curvatures satisfy $ \overline{k}_i = \frac{1}{c} k_i $, since they are the eigenvalues of $ \mathcal{S} $. The Gaussian curvature $ K $ and mean curvature $ H $ of $ S $ are then given by
        \begin{align*}
            \overline{K} = \overline{k}_1 \overline{k}_2 = \frac{1}{c^2} k_1 k_2 = \frac{1}{c^2} K, \\
            \overline{H} = \frac{\overline{k}_1 + \overline{k}_2}{2} = \frac{1}{c} \frac{k_1 + k_2}{2} = \frac{1}{c} H.
        \end{align*}
    \end{enumerate}
\end{solution}

% Exercise 4
\begin{exercise}[Do Carmo 3.3.24, Local Convexity and Curvature]
    ~

    A surface $S \subset \mathbb{R}^3$ is \emph{locally convex} at a point $p \in S$ if there exists a neighborhood $V \subset S$ of $p$ such that $V$ is contained in one of the closed half-spaces determined by $T_p(S)$ in $\mathbb{R}^3$. If, in addition, $V$ has only one common point with $T_p(S)$, then $S$ is called \emph{strictly locally convex} at $p$.
    \begin{enumerate}
        \item[\textbf{a.}] Prove that $S$ is strictly locally convex at $p$ if the principal curvatures of $S$ at $p$ are nonzero with the same sign (that is, the Gaussian curvature $K(p)$ satisfies $K(p) > 0$).
        
        \item[\textbf{b.}] Prove that if $S$ is locally convex at $p$, then the principal curvatures at $p$ do not have different signs (thus, $K(p) \ge 0$).
        
        \item[\textbf{c.}] To show that $K \ge 0$ does not imply local convexity, consider the surface
        \[
        f(x,y) = x^3(1 + y^2),
        \]
        defined in the open set $U = \{ (x,y) \in \mathbb{R}^2 : y^2 < \tfrac{1}{2} \}$. Show that the Gaussian curvature of this surface is nonnegative on $U$ and yet the surface is not locally convex at $(0,0) \in U$ (a deep theorem, due to R.~Sacksteder, implies that such an example cannot be extended to the entire $\mathbb{R}^2$ if we insist on keeping the curvature nonnegative; cf.\ Remark 3 of Sec.~5-6).

        \item[\textbf{*d.}] The example of part (c) is also very special in the following local sense. Let $p$ be a point in a surface $S$, and assume that there exists a neighborhood $V \subset S$ of $p$ such that the principal curvatures on $V$ do not have different signs (this does not happen in the example of part c). Prove that $S$ is locally convex at $p$.
    \end{enumerate}
\end{exercise}

\begin{solution}
    ~
    
    \begin{enumerate}[label=\textbf{\alph*.}]
        \item Without loss of generality, assume $ k_1, k_2 > 0 $, since if both are negative, just replace the chosen unit normal by its negative. Let $ \bvec{x}: U \subseteq \mathbb{R}^2 \to S \subseteq \mathbb{R}^3 $ be a local parametrization of $ S $ such that $ \{\bvec{x}_u, \bvec{x}_v\} $ is an \emph{orthonormal basis of principle directions at $ p \in S $}, where $ p = \bvec{x}(0,0) $. Following the definition of Exercise 3.3.22, define the \emph{height function} $ h: U \to \mathbb{R} $ of $ S $ relative to $ T_p(S) $ by
        \[
            h(u,v) = \langle \bvec{x}(u,v) - p, N(p) \rangle,
        \]
        where $ N(p) $ is the unit normal vector $ p $. We compute the derivatives as follows: 
        \begin{align*}
            h(p) &= \langle \bvec{x}(0,0) - p, N(p) \rangle = 0, \\
            h_u(p) &= \langle \bvec{x}_u(0,0), N(p) \rangle = 0, \\
            h_v(p) &= \langle \bvec{x}_v(0,0), N(p) \rangle = 0, \\
            h_{uu}(p) &= \langle \bvec{x}_{uu}(0,0), N(p) \rangle = e(p), \\
            h_{uv}(p) &= \langle \bvec{x}_{uv}(0,0), N(p) \rangle = f(p), \\
            h_{vv}(p) &= \langle \bvec{x}_{vv}(0,0), N(p) \rangle = g(p),
        \end{align*}
        where $ h_{ij}(p) $ are the coefficients of the second fundamental form at $ p $. Since $ \bvec{x}_u (0,0) $ and $ \bvec{x}_v (0,0) $ are principle directions and orthonormal, we have $ e(p) = k_1 $, $ f(p) = 0 $, and $ g(p) = k_2 $. Thus, the Hessian matrix of $ h $ at $ p $ is given by
        \[
            \nabla^2 h (p) = \begin{pmatrix}
                h_{uu}(p) & h_{uv}(p) \\
                h_{uv}(p) & h_{vv}(p)
            \end{pmatrix} = \begin{pmatrix}
                k_1 & 0 \\
                0 & k_2
            \end{pmatrix},
        \]
        and Taylor expansion gives 
        \[
            h(u,v) = \frac{1}{2} \left(k_1 u^2 + k_2 v^2 \right) + o\left(u^2 + v^2\right),
        \]
        Since $ k_1, k_2 > 0 $, the quadratic form $ Q = \frac{1}{2}\left( k_1 u^2 + k_2 v^2 \right) $ associated with $ \nabla^2 h (p) $ is positive definite. Hence, there exists a neighborhood $ W \subset U $ of $ p $ and some $ c > 0 $ such that $ Q(u,v) > c (u^2 + v^2) $ for all $ (u,v) \in W $. Now since 
        \[
            \frac{h(u,v) - Q(u,v)}{u^2 + v^2} \to 0 \quad \text{as } (u,v) \to (0,0),
        \]
        there exists a radius $ \delta>0 $ such that $ \sqrt{u^2 + v^2} < \delta $ implies $ \vert h(u,v) - Q(u,v) \vert < \frac{c}{2} (u^2 + v^2) $. Therefore, for all $ (u,v) \in W $ with $ \sqrt{u^2 + v^2} < \delta $, we have
        \[
            h(u,v) \geq Q(u,v) - \vert h(u,v) - Q(u,v) \vert > c (u^2 + v^2) - \frac{c}{2} (u^2 + v^2) = \frac{c}{2} (u^2 + v^2) > 0, 
        \]
        with $ h(u,v) = 0 $ if and only if $ (u,v) = (0,0) $. Thus, the neighborhood $ V = \bvec{x} (W \cap \{(u,v) : \sqrt{u^2 + v^2} < \delta \}) $ of $ p $ is contained in the half-space $ H^+ = \{ q \in \mathbb{R}^3 \mid \langle q - p, N(p) \rangle \geq 0 \} $, and $ V $ has only one common point with $ T_p(S) $. Therefore, $ S $ is strictly locally convex at $ p $. 

        \item Suppose $ S $ is locally convex at $ p $, so there exists a neighborhood $ V \subset S $ of $ p $ such that $ V $ is contained in one of the closed half-spaces determined by $ T_p(S) $. Define the height function as above, by local convexity we may choose an orientation $ N(p) $ such that $ h(u,v) \geq 0 $ in a neighborhood of $ (0,0) $, and $ h(0,0) = h_u (0,0) = h_v (0,0) = 0 $. Suppose that the principal curvatures at $ p $ have different signs, say $ k_1 > 0 > k_2 $. Then, along the coordinate axes, we have $ h(u,0) = \frac{1}{2}k_1 u^2 > 0 $ for all $ \vert u \vert < \delta_u $, and $ h(0,v) = \frac{1}{2}k_2 v^2 < 0 $ for all $ \vert v \vert < \delta_v $. Hence, in every neighborhood of $ (0,0) $, we can find points such that $ h(u,v) > 0 $ and others such that $ h(u,v) < 0 $, contradicting local convexity. Therefore, the principal curvatures at $ p $ do not have different signs, and hence $ K(p) \geq 0 $.
        
        \item The Gaussian curvature $ K $ of the surface defined by $ z = f(x,y) $ is given by
        \[
            K = \frac{f_{xx} f_{yy} - f_{xy}^2}{(1 + f_x^2 + f_y^2)^2}.
        \]
        Let's compute the necessary partial derivatives of $ f(x,y) = x^3(1 + y^2) $:
        \[
            f_x = 3x^2(1 + y^2), \quad f_y = 2x^3 y, \quad 
            f_{xx} = 6x(1 + y^2), \quad f_{yy} = 2x^3, \quad f_{xy} = 6x^2 y.
        \]
        Then, we have 
        \[
            K = \frac{(6x(1 + y^2))(2x^3) - (6x^2 y)^2}{(1 + (3x^2(1 + y^2))^2 + (2x^3 y)^2)^2} = \frac{12x^4(1 - 2y^2)}{(1 + 9x^4(1 + y^2)^2 + 4x^6 y^2)^2} \geq 0.
        \]
        However, the surface is not locally convex at $ (0,0) $, since for any neighborhood $ V $ of $ (0,0) $, there exist points with both positive and negative $ x $ values, and hence $ z $-coordinates, so $ V $ is not contained in one of the closed half-spaces determined by the tangent plane at $ (0,0) $. 

        \item Suppose $ V \subseteq S $ is a neighborhood of $ p $ such that the principal curvatures on $ V $ do not have different signs. Without loss of generality, assume $ k_1 (q), k_2 (q) \geq 0 $ for all $ q \in V $, since if at some point one of them were positive and later negative, it would have to cross zero alone, producing a point where the two have different signs, which is excluded by definition of $ V $. Follow the steps of \textbf{a.}, we define the height function $ h: U \to \mathbb{R} $ of $ S $ relative to $ T_p(S) $ by $ h(u,v) = \langle \bvec{x}(u,v) - p, N(p) \rangle $. Pick an orthonormal basis of principal directions $ \{\bvec{x}_u, \bvec{x}_v\} $. The Hessian matrix of $ h $ at $ p $ is given, again, by
        \[
            \nabla^2 h (p) = \begin{pmatrix}
                k_1 & 0 \\
                0 & k_2
            \end{pmatrix}.
        \]
        Near $ (0,0) $, we have 
        \[
            h(u,v) = \frac{1}{2} \left( k_1 u^2 + k_2 v^2 \right) + o\left(u^2 + v^2\right),
        \] 
        and the quadratic form $ Q = \frac{1}{2} \left(k_1 u^2 + k_2 v^2\right) $ is positive-definite. Now we consider two cases: 
        \begin{enumerate}
            \item At least one of the principal curvatures at $ p $ is positive, say $ k_1 > 0 $. Then, there exists a neighborhood $ W \subset U $ of $ p $ and some $ c > 0 $ such that $ Q(u,v) > c (u^2 + v^2) $ for all $ (u,v) \in W $. Following the same steps as in \textbf{a.}, we can show local convexity at $ p $.
            \item Both principal curvatures at $ p $ are zero, i.e., $ k_1 = k_2 = 0 $, so $ Q = 0 $. Since the principal curvatures are continuous functions on $ S $, we have $ h(0,0) = 0 $ and $ h(u,v) \geq 0 $ in a neighborhood of $ p $. Therefore, $ S $ is locally convex at $ p $.
        \end{enumerate}
    \end{enumerate}
\end{solution}

\end{CJK}
\end{document}