\documentclass[12pt]{article}

% Language setting
% Replace `english' with e.g. `spanish' to change the document language
\usepackage[english]{babel}

% Set page size and margins
% Replace `letterpaper' with `a4paper' for UK/EU standard size
\usepackage[letterpaper,top=2cm,bottom=2cm,left=3cm,right=3cm,marginparwidth=1.75cm]{geometry}

% Useful packages
\usepackage{amsmath}
\usepackage{graphicx}
\usepackage{mathtools}
\usepackage{amsfonts}
\usepackage{algorithm}
\usepackage{algorithmicx}
\usepackage[noend]{algpseudocode}
\usepackage[colorlinks=true, allcolors=blue]{hyperref}
\usepackage{bm}
\usepackage{amssymb}
\usepackage{esint}
\usepackage{tikz}
\usepackage{hyperref}
\usepackage{verbatim}
\DeclareMathOperator{\ord}{ord}
\DeclareMathOperator{\chr}{char}
\DeclareMathOperator{\spn}{span}
\DeclareMathOperator{\Img}{Im}
\DeclareMathOperator{\Syl}{Syl}
\DeclareMathOperator{\Conj}{Conj}
\DeclareMathOperator{\Aut}{Aut}
\DeclareMathOperator{\id}{id}
\DeclareMathOperator{\sign}{sign}
\DeclareMathOperator{\Inn}{Inn}
\DeclareMathOperator{\Out}{Out}
\DeclareMathOperator{\Frac}{Frac}
\DeclareMathOperator{\pre}{pre}
\DeclareMathOperator{\Gal}{Gal}
\DeclareMathOperator{\Orb}{Orb}
\DeclareMathOperator{\Stab}{Stab}
\DeclareMathOperator{\disc}{disc}
\setcounter{MaxMatrixCols}{20}
\def\acts{\curvearrowright}
\usetikzlibrary{tikzmark}
\newcommand{\surj}[0]{\xrightarrow[]{}\mathrel{\mkern-14mu}\rightarrow}
\newcommand{\inj}[0]{\xhookrightarrow{}}
\newcommand{\tikzarc}[1]{%
\tikzmarknode{a}{#1}
\begin{tikzpicture}[overlay,remember picture]
\draw ([yshift=1pt]a.north west) to[bend left=20] ([yshift=1pt]a.north east);
\end{tikzpicture}%
}

\title{Introduction to Algebra (II)}
\author{Cheng-Yun Yeh}

\begin{document}
\maketitle

\section*{Chapter 9. Fields (III)}
\subsection*{9.1 Splitting field revisited}
\indent

\textbf{Lemma 1. }Let $F_1,F_2$ be fields, $\phi:F_1 \overset{\sim}{\longrightarrow} F_2$ be an isomorphism, and $f_1(x) \in F[x]$, $f_2(x)=\phi(f_1(x)) \in F_2[x]$. ($\phi$ can be extended to $\phi:F_1[x] \overset{\sim}{\longrightarrow} F_2[x]$, $\phi:\sum_{i=0}^n a_ix^i \mapsto \sum_{i=0}^n \phi(a_i)x^i$) Suppose that $f_1$ is irreducible, if $\alpha_i$ is a root of $f_i$ in some field extension $K/F_i$ ($i=1,2$), then there is a unique isomorphism $\psi:F_1(\alpha_1) \overset{\sim}{\longrightarrow} F_2(\alpha_2)$ satisfying:

\[
    (\star)\left\{\begin{matrix}
        \psi(c)=\phi(c)&,&\forall c \in F_1 \\
        \psi(\alpha_1)=\alpha_2
    \end{matrix}\right.
\]

That is:

\begin{center}
    \includegraphics[height=0.23\textwidth, width=0.4\textwidth]{Iso.png}
\end{center}


pf. Step 1. First, $f_1$ is irreducible and $\phi$ is an isomorphism give us thee fact that $f_2$ is irreducible. Now, consider the evaluation map $F_1[x] \to F_1(\alpha_1)$, since $f(\alpha_1)=0$ by assumption, we have $f(x) \in \ker(ev_\alpha)$ and thus $(f_1) \subseteq \ker(ev_\alpha) \subseteq F_1[x]$. Since $f_1$ is irreducible, we have $(f_1)$ is maximal and thus $\ker(ev_\alpha)=(f_1)$ ($g(x)=x \notin \ker(ev_\alpha)$, so $\ker(ev_\alpha) \ne F_1[x]$). Then, by the first isomorphism theorem, $F_1[x]/(f_1)=F_1[x]/\ker(ev_\alpha) \simeq \Img(ev_\alpha)=F_1[\alpha]=F_1(\alpha)$ by the isomorphism $\Tilde{ev}_\alpha(g+(f_1))=g(\alpha)$. By the similar statements, we have $F_2[x]/(f_2) \simeq F_2(\alpha_2)$. \\

Step 2. On the other hand, we can define $\Tilde{\phi}:F_1[x]/(f_1) \to F_2[x]/(f_2)$, $\Tilde{\phi}:g+(f_1) \mapsto \phi(g)+(f_2)$. In this step, our goall is to show that $\Tilde{\phi}$ is an isomorphism. \\

First, we show that $\Tilde{\phi}$ is a well-defined homomorphism. Given $g_1,g_2$ s.t. $g_1-g_2=hf_1$ for some $h \in F_1[x]$, then:

\begin{align*}
    \Tilde{\phi}(g_1+(f_1))&=\phi(g_1)+(f_2) \\
    &=\phi(g_2+hf_1)+(f_2) \\
    &=\phi(g_2)+\phi(h)f_2+(f_2) \\
    &=\phi(g_2)+(f_2) \\
    &=\Tilde{\phi}(g_2+(f_1))
\end{align*}


Thus, it is well-defined. Moreover, given $g_1,g_2 \in F_1[x]$, we have 

\begin{align*}
    \Tilde{\phi}(g_1+g_2+(f_1))&=\phi(g_1+g_2)+(f_2) \\
    &=\phi(g_1)+\phi(g_2)+(f_2) \\
    &=(\phi(g_1)+(f_2))+(\phi(g_2)+(f_2)) \\
    &=\Tilde{\phi}(g_1+(f_1))+\Tilde{\phi}(g_2+(f_1))
\end{align*}


\begin{align*}
    \Tilde{\phi}(g_1g_2+(f_1))&=\phi(g_1g_2)+(f_2) \\
    &=\phi(g_1)\phi(g_2)+(f_2) \\
    &=(\phi(g_1)+(f_2))(\phi(g_2)+(f_2)) \\
    &=\Tilde{\phi}(g_1+(f_1))\Tilde{\phi}(g_2+(f_1))
\end{align*}

Thus, it is a homomorphsim. Moreover, the surjectivity is trivial since $\phi$ is an isomorphism between $F_1[x]$ and $F_2[x]$. For injectivity, given $g+(f_1) \in \ker\Tilde{\phi}$, then $\Tilde{\phi}(g+(f_1))=\phi(g)+(f_2)=(f_2)$, which means $\phi(g)=hf_2$ for some $h \in F_2[x]$ and thus $g=\phi^{-1}(h)\phi^{-1}(f_2)=\phi^{-1}(h)f_1 \in (f_1)$, i.e. $g+(f_1)=(f_1)$, the zero element in $F_1[x]/(f_1)$. In conclusion, we have $F_1[x]/(f_1) \simeq F_2[x]/(f_2)$. \\

Step 3. From step 1, 2, we have $F_1(\alpha_1) \simeq F_1[x]/(f_1) \simeq F_2[x]/(f_2) \simeq F_2(\alpha_2)$, i.e. $\exists \psi:F_1(\alpha_1) \overset{\sim}{\longrightarrow} F_2(\alpha_2)$. Now we show that $\forall c \in F_1$, $\psi(c)=\phi(c)$ and $\psi(\alpha_1)=\alpha_2$. \\

Follow the path: $\psi:F_1(\alpha_1) \to F_1[x]/(f_1) \to F_2[x]/(f_2) \to F_2(\alpha_2)$, note that $\phi(x)=\phi(1_{F_1})x=x \in F_2[x]$, we have $\alpha_1 \mapsto x+(f_1) \mapsto x+(f_2) \mapsto \alpha_2$. Moreover, given $c \in F$, we have $c\mapsto x-\alpha_1+c+(f_1) \mapsto x-\phi(\alpha_1)+\phi(c)+(f_2) \mapsto \alpha_2-\alpha_2+\phi(c)+(f_2)=\phi(c)+(f_2)$. \\

Step 4. For uniqueness, suppose $\psi':F_1(\alpha_1) \overset{\sim}{\longrightarrow} F_2(\alpha_2)$ satisfies $(\star)$, since $F_1(\alpha)=F_1[\alpha]=\{\sum_{i=0}^n a_i\alpha_1^i:a_i \in F_1\}$, given $\beta \in F_1(\alpha_1)$, we can write it as $\sum_{i=0}^n b_i\alpha_1^i$ and we have $\psi'(\beta)=\sum_{i=0}^n\psi'(b_i)\psi'(\alpha_1)^i=\sum_{i=0}^n\psi(b_i)\psi(\alpha_1)^i=\psi(\beta)$. Thus, $\psi'=\psi$, i.e. the isomorphism is unique. \\ \\

\textbf{Theorem 2. }Let $F_1,F_2$ be a fields, $f(x) \in F_1[x] \setminus \{0\}$, and $\phi:F_1\overset{\sim}{\longrightarrow} F_2$ be an isomorphism of fields. Let $f_2(x)=\phi(f_1)(x)$ and $K_i$ be a splitting field for $f_i$ over $F_i$ ($i=1,2$), then there is an isomorphism $\psi:K_1 \overset{\sim}{\longrightarrow} K_2$ satisfying $\psi(c)=\phi(c)$, $\forall c \in F_1$. \\

pf. We prove by induction on $d=\deg f_1=\deg f_2$. If $d=0,1$, then $F_1=K_1$ and $F_2=K_2$, take $\psi=\phi$, we're done. When $d \ge 2$, we first let $g_1$ be an irreducible factor of $f_1$ over $F_1$ and $g_2=\phi(g_1)$, then $f_1=g_1h_1 \in F_1[x]$, which means $K_1$ contains all roots of $g_1$ since $K_1$ contains all roots of $f_1$. Then, we can find $\alpha_1 \in K_1$ s.t. $g_1(\alpha_1)=0$. Similarly, we can find $\alpha_2 \in K_2$ s.t. $g_2(\alpha_2)=0$. \\

By lemma 1, $\exists! \psi:F_1(\alpha_1) \overset{\sim}{\longrightarrow} F_2(\alpha_2)$ s.t. $\psi_1 \vert_{F_1}=\phi:F_1 \overset{\sim}{\longrightarrow} F_2$ and $\psi_1(\alpha_1)=\alpha_2$. Since $\alpha_1,\alpha_2$ are roots of $f_1,f_2$, respectively, we have $f_1(x)=(x-\alpha_1)q_1(x)$ and $f_2(x)=(x-\alpha_2)q_2(x)$ for some $q_1 \in F_1[x]$ and $q_2 \in F_2[x]$. Then by $\psi_1(x-\alpha_1)=\psi_1(x)-\psi_1(\alpha_1)=x-\alpha_2$, we have $\psi_1(q_1)=q_2$ by $\psi_1(f_1)=\phi(f_1)=f_2$. (Note that $\psi_1,\phi$ map a polynomial by mapping their coefficients.) \\

Now, $\psi_1:F_1(\alpha_1) \overset{\sim}{\longrightarrow} F_2(\alpha_2)$ and $\psi_1(q_1)=q_2$. To use induction hypothesis, we should show that $K_1,K_2$ are splitting fields for $q_1,q_2$ over $F_1(\alpha_1),F_2(\alpha_2)$, respectively. Since $K_1$ is a splitting for $f_1$ and $\alpha_1$ is a root of $f_1$, we have $f_1(x)=c(x-\alpha_1)\prod_{i=1}^{d-1}(x-\gamma_i)$ for some $\gamma_i \in K_1$. Also, $f_1(x)=(x-\alpha_1)q_1(x)$, we have $q_1(x)=c\prod_{i=1}^{d-1}(x-\gamma_i)$, i.e. $q_1$ splits completely in $K_1$. Moreover, suppose there is a field $E$ s.t. $F_1(\alpha) \subseteq E \subsetneq K_1$ and $q_1$ splits completely in it, i.e. $\gamma_1,...,\gamma_{d-1} \in E$. Then, since $\alpha \in F_1 \subseteq E$, all the roots  of $f_1$ is in $E$, which contradicts to the fact that $K_1$ is a splitting field for $f_1$ over $F_1$. Hence, $K_1$ is a splitting field for $q_1$ over $F_1(\alpha_1)$, similar for $K_2$ for $q_2$ over $F_2(\alpha_2)$. \\

By induction hypothesis, there exists an isomorphism $\psi:K_1 \overset{\sim}{\longrightarrow} K_2$ s.t. $\psi(\alpha_1)=\alpha_2$ and $\psi \vert _{F_1}=\phi$, we're done. \\ \\

\textbf{Corollary 3. }Let $F$ be a field and $f \in F[x] \setminus \{0\}$, then all splitting fields for $f$ over $F$ are isomorphic. \\

pf. Take $F_1=F=F_2$ and $\phi$ be the identity map in theorem 2, we're done.


\subsection*{9.2 Finite field revisited}
\indent

\textbf{Recall. }Let $F$ be a finite field, then:

\begin{itemize}
    \item[] (i) $\chr F=p$ for some prime $p$

    \item[] (ii) $\mathbb{F}_p \xhookrightarrow{} F$

    \item[] (iii) See $F$ as a $\mathbb{F}_p$ vector space, if $[F:\mathbb{F}_p]=n$, then $\#F=p^n$. \\
\end{itemize}

\textbf{Theorem 4. }Let $p$ be a prime and $n \in \mathbb{Z}_{>0}$, then:

\begin{itemize}
    \item[] (i) There is a finite field $F$ with exactly $p^n$ elements. (Actually, its the splitting field for $x^{p^n}-x$ over $\mathbb{F}_p$.

    \item[] (ii) All fields with $p^n$ elements are isomorphic.
\end{itemize}

pf-1. Consider $f(x)=x^{p^n}-x \in \mathbb{F}_p[x]$, let $K$ be a splitting field for $f(x)$ over $\mathbb{F}_p$, then $K$ contains all roots of $f(x)$. Let $F=\{\alpha \in K:f(\alpha)=0\}$, we're going to show that $F$ is a field. \\

First, $f(0)=f(1)=0$, so $0,1 \in F$. Given $\alpha,\beta \in F$, then $\alpha^{p^n}=\alpha$ and $\beta^{p^n}=\beta$. So:

\begin{gather*}
    f(\alpha+\beta)=(\alpha+\beta)^{p^n}-(\alpha+\beta)=\alpha^{p^n}+\beta^{p^n}-\alpha-\beta=0 \\
    f(\alpha\beta)=(\alpha\beta)^{p^n}-\alpha\beta=\alpha^{p^n}\beta^{p^n}-\alpha\beta=0 \\
    p=2,\ f(-\alpha)=(-\alpha)^{p^n}-(-\alpha)=\alpha^{p^n}-\alpha=0 \\
    p\text{ is odd},\ f(-\alpha)=(-\alpha)^{p^n}-(-\alpha)=-\alpha^{p^n}+\alpha=0 \\
    f(\alpha^{-1})=(\alpha^{-1})^{p^n}-\alpha^{-1}=(\alpha^{p^n})^{-1}-\alpha^{-1}=0
\end{gather*}

Thus, $F$ is a field. On the other hand, $\gcd(f,f')=\gcd(f,1)=1$, so $f$ is separable, which gives us $\#F=p^{\deg f}=p^n$. Additionally, since $F \subseteq K$ and $F$ contains all roots of $f$, we have $F=K$ and thus $F$ is a splitting field for $f$ over $\mathbb{F}_p$.\\

pf-2. Given any finite field $F$ with $\#F=p^n$, then $F^\times=p^n-1$. By Lagrange theorem, for all $\alpha \in F^\times$, we have $\alpha^{p^n-1}=1$, and thus for all $\alpha \in F$, $\alpha^{p^n}-\alpha=0$. That is, $f(x)=x^{p^n}-x$ splits completely in $F$. Moreover, since $f$ is separable and $\#F=p^{\deg f}$, $F$ is exactly the set of the roots of $f$, which means $F$ is a splitting field for $f$ over $\mathbb{F}_p$. By corollary 3, all splitting fields for $f$ over $\mathbb{F}_p$ are isomorphic, we're done. \\ \\

\textbf{Example. }Let $f(x)=x^4+1$, its clear that $f$ is irreducible in $\mathbb{Z}[x]$. Claim: $f$ is reducible in $\mathbb{F}_p$, for all prime $p$. \\

pf. If $p=2$, then $f(x)=(x+1)^4$. If $p$ is an odd prime, then $p^2-1=0$ modulo $8$. Then, $f(x)=(x^4+1) \mid (x^8-1) \mid (x^{p^2-1}-1) \mid (x^{p^n}-x)$. Now suppose $f$ is irreducible in $\mathbb{F}_p[x]$. Let $F$ be the splitting field for $x^{p^n}-x$ over $\mathbb{F}_p$ and $\alpha$ be a root of $x^4+1$, we have $\mathbb{F}_p[x]/(f) \simeq \mathbb{F}_p(\alpha) \subseteq F$. But, $\#F=p^2$ by theorem 4 and $\#\mathbb{F}_p[x]/(f)=p^4$. (Note. $[\mathbb{F}_p[x]/(f):\mathbb{F}_p]=\deg f=4$ and each linear combination of $\{1,x,x^2,x^3\}$ are distinct, so we have $\#\mathbb{F}_p[x]/(f)=p^4$.) It leads to a contradiction. Therefore, $x^4+1$ is reducible in $\mathbb{F}_p$. \\ \\

\textbf{Prop 5. }For any prime $p$ and any $n \in \mathbb{Z}_{>0}$, there is an irreducible polynomial of degree $n$ over $\mathbb{F}_p$. \\

pf. By theorem 4, there exists $F$, a finite field, with $\#F=p^n$. By corollary 10 in chapter 8, we have $F^\times$ is cyclic, say $F^\times=\langle \theta \rangle$ for some $\theta \in F^\times$ s.t. $\ord(\theta)=p^n-1$. Consider $F=\mathbb{F}_p(\theta)$ and $h(x) \in \mathbb{F}_p[x]$ be the minimal polynomial of $\theta$ (i.e. the generator of $\ker(ev_\theta)$). We have proved that $h$ is irreducible and $\deg h=\dim_{\mathbb{F}_p}\mathbb{F}_p[x]/(h)=\dim_{\mathbb{F}_p}F=n$, we're done. \\ \\

\textbf{Prop 6. }The polynomial $f(x)=x^{p^n}-x$ over $\mathbb{F}_p$ is the product of all distinct (monic) irreducible polynomial over $\mathbb{F}_p$ with $\deg d$, where $d$ runs through all divisors of $n$. \\ \\

pf. Step 1. Let $h$ be an irreducible factor of $f$ in $\mathbb{F}_p[x]$. We first show that $\deg h$ must divide $n$. Let $K$ be a splitting field for $f$ over $\mathbb{F}_p$, then $K$ contains all roots of $h$ since $h$ is a factor of $f$. Let $\alpha$ be a root of $h$, then from $ev_\alpha:\mathbb{F}_p[x] \to K$, we have $F_1:=\mathbb{F}_p[x]/(h) \xhookrightarrow{} K$ by the first isomorphism theorem. Since $\mathbb{F}_p \xhookrightarrow{} F_1$ by identity map, we have $K/F_1/\mathbb{F}_p$ is an extension of fields. Since $n=[K:\mathbb{F}_p]$ by theorem 4, we have $n=[K:\mathbb{F}_p]=[K:F_1][F_1:\mathbb{F}_p]=[K:F_1]\deg h$. Thus, $\deg h \mid n$. \\

Step 2. We know that $f$ is separable, it cannot have any irreducible factors with multiplicity $\ge 2$. Thus, it suffices to prove $\forall h(x) \in \mathbb{F}_p[x]$, irreducible, with $\deg h \mid n$, then $h \mid f$. (That is, in step 1, we prove that $h$ is an irreducible factor of $f$ $\Rightarrow$ $\deg h \mid n$. In step 2, we show the converse.) \\

Consider $F_1=\mathbb{F}_p[x]/(h)$, a finite field with $\dim_{\mathbb{F}_p}F_1=\deg h=d \mid n$, then $\#F_1=p^d$ and thus $F_1$ is a splitting field for $g(x)=x^{p^d}-x$ over $\mathbb{F}_p$ by theorem 4. In the splitting field for $f$ over $\mathbb{F}_p$, $K$, let $\theta \in K$ be a generator of $K^\times$ with $\ord(\theta)=p^n-1$. Then, $p^d-1 \mid p^n-1$ since $d \mid n$ by assumption. Let $\theta_1=\theta^{\frac{p^n-1}{p^d-1}} \in K$, then $\ord(\theta_1)=p^d-1$. Since $\theta_1^{p^d-1}=\theta^{p^n-1}=1$ ($\theta \in K$ implies $\theta$ is a root of $f$), $\theta_1$ is a root of $g$ and thus $\{0,\theta_1,...,\theta_1^{p^d-1}\}$ is exactly the set of roots of $g$ (its size is $\deg g=p^d$). Therefore, $g$ splits completely in $K$, then there exists $K_1 \subseteq K$ s.t. $K_1$ is a splitting field for $g$ over $\mathbb{F}_p$. Then, both $F_1$ and $K_1$ are splitting fields for $g$ over $\mathbb{F}_p$, they are isomorphic. \\

Now, $K_1=\{0,\theta_1,...,\theta_1^{p^d-1}\}$, so $\#K_1=p^d=\#F_1$, they are isomorphic. Let $\alpha=x+(h) \in F_1$ be a root of $h$, since $h$ is irreducible in $\mathbb{F}_p$, it is the minimal polynomial for $\alpha$ over $\mathbb{F}_p$. Therefore, by $K_1 \simeq F_1$, $\alpha$ is a root of $g$ and thus $h \mid g$. Then, we get $h \mid f$ since $g \mid f$. \\ \\

\textbf{Example. }Find an irreducible polynomial of degree $2$ over $\mathbb{F}_2$. \\

Sol. Consider $x^{2^2}-x=x^4-x=x(x-1)(x^2+x+1)$, then $x^2+x+1$ is irreducible by proposition 5. \\ \\

\textbf{Example. }Find an irreducible polynomial of degree $3$ over $\mathbb{F}_2$. \\

Sol. Consider $x^{2^3}-x=x^8-x=x(x-1)(x^6+x^5+x^4+x^3+x^2+x+1)$. Observe that if $f \mid x^8-x$, then $\deg f=1$ or $3$ by proposition 6, i.e. $g(x)=x^6+x^5+x^4+x^3+x^2+x+1$ is reducible. Since $g(1)=1$, it does not have any factor of degree $1$ in $\mathbb{F}_2$. By some trick, we have:

\begin{align*}
    x^6+x^5+x^4+x^3+x^2+x+1&=x^6+x^5+x^4+x^3+(x^3+x^3)+x^2+x+1 \\
    &=(x^6+x^5+x^3)+(x^4+x^3+x)+(x^3+x^2+1) \\
    &=x^3(x^3+x^2+1)+x(x^3+x^2+1)+(x^3+x^2+1) \\
    &=(x^3+x+1)(x^3+x^2+1)
\end{align*}

So, $x^3+x+1$ and $x^3+x^2+1$ are irreducible polynomials in $\mathbb{F}_p$. \\ \\

\textbf{Example. }Find the number of irreducible polynomials of degree $5$ over $\mathbb{F}_2$. \\

Sol. Consider $x^{2^5}-x=x^{32}-x=x(x-1)(x^{30}+x^{29}+\cdots+x+1)$. By proposition 6, $x^{30}+x^{29}+\cdots+x+1$ is a product of distinct irreducible polynomials of degree $5$. So, there are $30/5=6$ distinct irreducible polynomials of degree $5$ over $\mathbb{F}_2$.

\subsection*{9.3 Automorphism of field and splitting fields}
\indent

\textbf{Def. }Let $K/F$ be an extension of fields. We say $\Aut(K/F)$ or $\Aut_F(K)$ is the set $\{\sigma \mid \sigma:K \to K,\ \text{$\sigma$ is an isomorphism s.t. }\sigma(c)=c,\ \forall c \in F\}$. \\ \\

\textbf{Prop 7. }Let $K/F$ be an extension of fields, then $\Aut(K/F)$ is a group. \\

pf. The neutral element of $\Aut(K/F)$ is $\id_K:K \to K$, and the group law is the composition of functions (which has associativity). For closure, let $\sigma,\tau \in \Aut(K/F)$, we have $\sigma \cdot \tau$ and $\sigma^{-1}$ are also isomorphisms. For all $c \in F$, $\sigma\tau(c)=\sigma(c)=c$ and $\sigma^{-1}(c)=\sigma^{-1}(\sigma(c))=\id_K(c)=c$. Thus, it has closure of group law. In conclusion, $\Aut(K/F)$ is a group. \\ \\

\textbf{Prop 8. }Let $K/F$ be an extension of fields. Let $f \in F[x] \setminus \{0\}$, $\alpha \in K$ be a root of $f$, and $\sigma \in \Aut(K/F)$. Then $\sigma(\alpha)$ is also a root of $f(x)$. \\

pf. Since $\alpha$ is a root of $f(x)=\sum_{i=0}^na_ix^i$ and $\sigma$ is an isomorphism, we have $f(\sigma(\alpha))=\sum_{i=0}^n a_i\sigma(\alpha)^i=\sigma(\sum_{i=0}^n a_i\alpha^i)=\sigma(0)=0$. Thus, $\sigma(\alpha)$ is a root of $f$. \\ \\

\textbf{Example. (Classification of $\Aut(\mathbb{Q}(\sqrt[4]{2},i)/\mathbb{Q})$) }Let $F=\mathbb{Q}$, $K$ be the splitting field for $x^4-2=(x+\sqrt[4]{2})(x-\sqrt[4]{2})(x+\sqrt[4]{2}i)(x-\sqrt[4]{2}i)$ over $\mathbb{Q}$, i.e. $K=\mathbb{Q}(\sqrt[4]{2},i)$. Claim: $\Aut(K/F) \simeq \mathcal{D}_4$. \\

pf. First, since $\id:\mathbb{Q} \overset{\sim}{\longrightarrow} \mathbb{Q}$, $x^4-2$ is irreducible in $\mathbb{Q}$, $\sqrt[4]{2}$, $\sqrt[4]{2}i$ are roots for $x^4-2$ and $\id(x^4-2)=x^4-2$, by lemma 1, there exists a field isomorphism $\sigma:\mathbb{Q}(\sqrt[4]{2}) \overset{\sim}{\longrightarrow} \mathbb{Q}(\sqrt[4]{2}i)$ s.t. $\sigma(c)=c$ for all $c \in \mathbb{Q}$ and $\sigma(\sqrt[4]{2})=\sqrt[4]{2}i$. \\

Since $i \notin \mathbb{Q}(\sqrt[4]{2})$, $x^2+1$ is irreducible in $\mathbb{Q}(\sqrt[4]{2})[x]$ and thus the minimal polynomial for $i$ over $\mathbb{Q}(\sqrt[4]{2})$ is $\Phi_{\mathbb{Q}(\sqrt[4]{2}),i}=x^2+1$. Again by lemma 1, we can extend $\sigma$ to $\sigma:\mathbb{Q}(\sqrt[4]{2},i) \overset{\sim}{\longrightarrow} \mathbb{Q}(\sqrt[4]{2}i,i)$, then $\sigma \in \Aut(K/F)$, $\sigma(\sqrt[4]{2})=\sqrt[4]{2}i$ and $\sigma(i)=i$.  \\

By the same token, there exists $\tau \in \Aut(K/F)$ s.t. $\tau(\sqrt[4]{2})=\sqrt[4]{2}$ and $\tau(i)=-i$, now we claim: $\Aut(K/F)=\langle \sigma,\tau \rangle$. \\

For $j=0,1,2,3$, we let $\alpha_j=\sqrt[4]{2} \cdot i^j$, i.e. $\alpha_0=\sqrt[4]{2},\ \alpha_1=\sqrt[4]{2}i,\ \alpha_2=-\sqrt[4]{2},\ \alpha_3=\sqrt[4]{2}$. Given $s \in \Aut(K/F)$, by proposition 8, $s(\alpha_0)$ is a root of $x^4-2$, so $s(\alpha_0)$ has only $4$ choices $\alpha_0,\alpha_1,\alpha_2,\alpha_3$. We can easily compute that $\sigma(\alpha_i)=\alpha_{i+1}$ for $i=0,1,2$ and $\sigma(\alpha_3)=\alpha_0$. Then we have $s(\alpha_0)=\alpha_i=\sigma^i(\alpha_0)$ for $i=0,1,2,3$, i.e. $(\sigma^{-1}s)(\alpha_0)=\alpha_0$, which means $\sigma^{-1}s$ fixes $\alpha_0$ and $\alpha_2$. On the other hand, again by proposition 8, $(\sigma^{-1}s)(i)=i$ or $(\sigma^{-1}s)(i)=-i$, i.e. $\sigma^{-1}s=\id$ or $\sigma^{-1}s=\tau$, which tells us $s=\sigma^i$ or $\sigma^i\tau$, $i=0,1,2,3$. Therefore, $\Aut(K/F)=\langle \sigma,\tau \rangle$. \\

Now, we show that $\langle \sigma,\tau \rangle=\mathcal{D}_4$. First, $\sigma^4=\tau^2=\id$. Moreover, since $\sigma^i(\sqrt[4]{2})=\sqrt[4]{2}$ and $\sigma^i(i)$, if $\sigma^i \in \langle \tau \rangle$, then $\sigma^i=\id$, i.e. $\langle \sigma \rangle \cap \langle \tau \rangle=\{\id\}$. Finally, $(\tau\sigma\tau)(i)=(\tau\sigma)(-i)=\tau(-i)=i$, $(\tau\sigma\tau)(\sqrt[4]{2})=(\tau\sigma)(\sqrt[4]{2})=\tau(\sqrt[4]{2}i)=\tau(\sqrt[4]{2})\tau(i)=-\sqrt[4]{2}i=\sigma^3=\sigma^{-1}$. So, in conclusion, $\Aut(K/F) \simeq \mathcal{D}_4$. In particular, $\#\Aut(K/F)=\#\mathcal{D}_4=8=[K:F]$. \\ \\


\textbf{Theorem 9. }Let $F$ be a field. Let $f \in F[x]$ be a separable polynomial, and let $K/F$ be a splitting field for $f$ over $F$. Then, $\#\Aut(K/F)=[K:F]$. \\

pf. Later, we have to use lemma 10. \\ \\

\textbf{Def. }In theorem 9, such a field $K$ is called a \textbf{Galois extension} of $F$. That is, we say $K/F$ is a \textbf{Galois extension} if $K$ is a splitting field of a separable polynomial $f \in F[x]$ over $F$. \\ \\

\textbf{Lemma 10. }Let $F$ be a field, $K/F$ be an extension of fields. Let $\alpha \in K$ be algebraic over $F$ and $\Phi_{F,\alpha}$ be the minimal polynomial with distinct roots $\alpha_1,...,\alpha_r$ in $K$. Then there are exactly $r$ field homomorphism $\sigma:F(\alpha) \xhookrightarrow{} K$ satisfying \textbf{condition ($\star$)}: $\sigma(c),\forall c \in F$. More precisely, for $1 \le i \le r$, there is exactly one such map $\sigma_i$ satisfying $\sigma(\alpha)=\sigma(\alpha_i)$ and ($\star$). \\

pf. Consider the minimal polynomial of $\alpha$ over $F$, say $\Phi_{F,\alpha}$. For any $\sigma:F(\alpha) \xhookrightarrow{}K$ satisfying ($\star$), since $\sigma$ and $\Phi_{F,\alpha}$ have coefficients in $F$, we have $\Phi_{F,\alpha}(\sigma(\alpha))=\sigma(\Phi_{F,\alpha}(\alpha))=\sigma(0)=0$. Thus, $\sigma(\alpha)$ is a root of $\Phi_{F,\alpha}$, then $\sigma$ has only $r$ choices ($\sigma(\alpha)$ can only be $\alpha_1,...,\alpha_r$). \\

Conversely, by lemma 1, since $F\simeq F$, there exists an unique field homormoophism $\sigma_i$ satisfies $\sigma(\alpha)=\alpha_i$ and ($\star$). So, there are at least $r$ field homomorphism satisfying ($\star$). In conclusion, there are exactly $r$ field homomorphism $\sigma(c)=c$, $\forall c \in F$. \\ \\

\textbf{Proof for theorem 9. }Since $K$ is a splitting field for $f$ over $F$, we can write $f(x)=c\prod_{i=1}^d(x-\alpha_i)$ with $\alpha_i \in K$, then $K=F(\alpha_1,...,\alpha_d)$. Now, set $K_0=F$ and $K_i=F(\alpha_1,...,\alpha_i)$, then we have a tower of fields: $F=K_0 \subseteq K_1 \subseteq ... \subseteq K_d=K$. \\

Since $f(\alpha_1)=0$, we have $\Phi_{F,\alpha_1} \mid f$ and thus all roots of $\Phi_{F,\alpha_1}$ are roots of $f$, which means $K$ contains all roots of $\Phi_{F,\alpha_1}$. We have assumed that $\alpha_1,...,\alpha_d$ are distinct ($\because$ $f$ is separable), $\Phi_{F,\alpha_1}$ has $\deg \Phi_{F,\alpha_1}$ distinct roots in $K$. By lemma 10, there are $\deg \Phi_{F,\alpha_1}$ field homormorphism satisfying ($\star$). \\

Claim: $\#\{\sigma_i:K_i \xhookrightarrow{} K \mid \text{$\sigma_i$ is a field homomorphism satisfying ($\star$)}\}=[K_i:F]$. \\

Proof for the claim. We prove by induction on $i$. For $i=1$, we have already proved $\#\{\sigma_1:K_1 \xhookrightarrow{} K \mid \text{$\sigma_1$ is a field homomorphism satisfying ($\star$)}\}=\deg \Phi_{F,\alpha_1}=[F[x]/(\Phi_{F,\alpha_1}):F]=[F(\alpha_1):F]$. Now, suppose the claim holds for $i=n \ge 1$, then for the case $i=n+1 \ge 2$, we want to find how many $\sigma_{n+1}:K_{n+1} \xhookrightarrow{} K$, field homomorphism, satisfying ($\star$). \\

By induction hypothesis, there are $[K_n:F]$ distinct $\sigma_n:K_n \xhookrightarrow{} K$ satisfying ($\star$). Choose any of them, let $\Phi_{K_n,\alpha_{n+1}}$ be the minimal polynomial of $\alpha_{n+1}$ over $K_n$, then $\sigma_n(\Phi_{K_n,\alpha_{n+1}})$ divides $f$ since $\Phi_{K_n,\alpha_{n+1}}$ divides $f$. ($f=\sigma(f)=\sigma(\Phi_{K_n,\alpha_{n+1}} \cdot q)=\sigma(\Phi_{K_n,\alpha_{n+1}})\sigma(q)$ for some $q$) So, $\sigma_n(\Phi_{K_n,\alpha_{n+1}})$ is also separable, and all roots of $\sigma_n(\Phi_{K_n,\alpha_{n+1}})$ are also roots of $f$. \\

Now, if we consider the field isomorphism, by lemma 1, there exists $\sigma_{n+1}:K_n(\alpha_{n+1})=K_{n+1} \overset{\sim}{\longrightarrow} \Img(\sigma_n)(\beta_n)$ s.t. $\sigma_{n+1}(c)=\sigma_n(c),\forall c \in K_n$ and $\sigma_{n+1}(\alpha_{n+1})=\beta_n$, for any $\beta_n$, roots of $\sigma_{n}(\Phi_{K_n,\alpha_{n+1}})$.  By above discussion, $\beta_n$ is also a root of $f$, and there are $\deg(\sigma_n(\Phi_{K_n,\alpha_{n+1}}))$ choices for $\beta_n$. Hence, for this $\sigma_n$, we have $\deg(\sigma_n(\Phi_{K_n,\alpha_{n+1}}))$ ways to extend it to $\sigma_{n+1}:K_{n+1} \xhookrightarrow{} K$ that satisfies ($\star$). \\

On the other hand, $\deg(\sigma_n(\Phi_{K_n,\alpha_{n+1}}))=\deg(\Phi_{K_n,\alpha_{n+1}})=[K_n[x]/(\Phi_{K_n,\alpha_{n+1}}):K_n]=[K_n(\alpha_{n+1}):K_n]=[K_{n+1},K_n]$, so by induction hypothesis, we have:

\begin{align*}
    &\ \#\{\sigma_{n+1}:K_{n+1} \xhookrightarrow{} K \mid \text{$\sigma_i$ is a field homomorphism satisfying ($\star$)}\} \\
    =&\ [K_{n+1}:K_n] \cdot \#\{\sigma_n:K_n \xhookrightarrow{} K \mid \text{$\sigma_i$ is a field homomorphism satisfying ($\star$)}\} \\
    =&\ [K_{n+1}:K_n][K_n:F] \\
    =&\ [K_{n+1}:F]
\end{align*}

By induction, the proof is completed. \\ \\

\textbf{Prop 11. }Let $F$ be a field, $\chr F=0$. Then every splitting field over $F$ is a splitting field of a separable polynomial and hence a Galois extension. \\

pf. Given $f \in F[x]$ and $K/F$ be a splitting field for $f$ over $F$. WLOG, $\deg F \ge 1$ (otherwise, we take $K=F$ and $F$ is a splitting field of constant polynomial, which is separable by definition.), then we can write $f(x)=c\prod_{i=1}^np_i(x)^{e_i}$ with monic, irreducible polynomials $p_i(x)$ in $F[x]$ and some scalar $c \in F$. Clearly, $K$ is also a splitting field for $\prod_{i=1}^n p_i(x)$. Since $\chr F=0$ and each $p_i$ is irreducible, by theorem 15 in chapter 8, $p_i$ is separable. \\

Moreover, if $p_i$ and $p_j$ have the a commom root, say $\alpha$, then $\Phi_{F,\alpha}$ divides both $p_i$ and $p_j$. Since $p_i,p_j$ are monic and irreducible, we have $p_i=p_j=\Phi_{F,\alpha}$, i.e. $i=j$. It tells us if $i \ne j$, then $p_i(x)$ and $p_j(x)$ do not have common roots. Thus, $\prod_{i=1}^np_i(x)$ is separable, which means $K$ is a splitting field for a separable polynomial. \\ \\

\textbf{Prop 12. }Let $K/F$ be a finite extension of fields, then there is a finite set of elements $\alpha_1,...,\alpha_n \in K$ s.t. $K=F(\alpha_1,...,\alpha_n)$. \\

pf. Let $K_0=F$, if $K_0=K$, then we're done. If $K_0 \subsetneq K$, then we can find $\alpha_1 \in K\setminus K_0$ s.t. $K_0 \subsetneq K_1=K_0(\alpha_1)$ and $[K_1:K_0] \ge 2$. If $K_1=K$, then we're done, if not, then we do the same process iteratively until we're done. Note that the process will halt since $[K:F]$ is finite and $[K:F] \ge [K_i:F]=\prod_{j=1}^i[K_j:K_{j-1}] \ge 2^i$. \\ \\

\textbf{Def. }An algebraic extension of fields $K/F$ is said to be \textbf{separable} if for any $\alpha \in K$, the minimal polynomial $\Phi_{F,\alpha}$ is separable. Alternatively, we can define that $K/F$ is \textbf{separable} if every irreducible polynomial $f \in F[x]$ having a root in $K$ is separable. \\ \\

\textbf{Remark. }The two definitions are equivalent. \\

pf ($\Rightarrow$). Given an irreducible polynomial $f \in F[x]$ and $\alpha \in K$ s.t. $f(\alpha)=0$, then since $f=c \cdot \Phi_{F,\alpha}$ for some $c \in F$ and $\Phi_{F,\alpha}$ is separable, we have $f$ is separable. \\

pf ($\Leftarrow$). Trivial, since $\Phi_{F,\alpha}$ is irreducible for all $\alpha \in K$. \\ \\

\textbf{Lemma 13. }Let $F$ be an infinite field and $K/F$ be a separable algebraic extension. Let $\alpha,\beta \in K$, then there is a $\gamma \in F(\alpha,\beta)$ s.t. $F(\alpha,\beta)=F(\gamma)$. \\

pf. Let $\gamma=\alpha+c\beta$ for some $c \in F^\times$ (we will choose a proper $c$ later), then $F(\gamma) \subseteq F(\alpha,\beta)$ since $\gamma \in F(\alpha,\beta)$. Our goal is to show that $\alpha$ (or $\beta$) is in $F(\gamma)$, then $\beta$ (or $\alpha$) is also in $F(\gamma)$ and thus $F(\gamma) \supseteq F(\alpha,\beta)$. \\

Let $f=\Phi_{F,\alpha}$ and $g=\Phi_{F,\beta}$. Since $K$ is separable, $f$ and $g$ are separable with distinct roots $\alpha_1,...,\alpha_n$ and $\beta_1,...,\beta_m$, respectively. WLOG, $\alpha_1=\alpha$ and $\beta_1=\beta$. Since $F$ is infinite, we want to choose $c$ s.t. $\gamma-c\beta_j =\alpha_1+c(\beta-\beta_j)\ne \alpha_i$, i.e. choose a $c$ that is not in the finite set $\left\{\dfrac{\alpha_i-\alpha_1}{\beta-\beta_j}:2 \le i \le n,\ 2 \le j \le m\right\}$. That is, $c \ne \frac{\alpha_i-\alpha_1}{\beta-\beta_j}$, then we have $\alpha_1+c(\beta-\beta_j) \ne \alpha_i$, for all $1 \le i \le n,\ 2 \le j \le m$. (It does not equal $\alpha_1$ since $c \in F^\times=F \setminus \{0\}$) \\

Now we consider $f(\gamma-cx) \in F(\gamma)[x]$, as $\alpha$ is a root of $f(x)$, we have $\beta$ is a root of $f(\gamma-cx)$. Thus, $\beta$ is a common root of $g(x)$ and $f(\gamma-cx)$. Furthermore, for $2 \le j \le m$, $f(\gamma-c\beta_j)=f(\alpha_1+c\beta_1-c\beta_j) \ne 0$, since $\alpha_1+c\beta_1-c\beta_j \ne \alpha_i$ for any $1 \le i \le n$, which means the only common root of $g(x)$ and $f(\gamma-cx)$ is $\beta$. Thus, $\Phi_{F(\gamma),\beta}$ is a common divisor of $g(x)$ and $f(\gamma-cx)$. By $\Phi_{F(\gamma),\beta} \mid g$, $\Phi_{F(\gamma),\beta}(x)$ can only have roots $\beta,\beta_2,...,\beta_m$. But $\beta_2,...,\beta_m$ are not a roots of $f(\gamma-cx)$, which tells us $\beta_2,...,\beta_m$ are not roots of $\Phi_{F(\gamma),\beta}(x)$, either. Therefore, $\Phi_{F(\gamma),\beta}(x)=(x-\beta)^p$ for some $p$. However, $g(x)$ is separable, $p$ can only be $1$. Thus, $\Phi_{F(\gamma),\beta}=x-\beta \in F(\gamma)[x]$, which means $\beta \in F(\gamma)$. Note that $\alpha=\gamma-c\beta$ is also in $F(\gamma)$, which gives us $F(\alpha,\beta) \subseteq F(\gamma)$ and thus $F(\alpha,\beta)=F(\gamma)$. \\ \\

\textbf{Theorem 14. (Primitive element theorem) }Let $K/F$ be a finite, separable extension, then there is an element $\gamma \in K$ s.t. $K=F(\gamma)$. \\ 

pf. Since lemma 13 needs $F$ to be infinite, we have to discuss two cases. \\

Case 1. $F$ is finite. Then since $[K:F]$ is finite, $K$ is also finite. Thus, there exists $\theta \in K^\times$ s.t. $K=\langle \theta \rangle$. Then, $K=F(\theta)$, it proves the claim. \\

Case 2. $F$ is infinite. Since $K/F$ is finite, by proposition 12, there exists $\alpha_1,...,\alpha_n \in K$ s.t. $K=F(\alpha_1,...,\alpha_n)$. Now, we choose a set $\{\alpha_1,...,\alpha_n\}$ s.t. $K=F(\alpha_1,...,\alpha_n)$ and $n$ is the minimum. That is, no set $\{\alpha_1,...,\alpha_{n-1}\}$ with cardinality $<n$ can satisfy $K=F(\alpha_1,...,\alpha_{n-1})$). Claim: $n=1$. Suppose NOT, i.e. $n \ge 2$, by lemma 13, there exists $\gamma \in K$ s.t. $F(\alpha_1,\alpha_2,\alpha_3,...,\alpha_n)=F(\gamma,\alpha_3,...,\alpha_n)$, which contradicts to the fact that $n$ is minimum. Thus, $n=1$, which proves the claim. \\ \\


\textbf{Notation. }Recall that we say $K/F$ is a finite Galois extension if $K$ is a splitting field for some separable polynomial $f \in F[x]$ over $F$. In this case, we denote $\Aut(K/F)$ by $\Gal(K/F)$ or $G(K/F)$. By theorem 9, $\#\Gal(K/F)=[K:F]$. \\ \\

\textbf{Prop 15. }Let $F$ be a field and $K$ be a splitting field of $f \in F[x]$. Let $\alpha_1,...,\alpha_n \in K$ be roots of $f(x)$, then there is a well-defined injective group homomorphism $\Pi:\Aut(K/F) \to \mathcal{S}_n$, $\sigma \mapsto \Pi_\sigma$, with the property $\sigma(\alpha_i)=\alpha_{\Pi_{\sigma}(i)}$, $\forall i=1,...,n$. Note that $\mathcal{S}_n$ is the symmetric group over $\{1,2,...,n\}$. \\

pf. By proposition 8, for all $i$, $\sigma(\alpha_i)$ is also a root of $f(x)$. Thus, there exists a unique $j$ s.t. $\sigma(\alpha_i)=\alpha_j$. Now, we define $\Pi(\sigma)=\Pi_\sigma \in \mathcal{S}_n$ by $\Pi_\sigma(i)=j$. We claim: $\Pi$ is a well-defined group homomorphism. \\

It is equivalent to show that $\sigma(\alpha_i)=\alpha_j$ defines a group action of $\Aut(K/F)$ on $\{\alpha_1,...,\alpha_n\}$. We check the two axioms:

\begin{enumerate}
    \item Given $\sigma,\tau \in \Aut(K/F)$, then $(\sigma\tau)(\alpha_i)=\sigma(\tau(\alpha_i))$ since the group law of $\Aut(K/F)$ is composition.
    \item The identity in $\Aut(K/F)$ is $\id_K$ and $\id_K(\alpha_i)=\alpha_i$ for all $i$.
\end{enumerate}

Thus, it defines a group action, then $\Pi:\Aut(K/F) \to \mathcal{S}_n$, $\sigma \mapsto \Pi_\sigma$ with $\sigma(\alpha_i)=\alpha_{\Pi_\sigma(i)}$ is a well-defined group homomorphism. \\

Moreover, we show that $\Pi$ is injective. Given $\sigma \in \ker\Pi$, then $\Pi_\sigma(i)=i$, for all $i$, which means $\sigma(\alpha_i)=\alpha_{\Pi_\sigma(i)}=\alpha_i$. Since $K=F(\alpha_1,...,\alpha_n)$, $\sigma$ fixes $c$, $\forall c \in K$. Thus, $\sigma=\id_K$, which means $\ker \Pi=\{\id_K\}$ and thus $\Pi$ is injective. 

\subsection*{9.4 Intermediate fields and fixed fields}
\indent

\textbf{Def. }Let $K/F$ be an extension of fields. A field $E$ satisfying $F \subseteq E \subseteq K$ is called an \textbf{intermediate field} to $K/F$. \\ \\

\textbf{Prop 16. }Let $K/F$ be a Galois extension, then:

\begin{itemize}
    \item[] (i) Let $H \subseteq \Gal(K/F)$ be a subgroup, then $K^H:=\{x \in K:\sigma(x)=x,\ \forall \sigma \in H\}$ is an intermediate field to $K/F$. Such a field is called a \textbf{fixed field} of $H$.

    \item[] (ii) Let $E$ be an intermediate field to $K/F$, then $K/E$ is a Galois extension and $\Gal(K/E)$ is a subgroup of $\Gal(K/F)$.
\end{itemize}

pf-1. Clearly, $F \subseteq K^H \subseteq K$ since all $\sigma \in H$ fix elements in $F$. We now show that $K^H$ is a field. First, $\forall \sigma \in H$, $\sigma(0)=0$ and $\sigma(1)=1$ since $\sigma$ is an automorphism. Moreover, given $\alpha,\beta \in K^H$, $\forall \sigma$, we have $\sigma(\alpha)=\alpha$ and $\sigma(\beta)=\beta$. Then:

\begin{gather*}
    \sigma(\alpha+\beta)=\sigma(\alpha)+\sigma(\beta)=\alpha+\beta \\
    \sigma(\alpha\beta)=\sigma(\alpha)\sigma(\beta)=\alpha\beta \\
    \sigma(-\alpha)=-\sigma(\alpha)=\alpha \\
    \sigma(\alpha^{-1})=\sigma(\alpha)^{-1}=\alpha^{-1}
\end{gather*}

Thus, $K^H$ is an intermediate field to $K/F$. \\

pf-2. Since $K/F$ is a Galois extension, there exists a separable polynomial $f \in F[x]$ s.t. $K$ is a splitting field for $f$ over $F$. Since $F \subseteq E$, we have $f \in E[x]$. Moreover, $f$ is still separable, and $K \supseteq E$ contains all roots of $f$. Thus, $K \supseteq K'$, for all $K'$, splitting field for $f$ over $H$. But, $K$ is a splitting field for $f$ over $F$, which means $K \subseteq K'$ since $K'$ contains all roots of $f$. Thus, $K=K'$, $K$ is a splitting field for a separable polynomial $f \in E[x]$, i.e. $K/E$ is a Galois extension. Note that $\forall \sigma \in \Gal(K/E)$, $\sigma$ also fixes elements in $F$ since $F \subseteq E$. Thus, $\Gal(K/E) \subseteq \Gal(K/F)$. \\ \\

\textbf{Remark. }Consider $K/F$, an extension of field. Then by proposition 16, $K/F$ is Galois implies that $K/E$ is Galois. But $E/F$ is not necessarily Galois. We can consider $\mathbb{Q} \subseteq \mathbb{Q}(\sqrt[4]{2}) \subseteq \mathbb{Q}(\sqrt[4]{2},i)$ as a counterexample. (By $f(x)=x^4-2$, $\mathbb{Q}(\sqrt[4]{2},i)/\mathbb{Q}$ is Galois, but $\mathbb{Q}(\sqrt[4]{2})$ is not Galois since $[\mathbb{Q}(\sqrt[4]{2}):\mathbb{Q}]=4 \ne 2=\Aut(\mathbb{Q}(\sqrt[4]{2})/\mathbb{Q})$.) \\ \\

\textbf{Remark. }Similarly, let $F \subseteq E \subseteq K$, if $E/F$ is Galois and $K/E$ is Galois, it doesn't mean that $K/F$ is Galois. We can consider $\mathbb{Q} \subseteq \mathbb{Q}(\sqrt{2}) \subseteq \mathbb{Q}(\sqrt[4]{2})$ as a counterexample. (By $f(x)=x^2-2$ and $g(x)=x^2-\sqrt{2}$, $\mathbb{Q}(\sqrt{2})/\mathbb{Q}$ and $\mathbb{Q}(\sqrt[4]{2})/\mathbb{Q}(\sqrt{2})$ are Galois, but again, $\mathbb{Q}(\sqrt[4]{2})$ is not Galois.) \\ \\

\textbf{Lemma 17. }Let $K/F$ be a Galois extension, then $K/F$ is separable, i.e. $\forall \alpha \in K$, $\Phi_{F,\alpha}$ is separable. \\

pf. Given $\alpha \in K$, let $\Phi_{F,\alpha}$ be its minimal polynomial over $F$. Since $K/F$ is Galois, by theorem 9, $[K:F]=\#\Aut(K/F)$. Moreover, by proposition 16, $K/F(\alpha)$ is also Galois, i.e. $\#\Aut(K/F(\alpha))=[K:F(\alpha)]$. Now, since $\deg \Phi_{F,\alpha}=[F(\alpha):F]$, we have:

\[
    \#\Aut(K/F)=[K:F]=[K:F(\alpha)][F(\alpha):F]=\#\Aut(K/F(\alpha))\deg \Phi_{F,\alpha}
\]

On the other hand, let $\alpha_1,...,\alpha_r$ be distinct roots of $\Phi_{F,\alpha}$ and $\alpha_1=\alpha$. By lemma 10, $\#\{\sigma:F(\alpha) \xhookrightarrow{}K \mid \sigma(c)=c,\ \forall c \in F\}=r$. Note that in the proof of lemma 10, these $\sigma$ is precisely the isomorphism between $F(\alpha)$ and $F(\alpha_i)$, for $i=1,2,...,r$. Then, $\#\Aut(K/F)=\sum_{i=1}^r\#\Aut(K/F(\alpha_i))$. (Each $\sigma$ constructs a relation between $F(\alpha)$ and $F(\alpha_i)$, with the automorphisms in $\Aut(K/F(\alpha_i))$, they can form a unique automorphism in $\Aut(K/F)$.) \\

By theorem 16, we know that each $K/F(\alpha_i)$ is Galois, we have $\#\Aut(K/F(\alpha_i))=[K:F(\alpha_i)]$. Since $F(\alpha) \simeq F(\alpha_i)$, we have $[K:F(\alpha_i)]=[K:F]/[F(\alpha_i)/F]=[K:F]/[F(\alpha):F]=[K:F(\alpha)]$. Thus, $\#\Aut(K/F(\alpha_i))=[K:F(\alpha)]=\#\Aut(K/F(\alpha))$. Then, we have $\#\Aut(K/F)=\sum_{i=1}^r\#\Aut(K/F(\alpha_i))=r \cdot \#\Aut(K/F(\alpha))$, which means:

\[
    \#\Aut(K/F(\alpha))\deg \Phi_{F,\alpha}=\#\Aut(K/F)=r \cdot \#\Aut(K/F(\alpha))
\]

Thus, $\deg\Phi_{F,\alpha}=r=\text{Number of distinct roots of $\Phi_{F,\alpha}$}$, i.e. $\Phi_{F,\alpha}$ is separable, which tells us that $K/F$ is a separable extension. \\ \\

\textbf{Theorem 18. }Let $K/F$ be a Galois extension, then:

\begin{itemize}
    \item[] (i) If $H \subseteq \Gal(K/F)$ is a subgroup, then $H=\Gal(K/K^H)$. 

    \item[] (ii) If $E$ is an intermediate field to $K/F$, then $E=K^{\Gal(K/E)}$.
\end{itemize}

pf-1. For $H \subseteq \Gal(K/K^H)$, given $\sigma \in H$, it is an automorphism that fixes $F$. Now given any $x \in K^H=\{y \in K:\sigma(y)=y,\ \forall \sigma \in H\}$, we have $\sigma(x)=x$ since $\sigma \in H$. Thus, $\sigma \in \Gal(K/K^H)$, i.e. $H \subseteq \Gal(K/K^H)$. \\

Next, we show that $\#H \ge \#\Gal(K/K^H)=[K:K^H]$. By lemma 17, $K/F$ is separable, and we always assume a Galois extension is finite, so it satisfies the condition of primitive element theorem, then $\exists \gamma \in K$ s.t. $K=F(\gamma)$. Now, we consider $g(x)=\prod_{\sigma \in H}(x-\sigma(\gamma))$, then the coefficients of $g$ are symmetric functions on the set $\{\sigma(\gamma):\sigma \in H\}$. (e.g. the coefficient of $x^{\lvert H \rvert-1}$ is $-\sum_{\sigma \in H}\sigma(\gamma)$) \\

Since all elements in $H$, say $\tau$, are automorphisms, they all satisfy $\tau(\{\sigma(\gamma):\sigma \in H\})=\{\sigma(\gamma):\sigma \in H\}$ ($\tau(\sigma(\gamma))=\sigma(\tau(\gamma))$ and $\tau$ is surjective). Thus, $H$ fixes the coefficients of $g$ since the coefficients are symmetric. It means that $g(x) \in K^H[x]$, along with $g(\gamma)=0$ (when $\sigma=\id$, $x-\sigma(\gamma)=x-\gamma$), we have $\Phi_{K^H,\gamma} \mid g$, which means $\#H=\deg g \ge \deg \Phi_{K^H,\gamma}=[K:K^H]=\#\Gal(K/K^H)$. By $H \subseteq \Gal(K/K^H)$, we have $H=\Gal(K/K^H)$. \\

pf-2. For $E \subseteq K^{\Gal(K/E)}$, given $x \in E$ and any $\sigma \in \Gal(K/E)$, then $\sigma$ fix any element in $E$, which means $\sigma(x)=x$. Thus, $x \in K^{\Gal(K/E)}$, i.e. $E \subseteq K^{\Gal(K/E)}$ \\

On the other hand, by theorem 16, we have $\#\Gal(K/E)=[K:E]$. Now, since $E \subseteq K^{\Gal(K/E)} \subseteq K$, we have $[K:E] \ge [K:K^{\Gal(K/E)}]$. Again by theorem 16, since $K^{\Gal(K/E)}$ is an intermediate field to $K/F$, $H:=K/K^{\Gal(K/E)}$ is Galois. Thus, $\#\Gal(K/K^H)=[K:K^H]$. To put it all together, we have:

\[
    \#\Gal(K/E)=[K:E]\ge [K:K^H]=\#\Gal(K/K^H)\underset{\text{by (i)}}{=}\#H=\#\Gal(K/E)
\]

That is, $[K:E]=[K:K^H]$, i.e. $E=K^H=K^{\Gal(K/H)}$ since $E \subseteq K^{\Gal(K/H)}$. \\ \\

\textbf{Lemma 19. }Let $K/F$ be a finite separable extension and $H \subseteq \Aut(K/F)$. Then, $K/K^H$ is a Galois extension and $\Gal(K/K^H)=H$. (It is a generalization of theorem 18.) \\

pf. Since $K/F$ is finite and separable, by primitive element theorem, $\exists \gamma \in K$ s.t. $K=F(\gamma)$. Similarly, we consider $g(x)=\prod_{\sigma \in H}(x-\sigma(\gamma)) \in K^H[x]$, since $H \subseteq \Aut(K/F)$, $\sigma(\gamma) \in K$, for all $\sigma$. Thus, $g$ splits completely over $K$. Since a splitting field for $g$ over $K^H$ must contain $\gamma$ ($\gamma$ is a root of $g$), $K$ is a splitting field for $f$ over $K^H$. Moreover, $\sigma(\gamma)$ are distinct because if $\sigma(\gamma)=\tau(\gamma)$ for some $\sigma,\tau \in H$, then $\sigma(c)=\tau(c)$, for all $c \in F(\gamma)=K$, i.e. $\sigma=\tau$. Thus, $g$ is separable, which means $K/K^H$ is Galois. By theorem 18, since $K^H$ is an intermediate field to $K/K^H$ and $K/K^H$ is Galois, we have $H=\Gal(K/K^H)$.

\subsection*{9.5 Normal extension and equivalent definitions of Galois extension}
\indent

\textbf{Def. }Let $K/F$ be an algebraic extension of field, then $K/F$ is called a \textbf{normal extension} if $\forall $irreducible polynomial $f \in F[x]$, $f$ either splits completely in $K$ or has no root in $K$. \\ \\

\textbf{Theorem 20. }Let $K/F$ be a finite, separable extension of field. Then TFAE:

\begin{itemize}
    \item[] (i) $K/F$ is a Galois extension.

    \item[] (ii) $K$ is a splitting field for some $f$.

    \item[] (iii) $K/F$ is a normal extension.

    \item[] (iv) $\#\Aut(K/F)=[K:F]$

    \item[] (v) $K^{\Aut(K/F)}=F$

    \item[] (vi) For any extension $L/K$ and every homomorphism $\phi:K \xhookrightarrow{}L$ satisfying $\phi(c)=c$, $\forall c \in F$, we have $\phi(K)=K$.
\end{itemize}

\textbf{Proof for theorem 20. (Part 1) }We first prove (ii) $\Rightarrow$ (i) $\Rightarrow$ (iii) $\Rightarrow$ (ii). \\

pf (ii) $\Rightarrow$ (i). Write $f(x)=c\prod_{i=1}^n p_i(x)^{e_i}$ with $p_i$ being distinct, irreducible, monic polynomials in $F[x]$. Let $g(x)=\prod_{i=1}^n p_i(x)$, then $g$ has the same roots as $f$, which means $K$ is also a splitting field for $g$ over $F$. Claim: $g$ is separable. Since each $p_i$ is irreducible and monic, it is the minimal polynomial of its own roots, which means if any $p_i$ and $p_j$ have common roots $\alpha$, then they must be the same (both are $\Phi_{F,\alpha}$). Moreover, $K/F$ is separable, which means all $p_i$ are separable. Thus, $g$ is separable, then $K$ is a splitting field for a separable polynomial over $F$, i.e. $K/F$ is Galois. \\


pf (i) $\Rightarrow$ (iii). Given any irreducible polynomial having a root $\beta \in K$, it suffices to show that $\Phi_{F,\beta}$ splits completely in $K$ since $f(x)=c \cdot \Phi_{F,\alpha}$ for some $c \in F$. Since $K/F$ is Galois, we have $\#\Aut(K/F)=[K:F]=[K:F(\beta)][F(\beta):F]$. Since $[F(\beta):F]=\deg \Phi_{F,\alpha}$, and $K/F(\beta)$ is also Galois by theorem 16, we have $\#\Aut(K/F)=[K:F(\beta)][F(\beta):F]=\#\Aut(K/F(\beta))\deg\Phi_{F,\alpha}$. Now define a group action $\Aut(K/F) \acts F$ by direct mapping. Consider the orbit $\Orb_{\Aut(K/F)}(\beta)=\{\sigma(\beta):\sigma \in \Aut(K/F)\}$, we notice that the stabilizer is $\Stab_{\Aut(K/F)}(\beta)=\{\sigma \in \Aut(K/F):\sigma(\beta)=\beta\}=\{\sigma \in \Aut(K/F):\sigma(x)=x,\ \forall x \in \beta\}=\Aut(K/F(\beta))$. Then, by orbit-stabilizer counting theorem,  we have:

\begin{align*}
    \#\Orb_{\Aut(K/F)}(\beta)&=\#\Aut(K/F)/\#\Stab_{\Aut(K/F)}(\beta) \\
    &=\#\Aut(K/F)/\#\Aut(K/F(\beta)) \\
    &=[F(\beta):F] \\
    &=\deg\Phi_{F,\beta}
\end{align*}

Note that $\sigma$ sends $\beta$ roots of $\Phi_{F,\alpha}$ by proposition 8, we have $\Orb_{\Aut(K/F)}(\beta) \subseteq \{\text{Roots of $\Phi_{F,\alpha}$ in $K$}\}$. Then, $\deg\Phi_{F,\beta}=\#\Orb_{\Aut(K/F)}(\beta) \le \text{Number of roots of $\Phi_{F,\beta}$ in $K$}$. But the number of roots of $\Phi_{F,\beta}$ cannot be greater than its degree, we thus have $\deg\Phi_{F,\beta}=\text{Number of roots of $\Phi_{F,\beta}$ in $K$}$, i.e. $\Phi_{F,\beta}$ splits completely in $K$, $K/F$ is normal. \\

pf (iii) $\Rightarrow$ (ii). Since $K/F$ is finite, separable, by primitive element theorem, $\exists \gamma \in K$ s.t. $K=F(\gamma)$. Let $\Phi_{F,\gamma}$ be the minimal polynomial of $\gamma$ over $F$, then since $K/F$ is normal, $\Phi_{F,\gamma}$ is irreducible, and it has a root $\gamma$, it has all roots in $K$. Thus, $K$ contains a splitting field for $\Phi_{F,\gamma}$ over $F$. But since all splitting fields for $\Phi_{F,\gamma}$ over $F$, say $K'$, must contain $\gamma$ as $\gamma$ is a root of $\Phi_{F,\gamma}$, we have $K=F(\gamma) \subseteq K'$. Thus, $K$ is a splitting field for $\Phi_{F,\gamma}$. \\ \\


\textbf{Proof for theorem 20. (Part 2) }Next, we prove (iii) $\Rightarrow$ (vi) $\Rightarrow$ (iv) $\Rightarrow$ (v) $\Rightarrow$ (iii). \\

pf (iii) $\Rightarrow$ (vi). Again by primitive element theorem, $K=F(\gamma)$ for some $\gamma \in K$. Now given any $\phi:K \xhookrightarrow{}L$ s.t. $\phi(c)=c$, $\forall c \in F$, then $\phi(\gamma)$ is still a root of $\Phi_{F,\gamma}$ since $\Phi_{F,\gamma}(\phi(\gamma))=\phi(\Phi_{F,\gamma}(\gamma))=\phi(0)=0$. Since $K/F$ is normal and $\Phi_{F,\gamma}$ is irreducible, $\Phi_{F,\gamma}$ has all roots in $K$, which means $\phi(\gamma)$, a root of $\Phi_{F,\alpha}$, can only be in $K$, which means $\phi(K)\subseteq K$ since $K$ is generated by $\gamma$. \\

Claim: $\phi$ is surjective. Since $\phi(\gamma) \in K$ and $\Img(\phi)$ is generated by $\phi(\gamma)$ (since $K$ is generated by $\gamma$), we have $\Img \phi= F(\phi(\gamma))$. Since $\phi(K)$, we have $F(\phi(\gamma)) \subseteq K$. We now show that $[K:F(\phi(\gamma))]=1$. First, we know:

\[
    [K:F(\phi(\gamma))][F(\phi(\gamma)):F]=[K:F]=[F(\gamma):F]=\deg \Phi_{F,\gamma}
\]

Since $\phi(\gamma)$ is a root of $\Phi_{F,\gamma}$ and $\Phi_{F,\gamma}$ is monic and irreducible, we have $\Phi_{F,\gamma}=\Phi_{F,\phi(\gamma)}$. Thus, we have:

\[
    [F(\phi(\gamma)):F]=\deg \Phi_{F,\phi(\gamma)}=\deg\Phi_{F,\gamma}
\]

Combine two formulas, we have $[K:F(\phi(\gamma))]\deg\Phi_{F,\gamma}=\deg\Phi_{F,\gamma}$, i.e. $[K:F(\phi(\gamma))]$, i.e. $K=F(\phi(\gamma))$ since $F(\phi(\gamma))$ is contained in $K$, which means $\phi(K)=K$. \\

pf (vi) $\Rightarrow$ (iv). Again and again, $K=F(\gamma)$ for some $\gamma \in K$. First, we show that $\#\Aut(K/F) \ge [K:F]$. Since $K/F$ is separable, we know that $\Phi_{F,\gamma}$ is separable, i.e. it has $\deg \Phi_{F,\gamma}$ distinct roots. Let $L$ be the splitting field for $\Phi_{F,\gamma}$ over $F$, then by lemma 1, for any roots $\gamma'$ of $\Phi_{F,\gamma}$ in $L$, there exists a unique $\sigma:F(\gamma) \to F(\gamma')$ s.t. $\sigma$ fixes elements in $F$. By assumption, $\sigma(K)=K$, it means that $\sigma \in \Aut(K/F)$. Thus, $\#\Aut(K/F) \ge \#\text{ of possible $\sigma$}=\#\text{ of distinct roots of $\Phi_{F,\gamma}$ in $L$}=\deg \Phi_{F,\gamma}=[F(\gamma):F]=[K:F]$. \\

Conversely, we show that $[K:F] \ge \#\Aut(K/F)$. For all $\sigma \in \Aut(K/F)$, $\sigma(\gamma)$ is a root of $\Phi_{F,\gamma}$ and $\sigma(\gamma)$ determines the uniqueness of $\sigma$ (that is, if $\sigma_1(\gamma)=\sigma_2(\gamma)$, then $\sigma_1=\sigma_2$), which means $\#\Aut(K/F) \le \#\text{ of distinct roots of $\Phi_{F,\gamma}$ in $L$}=[K:F]$. In conclusion, we have $\#\Aut(K/F)=[K:F]$. \\


pf (iv) $\Rightarrow$ (v). By lemma 19, $K/K^{\Aut(K/F)}$ is Galois and $\Gal(K/K^{\Aut(K/F)})=\Aut(K/F)$. By assumption, we also have $\Aut(K/F)=[K:F]$. Thus, $[K:F]=[K:K^{\Aut(K/F)}][K^{\Aut(K/F)}:F]=\#\Gal(K/K^{\Aut(K/F)})[K^{\Aut(K/F)}:F]=\#\Aut(K/F)[K^{\Aut(K/F)}:F]=[K:F][K^{\Aut(K/F)}:F]$. Then we have $[K^{\Aut(K/F)}:F]=1$, which means $K^{\Aut(K/F)}=F$ since $F \subseteq K^{\Aut(K/F)}$. \\


pf (v) $\Rightarrow$ (i). By lemma 19, $K/K^{\Aut(K/F)}$ is Galois, and by assumption, $K^{\Aut(K/F)}=F$, we have $K/F$ is Galois.

\end{document}