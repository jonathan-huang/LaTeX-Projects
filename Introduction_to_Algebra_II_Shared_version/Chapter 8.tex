\documentclass[12pt]{article}

% Language setting
% Replace `english' with e.g. `spanish' to change the document language
\usepackage[english]{babel}

% Set page size and margins
% Replace `letterpaper' with `a4paper' for UK/EU standard size
\usepackage[letterpaper,top=2cm,bottom=2cm,left=3cm,right=3cm,marginparwidth=1.75cm]{geometry}

% Useful packages
\usepackage{amsmath}
\usepackage{graphicx}
\usepackage{mathtools}
\usepackage{amsfonts}
\usepackage{algorithm}
\usepackage{algorithmicx}
\usepackage[noend]{algpseudocode}
\usepackage[colorlinks=true, allcolors=blue]{hyperref}
\usepackage{bm}
\usepackage{amssymb}
\usepackage{esint}
\usepackage{tikz}
\usepackage{hyperref}
\DeclareMathOperator{\ord}{ord}
\DeclareMathOperator{\chr}{char}
\DeclareMathOperator{\spn}{span}
\DeclareMathOperator{\Img}{Im}
\DeclareMathOperator{\Syl}{Syl}
\DeclareMathOperator{\Conj}{Conj}
\DeclareMathOperator{\Aut}{Aut}
\DeclareMathOperator{\id}{id}
\DeclareMathOperator{\sign}{sign}
\DeclareMathOperator{\Inn}{Inn}
\DeclareMathOperator{\Out}{Out}
\DeclareMathOperator{\Frac}{Frac}
\DeclareMathOperator{\pre}{pre}
\DeclareMathOperator{\Gal}{Gal}
\DeclareMathOperator{\Orb}{Orb}
\DeclareMathOperator{\Stab}{Stab}
\def\acts{\curvearrowright}
\usetikzlibrary{tikzmark}
\newcommand{\tikzarc}[1]{%
\tikzmarknode{a}{#1}
\begin{tikzpicture}[overlay,remember picture]
\draw ([yshift=1pt]a.north west) to[bend left=20] ([yshift=1pt]a.north east);
\end{tikzpicture}%
}

\title{Introduction to Algebra (II)}
\author{Cheng-Yun Yeh}

\begin{document}
\maketitle

\section*{Chapter 8. Fields (II)}
\subsection*{8.1 Algebraic numbers and transcendental numbers}
\indent

\textbf{Def. }Let $L/E$ be an extension of fields and $\alpha \in L$, we say $\alpha$ is \textbf{algebraic} over $E$ if $\alpha$ satiefies $\sum_{i=0}^n a_i\alpha^i=0$ for some $n \in \mathbb{Z}_{\ge 0}$, $a_i \in E$, not all zero. That is, $\exists f(x) \in E[x] \setminus \{0\}$ s.t. $f(\alpha)=0$. Moreover, if $\alpha$ is not algebraic, then we say $\alpha$ is \textbf{transcendental}. \\ \\

\textbf{Notation. }Let $L/E$ be an extension of fields and $\alpha \in L$, we denote $E[a]=\{\sum_{i=0}^n a_i\alpha^i:n \in \mathbb{Z}_{\ge 0},\ a_i \in E,\ \forall i\}$, and $E(\alpha)$: The smallest field in $L$ containing $E$ and $\alpha$. \\ \\

\textbf{Remark. }Clearly, $E[a]$ is a subring of $L$. We notice that for all subring $R$ of $L$ s.t. $\alpha \in R$, then $E[\alpha] \subset R$ by the closure of $R$. Thus, $E[\alpha]$ is the smallest subring of $L$ containing both $E$ and $\alpha$, while $E(\alpha)$ is the smallest subfield of $L$ containing both $E$ and $\alpha$, which tells us $E[\alpha] \subseteq E(\alpha)$. \\ \\

\textbf{Remark. }Let $ev_\alpha$ be the evaluation map $ev_\alpha:E[x] \to L$, $ev_\alpha:f(x) \mapsto f(\alpha)$, then $E[x]=\Img(ev_\alpha)$. \\ \\

\textbf{Theorem 1. }Let $E/F$ be an extension of fields and $\alpha \in E$, then $F[\alpha]=F(\alpha)$ if and only if $\alpha$ is algebraic over $F$. \\

(Note. If $\alpha=0$, then $F[\alpha]=\{0\}=F(\alpha)$, which is trivial, so we let $\alpha \ne 0$.) \\

pf ($\Rightarrow$). Since $\alpha \in F(\alpha)$ and $F(\alpha)$ is a field, we have $\alpha^{-1} \in F(\alpha)=F[\alpha]$, i.e. $\exists n \in \mathbb{Z}_{\ge 0}$ and $a_0,a_1,...,a_n \in F$, not all zero, s.t. $\alpha^{-1}=\sum_{i=0}^na_i\alpha^i$. Thus, we have $\sum_{i=0}^n a_i\alpha^{i+1}-1=0$. Consider $f(x)=\sum_{i=0}^n a_ix^i-1 \in F[x] \in \{0\}$, then $f(\alpha)=0$, i.e. $\alpha$ is algebraic over $F$. \\

pf ($\Leftarrow$). First, we know that $F[\alpha] \subseteq F(\alpha)$. We only need to show that $F[\alpha] \supseteq F(\alpha)$. It suffices to show that $F[\alpha]$ is a field. Since $\alpha$ is algebraic over $F$, there exists $f(x) \in F[x] \setminus \{0\}$ s.t. $f(\alpha)=0$. Consider the evaluation map $ev_\alpha:F[x] \to E$, $ev_\alpha:f(x) \mapsto f(\alpha)$, then $ev_\alpha$ is a ring homomorphism and thus $F[x]/\ker(ev_\alpha) \simeq \Img(ev_\alpha) \subseteq E$, Since $\Img(ev_\alpha)=F[\alpha]$ is a subring of the field, we have $F[x]/\ker(ev_\alpha)$ is an integral domain. Also, $\ker(ev_\alpha) \ne \{0\}$ since $f(x) \in \ker(ev_\alpha)$ and $F[x]$ is a PID, we have $F[x]/\ker(ev_\alpha)$ is a field. In conclusion, $F[\alpha]$ is a field, then $F[\alpha] \supseteq F(\alpha)$ and thus $F[\alpha]=F(\alpha)$. \\ \\

\textbf{Prop 2. }Let $F$ be a field and $f(x) \in F[x] \setminus \{0\}$. Let $E \supseteq F$ be an extension of fields of $F$ and $\alpha \in E$ s.t. $f(\alpha)=0$. Then:

\begin{itemize}
    \item[] (i) $\dim_F(F[x]/(f))=[F[x]/(f):F]=\deg f$
    
    \item[] (ii) $\dim_F(F(\alpha)) =[F(\alpha):F] \le \deg f$
    
    \item[] (iii) Suppose that $f$ is irreducible in $F[x]$, then the evaluation homomorphism $ev_\alpha:F[x] \to F(\alpha)$, $ev_\alpha:f(x) \mapsto f(\alpha)$, induces an isomorphism $F[x]/(f) \simeq F(\alpha)$. In this case, $\dim_F(F(\alpha))=[F(\alpha):F]=\deg(f)$.

    \item[] (Note. To check $F[x]/(f)$ is a $F$-vector space is easy.) 
\end{itemize}

pf-1. Let $I=(f)$, $d=\deg f$ and $\mathcal{B}=\{1+I,x+I,...,x^{d-1}+I\}$, we claim that $\mathcal{B}$ is a basis for $F[x]/I$. First, given $g(x)+I \in F[x]/I$, we have $g(x)=f(x)q(x)+r(x)$ for some $r(x)=0$ or $r(x)=\sum_{i=0}^{d-1} a_ix^i$. Then $g(x)+I=r(x)+I=\sum_{i=0}^{d-1} a_i(x^i+I) \in \spn(\mathcal{B})$. Moreover, if $\sum_{i=0}^{d-1}a_i(x^i+I)=I$ for some $a_i \in F$, then $\sum_{i=0}^{d-1}a_ix^i=f(x)q(x)$ for some $q(x) \in F[x]$. Since if $q(x) \ne 0$, then $\deg(\sum_{i=0}^{d-1}a_ix^i) \le d-1< d \le \deg(f(x)q(x))$. Thus, $q(x)=0$, which gives us $a_i=0$ for all $i$, i.e. $\mathcal{B}$ is linearly independent. In conclusion, $\mathcal{B}$ is a basis of $F[x]/(f)$, then $\dim_F(F[x]/(f))=\lvert \mathcal{B} \rvert=d=\deg(f)$. \\

pf-2. Consider the evaluation map $ev_\alpha$ and let $I=\ker(ev_\alpha)$, we have $0 \ne (f) \subset I$. By theorem 1, since $\alpha$ is algebraic over $F$, we have $F[\alpha]=F(\alpha)$ and thus the map $\phi:F[x]/(f) \to F(\alpha)$, $\phi:g+(f) \mapsto g(\alpha)$, is a surjection. (Any $\sum_{i=0}^na_i\alpha^i \in F[\alpha]$ is equal to $\phi(g+(f))$, where $g(x)=\sum_{i=0}^na_ix^i \in F[x]$.) Thus, $F(\alpha)=F[x]/I$ (the first isomorphism theorem) is a subspace of $F[x]/(f)$, which gives us $\dim_F(F(\alpha)) \le \dim_F(F[x]/(f))=\deg f$. \\

pf-3. Since $f$ is irreducible in $F[x]$, we have $(f)$ is maximal. We know that $ev_\alpha$ is not a zero map because $ev_\alpha\vert_F=\id$, it means $(f) \subseteq \ker(ev_\alpha) \subsetneq F[x]$, which gives us $(f)=\ker(ev_\alpha)$ and thus $F[x]/(f)=F[x]/\ker(ev_\alpha)\simeq F(\alpha)$, $\dim_F(F(\alpha))=\dim_F(F[x]/(f))=\deg f$.\\ \\

\textbf{Theorem 3. }Let $E/F$ be an extension of fields and $\alpha \in E$, then TFAE:

\begin{itemize}
    \item[] (i) $F[\alpha]$ is a finite dimensional $F$-vector space.

    \item[] (ii) $[F(\alpha):F]$ is finite.

    \item[] (iii) $\alpha$ is algebraic over $F$.
\end{itemize}

pf (i) $\Rightarrow$ (iii). Let $d=\dim_FF[a]$, consider $\{1,\alpha,\alpha^2,...,\alpha^d\}$, then it is linearly dependent since its size is $d+1 \ge d=\dim_FF[x]$. That is, there exists $f(x)=\sum_{i=0}^da_ix^i \in F[x] \setminus \{0\}$ s.t. $f(\alpha)=\sum_{i=0}^da_i\alpha^i=0$, i.e. $\alpha$ is algebraic over $F$. \\

pf (iii) $\Rightarrow$ (ii). Since $\alpha$ is algebraic, there exists $f(x) \in F[x] \setminus \{0\}$ s.t. $f(\alpha)=0$. By proposition 2-(ii), we have $[F(\alpha):F] \le \deg f<\infty$. \\

pf (ii) $\Rightarrow$ (i). Since $F[a] \subseteq F(\alpha)$ always holds, then $F[a]$ over $F$ is a subspace of $F(\alpha)$ over $F$, i.e. $\dim_FF[\alpha] \le \dim_FF(\alpha) < \infty$. \\ \\


\textbf{Corollary 4. }Let $E/F$ be an extension of fields. If $[E:F]$ is finite, then $\forall \alpha \in E$, $\alpha$ is algebraic over $F$. \\

pf. Since $F(\alpha) \subseteq E$, it is a subspace of $E$ (over $F$) and thus $[F(\alpha):F] \le [E:F]<\infty$, then $\alpha$ is algebraice over $F$ by theorem 3. \\ \\

\textbf{Def.} Let $E/F$ be an extension of fields. $E$ is said to be \textbf{algebraic} (i.e. an \textbf{algebraice extension}) over $F$ if $\forall \alpha \in E$, $\alpha$ is algebraic over $F$. (By corollary 4, $[E:F]$ is finite $\Rightarrow$ $E$ is an algebraic extension.)\\ \\

\textbf{Theorem 5. }Let $E/F$ be an extension of fields. If $\alpha,\beta \in E$, both algebraic over $F$, then $\alpha+\beta$, $\alpha\beta$, $\alpha^{-1}$, are all algebraic over $F$. \\

pf. Recall that $F(\alpha_1,...,\alpha_n)$, $\alpha_i \in E$, means the smallest field $K \subseteq E$ containing $F$ and $\alpha_1,...,\alpha_n$. Now, consider $F(\alpha,\beta) \supseteq F(\alpha) \supseteq F$. Since $\alpha,\beta$ are algebraic over $F$, by theorem 3, $[F(\alpha):F]<\infty$ and $[F(\beta):F]<\infty$. Now, we observe that $F(\alpha,\beta)=(F(\alpha))(\beta)$, i.e. $F(\alpha,\beta)$ is the smallest field containing $F(\alpha)$ and $\beta$. (Since for all such field $K$, it must contain $\alpha$ and $\beta$, and thus $K \supseteq F(\alpha,\beta)$.) Moreover, $f(\beta)=0$ for some $f(x) \in F[x] \setminus \{0\} \subseteq F(\alpha)[x] \setminus \{0\}$, we have $\beta$ is algebraic over $F(\alpha)$ and thus $[F(\alpha,\beta):F(\alpha)]=[(F(\alpha))(\beta):F(\alpha)]<\infty$. In conclusion, $[F(\alpha,\beta):F]=[F(\alpha,\beta):F(\alpha)][F(\alpha):F]<\infty$. Since $\alpha+\beta,\ \alpha\beta,\ \alpha^{-1} \in F(\alpha,\beta)$, $F(\alpha+\beta),\ F(\alpha\beta),\ F(\alpha^{-1})$ are contained in $F(\alpha,\beta)$ and hence their dimension over $F$ is finite. By theorem 3, $\alpha+\beta,\ \alpha\beta$, and $\alpha^{-1}$ are algebraic.\\ \\

\textbf{Corollary 6. }Let $E/F$ be an extensions and $K:=\{x \in E:x\text{ is algebraic over }F\}$. Then, $K$ is a subfield of $E$ containing $F$. \\

pf. Trivially, $F \subseteq K \subseteq E$, we only need to show that $K$ is a field. Given $\alpha,\beta \in K$, by theorem 5, we have $\alpha+\beta,\alpha\beta,\alpha^{-1} \in K$, i.e. $K$ is a field. \\ \\

\textbf{Def. }Let $E/F$ be an extension of fields and $\alpha \in E$ be an algebraic element over $F$. The \textbf{minimal polynomial} of $\alpha$ over $F$ is the unique monic polynomial in $F[x]$ that generates the ideal $I_\alpha:=\{f(x) \in F[x]:f(\alpha)=0\}$. We denote this polynomial to be $\Phi_{F,\alpha}(x)$. \\ \\

\textbf{Fact. }$I_\alpha=\{f(x) \in F[x]:f(\alpha)=0\}$ is the kernel of the evaluation map $ev_\alpha$, so it is indeed an ideal. \\ \\

\textbf{Theorem 7. }Let $E/F$ be an extension of field, $\alpha \in E$ be algebraic over $F$, and $\Phi_{F,\alpha}$ be the minimal polynomial of $\alpha$ over $F$. Then:

\begin{itemize}
    \item[] (i) $\Phi_{F,\alpha}(x)$ is irreducible in $F[x]$

    \item[] (ii) $ev_\alpha:F[x] \to F(\alpha)$, $ev_\alpha:f(x) \mapsto f(\alpha)$ induces an isomorphism $F[x]/(\Phi_{F,\alpha}) \xrightarrow{\sim} F(\alpha)$

    \item[] (iii) $[F(\alpha):F]=\deg(\Phi_{F,\alpha})$
\end{itemize}

pf-1. In the proof of theorem 1, we have $(\Phi_{F,\alpha})=\ker(ev_\alpha)$ is maximal (by proving that $F[x]/\ker(ev_\alpha)$ is a subring in a field and hence a domain). Since $F[x]$ is a PID, we have $\Phi_{F,\alpha}$ is irreducible. \\

pf-2. By proposition 2, $F(\alpha)=F[\alpha]=\Img(ev_\alpha) \simeq F[x]/(\Phi_{F,\alpha})$. \\

pf-3. By proposition 2, since $\Phi_{F,\alpha}$ is irreducible, we have $[F(\alpha):F]=\deg(\Phi_{F,\alpha})$.

\subsection*{8.2 Splitting fields}
\indent

\textbf{Theorem 8. }Let $R$ be a commutative ring and $f \in F[x] \setminus \{0\}$, then:

\begin{itemize}
    \item[] (i) If $f(\alpha)=0$ for some $\alpha \in R$, then $\exists g \in R[x]$ s.t. $f(x)=(x-\alpha)g(x)$. (Note that if $R$ is a field then $R[x]$ is a ED and thus trivial. Here, it is a general version.)

    \item[] (ii) Let $R$ be an integral domain. If $\alpha_1,...,\alpha_n$ are distinct roots of $f(x)$ in $R$, then $\exists g \in R[x]$ s.t. $f(x)=(x-\alpha_1)...(x-\alpha_n)g(x)$.

    \item[] (iii) With assumption in (ii), if $f \in R[x] \setminus \{0\}$ has degree $d$, then $f$ has at most $d$ roots in $R$.
\end{itemize}

pf-1. Suppose $f(x)=\sum_{i=0}^{n}a_ix^i$. By the formula $(x^i-\alpha^i)=(x-\alpha)(x^{i-1}+x^{i-2}\alpha+\cdots+x\alpha^{i-2}+\alpha^{i-1})$ and $f(\alpha)=0$, we have $f(x)=f(x)-f(\alpha)=\sum_{i=0}^{n}a_i(x^i-\alpha^i)=(x-\alpha)\sum_{i=1}^n a_i(x^{i-1}+\cdots+\alpha^i)$. \\

pf-2. By (i), we first have $f(x)=(x-\alpha_1)g_1(x)$ for some $g_1 \in F[x]$. Then, plug in $\alpha_2$, we have $0=f(\alpha_2)=(\alpha_1-\alpha_2)g_1(\alpha_2)$, i.e. $g_1(\alpha_2)=0$ since $R$ is an integral domain and $\alpha_1 \ne \alpha_2$. Iteratively, we have $f(x)=(x-\alpha_1)...(x-\alpha_n)g(x)$ for some $g \in F[x]$. \\

pf-3. From (ii), if $f$ has at least $d+1$ roots, then $f(x)=(x-\alpha_1)...(x-\alpha_{d+1})g(x)$ for some dinstinct $\alpha_1,...,\alpha_{d+1} \in R$ and $g \in R[x]$. Then, $d=\deg(f) \le \deg((x-\alpha_1)...(x-\alpha_{d+1}))=d+1$, which is a contradiction. Thus, $f$ has at most $d$ roots in $R$. \\ \\

\textbf{Lemma 9. }Let $A$ be an abelian group and $\alpha,\beta$ are elements in $A$ with finite order, say $\ord(\alpha)=n$ and $\ord(\beta)=m$. Then:

\begin{itemize}
    \item[] (i) If $\gcd(n,m)=1$, then $\ord(\alpha\beta)=nm$.

    \item[] (ii) Suppose $\max\{\ord(a):a \in A\}:=m$ is finite, then $\ord(a) \mid m$, $\forall a \in A$. (If $A$ is not abelian, then it may not be true. Counterexample: $\mathcal{D}_3$)
\end{itemize}

pf-1. First, $(\alpha\beta)^{nm}=(\alpha^n)^m(\beta^m)^n=e$, so $\ord(\alpha\beta):=d \mid nm$ by proposition 3 in chapter 2.1, which in particular gives us $d \le nm$. \\

Moreover, since $e=e^m=((\alpha\beta)^d)^m=\alpha^{dm}(\beta^m)^d=\alpha^{dm}$, so $n \mid dm$, which means $n \mid d$ since $n$ is irreducible and $\gcd(n,m)=1$. By symmetry, we have $m \mid d$, which gives us $nm \mid d$ and thus $d=nm$. \\

pf-2. Suppose $\alpha,\beta \in A$ and $\ord(\alpha)=n$, $\ord(\beta)=m=\max\{\ord(a):a \in A\}$, then we claim: for all prime number $p$, then the largest power of $p$ dividing $n$ cannot be larger than the largest power of $p$ dividing $m$. Then, the claim implies $n \mid m$. \\ 

Write $n=p^in'$, $m=p^jm'$ with $p \nmid n'$ and $p \nmid m'$, we consider $\gamma=\alpha^{n'}$ and $\delta=\beta^{p^j}$, then $\ord(\gamma)=p^i$ and $\ord(\delta)=m'$, which means $\ord(\gamma\delta)=p^im'$ as $\gcd(p^i,m')=1$. If $i>j$, then $\ord(\gamma\delta)=p^im'>p^jm'=\ord(\beta)$, which is a contradiction. Thus, $i \le j$. In conclusion, we have $\ord(a) \mid m$, for all $a \in A$.\\ \\

\textbf{Corollary 10. }Let $U \subseteq F^\times$ be a multiplicative group. If $U$ is finite, then $U$ is cyclic. (In particular, for any $n \in \mathbb{Z}_{\ge 0}$, we have $(\mathbb{Z}/n\mathbb{Z})^\times$ is cyclic.)\\

pf. Let $m=\max\{\ord(a):a \in U\}$ and consider $\alpha \in \{g \in U:\ord(g)=m\}$, we claim: $U=\langle\alpha\rangle$, i.e. $U=\{1,\alpha,...,\alpha^{m-1}\}$. It suffices to show that $\lvert U \rvert = m$ since it has already contained an element $\alpha$ with order $m$. First, $\{1,\alpha,...,\alpha^{m-1}\} \subseteq U$, so $\lvert U \rvert \ge m$. Now consider $f(x)=x^m-1 \in F[x]$, then by lemma 9-(ii), $\forall \beta \in U$, $\ord(\beta) \mid m$ and thus $f(\beta)=\beta^m-1=1-1=0$, i.e. each $\beta$ is a root of $f(x)$. By theorem 8-(iii), $f(x)$ has at most $\deg(f)=m$ roots, which tells us $\lvert U \rvert=(\text{choices of }\beta) \le m$. In conclusion, we have $\lvert U \rvert=m$ and thus $U=\{1,\alpha,...,\alpha^{m-1}\}=\langle \alpha \rangle$. \\ \\

\textbf{Def. }Let $L/F$ be an extension of fields, and $f \in F[x] \setminus \{0\}$ be a polynomial of degree $d$, then:

\begin{itemize}
    \item[] (i) We say $f$ \textbf{splits completely} in $L$ if $\exists c \in F^\times$ and $\alpha_i \in L$ s.t. $f(x)=c\prod_{i=1}^d (x-\alpha_i)$. Note that in this definition, a constant polynomial in $F[x]$ splits completely in $L$.

    \item[] (ii) We say $L$ is a \textbf{splitting field} for $f(x)$ over $F$ if $f$ splits completely in $L$ but does not split completely in any proper subfield of $L$ containing $F$. \\
\end{itemize}

\textbf{Example. }Consider $f(x)=x^4-2 \in \mathbb{Q}[x]$, then the splitting field for $f(x)$ is $\mathbb{Q}(\sqrt[4]{2},i)$, where $i=\sqrt{-1}$. \\ \\



\textbf{Theorem 11. }Let $F$ be a field and $f(x) \in F[x] \setminus \{0\}$, then:

\begin{itemize}
    \item[] (i) There is a field extension $E/F$ that is a splitting field for $f(x)$ over $F$.

    \item[] (ii) If $L/F$ is a splitting field for $f(x)$ over $F$, then $[L:F] \le (\deg f)!$.
\end{itemize}

pf-1. We use induction on $\deg f:=d$. For the base case, we consider $d \le 1$, then the spltting field for $f(x)$ over $F$ is $F$ itself. Suppose the claim holds as $d=n \ge 1$, then when $d=n+1 \ge 2$, we let $g(x)$ be an irreducible factor of $f(x)$. (Note. $g(x)$ can be $f(x)$, so $\deg g$ is not necessarily less than $\deg f$.) We consider $K=F[x]/(g)$, since $g$ is irreducible and $F[x]$ is a PID, we have $F[x]/(g)$ is a field with $F$ injected in it. \\

Now, suppose $f(x)=\sum_{i=0}^n a_ix^i$, then we can regard $f$ as an element in $K[x]$ since $a_i \in F$ is injected in $K$. (That is, regard $a_i$ as $a_i+(g)$.) Consider $\alpha=x+(g) \in K$, we have $f(\alpha)=\sum_{i=0}^n (a_i+(g))(x+(g))^i=\sum_{i=0}^n(a_ix^i+(g))=\sum_{i=0}^na_ix^i+(g)=f+(g)=(g)$, the zero element in $K$, since $f \in (g)$. Thus, $\alpha=x+(g)$ is a root in $K$, which means $f(x)=(x-\alpha)h(x)$ for some $h(x) \in K[x]$ and $\deg(h)=d-1$. By induction hypothesis, $\exists L/K$, field extension, s.t. $L$ is a splitting field for $h(x)$ over $K$. \\

So far, we have $F \subseteq K \subseteq L$ with $L$ being a splitting field for $h(x)$ over $K=F[x]/(g)$, where $f(x)=(x-\alpha)h(x)$ and $\alpha=x+(g)$. By our construction, $f(x)=c\sum_{i=0}^d(x-\alpha_i)$ for some $c \in F^\times$ and $a_i \in L$, i.e. $f$ splits completely in $L$. Now, we're going to show that if $L'=F(\alpha_1,...,\alpha_n) \subseteq L$, then $L$ is the splitting field for $f(x)$ over $F$. Clearly, $f$ splits completely in $L'$. For any field $E$ s.t. $F \subseteq E \subseteq L$, if $f$ splits completely in $E$, then $\alpha_1,...,\alpha_n \in E$ and thus $E \supseteq F(\alpha_1,...,\alpha_n)=L'$, i.e. $E=L'$, which means no proper subfield of $L'$ s.t. $f$ splits completely in it. In conlclusion, $L=F(\alpha_1,...,\alpha_n)$ is a splitting field for $f(x)$ over $F$. \\

(The key idea of proof 1 is to find $\alpha_1,...,\alpha_n$ by induction, and then consider the smallest field containing $F$ and $\alpha_1,...,\alpha_n$, which is $L'$.) \\

pf-2. Since $f(\alpha_1)=0$, by proposition 2, we have $[F(\alpha_1):F] \le \deg(f)=d$. Let $f(x)=(x-\alpha_1)f_1(x)$ for some $f_1(x) \in F[x]$ and $\deg(f_1)=d-1$. Similarly, $f_1(\alpha_2)=0$, so $[F(\alpha)(\beta),F(\alpha)] \le \deg(f_1)=d-1$. Iteratively, since $F(\alpha_1,...,\alpha_k)=(F(\alpha_1,...,\alpha_{k-1}))(\alpha_k)$ for all $k$, we have $[F(\alpha_1,...,\alpha_n):F]=\prod_{i=0}^{d-1}[F(\alpha_1,...,\alpha_{i+1}):F(\alpha_1,...,\alpha_i)]=\prod_{i=0}^{d-1}[(F(\alpha_1,...,\alpha_i))(\alpha_{i+1}):F(\alpha_1,...,\alpha_i)] \le d!$. \\ \\

\textbf{Def. }A field $K$ is said to be \textbf{algebraically closed} if  for all non-constant $f \in K[x]$, $f$ has a root in $K$. \\ \\

\textbf{Prop 12. }Let $K$ be a field, then TFAE:

\begin{itemize}
    \item[] (i) $K$ is algebraically closed.

    \item[] (ii) Every non-zero polynomial $f \in K[x]$ splits completely.
\end{itemize}

pf. We only prove (i) $\Rightarrow$ (ii) since $(\Leftarrow)$ is trivial. Given any non-zero polynomial $f \in K[x]$, if $f$ is a constant, the by definition, $f$ splits completely. If $\deg(f) \ge 1$, by assumption, there exists $\alpha_1 \in K$ s.t. $f(\alpha_1)=0$, i.e. $f(x)=(x-\alpha_1)f_1(x)$ for some $f_1 \in K[x]$. If $\deg(f_1) \ge 1$, then we can also find $\alpha_2 \in K$ s.t. $f_1(\alpha_2)=0$. Iteratively, we have $f(x)=(x-\alpha_1)...(x-\alpha_n)f_n(x)$ and $f_n$ is a non-zero constant in $K[x]$. Then, let $c=f_n(x)$, we have $f(x)=c\prod_{i=0}^n(x-\alpha_n)$, i.e. $f$ splits completely in $K$.

\subsection*{8.3 Separability and irreducibility}
\indent

\textbf{Def. }Let $F$ be a field and $f(x)=\sum_{i=0}^da_ix^i \in F[x]$. We defined the \textbf{formal derivative} of $f(x)$ to be the polonomial $f'(x):=\sum_{i=1}^dia_ix^{i-1}$. \\ \\

\textbf{Prop 13. }Let $f,g \in F[x]$, then:

\begin{itemize}
    \item[] (i) Sum rule: $(f+g)'=f'+g'$
    
    \item[] (ii) Product rule: $(fg)'=f'g+fg'$
    
    \item[] (iii) Chain rule: If $h(x)=f(g(x))$, then $h'(x)=f'(g(x))g'(x)$.
    
    \item[] (iv) If $\chr F=0$, then $f'(x)=0 \Leftrightarrow f(x)=c \in F$.
    
    \item[] (v) If $\chr F=p$ for some prime $p$, then $f'(x)=0 \Leftrightarrow \exists f_1 \in F[x]$ s.t. $f(x)=f_1(x^p)$.
\end{itemize}

pf-1, 2, 3. Expand them, trivial. \\

pf-4. By the equivalent statement: $f'(x)=0 \Leftrightarrow \sum_{i=1}^dia_ix^{i-1}=0 \Leftrightarrow ia_ix^{i-1}=0,\ \forall i \Leftrightarrow a_i=0,\ \forall i \ge 1 \Leftrightarrow f(x)=a_0 \in F$, since $\chr F=0$. \\

pf-5. Similarly, we have $f'(x)=0 \Leftrightarrow ia_ix^{i-1}=0,\ \forall i$. If $\chr F=p$, then it is equivalent to $p \mid i$ or $a_i=0$. Thus, $f'(x)=0 \Leftrightarrow f(x)=a_0+a_px^p+\cdots+a_{\lfloor d/p \rfloor}x^{\lfloor d/p \rfloor}=\sum_{i=0}^{\lfloor \frac{d}{p} \rfloor}a_{ip}x^{ip} \Leftrightarrow \exists f_1(x)=\sum_{i=0}^{\lfloor \frac{d}{p} \rfloor}a_{ip}x^i \in F[x]$ s.t. $f(x)=f_1(x^p)$. \\ \\

\textbf{Def. }Let $L/F$ be a splitting field for $f(x)$, say $f(x)=c\prod_{i=1}^d(x-\alpha_i)$ for some $c \in F$, $\alpha_i \in L$. We say $f$ is \textbf{separable} if all $\alpha_i$ are distinct; otherwise, we say $f$ is inseparable. \\ \\

\textbf{Prop 14. }Let $F$ be a field and $f(x) \in F[x] \setminus F$, then $f$ is separable $\Leftrightarrow \gcd(f,f')=1$. (Note that this "1" can be any unit.) \\

pf $(\Rightarrow)$. Suppose NOT, i.e. $\gcd(f,f')=g(x) \in F[x] \setminus F$. Let $L/F$ be the splitting field for $f(x)$, then there exists $\alpha \in L$ s.t. $f(\alpha)=g(\alpha)=0$. Thus, $f(x)=(x-\alpha)f_1(x)$ for some $f_1 \in F[x]$. Since $f$ is separable, $f_1(\alpha) \ne 0$. Now, suppose $f'(x)=g(x)q(x)$ for some $q(x) \in F[x]$, we have $f'(x)=f_1(x)+(x-\alpha)f'_1(x)$, which gives us $f_1(\alpha)=f'(\alpha)=g(\alpha)q(\alpha)=0$. It is clearly a contradiction. Therefore, $\gcd(f,f')=1$. \\

pf $(\Leftarrow)$. Suppose NOT, i.e. if $L/F$ is a splitting field for $f(x)$, then $f(x)=(x-\alpha)^2f_1(x)$ for some $\alpha \in L$ and $f_1(x) \in F[x]$. Then, $f'(x)=2(x-\alpha)f_1(x)+(x-\alpha)^2f_1'(x)$, which means $(x-\alpha) \mid \gcd(f,f')$. Thus, $\gcd(f,f')$ is not a unit, which contradicts to our assumption. Therefore, $f$ is separable. \\ \\

\textbf{Theorem 15. }Let $F$ be a field and $f \in F[x]$ be an irreducible polynomial, then:

\begin{itemize}
    \item[] (i) If $f'(x) \ne 0$, then $f(x)$ is separable.

    \item[] (ii) If $\chr F=0$, then $f$ is separable.
\end{itemize}

pf-1. By proposition 14, $f'$ is separable if and only if $ \Leftrightarrow \gcd(f,f')=1$. Since $f$ is irreducible and $f(x)=\gcd(f,f') \cdot q(x)$ for some $q(x) \in F[x]$, we have $\gcd(f,f')=f$ or $\gcd(f,f')=1$, up to a unit. But the degree of $\gcd(f,f')$ is less than $\deg(f')=\deg(f)-1$ ($f' \ne 0$ by assumption), so $\gcd(f,f')$ can only be $1$, which means $f$ is separable. \\ \\

pf-2. Since $\chr F=0$, by proposition 13, $f'(x)=0 \Leftrightarrow f(x)=c \in F$. But $f$ is irreducible, so $\deg(f) \ge 1$, which gives us $f'(x) \ne 0$. Thus, $f$ is separable by (i). \\ \\

\textbf{Example. }Let $p$ be a prime, $R=(\mathbb{Z}/p\mathbb{Z})[x]$, and $F=\Frac(R)=\{\frac{a}{b}:a,b \in R,\ b \ne 0\}$ be the fieid of fractions of $R$. Note that $\chr F=p$ since $\frac{a}{b}+\cdots+\frac{a}{b}=\frac{a+\cdots+a}{b}$ and $\chr R=p$. Consider $f(y)=y^p-x \in F[y]$, then $f'(y)=py^{p-1}=0$, which means $\gcd(f,f') \ne 1$, i.e. $f$ is inseparable by proposition 14.

\subsection*{8.4 Gauss's lemma and Eisenstein irreducibility criterion}
\indent

\textbf{Def. }Let $f(x)=\sum_{i=0}^da_ix^i \in \mathbb{Z}[x] \setminus \{0\}$. The \textbf{content} of $f$ is define to be $\gcd(a_0,...,a_d)$, denoted by $\text{content}(f)$ or $\mathcal{C}_f$. \\ \\

\textbf{Lemma 16. }Let $f,g \in \mathbb{Z}[x] \setminus\{0\}$, then $\mathcal{C}_{fg}=\mathcal{C}_f\mathcal{C}_g$. \\

pf. By definition, $f(x)=\mathcal{C}_ff_1(x)$ and $g(x)=\mathcal{C}_gg_1(x)$ for some $f_1,g_1 \in \mathbb{Z}[x] \setminus \{0\}$ s.t. $\mathcal{C}_{f_1}=\mathcal{C}_{g_1}=1$. It suffices to show that $\mathcal{C}_{f_1g_1}=1$. \\

Suppose NOT, then there exists a prime $p$ s.t. $p \mid \mathcal{C}_{f_1g_1}$. Consider the ring homomorphism $\pi:\mathbb{Z}[x] \to (\mathbb{Z}/p\mathbb{Z})[x]$, $\pi:\sum_{i=0}^da_ix^i \mapsto \sum_{i=0}^d \Bar{a}_ix^i$ with $0 \le \Bar{a}_i<p$ and $\Bar{a}_i \equiv a_i$ (mod $p$). Then we have $\pi(f_1g_1)=0$ since all of the coefficients of $f_1g_1$ can be divided by $p$. But it gives us $\pi(f_1)\pi(g_1)=0$ and thus $\pi(f_1)=0$ or $\pi(g_1)=0$ since $(\mathbb{Z}/p\mathbb{Z})[x]$ is an integral domain. (Since $\mathbb{Z}/p\mathbb{Z}$ is a field, we can prove by induction to show that $(\mathbb{Z}/p\mathbb{Z})[x]$ is an integral domain.) Thus, $p \mid \mathcal{C}_f$ or $p \mid \mathcal{C}_g$, which is a contradiction. Thus, $\mathcal{C}_{f_1g_1}=\mathcal{C}_{f_1}\mathcal{C}_{g_1}$, and it follows $\mathcal{C}_{fg}=\mathcal{C}_f\mathcal{C}_g$. \\ \\

\textbf{Theorem 17. (Gauss's lemma for $\mathbb{Z}[x]$) }Let $f \in \mathbb{Z}[x]$ and suppose that $f(x)=g(x)h(x)$ with $g,h \in \mathbb{Q}[x]$. Then, $\exists G,H \in \mathbb{Z}[x]$ s.t. $f(x)=G(x)H(x)$ with $G=\alpha g$, $H=\beta h$, for some $\alpha,\beta \in \mathbb{Q}^\times$. In particular, if $f$ is irreducible in $\mathbb{Z}[x]$, then $f$ is irreducible in $\mathbb{Q}[x]$, and if $f$ is irreducible in $\mathbb{Q}[x]$, then $f$ is irreducible in $\mathbb{Z}[x]$ up to a scalar in $\mathbb{Z}$. (That is, if $f=gh$ with $g,h \in \mathbb{Z}[x]$, then either $g \in \mathbb{Z}$ or $h \in \mathbb{Z}$, but not necessarily in $\mathbb{Z}^\times=\{1,-1\}$.) \\

pf. WLOG, we suppose $f \ne 0$. If $f=gh$ with $g,h \in \mathbb{Q}[x]$, then there exists $\alpha,\beta \in \mathbb{Q}^\times$ s.t. $\mathcal{C}_{\alpha g}=\mathcal{C}_{\beta h}=1$. (Note that $\alpha,\beta$ may be not in $\mathbb{Z}$, e.g. $g(x)=\frac{3}{5}(x+1)$.) Let $G=\alpha g$ and $H=\beta h$, then we have $f=(\alpha^{-1}G)(\beta^{-1}H)=(\alpha\beta)^{-1}GH$. Since $\alpha\beta \in \mathbb{Q}^\times$, we have $\alpha\beta=\frac{A}{B}$ for some $A,B \in \mathbb{Z} \setminus \{0\}$. Thus, $f=(\frac{A}{B})^{-1}GH=\frac{B}{A}GH$, i.e. $Af=BGH$. We have $A\mathcal{C}_f=\mathcal{C}_{Af}=\mathcal{C}_{BGH}=B\mathcal{C}_{GH}=B$, by lemma 16 and our choice of $G,H$. Therefore, $\frac{B}{A}=\mathcal{C}_f$, which gives us $f=\mathcal{C}_fGH$, i.e. $f$ can be factored into the product of $\mathcal{C}_fG$ and $H$ with $\mathcal{C}_fG,H \in \mathbb{Z}[x]$. \\ \\

\textbf{Def. }Let $R$ be a UFD, given $\alpha,\beta \in R$, we define $A=\gcd(\alpha,\beta) \Leftrightarrow $ $ A \mid \alpha$, $A \mid \beta$, and if $c \mid \alpha,\ c \mid \beta,\ $then $c \mid A$, where $c \mid \alpha$ means that $\alpha=c\gamma$ for some $\gamma \in R$. \\ \\

\textbf{Remark. }Given a UFD $R$ and $\alpha,\beta \in R$, write $\alpha=u_1\prod_{i=1}^n p_i^{e_i}$, $\beta=u_2\prod_{i=1}^n p_i^{\epsilon_i}$ for some $e_i,\epsilon_i \ge 0$, $u_1,u_2 \in R^\times$, $p_i$ irreducible in $R$ and non-associate to each other. Claim: $d=\prod_{i=1}^n p_i^{\min\{e_i,\epsilon_i\}}$ is the gcd of $\alpha,\beta$. Note that in a UFD, gcd is up to a unit. \\

pf. Clearly, $d \mid \alpha$ and $d \mid \beta$. Given any $c \in R$ s.t. $c \mid \alpha$, $c \mid \beta$, i.e. $\alpha=c\gamma_1$ and $\beta=c\gamma_2$ for some $\gamma_i \in R$. Write $c=u_c\prod_{i=1}^n p_i^{f_i}$ for some $u_c \ge 0$, $f_i \in \mathbb{Z}_{\ge 0}$, then we have $f_i \le e_i$ and $f_i \le \epsilon_i$, i.e. $c \mid d$. \\ \\

\textbf{Def. }Let $R$ be a UFD and $f(x)=\sum_{i=0}^d a_ix^i \in R[x]$, we define the \textbf{content} of $f$ to be $\mathcal{C}_f=\gcd(a_0,...,a_d)$, which is up to a unit. \\ \\

\textbf{Lemma 16'. (Generalization for lemma 16) }Let $R$ be a UFD, $f,g \in R[x] \setminus \{0\}$, then $\mathcal{C}_{fg}=u\mathcal{C}_f\mathcal{C}_g$ for some $u \in R^\times$. \\

pf. Write $f(x)=\mathcal{C}_ff_1(x)$ and  $g(x)=\mathcal{C}_gg_1(x)$ for some $f_1,g_1 \in R[x] \setminus \{0\}$ s.t. $\mathcal{C}_{f_1},\mathcal{C}_{g_1} \in R^\times$. Similarly, it suffices to show that $\mathcal{C}_{f_1g_1} \in R^\times$. \\

Suppose NOT, i.e. $\mathcal{C}_{f_1g_1} \in R^\times$. Since $R$ is a UFD, there exists an irreducible $r \in R$ s.t. $r \mid \mathcal{C}_{f_1g_1}$. Consider the map $\pi:R[x] \to (R[x]/(r))[x]$, $\pi:\sum_{i=0}^d a_ix^i \mapsto \sum_{i=0}^d \overline{a}_ix^i$, where $\overline{a}_i=a_i+(r)$. Since $R$ is a UFD, if $r$ is irreducible, $(r)$ is a prime ideal. Thus, $R/(r)$ is an integral domain, which implies that $(R/(r))[x]$ is also an integral domain. We know that $\pi(f_1g_1)=\pi(f_1)\pi(g_1)=f_1g_1+(r)=0$, which gives us $\pi(f_1)=0$ or $\pi(g_1)=0$, i.e. $r \mid \mathcal{C}_{f_1}$ or $r \mid \mathcal{C}_{g_1}$. Since $r$ is irreducible, we have $\mathcal{C}_{f_1} \notin R^\times$ or $\mathcal{C}_{g_1} \notin R^\times$, which is a contradiction. Therefore, $\mathcal{C}_{f_1g_1} \in R^\times$. \\ \\

\textbf{Theorem 17'. (Generalization for theorem 17) }Let $R$ be a UFD and $F=\Frac(R)$. Let $f \in R[x]$ and suppose that $f=gh$ for some $gh \in F[x]$, then $\exists G,H \in R[x]$ s.t. $f=GH$. \\

pf. WLOG, we suppose $f \ne 0$. If $f=gh$ with $g,h \in F[x]$, then there exists $\alpha,\beta \in F^\times$ s.t. $\mathcal{C}_{\alpha g},\mathcal{C}_{\beta h} \in R^\times$, and $\alpha g,\beta h \in R[x]$. Let $G=\alpha g$, $H=\beta h$, then $f=(\alpha\beta)^{-1}GH$. Since $(\alpha\beta)^{-1} \in F$, we have $(\alpha\beta)^{-1}=\frac{A}{B}$ with $A,B \in R$ and $A,B \ne 0$. Thus, $Bf=AGH$, which gives us $B\mathcal{C}_f=\mathcal{C}_{fg}=\mathcal{C}_{AGH}=uA\mathcal{C}_{G}\mathcal{C}_{H}=uA$ for some $u \in R^\times$ by our choice of $G,H$. Thus, $\mathcal{C}_f=\frac{A}{B}u$, then $f=\frac{A}{B}GH=u'\mathcal{C}_fGH$ for $u'=u^{-1} \in R^\times$, i.e. $f$ can be factored into the product of $u'\mathcal{C}_FG$ and $H$ with $u'\mathcal{C}_fG,H \in R[x]$. \\ \\

\textbf{Corollary 18. }Let $R$ be a integral domain, then:

\begin{itemize}
    \item[] (i) Suppose $R$ is a UFD, given $f \in R[x]$, assume that $\mathcal{C}_f \in R^\times$, then $f$ is irreducible in $R[x] \Leftrightarrow $ $f$ is irreducible in $F[x]$, where $F=\Frac(R)$.

    \item[] (ii) If $R[x]$ is a UFD, then $R$ is a UFD.
\end{itemize}

pf-1 ($\Rightarrow$). By Gauss's lemma, if $f$ is reducible in $F[x]$, then $f$ is reducible in $R[x]$, which is a contradiction. \\

pf-1 ($\Leftarrow$). Suppose $f=gh \in R[x]$, if $f$ is irreducible in $F[x]$, it means $g \in F[x]^\times=F^\times$ or $h \in F[x]^\times=F^\times$. WLOG, suppose $g \in F^\times$, then it means $g$ is a constant in $R[x]$, i.e. $g=c \in R \setminus \{0\}$. Thus, $\mathcal{C}_f=c\cdot\mathcal{C}_h$, which gives us $c \mid \mathcal{C}_f$. But $\mathcal{C}_f \in R^\times$, it means $c \in R^\times \subset R[x]^\times$, i.e. $f$ is irreducible in $R[x]$. \\

pf-2. Given $r \in R$, we have $r=\prod_{i=1}^n r_i$ for some irreducible $r_i \in R[x]$, unique up to a unit in $R[x]^\times=R^\times$, since $R \subset R[x]$. Thus, it suffices to show that each $r_i$ is irreducible in $R$. But this is trivial since if $r_i=pq$, we have $p \in R[x]^\times=R^\times$ or $q \in R[x]^\times=R^\times$ since $r_i$ is irreducible in $R[x]$. Then, $r_i$ is irreducible in $R$, which proves the claim. \\

\textbf{Remark. }Given a UFD $R$, then $r$ is irreducible in $R$ if and only if $r$ is irreducible in $R[x]$. \\

pf. ($\Leftarrow$) is shown above, we only prove ($\Rightarrow$). If $r$ is irreducible in $R$ and $r=f \cdot g \in R[x]$, then $0=\deg(r)=\deg(f)+\deg(g)$ since $R$ is an integral domain (the leading coefficients of $f,g$ are not zero divisors). Thus, $f \in R$ an $g \in R$, which means $f \in R^\times=R[x]^\times$ or $g \in R^\times=R[x]^\times$ since $r$ is irreducible in $R$. \\ \\

\textbf{Corollary 19. (Irreducibility via reduction mod $p$) }Let $d \ge 1$, $f(x)=\sum_{i=0}^d a_ix^i$. Let $p$ be a prime and $\pi:\mathbb{Z}[x] \to (\mathbb{Z}/p\mathbb{Z})[x]$, $\pi(f)=\sum_{i=0}^d\overline{a}_ix^i$, where $\overline{a}_i=a_i$ (mod $p$). Suppose $p \nmid a_d$, then $\pi(f)$ is irreducible $\Rightarrow$ $f$ is irreducible in $\mathbb{Q}[x]$. \\

pf. Since $p \nmid a_d$, we have $\deg(\pi(f))=\deg(f)$. Suppose that $f$ is reducible in $\mathbb{Q}[x]$, then $f=gh$ for some $g,h \in \mathbb{Q}[x]$. By Gauss's lemma, $f=GH$ for some $G,H \in \mathbb{Z}[x]$. Apply $\pi$, we have $\pi(f)=\pi(GH)=\pi(G)\pi(H)$. The assumption that $p \nmid a_d$ and that $p$ is a prime give us $\pi(G) \notin (\mathbb{Z}/p\mathbb{Z})^\times=(\mathbb{Z}/p\mathbb{Z})[x]^\times$ and $\pi(H) \notin (\mathbb{Z}/p\mathbb{Z})^\times=(\mathbb{Z}/p\mathbb{Z})[x]^\times$, which contradicts to fact that $\pi(f)$ is irreducible in $(\mathbb{Z}/p\mathbb{Z})[x]$. Thus, $f$ is irreducible in $\mathbb{Q}[x]$. \\ \\

\textbf{Notation. }We sometimes write $\pi(f)$ as $\bar{f}_p$. \\ \\

\textbf{Prop 20. (Eisenstein irreducibility criterion) }Let $R$ be an integral domain (then so is $R[x]$), $f(x)=\sum_{i=0}^da_ix^i \in R[x]$. and $P \trianglelefteq R$ be a prime ideal. Suppose that $a_d \notin P$, $a_i \in P$, $\forall 0 \le i \le d-1$, but $a_0 \notin P^2=\{\sum_{i=1}^n a_ib_i:a_i,b_i \in P,\ n \in \mathbb{Z}_{\ge 1}\}$. Then, $f$ cannot be written as $gh$ with $\deg g<d,\ \deg h<d$ in $R[x]$. \\

pf. The idea is to use corollary 19. Consier $\pi:R[x] \to (R/P)[x]$, $\pi(f)=\overline{a}_ix^i$, where $\overline{a}_i=a_i+P$. As $a_i \in P$, $\forall 0 \le i \le d-1$, and $a_d \notin P$, we have $\pi(f)=\overline{a}_dx^d \ne 0$ in $(R/P)[x]$. \\

If $f=gh$ for some non-constant $g,h \in R[x]$, then $\overline{a}_dx^d=\pi(f)=\pi(g)\pi(h)$. Write $g(x)=\sum_{i=0}^n b_ix^i$ and $h(x)=\sum_{i=0}^m c_ix^i$, we have $a_d=b_nc_m$ and $a_0=b_0c_0$. As $a_i \notin P$, we have $b_n \notin P$ and $c_m \notin P$. Also, as $a_0 \in P$ but $a_0 \notin P^2$, we have $b_0 \in P$ or $c_0 \in P$, but not both. WLOG, we assume $b_0 \notin P$ and $c \in P$, let $i_0=\min\{i:c_i \notin P\}$, then $0<i_0 \le m<d$ (since $\deg(h)<\deg(f)$). See the $x^{i_0}$ terms for $\pi(f)$, we have $P=a_{i_0}+P=(b_0c_{i_0}+b_1c_{i_0-1}+\cdots+b_{i_0-1}c_1+b_{i_0}c_0)+P=b_0c_{i_0}+P$. But $b_0,c_{i_0} \notin P$, which gives us a contradiction since $P$ is a prime ideal. Thus, $f$ is irreducible in $R[x]$. \\ \\

\textbf{Corollary 21. (Eisenstein irreducibility criterion for $\mathbb{Z}$) }Let $p \in \mathbb{Z}$ be a prime and $f(x)=\sum_{i=0}^d a_ix^i \in \mathbb{Z}[x]$. Suppose that $p \nmid a_d$, $p \mid a_i$, $\forall 0 \le i \le d-1$, but $p^2 \nmid a_0$. Then, $f$ is irreducible in $\mathbb{Q}[x]$. \\

pf. It suffices to show that $f$ cannot be written as $gh$ for some non-constant $g,h \in \mathbb{Z}[x]$. (By Gauss lemma, $f=gh$ for some $g,h \in \mathbb{Q}[x] \setminus \mathbb{Q} \Rightarrow f=GH$ for some $G,H \in \mathbb{Z}[x] \setminus \mathbb{Z}$) Apply proposition 20 on the UFD $R=\mathbb{Z}$ and the prime ideal $P=(p)$, we're done. \\ \\

\textbf{Theorem 22. }Let $R$ be an integral domain. Then, $R$ is a UFD if and only if $R[x]$ is a UFD. \\

pf. We only show ($\Rightarrow$) since ($\Leftarrow$) is shown in corollary 18-(ii). Let $f \in R[x]$, write $f(x)=\mathcal{C}_f\cdot f_1(x)$, then $\mathcal{C}_{f_1} \in R^\times$. Since $R$ is a UFD, $\mathcal{C}_f$ has a unique factorization in $R$. Let $F=\Frac(R)$, then since $f_1 \in R[x] \subset F[x]$, $f_1(x)=\prod_{i=1}^n p_i(x)$ for some irreducibles $p_i(x)$, non-associate with each other (not necessarily distinct). By theorem 17', $\exists P_i \in R[x]$ s.t. $f_1=\prod_{i=1}^n P_i$. \\

Now we claim: each $P_i$ is irreducible in $R[x]$. Since $\mathcal{C}_{f_1} \in R^\times$, we have $\mathcal{C}_{P_i} \in R^\times,\forall i$. By corollary 18-(i), we have $P_i$ is irreducible in $R[x]$, which gives $f$ a factorization of irreducible factors $\mathcal{C}_{f}\prod_{i=1}^n P_i$. \\

Now, we show that the factorization is unique up to a unit and rearrangement. Suppose $f(x)=\mathcal{C}_ff_1(x)$ with $\mathcal{C}_{f_1} \in R^\times$. Since $R$ is a UFD, the factorization of $\mathcal{C}_f$ is unique. Moreover, suppose $f_1=\prod_{i=1}^n P_i=\prod_{i=1}^m Q_i$, $P_i,Q_i \in R[x] \setminus R$, irreducible. Since $\mathcal{C}_{f_1} \in R^\times$, we have $\mathcal{C}_{P_i} \in R^\times$ and $\mathcal{C}_{Q_i} \in R^\times$. Since $F[x]$ is a UFD, for each $Q_i$, we have $Q_i(x)=P_j \cdot c_j$ for some $j$ depending on $i$, $c_j \in F^\times$. By comparing degree, we have $n=m$. Finally, since $c_j=\frac{A_i}{B_i}$ with $A_i,B_i \in R \setminus \{0\}$, then $B_iQ_i=A_iP_j$, which gives us $B_i\mathcal{C}_{Q_i}=\mathcal{C}_{B_iQ_i}=\mathcal{C}_{A_iP_j}=A_i\mathcal{C}_{P_j}$, i.e. $A_i=B_i \cdot \mu_i$, $\mu_i \in R^\times$. Thus, $c_i \in R^\times$, i.e. the factorization of $f_1$ is unique up to a unit follows from the prove for $F[x]$ is a UFD if $F$ is a field. \\

In conclusion, the factorization of both $\mathcal{C}_f$ and $f_1$ are unique up to a unit, the factorization of $f$ is unique. \\ \\

\textbf{Corollary 23. }If $R$ is a UFD, then $R[x_1,...,x_n]$ is a UFD. We can prove it by induction. 

\subsection*{8.5 Ruler and compass construction}
\indent

\textbf{Def. }Let $P,Q \in \mathbb{R}^2$, denote $L(P,Q) \subset \mathbb{R}^2$ be the line passing through $P$ and $Q$, and $C(P,Q) \subset \mathbb{R}^2$ be the circle centered at $P$ and passing through $Q$. \\ \\

\textbf{Def. }The set of \textbf{constructible points} is the smallest set of points in $\mathbb{R}^2$ having the following properties:

\begin{itemize}
    \item[] (i) $(0,0)$, $(1,0)$ are constructible.

    \item[] (ii) If $P,Q,R,S$ are constructible, then the follwing intersections are constructible: (a) $L(P,Q) \cap L(R,S)$ (if $L(P,Q) \ne L(R,S)$), (b) $L(P,Q) \cap C(R,S)$, (c) $C(P,Q) \cap C(R,S)$ (if $C(P,Q) \ne C(R,S)$). \\
\end{itemize}

\textbf{Def. }The set of \textbf{constructible numbers} is the set of numbers in $\mathbb{R}$ consisting of $X$ and $Y$ coordinates of all constructible points in $\mathbb{R}^2$. For example, any $\alpha \in \mathbb{Q}$ and $\sqrt{\beta}$ with $\beta \in \mathbb{Z}_{\ge 0}$ are constructible. \\ \\

\textbf{Remark. }Let $S$ be the set of constructible points and $F=\{\text{Coordinates of }P \in S\}$ be the set of constructible numbers. Then, $F$ is a field. \\

pf. Given $\alpha,\beta \in F$. Then $\alpha \pm \beta$ is easy to construct. Use similar triangle, we can also construct $\alpha\beta$ ($\alpha:\alpha\beta=1:\beta$) and $\alpha^{-1}$ ($\alpha:1=1:\alpha^{-1}$). \\ \\

\textbf{Lemma 24. }Let $P,Q,R,S$ are points in $\mathbb{R}^2$ with coordinates in a subfield $K$ of $R$, and $P \ne Q$, $R \ne S$, we have the following:

\begin{itemize}
    \item[] (i) If $L(P,Q) \ne L(R,S)$ and $(\alpha,\beta) \in L(P,Q) \cap L(R,S)$, then $\alpha,\beta \in K$.

    \item[] (ii) If $(\alpha,\beta) \in L(P,Q) \cap C(R,S)$, then $[K(\alpha,\beta):K] \le 2$.

    \item[] (iii) If $C(P,Q) \ne C(R,S)$ and $(\alpha,\beta) \in C(P,Q) \cap C(R,S)$, then $[K(\alpha,\beta):K] \le 2$.
\end{itemize}

pf-1. Let $L(P,Q)=\{(x,y):a_1x+b_1y+c_1=0\}$, $L(R,S)=\{(x,y):a_2x+b_2y+c_2=0\}$. Since $L(P,Q) \ne L(R,S)$ and $(\alpha,\beta) \in L(P,Q) \cap L(R,S)$ they have a unique intersection $(\alpha,\beta)$, and:

\[
    \alpha=\frac{\begin{vmatrix}
        c_1 & b_1 \\
        c_2 & b_2
    \end{vmatrix}}{\begin{vmatrix}
        a_1 & b_1 \\
        a_2 & b_2
    \end{vmatrix}},\ \beta=\frac{\begin{vmatrix}
        a_1 & c_1 \\
        a_2 & c_2
    \end{vmatrix}}{\begin{vmatrix}
        a_1 & b_1 \\
        a_2 & b_2
    \end{vmatrix}}
\]

Since the both numerators and the denominators are in $K$, their quotient is also in $K$, we have $\alpha,\beta \in K$. \\

pf-2. Let $P(a_0,b_0)$, $Q(a_1,b_1)$, $R(c_0,d_0)$, $S(c_1,d_1)$, then $L(P,Q)=\{(a_0,b_0)+t(a_1-a_0,b_1-b_0):t \in \mathbb{R}\}$ and $C(R,S)=\{(x,y):(x-c_0)^2+(y-d_0)^2=(c_1-c_0)^2+(d_1-d_0)^2\}$. When we solve the system of equations for $t$, say the solution is $t_0$, then $t_0$ can only appear a "$\sqrt{\cdot}$" since it is a system of equations of degree $2$ with coefficients in $K$. Thus, $[K(\alpha,\beta):K]=[K(t_0):K] \le 2$. ($\because t_0$ is a root of a polynomial of at most degree $2$) \\

pf-3. Let $C(P,Q)=\{(x,y):x^2+y^2+A_1x+B_1y+C_1=0\}$, $C(R,S)=\{(x,y):x^2+y^2+A_2x+B_2y+C_2=0\}$. The system of equations is equivalent to:

\[
    \left\{\begin{matrix}
        x^2+y^2+A_1x+B_1y+C_1&=0 \\
        (A_1-A_2)x+(B_1-B_2)y+(C_1-C_2)&=0
    \end{matrix}\right.
\]

Then it reduces to (ii). Thus, $[K(\alpha,\beta):K] \le 2$.  \\ \\

\textbf{Theorem 25. }If $\alpha \in \mathbb{R}$ is constructible, then $[\mathbb{Q}(\alpha):\mathbb{Q}]=2^n$ for some $n \in \mathbb{Z}_{\ge 0}$. \\

pf. First, we know that all rational numbers are constructible. We follow the algorithm:

\begin{algorithm}[H]
    \caption{CONSTRUCT-ALPHA($\alpha$)}
    \begin{algorithmic}[1]
        \State{$K_0=\mathbb{Q}$}
        \State{$i=0$}
        \While{$\alpha \notin K_i$}
            \State{Pick any $P,Q,R,S \in K_i$}
            \State{Pick a point $(\alpha_i,\beta_i)$ in one of the intersection of the follwing:}
            \State{$\ \ L(P,Q) \cap L(R,S)$, $L(P,Q) \cap C(R,S)$, $C(P,Q) \cap C(R,S)$}
            \State{$K_{i+1}=K_i(\alpha_i,\beta_i)$}
            \State{$i$ = $i + 1$}
        \EndWhile
        \State{Output $K_i$}
    \end{algorithmic}
\end{algorithm}

Note that $[K_{i+1}:K_i] \le 2$ by lemma 24., so $[K_{i+1}:K_i]=1$ or $[K_{i+1}:K_i]=2$. Thus, $[\mathbb{Q}(\alpha):\mathbb{Q}]=[K_{i}:\mathbb{Q}]=[K_{i}:K_0]=\prod_{j=1}^i[K_j:K_{j-1}]$ is a power of $2$. \\ \\

\textbf{Remark-1. }$\sqrt[3]{2}$ is not constructible. \\

pf. First, consider $f(x)=x^3-2 \in \mathbb{Q}[x]$, then $f(\sqrt[3]{2})=0$ By corollary 21 with $p=2$, $f$ is not reducible in $\mathbb{Q}[x]$. By proposition 2-(iii), $\mathbb{Q}(\sqrt[3]{2}) \simeq \mathbb{Q}[x]/(x^3-2)$. Since $[\mathbb{Q}(\sqrt[3]{2}):\mathbb{Q}]=3 \ne 2^n$ for any $n \in \mathbb{Z}_{\ge 0}$, we have $\sqrt[3]{2}$ is not constructible by theorem 25. \\ \\


\textbf{Remark-2. }We say an angle $\theta$ is constructible if $\cos \theta$ and $\sin \theta$ is constructible. Claim: $60^\circ$ can not be trisected by ruler and compass. \\

pf. We show that $\cos 20^\circ$ is not constructible. Consier $f(x)=8x^3-6x-1$, then $f(\cos 20^\circ)=0$. Claim: $f$ is irreducible in $\mathbb{Q}[x]$, then we can follow the proof for remark-1 to show that $\cos 20^\circ$ is not constructible. \\

We use corollary 19 (irreducibility via reduction mod $p$) and take $p=5$, then $\bar{f}_5(x)=3x^3-x-1$. Since $\deg \bar{f}_5=3$, if it is reducible, then it must have a linear factor and thus have a root. But, $\bar{f}_5(0)=-1$, $\bar{f}_3(1)=\bar{f}_5(2)=1$, $\bar{f}_5(3)=\bar{f}_5(4)=2$, so it is irreducible. By corollary 19, $f$ is irreducible in $\mathbb{Q}[x]$, which gives us $\cos 20^\circ$ is not constructible.

\end{document}