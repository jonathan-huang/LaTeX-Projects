\documentclass[a4paper]{article}
%% Formatting %%
\usepackage[margin=3cm]{geometry}
\usepackage{type1cm, titlesec, fancyhdr, titling}
\usepackage{multicol}
\usepackage[dvipsnames]{xcolor}
\usepackage{ulem}
\usepackage{parskip}
\setlength{\parindent}{2em}
\setlength{\headheight}{15pt}
\setlength{\droptitle}{-1.5cm}
\parindent=24pt
%% Math and Symbols %%
\usepackage{amsmath,amsthm,amssymb, mathtools}
\usepackage{yhmath, faktor, dsfont}
\usepackage{academicons, wasysym, marvosym}
\usepackage[scr]{rsfso} 
\usepackage{latexsym, amsmath, amscd, amsmath, amsthm}
\usepackage{amssymb,amsmath,amsthm,graphicx,dsfont}
\usepackage{hyperref}

%% Enhancement %%
\usepackage{graphicx, tabularx}
\usepackage[shortlabels,inline]{enumitem}
%% TikZ %%
\usepackage{tikz-cd}
\usepackage[breakable]{tcolorbox}
\usetikzlibrary{decorations.pathmorphing}
\usetikzlibrary{calc, arrows,matrix}

%% Other packages %%
\usepackage{amsopn}

%% Traditional Chinese %%
\usepackage{CJKutf8}

%% Math environments %%
\newtheoremstyle{mystyle}
  {6pt}{15pt}% 上下間距
  {}%          內文字體
  {}%              縮排
  {\bf}%       標頭字體
  {.}%       標頭後標點
  {1em}% 內文與標頭距離
  {}% Theorem head spec (can be left empty, meaning 'normal')
\theoremstyle{mystyle}	
\newtheorem{theorem}{Theorem}
\newtheorem{definition}{Definition}
\newtheorem{example}[theorem]{Example}
\newtheorem{exercise}{Exercise}
\newtheorem{solution}{Solution}
\newtheorem{corollary}[theorem]{Corollary}
\newtheorem{property}[theorem]{Property}
\newtheorem{proposition}[theorem]{Proposition}
\newtheorem{lemma}[theorem]{Lemma}
\newtheorem{problem}{Problem}
\newtheorem{answer}{Answer}[section]
\newtheorem{fact}[theorem]{Fact}
\newtheorem*{recall}{Recall}
\newtheorem*{remark}{Remark}
\newtheorem*{claim}{Claim}
\newtheorem*{observation}{Observation}

\begin{document}
\begin{CJK}{UTF8}{bkai}

    \title{%
  \textbf{Math 2213 Introduction to Analysis I} \\
  \vspace{0.5cm}
  \large Homework 6 Due October 17 (Friday), 2025
}
\author{物理、數學三 黃紹凱 B12202004}
\date{\today}

\maketitle

% Problem 1
\begin{problem}[20 pts]
    \begin{definition}[Totally ordered set]\label{def:total-order}
        A \emph{totally ordered set} (or \emph{linearly ordered set}) is a pair $(X, \leq)$ consisting of a nonempty set $X$ together with a binary relation $\leq$ on $X$ satisfying the following properties:
        \begin{enumerate}
            \item \textbf{Reflexivity:} For all $x \in X$, $x \leq x$.
            \item \textbf{Antisymmetry:} For all $x, y \in X$, if $x \leq y$ and $y \leq x$, then $x = y$.
            \item \textbf{Transitivity:} For all $x, y, z \in X$, if $x \leq y$ and $y \leq z$, then $x \leq z$.
            \item \textbf{Totality (or Comparability):} For all $x, y \in X$, either $x \leq y$ or $y \leq x$.
        \end{enumerate}
        A relation $\leq$ satisfying only (1)--(3) is called a \emph{partial order}.  
        If, in addition, (4) holds, the order is said to be \emph{total}, meaning that any two elements of $X$ can be compared.
    \end{definition}

    \begin{definition}[Hausdorff space]
        A topological space $(X,\mathcal{F})$ is called a \emph{Hausdorff space} (or $T_2$ space) if for every pair of distinct points $x,y \in X$ there exist neighborhoods 
        $U,V \in \mathcal{F}$ such that
        \[
        x \in U, \quad y \in V, \quad \text{and } U \cap V = \varnothing.
        \]
    \end{definition}

    \begin{enumerate}
        \item[(a)] Given any totally ordered set $X$ with order relation $\le$, declare a set $V \subseteq X$ to be open if for every $x \in V$ there exists a set $I$, which is an interval $\{y \in X : a < y < b\}$ for some $a, b \in X$, or $\{y \in X : a < y\}$ for some $a \in X$, or  
        $\{y \in X : y < b\}$ for some $b \in X$, or the whole space $X$, which contains $x$ and is contained in $V$. Let $\mathcal{F}$ be the set of all open subsets of $X$. Show that $(X, \mathcal{F})$ is a topology (this is the \emph{order topology} on the totally ordered set $(X, \le)$ which is Hausdorff in the sense of  Definition 2.5.4-2 or the definition above).
        \item[(b)] Show that on the real line $\mathbb{R}$ (with the standard ordering $\le$), the order topology matches the standard topology (i.e., the topology arising from the standard metric).   
        \item[(c)] If instead one defines $V$ to be open if the extended real line $\mathbb{R} \cup \{\pm \infty\}$ has an open set with boundary $\{\pm \infty\}$. Let $\{x_n\}^{\infty}_{n=1} $ be a sequence of numbers in $\mathbb{R} \cup \{\pm \infty\}$. Show that $x_n \to +\infty$ if and only if $\inf_{n \geq N} x_n \to +\infty$, and $x_n \to -\infty$ if and only if $\sup_{n \geq N} x_n \to -\infty$.
    \end{enumerate}
\end{problem}

\begin{solution}
    For each case below, we show that the topology axioms are satisfied.
    \begin{enumerate}[(i)]
        \item $ X $ and $ \varnothing $ are in $ \mathcal{F} $.  
        \item The union of any collection of sets in $ \mathcal{F} $ is also in $ \mathcal{F} $.
        \item The intersection of any finite number of sets in $ \mathcal{F} $ is also in $ \mathcal{F} $.
    \end{enumerate}
    \begin{enumerate}
        \item[(a)] The empty set and $ X $ are subsets of $ X $, so they are both in $ \mathcal{F} $. Let $ \mathcal{C} $ be a collection of open sets in $ \mathcal{F} $, and let $ U = \bigcup_{V \in \mathcal{C}} V $. For any $ x \in U $, there exists some $ V \in \mathcal{C} $ such that $ x \in V $. Since $ V $ is open, there exists an interval $ I $ containing $ x $ such that $ I \subseteq V \subseteq U $. Thus, $ U $ is open, and hence in $ \mathcal{F} $. Now, let $ V_1, V_2, \ldots, V_n $ be a finite collection of open sets in $ \mathcal{F} $, and let $ W = \bigcap_{i=1}^n V_i $. For any $ x \in W $, since each $ V_i $ is open, there exists an interval $ I_i $ containing $ x $ such that $ I_i \subseteq V_i $. Letting $ I = \bigcap_{i=1}^n I_i $, we have that $ I $ is also an interval containing $ x $, and since the intersection of a finite number of intervals is still an interval, we have that $ I \subseteq W $. Thus, $ W $ is open, and hence in $ \mathcal{F} $. Therefore, $(X, \mathcal{F})$ is a topology. The order topology is also Hausdorff: Let $ x, y \in X $ and assume $ x<y $. If there is some $ z \in X $ such that $ x < z < y $, where $ a < b $ if and only $ a \leq b $ and $ a \neq b $, then the intervals $ ( -\infty, z ) $ and $ ( z, +\infty ) $ are disjoint neighborhoods of $ x $ and $ y $, respectively. If there is no such $ z $, then the intervals $ ( -\infty, y ) $ and $ ( x, +\infty ) $ are disjoint neighborhoods of $ x $ and $ y $, respectively.
        \item[(b)] We will show that a set $ U \in \mathbb{R} $ is open in the order topology if and only if it is open in the standard topology. Suppose $ U $ is open in $ (\mathbb{R}, \vert \cdot \vert) $, then for all $ x \in U $, there exists $ \varepsilon>0 $ such that $ B(c, \varepsilon) = (x-\varepsilon, x+\varepsilon)\subseteq U $, and hence $ U $ is open in the order topology. COnversely, suppose $ V $ is open in the order topology. Then there exists an interval $ I \subseteq V $ containing $ x $. An interval is of the form $ (-\infty , a) $, $ (a,b) $, $ (b, \infty) $, or $ X $, so it is open in $ (\mathbb{R}, \vert \cdot \vert ) $. Hence $ V $ is open in the standard topology. 
        \item[(c)] 
    \end{enumerate}
\end{solution}

% Problem 2
\begin{problem}[15 pts]
    \begin{definition}[Metrizable space]\label{def:metrizable}
        A topological space $(X,\mathcal{F})$ is said to be \emph{metrizable} if there exists a metric $d : X \times X \to [0,\infty)$ such that the topology $\mathcal{F}$ coincides with the topology $\mathcal{F}_d$ induced by $d$.  
        That is,
        \[
        \mathcal{F} = \mathcal{F}_d := \{\, U \subseteq X : \forall x \in U, \exists\, \varepsilon > 0 \text{ such that } B_d(x,\varepsilon) \subseteq U \,\},
        \]
        where $B_d(x,\varepsilon) := \{\, y \in X : d(x,y) < \varepsilon \,\}$ denotes the open ball centered at $x$ with radius $\varepsilon$. If no such metric $d$ exists, then $(X,\mathcal{F})$ is said to be \emph{not metrizable}. In other words, its topology cannot arise from any metric on $X$.
    \end{definition}

        \begin{enumerate}
        \item[(a)]  Let $X$ be an uncountable set, and let $\mathcal{F}$ be the collection of all subsets $E$ in $X$ which are either empty or cofinite (which means that $X \setminus E$ is finite).  
        Show that $(X, \mathcal{F})$ is a topology (this is called the \emph{cofinite topology} on $X$) which is not Hausdorff  and is compact.  
        \item[(b)] Show that if $\{V_i : i \in I\}$ is any countable collection of open sets containing $x$, then $\bigcap_i V_i \neq \varnothing$.  
        Use this to show that the cofinite topology cannot be derived from any metric (i.e., $(X, \mathcal{F})$ is not metrizable).  
        (Hint: what is the set $\bigcap_{n=1}^\infty B(x, 1/n)$ equal to in a metric space?)
    \end{enumerate}
\end{problem}

\begin{solution}
    ~
    \begin{enumerate}
        \item[(a)] The complement of $ X $ in $ X $ is empty, so $ \varnothing, X \in \mathcal{F} $. Let $ \mathcal{C} $ be a collection of open sets in $ \mathcal{F} $, and let $ U = \bigcup_{V \in \mathcal{C}} V $. If $ U = \varnothing $, then $ U \in \mathcal{F} $. If $ U \neq \varnothing $, then there exists some $ V \in \mathcal{C} $ such that $ V \neq \varnothing $. Since $ V $ is cofinite, the complement of $ V $ in $ X $, denoted by $ X \setminus V $, is finite. Since $ U \supseteq V $, we have that
        \[ X \setminus U = \bigcap_{V \in \mathcal{C}} (X \setminus V) \]
        is a subset of $ X \setminus V $, and hence is finite. Thus, $ U $ is cofinite, and hence in $ \mathcal{F} $. Now, let $ V_1, V_2, \ldots, V_n $ be a finite collection of open sets in $ \mathcal{F} $, and let $ W = \bigcap_{i=1}^n V_i $. If $ W = \varnothing $, then $ W \in \mathcal{F} $. If $ W \neq \varnothing $, then for each $ i = 1, 2, \ldots, n $, the complement of $ V_i $ in $ X $, denoted by $ X \setminus V_i $, is finite. Since
        \[ X \setminus W = \bigcup_{i=1}^n (X \setminus V_i), \]
        we have that $ X \setminus W $ is a finite union of finite sets, and hence is finite. Thus, $ W \in \mathcal{F} $. Therefore, \emph{$ (X, \mathcal{F}) $ is a topology}. Let $ x, y \in X $ and assume $ x \neq y $. For any neighborhoods $ U $ of $ x $ and $ V $ of $ y $, $ X \setminus U $ and $ X \setminus V $ are finite. Since $ x \neq y $, we have that $ X \setminus (U \cap V) = (X \setminus U) \cup (X \setminus V) $ is finite, and hence $ U \cap V \neq \varnothing $. Therefore, \emph{$ (X, \mathcal{F}) $ is not Hausdorff}. Let $ \{V_i \vert i \in I\} $ be an open cover of $ X $. Since $ X \in \mathcal{F} $, there exists some $ i_0 \in I $ such that $ V_{i_0} \neq \varnothing $. Since $ V_{i_0} $ is cofinite, the complement of $ V_{i_0} $ in $ X $, denoted by $ X \setminus V_{i_0} $, is finite. For each $ x \in X \setminus V_{i_0} $, there exists some $ i_x \in I $ such that $ x \in V_{i_x} $. Letting $ J = \{i_x \vert x \in X \setminus V_{i_0}\} $, we have that $ J $ is a finite set, and
        \[ X = V_{i_0} \cup \bigcup_{i \in J} V_i. \]
        Thus, \emph{$ (X, \mathcal{F}) $ is compact}. 
        \item[(b)] Let $ \{V_i \vert i \in I\} $ be a countable collection of open sets containing $ x $. If there exists some $ i_0 \in I $ such that $ V_{i_0} = \varnothing $, then $ \bigcap_{i \in I} V_i = \varnothing $. If for each $ i \in I $, $ V_i \neq \varnothing $, then for each $ i \in I $, the complement of $ V_i $ in $ X $, denoted by $ X \setminus V_i $, is finite. Since
        \[ X \setminus \bigcap_{i \in I} V_i = \bigcup_{i \in I} (X \setminus V_i), \]
        we have that $ X \setminus \bigcap_{i \in I} V_i $ is a countable union of finite sets, and hence is countable. Since $ X $ is uncountable, we have that $ \bigcap_{i \in I} V_i \neq \varnothing $. Therefore, for any countable collection of open sets containing $ x $, $ \bigcap_i V_i \neq \varnothing $.

        Suppose $ (X, \mathcal{F}) $ were metrizable, then there exists a metric $ d : X \times X \to [0, \infty) $ such that $ \mathcal{F} = \mathcal{F}_d $. For any $ x \in X $, the collection of open balls $ \{B_d(x, 1/n) \vert n = 1, 2, \ldots\} $ satisfies $ \bigcap_{n=1}^\infty B_d(x, 1/n) = \{x\} $. By the definition of a metric, we have $ \bigcap_{n=1}^{\infty} B(x, \frac{1}{n}) = \{x\} $, which is a finite intersection of a countable collection of open sets, a contradiction. Hence, \emph{$ (X, \mathcal{F}) $ is not metrizable}.
    \end{enumerate}

    \begin{remark}
        The cofinite topology is compact but not Hausdorff. Since every metric space is Hausdorff, the result of (b) follows trivially from the contrapositive. In fact, in any infinite set with the cofinite topology, every sequence converges to every point in the space.
    \end{remark}
\end{solution}

% Problem 3
\begin{problem}[15 pts]
    Let $(X, \mathcal{F})$ be a compact topological space. Assume that this space is first countable, which means that for every $x \in X$ there exist countable collections of open sets $V_1, V_2, \ldots$ of neighborhoods of $x$, such that every neighborhood of $x$ contains one of the $V_n$. Show that every sequence in $X$ has a convergent subsequence (see Exercise 1.5.11).
\end{problem}

\begin{solution}
    Let $ (X, \mathcal{F}) $ be a compact first-countable space, and let $ \left(x_n\right)_{n=0}^{\infty} $ be a sequence in $ X $. Assume $ (x_n) $ does not have a convergent subsequence, then by the sequence lemma it does not have a limit point in $ X $. Then for each $ x\in X $, we may an open neighborhood $ U_x $ of $ x $ containing only finitely many terms of $ \left(x_n\right) $. The set $ U = \bigcup_{x\in X} U_x $ is an open cover of $ X $, by compactness there exists a finite subcover $ \{U_{x_i}\}_{i=1}^m $. Since each $ U_{x_i} $ contains only finitely many terms of $ (x_n) $, the union $ \bigcup_{i=1}^m U_{x_i} $ also contains only finitely many terms of $ (x_n) $, a contradiction since $ \bigcup_{i=1}^m U_{x_i} = X $ is at least countable. Therefore, $ (x_n) $ has a convergent subsequence.  

    \begin{lemma}[sequence lemma]
        Given a subset $ A $ of a first-countable space, a point $ x $ lies in $ \overline{A} $ if and only if there exists a sequence $ \left(x_n\right)_{n=0}^{\infty} $ in $ A $ such that $ x_n \to x $.
    \end{lemma}
    \begin{proof}
        Suppose the sequence $ \left(x_n\right)_{n=0}^{\infty} $ converges to $ x \in X$. For any open neighborhood of $ x $, there exists $ N \in \mathbb{N} $ such that for all $ n > N $, $ x_n \in U $. Thus, every open neighborhood of $ x $ intersects $ A $, and hence $ x \in \overline{A} $. Conversely, suppose $ x \in \overline{A} $. Since $ X $ is first-countable, there exists a countable collection of open neighborhoods $ \{V_n\}_{n=1}^{\infty} $ of $ x $ such that every neighborhood of $ x $ contains one of the $ V_n $. For each $ n \in \mathbb{N} $, since $ V_n $ is a neighborhood of $ x $, we have that $ V_n \cap A \neq \varnothing $. Thus, we may pick $ x_n \in V_1 \cap V_2 \cap \cdots \cap V_n \cap U $ for all $ n $. For every open neighborhood $ U $ of $ x $, there exists some $ N \in \mathbb{N} $ such that $ V_N \subseteq U $. Then for all $ n > N $, $ x_n \in V_N \cap V_{N+1} \cap \cdots \cap V_M \subseteq V_N \subseteq U $ with $ M>N $. Therefore, $ x_n \to x $. 
    \end{proof}
\end{solution}

% Problem 4
\begin{problem}[15 pts]
    Let $(X,\mathcal{F})$ be a compact topological space and $(Y,\mathcal{G})$ be a Hausdorff topological space. If $f:X\to Y$ is continuous, then $f$ is a \emph{closed map}; i.e., for every closed subset $F\subseteq X$, the image $f(F)$ is closed in $Y$.
\end{problem}

\begin{solution}
    Suppose $ f $ is continuous. Then for all $ A \subseteq X $, $ f(A) $ is compact in $ Y $ since the continuous image of a compact set is compact. Let $ K \subseteq X $ be a closed subset, then $ K $ is compact since given any open cover of $ K $, we may extend it to an open cover of $ X $ by adding the open set $ X \setminus K $, and since $ X $ is compact, there exists a finite subcover which also covers $ K $. Thus, $ K $ is compact, and $ f(K) $ is compact in $ Y $. Suppose $ f(K) $ is not closed, then there exists some $ y \notin f(K) $ such that all neighborhoods of $ y $ have nonempty intersection with $ f(K) $. Since $ Y $ is Hausdorff, for any $ z \in Y $ and fixed $ y $, there exist neighborhoods $ U_y, U_z $ such that $ U_y \cap U_z = \varnothing $. Then $ \left(U_z\right)_{z \in f(K)} $ is an open cover of $ f(K) $, and we may pick a finite subcover $ \left(U_{z_j}\right)_{j=1}^n $. However, $ U = \bigcap_{j=1}^n U_{y_j} $ is a neighborhood of $ y $ such that $ U \cap \bigcap_{j=1}^n U_{z_j} = U \cap f(K) = \varnothing $, a contradiction. Therefore, $ f(K) $ is closed in $ Y $, and hence $ f $ is a closed map.
\end{solution}

% Problem 5
\begin{problem}[20 pts]
    Let $\{f_n\}$ be a sequence of real-valued continuous functions defined on a compact metric space $S$ and assume that $\{f_n\}$ converges pointwise on $S$ to a limit function $f$. Prove that $f_n \to f$ uniformly on $S$ if, and only if, the following two conditions hold:
    \begin{enumerate}
        \item[(i)] The limit function $f$ is continuous on $S$.
        \item[(ii)] For every $\varepsilon > 0$, there exist $m > 0$ and $\delta > 0$ such that $n > m$ and 
        \[
        |f_k(x) - f(x)| < \delta \implies |f_{k+n}(x) - f(x)| < \varepsilon
        \]
        for all $x \in S$ and all $k = 1, 2, \dots$.
    \end{enumerate}

    \noindent\textbf{Hint.} To prove the sufficiency of (i) and (ii), show that for each $x_0 \in S$ there is a neighborhood $B(x_0, R)$ and an integer $k$ (depending on $x_0$) such that
    \[
    |f_k(x) - f(x)| < \delta \quad \text{if } x \in B(x_0,R).
    \]
    By compactness, a finite set of integers, say $A = \{k_1, \dots, k_r\}$, has the property that for each $x \in S$, some $k \in A$ satisfies $|f_k(x) - f(x)| < \delta$. Uniform convergence is an easy consequence of this fact.
\end{problem}

\begin{solution}
    Suppose $ f_n \to f $ uniformly on $ S $. Then for any $ \frac{\varepsilon}{3} > 0 $, there exists $ N \in \mathbb{N} $ such that for all $ n > N $ and all $ x, y \in S $, $ |f(x) - f(y)| < \frac{\varepsilon}{3} $. Since $ f_n $ is continuous, for each $ \frac{\varepsilon}{3} $ and $ y \in S $, there exists $ \delta $ such that $ \vert f_n(x) - f_n(y) \vert < \frac{\varepsilon}{3} $ whenever $ |x - y| < \delta $. Then, for all $ \varepsilon>0 $ and $ x \in S $, we have $ \vert f(x) - f(y) \vert \leq \vert f(x) - f_n(x) \vert + \vert f_n(x) - f_n(y) \vert + \vert f_n(y) - f(y) \vert < \frac{\varepsilon}{3} + \frac{\varepsilon}{3} + \frac{\varepsilon}{3} = \varepsilon $ for all $ y \in S $, whenever $ \vert x - y \vert < \delta $. Therefore, $ f $ is continuous on $ S $. Now, for any $ \varepsilon > 0 $, let $ m = N $ and let $ \delta = \varepsilon $. Then for all $ n > m $, if $ |f_k(x) - f(x)| < \delta $, then $ |f_{k+n}(x) - f(x)| < \varepsilon $ for all $ x \in S $ and all $ k \in \mathbb{N} $. 
    
    Conversely, suppose (i) and (ii) hold. For each $ x_0 \in S $, since $ f $ and $ f_k $ are continuous, for any $ \frac{\delta}{3} > 0 $, there exists $ r_x > 0 $ such that $ |f(x) - f(x_0)| < \frac{\delta}{3} $ and $ \vert f_k (x) - f_k(x_0) \vert < \frac{\delta}{3} $ whenever $ |x - x_0| < R_{x_0} $. Also, for $ \frac{\delta}{3}>0 $, there exists $ k_{x_0} $ such that $ \vert f_{k_x}(x_0) - f(x_0) \vert < \frac{\delta}{3} $. Then for $ \delta >0 $, $ \vert f_k (x) - f(x) \vert \leq \vert f_k(x) - f_k(x_0) \vert + \vert f_k(x_0) - f(x_0) \vert + \vert f(x_0) - f(x) \vert < \frac{\delta}{3} + \frac{\delta}{3} + \frac{\delta}{3} = \delta $ whenever $ \vert x - x_0 \vert < R_{x_0} $ (hence $ x \in B(x_0, R_{x_0}) $) and $ k > k_{x_0} $. Then $ \bigcup_{x\in S} B(x, R_x) $ is an open cover of $ X $, and by compactness there is a finite subcover $ \{B(x_j, R_{x_j})\}_{j=1}^n $. For all $ x \in S $, there exists $ 1 \leq k \leq n $ such that $ \vert f_k(x) - f(x) \vert < \delta $. Then for all $ \varepsilon > 0 $, by (ii) there exist $ m>0 $ such that $ \vert f_{n+k}(x) - f(x) \vert < \varepsilon $ for all $ n > m $. Therefore, $ f_n \to f $ uniformly on $ S $.
\end{solution}

% Problem 6
\begin{problem}[15 pts]
    The purpose of this exercise is to demonstrate a concrete relationship between continuity and pointwise convergence, and between uniform continuity and uniform convergence. 

    Let $f:\mathbb{R} \to \mathbb{R}$ be a function. For any $a \in \mathbb{R}$, let $f_a : \mathbb{R} \to \mathbb{R}$ be the shifted function defined by
    \[
    f_a(x) := f(x - a).
    \]

    \begin{enumerate}
    \item[(a)] Show that $f$ is continuous if and only if, whenever $(a_n)_{n=0}^\infty$ is a sequence of real numbers which converges to zero, the shifted functions $f_{a_n}$ converge pointwise to $f$.
    \item[(b)] Show that $f$ is uniformly continuous if and only if, whenever $(a_n)_{n=0}^\infty$ is a sequence of real numbers which converges to zero, the shifted functions $f_{a_n}$ converge uniformly to $f$.
    \end{enumerate}
\end{problem}

\begin{solution} 
    ~
    \begin{enumerate}
        \item[(a)] Suppose $ f $ is continuous. Let $ (a_n)_{n=0}^\infty $ be a sequence of real numbers which converges to zero, so there exists $ N \in \mathbb{N} $ such that for all $ n > N $, $ |a_n| < \delta $. For any $ \varepsilon > 0 $ and $ x \in \mathbb{R} $, since $ f $ is continuous at $ x $, there exists $ \delta > 0 $ such that $ |f(x - a_n) - f(x)| < \varepsilon $ whenever $ |a_n| < \delta $. Thus, for all $ n > N $, $ |f_{a_n}(x) - f(x)| = |f(x - a_n) - f(x)| < \varepsilon $, and $ f_{a_n} \to f $ pointwise. Conversely, let $ \left(a_n\right)_{n=0}^{\infty} $ be a sequence of real numbers converging to zero such that $ f_{a_n} \to f $ pointwise. Consider a rearrangement $ \left(a_{\phi (n)} \right)_{n=0}^{\infty} $ of the original sequence such that $ a_{i+1} \leq a_i $ for all $ i $, where $ \phi : \mathbb{N} \to \mathbb{N} $ is a bijection. We still have $ a_{\phi (n)} \to 0 $, so for any $ \delta > 0 $, there exists $ N \in \mathbb{N} $ such that $ \vert a_{\phi (n)} \vert < \delta $ if and only if $ n>N $. For any $ \varepsilon > 0 $ and $ x \in \mathbb{R} $, there exists $ N^{\prime} \in \mathbb{N} $ such that for all $ n > N^{\prime} $, $ |f_{a_n}(x) - f(x)| < \varepsilon $. Let $ M = \operatorname{max} (N, N^{\prime}) $, then for all $ \varepsilon>0 $, there exists $ \delta > 0 $ such that $ \vert f(x - a_n) - f(x) \vert < \varepsilon $ whenever $ n>M $ whenever $ \vert a_{\phi (n)} \vert < \delta $. Since $ \left(a_{\phi (n)}\right)_{n=0}^{\infty} $ is an arrangement and $ \left(a_n\right)_{n=0}^{\infty} $ is arbitrary, $ f $ is continuous. 
        \item[(b)] The proof follows the same logic as above. Suppose $ f $ is uniformly continuous. Let $ (a_n)_{n=0}^\infty $ be a sequence of real numbers which converges to zero, so there exists $ N \in \mathbb{N} $ such that for all $ n > N $, $ |a_n| < \delta $. For any $ \varepsilon > 0 $, since $ f $ is uniformly continuous, there exists $ \delta > 0 $ such that $ |f(x - a_n) - f(x)| < \varepsilon $ whenever $ |a_n| < \delta $ for all $ x \in \mathbb{R} $. Thus, for all $ n > N $, $ |f_{a_n}(x) - f(x)| = |f(x - a_n) - f(x)| < \varepsilon $ for all $ x \in \mathbb{R} $, and $ f_{a_n} \to f $ uniformly. Conversely, let $ \left(a_n\right)_{n=0}^{\infty} $ be a sequence of real numbers converging to zero such that $ f_{a_n} \to f $ uniformly. Consider a rearrangement $ \left(a_{\phi (n)} \right)_{n=0}^{\infty} $ of the original sequence such that $ a_{i+1} \leq a_i $ for all $ i $, where $ \phi : \mathbb{N} \to \mathbb{N} $ is a bijection. We still have $ a_{\phi (n)} \to 0 $, so for any $ \delta > 0 $, there exists $ N \in \mathbb{N} $ such that $ \vert a_{\phi (n)} \vert < \delta $ if and only if $ n>N $. For any $ \varepsilon > 0 $, there exists $ N^{\prime} \in \mathbb{N} $ such that for all $ n > N^{\prime} $, and all $ x \in \mathbb{R} $, we have  $ |f_{a_n}(x) - f(x)| < \varepsilon $. Let $ M = \operatorname{max} (N, N^{\prime}) $, then for all $ \varepsilon>0 $, there exists $ \delta > 0 $ such that for all $ x \in \mathbb{R} $, $ \vert f(x - a_n) - f(x) \vert < \varepsilon $ whenever $ n>M $ whenever $ \vert a_{\phi (n)} \vert < \delta $. Since $ \left(a_{\phi (n)}\right)_{n=0}^{\infty} $ is an arrangement and $ \left(a_n\right)_{n=0}^{\infty} $ is arbitrary, $ f $ is uniformly continuous.
    \end{enumerate}
\end{solution}

\newpage 

\begin{center}
    You can do the following problems to practice. You don't have to submit the following problems.
\end{center}

% Problem 7
\begin{problem}[Optional]
    Let $(X,\mathcal{F})$ be a  topological space and let $B$ be a subset of $X$. Prove the following set equality: 
    \begin{equation*}
        \overline{X\setminus B}=X\setminus Int(B).
    \end{equation*}
\end{problem}

\begin{solution}
    Notice that 
    \begin{align*}
        \overline{X \setminus B} &= (X \setminus B) \cup \partial (X \setminus B) \\
        &= X \setminus (\operatorname{int} B \cup \partial B) \cup \partial (X \setminus B) \\
        &= (X \setminus \operatorname{int} B) \cap ( X \setminus \partial B ) \cup \partial B \\
        &= X \setminus \operatorname{int} B.
    \end{align*}
\end{solution}

% Problem 8
\begin{problem}[Optional]
    Let $(X,\mathcal{F})$ be a topological space and $(Y,\mathcal{G})$ be a Hausdorff topological space. Suppose $f,g:X\rightarrow Y$ are continuous maps. Show that the set $Z=\{x\in X |f(x)=g(x)\}$ is closed in $X$. Give a counterexample if $Y$ is not Hausdorff. Hint: Show $X \setminus Z$ is open.
\end{problem}

\begin{solution}
    Consider $ x \in Z^c = \{ x \in X \,\vert\, f(x) \neq g(x) \} $. Since $ f(x) \neq g(x) $ and $ Y $ is Hausdorff, there exist open neighborhoods $ V_f, V_g \subseteq Y $ of $ f(x), g(x) $, respectively, such that $ V_f \cap V_g = \varnothing $. Then $ x \in f^{-1}(V_f) \cap g^{-1}(V_g) \subseteq Z^c $ is contained in an open set in $ X^c $. Since this is true for all $ x \in Z^c $, we have that $ Z^c $ is open, and hence $ Z $ is closed in $ X $. 
\end{solution}

% Problem 9
\begin{problem}[Optional]
    Suppose $X$ is a topological space, and for every $p\in X$ there exists a continuous function $f:X\rightarrow \mathbb{R}$ such that $f^{pre}(0)=\{p\}$. Show that $X$ is Hausdorff.
\end{problem}

\begin{solution}

\end{solution}

% Problem 10
\begin{problem}[Optional]
    Define two sequences $\{f_n\}$ and $\{g_n\}$ as follows:
    \[
    f_n(x) = x \left(1 + \frac{1}{n}\right), \qquad x \in \mathbb{R}, \quad n = 1, 2, \dots
    \]
    and
    \[
    g_n(x) =
    \begin{cases}
    \dfrac{1}{n}, & \text{if } x = 0 \text{ or } x \text{ is irrational}, \\[8pt]
    b + \dfrac{1}{n}, & \text{if } x \text{ is rational, say } x = \dfrac{a}{b}, \ b > 0.
    \end{cases}
    \]
    Let $h_n(x) = f_n(x) g_n(x)$.
    \begin{enumerate}
        \item[(a)] Prove that both $\{f_n\}$ and $\{g_n\}$ converge uniformly on every bounded interval.
        \item[(b)] Prove that $\{h_n\}$ does not converge uniformly on any bounded interval.
    \end{enumerate}
\end{problem}

\begin{solution}
\end{solution}

% Problem 11
\begin{problem}[Optional]
    Let $(X,d_X)$ be a metric space, and for every integer $n \ge 1$, let $f_n : X \to \mathbb{R}$ be a real-valued function. Suppose that $f_n$ converges pointwise to another function $f : X \to \mathbb{R}$ on $X$ (in this question we give $\mathbb{R}$ the standard metric $d(x,y) = |x - y|$). 

    Let $h:\mathbb{R}\to\mathbb{R}$ be a continuous function. Show that the functions $h \circ f_n$ converge pointwise to $h \circ f$ on $X$, where $h \circ f_n : X \to \mathbb{R}$ is defined by $h \circ f_n(x) := h(f_n(x))$, and similarly for $h \circ f$.
\end{problem}

\begin{solution}

\end{solution}

% Problem 12
\begin{problem}[Optional]
    ~
    \begin{enumerate}
        \item[(a)] Use Problem 5 in the first part to prove the following theorem of Dini:
        \begin{theorem}[Dini's Theorem]  
            If $\{f_n\}$ is a sequence of real-valued continuous functions converging pointwise to a continuous limit function $f$ on a compact set $S$ in a metric space, and if 
            \[
            f_n(x) \ge f_{n+1}(x)
            \quad \text{for each } x \in S \text{ and every } n = 1, 2, \dots,
            \]
            then $f_n \to f$ uniformly on $S$.
        \end{theorem}
        \item[(b)] Let 
        \[
        f_n(x) = \frac{1}{n x + 1}, \qquad 0 < x < 1, \quad n = 1, 2, \dots
        \]
        Prove that $\{f_n\}$ converges pointwise but not uniformly on $(0,1)$.
        \item[(c)] Use the sequence in part (b) to show that compactness of $S$ is essential in Dini's theorem.
    \end{enumerate}
\end{problem}

\end{CJK}
\end{document}