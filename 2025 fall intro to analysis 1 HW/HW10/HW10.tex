\documentclass[a4paper]{article}
%% Formatting %%
\usepackage[margin=3cm]{geometry}
\usepackage{type1cm, titlesec, fancyhdr, titling}
\usepackage{multicol}
\usepackage[dvipsnames]{xcolor}
\usepackage{ulem}
\usepackage{parskip}
\setlength{\parindent}{2em}
\setlength{\headheight}{15pt}
\setlength{\droptitle}{-1.5cm}
\parindent=24pt
%% Math and Symbols %%
\usepackage{amsmath,amsthm,amssymb, mathtools}
\usepackage{yhmath, faktor, dsfont}
\usepackage{academicons, wasysym, marvosym}
\usepackage[scr]{rsfso} 
\usepackage{latexsym, amsmath, amscd, amsmath, amsthm}
\usepackage{amssymb,amsmath,amsthm,graphicx,dsfont}
\usepackage{hyperref}

%% Enhancement %%
\usepackage{graphicx, tabularx}
\usepackage[shortlabels,inline]{enumitem}
%% TikZ %%
\usepackage{tikz-cd}
\usepackage[breakable]{tcolorbox}
\usetikzlibrary{decorations.pathmorphing}
\usetikzlibrary{calc, arrows,matrix}

%% Other packages %%
\usepackage{amsopn}

%% Traditional Chinese %%
\usepackage{CJKutf8}

%% Math environments %%
\newtheoremstyle{mystyle}
  {6pt}{15pt}% 上下間距
  {}%          內文字體
  {}%              縮排
  {\bf}%       標頭字體
  {.}%       標頭後標點
  {1em}% 內文與標頭距離
  {}% Theorem head spec (can be left empty, meaning 'normal')
\theoremstyle{mystyle}	
\newtheorem{theorem}{Theorem}
\newtheorem{definition}{Definition}
\newtheorem{example}[theorem]{Example}
\newtheorem{exercise}{Exercise}
\newtheorem{solution}{Solution}
\newtheorem{corollary}[theorem]{Corollary}
\newtheorem{property}[theorem]{Property}
\newtheorem{proposition}[theorem]{Proposition}
\newtheorem{lemma}[theorem]{Lemma}
\newtheorem{problem}{Problem}
\newtheorem{answer}{Answer}[section]
\newtheorem{fact}[theorem]{Fact}
\newtheorem*{recall}{Recall}
\newtheorem*{remark}{Remark}
\newtheorem*{claim}{Claim}
\newtheorem*{observation}{Observation}

\begin{document}
\begin{CJK}{UTF8}{bkai}

    \title{%
  \textbf{Math 2213 Introduction to Analysis I} \\
  \vspace{0.5cm}
  \large Homework 10 Due November 28 (Friday), 2025
}
\author{物理三 黃紹凱 B12202004}
\date{\today}

\maketitle

\begin{corollary}[3.7.3]
    Let $ [a,b] $ be an interval, and for every integer $ n\geq 1 $, let $ f_n : [a,b] \to \mathbb{R} $ be a continuously differentiable function. Suppose that the series $ \sum_{n=1}^{\infty} \lVert f_n^{\prime} \rVert_{\infty} $ is absolute convergent. Suppose also that $ \sum_{n=1}^{\infty} f_n(x_0) $ is convergent for some $ x_0 \in [a,b] $. Then the series $ \sum_{n=1}^{\infty} f_n $ converges uniformly on $ [a,b] $ to a differentiable function $ f : [a,b] \to \mathbb{R} $, and
    \[
        \frac{\mathrm{d}}{\mathrm{d}x} f(x) = \sum_{n=1}^{\infty} \frac{\mathrm{d}}{\mathrm{d}x}f_n (x).
    \]
\end{corollary}

% Exercise 1
\begin{exercise}[\textbf{Exercise 4.7.8}, 15 points]
    Let $\tan : (-\pi/2,\pi/2) \to \mathbb{R}$ be the tangent function $\tan(x) := \sin(x)/\cos(x)$. Show that $\tan$ is differentiable and monotone increasing, with 
    \[
    \frac{d}{dx}\tan(x) = 1 + \tan(x)^2,
    \]
    and that $\lim_{x\to \pi/2} \tan(x) = +\infty$ and $\lim_{x\to -\pi/2} \tan(x) = -\infty$. Conclude that $\tan$ is in fact a bijection from $(-\pi/2,\pi/2) \to \mathbb{R}$, and thus has an inverse function 
    \[
    \tan^{-1} : \mathbb{R} \to (-\pi/2,\pi/2)
    \]
    (this function is called the \emph{arctangent function}). Show that $\tan^{-1}$ is differentiable and 
    \[
    \frac{d}{dx} \tan^{-1}(x) = \frac{1}{1+x^2}.
    \]
\end{exercise}

\begin{solution}
    On $ (-\frac{\pi}{2}, \frac{\pi}{2}) $, we have $ \cos x > 0 $, so $ \tan x $ is defined on all of its domain and
    \[
    \frac{d}{dx} \tan x = \frac{\cos^2 x + \sin^2 x}{\cos^2 x} = 1 + \tan^2 x > 0.
    \]
    Hence $ \tan x $ is differentiable and monotone increasing. Now we show the limits of $ \tan $ as $ x \to \pm \frac{\pi}{2} $: Since $ \sin $ is continuous and $ \sin \frac{\pi}{2} = 1 $, there exists $ \delta_1 > 0 $ such that $ \sin x > \frac{1}{2} $ whenever $ \vert x - \frac{\pi}{2} \vert < \delta_1 $. Since $ \cos $ is continuous and $ \cos \frac{\pi}{2} = 0 $, for any $ \varepsilon>0 $ there exists $ \delta_2 > 0 $ such that $ \cos x < \varepsilon $ whenever $ \vert x - \frac{\pi}{2} \vert < \delta_2 $. Let $ M>0 $ be arbitray, $ \varepsilon=\frac{1}{2M} $, and $ \delta = \min \{\delta_1, \delta_2\} $. Then, for any $ x $ satisfying $ 0 < \vert x - \frac{\pi}{2} \vert < \delta $, we have
    \[
        \tan x = \frac{\sin x}{\cos x} > \frac{\frac{1}{2}}{\varepsilon} = M \implies \lim_{x \to \frac{\pi}{2}} \tan x = +\infty.
    \]
    By an analogous argument but with $ \sin x < -\frac{1}{2} $ and $ \cos x < \varepsilon $ for $ x $ close to $ -\frac{\pi}{2} $, we have, for arbitrary $ M>0 $, $ \varepsilon = \frac{1}{2M} $, and $ \tilde{\delta} = \min \{\tilde{\delta_1}, \tilde{\delta_2}\} $, that for any $ x $ satisfying $ 0 < \vert x + \frac{\pi}{2} \vert < \tilde{\delta} $, 
    \[
        \tan x = \frac{\sin x}{\cos x} < \frac{-\frac{1}{2}}{\varepsilon} = -M \implies \lim_{x \to -\frac{\pi}{2}} \tan x = -\infty.
    \]
    Since $ \tan $ is monotone increasing, it is injective. By the intermediate value theorem, it is also surjective onto $ \mathbb{R} $. Thus $ \tan $ is a bijection from $ \left(-\frac{\pi}{2}, \frac{\pi}{2}\right) $ to $ \mathbb{R} $, and has an inverse function $ \tan^{-1} : \mathbb{R} \to \left(-\frac{\pi}{2}, \frac{\pi}{2}\right) $. Differentiating both sides of the identity $ \tan \left(\tan^{-1} x \right) = x $, we have 
    \[
    \sec^2 \left(\tan^{-1} x \right) \cdot \frac{d}{dx} \tan^{-1} x = 1, 
    \]
    hence, 
    \[
        \frac{\mathrm{d}}{\mathrm{d}x} \tan^{-1} x = \cos^2 \left(\tan^{-1} x \right) = \frac{1}{1 + \tan^2 \left(\tan^{-1} x \right)} = \frac{1}{1 + x^2}. 
    \]
\end{solution}

% Exercise 2
\begin{exercise}[\textbf{Exercise 4.7.9}, 15 points]
    Recall the arctangent function $\tan^{-1}$ from Exercise 4.7.8. By modifying the proof of Theorem 4.5.6(e), establish the identity
    \[
    \tan^{-1}(x) = \sum_{n=0}^{\infty} \frac{(-1)^n x^{2n+1}}{2n+1}
    \]
    for all $x \in (-1,1)$. Using Abel's theorem (Theorem 4.3.1) to extend this identity to the case $x=1$, conclude in particular the identity
    \[
    \pi = 4 - \frac{4}{3} + \frac{4}{5} - \frac{4}{7} + \cdots 
    = 4 \sum_{n=0}^{\infty} \frac{(-1)^n}{2n+1}.
    \]
    (Note that the series converges by the alternating series test, Proposition 7.2.11.)  
    Conclude in particular that $4 - \tfrac{4}{3} < \pi < 4$.  
    (One can of course compute $\pi = 3.1415926\ldots$ to much higher accuracy, though if one wishes to do so it is advisable to use a different formula than the one above, which converges very slowly.)
\end{exercise}

\begin{solution}
    For $ x \in (-1, 1) $, we have that for any $ r < 1 $,
    \[
    \frac{\mathrm{d}}{\mathrm{d}x}\tan^{-1} x = \frac{1}{1+x^2} \rightrightarrows \sum_{n=0}^{\infty} (-1)^n x^{2n}
    \]
    on $ [-r, r] $. Since $ \tan^{-1}(0) = 0 $, integrating both sides from $ 0 $ to $ x $, we have
    \[
    \tan^{-1} x = \int_0^x \frac{\mathrm{d}t}{1+t^2} = \int_0^x \sum_{n=0}^{\infty} (-1)^n t^{2n} dt = \sum_{n=0}^{\infty} \frac{(-1)^n x^{2n+1}}{2n+1},
    \]
    since $ (-1)^n t^{2n} $ converges uniformly on compact subsets of $(-1,1)$ and is Riemann integrable for each $ n $. The resulting series converges by the alternating series test. Hence, by Abel's Theorem, we have 
    \[
    \frac{\pi}{4} = \tan^{-1} 1 = \lim_{x \to 1^-} \tan^{-1} x = \sum_{n=0}^{\infty} \frac{(-1)^n}{2n+1}.
    \]
    Therefore, 
    \[
    \pi = 4 - \frac{4}{3} + \frac{4}{5} - \frac{4}{7} + \cdots = 4 \sum_{n=0}^{\infty} \frac{(-1)^n}{2n+1}, 
    \]
    and $ 4 - \frac{4}{3} < \pi < 4 $ since the series is alternating with decreasing terms. 
\end{solution}

% Exercise 3
\begin{exercise}[\textbf{Exercise 4.7.10}, 30 points]
    Let $f : \mathbb{R} \to \mathbb{R}$ be the function
    \[
    f(x) := \sum_{n=1}^{\infty} 4^{-n} \cos(32^n \pi x).
    \]

    \begin{enumerate}
    \item[(a)] Show that this series is uniformly convergent, and that $f$ is continuous.

    \item[(b)] Show that for every integer $j$ and every integer $m \ge 1$, we have
    \[
    \left| f\!\left( \frac{j+1}{32^m} \right) - 
        f\!\left( \frac{j}{32^m} \right) \right| 
    \ge 4^{-m}.
    \]
    \textit{Hint: use the identity}
    \[
    \sum_{n=1}^{\infty} a_n 
    = \left( \sum_{n=1}^{m-1} a_n \right)
    + a_m 
    + \sum_{n=m+1}^{\infty} a_n
    \]
    \textit{for certain sequences } $a_n$. Also, use the fact that the cosine function is periodic with period $2\pi$, as well as the geometric series formula $\sum_{n=0}^{\infty} r^n = \frac{1}{1-r}$ for any $|r|<1$. Finally, you will need the inequality $|\cos(x)-\cos(y)| \le |x-y|$ for any real numbers $x$ and $y$; this can be proven by using the mean value theorem.

    \item[(c)] Using (b), show that for every real number $x_0$, the function $f$ is not differentiable at $x_0$. Hint: for every $x_0$ and every $m \ge 1$, there exists an integer $j$ such that $j \le 32^m x_0 \le j+1$, thanks to Exercise 5.4.3.

    \item[(d)] Explain briefly why the result in (c) does not contradict Corollary 3.7.3.
    \end{enumerate}
\end{exercise}

\begin{solution}
    ~

    \begin{enumerate}
        \item[(a)] Since $ |\cos(32^n \pi x)| \le 1 $, we have
        \[
        |4^{-n} \cos(32^n \pi x)| \le 4^{-n}.
        \]
        The series $\sum_{n=1}^{\infty} 4^{-n}$ is a geometric series with ratio $\frac{1}{4}$, which converges. Hence, by the Weierstrass M-test, the series defining $f(x)$ converges uniformly. Since each term $4^{-n} \cos(32^n \pi x)$ is continuous, the uniform limit $f$ is also continuous.

        \item[(b)] We can write
        \[
        f\left( \frac{j+1}{32^m} \right) - f\left( \frac{j}{32^m} \right)
        = \sum_{n=1}^{\infty} 4^{-n} \left[ \cos\left( 32^n \pi \frac{j+1}{32^m} \right) - \cos\left( 32^n \pi \frac{j}{32^m} \right) \right].
        \]
        For $n > m$, we have
        \[
        \cos\left( 32^n \pi \frac{j+1}{32^m} \right) = \cos\left( 32^n \pi \frac{j}{32^m} + 32^{n-m} \pi \right) = \cos\left( 32^n \pi \frac{j}{32^m} \right),
        \]
        so we are left with only the first $m$ terms, which can be split as
        \begin{align*}
            f\left(\frac{j+1}{32^m}\right) - f\left(\frac{j}{32^m}\right)
            &= \sum_{n=1}^{m-1} 4^{-n} \left[ \cos\left( 32^n \pi \frac{j+1}{32^m} \right) - \cos\left( 32^n \pi \frac{j}{32^m} \right) \right] \\
            &\quad + 4^{-m} \left[ \cos\left( \pi (j+1) \right) - \cos\left( j \pi \right) \right] \\
            &\equiv R_m (j) + 4^{-m} \left[ \cos\left( \pi (j+1) \right) - \cos\left( j \pi \right) \right].
        \end{align*}
        The first sum $ R_m(j) $ can be bounded using the inequality $|\cos(x) - \cos(y)| \le |x - y|$:
        \begin{align*}
            R_m (j) &= \left| \sum_{n=1}^{m-1} 4^{-n} \left[ \cos\left( \pi \frac{j+1}{32^{m-n}} \right) - \cos\left( \pi \frac{j}{32^{m-n}} \right) \right] \right| \\
            &\le \sum_{n=1}^{m-1} 4^{-n} \left| \pi \frac{j+1}{32^{m-n}} - \pi \frac{j}{32^{m-n}} \right| \\
            &= \sum_{n=1}^{m-1} \frac{4^{-n} \pi}{32^{m-n}} = \frac{\pi}{32^m} \sum_{n=1}^{m-1} 8^n \\
            &= \frac{\pi}{32^m} \cdot \frac{8}{7} (8^{m-1} - 1) = \frac{\pi}{7} \left( 4^{-m+1} - \frac{1}{32^m} \right) < \frac{4\pi}{7} 4^{-m}.
        \end{align*}
        and since $ \vert \cos \left((j+1) \pi\right) - \cos (j \pi) \vert = 2 $, we have 
        \begin{align*}
            \left| f\left( \frac{j+1}{32^m} \right) - f\left( \frac{j}{32^m} \right) \right| 
            &\ge \left| 4^{-m} \left[ \cos\left( \pi (j+1) \right) - \cos\left( j \pi \right) \right] \right| - |R_m (j)| \\
            &\ge 2 \cdot 4^{-m} - \frac{4\pi}{7} 4^{-m} = \left( 2 - \frac{4\pi}{7} \right) 4^{-m} > 4^{-m}.
        \end{align*}

        \item[(c)] For $ x_0 \in \mathbb{R} $, by Exercise 5.4.3, for each $ m \ge 1 $, there exists an integer $ j $ such that $ j \le 32^m x_0 \le j+1 $. Then,
        \begin{align*}
            \left| \dfrac{f\left( \dfrac{j+1}{32^m} \right) - f\left( \dfrac{j}{32^m} \right)}{\dfrac{j+1}{32^m} - \dfrac{j}{32^m}} \right| 
            &= 32^m \left| f\left( \frac{j+1}{32^m} \right) - f\left( \frac{j}{32^m} \right) \right| \\
            &\ge 32^m \cdot 4^{-m} = 8^m.
        \end{align*}
        As $ \frac{1}{32^m} \to 0 $, or, $ m \to \infty $, we have $ 8^m \to \infty $. Thus, the difference quotient does not converge, and by the definition of the derivative $ f $ is not differentiable at $ x_0 $.
        
        \item[(d)] Refer to the statement of Corollary 3.7.3 at the beginning of this document. The Corollary requires that $ \sum_{n=1}^{\infty} \lVert f_n^{\prime} \rVert $ converges absolutely. However,
        \[
            \vert f_n^{\prime} \vert = \left\vert 8^n \pi \sin \left(32^n \pi x\right) \right\vert \implies \lVert f_n^{\prime} \rVert_{\infty} = \sup_{x \in \mathbb{R}} \left\vert 8^n \pi \sin \left(32^n \pi x\right) \right\vert = 8^n \pi,
        \]
        which is unbounded for $ n \in \mathbb{N} $, and hence $ \sum_{n=1}^{\infty} \lVert f_n^{\prime} \rVert_{\infty} $ does not converge absolutely. Therefore, the result in (c) does not contradict Corollary 3.7.3.
    \end{enumerate}
\end{solution}

% Exercise 4
\begin{exercise}[20 points]
    ~
  
    \begin{enumerate}
        \item[(a)]  Prove that 
        \[
        (\cos\theta + i\sin\theta)^{n} = \cos(n\theta) + i\sin(n\theta)
        \] 
        for all integers $n$ and all real $\theta$.This is the classical \emph{DeMoivre’s theorem}.

        \item[(b)]
        By equating imaginary parts in DeMoivre's formula, prove that
        \[
        \sin n\theta = \sin^n \theta \left\{\binom{n}{1} \cot^{\,n-1}\theta - \binom{n}{3} \cot^{\,n-3}\theta + \binom{n}{5} \cot^{\,n-5}\theta-\ \cdots \right\}.
        \]

        \item[(c)]
        If $0<\theta<\pi/2$, prove that
        \[
        \sin(2m+1)\theta
        = \sin^{\,2m+1}\!\theta \; P_m(\cot^2\theta)
        \]
        where $P_m$ is the polynomial of degree $m$ given by
        \[
        P_m(x) = \binom{2m+1}{1} x^m - \binom{2m+1}{3} x^{m-1} + \binom{2m+1}{5} x^{m-2} - \cdots .
        \]

        Use this to show that $P_m$ has zeros at the $m$ distinct points
        \[
        x_k = \cot^2\!\left(\frac{\pi k}{2m+1}\right),
        \qquad k = 1,2,\dots,m.
        \]

        \item[(d)]
        Show that the sum of the zeros of $P_m$ is given by
        \[
            \sum_{k=1}^m \cot^2\!\left(\frac{\pi k}{2m+1}\right) = \frac{m(2m-1)}{3}.
        \]
    \end{enumerate}
\end{exercise}

\begin{solution}
    ~

    \begin{enumerate}
        \item[(a)] By Theorem 4.7.2 (f), for $ \theta \in \mathbb{R} $ we have $ e^{i \theta} = \cos \theta + i \sin \theta $. Raising both sides to the power $n$, we get $ (\cos \theta + i \sin \theta)^n = \left(e^{i \theta}\right)^n = e^{in \theta} =  = \cos(n\theta) + i \sin(n\theta) $.
        
        \item[(b)] Expanding $ (\cos \theta + i \sin \theta)^n $ using the binomial theorem gives 
        \[
            (\cos \theta + i \sin \theta)^n = \sum_{k=0}^n \binom{n}{k} \cos^{n-k} \theta (i \sin \theta)^k = \sum_{k=0}^n \binom{n}{k} i^k \cos^{n-k} \theta \sin^k \theta.
        \]
        The imaginary part is given by the sum over odd $k$, hence, 
        \begin{align*}
            \sin (n \theta) &= \sum_{\substack{k=1, \, k \text{ odd}}}^n \binom{n}{k} (-1)^{\frac{k-1}{2}} \cos^{n-k} \theta \sin^k \theta \\
            &= \sin^n \theta \sum_{\substack{k=1, \, k \text{ odd}}}^n \binom{n}{k} (-1)^{\frac{k-1}{2}} \cot^{n-k} \theta \\
            &= \sin^n \theta \left\{ \binom{n}{1} \cot^{n-1} \theta - \binom{n}{3} \cot^{n-3} \theta + \binom{n}{5} \cot^{n-5} \theta - \cdots \right\}.
        \end{align*}

        \item[(c)] For $ n = 2m + 1 $, we have
        \[
            \sin (2m + 1) \theta = \sin^{2m + 1} \theta \left\{ \binom{2m + 1}{1} \cot^{2m} \theta - \binom{2m + 1}{3} \cot^{2m - 2} \theta + \cdots \right\}
        \]
        by the result of (b). Hence, by the definition of $ P_m(x) $, we have
        \[
            \sin (2m + 1) \theta = \sin^{2m + 1} \theta \; P_m(\cot^2 \theta).
        \]
        Since $ 0 < \theta < \frac{\pi}{2} $, we have $ \sin (2m + 1) \theta = 0 $ when $ \theta = \frac{\pi k}{2m + 1} $ for $ k = 1, 2, \ldots, m $. Note that at these points, $ \sin \theta \neq 0 $. Thus, $ P_m (\cot^2 \theta) = \sin (2m+1)\theta / \sin^{2m+1}\theta = 0 $ at these points, so $ P_m $ has zeros at
        \[
            x_k = \cot^2 \left( \frac{\pi k}{2m + 1} \right), \quad k = 1, 2, \ldots, m.
        \]

        \item[(d)] By Vieta's formula (根與係數), the sum of the zeros of $ P_m(x) $ is given by 
        \[
            \sum_{k=1}^m \cot^2 \left(\frac{\pi k}{2m + 1}\right) = - \left. \left(-\binom{2m+1}{3}\right) \middle/ \binom{2m+1}{1} \right. = \frac{m(2m-1)}{3}.
        \]
    \end{enumerate}
\end{solution}

% Exercise 5
\begin{exercise}[20 points]
    This exercise outlines a simple proof of the formula $\zeta(2)=\sum_{n=1}^{\infty}\frac{1}{n^2}=\pi^2/6$. Start with the inequality
    \[
    \sin x < x < \tan x, \qquad 0<x<\frac{\pi}{2},
    \]
    take reciprocals, and square each member to obtain
    \[
    \cot^2 x < \frac{1}{x^2} < 1 + \cot^2 x.
    \]

    Now put $x = \dfrac{k\pi}{2m+1}$, where $k$ and $m$ are integers with $1 \le k \le m$,
    and sum on $k$ to obtain
    \[
    \sum_{k=1}^m \cot^2\!\left( \frac{k\pi}{2m+1} \right)
    < \frac{(2m+1)^2}{\pi^2} \sum_{k=1}^m \frac{1}{k^2}
    < m + \sum_{k=1}^m \cot^2\!\left( \frac{k\pi}{2m+1} \right).
    \]

    Use the formula in problem 4(d) to deduce the inequality
    \[
    \frac{m(2m-1)\pi^2}{3(2m+1)^2}
    < \sum_{k=1}^m \frac{1}{k^2}
    < \frac{2m(m+1)\pi^2}{3(2m+1)^2}.
    \]

    Now let $m\to\infty$ to obtain $\zeta(2) = \pi^2/6$.
\end{exercise}

\begin{solution}
    Following the steps in the problem statement, we have 
    \[
        \sum_{k=1}^m \cot^2 \left( \frac{k \pi}{2m + 1} \right) < \frac{(2m + 1)^2}{\pi^2} \sum_{k=1}^m \frac{1}{k^2} < m + \sum_{k=1}^m \cot^2 \left( \frac{k \pi}{2m + 1} \right). 
    \]
    By Exercise 4(d), we have
    \[
        \frac{m(2m - 1)}{3} < \frac{(2m + 1)^2}{\pi^2} \sum_{k=1}^m \frac{1}{k^2} < m + \frac{m(2m - 1)}{3}.
    \]
    Rearranging gives
    \[
        \frac{m(2m - 1) \pi^2}{3(2m + 1)^2} < \sum_{k=1}^m \frac{1}{k^2} < \frac{2m(m + 1) \pi^2}{3(2m + 1)^2}.
    \]
    Take the limit as $ m \to \infty $, we have
    \[
        \frac{m(2m-1)\pi^2}{3(2m+1)^2} \to \frac{\pi^2}{6}, \quad \frac{2m(m+1)\pi^2}{3(2m+1)^2} \to \frac{\pi^2}{6},
    \]
    hence by the Squeeze Theorem, we have that 
    \[
        \zeta(2) = \lim_{m \to \infty} \sum_{k=1}^m \frac{1}{k^2} = \sum_{k=1}^{\infty} \frac{1}{k^2} = \frac{\pi^2}{6}. 
    \]
\end{solution}

\end{CJK}
\end{document}