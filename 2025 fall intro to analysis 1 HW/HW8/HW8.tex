\documentclass[a4paper]{article}
%% Formatting %%
\usepackage[margin=3cm]{geometry}
\usepackage{type1cm, titlesec, fancyhdr, titling}
\usepackage{multicol}
\usepackage[dvipsnames]{xcolor}
\usepackage{ulem}
\usepackage{parskip}
\setlength{\parindent}{2em}
\setlength{\headheight}{15pt}
\setlength{\droptitle}{-1.5cm}
\parindent=24pt
%% Math and Symbols %%
\usepackage{amsmath,amsthm,amssymb, mathtools}
\usepackage{yhmath, faktor, dsfont}
\usepackage{academicons, wasysym, marvosym}
\usepackage[scr]{rsfso} 
\usepackage{latexsym, amsmath, amscd, amsmath, amsthm}
\usepackage{amssymb,amsmath,amsthm,graphicx,dsfont}
\usepackage{hyperref}

%% Enhancement %%
\usepackage{graphicx, tabularx}
\usepackage[shortlabels,inline]{enumitem}
%% TikZ %%
\usepackage{tikz-cd}
\usepackage[breakable]{tcolorbox}
\usetikzlibrary{decorations.pathmorphing}
\usetikzlibrary{calc, arrows,matrix}

%% Other packages %%
\usepackage{amsopn}

%% Traditional Chinese %%
\usepackage{CJKutf8}

%% Math environments %%
\newtheoremstyle{mystyle}
  {6pt}{15pt}% 上下間距
  {}%          內文字體
  {}%              縮排
  {\bf}%       標頭字體
  {.}%       標頭後標點
  {1em}% 內文與標頭距離
  {}% Theorem head spec (can be left empty, meaning 'normal')
\theoremstyle{mystyle}	
\newtheorem{theorem}{Theorem}
\newtheorem{definition}{Definition}
\newtheorem{example}[theorem]{Example}
\newtheorem{exercise}{Exercise}
\newtheorem{solution}{Solution}
\newtheorem{corollary}[theorem]{Corollary}
\newtheorem{property}[theorem]{Property}
\newtheorem{proposition}[theorem]{Proposition}
\newtheorem{lemma}[theorem]{Lemma}
\newtheorem{problem}{Problem}
\newtheorem{answer}{Answer}[section]
\newtheorem{fact}[theorem]{Fact}
\newtheorem*{recall}{Recall}
\newtheorem*{remark}{Remark}
\newtheorem*{claim}{Claim}
\newtheorem*{observation}{Observation}

\begin{document}
\begin{CJK}{UTF8}{bkai}

    \title{%
  \textbf{Math 2213 Introduction to Analysis I} \\
  \vspace{0.5cm}
  \large Homework 8 Due November 14 (Friday), 2025
}
\author{物理、數學三 黃紹凱 B12202004}
\date{\today}

\maketitle

% Exercise 1
\begin{exercise}[25 points]
    Give examples of a formal power series
    \[
    \sum_{n=0}^{\infty} c_n x^n
    \]
    centered at \(0\) with radius of convergence \(1\), which
    \begin{enumerate}[(a)]
        \item diverges at both \(x=1\) and \(x=-1\);
        \item diverges at \(x=1\) but converges at \(x=-1\);
        \item converges at \(x=1\) but diverges at \(x=-1\);
        \item converges at both \(x=1\) and \(x=-1\);
        \item converges pointwise on \((-1,1)\), but does not converge uniformly on \((-1,1)\).
    \end{enumerate}
\end{exercise}

\begin{solution}
    ~

    \begin{enumerate}[(a)]
        \item The series 
        \[
            \sum_{n=0}^{\infty} x^{2n} = 1 + x^2 + x^4 + x^6 + \cdots = \frac{1}{1-x^2}
        \]
        has radius of convergence \(1\) and diverges at both \(x=1\) and \(x=-1\).
        \item The series
        \[
            \sum_{n=0}^{\infty} x^n = 1 + x + x^2 + x^3 + \cdots = \frac{1}{1-x} 
        \]
        has radius of convergence \(1\), diverges at \(x=1\) but converges at \(x=-1\).
        \item The series
        \[
            \sum_{n=0}^{\infty} (-1)^n x^n = 1 - x + x^2 - x^3 + \cdots = \frac{1}{1+x} 
        \]
        has radius of convergence \(1\), diverges at \(x=1\) but converges at \(x=-1\).
        \item The series
        \[
            \sum_{n=0}^{\infty} (-1)^n x^{2n} = 1 - x^2 + x^4 - x^6 + \cdots = \frac{1}{1+x^2}   
        \]
        has radius of convergence \(1\) and converges at both \(x=1\) and \(x=-1\).
        \item The series 
        \[
            \sum_{n=0}^{\infty} x^n = 1 + x + x^2 + x^3 + \cdots = \frac{1}{1-x} 
        \] 
        converges pointwise on \((-1,1)\) as shown above. However, since it is unbounded, it does not converge uniformly on \((-1,1)\).
    \end{enumerate}
\end{solution}

% Exercise 2
\begin{exercise}[\textbf{Tao II Ex. 4.2.7.}, 25 points]
    Let \(m \ge 0\) be a positive integer, and let \(0 < r\) be real numbers. Prove the identity
    \[
    \frac{r}{\,r-x\,} = \sum_{n=0}^{\infty} x^n r^{-n}
    \]
    for all \(x \in (-r, r)\). Using Proposition 4.2.6, conclude the identity
    \[
    \frac{r}{(r-x)^{\,m+1}}
    = \sum_{n=m}^{\infty} \frac{n!}{m!(n-m)!}\, x^{\,n-m} r^{-n}
    \]
    for all integers \(m \ge 0\) and all \(x \in (-r, r)\). Also explain why the series on the right-hand side is absolutely convergent.
\end{exercise}

\begin{solution}
    First we show that the identity is true: 
    \[
        \frac{r}{r-x} = \frac{1}{1 - \frac{x}{r}} = \sum_{n=0}^{\infty} \left(\frac{x}{r}\right)^n = \sum_{n=0}^{\infty} x^n r^{-n},  
    \]
    for $ -1 < \frac{x}{r} < 1 $, which is equivalent to \(x \in (-r, r)\). By Proposition 4.2.6, there exists $ r>0 $ such that the power series on the right is $ m $ times differentiable on $ (-r, r) $. Differentiate both sides $ m $ times, we have
    \[
        \frac{d^m}{dx^m} \left( \frac{r}{r-x} \right) = m! \, \frac{r}{(r-x)^{m+1}},
    \]
    \[
        \frac{d^m}{dx^m} \left( \sum_{n=0}^{\infty} x^n r^{-n} \right) = \sum_{n=0}^{\infty} r^{-(n+m)} \frac{(n+m)!}{n!} x^{n} = \sum_{n=m}^{\infty} \frac{n!}{m!(n-m)!} x^{n-m} r^{-n}.
    \]
    Equating both sides gives the desired identity for all $ m\geq 0 $ and $ x \in (-r,r) $. 
\end{solution}

% Exercise 3
\begin{exercise}[25 points]
    Let \(E\) be a subset of \(\mathbb{R}\), let \(a\) be an interior point of \(E\), and let \(f:E\to\mathbb{R}\) be a function which is real analytic at \(a\) and has a power series expansion
    \[
    f(x)=\sum_{n=0}^{\infty} c_n (x-a)^n
    \]
    at \(a\) which converges on the interval \((a-r,\, a+r)\). Let \((b-s,\, b+s)\) be any subinterval of \((a-r,\, a+r)\) for some \(s>0\).
    \begin{enumerate}
        \item[(a)] Prove that \(|a-b| \le r-s\), so in particular \(|a-b| < r\).
        \item[(b)] Show that for every \(0<\varepsilon<r\), there exists a \(C>0\) such that \(|c_n| \le C(r-\varepsilon)^{-n}\) for all integers \(n\ge 0\).  
        \emph{(Hint: what do we know about the radius of convergence of the series \(\sum_{n=0}^{\infty} c_n(x-a)^n\)?)}
        \item[(c)] Show that the numbers \(d_0,d_1,\ldots\), given by the formula
        \[
        d_m := \sum_{n=m}^{\infty} \frac{n!}{m!(n-m)!}(b-a)^{\,n-m} c_n \qquad \text{for all integers } m\ge 0,
        \]
        are well-defined, in the sense that the above series is absolutely convergent. \emph{(Hint: use (b) and the comparison test, Corollary 7.3.2, followed by Exercise 4.2.7.)}
        \item[(d)] Show that for every \(0<\varepsilon<s\) there exists a \(C>0\) such that
        \[
        |d_m| \le C(s-\varepsilon)^{-m}
        \]
        for all integers \(m\ge 0\). \emph{(Hint: use the comparison test, and Exercise 4.2.7.)}
        \item[(e)] Show that the power series \(\sum_{m=0}^{\infty} d_m (x-b)^m\) is absolutely convergent for \(x \in (b-s,\, b+s)\) and converges to \(f(x)\).  
        (You may need Fubini’s theorem for infinite series, Theorem 8.2.2 of \emph{Analysis I}, as well as Exercise 4.2.5. One may also need to use a variant of the \(d_m\) in which the \(c_n\) are replaced by \(|c_n|\).) Note. You can use Exercise 4.2.5. Let \(a, b\) be real numbers, and let \(n \ge 0\) be an integer. Prove the identity
        \[
        (x-a)^n = \sum_{m=0}^{n} \frac{n!}{m!(n-m)!}(b-a)^{\,n-m}(x-b)^m
        \]
        for any real number \(x\).
        \item[(f)] Conclude that \(f\) is real analytic at \(b\), and thus analytic at every point in \((a-r,\, a+r)\).
    \end{enumerate}
\end{exercise}

  
\begin{solution}

\end{solution}

% Exercise 4
\begin{exercise}[25 points]
    ~ 
    
    \begin{enumerate}
        \item[(a)] If each \(a_n \ge 0\) and if \(\sum a_n\) diverges, show that \(\sum a_n x^n \to +\infty\) as \(x \to 1^{-}\).  
        (Assume \(\sum a_n x^n\) converges for \(|x|<1\).)
        \item[(b)] If each \(a_n \ge 0\) and if \(\lim_{x \to 1^{-}} \sum a_n x^n\) exists and equals \(A\), prove that \(\sum a_n\) converges and has sum \(A\).  
    \end{enumerate}
\end{exercise}  
  
\begin{solution}

\end{solution}

\newpage

\begin{center}
    You can do the following problems to practice. \\
    You don't have to submit the following problems.
\end{center}

% Exercise 5
\begin{exercise}[Optional]
    Let the power series
    \[
    f(x)=\sum_{n=0}^{\infty} a_n x^n
    \]
    converge for $-1<x<1$. For each $n$, define the partial sum
    \[
    s_n = \sum_{k=0}^{n} a_k, \qquad \sigma_n = \sum_{k=0}^{n} k|a_k|.
    \]
    Suppose that $\displaystyle \lim_{x\to 1^-} f(x) = S$ and $\displaystyle \lim_{n\to\infty} n a_n = 0$. In this problem, you will show that the series $\sum_{n=0}^{\infty} a_n$ converges and that its sum is $S$.
    \begin{enumerate}
        \item[(a)] \textbf{Preliminary Identity.} Show that for any $x \in (0,1)$,
        \[
        s_n - f(x) = \sum_{k=0}^{n} a_k (1-x^k) - \sum_{k=n+1}^{\infty} a_k x^k.
        \]
        \item[(b)] \textbf{Bounding the First Sum.} Show that for all $m\ge 1$ and $x\in(0,1)$,
        \[
        1 + x + \cdots + x^{m-1} \le \frac{1}{1-x},
        \]
        and deduce that
        \[
        |1 - x^k| = (1-x)(1 + x + \cdots + x^{k-1}) \le k(1-x).
        \]
        \item[(c)] \textbf{Application of the Bound.} Use part (b) to prove that for $x \in (0,1)$,
        \[
        \left| \sum_{k=0}^{n} a_k (1-x^k) \right| \le (1-x)\sigma_n.
        \]
        \item[(d)] \textbf{Estimate of the Tail.} Use the assumption that $\lim_{n\to \infty} n|a_n| = 0$ to show that for any $\varepsilon>0$, there exists $N$ such that for all $n\ge N$,
        \[
        n|a_n| < \frac{\varepsilon}{3}.
        \]
        Then prove that for such $n$ and all $x \in (0,1)$,
        \[
        \left| \sum_{k=n+1}^{\infty} a_k x^k \right| \le \frac{\varepsilon}{3(1-x)}.
        \]
        \item[(e)] \textbf{Putting the Estimates Together.} Combine parts (a)--(d) to show that for all $n \ge N$ and $x \in (0,1)$,
        \[
        |s_n - S|
        \le |f(x) - S| + (1-x)\sigma_n + \frac{\varepsilon}{3(1-x)}.
        \]
        \item[(f)] \textbf{Strategic Choice of $x$.} Let $x = x_n = 1 - \frac{1}{n}$. Use part (e) to show that when $n$ is sufficiently large,
        \[
        |s_n - S| < \varepsilon.
        \]
        Conclude that $s_n \to S$, and therefore
        \[
        \sum_{n=0}^{\infty} a_n = S.  
        \]
    \end{enumerate}
\end{exercise}

\begin{solution}

\end{solution}

% Exercise 6
\begin{exercise}[Optional]
    ~
    
    \begin{enumerate}
        \item[(1)] Let $\sum_{n=0}^{\infty} a_n x^n$ be a power series with radius of convergence $R>0$. Show that the radius of convergence of $\sum_{n=0}^{\infty} \frac{a_n}{n!} x^n$ is $+\infty$.
        \item[(2)] Suppose that the power series $\sum_{n=0}^{\infty} \frac{a_n}{n!} x^n$ has radius of convergence $R < +\infty$. What can we say about the radius of convergence of $\sum_{n=0}^{\infty}a_n x^n$?
    \end{enumerate}
\end{exercise}
 
\begin{solution}

\end{solution}

% Exercise 7
\begin{exercise}[Optional]
    Let $(a_n)_{n \ge 1}$ be a sequence of nonzero real numbers such that
    \[
    \frac{|a_{n+2}|}{|a_n|} \xrightarrow[n\to\infty]{} 2.
    \]
    Show that the radius of convergence of the power series $\sum_{n=0}^{\infty}  a_n x^n$ is $\frac{1}{\sqrt{2}}$.
\end{exercise}
 
\begin{solution}

\end{solution}

\end{CJK}
\end{document}