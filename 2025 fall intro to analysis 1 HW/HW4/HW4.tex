\documentclass[a4paper]{article}
%% Formatting %%
\usepackage[margin=3cm]{geometry}
\usepackage{type1cm, titlesec, fancyhdr, titling}
\usepackage{multicol}
\usepackage[dvipsnames]{xcolor}
\usepackage{ulem}
\usepackage{parskip}
\setlength{\parindent}{2em}
\setlength{\headheight}{15pt}
\setlength{\droptitle}{-1.5cm}
\parindent=24pt
%% Math and Symbols %%
\usepackage{amsmath,amsthm,amssymb, mathtools}
\usepackage{yhmath, faktor, dsfont}
\usepackage{academicons, wasysym, marvosym}
\usepackage[scr]{rsfso} 
\usepackage{latexsym, amsmath, amscd, amsmath, amsthm}
\usepackage{amssymb,amsmath,amsthm,graphicx,dsfont}
\usepackage{hyperref}

%% Enhancement %%
\usepackage{graphicx, tabularx}
\usepackage[shortlabels,inline]{enumitem}
%% TikZ %%
\usepackage{tikz-cd}
\usepackage[breakable]{tcolorbox}
\usetikzlibrary{decorations.pathmorphing}
\usetikzlibrary{calc, arrows,matrix}

%% Other packages %%
\usepackage{amsopn}

%% Traditional Chinese %%
\usepackage{CJKutf8}

%% Math environments %%
\newtheoremstyle{mystyle}
  {6pt}{15pt}% 上下間距
  {}%          內文字體
  {}%              縮排
  {\bf}%       標頭字體
  {.}%       標頭後標點
  {1em}% 內文與標頭距離
  {}% Theorem head spec (can be left empty, meaning 'normal')
\theoremstyle{mystyle}	
\newtheorem{theorem}{Theorem}
\newtheorem{definition}{Definition}
\newtheorem{example}[theorem]{Example}
\newtheorem{exercise}{Exercise}
\newtheorem{solution}{Solution}
\newtheorem{corollary}[theorem]{Corollary}
\newtheorem{property}[theorem]{Property}
\newtheorem{proposition}[theorem]{Proposition}
\newtheorem{lemma}[theorem]{Lemma}
\newtheorem{problem}{Problem}
\newtheorem{answer}{Answer}[section]
\newtheorem{fact}[theorem]{Fact}
\newtheorem*{recall}{Recall}
\newtheorem*{remark}{Remark}
\newtheorem*{claim}{Claim}
\newtheorem*{observation}{Observation}

\begin{document}
\begin{CJK}{UTF8}{bkai}

    \title{%
  \textbf{Math 2213 Introduction to Analysis I} \\
  \vspace{0.5cm}
  \large Homework 4 Due September 26 (Friday), 2025
}
\author{物理、數學三 黃紹凱 B12202004}
\date{\today}

\maketitle

% Problem 1
\begin{problem}[16 pts]
    ~
    \begin{enumerate}[(a)]
        \item Let
        \[
        X := \left\{ (a_n)_{n=0}^\infty : \sum_{n=0}^\infty |a_n| < \infty \right\}
        \]
        be the space of absolutely convergent sequences. Define the $\ell^1$ and $\ell^\infty$ metrics on this space by
        \[
        d_{\ell^1}\big((a_n)_{n=0}^\infty,(b_n)_{n=0}^\infty\big)
        := \sum_{n=0}^\infty |a_n - b_n|,
        \]
        \[
        d_{\ell^\infty}\big((a_n)_{n=0}^\infty,(b_n)_{n=0}^\infty\big)
        := \sup_{n\in\mathbb{N}} |a_n - b_n|.
        \]

        Show that these are both metrics on $X$, but show that there exist sequences 
        \[
        x^{(1)}, x^{(2)}, \dots
        \]
        of elements of $X$ (i.e.\ sequences of sequences) which are convergent with respect to the $d_{\ell^\infty}$ metric but not with respect to the $d_{\ell^1}$ metric. Conversely, show that any sequence which converges in the $d_{\ell^1}$ metric automatically converges in the $d_{\ell^\infty}$ metric.
        \item Let $(X,d_{\ell^1})$ be the metric space from part (a). For each natural number $n$, let $e^{(n)} = (e^{(n)}_j)_{j=0}^\infty$ be the sequence in $X$ such that  
        \[
        e^{(n)}_j := 
        \begin{cases}
        1, & \text{if } n=j,\\
        0, & \text{if } n\neq j.
        \end{cases}
        \]
        Show that the set
        \[
        \{ e^{(n)} : n \in \mathbb{N} \}
        \]
        is a closed and bounded subset of $X$, but is not compact.  

        (This is despite the fact that $(X,d_{\ell^1})$ is even a complete metric space---a fact which we will not prove here.  
        The problem is not that $X$ is incomplete, but rather that it is ``infinite-dimensional,'' in a sense that we will not discuss here.)
    \end{enumerate}
\end{problem}
\begin{solution}
    ~
    \begin{enumerate}[(a)]
        \item We will check the metric axioms for both $d_{\ell^1}$ and $d_{\ell^\infty}$. For brevity, we shall denote an element of $X$ by $(a)$ instead of $(a_n)_{n=0}^\infty$.
        \begin{enumerate}[(i)]
            \item For all $ (a), (b) \in X $, $ d_{\ell_1}((a), (b)) = 0 $ whenever $ (a) = (b) $, and $ d_{\ell_1}((a), (b)) > 0 $ whenever $ (a) \neq (b) $, since there exists some $ i \in \mathbb{N} $ such that $ \vert a_i - b_i \vert > 0 $.
            \item For all $ (a), (b) \in X $, $ d_{\ell_1}((a), (b)) = d_{\ell_1}((b), (a)) $ since $ \vert a_i - b_i \vert = \vert b_i - a_i \vert $ for all $ i \in \mathbb{N} $.
            \item For all $ (a), (b), (c) \in X $, by triangle inequality of real numbers with respect to $ \vert \cdot, \cdot \vert  $ , we have
            \[
            d_{\ell_1}((a), (c)) \leq d_{\ell_1}((a), (b)) + d_{\ell_1}((b), (c)).
            \]
        \end{enumerate}
        Similarly for $ d_{\ell^\infty} $: 
        \begin{enumerate}[(i)]
            \item For all $ (a), (b) \in X $, $ d_{\ell_\infty}((a), (b)) = 0 $ whenever $ (a) = (b) $, and $ d_{\ell_\infty}((a), (b)) > 0 $ whenever $ (a) \neq (b) $, since there exists some $ i \in \mathbb{N} $ such that $ \vert a_i - b_i \vert > 0 $.
            \item For all $ n \in \mathbb{N} $, we have $ \vert a_n - b_n \vert = \vert b_n - a_n \vert $, so $ d_{\ell_\infty}((a), (b)) = d_{\ell_\infty}((b), (a)) $.
            \item For all $ (a), (b), (c) \in X $, we have
            \begin{equation}
                \begin{split}
                    d_{\ell \infty}((a), (c)) &= \sup_{n \in \mathbb{N}} \vert a_n - c_n \vert \\
                    &\leq \sup_{n \in \mathbb{N}} (\vert a_n - b_n \vert + \vert b_n - c_n \vert) \\
                    &\leq \sup_{n \in \mathbb{N}} \vert a_n - b_n \vert + \sup_{n \in \mathbb{N}} \vert b_n - c_n \vert \\
                    &\leq d_{\ell_\infty}((a), (b)) + d_{\ell_\infty}((b), (c)),
                \end{split}
            \end{equation}
            where we used the triangle inequality of real numbers with respect to $ \vert \cdot \vert $.
        \end{enumerate}
        Suppose the sequence (of sequences) $ (x^{(m)}) \in X $ converges with respect to the metric $ d_{\ell_1} $. Then there exists $ x \in X $ such that for all $ \varepsilon > 0 $, there exists some $ N \in \mathbb{N} $ such that for all $ m > N $ we have 
        \[
            d_{\ell_1}((x^{(m)}), (x)) = \sum_{n=0}^{\infty} \vert a_n - b_n \vert  < \varepsilon.
        \]
        Then for all $ m > N $, we have 
        \[
            d_{\ell_\infty}((x^{(m)}), (x)) = \sup_{n \in \mathbb{N}} \vert a_n - b_n \vert \leq \sum_{n=0}^{\infty} \vert a_n - b_n \vert < \varepsilon, 
        \]
        so $ (x^{(m)}) $ also converges with respect to the metric $ d_{\ell_\infty} $. However, consider the sequence $ (x_n^{(m)})_{n=1}^{\infty} $ in $ \mathbb{R} $ defined by 
        \begin{equation*}
            x_n^{(m)} = 
            \begin{cases}
                \frac{1}{m}, & 0 \leq n < m, \\
                0, & n \geq m,
            \end{cases}
        \end{equation*}
        where $ \sum_{m=1}^{\infty} x_n^{(m)} = 1 < \infty $. This sequence converges to the zero sequence $ (0) $ in $ (\mathbb{R}, d_{\ell^{\infty}}) $ since for all $ \varepsilon > 0 $ and $ m \in \mathbb{N} $, let $ N = 1 / \varepsilon $, then
        \[
            d_{\ell_\infty}((x^{(m)}), (0)) = \sup_{0 \leq n < m} \frac{1}{m} < \frac{1}{N} = \varepsilon
        \]
        whenever $ n > N $. However, this sequence does not converge to in $ (\mathbb{R}, d_{\ell^1}) $ since for all $ n \in \mathbb{N} $, pick $ i, j \in \mathbb{N} $ such that $ i<j $, we have 
        \[
            d_{\ell_1}((x^{(i)}), (x^{(j)})) = \sum_{r=0}^{i-1} \frac{1}{i} + \sum_{r=i}^{j-1} \frac{1}{j} = 1 + \frac{j-i-1}{j} > 1.
        \]
        \item For all distinct $ i, j \in \mathbb{N} $, we have $ d_{\ell^1}(e^{(i)}, e^{(j)}) = \sum_{k=1}^{\infty} \left\vert e_k^{(i)} - e_k^{(j)} \right\vert = 2 $. Then for all $ x \in X $ and $ n \in \mathbb{N} $, we have
        \[
            d_{\ell^1}(x, e^{(n)}) = \sum_{k=1}^{\infty} \left\vert a_k - e_k^{(n)} \right\vert \leq \sum_{k=1}^{\infty} \left\vert a_k \right\vert + \sum_{k=1}^{\infty} \left\vert e_k^{(n)} \right\vert = \sum_{k=1}^{\infty} \left\vert a_k \right\vert + 1 < \infty.
        \]
        For all $ x \in X $, the sequence $ \left(e^{(n)}\right) $ is contained in $ B(x, \varepsilon) $ with $ \varepsilon = d_{\ell^1} \left(e^{(0)}, x\right) + 3 $, hence it is \emph{bounded}. For some $ x \in X \backslash \{e^{(n)} \,|\, n \in \mathbb{N}\} $, hence the complement of $ \{ e^{(n)} \,|\, n \in \mathbb{N} \} $ is open, and the set itself is closed. Finally, notice that $ d_{\ell^1}\left(e^{(i)}, e^{(j)}\right) = 2 $ for any pair of distinct $ i, j $, so the sequence $ \left(e^{(n)}\right) $ has no convergent subsequence, hence the $ \{e^{(n)} \,|\, n \in \mathbb{N} \} $ is \emph{not compact}. 
        \end{enumerate}
\end{solution}

\newpage 

% Problem 2
\begin{problem}[24 pts]
        A metric space $(X,d)$ is called \emph{totally bounded} if for every $\varepsilon > 0$, there exists a natural number $n$ and a finite number of balls
    \[
    B(x^{(1)},\varepsilon), \; B(x^{(2)},\varepsilon), \; \dots, \; B(x^{(n)},\varepsilon)
    \]
    which cover $X$ (i.e.\ $X = \bigcup_{i=1}^n B(x^{(i)},\varepsilon)$).

    \begin{enumerate}[(a)]
        \item Show that every totally bounded space is bounded.
        \item Show the following stronger version of Proposition~1.5.5: if $(X,d)$ is compact, then it is complete and totally bounded.  
        
        \emph{Hint:} if $X$ is not totally bounded, then there is some $\varepsilon > 0$ such that $X$ cannot be covered by finitely many $\varepsilon$-balls. Then use Exercise~8.5.20  (on page 182 of Analysis I) to find an infinite sequence of balls $B(x^{(n)},\varepsilon/2)$ which are disjoint from each other. Use this to construct a sequence which has no convergent subsequence.

        \item Conversely, show that if $X$ is complete and totally bounded, then $X$ is compact.  
        
        \emph{Hint:} if $(x^{(n)})_{n=1}^\infty$ is a sequence in $X$, use the total boundedness hypothesis to recursively construct a sequence of subsequences $(x^{(n;j)})_{n=1}^\infty$ of $(x^{(n)})_{n=1}^\infty$ for each positive integer $j$, such that for each $j$ the elements of the sequence $(x^{(n;j)})_{n=1}^\infty$ are contained in a single ball of radius $1/j$.  
        Also ensure that each sequence $(x^{(n;j+1)})_{n=1}^\infty$ is a subsequence of the previous one $(x^{(n;j)})_{n=1}^\infty$.  
        Then show that the ``diagonal'' sequence $(x^{(n;n)})_{n=1}^\infty$ is a Cauchy sequence, and then use the completeness hypothesis.
    \end{enumerate}
\end{problem}

\begin{solution}
    ~
    \begin{enumerate}[(a)]
        \item Let $ (X, d) $ be totally bounded . Then for all $ \varepsilon > 0 $, there exists some $ n \in \mathbb{N} $ and a sequence $\left( x^{(n)}\right)_{n=1}^{\infty} $ in $ X $ such that
        \[
            X = \bigcup_{i=1}^{n} B(x^{(i)}, \epsilon).
        \]
        Then for all $ x, y \in X $, there exists some $ i, j \in \{1, 2, \dots, n\} $ such that $ x \in B(x^{(i)}, \varepsilon) $ and $ y \in B(x^{(j)}, \varepsilon) $. Let $ R = \max\{d(x^{(i)}, x^{(j)}) : i,j \in \{1,2,\ldots,n\}\} $, by triangle inequality we have 
        \begin{equation*}
            d(x, y) \leq d(x, x^{(i)}) + d(x^{(i)}, x^{(j)}) + d(x^{(j)}, y) < 2\varepsilon + R.
        \end{equation*}
        Hence $ (X, d) $ is bounded.
        \item Suppose $ (X, d) $ is compact. Then by [Tao II] theorem 1.5.7, $ (X, d) $ is complete and bounded. Suppose $ (X, d) $ is not totally bounded, then there exists $ \varepsilon > 0 $ such that $ X $ cannot be covered by finitely many $ \varepsilon $-balls. Fix such $ \varepsilon $, choose $ x_1 \in X $ and write $ B_1 = B(x_1, \varepsilon/2) $, then we may choose $ x_2 \in X \backslash B_1 $ and write $ B_2 = B(x_2, \varepsilon/2) $, and so on. Note that $ X \backslash B_{i} \neq \varnothing $ since they do not cover $ X $. By induction, we obtain a sequence $ (x_n) \in X $ such that $ d(x_i, x_j) \geq \varepsilon $ for all distinct $ i, j \in \mathbb{N} $. Then for all $ N \in \mathbb{N} $, there exists some $ m, n > N $ such that $ d(x_m, x_n) \geq \varepsilon $, so $ (x_n) $ has no convergent subsequence, a contradiction. Hence $ (X, d) $ is totally bounded. 
        \item For each $ \varepsilon_n = \frac{1}{n} $, there exists finitely many $ \varepsilon_n $-balls that cover $ X $. Collect them into a sequence $ \left(B_{n;k}\right)_{k=1}^{m_n} $. Let $ (x^{(n)}) \subseteq X $ be a sequence, then $ x^{(1)} \in B_{1;k_1} $ for some $ k_1 \in \{1, 2, \dots, m_1\} $. Since $ (x^{(n)}) $ is infinite, there exists some subsequence $ (x^{(n;1)}) \subseteq (x^{(n)}) $ such that $ x^{(n;1)} \in B_{1;k_1} $ for all $ n $. Next, since $ x^{(n;1)} \in X $, there exists some $ k_2 \in \{1, 2, \dots, m_2\} $ such that $ x^{(n;1)} \in B_{2;k_2} $. Again, since $ (x^{(n;1)}) $ is infinite, there exists some subsequence $ (x^{(n;2)}) \subseteq (x^{(n;1)}) $ such that $ x^{(n;2)} \in B_{2;k_2} $ for all $ n $. Continuing this process, we obtain a sequence of subsequences $ (x^{(n;j)})_{n=1}^{\infty} $ for each $ j \in \mathbb{N} $ such that for all $ n $, $ x^{(n;j)} \in B_{j;k_j} $ for some $ k_j \in \{1, 2, \dots, m_j\} $, and $ (x^{(n;j+1)}) \subseteq (x^{(n;j)}) $. Consider the diagonal sequence $ (x^{(n;n)}) $, then for all $ \varepsilon > 0 $, let $ N = \operatorname{floor}\left(\frac{1}{\varepsilon}\right) $, and for all $ k, l > N $, we have $ d\left(x^{(k;k)}, x^{(l;l)}\right) \leq \frac{1}{N} < \varepsilon $. Hence $ (x^{(n;n)}) $ is Cauchy, and by completeness of $ (X, d) $, there exists some $ x \in X $ such that $ x^{(n;n)} \to x $. Thus $ (X, d) $ is compact. 
    \end{enumerate}
    \begin{remark}
        The sub-problems (b) and (c) becomes easy if we use the fact from Problem 3: \emph{a metric space is compact if and only if every open cover has a finite subcover.}
    \end{remark}
\end{solution}

% Problem 3
\begin{problem}[16 pts]
    ~
    \begin{enumerate}[(a)]
    \item A metric space $(X,d)$ is compact if and only if every sequence in $X$ has at least one limit point in $X$.
    \item Let $(X,d)$ have the property that every open cover of $X$ has a finite subcover. Show that $X$ is compact.  
    \emph{Hint:} If $X$ is not compact, then by part (a) there is a sequence $(x^{(n)})_{n=1}^\infty$ with no limit points.  
    Then for every $x \in X$ there exists a ball $B(x,\varepsilon)$ containing $x$ which contains at most finitely many elements of this sequence. Now use the hypothesis.
    \end{enumerate}
\end{problem}

\begin{solution}
    ~
    \begin{enumerate}[(a)]
        \item Suppose every sequence in $ X $ has at least one limit point in $ X $. Let $ (x^{(n)}) \in X $ be a sequence, then for all $ \varepsilon > 0 $ and $ N \in \mathbb{N} $, there exists some limit point$ x \in X $ and $ n > N $ such that $ d(x^{(n)}, x) < \varepsilon. $ For all $ N $, collect the corresponding $ n_N $, so $ (x^{(n)}) $ has a convergent subsequence  $ (x^{(n_N)}) \to x \in X $. Conversely, suppose $ (X, d) $ is compact. Then for all sequence $ (x^{(n)}) \in X $, there exists some subsequence $ (x^{(n_k)}) $ which converges to some $ x \in X $. Then for all $ \varepsilon > 0 $, there exists $ M \in \mathbb{N} $ such that $ d(x^{(n_k)}, x) < \varepsilon $ whenever $ k > M $. Hence $ x $ is a limit point of $ (x^{(n)}) $.
        \item Suppose that $ X $ is not compact. Then by part (a), there exists some sequence $ (x^{(n)})_{n=1}^{\infty} \in X $ with no limit points in $ X $. For all $ x \in X $, there exists some $ \varepsilon > 0 $ such that $ B(x, \varepsilon) $ contains finitely many elements of this sequence. Consider the collection of open sets
        \[
            R = \{ B(x, \varepsilon) : x \in X \}, \quad X = \bigcup_{U \in R} U.
        \]
        $ R $ is an open cover of $ X $, hence there exists some finite subcover $ R_0 \subseteq R $ such that
        \[
            X = \bigcup_{U \in R_0} U.
        \]
        However, since each $ U \in R_0 $ contains finitely many elements of $ (x^{(n)}) $, we conclude there are only finitely many elements of $ (x^{(n)}) $ in $ X $, a contradiction. Thus $ X $ is compact.
    \end{enumerate}
    \begin{remark}
        We have shown one direction of the result, where the other direcrtion is [Tao II] Theorem 1.5.8: \textit{a metric space is compact if and only if every open cover has a finite subcover.} 
    \end{remark}
\end{solution}

% Problem 4
\begin{problem}[10 pts]
    Let $(X,d)$ be a compact metric space. Suppose that $(K_\alpha)_{\alpha \in I}$ is a collection of closed sets in $X$ with the property that any finite subcollection of these sets necessarily has non-empty intersection, thus
    \[
    \bigcap_{\alpha \in F} K_\alpha \neq \varnothing \quad \text{for all finite } F \subseteq I.
    \]
    (This property is known as the \emph{finite intersection property}.)  

    Show that the entire collection has non-empty intersection, thus
    \[
    \bigcap_{\alpha \in I} K_\alpha \neq \varnothing.
    \]
    Show by counterexample that this statement fails if $X$ is not compact.
\end{problem}

\begin{solution}
    Since $ (X, d) $ is compact, every open cover of $ X $ has a finite subcover. Suppose $ R $ is a collection of open sets that cover $ X $, then 
    \[
        X = \bigcup_{U \in R_{0}} U,
    \] 
    for some $ R_0 \subseteq R $ which is a finite subcollection. Take the complement of the above equation, we have
    \[
        \varnothing = X^c = \left( \bigcup_{U \in R_0} U \right)^c = \bigcap_{U \in R_0} U^c.
    \]
    Since $ U $ is open, $ U^c $ is closed. Thus the claim is true.

    The statement fails if $ X $ is not compact. Consider the metric space $ (\mathbb{N}, d) $ where $ d(x, y) = \vert x - y \vert $. Since $ \mathbb{N} $ is not complete, it is not compact. Let $ K_n = [n, \infty) $, then for all finite $ F \subseteq \mathbb{N} $, we have 
    \[
        \bigcap_{n \in F} K_n = \bigcap_{n \in F} [n, \infty) = [\max F, \infty) \neq \varnothing.
    \]
    However, the entire collection has empty intersection:
    \[
        \bigcap_{n=1}^{\infty} K_n = \bigcap_{n=1}^{\infty} [n, \infty) = \varnothing.
    \]
\end{solution}

% Problem 5
\begin{problem}
       \begin{enumerate}

    \item[(a)] 
    Let $(X,d)$ be a metric space, and let $(E,d|_{E \times E})$ be a subspace of $(X,d)$.  
    Let $\iota_{E \to X} : E \to X$ be the inclusion map, defined by setting 
    \[
    \iota_{E \to X}(x) := x \quad \text{for all } x \in E.
    \]  
    Show that $\iota_{E \to X}$ is continuous.

    \item[(b)] Let $f : X \to Y$ be a function from one metric space $(X,d_X)$ to another $(Y,d_Y)$.  
    Let $E$ be a subset of $X$ (which we give the induced metric $d_X|_{E \times E}$), and let $f|_E : E \to Y$ be the restriction of $f$ to $E$, thus
    \[
    f|_E(x) := f(x) \quad \text{when } x \in E.
    \]  

    If $x_0 \in E$ and $f$ is continuous at $x_0$, show that $f|_E$ is also continuous at $x_0$.  
    (Is the converse of this statement true? Explain.)  

    Conclude that if $f$ is continuous, then $f|_E$ is continuous.  
    Thus restriction of the domain of a function does not destroy continuity.  

    \emph{Hint: use part (a).}
    
    \item[(c)] 
    Let $f : X \to Y$ be a function from one metric space $(X,d_X)$ to another $(Y,d_Y)$.  
    Suppose that the image $f(X)$ of $X$ is contained in some subset $E \subseteq Y$ of $Y$.  
    Let $g : X \to E$ be the function which is the same as $f$ but with the codomain restricted from $Y$ to $E$, thus $g(x) = f(x)$ for all $x \in X$.  

    \medskip
    \textbf{Note on codomain:}  
    The \emph{codomain} of a function is the declared target set of the function, in contrast to the \emph{image} (or range), which is the set of values the function actually takes.  
    So while $f$ is originally defined with codomain $Y$, its values all lie in the smaller set $E \subseteq Y$.  
    Therefore, one can equivalently regard $f$ as a function $g : X \to E$.  
    The metric on $E$ is the one \emph{induced from $Y$}, i.e.\ $d_Y|_{E \times E}$.

    \medskip
    Show that for any $x_0 \in X$, $f$ is continuous at $x_0$ if and only if $g$ is continuous at $x_0$.  
    Conclude that $f$ is continuous if and only if $g$ is continuous.  

    (Thus the notion of continuity is not affected if one restricts the codomain of the function.)
    \end{enumerate}
\end{problem}

\begin{solution}
    ~
    \begin{enumerate}[(a)]
        \item The preimage of an open ball in $ X $ is $ \iota ^{-1}\left(B_{(X,d)}\left(x, \varepsilon\right)\right) = B_{(E,d|_{E \times E})}\left(x, \varepsilon\right) $, hence it is open. By [Tao II] theorem 2.1.5, $ \iota $ is continuous.
        \item Suppose $ f $ is continuous at $ x_0 \in E $, then for all $ \varepsilon > 0 $, there exists $ \delta > 0 $ such that for all $ x \in E $, $ d_Y(f(x), f(x_0)) < \varepsilon $ whenever $ d_X(x, x_0) < \delta $. Since $ \iota_{E \to X} $ is continuous, we have $ d_Y(f|_E(x), f|_E(x_0)) = d_Y(f(\iota (x)), f(\iota (x_0))) < \varepsilon $ whenever $ d_X(x, x_0) < \delta $, hence $ f|_E $ is continuous at $ x_0 $. Conversely, suppose $ f|_E $ is continuous at $ x_0 \in E $, then for all $ \varepsilon > 0 $, there exists $ \delta > 0 $ such that for all $ x \in E $, $ d_X(x, x_0) < \delta \implies d_Y(f|_E(x), f|_E(x_0)) < \varepsilon $. Since $ \iota_{E \to X} $ is continuous, we have $ d_X(x, x_0) < \delta \implies d_Y(f(x), f(x_0)) < \varepsilon $, hence $ f $ is continuous at $ x_0 $. The converse is not true, however, since the floor function $ \lfloor \cdot \rfloor: \mathbb{R} \to \mathbb{R} $ is continuous on $ E = [0,1) $ but not on $ \mathbb{R} $.
        \item Suppose $ f $ is continuous at $ x_0 \in X $. Then for all $ \varepsilon > 0 $, there exists $ \delta > 0 $ such that for all $ x \in X $, $ d_Y(f(x), f(x_0)) < \varepsilon $ whenever $ d_X(x, x_0) < \delta $. Then $ d_Y|_{E \times E}(g(x), g(x_0)) = d_Y(f(x), f(x_0)) < \varepsilon $, hence $ g $ is continuous at $ x_0 $. Conversely, suppose $ g $ is continuous at $ x_0 \in E $. Then for all $ \varepsilon > 0 $, there exists $ \delta > 0 $ such that for all $ x \in E $, $ d_Y(g(x), g(x_0)) < \varepsilon $ whenever $ d_X(x, x_0) < \delta $. Since $ g(x) = f(x) $ for all $ x \in E $, we have $ d_Y(f(x), f(x_0)) < \varepsilon $ whenever $ d_X(x, x_0) < \delta $, hence $ f $ is continuous at $ x_0 $.
    \end{enumerate}
\end{solution}

% Problem 6
\begin{problem}
    Let $(X,d_X)$ and $(Y,d_Y)$ be metric spaces and $f:X \mapsto Y$ is a function from $X$ to $Y$.
    \begin{enumerate}[(a)]
        \item Prove that $f$ is continuous on $X$ if, and only if,
        \[
        f(\overline{A}) \subseteq \overline{f(A)}
        \]
        for every subset $A$ of $X$.
        \item Prove that $f$ is continuous on $X$ if and only if $f$ is continuous on every compact subset of $X$.  

        \textit{Hint:} If $x_n \to p$ in $X$, the set $\{p, x_1, x_2, \dots \}$ is compact.
    \end{enumerate}
\end{problem}

\begin{solution}
    ~
    \begin{enumerate}[(a)]
        \item Suppose $ f $ is continuous on $ X $. Let $ A \subseteq X $, for all $ x \in \overline{A} $, there exists some sequence $ (x^{(n)}) \in A $ such that $ x^{(n)} \to x $ under $ d_X $. Then by continuity of $ f $ and [Tao II] theorem 2.1.4, we have $ f(x^{(n)}) \to f(x) $ under $ d_Y $. Since $ f(x^{(n)}) \in f(A) $, we have $ f(x^{(n)}) \to f(x) \in \overline{f(A)} $. Thus $ f(\overline{A}) \subseteq \overline{f(A)} $. Conversely, suppose $ f(\overline{A}) \subseteq \overline{f(A)} $ for all $ A \subseteq X $. Let $ F \subseteq Y $ be a closed subset, then $ F = \overline{F} $ and $ f(f^{-1}(F)) \subseteq F $. Then by assumption $ f\left(\overline{f^{-1}(F)}\right) \subseteq \overline{f(f^{-1}(F))} \subseteq \overline{F} = F $, and $ \overline{f^{-1}(F)} \subseteq f^{-1}(F) $. Hence $ f^{-1}(F) $ is closed, and $ f $ is continuous by [Tao II] theorem 2.1.5. 
        \item Suppose $ f $ is continuous on $ X $, then for all compact subset $ K \subseteq X $, the restriction of $ f $ to $ K $ is continuous on $ K $. Conversely, suppose $ f $ is continuous on every compact subset of $ X $. Let $ F \subseteq Y $ be closed and let $ x \in \overline{f^{-1}(F)} $, then there exists a sequence $ (x^{(n)}) \in X $ such that $ x^{(n)} \to x \in X $. The set $ K = \{ x, x^{(1)}, x^{(2)}, \dots \} $ is compact since every sequence in $ K $ has a convergent subsequence. By assumption, $ f $ is continuous on $ K $, so by [Tao II] theorem 2.1.4, we have $ f(x^{(n)}) \to f(x) \in F $, hence $ x \in f^{-1}(F) $. Thus $ f^{-1}(F) $ is closed, and by the same theorem $ f $ is continuous on $ X $.
    \end{enumerate}
    
\end{solution}

\end{CJK}
\end{document}