\documentclass[12pt]{article}
\usepackage[a4paper, total={16cm, 20cm}]{geometry}
\usepackage{import}
\usepackage{xifthen}
\usepackage{pdfpages}
\usepackage{transparent}

\newcommand{\incfig}[1]{%
    \def\svgwidth{\columnwidth}
    \import{./Figures/}{#1.pdf_tex}
}

\usepackage{scrextend}
\usepackage{amsmath}
\usepackage[shortlabels]{enumitem} % enumerate with letters
\usepackage[normalem]{ulem}
\usepackage{indentfirst}
\usepackage{pifont}
\usepackage{fancyhdr}   % 頁首頁尾
\usepackage{amssymb}
\usepackage{empheq} % box around multiple equations
\newcommand*\widefbox[1]{\fbox{\hspace{2em}#1\hspace{2em}}}

% figures
\usepackage{graphicx} % Required for inserting images
\usepackage{wrapfig} 
\usepackage{subfig}
\usepackage{makecell} % table wrap text
\usepackage{booktabs}
\usepackage{tabularx}

\usepackage{lmodern,bm}                
\usepackage[T1]{sansmath} 
\SetMathAlphabet{\mathsfbf}{sans}{\sansmathencoding}{\sfdefault}{bx}{sl}
\usepackage{etoolbox}
\AtBeginEnvironment{sansmath}{\let\bm\mathsfbf}{}{}
\usepackage{mdframed}
\usepackage{amsmath, nccmath}
\usepackage{mathtools} % for text above and under arrows

\usepackage{bbm}
\usepackage{bbold} % blackboard bold font

\usepackage{relsize} % math symbol size
\usepackage{amsthm}

\theoremstyle{plain}
\newtheorem{theorem}{Theorem}[section]
\newtheorem{lemma}[theorem]{Lemma}
\newtheorem{proposition}[theorem]{Proposition}
\newtheorem{corollary}[theorem]{Corollary}

\usepackage{thmtools}
\usepackage{mdframed}
\usepackage[dvipsnames]{xcolor} % different colors


\declaretheoremstyle[
    headfont=\sffamily\bfseries\color{black}, % Sans-serif, bold, blue title
    bodyfont=\normalfont,
    headpunct={.}, % Adds a period after the theorem heading
    postheadspace=1em,
]{mydefstyle}

\declaretheorem[
    style=mydefstyle,
    name=Definition,
    numberwithin=section % <-- THIS makes it Definition 1.1, 1.2, etc.
]{definition}

\declaretheorem[
    style=mydefstyle,
    name=Example,
    numberwithin=section % <-- THIS makes it Definition 1.1, 1.2, etc.
]{example}

\theoremstyle{remark}
\newtheorem*{remark}{Remark}

\newcommand{\bvec}[1]{\mathbf{#1}} % vector
\newcommand{\mycomment}[1]{} % block comments

% figures
\usepackage{wrapfig} 
\usepackage{subfig}

\title{%
  Special Function for Michaelis-Menten Calculations \\
}
% \author{Jonathan Huang}
% \date{\today}

\begin{document}

\maketitle

\section{\large Special Function}
As defined by Dr. Lin, the equation 
\begin{equation}
    1 - mx = \frac{cx}{k+x} \label{equ:MM_special}
\end{equation}
admits the nonnegative solution given by
\begin{equation}
    p(c, k, m) = \frac{1}{2m}\left[(1 - c - mk) + \sqrt{(1 - c- mk)^2 + 4mk} \right]. 
\end{equation}
The quantity $ p(c,k,m) $ always lies between $ 0 $ and $ 1/m $, is an increasing function of $ c $, $ m $, and a decreasing function of $ k $. For our system, we can express the fixed point solution as 
\begin{subequations}
    \begin{align}
        Y^\ast_1 &= p\left(\frac{a_1 \theta_2}{b \theta_1}, k_1, 1\right) \equiv p(c_1, k_1, m_1), \\
        Y^\ast_2 &= p \left(\frac{a_2 \theta_3}{a_1 \theta_2} \frac{k_1 + Y^\ast_1}{Y^\ast_1}, k_2, \frac{b \theta_1}{a_1 \theta_2} \frac{k_1 + Y^\ast_1}{Y^\ast_1}\right) \equiv p(c_2, k_2, m_2), \\
        Y^\ast_{k+2} &= \theta_k W^\ast, \quad W^\ast \equiv 1 - Y^\ast_1 - Y^\ast_2, \quad k = 1,2,3.
    \end{align}
\end{subequations}

Let's consider the relevant limits of $ p(c,k,m) $ for the $ b \to 0 $ and $ b \to \infty $ limits. When $ b \to 0 $, we have $ c_1 \to \infty $ and $ c_2 \to \infty $. In this case, even though it seems like $ m_2 \to 0 $, we simultaneously have $ Y_1^\ast \to 0 $, and hence $ m_2 = O(1) $. It feels handwavy to conclude so, but a full expansion shown later will verify this fact. On the other hand, when $ b \to \infty $, we have $ c_1 \to 0 $, $ Y_1^\ast \to 1 $, and hence $ m_2 \to \infty $.
\begin{enumerate}[(1)]
    \item $ c \to 0 $: By direct expansion of $ p(c,k,m) $ about $ c=0 $, we have 
    \begin{mdframed}[linewidth=1.0pt, linecolor=black, innertopmargin=3pt, innerbottommargin=10pt]
    \begin{equation}
        p(c,k,m) = \frac{1}{m} - \frac{c}{m\,(k m + 1)} + \frac{k}{(k m + 1)^3}\,c^{2} + \frac{k\,(1 - k m)}{(k m + 1)^5}\,c^{3} + O(c^{4}).
    \end{equation}      
    \end{mdframed}
    \item $ c \to \infty $: By direct expansion of $ p(c,k,m) $ about $ c = \infty $, we have 
    \begin{mdframed}[linewidth=1.0pt, linecolor=black, innertopmargin=3pt, innerbottommargin=10pt]
    \begin{equation}
        p(c,k,m) = \frac{k}{c} + \frac{k - k^{2} m}{c^{2}} + \frac{k\,(k^{2} m^{2} - 3 k m + 1)}{c^{3}} + O(c^{-4}).
    \end{equation}
    \end{mdframed}
    \item $ m \to 0 $: By direct expansion of $ p(c,k,m) $ about $ m = 0 $, we have
    \begin{equation*}
        p(c,k,m) = \frac{|c - 1| - (c - 1)}{2m} + \frac{k}{2}\!\left(\frac{1 + c}{|c - 1|} - 1\right) - \frac{c\,k^{2}}{|c - 1|^{3}}\,m + \frac{k^{3} c (c + 1)}{|c - 1|^{5}}\,m^{2} + O(m^{3}).
    \end{equation*}
    Assuming $ c < 1 $, this becomes 
    \begin{mdframed}[linewidth=1.0pt, linecolor=black, innertopmargin=3pt, innerbottommargin=10pt]
    \begin{equation}
        p(c,k,m) = \frac{1-c}{2m} + \frac{ck}{1-c} - \frac{c\,k^{2}}{(1-c)^{3}}\,m + \frac{k^{3} c (c + 1)}{(1-c)^{5}}\,m^2 + O(m^{3}).
    \end{equation}
    \end{mdframed}
    If we instead assume $ c > 1 $, then 
    \begin{mdframed}[linewidth=1.0pt, linecolor=black, innertopmargin=3pt, innerbottommargin=10pt]
    \begin{equation}
        p(c,k,m) = \frac{k}{c - 1} - \frac{c\,k^{2}}{(c - 1)^{3}}\,m + \frac{k^{3} c (c + 1)}{(c - 1)^{5}}\,m^{2} + O(m^{3}).
    \end{equation}
    \end{mdframed}
    \item $ m \to \infty $: By direct expansion of $ p(c,k,m) $ about $ m = \infty $, we have  
    \begin{mdframed}[linewidth=1.0pt, linecolor=black, innertopmargin=3pt, innerbottommargin=10pt]
    \begin{equation}
        p(c,k,m) = \frac{1}{m} - \frac{c}{k\,m^{2}} + \frac{c(1 + c)}{k^{2}\,m^{3}} + O(m^{-4}).
    \end{equation}
    \end{mdframed}
\end{enumerate}

Notice that $ p $ diverges when $ m \to 0 $, but fortunately we do not need this limit in our analysis at all. Using these expansions, we can now give a preliminary analysis of the starvation and overabundance limits.

\section{\large Starvation Limit}
When $ b \to 0 $, we have $ c_1 \to \infty $, $ c_2 \to \infty $, and $ m_2 = O(1) $. Therefore, using the $ c \to \infty $ expansion on $ c_1 $, we have
\begin{mdframed}[linewidth=1.0pt, linecolor=black, innertopmargin=3pt, innerbottommargin=10pt]
\begin{equation}
    Y_1 = k_1 \left(\frac{b \theta_1}{a_1 \theta_2}\right) + k_1 (1 - k_1) \left(\frac{b \theta_1}{a_1 \theta_2}\right)^2 + O \left( n_1^3 \right), 
\end{equation}
\end{mdframed}
and
\begin{align}
    \frac{Y_1^\ast}{k_1 + Y_1^\ast} &= \left(\frac{\theta_1}{a_1 \theta_2}\right) b - (2-k_1) \left(\frac{\theta_1}{a_1 \theta_2}\right)^2 b^2 + O\left(\left(\frac{b \theta_1}{a_1 \theta_2}\right)^3\right), \\
    \frac{k_1 + Y_1^\ast}{Y_1^\ast} &= \left(\frac{a_1 \theta_2}{\theta_1}\right) \frac{1}{b} + (2 - k_1) + O\left(\frac{b \theta_1}{a_1 \theta_2}\right).
\end{align}
Then, using the same expansion on $ c_2 $, we have
\begin{equation}
    \frac{1}{c_2} = \left(\frac{a_1 \theta_2}{a_2 \theta_3}\right) \left(\frac{Y_1^\ast}{k_1 + Y_1^\ast}\right) = \left(\frac{\theta_1}{a_2 \theta_3}\right) b - (2-k_1) \left(\frac{\theta_1^{2}}{a_1 a_2 \theta_2 \theta_3}\right) b^{2} + O\left(\left(\frac{b \theta_1}{a_1 \theta_2}\right)^3\right),
\end{equation}
\begin{equation}
    m_2 = 1 + (2-k_1) \left(\frac{\theta_1}{a_1 \theta_2}\right)b + O\left(\frac{b \theta_1}{a_1 \theta_2}\right)^2.
\end{equation}
It is as expected that $ m_2 = O(1) $. Then, for $ Y_2^\ast $ we have 
\begin{equation}
    Y_2^\ast = k_2 (2 - k_2) \left(\frac{\theta_1}{a_2 \theta_3}\right) b - 2 k_2 (2-k_1) \left(\frac{\theta_1^2}{a_1 a_2 \theta_2 \theta_3}\right) b^2 + O\left(\left(\frac{b \theta_1}{a_1 \theta_2}\right)^3\right).
\end{equation}
This does not match the full expansion given in the next section. This may be because of the assumption that $ m_2 = O(1) $, which is not fully justified. A more rigorous derivation is given in the corresponding Subsection.

\section{\large Full Calculation}
We also show the result of a full on expansion for comparision with results from the special function method. Consider the series expansion of $Y_1, Y_2, Y_3$ in the limit $b \to 0$. In effect, we are computing the series expansion with respect to the dimensionless quantity $ n_1 = b \theta_1 / a_1 \theta_2 $ and $ n_2 = b \theta_1 / a_2 \theta_3 $. We have

\begin{mdframed}[linewidth=1.0pt, linecolor=black, innertopmargin=3pt, innerbottommargin=10pt]
\begin{align}
    Y_1 &= k_1 \left(\frac{b \theta_1}{a_1 \theta_2}\right) + \frac{k_1 \theta_2 (1 - k_1)}{\theta_2} \left(\frac{b \theta_1}{a_1 \theta_2}\right)^2 + O \left( n_1^3 \right), \\
    Y_2 &= k_2 \left(\frac{b \theta_1}{a_2 \theta_3}\right) - \frac{k_2 (a_1 k_2 \theta_2 - a_1 \theta_2 + a_2 k_1 \theta_3)}{a_1 \theta_2 } \left(\frac{b \theta_1}{a_2 \theta_3}\right)^2 + O\left(n_2^3\right) , \\
    Y_3 &= \theta_1 - b\theta_1^2\left(\frac{k_1}{a_1\theta_2} + \frac{k_2}{a_2\theta_3}\right) + O\left(\min \{n_1^2, n_2^2\}\right).
\end{align}
\end{mdframed}

\end{document}