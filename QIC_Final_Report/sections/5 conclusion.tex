\subsection{Summary of Results}
We have discussed various architectures for QRAM and their respective physical implementations. The architectures we mentioned include Fanout, Bucket-Brigade, Flip-Flop, Qudit-Based QRAM and applications of Parametrized Quantum Circuit (PQC)-Based QRAMs, specifically Approximate PQC and EQGAN QRAM.

Then we conducted error analysis of the various schemes, using simple mathematics and referring to existing analyses done in relevant literature.

Finally, we successfully realized Grover's unstructure search algorithm using Qiskit on three qubits, using both the Bucket-Brigade protocol and the Flip-Flop protocol. The simulation method is great for iterative schemes such as the flip-flop protocol. For the flip-flop simulation, we also verified the log-linear depth-to-qubit number relation.

\subsection{Future Study Directions}

\subsubsection{Scaling QRAM}
QRAMs, like traditional RAMs, would require millions or billions of cells or individual memory elements in order to support pratical calculation \cite{s23177462}. Even though quantum computing hardware can be expected to see great advancements in the upcoming future, it would remain a challenge to develop a QRAM that is capable of addressing this many memory elements. Research could also focus on more robust, i.e. noise-relient, protocols, such as the bucket-brigade scheme and its variants, to minimize possible hardware threshold.

\subsubsection{Do We Have to Scale?}
Apart from trying to extend the capabilities of quantum computing hardware to control millions of cells, it may be for the short term more feasible, and even more fruitful, to search for applications where smaller QRAMs can provide valuable results. Following on this, it may also be informative to study the resource estimations for these smaller QRAM projects.

\subsubsection{Noise-Resilience}
As observed in section \ref{sec:arch}, basically all the proposed QRAM architectures require exponential circuit depth or width, so increasing memory elements many have a large effect on the accuracy of QRAMs if they are not sufficiently noise-resilient.