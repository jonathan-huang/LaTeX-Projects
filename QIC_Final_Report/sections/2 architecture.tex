\subsection{Overview}

With the advancement of hardware and qubit technologies, the possibility of implementing quantum algorithms, such as Grover's algorithm, on a quantum computer becomes more and more within reach. The is section introduces the relevant technologies for physical implementation in further detail, and discusses a variety of potential QRAM architectures. This section describes several proposed QRAM architectures and evaluates them based on structure, complexity, platform, and practical considerations. Also, since it is easier to compare and demonstrate the quantum advantage (albeit only intuitively) of QRAM, we will also elaborate on their respective classical counterparts. 

\begin{table}[ht]
\centering
\begin{tabular}{@{}p{3cm} p{3cm} p{1.5cm} p{1cm} p{5cm}@{}}
\toprule
Architecture & Circuit Type & Width & Depth & Implementation \\
\midrule
Bucket-Brigade & Bifurcation graph & $O(2^n)$ & $O(2^n)$ & Photonic circuit/qutrits as trapped ions \\
Fanout QRAM & Bifurcation graph & $O(2^n)$ & $O(2^n)$ & Photonic circuit \\
Flip-Flop QRAM & Quantum circuit & $O(n)$ & $O(2^n)$ & Superconducting qubits \\
Qudit-based & qudits & $^*$ & $^*$ & Trapped ion/Photonic circuit \\
Approximate-PQC / EQGAN & Parametric Quantum Circuit & $O(n)$ & $O(1)$ &  \\
\bottomrule
\end{tabular}
\caption{Table of comparison. asterisk means "depends on $d$".}
\end{table}

Next, we will provide a comprehensive review of various architectures in detail, including an introduction to their classical counterparts.

\subsection{\label{sec:fanout}Fanout QRAM}
In the \emph{fanout QRAM} protocol, each bit of the address register controls multiple switches in a binary tree network, routing a quantum bus to the desired memory cell. 

\subsubsection{Classical Fanout RAM Scheme}
In the classical fanout scheme, there is an \textit{index register} that specifies the direction to follow to reach the memory cell we are interested in. Similar to the bucket-brigade architecture discussed above, the fanout protocol has a bifurcation graph circuit. Writing the index register in binary form, each bit of such register can be interpreted as an indicator of  direction, so $2^n$ paths can be described. 

We can realize this indexing procedure with a fanout RAM scheme, where the $k$-th index bit controls the value of $2^k$ (classical) bits in the $k$-th level of the tree. This uniquely specifies the path to the desired memory cell, but in the process it requires controlling all $2^n$ bits even though only $n$ bits participate in addressing the memory cell. However, we can do a 

The classical fanout scheme is implemented in chips, by using electronic circuits and pairs of transistors to replace the binary tree and respective switches \cite{PhysRevA.78.052310}.

\begin{figure}
    \centering
    \includegraphics[width=0.5\linewidth]{images/classical_fanout.png}
    \caption{Demonstration of the classical fanout scheme.}
    \label{fig:enter-label}
\end{figure}

\subsubsection{Quantization of the Fanout Scheme}
The fanout control structure discussed above is easy to realize classically, but quantum implementations suffer from decoherence due to entanglement between address qubits and many control gates. We take directly the classical fanout RAM and try to extend it using the principle pf superposition in quantum mechanics. The quantum version of the fanout RAM has the $k$th address qubit controlling $2^k$ quantum switches. Unlike the bucket-brigade protocol we'll introduce in the next section, fanout QRAM can be implemented on two-level systems, so the usual qubits are used.

We briefly illustrate this using part of a diagram from \cite{s23177462}, included as figure \ref{fig:fanout}. The quantum switches, which consist of traditional qubits, are initialized by being set to the $\ket{0}$ state. In the circuit, $\ket{0}$ describes a "left turn", while $\ket{1}$ describes a "right turn", so the sequence of qubits $(\ket{0}, \ket{1})$ in the address line would direct to the memory cell $\ket{X_{01}}$. We observe that only one path consisting of $2 = 2\times 1$ qubits is active at one time, instead of all $1 + 3 = 2^2$ in the bifurcation tree.

\begin{figure}
    \centering
    \includegraphics[width=0.5\linewidth]{images/fanout.png}
    \caption{Diagram illustrating only the case where the address line only consists of single state addresses.}
    \label{fig:fanout}
\end{figure}

Quantum versions require reversible operations and preservation of coherence through unitary transformations. Although scalable only to modest $n$, fanout QRAM is experimentally viable and foundational for understanding error propagation and decoherence in quantum memory systems.

\subsubsection{Implementation}
The quantum fanout precedure may be implemented with the following methods \cite{ref:arch}: 
\begin{enumerate}
    \item Optical implementation.
    \item Controlled phase gates implementation.
\end{enumerate}

\begin{figure}
    \centering
    \includegraphics[width=0.5\linewidth]{images/optical.png}
    \caption{Optical implementation of fanout protocol.}
    \label{fig:enter-label}
\end{figure}

\subsection{Bucket-Brigade QRAM}

This is the first proposal for the structure of a QRAM. The architecture , unlike the traditional $d$-dimensional lattice of memory array used in classical RAM, uses a \textit{bifurcation graph-based structure}. The structures is showcased in the diagrams [\ref{fig:RAM}], and also [\ref{fig:BB}] later, taken from reference \cite{s23177462}. 

To understand the architecture, we need a knowledge of "qutrits", the generalization of qubits consisting of a two-level system $\{\ket{0}, \ket{2}\}$ to three-level systems. In a qutrit, apart from the two basis states $\ket{0}$ and $\ket{1}$, a third state usually denoted as $\ket{\cdot}$ is present, which indicates "sending the incoming signal on without changing its state". This comes in very handy in extending the protocol described in section \ref{sec:fanout} to be more efficient.

The bucket-brigade architecture is designed to address the inefficiencies of conventional fanout models. It utilizes a tree of three-state quantum systems (qutrits) that route address and data signals by sequential activation. Initially, all qutrits are in a passive state. As each bit of the address register propagates through the tree, qutrits transition to active states (0 or 1), encoding the routing path. The bus qubit follows this path to the target memory cell, then reverses its route to complete the transaction.

\begin{figure}
    \centering
    \includegraphics[width=0.5\linewidth]{images/RAM.png}
    \caption{High dimensional lattice structure for a classical memory array.}
    \label{fig:RAM}
\end{figure}
\begin{figure}[!htb]    
    \centering
    \includegraphics[width=0.5\linewidth]{images/BB.png}
    \caption{Bifurcation graph structure of the bucket-brigade protocol for QRAM implemented on a quantum circuit. In the diagram, data in the memory cell denoted $\ket{m_{01}}$ is being accessed via a sequence of quantum gates. Also, the red path represents the active route of the QRAM.}
    \label{fig:BB}
\end{figure}

\subsubsection{The QRAM Scheme}

The advantage and thereof of the bucket-brigade QRAM scheme is discussed in the 2008 Giovannetti paper \cite{PhysRevLett.100.160501}. An advantage of the bucket-brigade method lies in its \textbf{robustness}, so it is less prone to the influence of noise. As most QRAM circuit architectures require exponential depth, this makes bucket-brigade a great architecture for physical implementation. This figures \ref{fig:BBnew} and \ref{fig:BBoriginal} show the bifurcation tree used in the bucket-brigade scheme.

\begin{figure}[!htb]
    \centering
    \begin{minipage}{0.5\textwidth}
        \centering
        \includegraphics[width=1.0\linewidth]{images/bif.png}
        \caption{Bifurcation graph of "RAM addressing", which inspires the bucket-brigade protocol.}
        \label{fig:BBoriginal}
    \end{minipage}%
    \begin{minipage}{0.5\textwidth}
        \centering
        \includegraphics[width=1.0\linewidth]{images/BBbif2.png}
        \caption{Illustration of the bucket-brigade protocol.}
        \label{fig:BBnew}
    \end{minipage}
\end{figure}

\subsubsection{Implementation}

This method activates only \( O(n) \) qutrits per memory call, compared to $O(2^n)$ in traditional models, significantly reducing decoherence risks and improving scalability with exponential speedup. The architecture is suitable for photonic and atomic systems where controlled routing is feasible. Because the bucket-brigade QRAM operates by sequential coupling of qutrits, it takes $O(n^2)$ steps to retrieve one of the $2^n$ memories in a coherent manner.

\begin{figure}
    \centering
    \includegraphics[width=0.5\linewidth]{images/BBphoton.png}
    \caption{At each node of the above bifurcation tree, the three-state switch corresponding to a qutrit consists of an atom in a cavity.}
    \label{fig:BBphoton}
\end{figure}

Alternatively, it is mentioned that these "atoms in a cavity" need not be real atoms, but could be artificial atoms consisting of superconducting qubits. For the atom-in-a-cavity implementation, refer to diagram \ref{fig:BBphoton}, taken from \cite{s23177462}. At the beginning, we initialize all the atoms to the lowest energy state, corresponding to $\ket{\cdot}$, which can be excited by laser to either $\ket{0}$ ($\ket{zero}$) or $\ket{1}$ ($\ket{one}$) depending on the polarization state of the incoming photon ($\ket{0}$ or $\ket{1}$) by \textit{Raman transition}. When a second photon hits the atom, we involve even higher energy states of the atom: $\ket{0} \to \ket{\gets}$ ($\ket{up}$) or $\ket{1} \to \ket{\to}$ ($\ket{down}$) depending on the polarization state of the second incoming photon. 

Then the photon is released according to the direction indicated by $\ket{\gets}, \ket{\to}$, coupling to the next atom in line. This is repeated until a path is constructed that reaches the desired memory cell, and the output register may load from the memory cell or store data into the said cell.

\subsection{Flip-Flop QRAM}

\subsubsection{Introduction}
Flip-Flop QRAM operates via dynamic switching between encoding and data-loading modes using a structured quantum circuit. Unlike fanout, it requires only linear circuit width (number of address lines), significantly reducing the number of ancilla qubits. However, the circuit depth remains exponential due to sequentially loading binary data. Hence the width is $O(n)$ and the depth is $O(2^n)$.

Storing data in a flip-flop QRAM involves the \textit{flip stage}, the \textit{register stage}, and the \textit{flop stage}. 
\begin{enumerate}
    \item Flip stage: switch all the address qubits that correspond to our desired data to $\ket{1}$. 
    \item Register stage: Construct a multi-controlled rotation gate that "rotates" the information of the flipped qubits into the data qubits. 
    \item Flop stage: "unflip" the information stored in the data qubits by doing the inverse operation.
\end{enumerate}

\subsubsection{Implementation}
This architecture is primarily suitable for superconducting qubit systems, where rapid and precise switching can be implemented. Flip-flop QRAM is non-iterative, so with the advent of quantum hardware technologies, we can expect to rnu flip-flop QRAM on quantum computing hardware. In the next section, we simulate flip-flop QRAM using Qiskit.

\subsection{Qudit-Based QRAM}
Qudit-based QRAMs utilize multi-level systems (qudits) to encode address and data values compactly. This approach can dramatically reduce circuit width and depth depending on the dimensionality \( d \) of the qudits used. The primary advantage is dense information representation, making them suitable for high-capacity systems.

\subsubsection{Fundamental Ideas}
In section (\ref{ssec:bbqram}) we discussed qutrits and their application in , qudits make up a higher dimensional extension of traditional qubis and qutrits. In a traditional qubit there are two basis states: $\ket{0}$ and $\ket{1}$, and we tensor them to form a computational basis for a system. In a \textit{qudit} system with $d$ computational basis states, a quantum state becomes 
\begin{equation}
    \ket{\psi} = a_0 \ket{0} + a_1 \ket{1} + \cdots + a_{d-1}\ket{d-1}.
\end{equation}

Compared to quantum system composed of qubits, a \textit{qudit} system with identical number of working bits can provide a larger state space to store and process information. Therefore, qudit and higher dimensional computing can provide reduction of the circuit complexity and simplification of the experimental setup \cite{10.3389/fphy.2020.589504}, though the theory of qudits is not of main focus here.

\subsubsection{Theory Behind Qudit QRAM}
Qudits-based quantum memory has been proposed, where qubits are compressed onto higher dimensional qudits, so that extra space on the qudits can be used as ancillas for other purposes when computing is not in progress. This can be reversibly, so the qudits can be transformed back into qubits. In reference \cite{s23177462}, it is mentioned that an $x$-$y$-$z$ \textit{qubit-qudit compression scheme} has been proposed.

However, the detailed scheme is complicated, so we will illustrate the main idea with a simple example of \textit{qubit-qutrit compression}. Consider three qubits with $d = 2$ and two qutrits with $d=3$. Three qubits can store a total of $2^3$ computational states, thus have a state space of size $8$, while two qudits have a state space with size $3^2 = 9$. Then it is possible to compress three qubits into two qudits and retrieve them using a compression/decompression circuit (figure \ref{fig:compress}).

\begin{figure}
    \centering
    \includegraphics[width=0.6\linewidth]{images/compress.png}
    \caption{An example compression and decompression circuit, which compresses a system of three qubits $A, B, C$ into a system of two qutrits $A^{\prime}, B^{\prime}$, and an ancilla $C^{\prime}$.}
    \label{fig:compress}
\end{figure}

\subsubsection{Is Qudit Too Good to be True? Note on Instability}
It is a fact that qudits with $d > 2$ are prone to instability \cite{6874964}, called the \textit{qudit instability}. This means that, while higher dimensional quantum computing using qudits can store more information yield more memory space, it is also more unstable than qubit systems. 

\subsubsection{Implementation}
Practical realization is difficult due to limited support in mainstream quantum programming frameworks. As in the case of qutrits mentioned in the bucket-brigade scheme (subsection \ref{ssec:bbqram}), physical implementation of qudits finds its way on physical quantum systems that have an \textit{infinite spectrum of states}, like superconducting qubits and trapped ion qubits. In such systems, we identify different energy levels with different states of the quantum system.

\subsection{EQGAN and approximate PQC QRAM}

\subsubsection{Overview and Parametrized Quantum Circuits}
Approximate parametric quantum circuit (PQC)-based QRAM and Enhanced Quantum Generative Adversarial Network (EQGAN) models use trained quantum circuits to approximate memory access behavior. As the name suggests, they are based on what we call \emph{parametrized quantum circuits}. 

A parametrized quantum circuits is a quantum circuit that includes one or more parameters (variables) in its quantum gates. These parameters are typically real numbers that can be adjusted during the execution. A quantum circuit consists of quantum gates that manipulate qubits. Each gate can perform operations like rotations or entangling, and we can define them with a parameter that is either continuous or discrete. For example, a rotation gate $R(\theta)$ has an angle parameter $\theta$, which determines the amount of rotation applied to a qubit.

\subsubsection{More on the Methods}
They are highly efficient in both depth ($O(1)$) and width ($O(n)$), making them ideal for near-term quantum devices. Unlike usual QRAM protocols, the approximate PQC-based QRAM does not leverage superposition to store data, but instead stores data in a sequential order, allowing it to store more comple data. It has been shown \cite{s23177462} that, in the context to loading them to a quantum neural network (QNN), sending images from approximate PQC-based QRAM may give faster convergence as compared to loading images without QRAM.

\subsubsection{Implementation}
Since they are both based on quantum circuits, they can be implemented on superconducting and trapped ion qubits. Another possibility is using quantum computing platforms such as Qiskit and IonQ (Trapped Ion Computing), given the quantum circuit is known by the experimenter. While this may seem like "cheating", it is a great choice for PQC-based QRAMs, since they, being based on training adjustable parameters, are inherently iterative.

\bigskip

In the section that follows, we deliver a simple error analysis for various archiectures, and showcase a Qiskit realization of Grover's search algorithm on the bucket-brigade and the flip-flop QRAM schemes.
