\documentclass[12pt, A4, twoside]{article}
\usepackage{scrextend}
\usepackage{geometry}
 \geometry{
	 a4paper,
 	total={170mm,257mm},
 	left=20mm,
	 top=20mm,
 }

\usepackage{amsmath}
\usepackage[shortlabels]{enumitem} %enumerate with letters
\usepackage[normalem]{ulem}
\usepackage{indentfirst}
\usepackage{pifont}
\usepackage{fancyhdr}   % 頁首頁尾
\usepackage{amssymb}
\usepackage{pgfplots}
\pgfplotsset{compat=1.15}
\usetikzlibrary{arrows}

% Mandarin
\usepackage{CJKutf8}
% bkai = 標楷體
% bsmi = 新細明體

\usepackage{graphicx} % Required for inserting images

% paragraph
%\setlength{\parskip}{\baselineskip}
\setlength{\parindent}{2em}
%
\usepackage{lmodern,bm}                
\usepackage[T1]{sansmath} 
\SetMathAlphabet{\mathsfbf}{sans}{\sansmathencoding}{\sfdefault}{bx}{sl}
\usepackage{etoolbox}
\AtBeginEnvironment{sansmath}{\let\bm\mathsfbf}{}{}
\usepackage{mdframed}
\usepackage{braket} % braket notation
\usepackage{amsmath, nccmath}

\usepackage{bbm}
\usepackage{dsfont} % bold numerals
\usepackage{bbold} % blackboard bold font
\usepackage{mathrsfs} % script

%% custom norm
\newcommand{\Norm}[1]{\left\lVert#1\right\rVert}
\newcommand{\Abs}[1]{\left\lvert#1\right\rvert}
%%

% circled number
\newcommand*\circlednum[1]{\raisebox{.5pt}{\textcircled{\raisebox{-.9pt} {#1}}}}

\usepackage{relsize} % math symbol size

\usepackage{amsthm}
\newtheorem{thm}{Theorem}[section]
\newtheorem{mydef}{Definition}[section]
\newtheorem{lemma}[thm]{Lemma}
\newtheorem{proposition}[thm]{Proposition}
\newtheorem{corollary}[thm]{Corollary}
\newtheorem{property}{Property}

\newtheoremstyle{problemstyle}
        {5pt} % <space above>
        {15pt} % <space below>
        {\normalfont} % <body font>
        {} % <indent amount}
        {\bfseries} % <theorem head font>
        {\normalfont\bfseries.} % <punctuation after theorem head>
        {.5em} % <space after theorem head>
        {} % <theorem head spec (can be left empty, meaning `normal')>
\theoremstyle{problemstyle}
\newtheorem{problem}{Problem}
\newtheorem{solution}{Solution}[section]
\newtheorem{example}{Example}[section]

\newenvironment{problembk}[1][]{%
  \begin{problem}[#1]$ $\par\nobreak\ignorespaces
}{%
  \end{problem}
}
\newenvironment{solutionbk}[1][]{%
  \begin{solution}[#1]$ $\par\nobreak\ignorespaces
}{%
  \end{solution}
}

\theoremstyle{remark}
\newtheorem*{remark}{Remark}

\usepackage{mdframed}
\newenvironment{fthm}
    {\begin{mdframed}\begin{thm}}
    {\end{thm}\end{mdframed}}

\newcommand{\bvec}[1]{\mathbf{#1}} % vector

% calculus
\newcommand{\D}{\mathrm{d}}
\newcommand{\Par}{\partial}
% Calligraphy (Euler script?)
\newcommand{\cA}{\mathcal{A}}
\newcommand{\cB}{\mathcal{B}}
\newcommand{\cC}{\mathcal{C}}
\newcommand{\cF}{\mathcal{F}}
\newcommand{\cH}{\mathcal{H}}
\newcommand{\cI}{\mathcal{I}}
\newcommand{\cL}{\mathcal{L}}
\newcommand{\cM}{\mathcal{M}}
\newcommand{\cP}{\mathcal{P}}
\newcommand{\cT}{\mathcal{T}}
% Calligraphy (blackboard bold)
\newcommand{\bC}{\mathbbm{C}}
\newcommand{\bD}{\mathbbm{D}}
\newcommand{\bF}{\mathbbm{F}}
\newcommand{\bH}{\mathbbm{H}}
\newcommand{\bI}{\mathbbm{I}}
\newcommand{\bK}{\mathbbm{K}}
\newcommand{\bN}{\mathbbm{N}}
\newcommand{\bO}{\mathbbm{O}}
\newcommand{\bP}{\mathbbm{P}}
\newcommand{\bQ}{\mathbbm{Q}}
\newcommand{\bR}{\mathbbm{R}}
\newcommand{\bT}{\mathbbm{T}}
\newcommand{\bZ}{\mathbbm{Z}}
% Calligraphy (fraktur font)
\newcommand{\fI}{\mathfrak{I}}
\newcommand{\fR}{\mathfrak{R}}
% lower indices (roman)
\newcommand{\rC}{\mathrm{C}} % Curie constant
\newcommand{\rf}{\mathrm{f}} % final
\newcommand{\ri}{\mathrm{i}} % initial
\newcommand{\rT}{\mathrm{T}} % isothermal

% figures
\usepackage{wrapfig} 
\usepackage{subfig}

% ckickable table of contents
\usepackage{hyperref}
\hypersetup{
    colorlinks,
    citecolor=black,
    filecolor=black,
    linkcolor=blue,
    urlcolor=black
}
\begin{document}
\begin{CJK}{UTF8}{bkai}

\title{蔣正偉 Quantum Mechanics I First Midterm Exam}
\author{Jonathan Huang (Giant Water Bird)}
\date{\today}
\cfoot{\thepage}
% header and footer
\pagestyle{fancy}
\rhead{蔣正偉}
\lhead{Quantum Mechanics I}

\maketitle

%\begin{figure}[h!]%
%    \centering
%    \subfloat{\includegraphics[width=8cm]{giant water bird.jpg}}
%    \caption{}
%\end{figure}

\begin{center}
    108-1 1st Midterm Instruction
\end{center}

    This is an open-book, 120-minute exam. Your are only allowed to use the main textbook (by Sakurai and Napolitano) and your own handwritten notes. Only derived results in the main text of Sakurai and Napolitano up to the range of the exam (i.e., Chapter 1) can be used. The score of each sub-problem is indicated by the number in square brackets. To avoid any misunderstanding, ask if you have any questions about the problems or notations. In your answers, define clearly your notations if they differ from those in the main textbook.

\newpage

\section{Problems}
Consider a spin-1/2 particle placed in a uniform magnetic field,
    \begin{equation}
        \vec{B}=\frac{B_0}{\sqrt{2}}(\hat{x}+\hat{z}),
    \end{equation}
    where $B_0 >0$. The dynamics of the particles are governed by the Hamiltonian
    \begin{equation}
        H=-γ\vec{S}\cdot\vec{B},
    \end{equation}
    where $\vec{S}$ is the vector spin operator for the particle, and $\gamma$ is a positive constant.
    
\begin{problem}[10 pts.]
    In the original Stern-Gerlach experiment, a beam of silver atoms (each carries 47 electrons) goes through a magnetic field $\vec{B}=B_0\hat{z}$, where $B_0$ has a spatial dependence, and splits into two.
    \begin{enumerate}[(a)]
        \item Explain why $B_0$ cannot be a constant.
        \item Explain why they can be seen as a spin-1/2 particles.
        \item Explain why the nuclear spin can be safely ignored in this case.
    \end{enumerate}
\end{problem}   

\begin{problem}[5 pts.]
    Calculate the matrix of the Hamiltonian in the $\{\ket{+},\ket{-}\}$ basis with 
    \begin{equation}
        S_z\ket{\pm} = \pm\frac{\hbar}{2}\ket{\pm}.
    \end{equation}
\end{problem}

\begin{problem}[10 pts.]
    Calculate the eigenvalues and the normalized eigenvectors of $H$. For the normalized eigenvectors, write them in the form of $N(1 )$, where $N$ and $\alpha$ are numbers.
\end{problem}

\begin{problem}[5 pts.]
    What is the matrix that diagonalizes $H$, with the elements of the diagonalized $H$ listed in the ascending order?
\end{problem}

\begin{problem}[10 pts.]
    At time $t=0$, the system is in the state $\ket{-}$, and the energy is measured. What are the possible outcomes of the energy measurement and with what probabilities?
\end{problem}

\begin{problem}[5 pts.]
    In the previous problem, what is the expectation value of the energy measurement?
\end{problem}

\begin{problem}[10 pts.]
    Evaluate $\Delta s_x\Delta s_z$ for the lower energy eigenstate, where $\Delta s_x$ and $\Delta s_z$ denote the uncertainties of $S_x$ and $S_z$ operators, respectively.
\end{problem}

\begin{problem}[5 pts.]
    Explain whether your result in the previous problem violates the uncertainty relation.
\end{problem}

\begin{problem}[10 pts.]
    Suppose the spin-$1/2$ particle is initially in the spin-up state along the $z$ axis, and has a fixed momentum $\bvec{p}$. Define a ket vector to describe such a state. Derive the final ket vector after one performs a finite translation of $\Delta\vec{x}$ to the particle. Explain what happens to the spin state.
\end{problem}

\begin{problem}[10 pts.]
    Suppose $\hat{n}$ is a unit vector in the $(\theta,\phi)$ direction, as shown in Fig.1(a). Find the normalized $\ket{\psi}$ such that
    \begin{equation}
        (\bvec{S}\cdot\hat{n})\ket{\psi} = +\frac{\hbar}{2}\ket{\psi}.
    \end{equation}
    Express your result in the $\{\ket{+},\ket{-}\}$ basis. Take the convention that the coefficient of $\ket{+}$ is real.
\end{problem}

\begin{problem}[10 pts.]
    Any spin-$1/2$ state can be visualized in the three dimensional space, called the Bloch sphere as shown in Fig. 1(b), as a Bloch vector, $(<σ_x>, <σ_y>,<σ_z>)$, evaluated with respect to the given state. Compute the Bloch vector for |$\ket{\psi}ψ$ in the previous problem, and show that it falls on the surface of the Bloch sphere. (The last property is a characteristic of
    pure states.)
\end{problem}

\begin{problem}[5 pts.]
    In the textbook, the spin raising and lower operators of the spin-$1/2$ particle are introduced respectively as
    \begin{equation}
        S_+ \equiv \hbar\ket{+}\bra{-}, \quad S_- \equiv \hbar\ket{-}\bra{+}.
    \end{equation}
    Suppose we know that $S_\pm = S_x\pm iS_y$ even for higher spin systems. Explain whether one can assume for a spin-$s$ particle that the raising and lowering operators are
    \begin{equation}
        S_+ = N\hbar\left(\ket{s,s}\bra{s,s-1}+\ket{s,s-1}\bra{s,s-2}+ \cdots +\ket{s,-s+1}\bra{<s,-s}\right),
    \end{equation}
    \begin{equation}
        S_- =N\hbar\left(\ket{s,s-1}\bra{s,s}+\ket{s,s-2}\bra{s,s-1}+\cdots+\ket{s,-s}\bra{s,-s+1}\right),
    \end{equation}
    where $\ket{s,m}$ denotes a spin-$s$ state with $s_z =m$, and $N$ is some overall factor.
\end{problem}

\end{CJK}
\end{document}