\documentclass[11pt]{article}

\usepackage[height=9.6in,width=6.45in]{geometry}
%\usepackage{showkeys}
\renewcommand{\baselinestretch}{1.08}

\usepackage{xspace}

\newcommand{\Poincare}{Poincar\'e\xspace}

\usepackage{amssymb}
%\usepackage{hyperref}
\usepackage{physics}
\usepackage{color}
\usepackage{graphicx}
\usepackage{srcltx}
\usepackage{mathrsfs}
\usepackage{lipsum}% Just for this example
\usepackage{slashed}
\usepackage[dvipsnames]{xcolor}
\usepackage[makeroom]{cancel}

\newcommand{\doublerule}[1][.4pt]{%
  \noindent
  \makebox[0pt][l]{\rule[.7ex]{\linewidth}{#1}}%
  \rule[.3ex]{\linewidth}{#1}}

%% My add-ons %%
%% Formatting %%
\usepackage{multicol}
\usepackage[dvipsnames]{xcolor}
\usepackage{ulem}
\usepackage{parskip}

%% Math and Symbols %%
\usepackage{amsmath,amsthm,amssymb, mathtools}
\usepackage{yhmath, faktor, dsfont}
\usepackage{academicons, wasysym, marvosym}
\usepackage[scr]{rsfso} 
\usepackage{latexsym, amsmath, amscd, amsmath, amsthm}
\usepackage{amssymb,amsmath,amsthm,graphicx,dsfont}
\usepackage{hyperref}

%% Enhancement %%
\usepackage{graphicx, tabularx}
\usepackage{booktabs}
\usepackage[shortlabels,inline]{enumitem}
%% TikZ %%
\usepackage{tikz}
\usepackage[breakable]{tcolorbox}
\usetikzlibrary{decorations.pathmorphing}
\usetikzlibrary{calc, arrows,matrix}
\usetikzlibrary{arrows.meta, decorations.pathmorphing}
\usepackage{tikz-feynman}

\newcommand{\bvec}[1]{\mathbf{#1}} % vector

%% Other packages %%
\usepackage{amsopn}

%% Traditional Chinese %%
\usepackage{CJKutf8}

%% Math environments %%
\newtheoremstyle{mystyle}
  {6pt}{15pt}% 上下間距
  {}%          內文字體
  {}%              縮排
  {\bf}%       標頭字體
  {.}%       標頭後標點
  {1em}% 內文與標頭距離
  {}% Theorem head spec (can be left empty, meaning 'normal')
\theoremstyle{mystyle}	
\newtheorem{theorem}{Theorem}
\newtheorem{definition}{Definition}
\newtheorem{example}[theorem]{Example}
\newtheorem{exercise}{Exercise}
\newtheorem{solution}{Solution}
\newtheorem{corollary}[theorem]{Corollary}
\newtheorem{property}[theorem]{Property}
\newtheorem{proposition}[theorem]{Proposition}
\newtheorem{lemma}[theorem]{Lemma}
\newtheorem{problem}{Problem}
\newtheorem{answer}{Answer}[section]
\newtheorem{fact}[theorem]{Fact}
\newtheorem*{recall}{Recall}
\newtheorem*{claim}{Claim}
\newtheorem*{observation}{Observation}

\theoremstyle{remark}
\newtheorem*{remark}{Remark}

\usepackage{listings}
\usepackage{xcolor}
\lstdefinestyle{matlab}{
  language=Matlab,
  basicstyle=\ttfamily\small,
  numbers=left,
  numberstyle=\tiny,
  stepnumber=1,
  frame=single,
  breaklines=true,
  showstringspaces=false,
  tabsize=2,
  keywordstyle=\color{blue},
  commentstyle=\color{teal!60!black}\itshape,
  stringstyle=\color{purple!70!black},
  backgroundcolor=\color{white}
}

\begin{document}
\begin{CJK}{UTF8}{bkai}

\centerline {\Large \bf Phys8018 QFT~(II) -- Fall 2025} 
\vspace{0.1cm}
\centerline {\large \bf Instructors: Chia-Hsien Shen $\&$ Chang-Tse Hsieh}
\vspace{0.2cm}
\centerline {\large \textbf{Student:} 物理三 黃紹凱 B12202004}
\doublerule\\
{\large \bf Problem Set \#2 [\textcolor{red}{110 pts}] \qquad Due \textcolor{red}{December 8, 2025}}
\\ \\
1.~\textbf{One-loop structure of QED} [\textcolor{red}{20 pts}]

In \textcolor{ForestGreen}{[Note-20251014\&21]} we argued that any photon-photon scattering amplitude from the diagram in Fig.\ 2(a) is finite, due to Lorentz covariance and Ward identity. Let's confirm this argument with explicit loop calculations. Consider the 1-loop photon four-point amplitude, which is a sum of six diagrams like Fig.\ 1(b). For each diagram, there is a logarithmically divergent part. Show that such divergence will be cancelled when summing over the six diagrams.

\begin{figure}[h]
    \centering
    \includegraphics[width=0.6\textwidth]{fig1.png}
    \caption{Photon four-point amplitudes in QED.}
\end{figure}

A useful (recursive) formula for the trace of any even number of $\gamma$ 
matrices for this calculation:
\begin{equation}
    \operatorname{tr}(\gamma^{\mu_1}\gamma^{\mu_2}\cdots\gamma^{\mu_n})
    = \sum_{k=2}^{n} (-1)^k\, g^{\mu_1 \mu_k}\;
    \operatorname{tr}\!\left(
    \gamma^{\mu_2}\cdots \cancel{\gamma^{\mu_k}}\cdots \gamma^{\mu_n}
    \right). \tag{1}
\end{equation}

For example, for $n=6$,
\begin{align}
    \operatorname{tr}(\gamma^{\mu_1}\gamma^{\mu_2}\cdots\gamma^{\mu_6})
    ={} &\;
    g^{\mu_1\mu_2}\, \operatorname{tr}(\gamma^{\mu_3}\gamma^{\mu_4}\gamma^{\mu_5}\gamma^{\mu_6})
    - g^{\mu_1\mu_3}\, \operatorname{tr}(\gamma^{\mu_2}\gamma^{\mu_4}\gamma^{\mu_5}\gamma^{\mu_6}) \notag \\
    &+ g^{\mu_1\mu_4}\,\operatorname{tr}(\gamma^{\mu_2}\gamma^{\mu_3}\gamma^{\mu_5}\gamma^{\mu_6})
    - g^{\mu_1\mu_5}\,\operatorname{tr}(\gamma^{\mu_2}\gamma^{\mu_3}\gamma^{\mu_4}\gamma^{\mu_6}) \notag \\
    &+ g^{\mu_1\mu_6}\,\operatorname{tr}(\gamma^{\mu_2}\gamma^{\mu_3}\gamma^{\mu_4}\gamma^{\mu_5}).
    \tag{2}
\end{align}

\begin{solution}
We need to show that the logarithmically divergent part of the one-loop photon-photon scattering amplitude vanishes. In momentum space with loop momentum $ p $, the amplitude for the permutation $ \mu , \nu , \rho , \sigma $ is 
\[
    \mathcal{M} = -e^4 \int \mathrm{d}^4 p\, \frac{\operatorname{tr} \left[ \gamma^{\mu} (\slashed{p} + m) \gamma^{\nu} (\slashed{p} + \slashed{k}_{\nu} + m) \gamma^{\rho} (\slashed{p} + \slashed{k}_{\nu} + \slashed{k}_{\rho} + m) \gamma^{\sigma} (\slashed{p} - \slashed{k}_{\mu} + m) \right]}{((p^{2} + k_1)^2 - m^2)((p+k_2)^2 - m^2)((p+k_3)^2 - m^2)((p+k_4)^2 - m^2)}.
\]
The denominator scales as $ p^8 $, the measure is $ \mathrm{d}^4 p \sim \mathrm{d}p\, p^3 $, so we require the numerator of the logarithmically divergent part scale as $ p^4 $. Isolate the leading term, asymptotically it is 
\[
    I \sim -e^4 \int \frac{\mathrm{d}^4 p}{p^8}\, \operatorname{tr} \left[ \gamma^{\mu} \slashed{p} \gamma^{\nu} \slashed{p} \gamma^{\rho} \slashed{p} \gamma^{\sigma} \slashed{p} \right].
\]
Notice that since $ \{\gamma^{\mu}, \gamma^{\nu}\} = 2 g^{\mu \nu} $, we have an identity for sandwiched gamma matrices:
\[
    \slashed{p} \gamma^{\mu} \slashed{p} = p_{\alpha} \gamma^{\alpha} \gamma^{\mu} \slashed{p} = p_{\alpha} (2 g^{\mu \alpha} - \gamma^{\mu} \gamma^{\alpha}) \slashed{p} = 2 p^{\mu} \slashed{p} - \gamma^{\mu} p_{\alpha} \gamma^{\alpha} \slashed{p} = 2 p^{\mu} \slashed{p} - p^2 \gamma^{\mu}, 
\] 
where 
\[
    \slashed{p} \slashed{p} = p_\mu p_\nu \gamma^\mu \gamma^\nu = p_\mu p_\nu \left( \frac{\{\gamma^\mu, \gamma^\nu\}}{2} + \frac{[\gamma^\mu, \gamma^\nu]}{2} \right) = p_\mu p_\nu frac{\{\gamma^\mu, \gamma^\nu\}}{2} = p^2.
\]
The trace is then decomposed into two parts \textbf{1.} and \textbf{2.}:
\[
    \operatorname{tr} \left[ \gamma^{\mu} \slashed{p} \gamma^{\nu} \slashed{p} \gamma^{\rho} \slashed{p} \gamma^{\sigma} \slashed{p} \right] = \operatorname{tr} \left[ \gamma^{\mu} \slashed{p} \gamma^{\nu} \left(2p^{\rho} \slashed{p} - p^2 \gamma^{\rho} \right) \gamma^{\sigma} \slashed{p} \right] = 2 p^{\rho} \operatorname{tr} \left[ \gamma^{\mu} \slashed{p} \gamma^{\nu} \slashed{p} \gamma^{\sigma} \slashed{p} \right] - p^2 \operatorname{tr} \left[ \gamma^{\mu} \slashed{p} \gamma^{\nu} \gamma^{\rho} \gamma^{\sigma} \slashed{p} \right] . 
\]
\textbf{1.} The trace in the first term is  
\[
    \operatorname{tr} \left[ \gamma^{\mu} \slashed{p} \gamma^{\nu} \slashed{p} \gamma^{\sigma} \slashed{p} \right] = \operatorname{tr} \left[ \gamma^{\mu} \left( 2 p^{\nu} \slashed{p} - p^2 \gamma^{\nu} \right) \gamma^{\sigma} \slashed{p} \right] = 2 p^{\nu} \operatorname{tr} \left[ \gamma^{\mu} \slashed{p} \gamma^{\sigma} \slashed{p} \right] - p^2 \operatorname{tr} \left[ \gamma^{\mu} \gamma^{\nu} \gamma^{\sigma} \slashed{p} \right].
\]
The traces with four matrices can be evaluated using (1): 
\begin{align*}
    \operatorname{tr} \left[ \gamma^{\mu} \slashed{p} \gamma^{\sigma} \slashed{p} \right] &= p_{\alpha} p_{\beta} \operatorname{tr} \left[ \gamma^{\mu} \gamma^{\alpha} \gamma^{\sigma} \gamma^{\beta} \right] \\
    &= p_{\alpha} p_{\beta} \left( g^{\mu \alpha} \operatorname{tr} \left[ \gamma^{\sigma} \gamma^{\beta} \right] - g^{\mu \sigma} \operatorname{tr} \left[ \gamma^{\alpha} \gamma^{\beta} \right] + g^{\mu \beta} \operatorname{tr} \left[ \gamma^{\alpha} \gamma^{\sigma} \right] \right) \\
    &= p_{\alpha} p_{\beta} \left( 4 g^{\mu \alpha} g^{\sigma \beta} - 4 g^{\mu \sigma} g^{\alpha \beta} + 4 g^{\mu \beta} g^{\alpha \sigma} \right) \\
    &= 8 p^{\mu} p^{\sigma} - 4 g^{\mu \sigma} p^2.
\end{align*}
Similarly, $ \operatorname{tr} \left[ \gamma^{\mu} \gamma^{\nu} \gamma^{\sigma} \slashed{p} \right] = p_{\alpha} \operatorname{tr} \left[ \gamma^{\mu} \gamma^{\nu} \gamma^{\sigma} \gamma^{\alpha} \right] $, and 
\[
     p_{\alpha} \operatorname{tr} \left[ \gamma^{\mu} \gamma^{\nu} \gamma^{\sigma} \gamma^{\alpha} \right]= p_{\alpha} \left( g^{\mu \nu} \operatorname{tr} \left[ \gamma^{\sigma} \gamma^{\alpha} \right] - g^{\mu \sigma} \operatorname{tr} \left[ \gamma^{\nu} \gamma^{\alpha} \right] + g^{\mu \alpha} \operatorname{tr} \left[ \gamma^{\nu} \gamma^{\sigma} \right] \right) = 4 (g^{\mu \nu} p^{\sigma} - g^{\mu \sigma} p^{\nu} + g^{\nu \sigma} p^{\mu}) .
\]
Combining them gives 
\[
    \mathbf{1.} = 16 p^{\mu} p^{\nu} p^{\sigma} - 4 p^2 (g^{\mu \nu} p^{\sigma} + g^{\mu \sigma} p^{\nu} + g^{\nu \sigma} p^{\mu}).
\]

\textbf{2.} Using the cyclic property $ \operatorname{tr} \left[ \gamma^{\mu} \slashed{p} \gamma^{\nu} \gamma^{\rho} \gamma^{\sigma} \slashed{p} \right] = \operatorname{tr} \left[ \slashed{p} \gamma^{\mu} \slashed{p} \gamma^{\nu} \gamma^{\rho} \gamma^{\sigma} \right] $, the trace in the second term is 
\[
     \operatorname{tr} \left[ \slashed{p} \gamma^{\mu} \slashed{p} \gamma^{\nu} \gamma^{\rho} \gamma^{\sigma} \right] = \operatorname{tr} \left[ \left(2 p^{\mu} \slashed{p} - p^2 \gamma^{\mu} \right) \gamma^{\nu} \gamma^{\rho} \gamma^{\sigma} \right] = 2 p^{\mu} \operatorname{tr} \left[ \slashed{p} \gamma^{\nu} \gamma^{\rho} \gamma^{\sigma} \right] - p^2 \operatorname{tr} \left[ \gamma^{\mu} \gamma^{\nu} \gamma^{\rho} \gamma^{\sigma} \right] . 
\]
We have 
\begin{align*}
    \operatorname{tr} \left[ \slashed{p} \gamma^{\nu} \gamma^{\rho} \gamma^{\sigma} \right] &= p_{\alpha} \operatorname{tr} \left[ \gamma^{\alpha} \gamma^{\nu} \gamma^{\rho} \gamma^{\sigma} \right] \\
    &= p_{\alpha} \left( g^{\alpha \nu} \operatorname{tr} \left[ \gamma^{\rho} \gamma^{\sigma} \right] - g^{\alpha \rho} \operatorname{tr} \left[ \gamma^{\nu} \gamma^{\sigma} \right] + g^{\alpha \sigma} \operatorname{tr} \left[ \gamma^{\nu} \gamma^{\rho} \right] \right) \\
    &= 4 (p^{\nu} g^{\rho \sigma} - p^{\rho} g^{\nu \sigma} + p^{\sigma} g^{\nu \rho}), \\
    \operatorname{tr} \left[ \gamma^{\mu} \gamma^{\nu} \gamma^{\rho} \gamma^{\sigma} \right] &= g^{\mu \nu} \operatorname{tr} \left[ \gamma^{\rho} \gamma^{\sigma} \right] - g^{\mu \rho} \operatorname{tr} \left[ \gamma^{\nu} \gamma^{\sigma} \right] + g^{\mu \sigma} \operatorname{tr} \left[ \gamma^{\nu} \gamma^{\rho} \right] \\
    &= 4 (g^{\mu \nu} g^{\rho \sigma} - g^{\mu \rho} g^{\nu \sigma} + g^{\mu \sigma} g^{\nu \rho}).
\end{align*}
Combining them gives 
\[
    \mathbf{2.} = 8 p^{\mu} (p^{\nu} g^{\rho \sigma} - p^{\rho} g^{\nu \sigma} + p^{\sigma} g^{\nu \rho}) - 4 p^2 (g^{\mu \nu} g^{\rho \sigma} - g^{\mu \rho} g^{\nu \sigma} + g^{\mu \sigma} g^{\nu \rho}). 
\]
By the result of \textbf{1.} and \textbf{2.}, the trace is given by 
\begin{align*}
    \operatorname{tr} \left[ \gamma^{\mu} \slashed{p} \gamma^{\nu} \slashed{p} \gamma^{\rho} \slashed{p} \gamma^{\sigma} \slashed{p} \right] &= 2 p^{\rho} (\mathbf{1.}) - p^2 (\mathbf{2.}) \\
    &= 32 p^{\mu} p^{\nu} p^{\rho} p^{\sigma} - 8 p^2 (g^{\mu \nu} p^{\rho} p^{\sigma} + g^{\mu \sigma} p^{\rho} p^{\nu} + g^{\nu \sigma} p^{\rho} p^{\mu}) \\
    &\quad - 8 p^2 p^{\mu} (p^{\nu} g^{\rho \sigma} - p^{\rho} g^{\nu \sigma} + p^{\sigma} g^{\nu \rho}) + 4 p^4 (g^{\mu \nu} g^{\rho \sigma} - g^{\mu \rho} g^{\nu \sigma} + g^{\mu \sigma} g^{\nu \rho}) \\
    &= 32 p^{\mu} p^{\nu} p^{\rho} p^{\sigma} - 8 p^2 (g^{\mu \nu} p^{\rho} p^{\sigma} + g^{\mu \sigma} p^{\rho} p^{\nu} + g^{\rho \sigma} p^{\mu} p^{\nu} + g^{\nu \rho} p^{\mu} p^{\sigma}) \\
    &\quad + 4 p^4 (g^{\mu \nu} g^{\rho \sigma} - g^{\mu \rho} g^{\nu \sigma} + g^{\mu \sigma} g^{\nu \rho})
\end{align*}

To carry out integration, there are two rules we need to use: 
\begin{enumerate}
    \item Since the average of $ p^{\mu} p^{\nu} \propto g^{\mu \nu} $, contractiing both sides with $ g_{\mu \nu} $ gives
    \[
        \int \mathrm{d}^4 p\, p^{\mu} p^{\nu} f(p^2) = \frac{g^{\mu \nu}}{4} \int \mathrm{d}^4 p\, p^2 f(p^2),
    \]

    \item The tensor $ p^{\mu} p^{\nu} p^{\rho} p^{\sigma} $ is totally symmetric in all indices, so its average must be a sum of products of metric tensors that is also totally symmetric. The only such combination is $ (g^{\mu \nu} g^{\rho \sigma} + g^{\mu \rho} g^{\nu \sigma} + g^{\mu \sigma} g^{\nu \rho}) $. Contracting both sides gives 
    \[
        \int \mathrm{d}^4 p\, p^{\mu} p^{\nu} p^{\rho} p^{\sigma} f(p^2) = \frac{1}{24} (g^{\mu \nu} g^{\rho \sigma} + g^{\mu \rho} g^{\nu \sigma} + g^{\mu \sigma} g^{\nu \rho}) \int \mathrm{d}^4 p\, p^4 f(p^2), 
    \]
    where $ f $ is some function.
\end{enumerate}

Writing the divergent part as $ K $, the first term in $ \operatorname{tr} \left[ \gamma^{\mu} \slashed{p} \gamma^{\nu} \slashed{p} \gamma^{\rho} \slashed{p} \gamma^{\sigma} \slashed{p} \right] $ is 
\[
    32 \int \mathrm{d}^4 p\, \frac{p^{\mu} p^{\nu} p^{\rho} p^{\sigma}}{p^8} = \frac{32}{24} (g^{\mu \nu} g^{\rho \sigma} + g^{\mu \rho} g^{\nu \sigma} + g^{\mu \sigma} g^{\nu \rho}) K, 
\]
the second term is
\[
    -8 \int \mathrm{d}^4 p\, \frac{p^2 (g^{\mu \nu} p^{\rho} p^{\sigma} + g^{\mu \sigma} p^{\rho} p^{\nu} + g^{\rho \sigma} p^{\mu} p^{\nu} + g^{\nu \rho} p^{\mu} p^{\sigma})}{p^8} = -\frac{8 \cdot 2}{4} (g^{\mu \nu} g^{\rho \sigma} + g^{\mu \sigma} g^{\nu \rho}) K,
\]
and the third term is
\[
    4 \int \mathrm{d}^4 p\, \frac{p^4 (g^{\mu \nu} g^{\rho \sigma} - g^{\mu \rho} g^{\nu \sigma} + g^{\mu \sigma} g^{\nu \rho})}{p^8} = 4 (g^{\mu \nu} g^{\rho \sigma} - g^{\mu \rho} g^{\nu \sigma} + g^{\mu \sigma} g^{\nu \rho}) K.
\]
Together, the contributions to each tensor structure are: $ \frac{4}{3} - 4 + 4 = \frac{4}{3} $ for $ g^{\mu \nu} g^{\rho \sigma} $, $ \frac{4}{3} + 0 - 4 = -\frac{8}{3} $ for $ g^{\mu \rho} g^{\nu \sigma} $, and $ \frac{4}{3} - 4 + 4 = \frac{4}{3} $ for $ g^{\mu \sigma} g^{\nu \rho} $. In total, for the permutation $ (\mu , \nu , \rho , \sigma) $ the divergent part of $ \mathcal{M} $ is  
\[
    \propto \frac{4}{3} \left(g^{\mu \nu} g^{\rho \sigma} - 2 g^{\mu \rho} g^{\nu \sigma} + g^{\mu \sigma} g^{\nu \rho}\right) K.
\]
The vertex $ \rho $ opposite to $ \mu $ gets a factor of $ -2 $, while neighboring vertices $ \nu $ and $ \sigma $ get a factor of $ 1 $. Summing over all six permutations, each vertex will be opposite to $ \mu $ twice and neighbor to $ \mu $ four times, so the total coefficient is zero, agreeing with the Ward identity.
\end{solution}

\newpage

2.~\textbf{Field-strength renormalization in $\varphi^4$ theory} [\textcolor{red}{25 pts}]

In \textcolor{ForestGreen}{[Note-20251007]} we learned how to fix the counterterm coefficients $\delta_Z$, $\delta_m$, and $\delta_\lambda$ for the $\lambda \varphi^4$ theory from the renormalization conditions and loop diagrams. The 1-loop calculations already give some nontrivial results for the mass and coupling counterterms $\delta_m$ and $\delta_\lambda$, while the field strength counterterm $\delta_Z$ is still trivial (zero) at this level. In this problem, we will identify the first nonzero contributions to $\delta_Z$ from 2-loop calculations. We thus look at all 2-loop 1PI (one-particle irreducible) diagrams for the self-energy $M_{\text{2-loop}}^2(p^2)$, as shown in Fig.\ 2. 

\begin{figure}[h]
    \centering
    \includegraphics[width=0.6\textwidth]{fig2.png}
    \caption{2-loop 1PI diagrams for $ M_{2-\text{loop}}^2 (p^2) $.}
\end{figure}

(a) Argue why \emph{only} the first diagram, the “sunset diagram”, contributes to $\delta_Z$.

(b) As we are going to determine $\delta_Z$, we will compute $\mathrm{d}M_{\text{sun}}^2/\mathrm{d}p^2$, where $M_{\text{sun}}^2(p^2)$ is the self-energy associated with the sunset diagram. Here are the steps:

(b-i) Write the amplitude $-iM^2_{\text{sun}}(p^2)$ from the sunset 
diagram. Express the momentum integral using Feynman parametrization, as
\begin{equation}
    -i M^2_{\text{sun}}(p^2)
    = (\text{number}) \times \lambda^2
    \int_0^1 \! dx\, dy\, dz\; \delta(x+y+z-1)
    \int \frac{d^4 q}{(2\pi)^4} \frac{d^4 k}{(2\pi)^4}\;
    \frac{1}{L^3}, \tag{3}
\end{equation}
where $L$ depends on loop momenta $q$ and $k$ and Feynman parameters $x,y,z$.

(b-ii) Shift $q$ and $k$ to some $\ell_1$ and $\ell_2$ so that $L$ 
takes a sum-of-squares form, yielding
\begin{equation}
    -iM^2_{\text{sun}}(p^2)
    = (\text{number}) \times \lambda^2
    \int_0^1 dx\, dy\, dz\; \delta(x+y+z-1)
    \int \frac{d^4 \ell_1}{(2\pi)^4}\, \frac{d^4 \ell_2}{(2\pi)^4}\;
    \frac{1}{(\alpha \ell_1^2 + \beta \ell_2^2 + \gamma p^2 - m^2 + i\epsilon)^3}. \tag{4}
\end{equation}

(b-iii) Take $i(d/dp^2)$ of both sides of (4). Wick-rotate the 
integral regarding $dM_{\text{sun}}^2/dp^2$ to Euclidean space, with 
$\ell^0_{E,i} = - i \ell^0_i$, $i=1,2$, and then use dimensional 
regularization to perform the momentum integral. Assemble your results as
\begin{equation}
    \frac{d M^2_{\text{sun}}(p^2)}{dp^2}
    = (\text{number}) \times \lambda^2
    \int_0^1 dx\, dy\, dz\; \delta(x+y+z-1)\, F(x,y,z)
    \left[
    \frac{1}{\epsilon} + (\text{finite terms})
    \right], \tag{5}
\end{equation}
where $F(x,y,z)$ is some function depending only on Feynman parameters.

(c) Requiring the renormalization condition
\[
\left.\frac{dM^2_{\text{ren}}(p^2)}{dp^2}\right|_{p^2=m^2}
= \left.\frac{dM^2_{\text{sun}}(p^2)}{dp^2}\right|_{p^2=m^2}
- \delta_Z = 0,
\]
we then have
\begin{equation}
\delta_Z = \left.\frac{dM^2_{\text{sun}}(p^2)}{dp^2}\right|_{p^2=m^2}
= (\text{number})\cdot \lambda^2 \left[ \frac{1}{\epsilon} + (\text{finite terms}) \right]. \tag{6}
\end{equation}
Find the exact values of the "number" and the "finite terms" in (6).

\begin{solution}
    ~

    (a) Since $ \delta_Z $ is fixed by the $ p^2 $-dependence of self-energy, only the sunset diagram contributes to $ \delta_Z $. The other diagram is independent of external momentum $ p $, so they do not contribute to $ \delta_Z $.

    (b-i) The amplitude from the sunset diagram is
    \[
        -i M^2_{\text{sun}}(p^2)
        = (-i \lambda)^2 \frac{1}{6}
        \int \frac{d^4 q}{(2\pi)^4} \frac{d^4 k}{(2\pi)^4}\;
        \frac{i}{q^2 - m^2 + i\epsilon}
        \frac{i}{k^2 - m^2 + i\epsilon}
        \frac{i}{(p - q - k)^2 - m^2 + i\epsilon}.
    \]
    Using Feynman parametrization as given in the , we have
    \[
        \begin{split}
            -i M^2_{\text{sun}}(p^2) &= -i (-i \lambda)^2 \frac{1}{6} (2!)  \int_0^1 \mathrm{d}x\, \mathrm{d}y\, \mathrm{d}z\; \delta(x+y+z-1) \int \frac{d^4 q}{(2\pi)^4} \frac{d^4 k}{(2\pi)^4}\; \frac{1}{L^3} \\
            &= \frac{i \lambda^2}{3} \int_0^1 \mathrm{d}x\, \mathrm{d}y \, \mathrm{d}z \, \delta(x+y+z-1) \int \frac{d^4 q}{(2\pi)^4} \frac{d^4 k}{(2\pi)^4}\; \frac{1}{L^3},
        \end{split}
    \]
    where
    \[
        L = x(q^2 - m^2 + i\epsilon) + y(k^2 - m^2 + i\epsilon) + z((p - q - k)^2 - m^2 + i\epsilon).
    \]
    
    (b-ii) Shifting $ q $ and $ k $ to $ \ell_1 = q - \frac{z}{x+z} p $ and $ \ell_2 = k - \frac{z}{y+z} p $, we have 
    \[
        L = (x+z) \ell_1^2 + (y+z) \ell_2^2 + \left( \frac{xyz}{(x+z)(y+z)} p^2 - m^2 + i\epsilon \right).
    \]
    Comparing with (5), we identify $ \alpha = x+z $, $ \beta = y+z $, and $ \gamma = \frac{xyz}{(x+z)(y+z)} $, and the number in the front of the integral is $ \frac{1}{3} $. Therefore, we have 
    \[
        -i M_{\text{sum}}^2 (p^2) = \frac{i \lambda^2}{3} \int_0^1 \! \mathrm{d}x\, \mathrm{d}y\, \mathrm{d}z\; \delta(x+y+z-1) \int \frac{\mathrm{d}^4 \ell_1}{(2\pi)^4}\, \frac{\mathrm{d}^4 \ell_2}{(2\pi)^4}\; \frac{1}{(\alpha \ell_1^2 + \beta \ell_2^2 + \gamma p^2 - m^2 + i\epsilon)^3}.  
    \]

    (b-iii) Taking $ i(d/dp^2) $ of both sides of (4) and applying Leibniz' rule of differentiating, we get 
    \[
        \begin{split}
            \frac{\mathrm{d}M_{\text{sum}}^2 (p^2)}{\mathrm{d}p^2} &= -\frac{\lambda^2}{3} \int_0^1 \! \mathrm{d}x\, \mathrm{d}y\, \mathrm{d}z\; \delta(x+y+z-1) \int \frac{\mathrm{d}^4 \ell_1}{(2\pi)^4}\, \frac{\mathrm{d}^4 \ell_2}{(2\pi)^4}\; \frac{\partial}{\partial p^2} \left[\frac{1}{(\alpha \ell_1^2 + \beta \ell_2^2 + \gamma p^2 - m^2 + i\epsilon)^3}\right] \\
            &= \gamma \lambda^2 \int_0^1 \! \mathrm{d}x\, \mathrm{d}y\, \mathrm{d}z\; \delta(x+y+z-1)\, \int \frac{\mathrm{d}^4 \ell_1}{(2\pi)^4}\, \frac{\mathrm{d}^4 \ell_2}{(2\pi)^4}\; \frac{1}{(\alpha \ell_1^2 + \beta \ell_2^2 + \gamma p^2 - m^2 + i\epsilon)^4}.
        \end{split}
    \]
    Wick-rotate by setting $ \ell^0_{E,i} = - i \ell^0_i $, $ i=1,2 $, we have $ L = \alpha \ell_1^2 + \beta \ell_2^2 + \gamma p^2 - m^2 \to - \alpha (\ell_{E,1})^2 - \beta (\ell_{E,2})^2 + \gamma p^2 - m^2 $, and hence  
    \[
        \frac{\mathrm{d}M_{\text{sun}}^2 (p^2)}{\mathrm{d}p^2} = -\gamma \lambda^2 \int_0^1 \! \mathrm{d}x\, \mathrm{d}y\, \mathrm{d}z\, \delta(x+y+z-1)\, \int \frac{\mathrm{d}^4 \ell_{E,1}}{(2\pi)^4}\, \frac{\mathrm{d}^4 \ell_{E,2}}{(2\pi)^4}\; \frac{1}{(\alpha (\ell_{E,1})^2 + \beta (\ell_{E,2})^2 - \gamma p^2 + m^2)^4}.
    \]
    Using dimensional regularization, introduce the parameter $ \mu $ such that 
    \[
      \int \frac{\mathrm{d}^4 \ell_1}{(2 \pi)^4} \to \mu^{4-d} \int \frac{\mathrm{d}^d \ell_1}{(2\pi)^d}, \quad \int \frac{\mathrm{d}^4 \ell_2}{(2 \pi)^4} \to \mu^{4-d} \int \frac{\mathrm{d}^d \ell_2}{(2\pi)^d}.
    \]
    Then, retaining the Euclidean signature in our integral, we have 
    \[
        \frac{\mathrm{d}M_{\text{sum}}^2 (p^2)}{\mathrm{d}p^2} = -\gamma \lambda^2 \mu^{8-2d} \int_0^1 \! \mathrm{d}x\, \mathrm{d}y\, \mathrm{d}z\, \delta(x+y+z-1) \int \frac{\mathrm{d}^d \ell_{E,1}}{(2\pi)^d}\, \frac{\mathrm{d}^d \ell_{E,2}}{(2\pi)^d}\; \frac{1}{(\alpha (\ell_{E,1})^2 + \beta (\ell_{E,2})^2 - \gamma p^2 + m^2)^4}
    \]
    where 
    \[
        \begin{split}
            &\int \frac{\mathrm{d}^d \ell_{E,1}}{(2\pi)^d}\, \frac{\mathrm{d}^d \ell_{E,2}}{(2\pi)^d}\; \frac{1}{(\alpha (\ell_{E,1})^2 + \beta (\ell_{E,2})^2 - \gamma p^2 + m^2)^4} \\
            &= \frac{1}{6} \int \frac{\mathrm{d}^d \ell_{E,1}}{(2\pi)^d}\, \frac{\mathrm{d}^d \ell_{E,2}}{(2\pi)^d}\; \int_0^{\infty} \mathrm{d}t\, t^3 \exp \left\{ -t \left[\alpha (\ell_{E,1})^2 + \beta (\ell_{E,2})^2 - \gamma p^2 + m^2 \right] \right\} \\
            &= \frac{1}{6} \int_0^{\infty} \mathrm{d}t\, t^3 e^{t(\gamma p^2 - m^2)} \left( \int \frac{\mathrm{d}^d \ell_{E,1}}{(2\pi)^d}\; e^{-t \alpha (\ell_{E,1})^2} \right) \left( \int \frac{\mathrm{d}^d \ell_{E,2}}{(2\pi)^d}\; e^{-t \beta (\ell_{E,2})^2} \right) \\
            &= \frac{1}{6} (4 \pi \alpha t)^{-d/2} (4 \pi \beta t)^{-d/2} \, \int_0^{\infty} \mathrm{d}t\, t^3 e^{-t (- \gamma p^2 + m^2)} = \frac{1}{6 (4 \pi )^{d} (\alpha \beta)^{d/2}} \Gamma (4-d) (m^2 - \gamma p^2)^{d-4}.
        \end{split}
    \]
    Here we used $ 6/A^4 = \int_0^{\infty} \mathrm{d}t\, t^3 e^{-At} $. Note that $ \Gamma (4-d) $ has a pole at $ d=4 $, but no poles for $ d<4 $. Let $ d = 4 - 2 \varepsilon $ for some small $ \varepsilon > 0 $, we have
    \[
        \Gamma  (2 \varepsilon) x^{\varepsilon} \sim \frac{1}{2 \varepsilon} - \gamma_E + \frac{1}{2} \log x + O(\varepsilon),
    \]
    where $ \gamma_E $ is the Euler-Mascheroni constant. Therefore, 
    \[
        \begin{split}
            &\frac{1}{(4 \pi )^{d} (\alpha \beta)^{d/2}} \Gamma (4-d) (m^2 - \gamma p^2)^{d-4} = \frac{1}{(4 \pi)^4 (\alpha \beta)^2} \Gamma (2 \varepsilon) \left(\frac{4 \pi \mu^2 \sqrt{\alpha \beta}}{m^2 - \gamma p^2}\right)^{2 \varepsilon} \\
            &\quad \sim \frac{1}{(4 \pi)^4 (\alpha \beta)^2} \left[ \frac{1}{2 \varepsilon} + \log \left( \frac{4 \pi \mu^2 \sqrt{\alpha \beta}}{m^2 - \gamma p^2} \right) - \gamma_E \right] \\
            &\quad = \frac{1}{12 (4 \pi)^4 (\alpha \beta)^2} \left[ \frac{1}{\epsilon} + 2 \log \frac{\mu^2}{m^2} + 2 \log (4 \pi) - 2 \gamma_E + \log \left(\frac{\alpha \beta}{(1 - (p^2 / m^2)\gamma )^2}\right) \right]. 
        \end{split}
    \]
    Plugging into the integral gives 
    \[
        \begin{split}
            \frac{\mathrm{d}M_{\text{sun}}^2 (p^2)}{\mathrm{d}p^2} &= - \frac{\lambda^2}{12 (4 \pi)^4} \frac{\gamma}{(\alpha \beta)^2} \int_0^1 \mathrm{d}x\, \mathrm{d}y\, \mathrm{d}z\, \delta (x+y+z-1) \frac{xyz}{(xy + xz + yz)^3} \\
            &\quad \times \left\{ \frac{1}{\varepsilon} + 2 \log \frac{\mu^2}{m^2} + 2 \log (4 \pi) - 2 \gamma_E + \log \left(\frac{\alpha \beta}{(1 - (p^2 / m^2)\gamma )^2}\right)  \right\}
        \end{split}
    \]

    (c) Set $ p^2 = m^2 $ in the above integral and plug in $ \alpha (x,y,z) $, $ \beta (x,y,z) $, $ \gamma (x,y,z) $, we have 
    \[
        \begin{split}
            \frac{\mathrm{d}M_{\text{sun}}^2 (p^2)}{\mathrm{d}p^2} &= - \frac{\lambda^2}{12 (4 \pi)^4} \frac{\gamma}{(\alpha \beta)^2} \int_0^1 \mathrm{d}x\, \mathrm{d}y\, \mathrm{d}z\, \delta (x+y+z-1) \frac{xyz}{(xy + xz + yz)^3} \\
            &\quad \times \left\{ \frac{1}{\varepsilon} + 2 \log \frac{\mu^2}{m^2} + 2 \log (4 \pi) - 2 \gamma_E + \log \left(\frac{\alpha \beta}{(1 - (p^2 / m^2)\gamma )^2}\right) \log \frac{(xy + yz + zx)^3}{(xy + yz + zx - xyz)^2} \right\}
        \end{split}
    \]
    The integrals are computed to be 
    \[
        \int_0^1 \mathrm{d}x\, \mathrm{d}y\, \mathrm{d}z\, \delta (x+y+z-1) \frac{xyz}{(xy + yz + zx)^3} = \frac{1}{2}, 
    \]
    \[
        \int_0^1 \mathrm{d}x\, \mathrm{d}y\, \mathrm{d}z\, \delta (x+y+z-1) \frac{xyz}{(xy + yz + zx)^3} \log \frac{(xy + yz + zx)^3}{(xy + yz + zx - xyz)^2} = - \frac{3}{4}.
    \]
    Finally, we have 
    \[
        \begin{split}
            \delta_Z &= - \frac{\lambda^2}{6144 \pi^4} \left\{ \frac{1}{\epsilon} + 2 \log \frac{\mu^2}{m^2} + 2 \log (4 \pi) - 2 \gamma_E - \frac{3}{2} \right\}.
        \end{split}
    \]
\end{solution}

\newpage 
3.~\textbf{RG functions in $\varphi^4$ theory} [\textcolor{red}{20 pts}]

The Lagrangian of $\varphi^4$ theory in terms of renormalized field and 
parameters is
\begin{equation}
\mathcal{L}
= \frac{1}{2} Z_\varphi (\partial \varphi_r)^2
- \frac{m^2}{2} Z_m Z_\varphi \varphi_r^2
- \mu^\epsilon \frac{\lambda}{4!} Z_\lambda Z_\varphi^2 \varphi_r^4,
\qquad \epsilon = 4 - d,
\tag{7}
\end{equation}
from which we can define the renormalization group (RG) functions:
\begin{equation}
\gamma(\lambda) \equiv \frac{1}{2} \mu \frac{1}{Z_\varphi}\frac{d Z_\varphi}{d\mu},
\qquad
\gamma_m(\lambda) \equiv \frac{\mu}{m^2}\frac{d m^2}{d\mu},
\qquad
\beta(\lambda) \equiv \mu \frac{d\lambda}{d\mu}.
\tag{8}
\end{equation}

(a) The renormalizability of $\varphi^4$ theory ensures the finiteness of these RG functions, basically determined from the renormalization constants 
$Z_\varphi$, $Z_m$, and $Z_\lambda$ (which are actually functions of $\lambda$ and $\epsilon$). Specifically, let’s write the renormalization constants in $1/\epsilon$ expansions,
\begin{equation}
Z_\varphi(\lambda,\epsilon)
= 1 + \sum_{n=1}^\infty Z_{\varphi,n}(\lambda)\, \frac{1}{\epsilon^n},
\qquad
Z_m(\lambda,\epsilon)
= 1 + \sum_{n=1}^\infty Z_{m,n}(\lambda)\, \frac{1}{\epsilon^n},
\qquad
Z_\lambda(\lambda,\epsilon)
= 1 + \sum_{n=1}^\infty Z_{\lambda,n}(\lambda)\, \frac{1}{\epsilon^n}.
\tag{9}
\end{equation}

Find explicit expressions for $\beta(\lambda)$, $\gamma_m(\lambda)$, and $\gamma(\lambda)$ in terms of $\epsilon$ and the expansion coefficients $Z_{\varphi,n}(\lambda)$, $Z_{m,n}(\lambda)$, and $Z_{\lambda,n}(\lambda)$ (as well as their derivatives with respect to $\lambda$). Accordingly, there exist an (infinite) set of relations among the expansion coefficients; 
find them as well.

(b) The renormalization constants up to two loops are
\begin{align*}
Z_\varphi &= 1 + \frac{1}{\epsilon} (\text{number}) \cdot \lambda^2, \\
Z_m &= 1 + \frac{1}{\epsilon}
\left[
  \frac{\lambda}{(4\pi)^2}
  - \frac{5\lambda^2}{2 \cdot (4\pi)^4}
\right]
+ \frac{1}{\epsilon^2}\frac{\lambda^2}{2 \cdot (4\pi)^4}, \\
Z_\lambda &= 1 + \frac{1}{\epsilon}
\left[
  \frac{3\lambda}{(4\pi)^2}
  - \frac{17\lambda^2}{6 \cdot (4\pi)^4}
\right]
+ \frac{1}{\epsilon^2}\frac{9\lambda^2}{(4\pi)^4}.
\tag{10}
\end{align*}

Here the "number" in $Z_\varphi$ is the one you computed in Problem 2(c) ($Z_\varphi = 1 + \delta Z$). Find $\beta(\lambda)$, $\gamma_m(\lambda)$, and $\gamma(\lambda)$ with these $Z$’s using the result from part (a). What is the location of the Wilson–Fisher fixed point up to this order? Is the critical exponent $\nu = 1/(2 - \gamma_m)$ at $\epsilon = 1$ closer to the real value $\nu \sim 0.63$ than the one-loop result $\nu = 0.5$?

\begin{solution}
~ 

(a) From the given Lagrangian, the bare parameters are related to the renormalized ones by $ \phi_0 = \sqrt{Z_{\phi}} \phi $, $ m_0^2 = Z_m m^2 $, and $ \lambda_0 = Z_{\lambda} (\lambda, \epsilon) \mu^{\epsilon} \lambda $. Since $ Z_{\lambda}(\lambda , \epsilon ) $ depends on $ \mu $ only through $ \lambda $, we have
\[
    \mu \frac{\mathrm{d}}{\mathrm{d}\mu} = \mu \frac{\partial}{\partial \mu} + \beta(\lambda) \frac{\partial}{\partial \lambda} = \beta(\lambda) \frac{\partial}{\partial \lambda}.
\]
\begin{enumerate}
    \item From $ \lambda_0 = \mu^{\epsilon} \lambda Z_{\lambda} $, we have 
    \[
        0 = \mu \frac{\mathrm{d} \lambda_0}{\mathrm{d} \mu} = \epsilon \mu^{\epsilon} \lambda Z_{\lambda} + \mu^{\epsilon} \beta(\lambda) Z_{\lambda} + \mu^{\epsilon} \lambda \beta(\lambda) \frac{\partial Z_{\lambda}}{\partial \lambda},
    \]
    hence 
    \[
        0 = \epsilon \lambda + \beta(\lambda) + \beta(\lambda) \lambda \frac{1}{Z_{\lambda}} \frac{\partial Z_{\lambda}}{\partial \lambda} \implies \beta (\lambda) \left(1 + \lambda \frac{\partial \ln Z_{\lambda}}{\partial \lambda}\right) = - \epsilon \lambda.
    \]
    Plug in $ Z_{\lambda} = 1 + \frac{Z_{\lambda , 1}}{\epsilon} + \cdots $ to get 
    \[
        \frac{\partial \ln Z_{\lambda}}{\partial \lambda} \frac{\partial}{\partial \lambda} \left(\frac{Z_{\lambda , 1}}{\epsilon} + \cdots \right) = \frac{1}{\epsilon} \frac{\partial Z_{\lambda , 1}}{\partial \lambda} + \cdots. 
    \]
    Since $ \beta (\lambda) $ is finite as $ \epsilon \to 0 $, we write $ \beta (\lambda) = - \epsilon \lambda + \beta ^{(0)} (\lambda) $. Substitue back to get 
    \[
        \left(- \epsilon \lambda + \beta^{(0)} (\lambda)\right) \left(1 + \frac{\lambda}{\epsilon} \frac{\mathrm{d} Z_{\lambda, 1}}{\mathrm{d} \lambda}\right) = - \epsilon \lambda \implies \beta^{(0)} (\lambda) = \lambda^2 \frac{\mathrm{d} Z_{\lambda , 1}}{\mathrm{d} \lambda}, 
    \]
    since the $ \epsilon $ terms cancel out. Therefore, we have
    \[
        \boxed{\beta (\lambda) = - \epsilon \lambda + \lambda^2 \frac{\mathrm{d} Z_{\lambda , 1}}{\mathrm{d} \lambda}, \quad \frac{\mathrm{d} \beta}{\mathrm{d}\lambda} = - \epsilon + \frac{\mathrm{d}}{\mathrm{d}\lambda} \left(\lambda^2 \frac{\mathrm{d} Z_{\lambda , 1}}{\mathrm{d} \lambda} \right).}
    \]

    \item From $ \gamma (\lambda) = \frac{1}{2} \mu \frac{\mathrm{d} \ln Z_{\phi}}{\mathrm{d} \mu} = \frac{1}{2} \beta({\lambda}) \frac{\partial \ln Z_{\phi}}{\partial \lambda} $. Plug in $ \beta (\lambda) $ from before and $ Z_{\phi} = \frac{1}{\epsilon} \frac{\mathrm{d} Z_{\phi, 1}}{\mathrm{d} \lambda} + \cdots $ to get 
    \[
        \gamma (\lambda) = \frac{1}{2} \left(- \epsilon \lambda + \lambda^2 \frac{\mathrm{d} Z_{\lambda , 1}}{\mathrm{d} \lambda}\right) \left(\frac{1}{\epsilon} \frac{\mathrm{d} Z_{\phi , 1}}{\mathrm{d} \lambda} + \cdots \right).
    \]
    The constant term as $ \epsilon \to 0 $ then gives 
    \[
        \boxed{\gamma (\lambda) = - \frac{1}{2} \lambda \frac{\mathrm{d} Z_{\phi , 1}}{\mathrm{d} \lambda}, \quad \frac{\mathrm{d} \gamma}{\mathrm{d} \lambda} = - \frac{1}{2} \frac{\mathrm{d}}{\mathrm{d} \lambda} \left( \lambda \frac{\mathrm{d} Z_{\phi , 1}}{\mathrm{d} \lambda} \right).}
    \]

    \item From $ m_0^2 = Z_m m^2 $, we have
    \[
        0 = \mu \frac{\mathrm{d} m_0^2}{\mathrm{d} \mu} = \mu \frac{\mathrm{d} Z_m}{\mathrm{d} \mu} m^2 + Z_m \mu \frac{\mathrm{d} m^2}{\mathrm{d} \mu} = \beta (\lambda) \frac{\partial Z_m}{\partial \lambda} m^2 + Z_m m^2 \gamma_m (\lambda).
    \]
    Hence, we have
    \[
        \gamma_m (\lambda) = - \frac{\beta (\lambda)}{Z_m} \frac{\partial Z_m}{\partial \lambda} = - \beta (\lambda) \frac{\partial \ln Z_m}{\partial \lambda}, 
    \]
    and plugging in $ \beta (\lambda) $ and $ Z_m = 1 + \frac{Z_{m,1}}{\epsilon} + \cdots $, we get
    \[
        \gamma_m (\lambda) = - \left(- \epsilon \lambda + \lambda^2 \frac{\mathrm{d} Z_{\lambda , 1}}{\mathrm{d} \lambda}\right) \left(\frac{1}{\epsilon} \frac{\mathrm{d}Z_{m,1}}{\mathrm{d} \lambda} + \cdots \right).
    \]
    The constant term as $ \epsilon \to 0 $ gives
    \[
        \boxed{\gamma _m (\lambda) = \lambda \frac{\mathrm{d}Z_{m,1}}{\mathrm{d} \lambda}, \quad \frac{\mathrm{d} \gamma_m}{\mathrm{d} \lambda} = \frac{\mathrm{d}}{\mathrm{d} \lambda} \left( \lambda \frac{\mathrm{d} Z_{m,1}}{\mathrm{d} \lambda} \right).}
    \]
\end{enumerate}
Since the RG functions are finite as $ \epsilon \to 0 $, the coefficients of $ 1/\epsilon^n $ for $ n \geq 1 $ must vanish in the above expressions. Define $ G_{i} (\lambda , \epsilon) = \ln Z_{i} (\lambda , \epsilon) $, where $ i $ may stand for $ \phi , m , \lambda $, and the coefficients are 
\[
    G_{i} (\lambda , \epsilon ) = \sum_{n=1}^{\infty} G_{i,n} (\lambda) \frac{1}{\epsilon^n}.
\] 
From previous calculation, all the RG functions can be written as
\[
    \left(\beta (\lambda) \frac{\partial}{\partial \lambda} + C \right) \ln Z_i = \left( \beta (\lambda) \frac{\partial}{\partial \lambda} + C \right) G_i < \infty, 
\]
where $ C = \epsilon $ or $ 0 $. Requiring the coefficients of $ 1/\epsilon^n $ for $ n \geq 1 $ to vanish, we have the relations
\[
    \boxed{ \frac{\mathrm{d} G_{i, n+1}}{\mathrm{d}\lambda} = \lambda \frac{\mathrm{d} Z_{\lambda , 1}}{\mathrm{d} \lambda} \frac{\mathrm{d} G_{i, n}}{\mathrm{d} \lambda} \implies \frac{\mathrm{d}Z_{i, n+1}}{\mathrm{d} \lambda} = \lambda \frac{\mathrm{d} Z_{\lambda, 1}}{\mathrm{d} \lambda} \frac{\mathrm{d}Z_{i, n}}{\mathrm{d} \lambda} + Z_{i,n}(\lambda) \frac{\mathrm{d} Z_{i,1}}{\mathrm{d} \lambda}.}
\] 

(b) From Problem 2, the second-order diagram contributes to $ Z_{\phi} $ as
\[
    Z_{\phi} = 1 - \frac{1}{\epsilon} \frac{\lambda^2}{12 (4\pi)^4}= 1 + (\text{number}) \frac{\lambda^2}{\epsilon}. 
\]
From (a), calculate
\[
    \frac{\mathrm{d} Z_{\lambda , 1}}{\mathrm{d} \lambda} = \frac{\mathrm{d}}{\mathrm{d} \lambda} \left[ \frac{3 \lambda}{(4\pi)^2} - \frac{17 \lambda^2}{6 (4\pi)^4} \right] = \frac{3}{(4\pi)^2} - \frac{17 \lambda}{3 (4\pi)^4},
\] 
\[
    \implies \beta (\lambda) = - \epsilon \lambda + \lambda^2 \left[ \frac{3}{(4\pi)^2} - \frac{17 \lambda}{3 (4\pi)^4} \right] = - \epsilon \lambda + \frac{3 \lambda^2}{(4\pi)^2} - \frac{17 \lambda^3}{3 (4\pi)^4}.
\]
Next, for $ \gamma_m (\lambda) $:  
\[
    \frac{\mathrm{d} Z_{m, 1}}{\mathrm{d}\lambda} = \frac{\mathrm{d}}{\mathrm{d} \lambda} \left[ \frac{\lambda}{(4\pi)^2} - \frac{5 \lambda^2}{2 (4\pi)^4} \right] = \frac{1}{(4\pi)^2} - \frac{5 \lambda}{(4\pi)^4},
\]
\[
    \implies \gamma_m (\lambda) = \lambda \left[ \frac{1}{(4\pi)^2} - \frac{5 \lambda}{(4\pi)^4} \right] = \frac{\lambda}{(4\pi)^2} - \frac{5 \lambda^2}{(4\pi)^4}.
\]
Finally, for $ \gamma (\lambda) $:
\[
    \frac{\mathrm{d} Z_{\phi , 1}}{\mathrm{d} \lambda} = \frac{\mathrm{d}}{\mathrm{d} \lambda} \left( - \frac{\lambda^2}{12 (4\pi)^4} \right) = - \frac{\lambda}{6 (4\pi)^4},
\]
\[
    \implies \gamma (\lambda) = - \frac{1}{2} \lambda \left( - \frac{\lambda}{6 (4\pi)^4} \right) = \frac{\lambda^2}{12 (4\pi)^4}.
\]
The Wilson-Fisher fixed point is located at $ \beta (\lambda^{*}) = 0 $, i.e. 
\[
    - \epsilon \lambda^{*} + \frac{3 (\lambda^{*})^2}{(4\pi)^2} - \frac{17 (\lambda^{*})^3}{3 (4\pi)^4} = 0 \implies - \epsilon + 3 u_{*} - \frac{17 u_{*}^2}{3} = 0, 
\]
where $ u_{*} = \frac{\lambda^{*}}{(4\pi)^2} $. Solving perturbatively in $ \epsilon $, let $ u_{*} = u_{\ast} ^{(1)} \epsilon + u_{*} ^{(2)} \epsilon^2 + \cdots $, we have 
\[
    u_{*} \approx \frac{\epsilon}{3} + \frac{17}{81} \epsilon^2 \implies \boxed{\lambda^{*} \approx \frac{(4\pi)^2}{3} \epsilon + \frac{17 (4\pi)^2}{81} \epsilon^2.} 
\] 
The critical exponent $ \nu = \frac{1}{2 - \gamma_m (\lambda^{*})} $. Plug in $ \lambda^{*} $ to get
\[
    \gamma_m (\lambda^{*}) = \frac{\lambda^{*}}{(4\pi)^2} - \frac{5 (\lambda^{*})^2}{(4\pi)^4} = u_{*} - 5 u_{*}^2 \approx \frac{\epsilon}{3} + \frac{17}{81} \epsilon^2 - 5 \left( \frac{\epsilon^2}{9} \right) = \frac{\epsilon}{3} - \frac{28}{81} \epsilon^2.
\]
Therefore, we have, at $ \epsilon = 1 $,  
\[
    \nu = \frac{1}{2 - \gamma_m (\lambda^{*})} \approx \left. \frac{1}{2 - \frac{\epsilon}{3} + \frac{28}{81} \epsilon^2} \right|_{\epsilon = 1} = \frac{1}{2 - \frac{1}{3} + \frac{28}{81}} \approx 0.497
\]
Therefore, the two-loop result is slightly worse than the one-loop result $ \nu = 0.5 $ when compared to the real value $ \nu \sim 0.63 $.
\end{solution}


\newpage

4.~\textbf{Real-space RG} [\textcolor{red}{25 pts}]

In \textcolor{ForestGreen}{[Note-20251104]} we demonstrated the real-space renormalization group (RG) calculation by considering the classical Ising models. Here, we consider a generalized version of the Ising model on a one-dimensional spatial lattice, with the energy of spins:
\begin{equation}
E[\sigma,\tau]
= - J_2 \sum_j (\sigma_j \sigma_{j+1} + \tau_j \tau_{j+1})
  - J_4 \sum_j \sigma_j \sigma_{j+1} \tau_j \tau_{j+1}.
\tag{11}
\end{equation}

In this model, there are two spin degrees of freedom, $\sigma$ and $\tau$, each taking values $\pm 1$. $J_2 > 0$ and $J_4 > 0$ are coupling constants for two-spin and four-spin interactions, respectively. The partition function of the system is
\begin{equation}
\mathcal{Z}
= \sum_{\{\sigma,\tau\}} e^{-E[\sigma,\tau]/T}
= \sum_{\{\sigma,\tau\}} e^{-S[\sigma,\tau]},
\tag{12}
\end{equation}
where the action $S[\sigma,\tau]$ has the same form as $E[\sigma,\tau]$ except that the couplings in $S[\sigma,\tau]$ are
\[
K_2 := J_2/T, \qquad K_4 := J_4/T.
\]

(a) Derive an exact RG transformation for the couplings $K_2$ and $K_4$.  
This can be done in the following steps:

\begin{itemize}
\item[(a-i)] Integrate out the degrees of freedom $\sigma_{2j+1}$ and $\tau_{2j+1}$ on the odd sites.

\item[(a-ii)] Rescale the system by defining
\[
\sigma'_j := \sigma_{2j}, \qquad \tau'_j := \tau_{2j}.
\]

\item[(a-iii)] Obtain the effective action of the new spin variables
$\sigma'_j$ and $\tau'_j$ with the coarse-grained couplings $K_2'$ and $K_4'$.

\item[(a-iv)] The relations between $\{K_2',K_4'\}$ and $\{K_2,K_4\}$ give the RG transformation and thus the beta functions.
\end{itemize}

\textbf{Identify all the fixed points associated with this RG
transformation. Draw the RG flows explicitly in the $K_2$–$K_4$ plane and
discuss the stability of each fixed point. Use the flows to derive the phase
diagram for this model, and briefly explain the physical meaning of each
phase (better in terms of the spin order parameters).}

\bigskip

(b) Extend the discussions in part (a) (\textbf{statements in bold}) to the
same model on a two-dimensional square lattice:
\begin{equation}
E[\sigma,\tau]
= - J_2 \sum_{\langle r,r'\rangle} (\sigma_r \sigma_{r'} + \tau_r \tau_{r'})
  - J_4 \sum_{\langle r,r'\rangle} \sigma_r \sigma_{r'} \tau_r \tau_{r'}.
\tag{13}
\end{equation}

Here, the sites are labeled by two-component lattice vectors $r$, and
$\langle r,r' \rangle$ denotes nearest-neighboring sites. In the 2d case, you may apply an approximation method, such as the \emph{Migdal–Kadanoff
approximation} (bond-moving procedure), to derive an RG transformation for
$K_2$ and $K_4$, similar to the derivation for the 2d Ising model in
\textcolor{ForestGreen}{[Note-20251104]}.

\begin{solution}
(a-i) Consider a block of three sites $ (2j, 2j+1, 2j+2) $, where we write the boundary spins as $ \sigma_L, \tau_L $ for $ 2j $ and $ \sigma_R, \tau_R $ for $ 2j+2 $. The contribution to the partition function from this block is
\[
    \overline{Z} = \sum_{\sigma , \tau = \pm 1} \exp \left\{ K_2 \sigma \left( \sigma_L + \sigma_R \right) + K_2 \tau \left(\tau_L + \tau_R\right) + K_4 \sigma \tau \left( \sigma_L \tau_L + \sigma_R \tau_R \right) \right\}.
\]
By symmetry considerations, there are only three cases for the boundary spins corresponding to distinct $ \overline{Z} $ values. 
\begin{enumerate}
    \item $ \sigma_L \sigma_R = \tau_L \tau_R = 1 $: $ Z_1 = e^{4 K_2 + 2 K_4} + e^{-2 K_4} + e^{-2 K_4} + e^{-4 K_2 + 2K_2} = 2 e^{2 K_4} \cosh (4 K_2) + 2 e^{-2 K_4} $.  
    \item $ \sigma_L \sigma_R \neq \tau_L \tau_R $, $ \sigma_L \sigma_R \tau_L \tau_R = -1 $: $ Z_2 = 2 \left(e^{2 K_2} + e^{-2 K_2}\right) = 4 \cosh (2 K_2) $. 
    \item $ \sigma_L \sigma_R = \tau_L \tau_R = -1 $: $ Z_3 = e^{2 K_4} + e^{-2 K_4} + e^{-2 K_4} + e^{2 K_4} = 4 \cosh (2 K_4) $.
\end{enumerate}

(a-ii) Define a coarse-grained system with spins $ \sigma'_j = \sigma_{2j} $, $ \tau'_j = \tau_{2j} $. 

(a-iii) Let $ \exp \left(-S_{\text{eff}}^{\prime}\right) = \exp \left[K_0^{\prime} + K_2^{\prime}  \left(\sigma_L \sigma_R + \tau_L \tau_R \right) + K_4^{\prime} \sigma_L \sigma_R \tau_L \tau_R \right] $ be the renormalized action, where apart from the renormalized $ K_2^{\prime} $, $ K_4^{\prime} $, we have also added a constant term $ K_0^{\prime} $. 

(a-iv) We will map the configurations to the new couplings as discussed in (a-i): 
\[
    \exp \left(-S_{\text{eff}}^{\prime}\right) = 
    \begin{cases}
        e^{K_0^{\prime} + 2 K_2^{\prime} + K_4^{\prime}} = Z_1, &\quad \sigma_L \sigma_R = \tau_L \tau_R = 1,\\
        e^{K_0^{\prime} - K_4^{\prime}} = Z_2, &\quad  \sigma_L \sigma_R \neq \tau_L \tau_R, \;\; \sigma_L \sigma_R \tau_L \tau_R = -1 \\
        e^{K_0^{\prime} - 2 K_2^{\prime} + K_4^{\prime}} = Z_3, &\quad \sigma_L \sigma_R = \tau_L \tau_R = -1.
    \end{cases}
\]
Solving these equations, we have
\[
    K_0^{\prime} = \frac{1}{4} \ln \left( Z_1 Z_2^2 Z_3 \right), \quad K_2^{\prime} = \frac{1}{4} \ln \left( \frac{Z_1}{Z_3} \right), \quad K_4^{\prime} = \frac{1}{4} \ln \left( \frac{Z_1 Z_3}{Z_2^2} \right).
\]
Plugging in $ Z_i $, we have the exact RG transformation
\[
    K_2^{\prime} = \frac{1}{4} \ln \left( \frac{e^{2 K_4} \cosh (4 K_2) + e^{-2 K_4}}{2 \cosh (2 K_4)} \right), \quad K_4^{\prime} = \frac{1}{4} \ln \left( \frac{\left( e^{2 K_4} \cosh (4 K_2) + e^{-2 K_4} \right) \cosh (2 K_4)}{2 \cosh^2 (2 K_2)} \right).
\]
The fixed points are given by solving $ K_2^{\prime} = K_2 $, $ K_4^{\prime} = K_4 $. There are two trivial fixed points and no non-trivial fixed points:
\begin{enumerate}
    \item $ (K_2 , K_4) = (0,0) $: Here $ Z_1 = Z_2 = Z_3 = 4 $, so $ K_2^{\prime} = K_4^{\prime} = 0 $. Stable fixed point. 
    \item $ (K_2 , K_4) = (\infty , \infty) $: Unstable fixed point.
\end{enumerate}

The beta functions are 
\[
    \beta_2 = \frac{K_2^{\prime} - K_2}{\ln 2} = \frac{1}{4 \ln 2} \ln \left( \frac{e^{2 K_4} \cosh (4 K_2) + e^{-2 K_4}}{2 \cosh (2 K_4)} \right) - \frac{K_2}{\ln 2},
\]
\[
    \beta_4 = \frac{K_4^{\prime} - K_4}{\ln 2} = \frac{1}{4 \ln 2} \ln \left( \frac{\left( e^{2 K_4} \cosh (4 K_2) + e^{-2 K_4} \right) \cosh (2 K_4)}{2 \cosh^2 (2 K_2)} \right) - \frac{K_4}{\ln 2}.
\]

The RG flows and nullclines of the beta functions are plotted in the $ K_2 $-$ K_4 $ plane in Figure~\ref{fig:1D}. For initial $ K_2, K_4 > 0 $, we have $ K_2^{\prime} < K_2 $ and $ K_4^{\prime} < K_4 $, and hence $ (0,0) $ is stable while $ (\infty , \infty) $ is unstable. The phase diagram consists of a single \emph{disordered paramagnetic phase}, and no finite-temperature phase transition occurs.
\\ \\

\begin{figure}
    \centering
    \includegraphics[width=0.8\textwidth]{1D.png}
    \caption{RG flows and nullclines for beta functions $ \beta_2 $ and $ \beta_4 $ in the $ K_2 $-$ K_4 $ plane (1D model).}
    \label{fig:1D}
\end{figure}

\begin{figure}
    \centering
    \includegraphics[width=0.8\textwidth]{2D.png}
    \caption{RG flows and nullclines for beta functions $ \beta_2 $ and $ \beta_4 $ in the $ K_2 $-$ K_4 $ plane (2D model).}
    \label{fig:2D}
\end{figure}

(b) The 2D case is not exactly solvable, and in this case we can apply the Migdal-Kadanoff approximation by replacing $ K $ with $ 2K $ in the 1D model, following \textcolor{ForestGreen}{[Note-20251104]}. Therefore, the RG transformation becomes
\[
    K_2^{\prime} = \frac{1}{4} \ln \left( \frac{e^{4 K_4} \cosh (8 K_2) + e^{-4 K_4}}{2 \cosh (4 K_4)} \right), \quad K_4^{\prime} = \frac{1}{4} \ln \left( \frac{\left( e^{4 K_4} \cosh (8 K_2) + e^{-4 K_4} \right) \cosh (4 K_4)}{2 \cosh^2 (4 K_2)} \right).
\]
We still have the trivial fixed points $ (K_2, K_4) = (0,0) $ and $ (\infty , \infty) $. However, there are now a non-trivial fixed points. 
\begin{itemize}
    \item Trivial fixed points: $ (0,0) $ is a paramagnetic sink, $ (\infty, \infty) $ is a ferromagnetic sink.
    \item Ising critical points: Let $ K_4 = 0 $, the system decouples into two Ising models. We have 
    \[
        K_2^{\prime} = \frac{1}{4} \ln \frac{1}{2} \left(\cosh (8 K_2) + 1\right) = \frac{1}{4} \ln \left(\cosh^2 (4 K_2 )\right) = \frac{1}{2} \ln \cosh (4 K_2). 
    \]  
    Then $ K_2^* = \frac{1}{2} \ln \cosh (4 K_2^*) $, and WolframAlpha returns $ K_2^* \approx 0.305 $. Setting $ K_2 = 0 $ returns a symmetric solution, so the fixed points are $ (0, K_c) \approx (0, 0.305) $ and $ (0, K_c) \approx (0, 0.305) $.
    \item Potts-like critical point: Let $ K_2 = K_4 = K_d $, we have $ K_2^{\prime} = K_4^{\prime} $. Solving for 
    \[
        K_d^{\prime} = \frac{1}{4} \ln \left( \frac{e^{4 K_d} \cosh (8 K_d) + e^{-4 K_d}}{2 \cosh (4 K_d)} \right) = K_d,
    \]
    WolframAlpha returns $ K_d \approx 0.189 $. So the fixed point is $ (K_d, K_d) \approx (0.189, 0.189) $. 
\end{itemize}

The beta functions are 
\[
    \beta_2 = \frac{K_2^{\prime} - K_2}{\ln 2} = \frac{1}{4 \ln 2} \ln \left( \frac{e^{4 K_4} \cosh (8 K_2) + e^{-4 K_4}}{2 \cosh (4 K_4)} \right) - \frac{K_2}{\ln 2},
\]
\[
    \beta_4 = \frac{K_4^{\prime} - K_4}{\ln 2} = \frac{1}{4 \ln 2} \ln \left( \frac{\left( e^{4 K_4} \cosh (8 K_2) + e^{-4 K_4} \right) \cosh (4 K_4)}{2 \cosh^2 (4 K_2)} \right) - \frac{K_4}{\ln 2}.
\]
The RG flows and nullclines of the beta functions are plotted in the $ K_2 $-$ K_4 $ plane in Figure~\ref{fig:2D}. The phase diagram for $ K_2, K_4 > 0 $ consists of three phases separated by critical lines that meet at the Potts-like critical point $ (K_d, K_d) $, as shown in Figure~\ref{fig:PhaseDiagram}.

\textbf{Phase diagram analysis} :
\begin{itemize}
    \item \textit{Paramagnetic phase (I)}: This is a disordered phase (thermal disorder), and all points flow into $(0,0)$. 
    \item \textit{Baxter phase (II)}: Strong $ K_2 $, and all points flow towards $(0, \infty) $.
    \item \textit{Ferromagnetic phase (III)}: Strong $ K_4 $ corresponds to an Ising-like phase, and all points flow towards $(\infty, 0) $.
\end{itemize}

\textbf{Spin-order parameter:}
\begin{enumerate}
    \item \textit{Ferromagnetic phase} ($K_2 > K_c, K_2 > K_4$): two coupled Ising models $\Rightarrow \langle \sigma \rangle \neq 0$ and $\langle \tau \rangle \neq 0$.
    \item \textit{Baxter phase} ($K_4 > K_c, K_4 > K_2$): minimize energy by fixing $\sigma\tau \Rightarrow \langle \sigma \rangle = 0$ and $\langle \tau \rangle = 0$ but $\langle \sigma\tau \rangle \neq 0$.
    \item \textit{Antiferromagnetic phase} ($ K_2, K_4 < 0 $): For $ K_2 < 0 $, the points flow into $(-\infty, 0) $, corresponding to the N\'{e}el phase. For $ K_4 < 0 $, the points flow into $(0, -\infty) $, and $ \sigma \tau $ alternates between sites. 
\end{enumerate}

\begin{figure}[htbp]
    \centering
    \includegraphics[width=0.75\textwidth]{phase_2D.png}
    \caption{Phase diagram for the 2D model in the $ K_2 $-$ K_4 $ plane. The relevant regions I, II, III correspond to the paramagnetic, Baxter, and ferromagnetic phases, respectively, as discussed above.}
    \label{fig:PhaseDiagram}
\end{figure}

\end{solution}

\newpage

5.~\textbf{Scalar QCD} [\textcolor{red}{20 pts}]

For a complex scalar $\phi_I$ in a complex representation of a non-Abelian
gauge group, we define the covariant derivative as
\begin{equation}
    (D_\mu \phi)_I = \partial_\mu \phi_I - ig A_\mu^a (T^a)_I^{\phantom{I} J} \phi_J,
\tag{14}
\end{equation}
where $I, J$ are the group indices. We use the upper and lower convention for
complex representation introduced in the class.

(a) Use the fact that $\phi^{\dagger I}$ lives in the conjugate representation, construct a “minimally-coupled” Lagrangian of $\phi$ that is renormalizable, invariant under the gauge group, and only contains quadratic terms in the scalar field.

(b) The relevant Feynman vertices of the action are $iV(\phi_J, \phi^{\dagger I}, A_\mu^a)$ and $iV(\phi_J, \phi^{\dagger I}, A_\mu^a, A_\nu^b)$. Derive the Feynman rules for both of them.

(c) Calculate the on-shell amplitude $A(\phi_J, \phi^{\dagger I}, A_\mu^a)$. Express your result in terms of the momenta $p_i{}_\mu$ and polarization of the gauge boson, $\varepsilon_\mu$.

(d) Verify that the amplitude $A(\phi_J, \phi^{\dagger I}, A^a)$ satisfies the Ward identity, i.e.\ the longitudinal mode of the gauge field decouples.

The Ward identity of three-particle amplitude is simply to see.
If you have time, try to calculate the four-particle amplitude
$A(\phi_J, \phi^{\dagger I}, A_\mu^a, A_\nu^b)$ and verify the Ward identity.

\begin{solution}
~ 

(a) The conjugate field $ \phi^{\dagger I} $ lives in the conjugate representation, so it has covariant derivative
\[
    (D_\mu \phi)^{\dagger I} = \partial_\mu \phi^{\dagger I} + ig A_\mu^a \phi^{\dagger J} (T^a)_J^{\phantom{J}I}.
\]
The minimally-coupled, renormalizable Lagrangian that is invariant under the gauge group and only contains quadratic terms in $ \phi $ is
\[
    \mathcal{L} = (D_\mu \phi)^{\dagger I} (D^\mu \phi)_I - m^2 \phi^{\dagger I} \phi_I.
\]

(b) Expand the covariant derivatives, we have
\[
    \begin{split}
        \mathcal{L} &= \partial_\mu \phi^{\dagger I} \partial^\mu \phi_I + ig A_\mu^a \phi^{\dagger I} (T^a)_I^{\phantom{I}J} \partial^\mu \phi_J - ig A_\mu^a \partial^\mu \phi^{\dagger I} (T^a)_I^{\phantom{I}J} \phi_J \\
        &\quad + g^2 A_\mu^a A^{b \mu} \phi^{\dagger K} (T^a)_K^{\phantom{K}I} (T^b)_I^{\phantom{I}J} \phi_J - m^2 \phi^{\dagger I} \phi_I.
    \end{split}
\]
Identify the cubic and quartic interaction terms as
\[
    \mathcal{L}^{(3)} = ig A_\mu^a \phi^{\dagger I} (T^a)_I^{\phantom{I}J} \partial^\mu \phi_J - ig A_\mu^a \partial^\mu \phi^{\dagger I} (T^a)_I^{\phantom{I}J} \phi_J,
\]
and
\[
    \mathcal{L}^{(4)} = g^2 A_\mu^a A^{b \mu} \phi^{\dagger K} (T^a)_K^{\phantom{K}I} (T^b)_I^{\phantom{I}J} \phi_J.
\]
For the cubic vertex $ i V(\phi_J, \phi^{\dagger I}, A_\mu^a) $, suppose $ \phi_J $ carries momentum $ p $ and $ \phi^{\dagger I} $ carries momentum $ p^{\prime} $. The derivative $ \partial_\mu $ acting on incoming $ \phi_J $ gives a factor of $ -i p_\mu $, while $ \partial_\mu $ acting on $ \phi^{I \dagger} $ gives a factor of $ i p^{\prime}_\mu $. The Feynman rule for this vertex is read from $ \mathcal{L}^{(3)} $ as 
\[
    i V(\phi_J, \phi^{\dagger I}, A_\mu^a) = i \cdot ig \cdot (T^a)_J^{\phantom{J}I} ( - ip -  ip^{\prime})_\mu = i g (T^a)_J^{\phantom{J}I} (p + p^{\prime})_\mu.
\]
Similarly, for the quartic vertex $ i V(\phi_J, \phi^{\dagger I}, A_\mu^a, A_\nu^b) $, notice that the two gluon are identical, so we sum over the permutations. The Feynman rule for this vertex is read from $ \mathcal{L}^{(4)} $ as
\[
    i V_{\mu \nu } (\phi_J, \phi^{\dagger I}, A_\mu^a, A_\nu^b) = ig^2 \{ T^a, T^b \}_J^{\phantom{J}I} g_{\mu \nu }, 
\]
where $ \{ T^a, \, T^b \}_J^{\phantom{J}I} = (T^a)_J^{\phantom{J}K} (T^b)_K^{\phantom{K}I} + (T^b)_J^{\phantom{J}K} (T^a)_K^{\phantom{K}I} $ is the anticommutator.
\begin{remark}
    In the special case of an abelian theory, take $ T^a \to Q $, and we recover $ i V_{\mu \nu } = 2ig^2 Q^2 g_{\mu \nu } $. 
\end{remark}

(c) Consider the amplitude with all vertices $ \phi_J (p_1) $, $ \phi^{\dagger I} (p_2) $, and $ A_\mu^a (p_3, \varepsilon_\mu) $ outgoing, where $ \varepsilon_\mu $ is the polarization vector of the gauge boson. Momemtum conservation gives $ p_1 + p_2 + p_3 = 0 $, while on-shell condition gives $ p_1^2 = p_2^2 = m^2 $, $ p_3^2 = 0 $. The tree-level diagram is a cubic vertex, so the vertex rule in (b) gives 
\[
    \begin{split}
        A(\phi_J, \phi^{\dagger I}, A_\mu^a) &= \varepsilon^\mu \cdot i V(\phi_J, \phi^{\dagger I}, A_\mu^a) = ig (T^a)_J^{\phantom{J}I} (p_1 - p_2)_\mu \varepsilon^\mu (p_3). 
    \end{split}
\]

(d) To check the Ward identity, take a longitudinal polarization $ \varepsilon_\mu (p_3) = p_{3 \mu } $. Then, we have
\[
    \begin{split}
        A(\phi_J, \phi^{\dagger I}, A^a) &= ig (T^a)_J^{\phantom{J}I} (p_1 - p_2)_\mu \, p_3^\mu = ig (T^a)_J^{\phantom{J}I} (p_1 - p_2) \cdot (- p_1 - p_2) \\
        &= - ig (T^a)_J^{\phantom{J}I} (p_1^2 - p_2^2) \propto (m^2 - m^2) = 0,
    \end{split}
\]
where we used $ p_1 \cdot p_2 = p_2 \cdot p_1 $ and the on-shell condition. Hence, we verify the Ward identity:
\[
    A \left. (\phi_J, \phi^{\dagger I}, A^a) \right|_{\varepsilon_\mu = p_{3 \mu}} = 0
\]

\end{solution}

\end{CJK}
\end{document}