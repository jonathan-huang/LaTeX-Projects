\documentclass[%
 sor,
%aip,
%twoside,
%groupedaddress,
%jmp,
 jor,
 amsmath,amssymb,
%preprint,%
 reprint,%
%author-year,%
%author-numerical,%
]{revtex4-2}

\usepackage{graphicx}% Include figure files
\usepackage{dcolumn}% Align table columns on decimal point
\usepackage{bm}% bold math
%\usepackage[mathlines]{lineno}% Enable numbering of text and display math
%\linenumbers\relax % Commence numbering lines

% \renewcommand{\emph}{\textsc}   % emphasize with small caps
% \renewcommand{\emph}{\textbf}
% \renewcommand{\emph}{\textit{\textbf}} 

%% Math and Symbols %%
\usepackage{amsmath,amsthm,amssymb, mathtools}
\usepackage{yhmath, faktor, dsfont}
\usepackage{academicons, wasysym, marvosym}
\usepackage[scr]{rsfso} 
\usepackage{latexsym, amsmath, amscd, amsmath, amsthm}
\usepackage{amssymb,amsmath,amsthm,graphicx,dsfont}
\usepackage{hyperref}

\usepackage{booktabs}
\usepackage{multirow}

%% Enhancement %%
\usepackage{graphicx, tabularx}
\usepackage[shortlabels,inline]{enumitem}
%% TikZ %%
\usepackage{tikz-cd}
\usepackage[breakable]{tcolorbox}
\usetikzlibrary{decorations.pathmorphing}
\usetikzlibrary{calc, arrows,matrix}

%% Other packages %%
\usepackage{amsopn}

%% Traditional Chinese %%
\usepackage{CJKutf8}

%% Math environments %%
\newtheoremstyle{mystyle}
  {6pt}{6pt}% 上下間距
  {}%          內文字體
  {}%              縮排
  {\bf}%       標頭字體
  {.}%       標頭後標點
  {1em}% 內文與標頭距離
  {}% Theorem head spec (can be left empty, meaning 'normal')
\theoremstyle{mystyle}	
\newtheorem{theorem}{Theorem}
\newtheorem{definition}{Definition}
\newtheorem{example}[theorem]{Example}
\newtheorem{exercise}{Exercise}
\newtheorem{solution}{Solution}
\newtheorem{corollary}[theorem]{Corollary}
\newtheorem{property}[theorem]{Property}
\newtheorem{proposition}[theorem]{Proposition}
\newtheorem{lemma}{Lemma}
\newtheorem{problem}[theorem]{Problem}
\newtheorem{answer}{Answer}[section]
\newtheorem{fact}[theorem]{fact}
\newtheorem*{claim}{Claim}
\newtheorem*{observation}{Observation}

\theoremstyle{remark}
\newtheorem*{remark}{Remark}

\newcommand{\bvec}[1]{\mathbf{#1}} % vector


\begin{document}

\preprint{AIP/123-QED}

\title[Term Paper]{Application of Field Theory to Percolation Processes}% Force line breaks with \\
\thanks{Contact: huang20041014@gmail.com}

\author{B12202004 Shao-Kai Jonathan Huang}
\affiliation{Department of Physics, National Taiwan University, Taipei 10617 \\
}%
% \altaffiliation[Also at ]{}%Lines break automatically or can be forced with \\

\date{\today}% It is always \today, today,
             %  but any date may be explicitly specified

\begin{abstract}
The bulk of the term paper will be based on the exposition \textit{The field theory approach to percolation} (\cite{JANSSEN2005147}), where both dynamical isotropic percolation (dIP) and directed percolation (DP) are discussed. Throughout the paper, I provide an overview of the field theory approach to percolation processes and apply it to \emph{directed percolation processes}, and leave out dIP for the sake of brevity. The paper aims to elucidate the concept of universality classes, classify some commonly encountered classes, and provide a brief overview of how field theory techniques can be employed to analyze critical phenomena. I will motivate important concepts by incorporating examples from pure mathematics, for example some well-known results from probability theory and graph theory. Finally, I will cite and organize important loop calculations from Janssen's relevant paper \textit{Directed Percolation with Colors and Flavors} (\cite{JANSSEN2005147}) in the appendix.
\end{abstract}

\keywords{Field theory, percolation, renormalization, phase transition, universality classes}%Use showkeys class option if keyword
                              %display desired
\maketitle

\newpage

\section{\label{sec:introduction}Introduction}

Percolation processes are fundamental models in statistical physics that describe the behavior of connected clusters in a random medium. Since
it provides an intuitively model of the irregular geometry emerging in disordered systems, percolation has provided a new paradigm for random structures. Related models were first introduced by Broadbent and Hammersley in 1957 to model fluid flow through porous media. Percolation processes are one of the simplest systems exhibiting non-equilibrium continuous phase transitions, and due to their display of universal behavior near the critical points, they can be studied with renormalization group methods.

To motivate this study, we will first consider a simple example of percolation called \emph{bond percolation} to provide an intuitive picture. In bond percolation, we designate each edge $ E $ of some graph $ G(V,E) $ to be either open or connected randomly, according to independently identically distributed random variables. In site percolation, we do the same for vertices. They may be classified as either \emph{directed percolation (DP)}, where connections between sites have a preferred irreversible direction (which we may identify as the "time direction"). This is known as \emph{isotropic percolation (IP)}.

\begin{figure}[htbp]
    \centering
    \includegraphics[width=0.6\linewidth]{Images/bond_site.png}
    \caption{Site versus bond percolation. Percolation is the emergence of an infinite connected component (a cluster) containing 0}
    \label{fig:bond_site}
\end{figure}

\begin{example}[Bernoulli percolation]
    To motivate our study, consider the model of \emph{Bernoulli percolation} from probability theory. Let's define an infinite $ d $-dimensional lattice $ \mathbb{Z}^d $, where each bond (edge) between neighboring vertices is open with probability $p$ and closed with probability $1-p$, and hence are i.i.d. Beroulli random variables. Naturally, we consider the probability space $ (\{0,1\}^{\mathbb{E}}, \mathfrak{F}, \mathbb{P}_p) $. We will omit the details here, but the reader may refer to \cite{DuminilCopin2018IntroductionTB}. Call a connected component of the set of edges a \emph{cluster}. We are interested in the behavior of cluster sizes as $ p $ varies, in particular, whether an infinite cluster may emerge. Intuitively, when $ p=0 $, all clusters are finite; when $ p=1 $, almost all edges are connected, and an infinite cluster exists. Define distance by the taxicab measure $ \vert \cdot \vert_1 $, $ C(x) $ a cluster containing $ x $, and
    \begin{equation}
        \{ C(x) \text{ is an infinite cluster}\} = \bigcap_{n=1}^{\infty} \{ C(x) \text{ contains a vertex } x^{\prime} \text{ such that } |x - x^{\prime} |_1 = n \}.
    \end{equation}
    Define the percolation probability $ p_c $ to be 
    \begin{equation}
        p_c = \inf \{ p \in [0,1] : \mathbb{P}_p ( C(0) \text{ is an infinite cluster} ) > 0 \}.
    \end{equation}
    An important result then says that $ 0 < p_c < 1 $ for all $ d \geq 2 $, and thus there exists a non-trivial critical threshold for the emergence of an infinite cluster, which is an example of a phase transition. This can be established using simple arguments, as presented in \cite{DuminilCopin2018IntroductionTB}.
\end{example}

\begin{theorem}
    For Bernoulli bond percolation on the lattice $ \mathbb{Z}^d $ with $ d \geq 2 $, there exists a critical probability $ p_\text{c} \in (0,1) $ such that $ \theta (p) = 0 $ for $ p < p_\text{c} $ and $ \theta (p) $ is bounded increasing for $ o > p_\text{c} $. 
\end{theorem}

Here, $ \theta (p) = \mathbb{P}_p ( C(0) \text{ is an infinite cluster} ) $ is the percolation threshold. The theorem indicates that bond percolation on $ \mathbb{Z}^d $ undergoes phase transition at "finite temperature" $ p_\text{c} $.

\begin{remark}
    Whether $ \theta (p) $ is continuous at $ p=p_\text{c} $, and thus this constitutes a \emph{continuous phase transition}, is only affirmative in dimensions $ d \leq 2 $ and $ d \geq 19 $. The fact that $ \theta (p) $ is continuous for all dimensions is still an open problem. The phase diagram is illustrated in Figure~\ref{fig:percolation_phase_diagram}.
\end{remark}

\begin{figure}[htbp]
    \centering
    \includegraphics[width=0.5\linewidth]{Images/phase_diagram.png}
    \caption{Phase diagram of bond percolation on $ \mathbb{Z}^d $. There exists a critical threshold $ p_c $ such that for $ p < p_c $, there is almost surely no infinite cluster, while for $ p > p_c $, there exists almost surely a unique infinite cluster.}
    \label{fig:percolation_phase_diagram}
\end{figure}

As in the above example, many percolating systems exhibit \emph{non-equilibrium phase transition}, where there exists a critical threshold $ p_c $ whereby the probability for the existence of an infinite connected accessible cluster becomes non-zero, and an agent may percolate through the entire system. This percolation transition defines a genuine critical phenomenon as encountered in statistical mechanics, and thus define genuine \emph{universality classes}. Although it is known that finite-temperature phase temperatures always occur, it is in general difficult to determine the exact numbers. For square bond percolation, it is intuitively clear that $ p_c = \frac{1}{2} $, but was not proven until 1980, by Harris (1960, $ p \geq \frac{1}{2} $) and Kesten (1980, $ p \leq \frac{1}{2} $). For a list of proven and conjectured percolation thresholds, see Table \ref{tab:percolation_thresholds}.

\begin{table}[htbp]
    \centering
    \caption{Percolation thresholds $p_c$ for selected lattices (site and bond). An asterisk ($^{\ast}$) indicates an exact value, while the rest are agreed numerical values by various numerical computations in the literature.}
    \label{tab:percolation_thresholds}
    \begin{tabular}{lcc}
        \toprule
        \textbf{Lattice} & \textbf{$p_c$ (site percolation)} & \textbf{$p_c$ (bond percolation)} \\
        \midrule
        Cubic (body-centered) & 0.246   & 0.1803 \\
        Cubic (face-centered) & 0.198   & 0.119  \\
        Cubic (simple)        & 0.3116  & 0.2488 \\
        Diamond               & 0.43    & 0.388  \\
        Honeycomb             & 0.6962  & 0.65271$^{\ast}$ \\
        Square                & 0.592746& 0.50000$^{\ast}$ \\
        Triangular            & 0.50000$^{\ast}$ & 0.34729$^{\ast}$ \\
        \bottomrule
    \end{tabular}
\end{table}

\begin{theorem}[Site percolation is more general than bond percolation]
    Bond percolation on a graph $G(V,E)$ is equivalent to site percolation on its line graph $L(G)$, whose construction we show in Figure~\ref{fig:line_graph}. However, given a graph $\bar{G}(\bar{V}, \bar{E})$, there does not necessarily exists a graph $G$ such that $L(G)=\bar{G}(\bar{V}, \bar{E})$. In particular, site percolation also exhibit finite-temperature phase transitions.
\end{theorem}

\begin{figure}[htbp]
    \centering
    \includegraphics[width=0.95\linewidth]{Images/line_graph.png}
    \caption{Construction of the line graph $ L(G) $ from a graph $ G $. Each edge in $ G $ corresponds to a vertex in $ L(G) $, and two vertices in $ L(G) $ are connected if their corresponding edges in $ G $ share a common vertex.}
    \label{fig:line_graph}
\end{figure}

\subsection{Directed Percolation}

To motivate a field theory (FT) approach, we will frame the percolation problem as an epidemic process, as it is well-known that population models in mathematical biology are well-described by FT \cite{Peliti1985PathIA}. Let's illustrate this with a hypothetical forest: Trees may catch blight disease and allow the agent, blight beetles, to spread and infect other trees. If they are able to recover, then this process defines a \emph{simple epidemic process (SEP)}. On the other hand, trees may catch fire and spread it to other trees, and after burning, they become burnt trunks which cannot catch fire again. This is the \emph{general epidemic process (GEP)}, and its universal properties are governed by the dIP universality class, a topic which will not be pursued further. 

\begin{figure}[htbp]
    \centering
    \includegraphics[width=0.6\linewidth]{Images/forest_fire.png}
    \caption{In a model of a forest, the healthy trees are represented by green, the burning trees are depicted as red, while the burnt trunks (which cannot catch fire again) are shown in black. This illustrates a directed percolation (DP) process, where the fire spreads from burning trees to adjacent healthy ones.}
    \label{fig:bark_beetle}
\end{figure}

\section{\label{sec:field_theory}Field Theory for Percolation Processes}
We now list four principles that allow for the explicit construction of a mesoscopic stochastic theory of percolation processes, where we have to a-priori define the slow dynamic fields in the theory, and assume that on a mesoscopic scale, the fast microscopic variables are absorbed into the stochastic equations of motion for the relevant slow variables.

A \emph{universality class} is defined as a collection of models that share the same set of critical exponents and scaling functions near the critical point. More concretely, let's observe that some classes of systems exhibit
\emph{universality}, in that order parameters are less sensitive to details of the system near their critical value, and obey a power-law scaling given by 
\begin{equation}
    a = a_0 \vert \beta - \beta_0 \vert^{\alpha}.  
\end{equation}

\begin{definition}[universality class]
    A universality class is an equivalence class of physical models $[A]$, with relation $A \sim B$ if $A, B$ have the same critical exponents.
\end{definition}

DP has been shown to define a universal class, which we will denote $ \mathbf{DP} $. For 2 dimensional percolation, which is below the critical dimension of DP, the critical exponents can be computed exactly as 
\begin{equation}
    \alpha = -\frac{2}{3}, \quad \beta = \frac{5}{36}, \quad \gamma = \frac{43}{18}, \quad \delta = \frac{91}{5}, \quad \nu = \frac{4}{3}, \eta = \frac{5}{24}. 
\end{equation}

Building on the observation that DP defines a universal class, we further note for reference that the SEP mentioned earlier as an ecological model is an element of $ \mathbf{DP} $, and for instance the liquid-gas phase transition belogns to the universality class of $ \mathbf{3D Ising Model} $. For DP universal class, the author propose the following principles for constructing a mesoscopic stochastic theory of percolation processes, based on birth-death processes with diffusion in mathematical biology, such as \cite{Peliti1985PathIA}:
\begin{itemize}
    \item[(i)] The susceptible medium becomes locally infected, depending on the density $n$ of neighboring sick individuals. The infected regions recover after a brief time interval.
    \item[(ii)] The state with $n \equiv 0$ is absorbing. This state is equivalent to the extinction of the disease.
    \item[(iii)] The disease spreads out diffusively via the short-range infection (i) of neighboring susceptible regions.
    \item[(iv)] Microscopic fast degrees of freedom are represented as stochastic forces respecting (ii).
\end{itemize}

The assertion that the above principles lead directly to $ \mathbf{DP} $ is known in the literature as the \emph{DP conjecture}. 

he stochastic variables V(tk) are uncorrelated at
different times, and that the statistical properties of V(tk) depend solely upon
n(tk) and m(tk). As a consequence, the generating function of the cumulants
of V (i.e., the Laplace transform of the corresponding probability distribution)
must have the general form

\[
\overline{\exp\!\left( \Delta \sum_k \mathcal{V}(t_k)\,\tilde n(t_{k+1}) \right)}
= \exp\!\left( 
\Delta \sum_k \sum_{l_k=1}^{\infty} \frac{\tilde n(t_{k+1})^{\,l_k}}{l_k!}\,
K_{l_k}\!\left[n(t_k),m(t_k)\right]
\right),
\tag{3}
\]

where the overbar denotes the statistical average over the fast microscopic
degrees of freedom. The $\tilde n(t_k)$ constitute independent new variables. Of course, in the DP process the cumulants $K_l$ are independent of the debris $m$.

Similarly to building a path integral formulation in QFT, the statistical properties of the stochastic theory are fully encoded in the probability density of the history $ \mathcal{H} = \{ n(t_0), n(t_1), \dots , n(t_k), \dots \} $ of the density field $ n(t) $, and we may write 
\begin{equation}
    \begin{split}
        \mathcal{P}(\{n(t)\}) &= \prod_k \delta\!\left[n(t_{k+1}) - n(t_k) - \Delta \mathcal{V}(t_k)\right] \\
        &= = \int \prod_k \frac{d\tilde n(t_{k+1})}{2\pi i}\, \exp\!\left\{\sum_k \tilde n(t_{k+1})\left[ \Delta \mathcal{V}(t_k) + n(t_k) - n(t_{k+1}) \right]\right\}.
    \end{split}
\end{equation}

We may formally write $\mathcal{P}$ as a path integral over functions $\tilde n(t)$: 
\begin{equation}
    \mathcal{P}(\{n(t)\})
    = \int \mathcal{D}[\tilde n]\,
    \exp \int dt \left[
    \sum_{l=1}^{\infty} \frac{\tilde n(t)^{\,l}}{l!} K_l[n(t),m(t)]
    \;-\; \tilde n(t)\,\dot n(t)
    \right].
\end{equation}

As will be shown, the cumulants or order $ l\geq 3 $ are irrelevant, and the task is to find the functional forms of the mean-field part $K_1$ and the Gaussian stochastic correlator $K_2$. This is because the first approximation when evaluating path integrals usually consists of a Gaussian truncation in the action $ J $. We can write the stochastic equation of motion usual Langevin form as: 
\begin{subequations}
    \begin{align}
        \partial_t n(t) &= K_1 [n(t), m(t)] + \zeta (t), \\
        \overline{\zeta (t) \zeta (t^{\prime})} &= K_2 [n(t), m(t)] \delta (t - t^{\prime}).
    \end{align}
\end{subequations}

\begin{remark}
    The relevant cumulants $K_i$ here are exactly the mean and the variance of the stochastic variable $ \mathcal{V} $, and thus the stochastic process is completely determined by its first two moments.
\end{remark}

The diffusive spreading behevior implies that the cumulants $ K_1 $ and $ K_2 $ must be local functionals of the density fields, and may thus be expanded in powers of spatial gradients. Furthermore, the absorbing state condition requires that both $ K_1 $ and $ K_2 $ vanish for $ n=0 $. Taking into account only the most relevant contributions, we may write:

\begin{align}
    K_1(n,m) &= R(n,m)\,n+\lambda\,\nabla^2 n+\ldots, \\
    K_2\!\left(n,m;\,r-r'\right) &= 2\Big[\Gamma(n,m)\,n+\lambda'\,(\nabla^2 n)-\lambda''\,n\nabla^2+\ldots\Big]\, \delta(r-r') .
\end{align}

\begin{equation}
    R(n,m)=-\lambda\bigl(\tau+g_{1}n+g_{2}m+\ldots\bigr), \quad \Gamma(n,m)=\lambda\bigl(g_{3}+\ldots\bigr). 
\end{equation}

For DP, which we will focus on here, we will set $ g_2 = 0 $, and we assume all the coupling constants $ g_i $ to be positive. The parameter $ \tau $ is the control parameter that tunes the system through the phase transition, and $ q $ is an external source term that allows for spontaneous creation of active particles.

\begin{equation}
    \mathcal{J} =\int d^d r\,dt\left\{ \tilde{n}\Big[\partial_t+\lambda(\tau-\nabla^2)+\lambda\big(g_1 n+g_2 m-g_3 \tilde{n}\big)\Big]n -q\,\tilde{n} \right\}.
\end{equation}

\section{\label{sec:field_theory_DP}Field Theory of Directed Percolation}

\begin{center}
    \textit{DP is one of the simplest model of a strictly non-equilibrium system displaying a continuous phase transition.} 
\end{center}

Recall that the Ising model, described by the Landau-Ginzburg-Wilson Hamiltonian, is a canonical example model for equilibrium critical phenomena. Analogously, the \emph{Reggeon field theory} (RFT) provides a minimal field-theoretic model for DP universality class, and may be regarded as the non-equilibrium analog of the $ \phi^4 $ field theory. 

To study the critical behavior of DP near the phase transition, we start from the dynamic functional for DP, and decide a scheme for renormalization. Similarly to the renormalization scheme for the Ising model introduced in class, where we replace a collection of sites by super-sites given a coarse-graining scheme, and study percolation of the new super-lattice. See Figure~\ref{fig:coarse_graining} from \cite{Roemer2001} for an illustration of the coarse-graining scheme for percolation processes.

\begin{figure}
    \centering
    \includegraphics[width=0.4\linewidth]{Images/coarse_graining.png}
    \caption{Coarse-graining scheme for renormalization of percolation processes. These combination of bonds lead to a horizontal super-bond after renormalization, and the probability $ p^{\prime} $ of the thick bond occuring is the sum of probabilities of each scenario occuring.}
    \label{fig:coarse_graining}
\end{figure}

\subsection{Critical Dimension}
In the field theory for spreading phenomena with an absorbing state, the dynamic response functional $ J $ contains a redundant parameter that must be eliminated. We can achieve this by a suitable rescaling, which we will detail now. Assume $ \tilde{n}=K^{-1}\tilde{s} $, $ n=Ks $, $ m=KS $, where $ K $ is an amplitude with non-vanishing scaling dimension. We choose $ K $ such that the couplings attain identical scaling dimensions: $ 2K g_1 = 2K^{-1} g_3 = g $. 

We can write 
\begin{equation}
    \mathcal{J}_{\mathrm{DP}} =\int d^{d}r\,dt\left\{ \tilde{s}\left[\partial_{t}+\lambda(\tau-\nabla^{2})+\frac{\lambda g}{2}\,(s-\tilde{s})\right]s -\lambda h\,\tilde{s} \right\},
\end{equation}
and, fixing the redundancy of the theory, we have the naive scaling dimensions given by 
\begin{equation}
    s \sim \tilde{s} \sim \mu^{d/2}, \quad g \sim \mu^{(4-d)/2}. 
\end{equation}
From the scaling law of the coupling constant $ g $, we can infer directly that the upper critical dimension for DP is $ d_c = 4 $, above which mean-field theory becomes exact. 

\subsection{Renormalization Group and Propagator}
To get into relevant loop calculations, we first determine the correlation and response functions of the dynamical variables, as functions of their space-time coordinates. According to \cite{JANSSEN2005147}, in a compact form, one attempts to determine the cumulant generating functional

\begin{align}
\mathcal{W}[H,\tilde{H}]
&=\ln\int \mathcal{D}[\tilde{s},s]\,
\exp\!\left[-\mathcal{J}[\tilde{s},s]+(H,s)+(\tilde{H},\tilde{s})\right].
\end{align}

Functional derivatives with respect to the sources $H$ and $\tilde{H}$ gives the Green function $ G_{N, \tilde{N}} $:
\begin{align}
    \left. \frac{\delta^{N+\tilde{N}}\mathcal{W}} {\bigl[\delta H^{N}\bigr]\bigl[\delta \tilde{H}^{\tilde{N}}\bigr]} \right|_{H=\tilde{H}=0} &=\bigl\langle [s^{N}][\tilde{s}^{\tilde{N}}]\bigr\rangle^{(\mathrm{cum})} =: G_{N,\tilde{N}}.
\end{align}

The Gaussian parts of the response functionals define the propagator, and we may write directly: 
\begin{equation}
    \langle s(\mathbf{r},t)\,\tilde{s}(\mathbf{r}',t')\rangle_{0} = G(\mathbf{r}-\mathbf{r}',\,t-t'), \quad G(\mathbf{r},t) = \int_{q,\omega}\frac{\exp\!\big(i\mathbf{q}\cdot\mathbf{r}-i\omega t\big)} {-i\omega+\lambda\,(q^{2}+\tau)}.
\end{equation}

Next, we proceed by noticing that the generating functional for the \emph{vertex functions} $\Gamma [\tilde{s}, s]$ is related to the cumulant generating functional via the Legendre transformation.
\begin{equation}
    \Gamma[\tilde{s},s]+\mathcal{W}[H,\tilde{H}]=(H,s)+(\tilde{H},\tilde{s}), \quad \text{with}\quad s=\frac{\delta\mathcal{W}}{\delta H}, \quad \tilde{s}=\frac{\delta\mathcal{W}}{\delta\tilde{H}}\,.
\end{equation}

We may start with the perturbation contribution of the Gaussian cutoff $ \exp (\mathcal{J}_0) $, and the subsequent expansion in terms of Feynman diagrams. The different contributions to the series can be graphically organized in successive order of closed loops. The different contributions can be decomposed into one-line irreducible amputated Feynman diagrams that represent the building blocks for the vertex functions. These Feynman diagrams up to two-loop order are shown in Figures~\ref{fig:one_loop}, \ref{fig:two_loop_self}, and \ref{fig:propagator}. For the detailed calculations, please refer to \cite{Janssen2001DirectedPercolationWithColorsFlavors}.

\begin{figure}[htbp]
    \centering
    \begin{minipage}[b]{0.4\linewidth}
        \includegraphics[width=0.5\linewidth]{Images/one_loop.png}
        \caption{One-loop diagram contributing to the renormalization of the coupling constant in DP.}
        \label{fig:one_loop}
    \end{minipage} 
    \hfill
    \begin{minipage}[b]{0.55\linewidth}
        \includegraphics[width=0.9\linewidth]{Images/two_loop_self.png}
        \caption{Two-loop diagrams contributing to the renormalization of the coupling constant in DP.}
        \label{fig:two_loop_self}
    \end{minipage}
\end{figure}

\begin{figure}
    \centering
    \includegraphics[width=0.8\linewidth]{Images/two_loop_vertex.png}
    \caption{Propagator in DP. The arrow indicates the direction of time.}
    \label{fig:propagator}
\end{figure}

As is common with field theories, such as the stastical field theory considered here. The naive perturbation expansion, here based on our response functionals, is problematic, especially in the zero mass limit. We rename the original bare fields and parameters according to multiplicative renormalization. For DP, the main topic of concern here, we set $ \tilde{Z} = Z $. This gives the following renormalization scheme:

\begin{align}
\mathring{s} &= Z^{1/2}s,
&
\mathring{\tilde{s}} &= \tilde{Z}^{1/2}\tilde{s},
&
G_{\epsilon}\,\mathring{g}^{2} &= \tilde{Z}^{-1}Z_{\lambda}^{-2}Z_{u}\,u\,\mu^{\epsilon},
\\
\mathring{\lambda} &= (Z\tilde{Z})^{-1/2}Z_{\lambda}\lambda,
&
\mathring{\tau} &= Z_{\lambda}^{-1}Z_{\tau}\tau+\mathring{\tau}_{c},
&
\mathring{h} &= Z^{1/2}Z_{\lambda}^{-1}h,
\\
\tilde{Z} &= Z \quad \text{for DP}, 
&
\tilde{Z} &= Z_{\lambda} \quad \text{for dIP}.
\end{align}

Hence, the expressions for the RG functions are given by 
\begin{align}
Z &= 1+\frac{u}{4\varepsilon}
+\left(\frac{7}{\varepsilon}-3+\frac{9}{2}\ln\frac{4}{3}\right)\frac{u^{2}}{32\varepsilon}
+ O\!(u^{3}),
\\
Z_{\lambda} &= 1+\frac{u}{8\varepsilon}
+\left(\frac{13}{4\varepsilon}-\frac{31}{16}+\frac{35}{8}\ln\frac{4}{3}\right)\frac{u^{2}}{32\varepsilon}
+ O\!(u^{3}),
\\
Z_{\tau} &= 1+\frac{u}{2\varepsilon}
+\left(\frac{16}{\varepsilon}-5\right)\frac{u^{2}}{32\varepsilon}
+ O\!(u^{3}),
\\
Z_{u} &= 1+\frac{2u}{\varepsilon}
+\left(\frac{4}{\varepsilon}-1\right)\frac{7u^{2}}{8\varepsilon}
+ O\!(u^{3}).
\end{align}

The beta function can be computed from the dimensional regularization variable $ \varepsilon = 4 - d $, and the following expression for the beta function is obtained: 
\begin{equation}
    \beta(u) =\left[-\varepsilon+\frac{3u}{2}-\left(169+106\ln\frac{4}{3}\right)\frac{u^{2}}{128} + O\!(u^{3})\right]u. 
\end{equation}

The function has a nontrivial fixed point $ u_{\ast} $ for $ \varepsilon > 0 $, which is infrared stable, and as an expansion in the parameter $ \varepsilon $, we have
\begin{equation}
    u_{*} =\frac{2\varepsilon}{3}\left[1+\left(\frac{169}{288}+\frac{53}{144}\ln\frac{4}{3}\right)\varepsilon + O\!(\varepsilon^{2})\right].
\end{equation}

\section{Conclusion}
From the introduction section, we established the mathematical fact that percolation processes on lattices exhibit finite-temperature phase transitions, and they are thus a simple example model of non-equilibrium critical phenomena. Next, near-critical behaviors of these percolating systems are described by field theories with few parameters.

In order to study the near-critical behaviors and determine the appropriate renormalization group properties, the author construct a path integral based on birth-lattice epidemic models, treating directed percolation as modeling the irreversible diffusion of an agent through a disordered medium. 

The dynamic response functional is identified as the appropriate effective actionm, and loop expansions up to second order can be written down after dimensional regularization. This formalism gives a systematic way to:
classify universality classe and compute critical exponents and RG functions
of percolation processes.

\appendix

\section{Appendixes}

\subsection{Comparison of DP and Ising Model}

The symmetry group for \textbf{DP} is trivial, as there is only one absorbing state, while the Ising model has a $ \mathbb{Z}_2 $ symmetry corresponding to spin-flip symmetry. Notice that the two-variable dynamic response functional in the theory discussion includes discussion of the \textbf{dIP} class. See Table~\ref{tab:crit-exponents-dp-ising} for a comparison of critical exponents between DP and Ising model universality classes.

\begin{table}[htbp]
\centering
\renewcommand{\arraystretch}{1.15}
    \caption{Critical exponents for the \textbf{DP} and \textbf{Ising Model} universality classes. The numerical numbers have been truncated manually to at most six decimal places. The entry $4^{+}$ denotes mean-field exponents above the upper critical dimension $ d_{\text{c}} $.}
    \label{tab:crit-exponents-dp-ising}
    \begin{tabular}{lllcccccc}
    \toprule
    Class & Dimension & Symmetry & $\alpha$ & $\beta$ & $\gamma$ & $\delta$ & $\nu$ & $\eta$ \\
    \midrule
    \multirow{4}{*}{\textbf{DP}}
    & 1 & 1 & 0.159464 & 0.276486 & 2.277730 & 0.159464 & 1.096854 & 0.313686 \\
    & 2 & 1 & 0.451 & 0.536 & 1.60 & 0.451 & 0.733 & 0.230 \\
    & 3 & 1 & 0.73 & 0.813 & 1.25 & 0.73 & 0.584 & 0.12 \\
    & $4^{+}$ & 1 & 1 & 1 & 1 & 1 & $\tfrac{1}{2}$ & 0 \\
    \midrule
    \multirow{3}{*}{\textbf{IsingModel}}
    & 2 & $\mathbb{Z}_2$ & 0 & $\tfrac{1}{8}$ & $\tfrac{7}{4}$ & 15 & 1 & $\tfrac{1}{4}$ \\
    & 3 & $\mathbb{Z}_2$ & 0.110087 & 0.326418 & 1.237075 & 4.789842 & 0.629971 & 0.036298 \\
    & $4^{+}$ & $\mathbb{Z}_2$ & 0 & $\tfrac{1}{2}$ & 1 & 3 & $\tfrac{1}{2}$ & 0 \\
    \bottomrule
    \end{tabular}
\end{table}


\nocite{*}
\bibliography{references}% Produces the bibliography via BibTeX.

\end{document}
%
% ****** End of file sorsamp.tex ******
