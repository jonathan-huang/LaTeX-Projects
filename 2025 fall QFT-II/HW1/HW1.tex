\documentclass[11pt]{article}

\usepackage[height=9.6in,width=6.45in]{geometry}
%\usepackage{showkeys}
\renewcommand{\baselinestretch}{1.08}

\usepackage{xspace}

\newcommand{\Poincare}{Poincar\'e\xspace}

\usepackage{amssymb}
%\usepackage{hyperref}
\usepackage{physics}
\usepackage{color}
\usepackage{graphicx}
\usepackage{srcltx}
\usepackage{mathrsfs}
\usepackage{lipsum}% Just for this example
\usepackage{slashed}
\usepackage[dvipsnames]{xcolor}
\usepackage[makeroom]{cancel}

\newcommand{\doublerule}[1][.4pt]{%
  \noindent
  \makebox[0pt][l]{\rule[.7ex]{\linewidth}{#1}}%
  \rule[.3ex]{\linewidth}{#1}}

%% My add-ons %%
%% Formatting %%
\usepackage{multicol}
\usepackage[dvipsnames]{xcolor}
\usepackage{ulem}
\usepackage{parskip}

%% Math and Symbols %%
\usepackage{amsmath,amsthm,amssymb, mathtools}
\usepackage{yhmath, faktor, dsfont}
\usepackage{academicons, wasysym, marvosym}
\usepackage[scr]{rsfso} 
\usepackage{latexsym, amsmath, amscd, amsmath, amsthm}
\usepackage{amssymb,amsmath,amsthm,graphicx,dsfont}
\usepackage{hyperref}

%% Enhancement %%
\usepackage{graphicx, tabularx}
\usepackage{booktabs}
\usepackage[shortlabels,inline]{enumitem}
%% TikZ %%
\usepackage{tikz}
\usepackage[breakable]{tcolorbox}
\usetikzlibrary{decorations.pathmorphing}
\usetikzlibrary{calc, arrows,matrix}
\usetikzlibrary{arrows.meta, decorations.pathmorphing}
\usepackage{tikz-feynman}

\newcommand{\bvec}[1]{\mathbf{#1}} % vector

%% Other packages %%
\usepackage{amsopn}

%% Traditional Chinese %%
\usepackage{CJKutf8}

%% Math environments %%
\newtheoremstyle{mystyle}
  {6pt}{15pt}% 上下間距
  {}%          內文字體
  {}%              縮排
  {\bf}%       標頭字體
  {.}%       標頭後標點
  {1em}% 內文與標頭距離
  {}% Theorem head spec (can be left empty, meaning 'normal')
\theoremstyle{mystyle}	
\newtheorem{theorem}{Theorem}
\newtheorem{definition}{Definition}
\newtheorem{example}[theorem]{Example}
\newtheorem{exercise}{Exercise}
\newtheorem{solution}{Solution}
\newtheorem{corollary}[theorem]{Corollary}
\newtheorem{property}[theorem]{Property}
\newtheorem{proposition}[theorem]{Proposition}
\newtheorem{lemma}[theorem]{Lemma}
\newtheorem{problem}{Problem}
\newtheorem{answer}{Answer}[section]
\newtheorem{fact}[theorem]{Fact}
\newtheorem*{recall}{Recall}
\newtheorem*{remark}{Remark}
\newtheorem*{claim}{Claim}
\newtheorem*{observation}{Observation}

\usepackage{listings}
\usepackage{xcolor}
\lstdefinestyle{matlab}{
  language=Matlab,
  basicstyle=\ttfamily\small,
  numbers=left,
  numberstyle=\tiny,
  stepnumber=1,
  frame=single,
  breaklines=true,
  showstringspaces=false,
  tabsize=2,
  keywordstyle=\color{blue},
  commentstyle=\color{teal!60!black}\itshape,
  stringstyle=\color{purple!70!black},
  backgroundcolor=\color{white}
}

\begin{document}
\begin{CJK}{UTF8}{bkai}

\centerline {\Large \bf Phys8018 QFT~(II) -- Fall 2025} 
\vspace{0.1cm}
\centerline {\large \bf Instructors: Chia-Hsien Shen $\&$ Chang-Tse Hsieh}
\vspace{0.2cm}
\centerline {\large \textbf{Student:} 物理三 黃紹凱 B12202004}
\doublerule\\
{\large \bf Problem Set \#1 [\textcolor{red}{120 pts}] \qquad Due \textcolor{red}{October 20, 2025}}
\\ \\
1.~\textbf{Scattering Amplitudes from Feynman Diagrams} [\textcolor{red}{20 pts}]  \\ 
Consider a theory of a massless scalar field $\phi(x)$ with the following Lagrangian
\begin{align}
    \mathcal{L} &= \frac{1}{2} \partial^\mu \phi \partial_\mu \phi 
    + \frac{c_3}{2!} \phi (\partial^\mu \phi \partial_\mu \phi)
    + \frac{c_4}{4} \phi^2 (\partial^\mu \phi \partial_\mu \phi).
\end{align}
We use $(+,-,-,-)$ metric from now on.\\
(a) Using this Lagrangian, derive the Feynman vertices $V_3(p_1,p_2,p_3)$ and $V_4(p_1,p_2,p_3,p_4)$ with three and four scalar fields. \\ \\
(b) Derive the scattering amplitude of three scalar scattering, $A_3(p_1,p_2,p_3)$. For simplicity, we use the convention that all the momenta are out-going, such that $p_1+p_2+p_3=0$.
Make sure that you simplify the result by applying all the constraints from momentum conservation and on-shell conditions on the external particles, $p_i^2=0$. \\ \\
(c) For the scattering of four massless particles, we have the momentum conservation $p_1+p_2+p_3+p_4=0$ and on-shell conditions. Show that any pair of $p_i\cdot p_j$ can be written as the following three variables,
\begin{align}
    s \equiv (p_1+p_2)^2, \quad t\equiv (p_1+p_4)^2, \quad 
    u \equiv (p_1+p_3)^2,
\end{align}
and only two out of the three are independent because
\begin{align}
    s+t+u=0.
\end{align}
These Lorentz-invariant inner products are called the Mandelstam invariants. Each of them corresponds to the propagator of an exchange diagram.\\ \\
(d) Derive the four-scalar amplitude, $A_4(p_1,p_2,p_3,p_4)$, as a function of $c_3$ and $c_4$. Make sure that you include all the Feynman diagrams and write the result in terms of $s$ and $t$. Do you find simple results for $A_3(p_1,p_2,p_3)$ and $A_4(p_1,p_2,p_3,p_4)$?

\begin{solution}
    ~ 

    (a) For $ \frac{c_3}{2} \phi (\partial \phi)^2 $, there are $ 2! $ ways to assign the differentiated vertices. For instance, if the undifferentiated leg is $ 1 $, then the derivative factor is
    \begin{equation}
        (-i p_2) \cdot (-i p_3) = - p_2 \cdot p_3.
    \end{equation}
    Therefore,
    \begin{equation}
        V(p_1, p_2, p_3) = - i \times 2! \times \frac{c_3}{2} \sum_{i<j} p_{i} \cdot p_{j} = -i c_3 \sum_{i<j}^3  (p_i \cdot p_j). 
    \end{equation}
    Similarly, for $ \frac{c_4}{4} \phi^2 (\partial \phi)^2 $, there are $ 2! \times 2! $ ways to assign the differentiated and undifferentiated vertices. For instance, if the undifferentiated legs are $ 1, 2 $, then the derivative factor is
    \begin{equation}
        (-i p_3) \cdot (-i p_4) = - p_3 \cdot p_4.
    \end{equation}  
    Therefore,
    \begin{equation}
        V(p_1, p_2, p_3, p_4) = -i \times (2! \times 2!) \times \frac{c_4}{4} \sum_{i<j} p_i \cdot p_j = - i c_4 \sum_{i<j}^4 p_i \cdot p_j . 
    \end{equation}
    These vertices are totally symmetric in their legs.

    (b) Tree-level $ A_3 $ is the vertex evaluated on-shell. Using momentum conservation $ p_1 + p_2 + p_3 = 0 $, we have 
    \begin{equation}
        (p_1 + p_2 + p_3)^2 = 0 = \sum_{i=1}^3 p_i^2 + 2 \sum_{i<j} p_i \cdot p_j = 2 \sum_{i<j} p_i \cdot p_j, 
    \end{equation}
    where we used the on-shell condition for massless external legs $ p_i^2 = 0 $. Therefore, the scalar three-amplitude vanishes: $ A_3 (p_1, p_2, p_3) = 0 $.
    
    (c) We can expand the Mandelstam invariants for on-shell massless external legs as follows:
    \begin{equation}
        \begin{split}
            s &= (p_1 + p_2)^2 = 2 (p_1 \cdot p_2), \\
            t &= (p_1 + p_4)^2 = 2 (p_1 \cdot p_4), \\
            u &= (p_1 + p_3)^2 = 2 (p_1 \cdot p_3).
        \end{split}
    \end{equation}
    Using momentum conservation $ p_1 + p_2 + p_3 + p_4 = 0 $, we also have 
    \begin{equation}
        \begin{split}
            s = (p_3 + p_4)^2 = 2 (p_3 \cdot p_4), \\
            t = (p_2 + p_3)^2 = 2 (p_2 \cdot p_3), \\
            u = (p_2 + p_4)^2 = 2 (p_2 \cdot p_4).
        \end{split}
    \end{equation}   
    Therefore, together we have 
    \begin{equation}
        s + t + u = 2 \sum_{i<j} p_i \cdot p_j = 0,
    \end{equation}
    and only two out of the three Mandelstam variables are independent.

    (d) At tree-level, the expansion includes contact and exchange diagrams. The contact diagram is simply the four-point vertex evaluated on-shell:
    \begin{equation}
        i \mathcal{M} = i V(p_1, p_2, p_3, p_4) = c_4 \sum_{i<j} p_i \cdot p_j = - i c_4 (s + t + u).
    \end{equation}
    The exchange diagrams include $ s, t, u $ channels. For the $ s $-channel, the internal momentum is $ q_s = p_1 + p_2 $, with $ q_s^2 = s $. Similarly we define the internal momenta for $ t $ and $ u $ channels as $ q_t = p_1 + p_4 $, $ q_u = p_1 + p_3 $. Then 
    \begin{equation}
        \begin{split}
            V_3 (p_1, p_2, -q_s) &= - i c_3 \left(p_1 \cdot p_2 - p_1 \cdot q_s - p_2 \cdot q_s \right) \\
            &= - \frac{i c_3}{2} \left[(p_1 + p_2 - q_s)^2 - p_1^2 - p_2^2 - (-q_s)^2 \right] = - \frac{c_3}{2} s, \\
            V_3 (p_1, p_4, -q_t) &= - i c_3 \left(p_1 \cdot p_4 - p_1 \cdot q_t - p_4 \cdot q_t \right) \\
            &= - \frac{i c_3}{2} \left[(p_1 + p_4 - q_t)^2 - p_1^2 - p_4^2 - (-q_t)^2 \right] = - \frac{c_3}{2} t, \\
            V_3 (p_1, p_3, -q_u) &= - i c_3 \left(p_1 \cdot p_3 - p_1 \cdot q_u - p_3 \cdot q_u \right) \\
            &= - \frac{i c_3}{2} \left[(p_1 + p_3 - q_u)^2 - p_1^2 - p_3^2 - (-q_u)^2 \right] = - \frac{c_3}{2} u.
        \end{split} 
    \end{equation} 
    The corresponding amplitudes are
    \begin{equation}
        \mathcal{M}_s = V_3 (p_1, p_2, -q_s) \frac{i}{s + i \varepsilon} V_3 (p_3, p_4, q_s) \to - \frac{c_3^2}{4} s, 
    \end{equation}
    and similarly,
    \begin{equation}
        \begin{split}
            \mathcal{M}_t &= V_3 (p_1, p_4, -q_t) \frac{i}{t + i \varepsilon} V_3 (p_2, p_3, q_t) \to - \frac{c_3^2}{4} t, \\
            \mathcal{M}_u &= V_3 (p_1, p_3, -q_u) \frac{i}{u + i \varepsilon} V_3 (p_2, p_4, q_u) \to - \frac{c_3^2}{4} u.
        \end{split}
    \end{equation}
    Therefore, the total four-point amplitude is
    \begin{equation}
        \begin{split}
            A_4 (p_1, p_2, p_3, p_4) &= \mathcal{M} + \mathcal{M}_s + \mathcal{M}_t + \mathcal{M}_u \\
            &= - i c_4 (s + t + u) - \frac{c_3^2}{4} (s + t + u) = 0.
        \end{split}
    \end{equation}
    The Feynman diagrams involved in the amplitude calculation are shown in Fig.~\ref{fig:4point}.

    \begin{figure}
        \centering 
        \includegraphics[width=0.9\linewidth]{4point.jpg}
        \caption{The diagrams for four-point scattering amplitude. (a): contact diagram; (b)-(d): exchange diagrams in $ s, t, u $ channels respectively.}
        \label{fig:4point}
    \end{figure}

    The interactions are removable by a field redefinition of the free theory, given by 
    \begin{equation}
        \begin{split}
            \phi &\to \phi^{\prime} = \phi + \frac{c_3}{4} \phi^2 + \frac{c_4}{12} \phi^3, \\
            \frac{1}{2} (\partial \phi)^2 &\to \frac{1}{2} (\partial \phi^{\prime})^2= \frac{1}{2} (\partial \phi)^2 + \frac{c_3}{2} \phi (\partial \phi)^2 + \frac{c_4}{4} \phi^2 (\partial \phi)^2.
        \end{split}
    \end{equation}
    By LSZ reduction, the S-matrix is invariant under field redefinition, so the on-shell amplitudes vanish as they do in the free theory case.
\end{solution}

\newpage 

% Problem 2
2.~\textbf{Spinor Helicity} [\textcolor{red}{30 pts}]  \\
(a) Spinor identity 1: since the angle spinor is a 2-dimensional vector, any spinor $|p_3\rangle \equiv |3\rangle $ must be spanned by two other spinors,
\begin{equation}
    |3\rangle = \alpha |1\rangle + \beta |2\rangle,
    \label{eq:Schouten1}
\end{equation}
where $\alpha,\beta$ are two coefficients and we drop the spinor indices for simplicity. Determine the coefficients $\alpha,\beta$ in terms of the spinor inner product $\langle i j \rangle$. Note that $p_1,p_2,p_3$ are not related by momentum conservation. This identity is called the Schouten identity.\\
\textit{Comment: Contract the spinor relation~\eqref{eq:Schouten1} with $\langle 4|$ yields $\langle 43\rangle = \alpha \langle 41\rangle + \beta \langle 42\rangle$ which is closely related to the Fierz identity among fermionic fields.} \\ \\
(b) Spinor identiy 2: consider four-particle scattering with $p_1+p_2+p_3+p_4=0$. Derive
\begin{equation}
    \langle 1 3\rangle [32] = -\langle 1 4\rangle [42].
\end{equation}
\textit{Hint: first show that $\langle 1 3\rangle [32] = \langle 1|_{\dot{a}} (p_3)^{\dot{a}a} {}_a |2] \equiv \langle 1| p_3 |2]$}
\\ \\
(c) Four-gluon scattering at a glance: using the factorization formula and the two spinor identities above, we can extend the simple three-particle amplitudes in spinor helicity variables into four particle scattering. Consider the two three-gluon amplitudes
\begin{align}
    A_3(4^+,1^-,I^-) &= \frac{\langle 1I\rangle^3}{\langle I4\rangle \langle 41\rangle } \\
    A_3(2^-,3^+,I^+) &= \frac{[3I]^3}{[23][I2]},
\end{align}
where the superscripts denote the helicities and we have factored out the prefactors (color factors) for the simplicity. (The removal of the color factors implies that gluons do not have Bose symmetry anymore.) As shown from the diagram in Fig.~\ref{fig:1}, when the internal line $I$ becomes on-shell, the four-gluon amplitude factorizes into the product of the two three-gluon amplitudes
\begin{align}
    \lim_{(p_2+p_3)^2=0}A_4(1^-,2^-,3^+,4^+) &\rightarrow 
    -\frac{A_3(4^+,1^-,I^-) A_3(2^-,3^+,I^+)}{\langle 23\rangle [32]} \bigg\vert_{p_I = p_2+p_3}.
\end{align}
The momentum $p_I =p_2+p_3$ is on-shell, with $|I] \propto |1]$ and $|I\rangle \propto |2\rangle$.
Using the type of spinor identities in (a) and (b), show that the right-hand side of the above simplifies to
\begin{equation}
    -\frac{A_3(4^+,1^-,I^-) A_3(2^-,3^+,I^+)}{\langle 23\rangle [32]} \bigg\vert_{p_I = p_2+p_3} = -\frac{\langle 12\rangle^4}{\langle 12 \rangle \langle 23 \rangle \langle 34 \rangle \langle 41 \rangle}\,.
\end{equation}
\textit{Comment: It turns out that the 1-line expression $\frac{\langle 12\rangle^4}{\langle 12 \rangle \langle 23 \rangle \langle 34 \rangle \langle 41 \rangle}$ is the correct four-gluon amplitude in generic kinematics. The result immediately generalizes to $n$-gluon scattering. This is the famous MHV formula discovered by Parke and Taylor.}
\begin{figure}
    \centering
    \includegraphics[width=0.5\linewidth]{fact.png}
    \caption{The factorization of a four-gluon amplitude into two three-gluon amplitudes}
    \label{fig:1}
\end{figure}

\begin{solution}
    ~

    (a) The spinor inner product is antisymmetric, so we can contract equation (\ref{eq:Schouten1}) with \( \langle 1 \vert \) and \( \langle 2 \vert \) to get
    \begin{equation}
        \langle 2 3 \rangle = \alpha \langle 2 1 \rangle + \beta \langle 2 2 \rangle = \alpha \langle 2 1 \rangle, \quad \langle 1 3 \rangle = \alpha \langle 1 1 \rangle + \beta \langle 1 2 \rangle = \beta \langle 1 2 \rangle.
    \end{equation}
    Then we have the \emph{Schouten identity}:
    \begin{equation}
        \vert 3 \rangle = \frac{\langle 2 3 \rangle}{\langle 2 1 \rangle} \vert 1 \rangle + \frac{\langle 1 3 \rangle}{\langle 1 2 \rangle} \vert 2 \rangle. 
    \end{equation}

    (b) Following the hint, we can write the momenta as a bispinor $ p_{a \dot{a}} = \vert p \rangle_{a} [ p \vert_{\dot{a}} $, so $ \langle 1 3 \rangle [32] = \langle 1|^{\dot{a}} (p_3)_{\dot{a}a} {}^a |2] \equiv \langle 1| p_3 |2] $. Then using the momentum conservation $ p_1 + p_2 + p_3 + p_4 = 0 $ condition, we have
    \begin{equation}
        \langle 1 3 \rangle [32] = \langle 1| p_3 |2] = - \langle 1| (p_1 + p_2 + p_4) |2] = - \langle 1| p_4 |2] = - \langle 1 4 \rangle [42].
    \end{equation}

    (c) Apply (b) to the expressions involving $ p_I = p_2 + p_3 $, and use antisymmetry properties of the spinor inner products, we have
    \begin{enumerate}[(i)]
        \item Numerator: $ - \langle 1 I \rangle^3 [I 3]^3 = - \langle 1 \vert p_I \vert 3 \rangle^3 = - \langle 1 \vert p_2 + p_3 \vert 3 \rangle ^3 = - \langle 1 \vert p_2 \vert 3 \rangle ^3 = - \langle 1 2 \rangle^3 [2 3]^3. $ 
        \item Denominator: $ \langle I 4\rangle [I 2] = - \langle 4 I \rangle [I 2] = - \langle 4 \vert p_I \vert 2 ] = - \langle 4 \vert p_2 + p_3 \vert 2 ] = - \langle 4 \vert p_3 \vert 2 ] = - \langle 4 3 \rangle [3 2]. $
    \end{enumerate}
    Then, the right hand side becomes 
    \begin{equation}
        \begin{split}
            \text{RHS} &= - \frac{A_3 (4^+, 1^-, I^-) A_3 (2^-, 3^+, I^+)}{\langle 23 \rangle [32]} \bigg\vert_{p_I = p_2+p_3} \\
            &= - \frac{\langle 1 I \rangle^3 [3 I]^3}{\langle I 4 \rangle \langle 4 1 \rangle [23] [I2] \langle 2 3 \rangle [32]} \bigg\vert_{p_1 = p_2 + p_3} \\
            &= - \frac{- \langle 1 2 \rangle^3 [2 3]^3}{\langle 41 \rangle [23] \langle 23 \rangle [32] (- \langle 4 3 \rangle [3 2])} \\
            &= \frac{\langle 12 \rangle^4}{\langle 12 \rangle \langle 23 \rangle \langle 34 \rangle \langle 41 \rangle}. 
        \end{split}
    \end{equation}
\end{solution}

\newpage 

% Problem 3
3.~\textbf{Path integrals in QM} [\textcolor{red}{20 pts}]  \\ 
(a) Given a quantum Hamiltonian $H(\hat{p}, \hat{q})$ that is Weyl-ordered, i.e.,
\begin{equation}
H(\hat{p}, \hat{q}) = \int\frac{dx}{2\pi}\frac{dk}{2\pi}\, e^{i(x\hat{p}+k\hat{q})}\int dpdq\, e^{-i(xp+kq)} H(p, q)
\end{equation} 
from a classical Hamiltonian $H(p, q)$, show that the transition amplitude from $|q_1\rangle$ to $|q_2\rangle$ within a small time interval $\delta t$ is 
\begin{equation}
\langle q_2| e^{-iH(\hat{p}, \hat{q})\delta t}|q_1\rangle = \int \frac{dp_1}{2\pi}\, e^{-iH\left(p_1, \frac{q_1+q_2}{2}\right)\delta t} e^{ip_1(q_2-q_1)}.
\end{equation}
\\ \\ 
(b) Consider a Lagrangian with a velocity-dependent potential:
\begin{equation}
L (\dot{q}, q) = \frac{1}{2} \dot{q}^2 f(q),
\end{equation} 
where $f(q)$ is a non-singular function of $q$. Let's quantize the classical system described by $L$ using Weyl ordering.  The quantum mechanical amplitude from $q_i$ at time $t_i$ to $q_f$ at time $t_f$ can then be expressed by the following functional integral:
\begin{equation}
\langle q_f, t_f| q_i, t_i\rangle = \int\mathcal{D}q\, e^{i\int_{t_i}^{t_f}dt\, L_{\text{eff}}(\dot{q}, q)},
\end{equation} 
Find the exact form of $L_{\text{eff}}(\dot{q}, q)$.
Is it identical to $L(\dot{q}, q)$?

\begin{solution}
    ~

    (a) Recall the BCH formula: for two operators $ X, Y $, we have 
    \begin{equation}
        e^X e^Y = e^{X + Y + \frac{1}{2} [X, Y] + \frac{1}{12} ([X, [X, Y]] + [Y, [Y, X]]) + \cdots },
    \end{equation}
    where the higher order terms involve nested commutators. Notice that 
    \begin{equation}
        e^{i(k \hat{q} + x \hat{p})} = e^{\frac{1}{2}k \hat{q}} e^{ix \hat{p}} e^{\frac{i}{2} k \hat{q}}
    \end{equation}
    by the canonical commutation relation $ [k \hat{q}, x \hat{p}] = i k x $, hence 
    \begin{equation}
        \langle q_2 \vert e^{i(x \hat{p} + k \hat{q})} \vert q_1 \rangle = e^{\frac{i}{2} k (q_1 + q_2)} \langle q_2 \vert e^{i x \hat{p}} \vert q_1 \rangle = e^{\frac{i}{2} k (q_1 + q_2)} \delta (q_1 - q_2 - x).
    \end{equation}
    The Weyl-ordered Hamiltonian is then
    \begin{equation}
        \begin{split}
            \langle q_2 \vert H(\hat{p} , \hat{q} ) \vert q_1 \rangle &= \int \frac{\mathrm{d}x}{2 \pi} \int \frac{\mathrm{d}k}{2 \pi} \, \langle q_2 \vert e^{\frac{1}{2} k \hat{q}} e^{i x \hat{p}} e^{\frac{1}{2} k \hat{q}} \vert q_1 \rangle \int \mathrm{d}p \, \int \mathrm{d}q\, e^{-i(xp + kq)} H(p,q) \\
            &= \int \frac{\mathrm{d}x}{2\pi} \int \frac{\mathrm{d}k}{2\pi} \int \mathrm{d}p \int \mathrm{d}q\, e^{\frac{i}{2}k (q_1 + q_2)} \delta (q_1 - q_2 - x)  e^{-i(xp + kq)} H(p,q) \\
            &= \int \frac{\mathrm{d}k}{2\pi} \int \mathrm{d}p \int \mathrm{d}q \, e^{\frac{i}{2} k (q_1 + q_2)} e^{i((q_2 - q_1)p - kq)} H(p,q) \\
            &= \int \frac{\mathrm{d}p}{2\pi} \int \mathrm{d}q\, H(p,q) e^{i(q_2 - q_1)p} \int \mathrm{d}k \, e^{i k (\frac{q_1 + q_2}{2} - q)} \\
            &= \int \frac{\mathrm{d}p}{2 \pi} \, e^{i(q_2 - q_1)p} \int \mathrm{d}q \, H(p,q) \, \delta \left(q - \frac{q_1 + q_2}{2}\right) \\ 
            &= \int \frac{\mathrm{d}p}{2\pi} \, e^{i(q_2 - q_1)p} H\left(p, \frac{q_1 + q_2}{2}\right).
        \end{split}
    \end{equation}

    Within an infinitesmal time interval, we have $ e^{-i H(\hat{p}, \hat{q}) \delta t} \approx 1 - i H(\hat{p}, \hat{q}) \delta t $. Thus
    \begin{equation}
        \begin{split}
            \langle q_2 \vert e^{-i H(\hat{p}, \hat{q}) \delta t} \vert q_1 \rangle &= \langle 1 - i \delta t H(\hat{p} , \hat{q}) + O(\delta t^2)\rangle \\
            &= \delta (q_2 - q_1) - i \delta t \langle q_2 \vert H(\hat{p}, \hat{q}) \vert q_1 \rangle + O(\delta t^2) \\
            &= \int \frac{\mathrm{d}}{2\pi} \, e^{i(q_2 - q_1)p} \left(1 - i \delta t H\left(p, \frac{q_1 + q_2}{2}\right) + O(\delta t^2)\right) \\
            &= \int \frac{\mathrm{d}p}{2\pi} \, e^{i(q_2 - q_1)p} e^{-i \delta t H\left(p, \frac{q_1 + q_2}{2}\right)}.
        \end{split}
    \end{equation}
    In the last step, we used the linear approximation of the exponential function again.
    \\ \\
    (b) Do a Legendre transformation to get 
    \begin{equation}
        L = \frac{1}{2} f(q) \dot{q}^2 \longrightarrow p = f(q) \dot{q}, \quad H = \frac{p^2}{2 f(q)}.
    \end{equation}
    From (a): Let $ \varepsilon = (t_{f} - t_{i}) / (N+1) $, $ \overline{q}_n = \frac{1}{2} (q_n + q_{n+1}) $, and $ \Delta q_n = q_{n+1} - q_n $, then the infinitesmal transition amplitude given the Weyl-ordered Hamiltonian is
    \begin{equation}
        \begin{split}
            \langle q_{n+1}, t_{n+1} \vert q_n, t_n \rangle &= \int \frac{\mathrm{d}p}{2\pi} \, \exp \left(i p \Delta q_n - i \varepsilon H(p, \overline{q}_n)\right) \\
            &= \int \frac{\mathrm{d}p}{2\pi} \, \exp \left(i p \Delta q_n - i \varepsilon \frac{p^2}{2 f(\overline{q}_n)}\right) \\
            &= \sqrt{\frac{f(\overline{q}_n)}{2 \pi i \varepsilon}} \exp \left(i \frac{f(\overline{q}_n)}{2 \varepsilon} (\Delta q_n)^2 \right).
        \end{split}
    \end{equation}
    Then, we can build the total transition amplitude as
    \begin{equation}
        \begin{split}
            \langle q_f, t_f \vert q_i, t_i \rangle &= \lim_{N \to \infty} \prod_n \left(\int \mathrm{d}q_n \, \sqrt{\frac{f(\overline{q}_n)}{2\pi i \varepsilon}} \right) \exp \left[i \sum_{n} \frac{1}{2}\varepsilon f(\overline{q}_n) \left(\frac{\Delta q_n}{\varepsilon}\right)^2 \right] \\
            &= \left[\lim_{N \to \infty} \prod_n \int \mathrm{d}q_n \, \sqrt{\frac{f(\overline{q}_n)}{2\pi i \varepsilon}} \right] \, \exp \left[i \int_{t_i}^{t_f} \mathrm{d}t \, L(q, \dot{q}, t) \right].
        \end{split}
    \end{equation}
    However, the measure still contains a non-trivial factor, which we shall absorb into the exponential. As mentioned in Sakurai and Napolitano (2008), the leading weight factor in the measure should be independent of $ V(x) $, and hence $ L(x) $. Consider the following expansion: 
    \begin{equation}
        \begin{split}
            \ln f(\overline{q}_n) &= \ln f \left(q_n + \frac{\Delta q_n}{2}\right) \\
            &= \ln f(q_n) + \frac{\Delta q_n}{2} \frac{f'(q_n)}{f(q_n)} + \frac{(\Delta q_n)^2}{8} \left[\frac{f''(q_n)}{f(q_n)} - \left(\frac{f'(q_n)}{f(q_n)}\right)^2\right] + O((\Delta q_n)^3).
        \end{split}
    \end{equation}
    Then, ignoring terms of order higher than $ O(\Delta q_n^3) $, we have
    \begin{equation}
        \begin{split}
            &\lim_{N \to \infty} \prod_n \int \mathrm{d}q_n \, \sqrt{\frac{f(\overline{q}_n)}{2\pi i \varepsilon}} = \lim_{N \to \infty} \prod_n \int \mathrm{d}q_n \, \lim_{N \to \infty} \prod_n \sqrt{\frac{f(\overline{q}_n)}{2\pi i \varepsilon}} \\
            &= \lim_{N \to \infty} \prod_n \int \frac{\mathrm{d}q_n}{\sqrt{2\pi i \varepsilon}} \, \exp \left\{ \frac{1}{2} \sum_{n} \left[\ln f(q_n) + \frac{\Delta q_n}{2} \frac{f'(q_n)}{f(q_n)} + \frac{\Delta q_n^2}{8}\left(\frac{f''(q_n)}{f(q_n)} - \frac{f'(q_n)^2}{f(q_n)^2} \right) \right] \right\} \\
            &= \lim_{N \to \infty} \prod_n \int \frac{\mathrm{d}q_n}{\sqrt{2\pi i \varepsilon}} \, \exp \left[ \frac{1}{4} \sum_{n} \ln f(q_n) \right] \exp \left[\frac{1}{2} \int \mathrm{d}q \, \frac{\partial}{\partial q} \ln L \left(q, \dot{q}, t\right) \right] \\
            & \quad \quad \quad \times \exp \left\{ \frac{(\Delta q_n)^2}{16} \left[\frac{f''(q_n)}{f(q_n)} - \frac{f^{\prime} (q_n)^2}{f(q_n)^2} \right] \right\} \\
            &= \tilde{N} \int \mathcal{D} q \, \exp \left\{ i \int^{t_f}_{t_i} \mathrm{d}t\, \frac{1}{16} \left(\frac{f''(q)}{f(q)^2} - \frac{f^{\prime} (q)^2}{f(q)^3}\right) \right\},
        \end{split}
    \end{equation}
    where we absorbed the leading term into a normalization constant \( \tilde{N} \), left out the boundary term corresponding $ \Delta q $ that does not affect the action, and used the identity
    \begin{equation}
        \lim_{N \to \infty} \left(\Delta q_n\right)^2 = \frac{i \varepsilon}{f(q)} \to \frac{i \mathrm{d}t}{f(q)}, 
    \end{equation}
    where we identify $ \varepsilon $ with $ \mathrm{d}t $. Therefore, the effective Lagrangian is given by 
    \begin{equation}
        L_{\text{eff}} = \frac{1}{2} f(q) \dot{q}^2 + \frac{1}{16} \left(\frac{f''(q)}{f(q)^2} - \frac{f^{\prime} (q)^2}{f(q)^3}\right) = L + \frac{1}{16} \left(\frac{f''(q)}{f(q)^2} - \frac{f^{\prime} (q)^2}{f(q)^3}\right),
    \end{equation} 
    which vanishes when $ f = \text{const.} $ as expected. 
\end{solution}
 
\newpage

% Problem 4
4.~\textbf{Path integral for the EM field} [\textcolor{red}{20 pts}]  \\ 
The free photon propagator $\Delta^{\mu\nu}(x-y)=\langle 0| TA^{\mu}(x)A^{\nu}(y)|0\rangle$ in the Coulomb gauge $\nabla\cdot {\bf{A}}=0$ is given by, in momentum space, 
\begin{align}
\label{propagator_Coulomb}
\tilde{\Delta}^{00}(k) = \frac{i}{{\bf{k}}^2},\qquad \tilde{\Delta}^{i0}(k) = \tilde{\Delta}^{0i}(k) = 0,\qquad
\tilde{\Delta}^{ij}(k) = \frac{i}{k_0^2-{\bf{k}}^2}\cdot \left(\delta^{ij} - \frac{k^i k^j}{{\bf{k}}^2}\right).
\end{align} 
Let's derive this in the following two ways: \\
(a) Use the fact that $-i\Delta^{\mu\nu}(x-y)$ is a Green's function of the Maxwell's equations for $A^{\mu}(x)$, i.e., 
\begin{align}
\partial^2A^{\nu}(x) - \partial^{\nu}(\partial_{\mu}A^{\mu}(x)) = J^{\nu}(x),\qquad
A^{\mu} (x) = \int d^4y\, (-i) \Delta^{\mu\nu}(x-y) J_{\nu}(y),
\end{align} 
to derive the explicit form of $\tilde{\Delta}^{\mu\nu}(k)$ under the condition $\nabla\cdot {\bf{A}}=0$. 
\\ \\
(b) Follow the discussion of the Faddeev-Popov method for deriving the photon propagator in my lecture note \textcolor{OliveGreen}{[Note-20250923]}. Instead of taking the gauge function $G = \partial_{\mu}A^{\mu}(x) -\sigma(x)$, here we choose $G = \nabla\cdot {\bf{A}}(x) -\sigma(x)$. Show that, with the same weighting function $F[\sigma] = \exp{-\frac{i}{2\xi}\int d^4x\, \sigma^2}$ and in the limit $\xi\rightarrow 0$,  the photon propagator obtained using the path integral formulation is exactly given by~\eqref{propagator_Coulomb}.

\begin{solution}
    ~

    (a) Fourier transforming to momentum space, we have 
    \begin{equation}
        (-k_0^2 + \vert \bvec{k} \vert^2 ) \tilde{A}^{\nu} + k^{\nu} k_{\mu} \tilde{A}^\mu = \tilde{J}^{\nu} (k),
    \end{equation}
    and the propagator satisfies 
    \begin{equation}
        \left(-k^2 g^{\mu}_{\phantom{\mu} \lambda} + k^{\mu} k_{\lambda} \right) \tilde{\Delta}^{\lambda \nu} (k) = i g^{\mu \nu}
    \end{equation}
    as a Green function. Impose the Coulomb gauge $ \partial_\mu \tilde{A}^{\mu} = \partial_0 \tilde{A}^0 $. Then, 
    \begin{equation}
        \left(- k_0^2 + \vert \bvec{k} \vert^2 \right) \tilde{A}^{\nu} + k^{\nu} k_0 \tilde{A}^0 = \tilde{J}^{\nu} (k).
    \end{equation}
    \begin{itemize}
        \item $ \nu = 0 $: We have
        \begin{equation}
            \left(- k_0^2 + \vert \bvec{k} \vert^2 \right) \tilde{A}^{0} + k^{0} k_0 \tilde{A}^0 = \tilde{J}^{0} (k) \Longrightarrow \vert \bvec{k} \vert^2 \tilde{A}^0 = \tilde{J}^0, 
        \end{equation}
        so $ \tilde{A}^0 = \tilde{J}^0 / \vert \bvec{k} \vert^2 $. Comparing this with $ \tilde{A}^{\mu} = -i \tilde{\Delta}^{\mu \nu } \tilde{J}_\nu $, we get 
        \begin{equation}
            \tilde{\Delta}^{00} (k) = \frac{i}{\vert \bvec{k} \vert^2}.
        \end{equation} 
        \item $ \nu = i $: We have
        \begin{equation}
            \left(- k_0^2 + \vert \bvec{k} \vert^2 \right) \tilde{A}^{i} + k^{i} k_0 \tilde{A}^0 = \tilde{J}^{i} (k) \Longrightarrow \tilde{A}^{i} = \frac{\tilde{J}^{i} (k) - k^{i} k_0 \tilde{A}^0}{- k_0^2 + \vert \bvec{k} \vert^2}.
        \end{equation}
        Substitute $ \tilde{A}^0 $ into the above, we have
        \begin{equation}
            \tilde{A}^{i} = \frac{\tilde{J}^{i} (k) - \frac{k^{i} k_0}{\vert \bvec{k} \vert^2} \tilde{J}^0}{- k_0^2 + \vert \bvec{k} \vert^2} = -i \tilde{\Delta }^{i \nu } \tilde{J}_\nu .
        \end{equation}
        Set $ \nu = j $, we have 
        \begin{equation}
            \tilde{\Delta }^{ij} (k) \tilde{J}_j = \frac{i}{k_0^2 - \vert \bvec{k} \vert^2} \left(\delta^{ij} + \frac{k^i k^j}{\vert \bvec{k} \vert^2 } \tilde{J}_j\right).
        \end{equation}  
        Since this holds for arbitrary $ \tilde{J}_j $, using previous derived relations, we also have
        \begin{equation}
            \tilde{\Delta }^{0j} = 0 = \tilde{\Delta}^{i0}. 
        \end{equation}
    \end{itemize}
    Therefore, we have proven the desired result.

    (b) Consider the gauge function $ G(A^{\mu}) = \nabla \cdot \bvec{A} - \sigma (x) = \partial_\mu A^\mu + \partial_0 A^0 - \sigma (x) $. The Faddeev-Popov determinant is
    \begin{equation}
        \Delta_{FP} [A^{\mu}] = \det \left(\frac{\delta G(A^{\mu \alpha})}{\delta \alpha} \right) = \det (\nabla^2),
    \end{equation}
    which is independent of the gauge field $ A^{\mu} $. Then, the path integral is 
    \begin{equation}
        \begin{split}
            Z &= \int \mathcal{D} \sigma \, e^{-\frac{i}{2 \xi} \int \mathrm{d}^4 x\, \sigma ^2} \int \mathcal{D} A^{\mu} \, e^{i S[A]} \delta \left(\nabla \cdot \bvec{A} - \sigma (x)\right) \\
            &= \int \mathcal{D} A^{\mu} \, e^{i S[A]} \exp \left[- \frac{i}{2 \xi} \int \mathrm{d}^4 x\, (\nabla \cdot \bvec{A})^2 \right],
        \end{split}
    \end{equation}
    and we may write the effective action as
    \begin{equation}
        \begin{split}
            S_{\text{eff}} &= S[A] - \frac{1}{2 \xi} \int \bvec{d}^4 x\, (\nabla \cdot \bvec{A})^2 \\
            &= \int \mathrm{d}^4 x \, \left\{ \frac{1}{2} A_\mu \left[ \partial^2 g^{\mu \nu} - \partial^\mu \partial^{\nu} \right] A_{\nu} + A_\mu J^{\mu} - \frac{1}{2\xi} \left(\nabla \cdot \bvec{A}\right)^2\right\} \\
            &= \frac{1}{2} \int \mathrm{d}^4 x\, \left\{ A_{\mu} \left[\partial^2 g^{\mu \nu} - \partial^{\mu} \partial^{\nu} + \frac{1}{\xi} \delta^{\mu i} \delta^{\nu j} \partial_i \partial_j \right] A_{\nu} + A_{\mu} J^{\mu} \right\}.
        \end{split}
    \end{equation}

    Let's consider the term in front of $ A_\nu $ and Fourier transform to momentum space. Let $ \tilde{O} $ be an operator such that 
    \begin{equation}
        \tilde{O}^{\mu \nu }(k) = - k^2 g^{\mu \nu} + k^{\mu} k^{\nu} - \frac{1}{\xi} \delta^{\mu i} \delta^{\nu j} k_i k_j.
    \end{equation}

    Then in the basis $ (\mu, \nu) = (0, i) $, we have
    \begin{equation}
        \begin{split}
            O &\sim \begin{pmatrix}
                \bvec{k}^2 & k^0 k^j \\
                k^i k^0 & (k^0)^2 - \vert \bvec{k} \vert^2 + \left(1+ \frac{1}{\xi}\right) k^i k^j \\
            \end{pmatrix} \\
            &= \begin{pmatrix}
                \bvec{k}^2 & k^0 k^j \\
            k^i k^0 & k^2 + \left(1+ \frac{1}{\xi}\right) k^i k^j \\
            \end{pmatrix}
        \end{split}
    \end{equation}
    Then $ O^{\mu \lambda} \tilde{\Delta}_{\lambda \nu} = i \delta^{\mu}_{\phantom{\mu} \nu} $. Act $ \tilde{O} $ on a transverse vector $ A_T^{\nu} $ such that $ k_{\nu} A_T^{\nu} = 0 $, we have
    \begin{equation}
        O^{ij} A_{T,j} = k^2 A_{T}^i \Longrightarrow \left(\tilde{\Delta}\right)^{ij} = i \frac{1}{k^2} P^{ij}_T, 
    \end{equation}
    where the transverse projection operator is defined as
    \begin{equation}
        P_T^{ij} = \left(\delta^{ij} - \frac{k^i k^j}{\vert \bvec{k} \vert^2}\right)
    \end{equation}
    Hence, 
    \begin{equation}
        \tilde{\Delta}_T^{ij} (k) = \frac{i}{k_0^2 - \vert \bvec{k} \vert^2} \left(\delta^{ij} - \frac{k^i k^j}{\vert \bvec{k} \vert^2}\right).
    \end{equation}

    Now we would like to discuss the longitudinal components. Define $ \tilde{M} $, then 
    \begin{equation}
        \begin{split}
            \tilde{M}^{\mu \nu} (k) &= - k^2 g^{\mu \nu} + \left(1- \frac{1}{\xi}\right) k^{\mu} k^{\nu} \\
            &\sim \begin{pmatrix}
                |\bvec{k}|^2 & k^0 |\bvec{k}| \\
                k^0 |\bvec{k}| & (k^0)^2 - \frac{1}{\xi} |\bvec{k}|^2
            \end{pmatrix}
        \end{split}
    \end{equation}
    in the basis $ (\mu, \nu) = (0, i) $. The Green's function equation now satisfies $ \tilde{M}^{\mu \lambda} \tilde{\Delta}_{\lambda \nu } = i \delta^{\mu}_{\phantom{\mu} \nu} $. 
    Since $ \operatorname{det} (\tilde{M}) = - |\bvec{k}|^4 / \xi $ and $ \tilde{\Delta} = i\tilde{M}^{-1} $, we have 
    \begin{equation}
        \begin{split}
            \tilde{\Delta}^{\mu \nu}_L (k) &\sim \begin{pmatrix}
                \tilde{\Delta}^{00} & \tilde{\Delta}^{0 \ell } \\
                \tilde{\Delta}^{\ell 0} & \tilde{\Delta}^{\ell \ell} 
            \end{pmatrix} \\
            &= \frac{i}{\operatorname{det} (\tilde{M})} \begin{pmatrix}
                (k^0)^2 - \frac{1}{\xi} |\bvec{k}|^2 & - k^0 |\bvec{k}| \\
                - k^0 |\bvec{k}| & |\bvec{k}|^2 
            \end{pmatrix} \\
            &= \frac{i \xi}{|\bvec{k}|^4} \begin{pmatrix}
                (k^0)^2 - \frac{1}{\xi} |\bvec{k}|^2 & - k^0 |\bvec{k}| \\
                - k^0 |\bvec{k}| & |\bvec{k}|^2 
            \end{pmatrix}.
        \end{split}
    \end{equation}

    So, we find that the propagator is give by $ \tilde{\Delta} = \tilde{\Delta}_T + \tilde{\Delta}_L $: 
    \begin{equation}
        \begin{split}
            \left(\tilde{\Delta}\right)^{00} = i \frac{\xi (k^0)^2 - |\bvec{k}|^2}{|\bvec{k}|^4} &\quad \xrightarrow[]{\xi \to 0} \quad \frac{i}{|\bvec{k}|^2}, \\
            \left(\tilde{\Delta}\right)^{0j} = \left(\tilde{\Delta}\right)^{i0} = - i \frac{\xi k^0 k^i}{|\bvec{k}|^4} &\quad \xrightarrow[]{\xi \to 0} \quad 0 , \\
            \left(\tilde{\Delta}\right)^{ij} = i \frac{\xi \delta^{ij}}{|\bvec{k}|^2} + \frac{i}{k_0^2 - |\bvec{k}|^2} \left(\delta^{ij} - \frac{k^i k^j}{|\bvec{k}|^2}\right) &\quad \xrightarrow[]{\xi \to 0} \quad \frac{i}{k_0^2 - |\bvec{k}|^2} \left(\delta^{ij} - \frac{k^i k^j}{|\bvec{k}|^2}\right). 
        \end{split}
    \end{equation}
\end{solution}

\newpage 

% Problem 5
5.~\textbf{Asymptotic expansions of ordinary integrals} [\textcolor{red}{30 pts}]  \\ 
The integral 
\begin{align}
    \label{q4}
    F(\lambda) = \frac{1}{\sqrt{2\pi}}\int_{-\infty}^{\infty}dq~e^{-\frac{1}{2}q^2-\frac{\lambda}{4!}q^4}
\end{align} 
has an analytic form that involves the modified Bessel function. Nevertheless, let's express this integral as a power series in $\lambda$ by Taylor expanding $\exp{-\frac{\lambda}{4!}q^4}$ and discuss some properties of this series. \\
(a) So we write
\begin{equation}
    \label{expandF}
    F(\lambda) = \sum_{n=0}^{\infty} C_n\lambda^n.
\end{equation} 
Find the coefficient $C_n$ for each $n$. \\ \\
(b) Explain why the magnitude of the summand in~\eqref{expandF} diverges when $n\rightarrow\infty$, no matter how small $\lambda$ is (except for $\lambda=0$). For finite  $n$, estimate the value of $n = n_*(\lambda)$ that minimizes the summand.  \\ \\
(c) Define the relative error
\begin{equation}
    \label{errorF}
    S_N(\lambda) = 1-\frac{\sum_{n=0}^{N}C_n\lambda^n}{F(\lambda)}
\end{equation} 
from the ($N$+$1$)-term approximation for $F(\lambda)$. Use a mathematical tool (e.g.~Wolfram Mathematica) to numerically evaluate $F(\lambda)$ and $S_N(\lambda)$ and plot ``$|S_N(\lambda)|$ versus $N$" (better on a log scale) for several values of $\lambda$. What do you find? Try $\lambda=0.05,\ 0.2,\ 1,\ 10$ with $0\leq N\leq 50$. For these values of $\lambda$, compare $n_*(\lambda)$ in part~(b) with the value of $n$ where $|S_N(\lambda)|$ reaches a minimum in your plot. \\ \\
(d)  Use a diagrammatic expansion to express $F(\lambda)$ by assigning appropriate factors to each vertex and each line in a diagram, as well as assigning a symmetry factor to each diagram. Draw all diagrams, including both connected and disconnected diagrams, up to order $\lambda^3$, and identify the symmetry factor of each diagram. Confirm that summing diagrams of the same order in $\lambda$ yields exactly the coefficients $C_1$, $C_2$, and $C_3$ in the series~\eqref{expandF}. \\ \\ 
(e) Repeat parts~(a), (b), (c), and (d) for another integral
\begin{equation}
    \label{q6}
    G(\lambda) = \frac{1}{\sqrt{2\pi}}\int_{-\infty}^{\infty}dq~e^{-\frac{1}{2}q^2-\frac{\lambda}{6!}q^6}.
\end{equation} 
For part~(d) in this problem, you just need to do the diagrammatic expansion of $G(\lambda)$ up to order $\lambda^2$. 
\\ \\ \\

\begin{solution}
    ~

    (a) The integrand can be written as $ e^{-\frac{1}{2}q^2} e^{-\frac{\lambda}{4!}q^4} $. Since $ e^{-\frac{\lambda}{4!}q^4} $ is analytic, it admits a Taylor expansion: 
    \begin{equation}
        \begin{split}
            F(\lambda) &= \frac{1}{\sqrt{2 \pi}} \int^{\infty}_{-\infty} \mathrm{d}q \, \sum_{n=1}^{\infty} \frac{(-1)^n}{n!} \left(\frac{\lambda}{4!}\right)^n q^{4n} e^{-\frac{1}{2}q^2} \\ 
            &= \frac{1}{\sqrt{2 \pi}} \sum_{n=1}^{\infty} \frac{(-1)^n}{n!} \left(\frac{\lambda}{4!}\right)^n \int^{\infty}_{-\infty} \mathrm{d}q \, q^{4n} e^{-\frac{1}{2}q^2}, 
        \end{split}
    \end{equation} 
    where we interchanged summation and integration due to absolute convergence. With the Gaussian integral formula 
    \begin{equation}
        \label{eq:gaussian}
        \int^{\infty}_{-\infty} \mathrm{d}q \, q^{2n} e^{-\frac{1}{2}q^2} = \sqrt{2 \pi} (2n-1)!! = \sqrt{2 \pi} \frac{(2n)!}{2^n n!},
    \end{equation}
    the coefficients $C_n$ can be read off as
    \begin{equation}
        C_n = \frac{(-1)^n (4n)!}{96^n n! (2n)!}.
    \end{equation}

    (b) We use the ratio test to assess convergence: 
    \begin{equation}
        \left\vert \frac{C_{n+1} \lambda^{n+1}}{C_n \lambda^n} \right\vert = \left\vert \frac{1}{4!} \frac{(4n+1)(4n+3)}{n+1} \lambda \right\vert .
    \end{equation}
    Therefore the series has zero radius of convergence. For finite $n$, the magnitude of the summand is minimized when
    \begin{equation}
        n = n_{\ast} (\lambda) = \left\lfloor \frac{3}{2\vert \lambda \vert} \right\rfloor .
    \end{equation}
    We can see this by applying Stirling's approximation to $ C_n $: 
    \begin{equation}
        \begin{split}
            \left\vert C_n \lambda^n \right\vert &\sim \frac{1}{96^n} \frac{\sqrt{8\pi n} (4n)^{4n} e^{-4n}}{\sqrt{2\pi n} n^n e^{-n} \sqrt{4\pi n} (2n)^{2n} e^{-2n}} \vert \lambda \vert^n \\
            &\sim \frac{1}{\sqrt{\pi n}} \left(\frac{2n \vert \lambda \vert}{3}\right)^n e^{-n} \left(1 + O\left[n^{-1}\right]\right) \\ 
            &\sim \frac{1}{\sqrt{\pi n}} \left(\frac{2n \vert \lambda \vert}{3}\right)^n e^{-n} . 
        \end{split}
    \end{equation}
    Treat $ n $ as a continuous variable, then 
    \begin{equation}
        \frac{\mathrm{d}}{\mathrm{d}n} \log \left\vert C_n \lambda^n \right\vert = 0 \;\Longrightarrow\; n_{\ast} (\lambda) = \frac{3}{2\vert \lambda \vert} e^{O(n^{-1})} \sim \frac{3}{2\vert \lambda \vert }. 
    \end{equation}
    The largest integer less than $ n_{\ast} $ is taken since $ n $ is discrete.
    \\ \\

    (c) The numerical results are shown in Fig.~\ref{fig:plot1}, and the results are organized in Table~\ref{tab:relative1}, which shows good agreement.
    \begin{figure}
        \centering
        \includegraphics[width=0.8\linewidth]{plot1.png}
        \caption{The relative error $ \vert S_N (\lambda) \vert $ versus $ N $ for various $ \lambda $. A $ \log \log $ scale is used for better visualization. The vertical dashed lines indicate $ n_{\ast} (\lambda) = \lfloor 3/(2\vert \lambda \vert) \rfloor $ for each $ \lambda $.}
        \label{fig:plot1}
    \end{figure}

    \begin{table}[h]
        \centering
        \caption{Minimum index \(N\) where \(|S_N(\lambda)|\) is smallest, compared with \(\lfloor n_*(\lambda)\rfloor\).}
        \label{tab:relative1}
        \begin{tabular}{ccc}
        \toprule
        \(\lambda\) & \(\arg\min_N |S_N(\lambda)|\) & \(\lfloor n_*(\lambda)\rfloor\) \\
        \midrule
        0.05 & 30 & \textbf{30} \\
        0.2  & 7  & \textbf{7}  \\
        1    & 1  & \textbf{1}  \\
        10   & 0  & \textbf{0}  \\
        \bottomrule
        \end{tabular}
    \end{table}

    (d) The first three non-unity expansion coefficients are 
    \begin{equation}
        C_1 = -\frac{1}{8}, \quad C_2 = \frac{35}{384}, \quad C_3 = - \frac{385}{49152}.
    \end{equation}
    When computing an $ n $-point function, the factor to be placed in front of each diagram is given by 
    \begin{equation}
        \frac{C}{V! (4!)^V} = \frac{1}{\mathcal{D}},
    \end{equation} 
    where $ V $ is the number of vertices, $ C $ is the multiplicity, or number of contractions according to Wick's theorem, and $ \mathcal{D} $ is the symmetry factor. The diagrams up to order \( \lambda^3 \) are shown in Fig.~\ref{fig:diagrams_phi4} along with their multiplicities. For each vertex, we associate with it a factor of $ -\frac{\lambda}{4!} $. 
    \begin{equation}
        \begin{split}
            \lambda^1 \text{: } & \quad C_1 = -\frac{1}{4!} \cdot 3 = -\frac{1}{8}, \\
            \lambda^2 \text{: } & \quad C_2 = \frac{1}{2!} \frac{1}{(4!)^2} \left(4! + 3^2 \cdot 2^3 + 3^2\right) = \frac{35}{384}, \\
            \lambda^3 \text{: } & \quad C_3 = -\frac{1}{3!} \frac{1}{(4!)^3} \left( 2^6 \cdot 3^3 + 2^7 \cdot 3^3 + 2^5 \cdot 3^4 + 2^6 \cdot 3^3 + 2^3 \cdot 3^3 + 2^3 \cdot 3^4 + 3^3 \right) = - \frac{385}{3072}.
        \end{split}
    \end{equation}
    In the above calculation, the symmetry factors of disconnected diagrams are computed by multiplying those of each connected component, and tagging on a $ n! $ factor for identical components. The results agree with the coefficients obtained in part (a).

    (e-1) As above, by Taylor expansion we have 
    \begin{equation}
        G(\lambda) = \frac{1}{\sqrt{2 \pi}} \sum_{n=1}^{\infty} \frac{(-1)^n}{n!}\left(\frac{\lambda}{6!}\right)^n \int^{\infty}_{-\infty} \mathrm{d}q \, q^{6n} e^{-\frac{1}{2}q^2}.
    \end{equation}
    Using equation (\ref{eq:gaussian}), the coefficients \(C_n\) are given by
    \begin{equation}
        D_n = \frac{(-1)^n (6n)!}{5760^n n! (3n)!}. 
    \end{equation}
    (e-2) By the ratio test, the series also has zero radius of convergence: 
    \begin{equation}
        \left\vert \frac{D_{n+1} \lambda^{n+1}}{D_n \lambda^n} \right\vert = \frac{1}{6!} \frac{(6n+1)(6n+3)(6n+5)}{n+1} \vert \lambda \vert .
    \end{equation}
    The magnitude of the summand is minimized when
    \begin{equation}
        n_{\ast} (\lambda) = \left\lfloor \sqrt{\frac{10}{3 \vert \lambda \vert}} \right\rfloor .
    \end{equation}
    We can see this by applying Stirling's approximation to \( D_n \):
    \begin{equation}
        \begin{split}
            \left\vert D_n \lambda^n \right\vert &\sim \frac{1}{5760^n} \frac{\sqrt{12\pi n} (6n)^{6n} e^{-6n}}{\sqrt{2\pi n} n^n e^{-n} \sqrt{6\pi n} (3n)^{3n} e^{-3n}} \vert \lambda \vert^n \\
            &\sim \frac{1}{\sqrt{\pi n}} \left(\frac{3 n^{2} \vert \lambda \vert }{10 e^2}\right)^n \left[1 + O\left(\frac{1}{n}\right)\right] \\
            &\sim \frac{1}{\sqrt{\pi n}} \left(\frac{3 n^2 \vert \lambda \vert}{10 e^2}\right).
        \end{split}
    \end{equation}
    Treating \( n \) as a continuous variable, then
    \begin{equation}
        \frac{\mathrm{d}}{\mathrm{d}n} \log \left\vert D_n \lambda^n \right\vert = 0 \;\Longrightarrow\; n_{\ast} (\lambda) = \sqrt{\frac{10}{3 \vert \lambda \vert}} e^{O(n^{-1})} \sim \sqrt{\frac{10}{3 \vert \lambda \vert}}.
    \end{equation}

    (e-3) The numerical results are shown in Fig.~\ref{fig:plot2}, and the results are organized in Table~\ref{tab:relative2}, which shows good agreement again.
    \begin{figure}
        \centering
        \includegraphics[width=0.9\linewidth]{plot2.png}
        \caption{The relative error $ \vert S_N (\lambda) \vert $ versus $ N $ for various $ \lambda $. A $ \log \log $ scale is used for better visualization. The vertical dashed lines indicate $ n_{\ast} (\lambda) = \lfloor 3/(2\vert \lambda \vert) \rfloor $ for each $ \lambda $.}
        \label{fig:plot2}
    \end{figure}

    \begin{table}[h]
        \centering
        \caption{Minimum index \(N\) where \(|S_N(\lambda)|\) is smallest, compared with \(\lfloor n_*(\lambda)\rfloor\).}
        \label{tab:relative2}
        \begin{tabular}{ccc}
        \toprule
        \(\lambda \times\) & \(\arg\min_N |S_N(\lambda)|\) & \(\lfloor n_*(\lambda)\rfloor\) \\
        \midrule
        0.05 & 8 & \textbf{8} \\
        0.2  & 4  & \textbf{4}  \\
        1    & 2  & 1 \\
        10   & 0  & \textbf{0}  \\
        \bottomrule
        \end{tabular}
    \end{table}

    (e-4) The first two non-unity expansion coefficients are 
    \begin{equation}
        D_1 = -\frac{1}{48}, \quad D_2 = \frac{77}{7680}.
    \end{equation}
    For $ \lambda^2 $ diagrams, let $ z = \# \left(\text{lines connecting two vertices}\right) $, then $ z $ must be even, and $ x = y = (6-z)/2 $ is the number of "tadpoles" on each vertex.  
    The diagrams up to order \( \lambda^3 \) are shown in Fig.~\ref{fig:diagrams_phi4} along with their multiplicities. For each vertex, we associate with it a factor of $ -\frac{\lambda}{6!} $. Then we have four diagrams corresponding to $ z = 0, 2, 4, 6 $.
 
    \begin{equation}
        \begin{split}
            \lambda^1 \text{: } & \quad D_1 = -\frac{1}{6!} \cdot {6 \choose 2} = -\frac{1}{48}, \\
            \lambda^2 \text{: } & \quad D_2 = \frac{1}{2!} \frac{1}{(6!)^2} \left(225 + 4050 + 5400 + 6!\right)= \frac{77}{7680}.
        \end{split}
    \end{equation}
    The results agree with the coefficients obtained in part (a).

    \newpage
    \begin{figure}
        \centering 
        \includegraphics[width=0.9\linewidth]{diagrams_phi4.jpg}
        \caption{Diagrams up to order \( \lambda^3 \) in \( \phi^4 \) theory, and up to order $ \lambda^2 $ in $ \phi^6 $ theory.}
        \label{fig:diagrams_phi4}
    \end{figure}

    \begin{remark}
        The high-precision MATLAB code used to generate the numerical results in part (c) is given below, while the code for part (e) is similar and thus omitted.
        \vspace{1em}
        \begin{lstlisting}[style=matlab, caption={Code to plot $|S_N(\lambda)|$ vs.~$N$}, label={lst:sn}]
clear; clc; digits(100);

Nmax    = 50;
lambdas = [0.05, 0.2, 1, 10];
Ns      = 0:Nmax;

f  = figure('Color','w');
ax = axes('Parent',f,'Color','w');
hold(ax,'on'); grid(ax,'on');
set(ax,'XColor','k','YColor','k'); 
set(f,'InvertHardcopy','off');  
title(ax,'|S_N(\lambda)| vs N','Color','k');
xlabel(ax,'N','Color','k');
ylabel(ax,'|S_N(\lambda)|','Color','k');

C = sym(zeros(1, Nmax+1));   % Precompute C_n
for n = 0:Nmax
    C(n+1) = (-1)^n * factorial(sym(4*n)) / ...
             (sym(96)^n * factorial(sym(n)) * factorial(sym(2*n)));
end

for lam = lambdas
    fint = @(q) exp(-0.5*q.^2 - lam*(q.^4)/24)/sqrt(2*pi);
    Fval = 2*integral(fint, 0, Inf, 'RelTol',1e-13, 'AbsTol',1e-15);
    Fsym = vpa(Fval);   % F(lambda)

    lamSym  = vpa(lam);
    Sseries = cumsum( C .* (lamSym).^Ns );
    SNvpa   = abs( vpa(1) - Sseries./Fsym );   % Relative error |S_N|
    
    lo = vpa(realmin('double'));   
    hi = vpa(realmax('double')); 
    SNvpa = min(max(SNvpa, lo), hi);
    SN    = double(log(SNvpa));

    % Plot
    p = semilogy(Ns, SN, 'o-', 'LineWidth', 1.3, ...
        'DisplayName', sprintf('\\lambda = %.2g', lam));
    drawnow;
    
    nstar_floor = floor(3/(2*lam));   % floor(n*)
    clr = get(p, 'Color');
    lbl = sprintf(' floor(n*) = %d', nstar_floor);
    xline(nstar_floor, '-', lbl, ...
        'Color', clr, 'LineWidth', 1.2, 'HandleVisibility','off');

    [~, idxMin] = min(SN);
    fprintf('lambda = %.3g: min |S_N| at N = %d, floor(n*) = %d\n', ...
            lam, Ns(idxMin), nstar_floor);   % report minimum
end

legend(ax,'Location','best','TextColor','k','Color','w','EdgeColor',[.8 .8 .8]);
        \end{lstlisting}
    \end{remark}
\end{solution}

\end{CJK}
\end{document}

