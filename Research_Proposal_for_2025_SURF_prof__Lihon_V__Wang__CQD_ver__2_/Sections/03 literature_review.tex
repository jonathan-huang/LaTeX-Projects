\subsection{Co-Quantum Dynamics}

Co-Quantum Dynamics (CQD) \textcolor{blue}{\hyperref[sec:ref]{[1]}\hyperref[sec:ref]{[2]}\hyperref[sec:ref]{[3]}\hyperref[sec:ref]{[4]}} is a semiclassical model that describes atomic spin dynamics and wave function collapse. Currently, the model succeeds in providing a precise description of the spin-flip phenomena in the multi-stage Stern–Gerlach experiment, in particular the Frisch-Segr\`e experiment of 1933 \textcolor{blue}{\hyperref[sec:ref]{[5]}}, without fitting parameters.

CQD models the interaction between electron and nuclear magnetic moments ($\mu_\text{e}$ and $\mu_\text{n}$), treating $\mu_\text{n}$ as a "co-quantum" which guides the collapse of $\mu_e$, but does not itself collapse upon measurement of atomic spin. 

Figure 1(a) illustrates the apparatus used in the Frisch-Segr\`e experiment. As an atom traverses the central magnetic field, the model incorporates induction effects and torque-averaged magnetic dynamics instead of traditional self-averaging to describe its evolution. A key postulate is a semiclassical branching condition, where the relative orientation of $\mu_\text{e}$ and $\mu_\text{n}$ dictates the collapse direction of $\mu_\text{e}$. Denoting the electron and nuclear co-quantum state as $\lvert \mu_\text{e} \copyright \mu_\text{n} \rangle$, we have \textcolor{blue}{\hyperref[sec:ref]{[3]}}.

\begin{equation}
    \lvert \mu_\text{e} \copyright \mu_\text{n} \rangle = \frac{1 - \operatorname{sgn}(\theta_\text{e} - \theta_\text{n})}{2} \lvert +z \rangle + \frac{1 + \operatorname{sgn}(\theta_\text{e} - \theta_\text{n})}{2} \lvert -z \rangle.    
\end{equation}

The result of the Frishc-Segr\`e experiment is closely replicated by the current model, Cf. Figure 1(b) \textcolor{blue}{\hyperref[sec:ref]{[2]}}, including the signature fact that the spin-flip fraction curve bends down at higher current values.

\begin{figure}%
    \centering
    \subfloat[\centering]{{\includegraphics[width=.5\linewidth]{Figures/original.png}}}%
    \qquad
    \subfloat[\centering]{{\includegraphics[width=.4\linewidth]{Figures/graph.png}}}%
    \caption{(a) Original setup of the Frisch-Segr\`e experiment from Ref. \textcolor{blue}{\hyperref[sec:ref]{[1]}}. (b) Result of current simulation using the CQD model, taken from Ref. \textcolor{blue}{\hyperref[sec:ref]{[2]}}. The horizontal axis is current that induces the central magnetic field; the vertical axis is the spni-flip fraction.}%
    \label{fig:example}%
\end{figure}

\subsection{Schwinger Oscillator Model}
\label{ssec:SOM}
Julian Schwinger showed that the algebra of a system of two simultaneous harmonic oscillators is equivalent to an angular momentum algebra \textcolor{blue}{\hyperref[sec:ref]{[6]}}.

Consider two independent oscillators, with the annihilation and creation operators given respectively by \( a_\pm \) and \( a_\pm^\dagger \). Define the number operators as usual: $N_\pm \equiv a_\pm^\dagger a_\pm$.

If we assume the usual commutation relations among \( a_\pm \), \( a_\pm^\dagger \), and \( N_\pm \),
\begin{equation}
    [a_\pm, a_\pm^\dagger] = 1, \quad [N_\pm, a_\pm] = -a_\pm, \quad [N_\pm, a_\pm^\dagger] = a_\pm^\dagger,
\end{equation}

we will obtain
\begin{equation}
    |n_+, n_-\rangle = \frac{(a_+^\dagger)^{n_+} (a_-^\dagger)^{n_-}}{\sqrt{n_+!} \sqrt{n_-!}} |0,0\rangle.
\end{equation}

Then define
\begin{equation}
    J_+ \equiv \hbar a_+^\dagger a_-, \quad J_- \equiv \hbar a_-^\dagger a_+, \quad J_z \equiv \left( \frac{\hbar}{2} \right) (a_+^\dagger a_+ - a_-^\dagger a_-).
\end{equation}

The operators in equation (4) can be shown to satisfy the usual angular-momentum commutation relations. As far as the transformation properties under rotations are concerned, any spin-\( j \) object can be understood as a composite system of \( 2j \) spin-\(\frac{1}{2}\) particles.