The methods involved in the research project are outlined below:
\begin{enumerate}
    \item Collect and organize existing experimental data on atoms with spin higher than $\frac{1}{2}$. The amount and quality of the data collected could be used to decide the type of experiment that could be used to verify step 2, which is an extension of the model to atoms with higher spins. 
    \item Although the current model predicts the result of the Frisch-Segr\`e experiment very well, it has limited applications for atoms of higher spin. Section \textcolor{blue}{\hyperref[ssec:SOM]{3.2}} hints at the fact that we can extend the model for spin-$\frac{1}{2}$ particles to particles with higher spin. We will investigate possible ways to modify the current model to accommodate atoms of higher spin, for example by adopting an appropriate generalization of the branching condition equation (1). Atoms with nuclear spin $0$ is a special case that is not considered by the current model, this may also be considered. Using the data organized in step 1, we will compare the theoretical calculations with experimental data.
\end{enumerate}

The research will be guided by principal investigator Professor Lihong V. Wang. The proposed research will happen in the summer of 2025. This research will be performed in the Caltech Optical Imaging Laboratory.

