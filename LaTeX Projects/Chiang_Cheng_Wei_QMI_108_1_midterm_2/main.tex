\documentclass[12pt, A4, twoside]{article}
\usepackage{scrextend}
\usepackage{geometry}
 \geometry{
	 a4paper,
 	total={170mm,257mm},
 	left=20mm,
	 top=20mm,
 }

\usepackage{amsmath}
\usepackage[shortlabels]{enumitem} %enumerate with letters
\usepackage[normalem]{ulem}
\usepackage{indentfirst}
\usepackage{pifont}
\usepackage{fancyhdr}   % 頁首頁尾
\usepackage{amssymb}
\usepackage{pgfplots}
\pgfplotsset{compat=1.15}
\usetikzlibrary{arrows}

% Mandarin
\usepackage{CJKutf8}
% bkai = 標楷體
% bsmi = 新細明體

\usepackage{graphicx} % Required for inserting images

% paragraph
%\setlength{\parskip}{\baselineskip}
\setlength{\parindent}{2em}
%
\usepackage{lmodern,bm}                
\usepackage[T1]{sansmath} 
\SetMathAlphabet{\mathsfbf}{sans}{\sansmathencoding}{\sfdefault}{bx}{sl}
\usepackage{etoolbox}
\AtBeginEnvironment{sansmath}{\let\bm\mathsfbf}{}{}
\usepackage{mdframed}
\usepackage{braket} % braket notation
\usepackage{amsmath, nccmath}

\usepackage{bbm}
\usepackage{dsfont} % bold numerals
\usepackage{bbold} % blackboard bold font
\usepackage{mathrsfs} % script

%% custom norm
\newcommand{\Norm}[1]{\left\lVert#1\right\rVert}
\newcommand{\Abs}[1]{\left\lvert#1\right\rvert}
%%

% circled number
\newcommand*\circlednum[1]{\raisebox{.5pt}{\textcircled{\raisebox{-.9pt} {#1}}}}

\usepackage{relsize} % math symbol size

\usepackage{amsthm}
\newtheorem{thm}{Theorem}[section]
\newtheorem{mydef}{Definition}[section]
\newtheorem{lemma}[thm]{Lemma}
\newtheorem{proposition}[thm]{Proposition}
\newtheorem{corollary}[thm]{Corollary}
\newtheorem{property}{Property}

\newtheoremstyle{problemstyle}
        {5pt} % <space above>
        {15pt} % <space below>
        {\normalfont} % <body font>
        {} % <indent amount}
        {\bfseries} % <theorem head font>
        {\normalfont\bfseries.} % <punctuation after theorem head>
        {.5em} % <space after theorem head>
        {} % <theorem head spec (can be left empty, meaning `normal')>
\theoremstyle{problemstyle}
\newtheorem{problem}{Problem}
\newtheorem{solution}{Solution}[section]
\newtheorem{example}{Example}[section]

\newenvironment{problembk}[1][]{%
  \begin{problem}[#1]$ $\par\nobreak\ignorespaces
}{%
  \end{problem}
}
\newenvironment{solutionbk}[1][]{%
  \begin{solution}[#1]$ $\par\nobreak\ignorespaces
}{%
  \end{solution}
}
\theoremstyle{remark}
\newtheorem*{remark}{Remark}


\usepackage{mdframed}
\newenvironment{fthm}
    {\begin{mdframed}\begin{thm}}
    {\end{thm}\end{mdframed}}

\newcommand{\bvec}[1]{\mathbf{#1}} % vector

% calculus
\newcommand{\D}{\mathrm{d}}
\newcommand{\Par}{\partial}
% Calligraphy (Euler script?)
\newcommand{\cA}{\mathcal{A}}
\newcommand{\cB}{\mathcal{B}}
\newcommand{\cC}{\mathcal{C}}
\newcommand{\cF}{\mathcal{F}}
\newcommand{\cH}{\mathcal{H}}
\newcommand{\cI}{\mathcal{I}}
\newcommand{\cL}{\mathcal{L}}
\newcommand{\cM}{\mathcal{M}}
\newcommand{\cP}{\mathcal{P}}
\newcommand{\cT}{\mathcal{T}}
% Calligraphy (blackboard bold)
\newcommand{\bC}{\mathbbm{C}}
\newcommand{\bD}{\mathbbm{D}}
\newcommand{\bF}{\mathbbm{F}}
\newcommand{\bH}{\mathbbm{H}}
\newcommand{\bI}{\mathbbm{I}}
\newcommand{\bK}{\mathbbm{K}}
\newcommand{\bN}{\mathbbm{N}}
\newcommand{\bO}{\mathbbm{O}}
\newcommand{\bP}{\mathbbm{P}}
\newcommand{\bQ}{\mathbbm{Q}}
\newcommand{\bR}{\mathbbm{R}}
\newcommand{\bT}{\mathbbm{T}}
\newcommand{\bZ}{\mathbbm{Z}}
% Calligraphy (fraktur font)
\newcommand{\fI}{\mathfrak{I}}
\newcommand{\fR}{\mathfrak{R}}
% lower indices (roman)
\newcommand{\rC}{\mathrm{C}} % Curie constant
\newcommand{\rf}{\mathrm{f}} % final
\newcommand{\ri}{\mathrm{i}} % initial
\newcommand{\rT}{\mathrm{T}} % isothermal

% figures
\usepackage{wrapfig} 
\usepackage{subfig}

% ckickable table of contents
\usepackage{hyperref}
\hypersetup{
    colorlinks,
    citecolor=black,
    filecolor=black,
    linkcolor=blue,
    urlcolor=black
}

\begin{document}
\begin{CJK}{UTF8}{bkai}

\title{蔣正偉 Quantum Mechanics I Second Midterm Exam}
\author{Jonathan Huang (Giant Water Bird)}
\date{\today}
\cfoot{\thepage}
% header and footer
\pagestyle{fancy}
\rhead{蔣正偉}
\lhead{Quantum Mechanics I}

\maketitle

%\begin{figure}[h!]%
%    \centering
%    \subfloat{\includegraphics[width=8cm]{giant water bird.jpg}}
%    \caption{}
%\end{figure}

\begin{center}
    Instruction
\end{center}

    This is an open-book, 120-minute exam. Your are only allowed to use the main textbook (by Sakurai and Napolitano) and your own handwritten notes. Only derived results in the main text of Sakurai and Napolitano up to the range of this exam (i.e., Chapter 2) can be used. The score of each sub-problem is indicated by the number in square brackets. To avoid any misunderstanding, ask if you have any queations about the problems or notations. In your answers, define clearly your notations if they differ from those in the main textbook. Note that we also do not distinguish the notation of an operator from that of its corresponding variable.

\newpage

\section{Problems}
\begin{problem}[70 pts.]
    Consider a one-dimensional simple harmocis oscillator (SHO) with the potential given by
    \begin{equation}
        V(x)=\frac{1}{2}m\omega^2 x^2.
    \end{equation}
    The $n$-th energy level is denoted by $\ket{n}$. Recall that
    \begin{equation}
        \hat{a} = \frac{1}{\sqrt{2}}\left(\frac{x}{x_0}+i\frac{x_0 p}{\hbar}\right), \quad \hat{a}^\dagger = \frac{1}{\sqrt{2}}\left(\frac{x}{x_0}-i\frac{x_0 p}{\hbar}\right).
    \end{equation}
    \begin{equation}
        \hat{a}\ket{n} = \sqrt{n}\ket{n-1}, \quad \hat{a}^\dagger\ket{n} = \sqrt{n+1}\ket{n+1}.
    \end{equation}
    where $x_0 = \sqrt{\hbar/m\omega}$. Also, recall that the position and momentum operators in the Heisenberg pircture for the SHO are solved to be
    \begin{equation}
        x(t) = x(0)\cos(\omega t) + \frac{p(0)}{m\omega}\sin(\omega t),
    \end{equation}
    \begin{equation}
        p(t) = -m\omega x(0)\sin(\omega t) + p(0)\cos{(\omega t)}.
    \end{equation}
    \begin{enumerate}[(a)]
        \item (10 pts.) Work out the matrix forms of the following operators: $\hat{a}$, $\hat{a}^{\dagger}$, $\hat{x}$, and $\hat{p}$ in the Schr\"{o}dinger picture. Show at least the upper left 4×4 sub-matrix.
        \item (10 pts.) Calculate $(\Delta x)(\Delta p)$ for the $n$-th level state, where for your convenience $\Delta a \equiv \sqrt{\langle A^2\rangle - \langle A\rangle^2}$.
        \item (10 pts.) Calculate the expectation values of kinetic and potential energy of the $n$-th level state, and show that they satisfy the virial theorem.
        \item (10 pts.) Compute the expectation values of x(t) and p(t) for the n-th level state. Explain physically what your results mean. In particular, are they consistent with the classical picture of an oscillator?
        \item (10 pts.) Use the WKB approximation method to work out the eigenstate energies.
\newline

        Define the coherent state, denoted by $\ket{\alpha}$, satisfying $A\ket{\alpha}=\alpha \ket{\alpha}$ with the normalization $\braket{\alpha | \alpha} = 1$.
        \item (10 pts.) Derive the normalized coherent state in the basis of $\{\ket{n}\}$.
        \item (10 pts.) Explain what kind of quantity the eigenvalue $\alpha$ can be, and whether or not one can set it to be real at all times.
    \end{enumerate}
\end{problem}

\begin{problem}[10 pts.]
    Consider a one-dimensional quantum mechanical problem of a particle with a time-independent Hamiltonian $H$. In the path integral calculation, we have shown in class that for a particle with a general Hamiltonian $H(p,x)$:
    \begin{equation}
        \Braket{x_{i+1}|H|x_i} = \int^\infty_{-\infty}\frac{\D p}{2\pi\hbar} H(p,\bar{x}_i)\exp{\left[\frac{i}{\hbar}p(x_{i+1}-x_i)\right]},
    \end{equation}
     where $\bar{x}_i = \frac{x_i+x_{i+1}}{2}$, and $x_i$ and $x_{i+1}$ denote respectively the positions of the particle at two infinitesimally separated moments $t_i$ and $t_{i+1}$. Explain why one should use $\bar{x}_i=(x_i+x_{i+1})/2$ instead of $x_i$ or $x_{i+1}$.
\end{problem}

\begin{problem}[20 pts.]
    Consider a particle of electric charge $Qe$ and mass $m$ in an electromagnetic potential, described by the Hamiltonian
    \begin{equation}
        H = \frac{\Pi^2}{2m}+Qe\phi,
    \end{equation}
    with $Π=\bvec{p}-Qe/c\bvec{A}$, where $\phi(\bvec{x})$ and $\bvec{A}(\bvec{x})$ are time-independent scalar and vector potentials, respectively. Use the Heisenberg picture throughout this problem.
    \begin{enumerate}[(a)]
        \item Explain whether $\D\bvec{x}/\D t$ and $\bvec{x}$ can commute with each other. If so, prove it. If not, give a simple example.
        \item Derive the quantum mechanical version of the Lorentz force
        \begin{equation}
             m\frac{\D^2\bvec{r}}{\D t^2} = Qe\left[\bvec{E} + \frac{1}{2c}\left(\frac{\D\bvec{r}}{\D t}\times\bvec{B} - \frac{\D\bvec{r}}{\D t}\times\bvec{B} \right)\right].
        \end{equation}
    \end{enumerate}
\end{problem}

\end{CJK}
\end{document}