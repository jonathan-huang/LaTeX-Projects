\documentclass[12pt]{article}

% Language setting
% Replace `english' with e.g. `spanish' to change the document language
\usepackage[english]{babel}

% Set page size and margins
% Replace `letterpaper' with `a4paper' for UK/EU standard size
\usepackage[letterpaper,top=2cm,bottom=2cm,left=3cm,right=3cm,marginparwidth=1.75cm]{geometry}

% Useful packages
\usepackage{amsmath}
\usepackage{graphicx}
\usepackage{mathtools}
\usepackage{amsfonts}
\usepackage{algorithm}
\usepackage{algorithmicx}
\usepackage[noend]{algpseudocode}
\usepackage[colorlinks=true, allcolors=blue]{hyperref}
\usepackage{bm}
\usepackage{amssymb}
\usepackage{esint}
\usepackage{tikz}
\usepackage{hyperref}
\DeclareMathOperator{\ord}{ord}
\DeclareMathOperator{\chr}{char}
\DeclareMathOperator{\spn}{span}
\DeclareMathOperator{\Img}{Im}
\DeclareMathOperator{\Syl}{Syl}
\DeclareMathOperator{\Conj}{Conj}
\DeclareMathOperator{\Aut}{Aut}
\DeclareMathOperator{\id}{id}
\DeclareMathOperator{\sign}{sign}
\DeclareMathOperator{\Inn}{Inn}
\DeclareMathOperator{\Out}{Out}
\DeclareMathOperator{\Frac}{Frac}
\DeclareMathOperator{\pre}{pre}
\DeclareMathOperator{\Gal}{Gal}
\DeclareMathOperator{\Orb}{Orb}
\DeclareMathOperator{\Stab}{Stab}
\def\acts{\curvearrowright}
\usetikzlibrary{tikzmark}
\newcommand{\tikzarc}[1]{%
\tikzmarknode{a}{#1}
\begin{tikzpicture}[overlay,remember picture]
\draw ([yshift=1pt]a.north west) to[bend left=20] ([yshift=1pt]a.north east);
\end{tikzpicture}%
}

\title{Introduction to Algebra (II)}
\author{Cheng-Yun Yeh}

\begin{document}
\maketitle

\section*{Chapter 6. Groups (II)}
\subsection*{6.1 Normal subgroups and quotient groups}
\indent

\textbf{Def. }Let $G$ be a group and $H \subseteq G$ be a subgroup. We denote the collection of the (left) cosets of $H$ as $G/H:=\{\text{Left cosets of $H$}\}$, and each element in $G/H$ is in the form $gH$ with some $g \in G$. \\

Note. $G/H$ is only a set, not always as group. If we want $G/H$ to be a quotient group, we need a better "$H$", just like we use ideal instead of subring in quotient ring. \\ \\

\textbf{Def. }Let $G$ be a group and $H \subseteq G$ be a subgroup. We say $gHg^{-1}:=\{ghg^{-1}:h \in H\}$ is the \textbf{$g$-conjugate of $H$}, and we say $H$ is a \textbf{normal subgroup} of $G$ if $\forall g \in G$, $gHg^{-1}=H$, or equivalently, $g^{-1}Hg=H$. We denote it as $H \trianglelefteq G$. \\

Example-1: If $G$ is abelian, then all subgroups of $G$ is normal. \\

Example-2: Let $R=\langle r \rangle$ be a subgroup of $\mathcal{D}_3$, then $R$ is normal. \\ \\

\textbf{Prop 1. }Let $G$ be a group and $H \subseteq G$ be a subgroup. Let $\phi:G \to G'$ be a group homomorphism, then $\ker\phi \trianglelefteq G$. \\

pf. For all $g \in G$, for all $h \in \ker\phi$, we have $\phi(ghg^{-1})=\phi(g)\phi(h)\phi(g^{-1})=\phi(g)\phi(g)^{-1}=e$. Thus, $g(\ker\phi)g^{-1} \subseteq \ker\phi$. Also, since $ghg^{-1}=h'$ for some $h' \in \ker\phi$, we have $h=g^{-1}h'g=g'h'g'^{-1}$, where $g'=g^{-1} \in G$. Thus, $\ker\phi \subseteq g(\ker\phi)g^{-1}$ for all $g \in G$ (Since $g \mapsto g^{-1}$ is bijective). In conclusion, we have $g(\ker\phi)g^{-1}=\ker\phi$ for all $g \in G$, i.e. $\ker\phi \trianglelefteq G$. \\ \\

\textbf{Prop 2. }Let $G$ be a group and $H \subseteq G$ is a subgroup. Then: \\

(i) If $gHg^{-1} \subseteq H$, $\forall g \in G$, then $H \trianglelefteq G$. \\

(ii) The $g$-conjugate of $H$, $gHg^{-1}$, is a subgroup of $G$, $\forall g \in G$. \\

(iii) Let $\phi:H \to gHg^{-1}$, $\phi(h)=ghg^{-1}$, then $\phi$ is a group isomorphism. \\

pf-1. First, $gHg^{-1} \subseteq H$ and $g^{-1}Hg \subseteq H$ by assumption. Then, given any $h \in H$, we have $h=(gg^{-1})h(gg^{-1})=g(g^{-1}hg)g^{-1}=gh'g^{-1}$ for some $h' \in H$, i.e. $h \in gHg^{-1}$, $H \subseteq gHg^{-1}$. Thus, we have $gHg^{-1}=H$, $H \trianglelefteq G$. \\

pf-2. Given $h_1,h_2,h_3 \in gHg^{-1}$, then $gh_1'g^{-1}=h_1$, $gh_2'g^{-1}=h_2$, $gh_3'g^{-1}$ for some $h_1',h_2',h_3' \in H$. Thus, we have:

\begin{gather*}
    h_1 \cdot h_2=(gh_1'g^{-1})\cdot(gh_2'g^{-1})=g(h_1'\cdot h_2')g^{-1} \in gHg^{-1} \\
    h_2 \cdot h_1=(gh_2'g^{-1})\cdot(gh_1'g^{-1})=g(h_2'\cdot h_1')g^{-1} \in gHg^{-1}
\end{gather*}

And since $e=geg^{-1}$ and $\forall h \in gHg^{-1}$, we have $h=gh'g^{-1}$ for some $h' \in H$, then $h^{-1}=gh'^{-1}g^{-1} \in gHg^{-1}$. Therefore, $gHg^{-1}$ is a subgroup of $G$, $\forall g \in G$. \\

pf-3. First, since for all $h_1,h_2 \in H$, we have $\phi(h_1\cdot h_2)=g(h_1\cdot h_2)g^{-1}=(gh_1g^{-1})\cdot(gh_2g^{-1})=\phi(h_1)\cdot\phi(h_2)$, then $\phi$ is a group homomorphism. In addition, $gHg^{-1}$ is surjective since every element in it is mapped by some $h \in H$. Also, consider $h \in H$ such that $\phi(h)=e$, then $ghg^{-1}=e$, i.e. $h=g^{-1}eg=e$. Thus, $\ker\phi=\{e\}$. In conclusion, $\phi$ is bijective, it is a group isomorphism. \\

Note. If we want to show $\phi$ is an isomorphism, instead of showing that is is bijective, we can also construct a map $\psi$ such that $\phi\circ\psi=\psi\circ\phi=$The identity map. This implies the inverse of $\phi$ exists on the whole codomain. Here, we can consider $\psi:gHg^{-1} \to H$, $\psi(\alpha)=g^{-1}\alpha g$. \\ \\

\textbf{Lemma 3. }Let $G$ be a group and $H \trianglelefteq G$ be a normal subgroup of $G$. Suppose $g_1,g_1',g_2,g_2' \in G$ and $g_1' \in g_1H$, $g_2'=g_2H$, where $g_1H$ and $g_2H$ are the cosets of $H$ attached on $g_1$ and $g_2$, respectively. Then, $g_1g_2H=g_1'g_2'H$. \\

pf. Since $g_1' \in g_1H$ and $g_2' \in g_2H$, there exists $h_1,h_2$ such that $g_1'=g_1h_1$ and $g_2'=g_2h_2$, we're going to show that $g_1'g_2' \in g_1g_2H$, i.e. $g_2^{-1}g_1^{-1}g_1'g_2' \in H$. We know that:

\[
    g_2^{-1}g_1^{-1}g_1'g_2'
    =g_2^{-1}g_1^{-1}g_1h_1g_2h_2
    =g_2^{-1}h_1g_2h_2
\]

Since $g_2^{-1} \in G$ and $H$ is normal, we have $g_2^{-1}h_1g_2 \in H$. As a consequence, $g_2^{-1}h_1g_2h_2 \in H$. Then we have $g_2^{-1}g_1^{-1}g_1'g_2' \in H$, $g_2^{-1}g_1^{-1}g_1'g_2'=h'$ for some $h' \in H$, i.e. $g_1'g_2'=g_1g_2h' \in g_1g_2H$. \\ \\

\textbf{Theorem 4. }Let $G$ be a group and $H \trianglelefteq G$ be a normal subgroup of $G$, then: \\

(i) The collection of cosets $G/H$ is a group via the well-defined group operation $(g_1H)\cdot(g_2H)=(g_1\cdot g_2)H$, for all $g_1H,g_2H \in G/H$. We call it the \textbf{quotient group}. \\

(ii) The map $\Pi:G \to G/H$, $g \mapsto gH$ is a group homomorphism whose kernel is $H$. \\

(iii) Let $\phi:G \to G'$ be a group homomorphism and $H \subseteq \ker\psi$, then there exists a unique homomorphism $\lambda:G/H \to G'$ satisfying $\lambda(gH)=\phi(g)$. \\

(iv) If we take $H=\ker\phi$ (By prop 1, $\ker\phi$ is normal), then $\lambda:G/H \to G'$ is a unique injective homomorphism. In particular, we get an isomorphism $G/H \simeq \Img(\lambda) \subseteq G'$. This is called the \textbf{first theorem of isomorphism}. \\

pf-1. The associative law is inherited from $G$, and the identity is $H=eH$. For all $gH \in G/H$, $(gH)^{-1}=g^{-1}H$. \\

pf-2. Given any $g_1,g_2 \in G$, we have $\Pi(g_1 \cdot g_2)=(g_1\cdot g_2)H=(g_1H)\cdot(g_2H)=\Pi(g_1)\cdot\Pi(g_2)$. Thus, is a group homomorphism. Moreover, if $\Pi(g)=H$ for some $g \in G$, then it means $gH=H=eH$, i.e. $g \in H$ (The coset attached to elements in $H$ is still $H$). \\

pf-3. We first check if it is well-defined. Since $H \subseteq \ker\phi$, given $g_1,g_1' \in G$ such that $g_1H=g_1'H$, i.e. $g_1=g_1'h$ for some $h \in H$, then $\lambda(g_1H)=\phi(g_1)=\phi(g_1')\phi(h)=\phi(g_1')=\lambda(g_1'H)$. Thus, it is well-defined. \\

Next, given $g_1H,g_2H \in G/H$, we have $\lambda((g_1H)\cdot(g_2H))=\lambda((g_1\cdot g_2)H)=\phi(g_1\cdot g_2)=\phi(g_1)\cdot\phi(g_2)=\lambda(g_1H)\cdot\lambda(g_2H)$. Thus, it is a homomorphism. \\

Finally, since the value of $\lambda(gH)$ is determined by the statement, $\lambda$ can only be unique. \\

pf-4. From pf-3, we know that $\lambda$ is a well-defined homomorphism, we want to check that if it is injective. If $gH \in G/H$ satisfies $\lambda(gH)=\psi(g)=e$, then $g \in H$ since we take $H=\ker\phi$. Then, $gH=H$, $\ker\lambda=\{H\}$, we're done. \\ \\

\subsection*{6.2 Group actions}
\indent

\textbf{Def. }Let $G$ be a group and $X$ be a set. An \textbf{action} of $G$ on $X$ is a map $G \times X \to X$, $(g,x) \mapsto g \cdot x$, denoted as $G \acts A$, satisfying the following two conditions: \\

(i) Identity axiom: $e \cdot x=x$, $\forall x \in X$

(ii) Associative axiom: $\forall g_1,g_2 \in G$, $\forall x \in X$, $(g_1 \cdot g_2) \cdot x=g_1 \cdot (g_2 \cdot x)$ \\ 

Example: $S_X$: All permutations on $X$, i.e. all bijections from $X$ to $X$, and $S_x$ is a group under the composition of maps. \\ \\

\textbf{Remark. }A group action is bijective for each $g$, i.e. $g$ is can be regarded as a permutation of $X$. \\

pf. If $g \cdot x_1=g \cdot x_2$, then $g^{-1} \cdot (g \cdot x_1)=g^{-1} \cdot (g \cdot x_2)$, $x_1=x_2$. For all $x \in X$, $g \cdot (g^{-1} \cdot x)=x$. Thus, $g$ acts bijectively on $X$. \\ \\

\textbf{Remark. }We can show that a group action of $G$ on $X$ is equivalent to a group homomorphism $\alpha:G \to S_x$, where $\alpha$ sends $g$ to a permutation $\alpha(g):X \to X$. That is, define $g \cdot x=\alpha(g)(x)$. By (i), (ii), we can show that $\alpha$ is a group homomorphism. \\ \\

\textbf{Def. }Let $X$ be a group, an \textbf{endomorphism} is a map $\phi:X \to X$ s.t. $\phi$ is a homomorphism. If $\phi$ is an isomorphism, then we say $\phi$ is an \textbf{automorphism}. All automorphisms of $X$ is denoted as $\Aut(X)$. ($\Aut(X)$ is a group with composition of functions, which will be discussed in chapter 12.3)\\ \\

\textbf{Remark. }If $X$ is a group of order $p$ where $p$ is a prime, then $\Aut(X) \simeq \Aut(\mathbb{Z}/p\mathbb{Z}) \simeq (\mathbb{Z}/p\mathbb{Z})^\times$. \\

pf. Suppose $X=\langle x \rangle$, then $x^i \mapsto i+p\mathbb{Z}$ is an isomorphism, we have $X \simeq \mathbb{Z}/p\mathbb{Z}$ and thus $\Aut(X) \simeq \Aut(\mathbb{Z}/p\mathbb{Z})$. For $\Aut(\mathbb{Z}/p\mathbb{Z})=(\mathbb{Z}/p\mathbb{Z})^\times$, it is proved in chapter 8. \\ \\

\textbf{Def. }Given a group $G$ action on a set $X$. For some $x \in X$, we define: \\

(i) The \textbf{orbit} of $x$: $G \cdot x:=\{g \cdot x:g \in G\} \subseteq X$

(ii) The \textbf{stabilizer} of $x$: $G_x:=\{g \in G:g \cdot x=x\} \subseteq G$ \\ \\

\textbf{Prop 5. }Let $G$ be a group, $X$ be a set and $G \acts X$, then: \\

(i) $\forall x \in X$, $G_x$ is a subgroup of $G$ \\

(ii) Define a relation on $X$ by $\forall x,y \in X$, $x \sim y$ if $y=g \cdot x$ for some $g \in G$. Then, "$\sim$" is an equivalence relation. \\

(iii) Let $x \in X$, there is a well-defined bijection $\alpha:G/G_x \to G \cdot x$, $gG_x \mapsto g \cdot x$. In particular, if $G$ is finite, then $\#(G \cdot x)=\#(G/G_x)=\#G/\#G_x$. \\

pf-1. First, $e \cdot x=x$ by identity axiom, so $e \in G_x$. Next, if $\forall g \in G_x$, we have $g^{-1} \cdot x=g^{-1} \cdot (g \cdot x)=(g \cdot g^{-1}) \cdot x=e \cdot x=x$, so $g^{-1} \in G_x$. Finally, $\forall g_1,g_2 \in G_x$, we have $(g_1 \cdot g_2) \cdot x=g_1 \cdot (g_2 \cdot x)=g_1 \cdot x=x$ and $(g_2 \cdot g_1) \cdot x=g_2 \cdot (g_1 \cdot x)=g_2 \cdot x=x$, so $g_1 \cdot g_2 \in G_x$ and $g_2 \cdot g_1 \in G_x$. In conclusion, $G_x$ is a subgroup of $G$. \\

pf-2. Since $x=e \cdot x$, we have $x \sim x$. Suppose $x \sim y$, i.e. $y=g \cdot x$ for some $g \in G$, then $g^{-1} \cdot y=g^{-1} \cdot (g \cdot x)=(g^{-1} \cdot g) \cdot x=e \cdot x=x$, so $y \sim x$. Suppose $x \sim y$ and $y \sim z$, i.e. $y=g_1 \cdot x$ and $z=g_2 \cdot y$ for some $g_1,g_2 \in G$. Then $z=g_2 \cdot y=g_2 \cdot (g_1 \cdot x)=(g_2 \cdot g_1) \cdot x$, so $x \sim z$. In conclusion, "$\sim$" is an equivalence relation. \\

pf-3. Given $g_1G_x=g_2G_x$, then there exists $h \in G_x$ such that $g_1=g_2h$. Then, $\alpha(g_1G_x)=g_1 \cdot x=(g_2 \cdot h) \cdot x=g_2 \cdot (h \cdot x)=g_2 \cdot x=\alpha(g_2G_x)$. Thus, $\alpha$ is well-defined. \\

Since each $g \cdot x \in G \cdot x$ is an element $g$ acts on $x$, so $\alpha(gG_x)=g \cdot x$, $\alpha$ is surjective. Suppose $\alpha(g_1G_x)=\alpha(g_2G_x)$ for some $g_1,g_2 \in G$, then $g_1 \cdot x=g_2 \cdot x$, $x=(g_1^{-1} \cdot g_2) \cdot x$, i.e. $g_1^{-1} \cdot g_2 \in G_x$, we have $g_2=g_1h$ for some $h \in G_x$, $g_1G_x=g_2G_x$. Thus, $\alpha$ is injective. In conclusion, $\alpha$ is a bijection. \\

If $G$ is finite, since $\alpha$ is a bijection, we have $\#(G \cdot x)=\#(G/G_x)$. By Lagrange theorem, we have $\#(G \cdot x)=\#(G/G_x)=\#G/\#G_x$.

\subsection*{6.3 Orbit-stabilizer counting theorem}
\indent

\textbf{Theorem 6. (Orbit-stabilizer counting theorem)} \\

Let $G$ be a finite group acting on a finite set $X$, by prop 5-(ii), the "$\sim$ defined by $x \sim y \Leftrightarrow y=g \cdot x$ is an equivalence relation. Then, we can choose $x_1,...,x_k \in X$ s.t. $G \cdot x_1,...,G \cdot x_k$ are distinct orbit of elements of $X$ and $\bigsqcup_{i=1}^k G \cdot x_i=X$. Then:

\[
    \#X=\#\bigsqcup_{i=1}^k G \cdot x_i=\sum_{i=1}^k \#G \cdot x_i=\sum_{i=1}^k \#(G/G_{x_i})=\sum_{i=1}^k \#G/\#G_{x_i}
\]

pf. Trivial, just use the theorem of equivalence class and prop 5-(iii). \\ \\

\textbf{Def. }Let $G$ be a group, the \textbf{center} of $G$, denoted as $Z(G)$, is the set $Z(G)=\{g \in G:g \cdot \sigma=\sigma \cdot g,\ \forall \sigma \in G\}$. \\ \\

\textbf{Fact. }Let $G$ be a group, then $Z(G)$ is a normal subgroup of $G$. \\

pf. Since $e$ commutes with all $g \in G$, $e \in Z(G)$. For all $h \in Z(G)$, for all $\sigma \in G$, we have $h^{-1}\sigma=h^{-1}\sigma hh^{-1}=h^{-1}h\sigma h^{-1}=\sigma h^{-1}$, then $h \in Z(G)$. Finally, if $h_1,h_2 \in Z(G)$, then for all $\sigma \in G$, we have $(h_1h_2)\sigma=h_1\sigma h_2=\sigma(h_1h_2)$, then $h_1h_2 \in Z(G)$. So, $Z(G)$ is a subgroup of $G$. Since for all $g \in G$, we have $gZg^{-1}=Zgg^{-1}=Z$. Thus, $Z(G) \trianglelefteq G$. \\ \\

\textbf{Theorem 7. }Let $p$ be a prime number and $G$ be a finite group. If $\#G=p^n$, $n \in \mathbb{Z}_{>0}$, then $Z(G)$ is nontrivial, i.e. $\exists h \in Z(G)$, $h \ne e$ such that $h\sigma=\sigma h$ for all $\sigma \in G$. \\

pf. Consider $G$ acts on itself by conjugation, i.e. $\forall g,x \in G$, $g \star x=g \cdot x \cdot g^{-1}$, where $\star$ means the action. Then, there exists $x_1,...,x_k \in G$ such that $G=\bigsqcup_{i=1}^k G \star x_i$. \\

We notice that $x_i \in Z(G) \Leftrightarrow G_{x_i}=G \Leftrightarrow G \star x_i=\{x_i\}$, then, write:

\[
    \#G=\sum_{1 \le i \le k \atop x_i \in Z(G)} \#(G \star x_i)+\sum_{1 \le i \le k \atop x_i \notin Z(G)} \#(G \star x_i)=\#Z(G)+\sum_{1 \le i \le k \atop x_i \notin Z(G)} \#(G \star x_i)
\]

Since $\#G=p^n$, by Lagrange theorem, we have $\#G_{x_i}=p^{n_i}$, $n_i \le n$. If $x_i \notin Z(G)$, then $G_{x_i} \ne G$, i.e. $\#G_{x_i}=p^{n_i}$, $n_i<n$. By orbit-stabilizer counting theorem, we have:

\[
    \sum_{1 \le i \le k \atop x_i \notin Z(G)} \#(G \star x_i)=\sum_{1 \le i \le k \atop x_i \notin Z(G)} \#G/\#G_{x_i}=\sum_{1 \le i \le k \atop x_i \notin Z(G)}p^{n-n_i},\ n-n_i \ge 1
\]

Thus, $p$ divides $\sum_{1 \le i \le k \atop x_i \notin Z(G)} \#(G \star x_i)$. Also, we know that $p$ divides $\#G$, we have $p$ divides $\#Z(G)$, then $\#Z(G)>1$. (Or say $\#Z(G) \ge p$) \\ \\

\textbf{Corollary 8. }Let $G$ be a finite group of order $p^2$ for some prime $p$. Then, $G$ is abelian. \\

pf. Since $\#Z(G)=1$ or $p$ or $p^2$ by Lagrange theorem, by theorem 7, we knoe $\#Z(G) \ne 1$. Suppose $\#Z(G)=p$, since $Z(G) \trianglelefteq G$, $G/Z(G)$ is group. We know that $\#(G/Z(G))=\#G/\#Z(G)=p^2/p=p$, then $G/Z(G)$ is cyclic, i.e. $G/Z(G)=\langle gZ(G) \rangle$ for some $g \in G$. \\

Then, given $\sigma,\tau \in G$, we can find $i,j$, $0 \le i,j \le p-1$, such that $\sigma \in g^iZ(G)$, $\tau \in g^jZ(G)$, i.e. $\sigma=g^ih$, and $\tau=g^jh'$ for some $h,h' \in Z(G)$. Then, $\sigma\tau=g^ihg^jh'=g^{i+j}hh'$, $\tau\sigma=g^jh'g^ih=g^{i+j}hh'=\sigma\tau$, i.e. $\sigma,\tau \in Z(G)$ since this property holds for all $\sigma,\tau \in G$. Thus, $Z(G)=G$, which contradicts to $\#Z(G)=p$. In conclusion, $\#Z(G)=p^2$, i.e. $Z(G)=G$, $G$ is abelian. \\ \\

\textbf{Def. }Let $G$ be a group and $H \subseteq G$ be a subgroup, we define: \\

(i) The \textbf{centralizer} of $H$, $Z_G(H)=\{g \in G:hg=gh,\ \forall h \in H\}$

(ii) The \textbf{normalizer} of $H$, $N_G(H)=\{g \in G:gHg^{-1}=H\}$ \\

Then, $Z_G(H) \subseteq N_G(H)$ since if $z \in Z_G(H)$, then $zHz^{-1}=Hzz^{-1}=H$. \\ \\

\textbf{Fact. }$Z_G(H)$ and $N_G(H)$ are subgroups of $G$. \\

pf ($Z_G(H)$). Since $e$ commutes with all elements in $H$, we have $e \in Z_G(H)$. For all $z \in Z_G(H)$, since for all $h \in H$, $z^{-1}h=z^{-1}hzz^{-1}=z^{-1}zhz^{-1}=hz^{-1}$, then $z^{-1} \in Z_G(H)$. Finally, given $z_1,z_2 \in Z_G(H)$, for all $h \in H$, we have $(z_1z_2)h=z_1hz_2=h(z_1z_2)$, then $z_1z_2 \in Z_G(H)$. In conclusion, 
$Z_G(H)$ is a subgroup of $G$. \\

pf ($N_G(H)$). Since $eHe^{-1}=eHe=H$, we have $e \in Z_G(H)$. For all $n \in N_G(H)$, since $n^{-1}H(n^{-1})^{-1}=n^{-1}Hn=n^{-1}(nHn^{-1})n=H$, then $n^{-1} \in N_G(H)$. Finally, given $n_1,n_2 \in N_G(H)$, we have $(n_1n_2)H(n_1n_2)^{-1}=n_1(n_2Hn_2^{-1})n_1^{-1}=n_1Hn_1^{-1}=H$, then $n_1n_2 \in N_G(H)$. In conclusion, $N_G(H)$ is a subgroup of $G$. \\ \\

\textbf{Theorem 9. (Sylow's 1st theorem)} \\

Let $G$ be a finite group, let $p$ be a prime number and let $p^n$ be the largest power of $p$ that divides $\#G$, i.e. $\#G=p^nm$ where $p \nmid m$, or say $p^{n+1} \nmid \#G$. Then, $G$ has a subgroup of order $p^n$. \\

pf. Let $S=\{A \subset G:\#A = p^n\}$, i.e. $S$ is the collection of all "subsets" of $G$ whose cardinality divides $p^n$. Then, $\#S=\binom{p^nm}{p^n}$. Consider $G \acts S$, defined by $\forall g \in G$, $\forall A \in S$, $g \cdot A=gA=\{g \cdot a:a \in A\}$. We can easily check that $a \mapsto ga$ is a bijection. (Sujection is trivial, and if $g \cdot a_1=g \cdot a_2$, then $a_1 = a_2$ since
$g^{-1}$ exists, then it is injective). Thus, $\#gA=p^n$, $gA \in S$, this is a well-defined group action. ($e \cdot A=A$, $(g_1g_2) \cdot A=g_1\cdot(g_2\cdot A)$ since $(g_1g_2)a=g_1(g_2a)$ for all $a \in A$)\\

Consider $A_1,...,A_k \in S$ such that $S=\bigsqcup_{i=1}^k G \cdot A_i$, then by orbit-stabilizer counting theorem, $\#G=\sum_{i=1}^k\#G/\#G_{A_i}$, then we have:

\[
    \binom{p^nm}{p^n}=\sum_{i=1}^k \dfrac{p^nm}{\#G_{A_i}}
\]

Use the fact that $\binom{p^nm}{p^n}$ is not divisible by $p$, which will be proved in lemma 10. Then, it follows the result that there is a $A_i \in S$ such that $\#G_{A_i}$ divides $p^n$. \\

Moreover, given any $g \in G_{A_i}$, we have $gA_i=A_i$, i.e. $ga \in A_i$ for all $a \in A_i$. Then, we have $g \in A_ia^{-1}$, i.e. $G_{A_i} \subseteq A_ia^{-1}$. Therefore, $\#G_{A_i} \le \#(A_ia^{-1})=p^n$ (Cosets have the same number of elements as the group). Since $\#G_{A_i}$ divides $p^n$, we have $\#G_{A_i}=p^n$, and this is the subgroup we desire. \\ \\

\textbf{Lemma 10. }Suppose $p$ is prime and $m \ge 1$ with $p \nmid m$, then for any $n \in \mathbb{Z}_{\ge 0}$, $\binom{p^nm}{p^n}$ is not divisible by $p$. \\

pf. We compute:

\[
    \binom{p^nm}{p^n}=\dfrac{p^nm\cdot(p^nm-1)\cdot...\cdot(p^nm-p^n+1)}{p^n!}=\prod_{r=0}^{p^n-1}\left( \dfrac{p^nm-r}{p^n-r}\right)
\]

If $r=0$, then $\dfrac{p^nm}{p^n}=m$ is not divisible by $p$. If $r>0$, suppose $r=p^is$ with $0 \le i < n$ and $p \nmid s$, then:

\[
    \dfrac{p^nm-i}{p^n-r}=\dfrac{p^nm-p^is}{p^n-p^is}=\dfrac{p^{n-i}m-s}{p^{n-i}-s}
\]

Since $p \nmid s$, $p^{n-1}m-s$ is not divisible by $p$, the whole term is not divisible by $p$. Hence, $\binom{p^nm}{p^n}$ is not divisible by $p$. \\ \\

\textbf{Def. }Let $G$ be a group of order $p^nm$ with the prime $p \nmid m$. A subgroup $H \subseteq G$ is called a $p$-\textbf{Sylow subgroup} of $G$ if $\#H=p^n$. The collection of a $p$-Sylow subgroups of $G$ is denoted by $\Syl_p(G)$. Sylow's 1st theorem tells us $\Syl_p(G) \ne \phi$ if $p$ is a prime factor of $\#G$.

\subsection*{6.4 Two counting lemmas and double cosets}
\indent

\textbf{Def. }Let $G$ be a finite group and let $H_1,H_2 \subseteq G$ be a subgroup. We say $H_2$ is a \textbf{conjugate} of $H_1$ if there exists $g \in G$ such that $H_2=gH_1g^{-1}$. We denote the conjugates of $H_1$ as $\Conj(H)=\{gH_1g^{-1}:g \in G\}$. \\ \\

\textbf{Lemma 11. }Let $G$ be a finite group and let $H \subseteq G$ be a subgroup, then $H$ has exactly $\#G/\#N_G(H)$ distinct conjugates in $G$. \\

pf. Clearly, since the normalizer of $H$ in $G$, $N_G(H)=\{g \in G:gHg^{-1}=H\}$, is a subgroup of $G$, a simple way to show that $\Conj(H)=\#G/\#N_G(H)=\#(G/N_G(H))$ is to construct a bijection between $G/N_G(H)$ and $\Conj(H)$. \\

Consider $\phi:G/N_G(H) \to \Conj(H)$, $\phi(gN_G(H))=gHg^{-1}$, we should check if it is well-defined. Given $g_1,g_2 \in G$ such that $g_1N_G(H)=g_2N_G(H)$, then there exists a $n \in N_G(H)$ such that $g_2=g_1n$. Then, $\phi(g_2N_G(H))=g_2Hg_2^{-1}=g_1(nHn^{-1})g_1^{-1}=g_1Hg_1^{-1}=\phi(g_1N_G(H))$, $\phi$ is well-defined. \\

By the definition of $\phi$, it is surjective. Suppose $\phi(g_1N_G(H))=\phi(g_2N_G(H))$ for some $g_1,g_2 \in G$, then $g_1Hg_1^{-1}=g_2Hg_2^{-1}$, $(g_1^{-1}g_2)H(g_1^{-1}g_2)^{-1}=H$. Thus, $g_1^{-1}g_2 \in N_G(H)$, i.e. $g_2=g_1n$ for some $n \in N_G(H)$, $g_1N_G(H)=g_2N_G(H)$. In conclusion, $\phi$ is a bijective, then we have:

\[
    \#\Conj(H)=\#(G/N_G(H))=\#G/\#N_G(H)
\]

\textbf{Lemma 12. }Let $D$ be a finite group and $A \subseteq D$, $B \subseteq D$ be subgroups. Let $AB=\{ab:a \in A,\ b \in B\}$, then $\#AB=\#A\#B/\#(A \cap B)$. \\

pf. We notice that $A \times B$ and $AB$ are different, but $\#(A \times B)=\#A\#B$. We want to find a subgroup of $A \times B$ that has the same order as $A \cap B$. Clearly, the candidate is $H:=\{(c,c):c \in A \cap B\}$. \\

We show that $H$ is a subgroup of $A \times B$. First, $(e,e) \in H$, and for all $(c,c) \in H$, $(c^{-1},c^{-1})$ is also in $H$. Finally, $\forall (c_1,c_1),(c_2,c_2) \in H$, $(c_1 \cdot c_1,c_2 \cdot c_2) \in H$. Thus, $H$ is a subgroup of $A \times B$. \\

So, consider the map $\phi:(A \times B)/H \to AB$, $\phi((a,b)H)=ab^{-1}$, we should check if it is well-defined. Given $(a_1,b_1),(a_2,b_1) \in A \times B$ such that $(a_1,b_1)H=(a_2,b_2)H$, then there exists a $(c,c) \in H$ such that $(a_2,b_2)=(a_1,b_1) \cdot (c,c)=(a_1c,b_1c)$. Then, $\phi((a_2,b_2)H)=a_2b_2^{-1}=(a_1c)(b_1c)^{-1}=a_1cc^{-1}b_1^{-1}=a_1b_1^{-1}=\phi((a_1,b_1)H)$, $\phi$ is well-defined. \\

Given any $ab \in AB$, we have $\phi((a,b^{-1})H)=ab$, so $\phi$ is surjective. Suppose $\phi((a_1,b_1)H)=\phi((a_2,b_2)H)$ for some $(a_1,b_1),(a_2,b_2) \in A \times B$, then $a_1b_1^{-1}=a_2b_2^{-1}$, Let $a_1^{-1}a_2=b_1^{-1}b_2=c$, we have $(a_2,b_2)=(a_1c,b_1c)=(a_1,b_1) \cdot (c,c)$
, i.e. $(a_1,b_1)H=(a_2,b_2)H$. In conclusion, $\phi$ is bijective, then we have:

\[
    \#AB=\#((A \times B)/H)=\#(A \times B)/\#H=\#A\#B/\#(A \cap B)
\] \\

\textbf{Def. }Let $G$ be a group and $H_1 \subseteq G$, $H_2 \subseteq G$ be subgroups. The \textbf{double coset} associated to $g \in G$ is the set $H_1gH_2=\{h_1gh_2:h_1 \in H_1,\ h_2 \in H_2\}$. \\ \\

\textbf{Remark. }Regarding to $H_1,H_2$, define a relation "$\sim$": for all $x,y \in G$, $x \sim y$ if $y=h_1xh_2$ for some $h_1 \in H_1$, $h_2 \in H_2$, then $\sim$ is an equivalence relation. \\

pf. Since for all $x \in G$, $x=exe$ and $e \in H_1$, $e \in H_2$, it is reflexive. If $x \sim y$, then $y=h_1xh_2$ for some $h_1 \in H_1$, $h_2 \in H_2$. Then, $x=h_1^{-1}yh_2^{-1}$ and $h_1^{-1} \in H_1$, $h_2^{-1} \in H_2$. Thus, $y \sim x$, it is symmetric. Finally, if $x \sim y$ and $y \sim z$, then $y=h_1xh_2$ and $z=h_1'yh_2'$ for some $h_1,h_1' \in H_1$, $h_2,h_2' \in H_2$. Then, $z=h_1'yh_2'=(h_1'h_1)x(h_2'h_2)$ and $h_1'h_1 \in H_1$, $h_2'h_2 \in H)2$. Thus, $x \sim z$, it is transitive. In conclusion, it is an equivalence relation. \\

Note. Since it is an equivalence relation, we can find $g_1,...,g_n \in G$, distinct, such that $G=\bigsqcup_{i=1}^n H_1g_iH_2$. \\ \\

\textbf{Lemma 13. }Let $G$ be a group and $H_1,H_2$ be subgroups of $G$, $g \in G$. Then, $\#H_1gH_2=\#H_1\#H_2/\#(g^{-1}H_1g \cap H_2)$. It is important to be aware that $\#H_1g\#H_2 \ne \#H_1\#H_2/\#(H_1 \cap H_2)$ in general. So, each double coset (assciated to $g \in G$) may not have the same number of elements. \\

pf. Since $g^{-1}H_1g$ is a group in $G$ and $H_1 \to g^{-1}H_1g$ is a bijection (both are easy to check), we have $\#H_1\#H_2/\#(g_1^{-1}H_1g \cap H_2)=\#((g^{-1}H_1g)H_2)$. Consider the map $\phi:H_1gH_2 \to (g^{-1}H_1g)H_2$, $\phi:x \mapsto g^{-1}x$. Since the map $\psi:(g^{-1}H_1g)H_2 \to H_1gH_2$, $\psi:y \mapsto gy$ is the inverse of $\phi$ and $\phi \circ \psi=\id$, $\psi \circ \phi=\id$, $\phi$ is invertible and thus bijective. Then, we have:

\[
    \#H_1gH_2=\#g^{-1}H_1gH_2=\#(g^{-1}H_1g)\#H_2/\#((g^{-1}H_1g) \cap H_2)=\#H_1\#H_2/\#(g^{-1}H_1g_1 \cap H_2)
\]

\subsection*{6.5 Sylow's theorem}
\indent

\textbf{Theorem 14. (Sylow's theorem)} \\

Let $G$ be a finite group of order $p^nm$ with $p \nmid m$, then: \\

(i) $\Syl_p(G):=\{H \subseteq G:H \text{ is a subgroup of order }p^n\}$, then $\Syl_p(G) \ne \phi$. \\

(ii) For $H_1,H_2 \subseteq \Syl_p(G)$, there exists some $g \in G$ such that $H_2=gH_1g^{-1}$. In particular, if $\#\Syl_p(G)=1$, then the subgroup in it is normal. \\

(iii) Define $n_p=\#\Syl_p(G)$, then $n_p=\#G/\#N_G(H)$ for any $H \in \Syl_p(G)$. In particular, $n_p \mid \#G$. Moreover, $n_p \equiv 1$ (mod $p$) \\

pf-1. This is the restatement of theorem 9. \\

pf-2. Given $g \in G$, consider the double coset $H_1gH_2$, by lemma 13, we have:

\[
    \#(H_1gH_2)=\#H_1\#H_2/\#(g^{-1}H_1g \cap H_2)
\]

Since $H_1,H_2 \in \Syl_p(G)$, $\#H_1=\#H_2=p^n$. Also, $g^{-1}H_1g \cap H_2$ is a subgroup of $H_2$, so $\#(g^{-1}H_1gH_2) \mid p^n$, say $\#(g^{-1}H_1gH_2)=p^e$ for some $0 \le e \le n$. Then:

\[
    \#(H_1gH_2)=\dfrac{p^{2n}}{p^e}=p^{n+t},\ 0 \le t=n-e \le n
\]

Since $G=\bigsqcup_{i=1}^r H_1g_iH_2$ for some $g_1,...,g_r \in G$, we have:

\[
    \#G=\sum_{i=1}^r p^{n+t_i},\ 0 \le t_i \le n
\]

Since $\#G$ can not divide $p^{n+1}$, there exists $i$ such that $t_i=0$, i.e. $e_i=n$. Then, $\#(g_i^{-1}H_1g_i \cap H_2)=\#H_2$ and $g_i^{-1}H_1g_i \cap H_2$ is a subgroup of $H_2$, we have $g_i^{-1}H_1g_i \cap H_2=H_2$, $H_2 \subseteq g_i^{-1}H_1g_i$. However, since $H_1 \simeq g_i^{-1}H_1g_i$, we have $\#H_2=p^2=\#H_1=\#(g_i^{-1}H_1g_i)$. Therefore, there exists some $g \in G$ such that $H_2=gH_1g^{-1}$, i.e. $H_2 \in \Conj(H_1)$. \\

In particular, if $\Syl_p(G)=\{H\}$, ($\#\Syl_p(G)=1$) then $gHg^{-1}$ can only be $H$ (since $\#H=\#gHg^{-1}$), i.e. $H \trianglelefteq G$. \\

pf-3. The first part is easy, let $n_p=\#\Syl_p(G)$, by 1st Sylow, $n_p \ge 1$ and $\exists H \in \Syl_p(G)$. By 2nd Sylow, $\Syl_p(G)=\Conj(H)=\{gHg^{-1}:g \in G\}$. By lemma 11, we have:

\[
    n_p=\#\Syl_p(G)=\#\Conj(H)=\#G/\#N_G(H)
\]

For the next part, we use the fact that $\#G=\sum_{i=1}^r p^{n+t_i}$, $0 \le t_i \le n$, we write it as:

\[
    \#G=\sum_{i=1}^r p^{n+t_i}=\sum_{i=1 \atop t_i=0}^r p^n+\sum_{i=1 \atop t_i \ne 0}^r p^{n+t_i}
\]

Here is the question: how many $i$ such that $t_i=0$? In pf-2, we know that if $H_1,H_2 \in \Syl_p(G)$, then $t_i=0 \Leftrightarrow H_2=g_i^{-1}H_1g_i$. We take $H_1=H_2=H$ for some $H \in \Syl_p(G)$, then $t_i=0 \Leftrightarrow H=g_i^{-1}Hg_i \Leftrightarrow g_i^{-1} \in N_G(H) \Leftrightarrow g_i \in N_G(H)$ since $N_G(H)$ is a subgroup of $G$. \\

Now, since $g_i \in N_G(H)$, we find that the double coset $Hg_iH$ equals the left coset $g_iH$, and thus $\#Hg_iH=\#g_iH=\#H$. Recall that the choice of $g_i$ is distinct and satisfies $G=\bigsqcup_{i=1}^r Hg_iH$, we want to count how many such $g_i$ are in $N_G(H)$. By the discussion above, the number of distinct double cosets is the exactly the same of the number of distinct left coset, i.e. we should count how many left cosets $g_iH$ when $g_i$ is take from $N_G(H)$, for $1 \le i \le r$. The answer is to check $\#(N_G(H)/H)$ since $N_G(H)/H$ is a subset of $G/H$ such that each left coset $g_iH$ in $N_G(H)/H$ satisfies $g_i \in N_G(H)$. Thus, we have:

\[
    \#\{1 \le i \le r:t_i=0\}=\#(N_G(H)/H)=\#N_G(H)/\#H
\]

Then, we have:

\begin{align*}
    \#G&=\sum_{i=1}^r p^{n+t_i}=\sum_{i=1 \atop t_i=0}^r p^n+\sum_{i=1 \atop t_i \ne 0}^r p^{n+t_i} \\
    &=\dfrac{\#N_G(H)}{\#H} \cdot p^n+(\text{A multiple of $p^{n+1}$}) \\
    &=\#N_G(H)+(\text{A multiple of $p^{n+1}$})
\end{align*}

Since $H \subseteq N_G(H) \subseteq G$ ($\forall h \in H$, $hHh^{-1}=H$), the exact power of $p$ dividing $\#N_G(H)$ is $p^n$, so we have:

\[
    n_p=\dfrac{\#G}{\#N_G(H)}=1+(\text{A multiple of $p$})
\]

, that is, $n_p \equiv 1$ (mod $p$).

\subsection*{6.6 Examples of applying Sylow's theorem}
\indent

\textbf{Example. }Let $G$ be a group of order $10=2 \cdot 5$, then there is a nontrivial normal subgroup of $G$. \\

pf. By Sylow 1st, we have $\Syl_5(G) \ne \phi$, and by Sylow 3rd, let $n_5=\#\Syl_5(G)$, then $n_5=\#G/\#N_G(H)$. Since $H \subseteq N_G(H) \subseteq G$, we have $\#N_G(H) \ge 5$ and $\#N_G(H) \mid \#G=10$, i.e. $n_5 \mid 10/5=2$. Also by Sylow 3rd, $n_5 \equiv 1$ (mod $5$), then $n_5=1$, the only $H \in \Syl_5(G)$ is normal. \\ \\

\textbf{Example 15. }Let $G$ be a group of order $pq$, $p,q$ are primes with $p>q$, then there is a nontrivial normal subgroup of $G$. \\

pf. By Sylow 1st, we have $\Syl_p(G) \ne \phi$, and by Sylow 3rd, let $n_p=\#\Syl_p(G)$, then $n_p \mid q$ and $n_p \equiv 1$ (mod $p$). Thus, $n_p=1$ since $p>q$, then the only $H \in \Syl_p(G)$ is normal. \\ \\

\textbf{Remark. }Assume furthermore in example-15, if $q \nmid p-1$, then $G$ is abelian and hence cyclic. \\

pf. First, let $n_q=\#\Syl_q(G)$, then $n_q \mid p$ (then $n_q=1$ or $n_q=p$) and $n_q \equiv 1$ (mod $q$). Since $q \nmid p-1$, we have $n_q=1$. \\

Let the only group in $\Syl_p(G)$ be $H$ and the only group in $\Syl_q(G)$ be $Q$. Since $p,q$ are primes, we have $H=\langle h \rangle$ and $Q=\langle l \rangle$ for some $h \in H$, $l \in Q$. Now claim: $hl=lh$, and it'll give us $G$ is abelian and $G=\langle hl \rangle$ ($\gcd(\ord(h),\ord(l))=\gcd(p,q)=1$, so $\ord(hl)=pq$). \\

Consider $H \acts Q$ by conjugation, i.e. $\forall h \in H$, $\forall l \in Q$, $h \cdot l=hlh^{-1}$. Then, $h$ is equivalent to the automorphism $\phi_h:Q \to Q$, $\phi_h:l \mapsto hlh^{-1}$. \\

Then, $\Pi:H \to \Aut(Q)$, $\Pi:h^n \mapsto \phi_{h^n}$, is a homomorphism, which gives us $\Img\Pi$ is a subgroup of $\Aut(Q)$. Then the first theorem of isomorphism gives us $H/\ker\Pi \simeq \Img\Pi$. Thus, $\#H/\#\ker\Pi=\#\Img(\Pi)$, then $\#\Img\Pi \mid \#H$, i.e. $\#\Img\Pi=1\text{ or }p$. But in section 6.2, we know that $\#\Aut(Q)=q-1$, then $\#\Img\Pi \le q-1$, i.e. $\#\Img\Pi=1$. Finally, we get $\#\ker\Pi=\#H$, i.e. $\ker\Pi=H$, then for any $h^i \in H$, for all $l \in Q$, we have $h^il(h^i)^{-1}=l$, i.e. $h^il=lh^i$, $G$ is abelian. \\ \\

\textbf{Example 16. }Let $G$ be a group of order $12=2^2 \cdot 3$, then there is a nontrivial normal subgroup of $G$. \\

pf. By Sylow 1st, we have $\Syl_3(G) \ne \phi$, and by Sylow 3rd, let $n_3=\#\Syl_3(G)$, then $n_3 \mid 4$ and $n_3 \equiv 1$ (mod $3$). Thus, $n_3=1$ or $n_3=4$. \\

If $n_3=1$, then the only group in $\Syl_3(G)$ is normal. If $n_3=4$, then there are $4$ cyclic subgroups of $G$. Given $H_1,H_2 \in \Syl_3(G)$, distinct, then since $\#(H_1 \cap H_2) \mid \#H_1=3$ and $H_1 \ne H_2$, we have $H_1 \cap H_2=\{e\}$. Each $H \in \Syl_3(G)$ has $2$ elements of order $3$ (since $\ord(e)=1$), thus, there are $(3-1) \cdot 4=8$ elements having order $3$. Again by Sylow 1st, $\Syl_2(G) \ne \phi$. Consider $Q \in \Syl_2(G)$, then $\#Q=4$. Thus, there is only one Sylow $2$-subgroup ($8+4=12$), then $Q \trianglelefteq G$. \\ \\

\textbf{Example 17. }Let $G$ be a group of order $p^2q$, $p,q$ are primes, then there is a nontrivial normal subgroup of $G$. \\

pf. We break into two case: \\

Case 1. Suppose $p>q$, by Sylow 1st, we have $\Syl_p(G) \ne \phi$, and by Sylow 3rd, let $n_p=\#\Syl_p(G)$, then $n_p \mid q$ and $n_p \equiv 1$ (mod $p$). Thus, $n_p=1$, i.e. the only group in $\Syl_p(G)$ is normal. \\

Case 2. Suppose $p<q$, By Sylow 1st, we have $\Syl_q(G) \ne \phi$, and by Sylow 3rd, let $n_q=\#\Syl_q(G)$, then $n_q \mid p^2$ and $n_q \equiv 1$ (mod $q$). Thus, $n_q=1$, $n_q=p$, or $n_q=p^2$. \\

If $n_q=1$, then we're done. If $n_q=p$, we have assumed $p<q$, then $n_q \equiv p$ (mode $q$), which is not true. Finally, if $n_q=p^2$, then $p^2-1 \equiv 0$ (mod $q$), i.e. $q \mid (p+1)$ or $q \mid (p-1)$. Since $p<q$, we know $q \nmid (p-1)$, then we have $q \mid (p+1)$. Then, we have $q \le p+1$ and $p<q$, which tells us $q=p+1$. This situation only occurs as $p=2,q=3$, by example 16, there is a normal subgroup of order $3$. \\ \\

\textbf{Example 18-a. }Let $G$ be a group of order $30=2 \cdot 3 \cdot 5$, then $G$ has a normal Sylow $3$-subgroup or a normal Sylow $5$-subgroup. \\

pf. By Sylow 1st, we have $\Syl_3(G) \ne \phi$ and $\Syl_5(G) \ne \phi$, and by Sylow 3rd, let $n_3=\#\Syl_3(G)$, $n_5=\#\Syl_5(G)$, then $n_3 \mid 10$, $n_5 \mid 6$ and $n_3 \equiv 1$ (mod $3$), $n_5 \equiv 1$ (mod $5$). Thus, $n_3=1$ or $n_3=10$, $n_5=1$ or $n_5=6$. If either of them is $1$, then we're done. If $n_3=10$ and $n_5=6$, then there are $(3-1) \cdot 10=20$ elements of order $3$ and $(5-1) \cdot 6=24$ elements of order $5$. But, $20+24>30=\#G$, it cannot be true. Then, we conclude that there is a normal Sylow $3$-subgroup or a normal Sylow $5$-subgroup. \\ \\

\textbf{Example 18-b. }Consider $P \in \Syl_3(G)$, $Q \in \Syl_5(G)$, then $PQ$ is cyclic and normal. \\

pf. Since $\#P=3$, $\#Q=5$ and $\gcd(3,5)=1$, we know that $PQ$ is cyclic and abelian by example 15. We now show that if $H$ is a subgroup of $G$ and $\#(G/H)=2$, then $H \trianglelefteq G$. \\

First, $G=H \sqcup gH$ for any $g \in G$ and $g \notin H$. We know that $Hg \ne H$ since if NOT, then $Hg=H$, i.e. $g \in H$, which is a contradiction. But $G$ is also $H \sqcup Hg$, so we conclude that $gH=Hg$, i.e. $gHg^{-1}=H$, for any $g \in G$ and $g \notin H$. For $g \in H$, $gHg^{-1}=H$ is trivial. Thus, $H \trianglelefteq G$. \\

Now, back to our proof, since $\#(G/PQ)=30/15=2$, we have $PQ \trianglelefteq G$. \\ \\

\textbf{Example 18-c. }$PQ$ has a normal Sylow $5$-subgroup by example 15, which means $G$ has a normal Sylow $5$-subgroup. \\ \\



\textbf{Prop 19. }Let $G$ be a group of order $60=2^2 \cdot 3 \cdot 5$. Suppose $G$ has more than one Sylow $5$-subgroup, then $G$ is \textbf{simple}, i.e. $G$ has no normal subgroup other than $\{e\}$ and $G$. \\

pf. First, suppose $H \trianglelefteq G$ and $1 < \#H < \#G$. We break into $3$ cases. \\

\textbf{Case 1.} $5 \mid \#H$ \\

By Sylow 1st, $\exists P \in \Syl_5(H)$, and the exact power of $5$ of $P$ is $1$ since $\#H \mid \#G$. Then, we have $P \in \Syl_5(G)$, $\Syl_5(H) \subseteq \Syl_5(G)$. On the other hand, $\forall P' \in \Syl_5(G)$, since $P$ is also in $\Syl_5(G)$, by Sylow 2nd, we have $P' \in \Conj(P)$, i.e. $P'=gPg^{-1}$ for some $g \in G$. However, $H \trianglelefteq G$, we have $P'=gPg^{-1} \subseteq gHg^{-1}=H$, i.e. $\Syl_5(G) \subseteq \Syl_5(H)$. In conclusion, $\Syl_5(H)=\Syl_5(G)$. \\

Let $n_5=\#\Syl_5(G)$, then by Sylow 3rd, $n_5 \mid 12$ and $n_5 \equiv 1$ (mod $5$). We soon have $n_5=1$ or $n_5=6$. By assumption, $n_5 \ne 1$, but if $n_5=6$, then there is $(5-1) \cdot 6=24$ elements of order $5$, and these elements are in the groups in $\Syl_5(H)$, and hence in $H$. By Lagrange theorem, we have $\#H=30$ since we assume $\#H \ne 60$. By example 18-a, $H$ has a normal $5$-Sylow subgroup, that is, $n_5=1$, which contradicts our assumption $n_5=6$. So, case 1 does not hold. \\

\text{Case 2.} From case 1 and Lagrange theorem, $\#H \mid 12$. Here, consider $\#H=6$ or $\#H=12$. \\

If $\#H=6$, then by example 15, $H$ has a normal $3$-Sylow subgroup, which reduces to case 3. \\

If $\#H=12$, then by example 16 (or 17), $H$ has a normal $2$-Sylow subgroup or a normal $3$-Sylow subgroup, which also reduces to case 3. \\

\textbf{Case 3.} $\#H=2,3,4$. \\

By Lagrange theorem, $\#(G/H)=30,20,15$. Since $H \trianglelefteq G$, $G/H$ is a group, denoted as $\overline{G}$. \\

If $\#H=2$, i.e. $\#\overline{G}=30$, then by example 18, $\overline{G}$ has a normal $5$-Sylow subgroup $\overline{K}$. \\

If $\#H=3$, i.e. $\#\overline{G}=20=2^2 \cdot 5$. By Sylow's theorem, $n_5=\#\Syl_5(\overline{G}) \mid 4$ and $n_5 \equiv 1$ (mod $5$). Then, $\overline{G}$ has a normal $5$-Sylow subgroup $\overline{K}$. \\

If $\#H=4$, i.e. $\#\overline{G}=15=3 \cdot 5$, then by example 16, $\overline{G}$ has a normal $5$-Sylow subgroup $\overline{K}$. \\

So, we deal with the situation that $\overline{G}$ has a normal $5$-Sylow subgroup $\overline{K}$. Consider $\Pi:G \to \overline{G}$, $\Pi:g \to gH$, which is a group homomorphism. Then since $\overline{K} \trianglelefteq \overline{G}$, Since $\overline{K}$ is a group, we have the preimage $\Pi^{-1}(\overline{K})$ is also a group. Moreover, we can easily show that $\Pi^{-1}(\overline{K}) \trianglelefteq G$ since $\overline{K} \trianglelefteq \overline{G}$\\


Consider the homomorphism $\Pi'=\Pi\vert_{\Pi^{-1}(\overline{K})}$, by the first theorem of isomorphism, $\Pi^{-1}(K)/\ker(\Pi') \simeq \overline{K}$, then $5=\#\overline{K} \mid \#\Pi^{-1}(\overline{K})$. But $\#\Pi^{-1}(K)$ is a normal subgroup of $G$, then by case 1, it is a contradiction. \\

\textbf{In conclusion, }$G$ is simple.

\newpage



\end{document}