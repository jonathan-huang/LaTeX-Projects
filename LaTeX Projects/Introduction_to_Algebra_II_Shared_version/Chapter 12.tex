\documentclass[12pt]{article}

% Language setting
% Replace `english' with e.g. `spanish' to change the document language
\usepackage[english]{babel}

% Set page size and margins
% Replace `letterpaper' with `a4paper' for UK/EU standard size
\usepackage[letterpaper,top=2cm,bottom=2cm,left=3cm,right=3cm,marginparwidth=1.75cm]{geometry}

% Useful packages
\usepackage{amsmath}
\usepackage{graphicx}
\usepackage{mathtools}
\usepackage{amsfonts}
\usepackage{algorithm}
\usepackage{algorithmicx}
\usepackage[noend]{algpseudocode}
\usepackage[colorlinks=true, allcolors=blue]{hyperref}
\usepackage{bm}
\usepackage{amssymb}
\usepackage{esint}
\usepackage{tikz}
\usepackage{hyperref}
\DeclareMathOperator{\ord}{ord}
\DeclareMathOperator{\chr}{char}
\DeclareMathOperator{\spn}{span}
\DeclareMathOperator{\Img}{Im}
\DeclareMathOperator{\Syl}{Syl}
\DeclareMathOperator{\Conj}{Conj}
\DeclareMathOperator{\Aut}{Aut}
\DeclareMathOperator{\id}{id}
\DeclareMathOperator{\sign}{sign}
\DeclareMathOperator{\Inn}{Inn}
\DeclareMathOperator{\Out}{Out}
\DeclareMathOperator{\Frac}{Frac}
\DeclareMathOperator{\pre}{pre}
\DeclareMathOperator{\Gal}{Gal}
\DeclareMathOperator{\Orb}{Orb}
\DeclareMathOperator{\Stab}{Stab}
\def\acts{\curvearrowright}
\usetikzlibrary{tikzmark}
\newcommand{\surj}[0]{\xrightarrow[]{}\mathrel{\mkern-14mu}\rightarrow}
\newcommand{\inj}[0]{\xhookrightarrow{}}
\newcommand{\tikzarc}[1]{%
\tikzmarknode{a}{#1}
\begin{tikzpicture}[overlay,remember picture]
\draw ([yshift=1pt]a.north west) to[bend left=20] ([yshift=1pt]a.north east);
\end{tikzpicture}%
}

\title{Introduction to Algebra (II)}
\author{Cheng-Yun Yeh}

\begin{document}
\maketitle

\section*{Chapter 12. Groups (III)}
\subsection*{12.1 Permutation groups}
\indent

\textbf{Def. }Let $X$ be a finite set, define the \textbf{permutation group} on $X$ to be $\mathcal{S}_X=\{\pi:\text{$\pi$ is a bijection $X \to X$}\}$ with composition law of functions. We often discuss $\mathcal{S}_n=\{\pi:\text{$\pi$ is a bijection $\{1,...,n\} \to \{1,...,n\}$}\}$ since $\mathcal{S}_X \simeq \mathcal{S}_n$ if $\lvert X \rvert=n$.\\ \\

\textbf{Prop 1. }Let $X=\{1,...,n\}$ and $\mathcal{S}_n$ be the permutation group on $X$, then: \\

(i) $\#\mathcal{S}_n=n!$

(ii) $\mathcal{S}_n$ is nonabelian for $n \ge 3$ \\

pf-1. $\pi(1)$ has $n$ choices, $\pi(2)$ has $(n-1)$ choices,..., $\pi(n)$ has $1$ choice. Thus, there are $n!$ permutations of $X$, $\#\mathcal{S}_n=n!$. \\

pf-2. For $n \ge 3$, consider a $\pi_1$, $\pi_2$ defined as:

\begin{gather*}
    \pi_1(1)=2,\ \pi_1(2)=1,\ \pi_1(3)=3,\ \pi_1(k)=k,\ \forall k \ge 4 \\
    \pi_2(1)=2,\ \pi_2(2)=3,\ \pi_2(3)=1,\ \pi_2(k)=k,\ \forall k \ge 4
\end{gather*}

Then, $(\pi_1\pi_2)(1)=1$, $(\pi_2\pi_1)(1)=3$, which means $\mathcal{S}_n$ is nonabelian, $\forall n \ge 3$. \\ \\

\textbf{Remark. }A \textbf{group action} $G \acts X$ is actually a group homomorphism $\phi:G \to \Aut(X)$. \\ \\

\textbf{Def. }Given $\pi \in \mathcal{S}_n$, suppose the $\pi$-orbit of $\alpha \in X$, has $k$ elements, i.e. $\langle \pi \rangle \cdot \alpha=\{\pi^{i}(\alpha):0 \le i \le k-1\}$, then it can be written as $(\alpha\ \pi(\alpha)\ \pi^2(\alpha)\ \cdots\ \pi^{k-1}(\alpha))$. \\ \\

\textbf{Def. }We say $\pi \in \mathcal{S}_n$ is a \textbf{cycle} of length $r$ if $\pi=(a_1\ \cdots\ a_r)$, each $a_i$ is distinct. That is:

\begin{gather*}
    \left\{\begin{matrix}
        \pi(a)&=&a &,&\ \text{if }a \notin \{a_1,...,a_r\} \\
        \pi(a_i)&=&a_{i+1} &,&\ \forall 1 \le i \le r-1 \\
        \pi(a_r)&=&a_1
    \end{matrix}\right.
\end{gather*}\\

\textbf{Def. }We say two cycles $\pi_1=(a_1\ \cdots\ a_r)$ and $\pi_2=(b_1\ \cdots\ b_s)$ are \textbf{disjoint} if $\{a_1,...,a_r\} \cap \{b_1,...,b_s\}=\phi$. \\ \\

\textbf{Lemma 2. }Let $\pi_1,\pi_2 \in \mathcal{S}_n$ are two disjoint cycles, then $\pi_1\pi_2=\pi_2\pi_1$. \\

pf. Given $x \in X=\{1,...,n\}$, suppose $\pi_1=(a_1,...,a_r)$, $\pi_2=(b_1,...,b_s)$, $A=\{a_1,...,a_r\}$, $B=\{b_1,...,b_s\}$, then $A \cap B=\phi$. \\

If $x=a$ and $\pi_1(a)=a'$ for some $a,a' \in A$, then $(\pi_1\pi_2)(a)=\pi_1(a)=a'=\pi_2(a')=(\pi_2\pi_1)(a)$. \\

If $x=b$ and $\pi_2(b)=b'$ for some $b,b' \in B$, then $(\pi_1\pi_2)(b)=\pi_1(b')=b'=\pi_2(b)=(\pi_2\pi_1)(b)$. \\

If $x=c$ for some $c \in X \backslash (A \cup B)$, then $(\pi_1\pi_2)(c)=\pi_1(c)=c=\pi_2(c)=(\pi_2\pi_1)(c)$. \\

In conclusion, $\pi_1\pi_2=\pi_2\pi_1$ since they give the same automorphism. \\ \\

\textbf{Theorem 3. }Let $\pi \in \mathcal{S}_n \backslash \{\id\}$, there are pairwisely disjoint cycles $\sigma_1,...,\sigma_k \in \mathcal{S}_n$ s.t. $\pi=\sigma_1...\sigma_k$ and each $\sigma_i$ has length larger than or equal to $2$. Moreover, the list of cycles $\sigma_1,...,\sigma_k$ is uniquely determined by $\pi$, up to listing them in a different order. We say the list is the \textbf{cycle decomposition} of $\pi$. \\

pf. Let $G=\langle \pi \rangle$, we know that $X=\bigsqcup_{i=1}^t G \cdot x_i$, for some $x_1,...,x_t \in X$. WLOG, we suppose $\#G \cdot x_i \ge 2$ for $1 \le i \le k$, and $\#G \cdot x_i=1$ for $i>k$. Let $\#G\cdot x_i=l_i$, for $1 \le i \le k$, then if $\sigma_i=(x_i\ \pi(x_i)\ \cdots\ \pi^{l_i-1}(x_i))$, we have $\pi=\sigma_1\sigma_2...\sigma_k$ with $\sigma_i$ and $\sigma_j$ being disjoint if $i \ne j$. Thus, the cycle decomposition of $\pi$ exists. \\

Moreover, given another cycle decomposition $\pi=\tau_1...\tau_m$. Since the decomposition is pariwisely disjoint, consider the aforementioned $x_i$, if $x_i \in X$ is in some $\tau_j$, then it is not in any $\tau_p$ for all $p \ne j$. Then, $\tau_j=G \cdot x_i$ since $\tau_j$ contains $\pi(x_i)$, $\pi^2(x_i),...$. Moreover, since every $x_i$ is in different cycle, we know that $x_1,...,x_k$ are in different $\tau_i$. By counting the elements, the sum of length of $\tau_i$ is equal to the sum of length of $\sigma_i$, then the remaining elements (neither in $\tau_i$ nor $\pi_i$) is fixed by $\pi$. Thus, the decomposition is unique. \\ \\

\textbf{Def. }A \textbf{transposition} $\tau \in \mathcal{S}_n$ is a cycle of length $2$. \\ \\

\textbf{Prop 4. }Let $\pi \in \mathcal{S}_n$, $n \ge 2$, then there are transpositions $\tau_1,...,\tau_m \in \mathcal{S}_n$ s.t. $\pi=\tau_1\tau_2...\tau_m$, but $m$ is NOT unique. (Example: $\id=(1\ 2)(1\ 2)=(1\ 2)(1\ 2)(1\ 2)(1\ 2)$) \\

pf. Given $\pi \in \mathcal{S}_n$, if $\pi=\id$, then $\pi=(1\ 2)(1\ 2)$, we're done. So WLOG, $\pi \ne \id$. By theorem 3, there are pairwisely disjoint cycles $\sigma_1,...,\sigma_k$ such that $\pi=\sigma_1...\sigma_k$. It suffices to show that each cycle of length larger than or equal to $2$ can be written as product of transpositions. \\

Let $\pi=(a_1\ \cdots\ a_r)$, $r \ge 2$, we claim that $\pi=\tau_r...\tau_3\tau_2$, where $\tau_i=(a_1\ a_i)$, $i=2,...,r$. Given $c \in X=\{1,...,n\}$, we consider the following cases: \\

1. If $c \notin \{a_1,...,a_r\}$, then $\pi(c)=c$, $\tau_1(c)=c$, $i=2,...,r$. Thus, $\pi(c)=(\tau_r...\tau_3\tau_2)(c)$. \\

2. If $c=a_j$, $j \ne 1$ and $j \ne r$, then $\pi(c)=a_{j+1}$, $(\tau_r...\tau_3\tau_2)(c)=...=(\tau_r...\tau_{j+1}\tau_j)(c)=(\tau_r...\tau_{j+1})(a_1)=(\tau_r...\tau_{j+2})(a_{j+1})=a_{j+1}$ (if $j+2>r$, then $\tau_{r+1}=r$ since $j \ne r$, we also have $\tau_r(a_1)=a_r=a_{j+1}$). Thus, $\pi(c)=(\tau_r...\tau_3\tau_2)(c)$. \\

3. If $c=a_1$, then $\pi(c)=a_2$, $\tau_2(c)=a_2$ and $\tau_i(a_2)=a_2$, $i=3,...,r$. Thus, $\pi(c)=(\tau_r...\tau_3\tau_2)(c)$.  \\

4. If $c=a_r$, then $\pi(c)=a_1$, $\tau_i(c)=c$, $i=2,...,r-1$, $\tau_r(c)=a_1$. Thus, $\pi(c)=(\tau_r...\tau_3\tau_2)(c)$. \\

In conclusion, $\pi=\tau_r...\tau_3\tau_2$, we're done. \\ \\

\textbf{Theorem 5. }Let $n \ge 2$ and $\pi \in \mathcal{S}_n$. Suppose that we write $\pi$ as a product of transpositions in 2 ways, say $\pi=\tau_1...\tau_k=\sigma_1...\sigma_l$ with $\tau_i$, $\sigma_j$ are transpositions, $\forall i,j$. Then, $k \equiv l$ (mod $2$). \\

pf. Consider the polynomial ring on $\mathbb{Z}$ in $n$ variables, say $R=\mathbb{Z}[x_1,...,x_n]$, and consider $\mathcal{S}_n \acts R$: if $\sigma \in \mathcal{S}_n$, $f(x_1,...,x_n) \in R$, then $(\sigma f)(x_1,...,x_n)=f(x_{\sigma(1)},...,x_{\sigma(n)})$. For instance, if $f(x_1,...,x_n)=3x_1^2+x_2$ and $\sigma=(1\ 3)(2\ 4)$, then $(\sigma f)(x_1,...,x_n)=3x_3^2+x_4$. \\

We first check this definition is actually a group action. For identity axiom, if $e=\id \in \mathcal{S}_n$, then:

\[
    (ef)(x_1,...,x_n)=f(x_{\id(1)},...,x_{\id(n)})=f(x_1,...,x_n)
\]

For associative law, if $\sigma,\tau \in \mathcal{S}_n$, then:

\[
    ((\sigma\tau)f)(x_1,...,x_n)=f(x_{(\sigma\tau)(1)},...,x_{(\sigma\tau)(n)})=f(x_{\sigma(\tau(1))},...,x_{\sigma(\tau(n))})=(\sigma(\tau(f)))(x_1,...,x_n)
\]

Thus, it is a group action, which means $\psi:\mathcal{S}_n \to \Aut(R)$, $\psi:\sigma \to [\phi_\sigma:f \mapsto \sigma f]$, is a group homomorphism. Also, we use the fact (which may be proved in chapter 7) that $\sigma(f+g)=\sigma f+\sigma g$ and $\sigma(fg)=(\sigma f)(\sigma g)$, i.e. $\phi_\sigma$ is a ring homomorphism. We now let:

\[
    D=\prod_{1 \le i<j \le n}(x_i-x_j)
\]

, where $D$ has $\binom{n}{2}$ terms. Consider $D^2$, let $\sigma$ be an element of $\mathcal{S}_n$, then we have:

\[
    \sigma(D^2)=(\sigma D)^2=\prod_{1 \le i < j \le n}(x_{\sigma(i)}-x_{\sigma(j)})^2=D^2
\]

, since $\sigma$ is a bijection. We thus have $\sigma D=\pm D$, $\forall \sigma \in \mathcal{S}_n$. Now we claim: if $\sigma$ is a transposition, then $\sigma D=-D$. WLOG, let $\sigma=(a\ b)$ with $a<b$, then:

\[
    \sigma(D)=\prod_{1 \le i < j \le n}(x_{\sigma(i)}-x_{\sigma(j)})
\]

We discuss the following case: \\

Case 1: $\{i,j\} \cap \{a,b\}=\phi$. Then $\sigma(x_i-x_j)=x_i-x_j$ (no negative). \\

Case 2-1: $\{i,j\} \cap \{a,b\}=a$, $i=a$ and $a<j<b$. Then $\sigma(x_i-x_j)=x_b-x_j=(-1)(x_j-x_b)$ (one negative). \\

Case 2-2: $\{i,j\} \cap \{a,b\}=a$, $i=a$ and $j>b$. Then $\sigma(x_i-x_j)=x_b-x_j$ (no negative). \\

Case 2-3: $\{i,j\} \cap \{a,b\}=a$, $j=a$ and we must have $i<a$. Then $\sigma(x_i-x_j)=x_i-x_b$ (no negative). \\

Case 3-1: $\{i,j\} \cap \{a,b\}=b$, $i=b$ and we must have $j>b$. Then $\sigma(x_i-x_j)=x_a-x_j$ (no negative). \\

Case 3-2: $\{i,j\} \cap \{a,b\}=b$, $j=b$ and $i<a$. Then $\sigma(x_i-x_j)=x_i-x_a$ (no negative). \\

Case 3-3: $\{i,j\} \cap \{a,b\}=b$, $j=b$ and $a<i<b$. Then $\sigma(x_i-x_j)=x_i-x_a=(-1)(x_a-x_i)$ (one negative). \\

Case 4: $\{i,j\} \cap \{a,b\}=\{a,b\}$, we must have $i=a,j=b$. Then $\sigma(x_i-x_j)=x_b-x_a=(-1)(x_a-x_b)$ (one negative). \\

In conclusion, $\sigma D$ has $0+(b-a-1)+(b-a-1)+1=2(b-a-1)+1$ negative $1$ on it. That is, $\sigma D=(-1)^{2(b-a-1)+1}D=-D$. \\

In general, we back to $\pi \in \mathcal{S}_n$ and $\pi=\tau_1...\tau_k=\sigma_1...\sigma_l$ with $\tau_i$, $\sigma_j$ are transpositions, $\forall i,j$. Then, we have:

\begin{gather*}
    \pi D=(\tau_1...\tau_k)D=(-1)^k D \\
    \pi D=(\sigma_1...\sigma_l)D=(-1)^l D
\end{gather*}

Thus, $(-1)^k=(-1)^l$, we have $k \equiv l$ (mod $2$). \\ \\

\textbf{Def. }Given $\pi \in \mathcal{S}_n$, if we write $\pi=\tau_1...\tau_r$ with each $\tau_i$ is a transposition, $i=1,...,r$, define $\sign(\pi)=(-1)^r$. By theorem 5, $\sign$ is well-defined. \\ \\

\textbf{Remark. }$\{0,1\}$ forms a group $G_1$ with addition, and $\{\pm 1\}$ forms a group $G_2$ with multiplication. . Moreover, $\phi:G_1 \to G_2$, $\phi:x \to (-1)^x$ is a group isomorphism. \\ \\

\textbf{Prop 6. }Let $G=(\{\pm 1\},\times)$ be a group. The map $\sign:\mathcal{S}_n \to G$ is a surjective group homomorphism for $n \ge 2$. \\

pf. Let $\pi_1=\tau_1...\tau_r$ and $\pi_2=\sigma_1...\sigma_s$ be transposition decompositions of $\pi_1,\pi_2 \in \mathcal{S}_n$, then:

\[
    \sign(\pi_1\cdot\pi_2)=\sign((\tau_1...\tau_r)(\sigma_1...\sigma_s))=(-1)^{r+s}=(-1)^r(-1)^s=\sign(\pi_1)\sign(\pi_2)
\]

Moreover, $\sign(\id)=1$ and $\sign((1\ 2))=-1$, it is surjective. \\ \\

\textbf{Def. }Let $n \ge 1$ and $\sign:\mathcal{S}_n \to (\{\pm1\},\times)$ be the sign map, then $\mathcal{A}_n:=\ker(\sign)$ is called the \textbf{alternating group}. That is, $\mathcal{A}_n$ collects permutations in $\mathcal{S}_n$ that have even number transpositions in their transposition decomposition. \\ \\

\textbf{Prop 7. }Let $n \ge 2$, then: \\

(i) $\mathcal{A}_n \trianglelefteq \mathcal{S}_n$.

(ii) We have an isomorphism $\mathcal{S}_n/\mathcal{A}_n \simeq \mathcal{C}_2$.

(iii) $[\mathcal{S}_n:\mathcal{A}_n]=2$. \\

pf-1. The kernel is normal. \\

pf-2. By the first isomorphism theorem, we have $\mathcal{S}_n/\mathcal{A}_n \simeq \Img(\sign)=(\{\pm1\},\times) \simeq \mathcal{C}_2$. \\

pf-3. From (ii), we have $[\mathcal{S}_n:\mathcal{A}_n]=\lvert \mathcal{C}_2 \rvert=2$. \\ \\

\textbf{Remark. }We know that a $k$-cycle $\pi=(a_1\ ...\ a_k)$ has transposition decomposition $\tau_k...\tau_3\tau_2$ where $\tau_i=(a_1\ a_i)$, $i=2,...,k$. Thus, $\sign(\pi)=(-1)^{k-1}$, which tells us if $k$ is odd, then $\pi \in \mathcal{A}_n$.

\end{document}