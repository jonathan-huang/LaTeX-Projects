\documentclass[12pt]{article}

% Language setting
% Replace `english' with e.g. `spanish' to change the document language
\usepackage[english]{babel}

% Set page size and margins
% Replace `letterpaper' with `a4paper' for UK/EU standard size
\usepackage[letterpaper,top=2cm,bottom=2cm,left=3cm,right=3cm,marginparwidth=1.75cm]{geometry}

% Useful packages
\usepackage{amsmath}
\usepackage{graphicx}
\usepackage{mathtools}
\usepackage{amsfonts}
\usepackage{algorithm}
\usepackage{algorithmicx}
\usepackage[noend]{algpseudocode}
\usepackage[colorlinks=true, allcolors=blue]{hyperref}
\usepackage{bm}
\usepackage{amssymb}
\usepackage{esint}
\usepackage{tikz}
\usepackage{hyperref}
\usepackage{verbatim}
\DeclareMathOperator{\ord}{ord}
\DeclareMathOperator{\chr}{char}
\DeclareMathOperator{\spn}{span}
\DeclareMathOperator{\Img}{Im}
\DeclareMathOperator{\Syl}{Syl}
\DeclareMathOperator{\Conj}{Conj}
\DeclareMathOperator{\Aut}{Aut}
\DeclareMathOperator{\id}{id}
\DeclareMathOperator{\sign}{sign}
\DeclareMathOperator{\Inn}{Inn}
\DeclareMathOperator{\Out}{Out}
\DeclareMathOperator{\Frac}{Frac}
\DeclareMathOperator{\pre}{pre}
\DeclareMathOperator{\Gal}{Gal}
\DeclareMathOperator{\Orb}{Orb}
\DeclareMathOperator{\Stab}{Stab}
\DeclareMathOperator{\disc}{disc}
\setcounter{MaxMatrixCols}{20}
\def\acts{\curvearrowright}
\usetikzlibrary{tikzmark}
\newcommand{\surj}[0]{\xrightarrow[]{}\mathrel{\mkern-14mu}\rightarrow}
\newcommand{\inj}[0]{\xhookrightarrow{}}
\newcommand{\tikzarc}[1]{%
\tikzmarknode{a}{#1}
\begin{tikzpicture}[overlay,remember picture]
\draw ([yshift=1pt]a.north west) to[bend left=20] ([yshift=1pt]a.north east);
\end{tikzpicture}%
}

\title{Introduction to Algebra (II)}
\author{Cheng-Yun Yeh}

\begin{document}
\maketitle

\section*{Chapter 9. Fields (III)}
\subsection*{9.6 Fundamental theorem of Galois theory}
\indent

\textbf{Theorem 21. (Fundamental theorem of Galois theory) }Let $K/F$ be a Galois extension, then:

\begin{itemize}
    \item[] (i) There is a one-to-one correspondence:

    \begin{align*}
    \{\text{Intermediate fields to $K/F$}\} &\longleftrightarrow \{\text{Subgroups of $\Gal(K/F)$}\} \\
    E &\longleftrightarrow \Aut(K/E) \\
    K^H &\longleftrightarrow H
    \end{align*}

    and we have $E=K^{\Gal(K/E)}$ and $H=\Gal(K/K^H)$ by theorem 18.

    \item[] (ii) The correspondence in (a) is \textbf{inclusion reversing}, i.e. if $E_1 \subseteq E_2$, then $\Gal(K/E_2) \subseteq \Gal(K/E_1)$. If $H_1 \subseteq H_2$, then $K^{H_2} \subseteq K^{H_1}$.

    \item[] (iii) Let $E$ be an intermediate field to $K/F$, i.e. $F \subseteq E \subseteq K$, then $E/F$ is Galois $\Leftrightarrow \Aut(K/E)$ is normal in $\Gal(K/F)$.

    \item[] (iv) If $E/F$ is Galois, then $\sigma(E)=E$, for all $\sigma \in \Gal(E/F)$, and there is a natural isomorphism $\Gal(K/F)/\Gal(K/E) \simeq \Gal(E/F)$.
\end{itemize}

pf-1. Done in theorem 18. \\

pf-2. Assume that $ E_1,E_2$ are intermediate fields to $K/F$ s.t. $E_1 \subseteq E_2$. Given $\sigma \in \Gal(K/E_2)$, then for all $x \in E_1$, it is also in $E_2$, so $\sigma(x)=x$, i.e. $\sigma \in \Gal(K/E_1)$. On the other hand, assume that $H_1,H_2$ are subgroups of $\Aut(K/F)$ s.t. $H_1 \subseteq H_2$. Given $x \in K^{H_2}$, then for all $\sigma \in H_1$, it is also in $H_2$, then $\sigma(x)=x$, i.e. $x \in K^{H_1}$. \\

pf-3. We use the equivalent statements:

\begin{align*}
    E/F\text{ is Galois} &\Longleftrightarrow \forall \tau \in \Gal(K/F),\ \tau(E)=E\text{ ($\Rightarrow$ by theorem 20, $\Leftarrow$ prove later)} \\
    &\Longleftrightarrow \forall \sigma \in \Gal(K/E),\ \forall \tau \in \Gal(K/F),\ \forall x \in E,\ \sigma(\tau(x))=\tau(x) \\
    &\quad\quad\text{ ($\Rightarrow \tau(x) \in E$, $\Leftarrow$ prove later)} \\
    &\Longleftrightarrow \forall \sigma \in \Gal(K/E),\ \forall \tau \in \Gal(K/F),\ \forall x \in E,\ (\tau^{-1}\sigma\tau)(x)=x \\
    &\Longleftrightarrow \forall \sigma \in \Gal(K/E),\ \forall \tau \in \Gal(K/F),\ \forall x \in E,\ \tau^{-1}\sigma\tau \in \Aut(K/E) \\
    &\Longleftrightarrow \Gal(K/E) \trianglelefteq \Gal(K/F)
\end{align*}

We now prove the first $\Leftarrow$ and the second $\Leftarrow$. \\

The first $\Leftarrow$: We show that if $\forall \tau \in \Gal(K/F),\ \tau(E)=E$, then $E/F\text{ is Galois}$. First, since $K/F$ is Galois, by lemma 17, it is also separable. Then since $E \subseteq K$, $E/F$ is also finite and separable. Thus, $\exists \gamma \in E$ s.t. $E=F(\gamma)$ by primitive element theorem. Consider $\Phi_{F,\gamma}$, by theorem 20 (i) $\Rightarrow$ (iii), $K/F$ is normal, which means $\Phi_{F,\gamma}$ splits completely in $K$. Now, for all $\gamma'$, a root of $\Phi_{F,\gamma}$ in $K$, there exists a unique $\phi:E \to F(\gamma') \subseteq K$ by lemma 1. Thus, by theorem 20 (i) $\Rightarrow$ (vi), $\phi(E)=E$, we have $\gamma' \in E$. This holds for all roots of $\Phi_{F,\gamma}$, so $E$ is a splitting field for the separable polynomial $\Phi_{F,\gamma}$ over $F$, i.e. $E/F$ is Galois. \\

The second $\Leftarrow$: We show that if $\forall \sigma \in \Gal(K/E),\ \forall \tau \in \Gal(K/F),\ \forall x \in E,\ $we always have $\sigma(\tau(x))=\tau(x)$, then it implies $\tau(E)=E,\ \forall \tau \in \Gal(K/E)$. The proof is easy, for all $x \in E$, then $\tau(x) \in K^{\Gal(K/E)}$. By theorem 21-(i), we have $\tau(x) \in K^{\Gal(K/E)}=E$, then $\tau(E) \subseteq E$. On the other hand, $\forall x \in E$, $\tau(\tau^{-1}(x))=x \in E$ and $\tau^{-1}(x)$ is also in $E$, which means $x \in \tau(E)$ and thus $E \subseteq \tau(E)$. Therefore, $\tau(E)=E$. \\

pf-4. Since $E/F$ is Galois, by theorem 20 (i) $\Rightarrow$ (vi). $\forall \tau \in \Gal(K/F)$, $\tau(E)=E$. Thus, we have the group homomorphism:

\begin{align*}
    \pi:\Gal(K/F) &\to \Gal(E/F) \\
    \tau &\mapsto \tau \vert_{E}
\end{align*}

Clearly, $\ker\pi=\{\text{Those who fix $E$}\}=\Gal(K/E)$, then by the first isomorphism theorem, $\Gal(K/F)/\Gal(K/E) \simeq \Img\pi$. For the surjectivity of $\pi$, we count the number of elements:

\begin{align*}
    \#\left(\Gal(K/F)/\Gal(K/E)\right)&=\#\Gal(K/F)/\#\Gal(K/E) \\
    &=[K:F]/[K:E] \\
    &=[E:F] \\
    &=\#\Gal(E/F)
\end{align*}

Thus, $\pi$ is surjective and thus $\Img\pi=\Gal(E/F)$, i.e. $\Gal(K/F)/\Gal(K/E) \simeq \Gal(E/F)$. \\ \\

\textbf{Example. (Computation for fixed fields) }Let $K$ be a splitting field for $f(x)=x^4-2$ over $\mathbb{Q}$, then $K=\mathbb{Q}(\sqrt[4]{2},i)$, and we have seen that $K/\mathbb{Q}$ is Galois. Let $\sigma,\tau \in \Gal(K/\mathbb{Q})$ s.t.:

\[
    \begin{matrix}
        K&\overset{\sigma}{\longrightarrow} &K \\
        \sqrt[4]{2} &\longmapsto &\sqrt[4]{2}i \\
        i &\longmapsto &i
    \end{matrix}
\]

and:

\[
    \begin{matrix}
        K&\overset{\tau}{\longrightarrow} &K \\
        \sqrt[4]{2} &\longmapsto &\sqrt[4]{2} \\
        i &\longmapsto &-i
    \end{matrix}
\]

We also know that $\Gal(K/F)=\langle\sigma{,}\tau\rangle \simeq \mathcal{D}_4$. Now consider $H=\langle \sigma\tau \rangle$, then what is $K^H$? \\

Sol-1 (Brute-force). Consider a basis for $K$ over $F$, say: 

\[
    \mathcal{B}=\{1,\sqrt[4]{2},(\sqrt[4]{2})^2,(\sqrt[4]{2})^3,i,\sqrt[4]{2}i,(\sqrt[4]{2})^2i,(\sqrt[4]{2})^3i\},
\]

then we can write any $x \in K$ as a (unique) linear combination of vectors in $\mathcal{B}$. That is, $x=c_1+c_2\sqrt[4]{2}+c_3(\sqrt[4]{2})^2+c_4(\sqrt[4]{2})^3+c_5i+c_6\sqrt[4]{2}i+c_7(\sqrt[4]{2})^2i+c_8(\sqrt[4]{2})^3i$ for some $c_1,...,c_8 \in \mathbb{Q}$. Since $H=\langle \sigma\tau \rangle$, we want to determined when $\sigma\tau(x)=x$ holds. Compute:

\begin{align*}
    \sigma\tau(x)&=\sigma\tau(c_1+c_2\sqrt[4]{2}+c_3(\sqrt[4]{2})^2+c_4(\sqrt[4]{2})^3+c_5i+c_6\sqrt[4]{2}i+c_7(\sqrt[4]{2})^2i+c_8(\sqrt[4]{2})^3i) \\
    &=\sigma(c_1+c_2\sqrt[4]{2}+c_3(\sqrt[4]{2})^2+c_4(\sqrt[4]{2})^3-c_5i-c_6\sqrt[4]{2}i-c_7(\sqrt[4]{2})^2i-c_8(\sqrt[4]{2})^3i) \\
    &=c_1+c_2\sqrt[4]{2}i-c_3(\sqrt[4]{2})^2-c_4(\sqrt[4]{2})^3i-c_5i+c_6\sqrt[4]{2}+c_7(\sqrt[4]{2})^2i-c_8(\sqrt[4]{2})^3 \\
    &=c_1+c_6\sqrt[4]{2}-c_3(\sqrt[4]{2})^2-c_8(\sqrt[4]{2})^3-c_5i+c_2\sqrt[4]{2}i+c_7(\sqrt[4]{2})^2i-c_4(\sqrt[4]{2})^3i
\end{align*}

Compare the coefficients, we have:

\[
    \left\{\begin{matrix}
        c_2&=&c_6 \\
        c_3&=&c_5&=&0 \\
        c_4&=&-c_8
    \end{matrix}\right.
\]

It tells us $x=c_1+c_2\sqrt[4]{2}(1+i)+c_4(\sqrt[4]{2})^3(1-i)+c_7\sqrt{2}i$. We notice that $(\sqrt[4]{2}(1+i))^2=2\sqrt{2}i$, $(\sqrt[4]{2}(1+i))^3=-2(\sqrt[4]{2})^3(1-i)$, so $x=c_1+c_2\sqrt[4]{2}(1+i)+\frac{c_7}{2}(\sqrt[4]{2}(1+i))^2-\frac{c_4}{2}(\sqrt[4]{2}(1+i))^3$. Thus, $K^H=\{x \in K:\sigma\tau(x)=x\}=\{c_1+c_2\sqrt[4]{2}(1+i)+d_1(\sqrt[4]{2}(1+i))^2+d_2(\sqrt[4]{2}(1+i))^3:c_1,c_2,d_1,d_2 \in \mathbb{Q}\}=\mathbb{Q}(\sqrt[4]{2}(1+i))$. \\

Sol-2 (Notice that). We notice that $\sigma\tau(\sqrt[4]{2}(1+i))=\sqrt[4]{2}(1+i)$, so $\mathbb{Q}(\sqrt[4]{2}(1+i)) \subseteq K^H$. Since $\alpha:=\sqrt[4]{2}(1+i)=\sqrt[4]{2} \cdot \sqrt{2}e^{\frac{i\pi}{4}}$, we have $\alpha^4=-8$, i.e. $\alpha$ is a root of $x^4+8$. If we can show that $f(x)=x^4+8$ is irreducible, then $[\mathbb{Q}(\alpha):\mathbb{Q}]=\deg\Phi_{\mathbb{Q},\alpha}=\deg f=4=\#\Gal(K/\mathbb{Q})/\#\Gal(K/K^H)=[K:\mathbb{Q}]/[K:K^H]=[K^H:\mathbb{Q}]$, i.e. $\mathbb{Q}(\alpha)=K$ since $\mathbb{Q}(\alpha)=K$. \\

How can we show that $x^4+8$ is irreducible in $\mathbb{Q}[x]$? Consider $p=5$ and use irreducibility via reduction mod $p$ (chapter 8, corollary 19), that is, it suffices to show that $x^4+8$ is irreducible in $\mathbb{F}_5$. First, clearly, $x^4+8 \equiv x^4-2$ has no root in $\mathbb{F}_5$, so it doesn't have any factors of degree $1$ and $3$ in $\mathbb{F}_5[x]$. Next, $\alpha^4=-8=2$ (mod $5$), so $\alpha^{16}=1$ (mod $5$) since $\ord(2)=4$, and $16$ is the smallest integer to make this true (i.e. $\alpha^k \ne 1$, $\forall 1 \le k \le 15$). Suppose that $x^4+8$ factors into two irreducible polynomials of degree $2$ in $\mathbb{F}_5[x]$, then by proposition 6, $x^4+8 \mid x^{5^2}-x$, which gives us $\alpha^{25}-\alpha=0$, i.e. $\alpha^{24}=1$. But is leads to a contradiction that $\alpha^8=\alpha^{8} \cdot 1=\alpha^8 \cdot \alpha^{16}=\alpha^{24}=1$. In conclusion, $x^4+8$ is irreducible in $\mathbb{F}_5[x]$, and hence $x^4+8$ is the minimal polynomial of $\alpha$, we have $\mathbb{Q}(\alpha)=K^H$. \\ \\

\textbf{Remark. }Let $K/F$ be a field extension. Given $E$, an intemediate field to $K/F$, then $[K:F]=[K:E][E:F]$ holds by linear algebra, and $\#\Gal(K/F)=\#\Gal(K/E) \cdot \#(\Gal(K/F)/\Gal(K/E))$ holds by Lagrange theorem. Note that $\Gal(K/F)/\Gal(K/E)$ is only the collection of cosets if we do not set any contraint on $E$. Now, if we suppose that $K/F$ is Galois, then $K/E$ is also Galois by proposition 16. Thus, $[E:F]=[K:F]/[K:E]=\#\Gal(K/F)/\#\Gal(K/E)=\#(\Gal(K/F)/\Gal(K/E))$. \\

If we want to determine that wheter $E/F$ is Galois, by theorem 21-(iii), we have $E/F$ is Galois $\Leftrightarrow \Gal(K/E) \trianglelefteq \Gal(K/F)$. Then:

\[
    E/F \text{ is Galois }\Leftrightarrow \Gal(K/E) \trianglelefteq \Gal(K/F) \Leftrightarrow\Gal(K/E)/\Gal(K/F)\text{ is a group} 
\]

In this case, $\Gal(E/F)$ is exactly $\Gal(K/F)/\Gal(K/E)$.

\subsection*{9.7 Fundamental theorem of algebra}
\indent

\textbf{Def. }Let $G$ be a group and $H \subseteq G$ be a subgroup. If $\#H=p^e$ for some prime $p$ and positive integer $e$, then we say $H$ is a \textbf{$p$-group}. \\ \\

\textbf{Fact. }Let $G$ be a $p$ group, i.e. $\#G=p^e$, then there exists a subgroup $H \subseteq G$ s.t. $\#H=p^{e-1}$. The same can be done on $H$, so $G$ contains a subgroup of order $p^{e-i}$, $0 \le i \le e$. \\

pf. If $e=1$, then $p^{e-1}=1$, we can take $H=\{\id_G\}$. Suppose the claim holds up to some $e-1 \ge 1$, we first show that $G$ has a normal subgroup of order $p^k$ for some $1 \le k \le e-1$. \\

If $G$ is abelian, then any subgroups of $G$ is normal. Take $g \neq \id_G$, we discuss two cases. If $G=\langle g \rangle$, then $\langle g^p \rangle$ is a subgroup of order $p^{e-1}$. If $G \ne \langle g \rangle$, then $\langle g \rangle$ itself is a group of order $p^k$, $1 \le k \le e-1$. Thus, $G$ has a normal subgroup of order $p^k$, $1 \le k \le e-1$. \\

If $G$ is not abelian, then $\#Z(G)<\#G=p^e$. Also, we know that $Z(G)$ is nontrivial by theorem 7 in chapter 6. Thus, $G$ also has a normal subgroup of order $p^k$, $1 \le k \le e-1$. \\

Now, let $H$ be a subgroup of $G$, $\#H=p^k$, $1 \le k \le e-1$. Consider $G/H$, $\#(G/H)=\#G/\#H=p^f$ for some $1 \le f \le e-1$. By indcution hypothesis, there exists a subgroup $\overline{H}$ of $G/H$ s.t. $\#\overline{H}=p^{f-1}$. Consider the homormorphsim $\pi:G \to G/H$, $\pi(g)=gH$, we can easily check that $\pi^{-1}(\overline{H})$ is a group. Moreover, all cosets in $G$ are disjoint and of the same cardinality $\#H=p^{e-f}$, so $\#\pi^{-1}(\overline{H})=\#\overline{H} \cdot p^{e-f}=p^{f-1} \cdot p^{e-f}=p^{e-1}$. By mathematical induction, the proof is completed. \\ \\

\textbf{Theorem 22. }Let $K/\mathbb{R}$ be a field extension and $\alpha \in K$ be algebraic over $\mathbb{R}$, then $\mathbb{R}(\alpha)=\mathbb{R}$ or $\mathbb{R}(\alpha)=\mathbb{C}$. \\

pf. To prove theorem 22, we need three claims. \\ \\

\textbf{Claim 1 for theorem 22. }Let $K/\mathbb{R}$ be a field extension. If $[K:\mathbb{R}]=2$, then $K=\mathbb{C}$. \\

pf. Since $[K:\mathbb{R}]=2$, there exists $\alpha \in K \setminus \mathbb{R}$ s.t. $K=\mathbb{R}(\alpha)$. Let $\Phi_{\mathbb{R},\alpha}$ be the minimal polynomial of $\alpha$ over $\mathbb{R}$, then $\deg \Phi_{\mathbb{R,\alpha}}=[\mathbb{R}(\alpha):\mathbb{R}]=[K:\mathbb{R}]=2$, which means we can write $\Phi_{\mathbb{R},\alpha}$ as $x^2+bx+c$ for some $b,c \in \mathbb{R}$. Since the root of $\Phi_{\mathbb{R},\alpha}(x)=x^2+bx+c$ is $\frac{-b\pm\sqrt{b^2-4c}}{2}$ and $\alpha \notin \mathbb{R}$ is a root of $\Phi_{\mathbb{R},\alpha}$, $b^2-4c<0$. Thus, $\mathbb{R}(\alpha)=\mathbb{R}(\sqrt{b^2-4c})=\mathbb{R}(\sqrt{4c-b^2}\sqrt{-1})=\mathbb{R}(i)=\mathbb{C}$. \\ \\

\textbf{Claim 2 for theorem 22. } Let $L/\mathbb{C}$ be a field extension, $[L:\mathbb{C}]=2$ is impossible. \\

pf. Suppose NOT, $[L:\mathbb{C}]=2$, then there exists $\alpha \in L \setminus \mathbb{C}$ s.t. $L=\mathbb{C}(\alpha)$. Let $\Phi_{\mathbb{C},\alpha}$ be the minimal polynomial of $\alpha$ over $\mathbb{C}$, then $\deg \Phi_{\mathbb{C,\alpha}}=[\mathbb{C}(\alpha):\mathbb{C}]=[L:\mathbb{C}]=2$, which means we can write $\Phi_{\mathbb{C},\alpha}$ as $x^2+bx+c$ for some $b,c \in \mathbb{C}$. Since the root of $\Phi_{\mathbb{C},\alpha}(x)=x^2+bx+c$ is $\frac{-b\pm\sqrt{b^2-4c}}{2}$ and each element in $\mathbb{C}$ has square roots in $\mathbb{C}$, (each $y=re^{i\theta} \in \mathbb{C}$ has square roots $\sqrt{y}=re^{i\theta/2}$,$re^{i\theta/2+\pi}$), we have $\alpha \in \mathbb{C}$. Then, $\mathbb{C}(\alpha)=\mathbb{C}$, which contradicts to $[\mathbb{C}(\alpha):\mathbb{C}]=2$. Therefore, such $L$ doesn't exists. \\ \\

\textbf{Claim 3 for theorem 22. } Let $K/\mathbb{R}$ be a field extension, then $[K:\mathbb{R}]\text{ is odd} \Rightarrow K=\mathbb{R}$. \\

pf. Suppose NOT, $K \supsetneq \mathbb{R}$, then $\exists \alpha \in K/\mathbb{R}$ s.t. $[\mathbb{R}(\alpha):\mathbb{R}]>1$. Since $[K:\mathbb{R}]=[K:\mathbb{R}(\alpha)][\mathbb{R}(\alpha):\mathbb{R}]$ is odd, $\deg\Phi_{\mathbb{R},\alpha}=[\mathbb{R}(\alpha):\mathbb{R}]$ can only be odd. Hence, $\Phi_{\mathbb{R},\alpha}(x)$ is an odd polynomial of degree $>1$, which gives us $\lim_{x \to \infty}\Phi_{\mathbb{R},\alpha}(x)=+\infty$ and $\lim_{x \to -\infty}\Phi_{\mathbb{R},\alpha}(x)=-\infty$. Since $\Phi_{\mathbb{R},\alpha}(x)$ is a polynomial, it is continuous. Then, by intermediate value theorem, $\Phi_{\mathbb{R},\alpha}(x)$ has a root $\gamma \in \mathbb{R}$. We can write $\Phi_{\mathbb{R},\alpha}(x)=(x-\gamma) \cdot q(x)$ for some $q(x) \in \mathbb{R}[x]$. But $\Phi_{\mathbb{R},\alpha}$ is irreducible and monic, we have $q(x)\equiv 1$, so $\Phi_{\mathbb{R},\alpha}(x)=x-\gamma$, contradicting to $\deg\Phi_{\mathbb{R,\alpha}}>1$. So, the assumption is false, which means $K=\mathbb{R}$. \\ \\


\textbf{Proof for theorem 22. }First, by claim 3, if $[\mathbb{R}(\alpha):\mathbb{R}]$ is odd, then $\mathbb{R}(\alpha)=\mathbb{R}$. If $[\mathbb{R}(\alpha):\mathbb{R}]$ is even, let $L$ be a splitting field for $\Phi_{\mathbb{R},\alpha}$ over $\mathbb{R}$. As $\chr\mathbb{R}=0$, by theorem 15 in chapter 8, $\Phi_{\mathbb{R},\alpha}$ is separable. Thus, $L/\mathbb{R}$ is Galois and hence $[L:\mathbb{R}]=\#\Gal(L/\mathbb{R})$. Moreover, $[L:\mathbb{R}]=[L:\mathbb{R}(\alpha)][\mathbb{R}(\alpha):\mathbb{R}]$ is even since we assume $[\mathbb{R}(\alpha):\mathbb{R}]$ is even. Thus, we can write $\#\Gal(L/\mathbb{R})=[L:\mathbb{R}]=2^\ell \cdot m$ for some odd $m$ and $\ell \ge 1$. \\

Now, let $H$ be a Sylow $2$-subgroup of $\Gal(L/\mathbb{R})$, then $\#H=2^\ell$. By proposition 16 and theorem 18, $L/L^H$ is Galois with $H=\Gal(L/L^H)$, we have $[L^H:\mathbb{R}]=[L:\mathbb{R}]/[L:L^H]=\#\Gal(L/\mathbb{R})/\#\Gal(L/L^H)=\#\Gal(L/\mathbb{R})/\#H=(2^\ell \cdot m)/2^\ell=m$, which is odd. Again by claim 3, we have $L^H=\mathbb{R}$, so $\#\Gal(L/\mathbb{R})=[L:\mathbb{R}]=[L:L^H]=2^\ell$, which means $\Gal(L/\mathbb{R})$ is a $2$-group. \\

By the fact above theorem 22, $\Gal(L/\mathbb{R})$ has a subgroup $H'$ of order $2^{\ell-1}$. Then $[L^{H'}:\mathbb{R}]=[L:\mathbb{R}]/[L/L^{H'}]=\#\Gal(L/\mathbb{R})/\#\Gal(L/L^H)=\#\Gal(L/\mathbb{R})/\#H'=2^{\ell}/2^{\ell-1}=2$. By claim 1, $L^{H'}=\mathbb{C}$. So, $[L:\mathbb{C}]=[L:L^{H'}]=2^{\ell-1}$. \\

If $\ell=1$, then $L=\mathbb{C}$, we're done. If $\ell \ge 2$, then $\Gal(L/\mathbb{C})$ is a $2$-group, which again gives us $\Gal(L/\mathbb{C})$ has a subgroup $\Tilde{H}$ of order $2^{\ell-2}$. But it leads to a contradiction to claim 2 since $[L^{\Tilde{H}}:\mathbb{C}]=\#\Gal(L/\mathbb{C})/\#\Tilde{H}=2^{\ell-1}/2^{\ell-2}=2$. Thus, $\ell=1$, which means $L=\mathbb{C}$. \\ \\

\textbf{Corollary 23. (Fundamental theorem of algebra) }The field $\mathbb{C}$ is algebraically closed, i.e. $\forall f(x) \in \mathbb{C}[x] \setminus \mathbb{C}$, $f(x)$ has a root in $\mathbb{C}$. \\

pf. Given $f(x)=\sum_{i=0}^na_ix^i \in \mathbb{C}[x] \setminus \mathbb{C}$, consider the complex conjugation $\overline{\cdot}:\mathbb{C} \to \mathbb{C}$, $\overline{a+bi}=a-bi$ for any $a,b \in \mathbb{R}$. We can extend $\overline{\cdot}$ to $\mathbb{C}[x] \to \mathbb{C}[x]$ by $\overline{\sum_{i=0}^na_ix^i}=\sum_{i=0}^n\overline{a_i}x^i$. \\

Set $g(x)=f(x)\overline{f}(x) \in \mathbb{C}[x]$, then $\overline{g}=\overline{f\overline{f}}=\overline{f}f=g$, which means $g(x) \in \mathbb{R}[x]$. Let $K$ be a splitting field for $g$ over $\mathbb{R}$, then there is a root $\alpha \in K$ of $g(x)$. By theorem 22, $\mathbb{R}(\alpha)=\mathbb{R}$ or $\mathbb{R}(\alpha)=\mathbb{C}$, i.e. $\alpha \in \mathbb{C}$. Since $g(\alpha)=f(\alpha)\overline{f}(\alpha)=0$, $f(\alpha)=0$ or $\overline{f}(\alpha)=0$. If $f(\alpha)=0$, then $\alpha$ is a root of $f$ in $\mathbb{C}$. If $\overline{f}(\alpha)=\sum_{i=0}^n\overline{a_i}\alpha^i=0$, take conjugation on both sides, we have $0=\sum_{i=0}^na_i\overline{\alpha}^i=f(\overline{\alpha})$, i.e. $\overline{\alpha}$ is a root of $f$ in $\mathbb{C}$. Either way, $f$ has a root in $\mathbb{C}$, which proves the claim.

\subsection*{9.8 Galois theory of finite fields}
\indent

\textbf{Def. }Let $p$ be a prime and $d \in \mathbb{Z}_{>0}$, we know that there is a finite field with $p^d$ elements and is unique up to isomorphism (theorem 4 in 9.2). We denote this field as $\mathbb{F}_{p^d}$. \\ \\

\textbf{Theorem 24. }Let $p$ be a prime, $d \in \mathbb{Z}_{>0}$, then:

\begin{itemize}
    \item[] (i) The field $\mathbb{F}_{p^d}$ is a Galois extension of $\mathbb{F}_p$.

    \item[] (ii) The map $\phi:\mathbb{F}_{p^d} \to \mathbb{F}_{p^d}$, $\phi:x \mapsto x^p$ is an isomorphism which fixes elements in $\mathbb{F}_p$, i.e. $\phi \in \Gal(\mathbb{F}_{p^d}/\mathbb{F}_p)$. Moreover, $\ord(\phi)=d$ and $\Gal(\mathbb{F}_{p^d}/\mathbb{F}_p)=\langle \phi \rangle$. This $\phi$ is called the \textbf{Frobenius elements} in $\Gal(\mathbb{F}_{p^d}/\mathbb{F}_p)$.

    \item[] (iii) Let $e \ge 1$ be another integer, if $d \mid e$, then $\mathbb{F}_{p^e}$ contains a unique subfield isomorphic to $\mathbb{F}_{p^d}$. If $d \nmid e$, then $\mathbb{F}_{p^e}$ does not contain a subfield isomorphic to $\mathbb{F}_{p^d}$.
    
    \item[] (Note. In the proof of theorem 6 in 9.2, we have shown that if $d \mid e$, then there exists a $\mathbb{F}_p$-embedding $\iota:\mathbb{F}_{p^d} \xhookrightarrow{} \mathbb{F}_{p^e}$. Here, (iii) implies that these embeddings (of the same order) are not only isomorphic to each other, they are exactly the same.)
\end{itemize}

pf-1. Given $x \in \mathbb{F}_{p^d}$, if $x \in \mathbb{F}_{p^d}^\times$, then $x^{p^d}=1$ by Lagrange theorem and thus $x^{p^d}-x=0$. If $x=0$, then $x^{p^d}-x=0$ is trivial. Thus, all elements in $\mathbb{F}_{p^d}$ are roots of $f(x)=x^{p^d}-x$. Moreover, $f(x)$ has at most $p^d$ roots by theorem 8 in 8.2. Thus, $\mathbb{F}_{p^d}$ is the splitting field for $f(x)$ over $\mathbb{F}_{p^d}$. Since $\gcd(f,f')=\gcd(x^{p^d}-x,\ p^dx^{p^d-1}-1)=\gcd(x^{p^d}-x,-1)=1$, $f$ is separable. Thus, $\mathbb{F}_{p^d}$ is a splitting field for a separable polynomial over $\mathbb{F}_{p^d}$, which means $\mathbb{F}_{p^d}/\mathbb{F}_p$ is Galois. \\

pf-2. First, for all $x \in \mathbb{F}_p^\times$, we have $x^{p-1}=1$ by Lagrange theorem, so all elements in $\mathbb{F}_p=\mathbb{F}_p^\times \cup \{0\}$ are roots of $x^p-x$. Thus, $\forall x \in \mathbb{F}_p$, we have $\phi(x)=x^p=x$, i.e. $\phi$ fixes elements in $\mathbb{F}_p$. \\

Now, since $\chr\mathbb{F}_{p^d}=p$, we have shown, in the last semester, $\phi$ is a homomorphism called Frobenius endomorphsim. Clearly, $\phi$ is not a zero map, so $\phi$ is injective (which is also proved in the last semester). Finally, since $\#\mathbb{F}_{p^d}=\#\mathbb{F}_{p^d}$, $\phi$ is an isomorphism, then $\phi \in \Gal(\mathbb{F}_{p^d}/\mathbb{F}_p)$. \\

Next, we show that $\ord(\phi)=d$. First, $\mathbb{F}_{p^d}^\times$ is cyclic, then there exists $\theta \in \mathbb{F}_{p^d}^\times$ s.t. $\mathbb{F}_{p^d}^\times=\langle \theta \rangle$, $\ord(\theta)=p^d-1$. To find the order of $\phi$, it suffices to find the smallest positive integer $i$ s.t. $\phi^i(\theta)=\theta^{p^i}=\theta$ since the neutral element in $\Gal(\mathbb{F}_{p^d}/\mathbb{F}_p)$ is identity map and $\theta$ is a generator of $\mathbb{F}_{p^d}$. But $\ord(\theta)=p^d-1$, if we want $\theta^{p^i-1}=1$, the smallest $i$ is $d$, which means $\ord(\phi)=d$. Finally, $\#\Gal(\mathbb{F}_{p^d}/\mathbb{F}_p)=[\mathbb{F}_{p^d}:\mathbb{F}_p]=d=\ord(\phi)$, which gives us $\Gal(\mathbb{F}_{p^d}/\mathbb{F}_p)=\langle \phi \rangle$. \\

pf-3. Given an integer $e$, by (ii), $\Gal(\mathbb{F}_{p^e}/\mathbb{F}_p)$ is cyclic, say $\Gal(\mathbb{F}_{p^e}/\mathbb{F}_p)=\langle \phi \rangle$, where $\ord(\phi)=\#\Gal(\mathbb{F}_{p^e}/\mathbb{F}_p)=e$. If $d \mid e$, then there is a unique subgroup of order $e/d$ of $\Gal(\mathbb{F}_{p^e}/\mathbb{F}_p)=\langle \phi \rangle$, which is exactly $\langle \phi^d \rangle:=H$. By Galois theory, $E:=\mathbb{F}_{p^e}^H$ is the unique intermediate field to $\mathbb{F}_{p^e}/\mathbb{F}_p$ s.t. $\Gal(\mathbb{F}_{p^e}/E)=H$. We compute the dimension of $E$ over $\mathbb{F}_p$:

\begin{align*}
    [E:\mathbb{F}_p]
    &=[\mathbb{F}_{p^e}:\mathbb{F}_p]/[\mathbb{F}_{p^e}:E] \\
    &=\#\Gal(\mathbb{F}_{p^e}/\mathbb{F}_p)/\#\Gal(\mathbb{F}_{p^e}/E) \\
    &=\#\Gal(\mathbb{F}_{p^e}/\mathbb{F})/\#H \\
    &=e/(e/d) \\
    &=d
\end{align*}

Thus, $E \simeq \mathbb{F}_{p^d}$, and it's an intermediate field to $\mathbb{F}_{p^d}/\mathbb{F}_p$, so it is our desired field. Note that the subgroup $H=\langle \phi^{d} \rangle$ is the only subgroup of $\Gal(\mathbb{F}_{p^e}/\mathbb{F}_p)$ of order $e/d$, so the field $E$ is unique. Moreover, $\mathbb{F}_{p^d} \subseteq \mathbb{F}_{p^e}$ implies that $[\mathbb{F}_{p^e}:\mathbb{F}_{p^d}]=[\mathbb{F}_{p^e}:\mathbb{F}_p]/[\mathbb{F}_{p^d}:\mathbb{F}_p]=e/d$ is an integer, i.e. $d \mid e$. Therefore, $d \nmid e \implies \mathbb{F}_{p^d} \nsubseteq \mathbb{F}_{p^e}$.

\subsection*{9.9 Cyclotomic fields and Kummer fields}
\indent

\textbf{Notation. }Let $F$ be a field, we denote $x^n-1$ as $f_n$, where $n \in \mathbb{Z}_{>0}$. Also, we denote $K_n$ as the splitting field of $f_n$ over $F$. \\ \\

\textbf{Remark. }In a field of characteristic $c =0$ or $c \nmid n$, then since $\gcd(f_n,f_n')=\gcd(x^n-1,nx^{n-1})=1$, $f_n$ is separable. \\ \\

\textbf{Remark. }We first consider $F=\mathbb{Q}$. Since $\chr \mathbb{Q}=0$, $f_n$ is separable. \\ \\

\textbf{Def. }We let $\mu_n=\{\alpha:\alpha\text{ is a root of $f_n$ in $K_n$}\}=\{\alpha \in K_n^\times:\alpha^n=1\}$, then we can easily check that $\mu_n$ is a (multiplicative) subgroup of $K_n^\times$, which means $\mu_n$ is cyclic, i.e. $\mu_n=\langle \zeta_n \rangle$ for some $\zeta_n \in \mu_n$, $\ord(\zeta_n)=n$. Any generator of $\mu_n$ is called a \textbf{primitive $n$-th root of unity}. \\ \\

\textbf{Notation. }We let $\Phi_n(x)=\prod_{\zeta:\zeta\text{ is a primitive $n$-th root of unity in $\mathbb{C}$}}(x-\zeta) \in \mathbb{C}[x]$. \\ \\

\textbf{Example. }By direct computation, $\Phi_1(x)=x-1$, $\Phi_2(x)=x-(-1)=x+1$, $\Phi_3(x)=\frac{x^3-1}{x-1}=x^2+x+1$ (since only $1$ is not a generator), $\Phi_4(x)=\frac{x^4-1}{\Phi_1(x)\Phi_2(x)}=x^2+1$ (since the primitive $1$-st root of unity, $1$, and primitive $2$-nd root of unity, $-1$, can only generate groups of order $1,2$, respectively.). \\ \\

\textbf{Remark. }We can observe that $\Phi_n$ is exactly $\frac{x^n-1}{g(x)}$, where $g(x)$ is the product of $\Phi_m(x)$, $m$ goes through the factors of $n$. Actually, this can be proved. \\

pf. Wrtie $f_n(x)=x^n-1=\prod_{i=0}^{n-1}(x-\zeta^i)$, where $\zeta$ is a primitive $n$-th root of unity. Note that if $\ord(\zeta^i)=d$, then $\zeta^i$ is a primitive $d$-th root of unity, i.e. $\zeta^i$ generates $\mu_d$. By Lagrange theorem, $d \mid n$, so we can write $f_n$ as:

\[
    f_n(x)=x^n-1=\prod_{i=0}^{n-1}(x-\zeta^i)=\prod_{d \mid n}\prod_{\alpha:\alpha\text{ is a primitive}\atop\text{$d$-th root of unity}}(x-\alpha)=\prod_{d \mid n}\Phi_d(x)=\Phi_n(x)\prod_{d \mid n \atop d<n}\Phi_d(x)
\]

So, $\Phi_n(x)=\frac{x^n-1}{g(x)}$, where $g(x)=\prod_{d \mid n,\ d<n}\Phi_d(x)$. \\ \\

\textbf{Lemma 25. }The $\Phi_n$ defined above is a monic polynomial in $\mathbb{Z}[x]$. \\

pf. By construction, $\Phi_n(x)$ is monic. We prove by induction to show that $\Phi_n(x) \in \mathbb{Z}[x]$. If $n=1$, then $\Phi_1(x)=x-1 \in \mathbb{Z}[x]$, the claim holds. Suppose the claim holds up to some $n-1 \ge 1$, then for $n \ge 2$, we first use $x^n-1=\Phi_n(x) \cdot \prod_{d \mid n,\ d<n}\Phi_d(x)$. Since $x^n-1 \in \mathbb{Z}[x] \subseteq \mathbb{Q}[x]$, $\prod_{d \mid n,\ d<n}\phi_d(x) \in \mathbb{Z}[x] \subseteq \mathbb{Q}[x]$ by induction hypothesis, and $\mathbb{Q}[x]$ is a ED, we can use division formula (and some high school math, like long division) to deduce that $\Phi_n$ is also in $\mathbb{Q}[x]$. \\

So, $x^n-1 \in \mathbb{Z}[x]$ has a factorization in $\mathbb{Q}[x]$, which is $x^n-1=\Phi_n(x)g(x)$ with $g(x):=\prod_{d \mid n,\ d<n}\Phi_d(x)$. By Gauss's lemma, there exists $G,H \in \mathbb{Q}[x]$ s.t. $x^n-1=(G\Phi_n(x))(Hg(x))$ and $G\Phi_n(x) \in \mathbb{Z}[x]$, $Hg(x) \in \mathbb{Z}[x]$. Therefore, $(G \cdot \text{The leading coefficient of $\Phi_n$}) \in \mathbb{Z}$ and $(H \cdot \text{The leading coefficient of $g$}) \in \mathbb{Z}$, and:

\begin{align*}
    &\ (G\ \cdot\ \text{The leading coefficient of $\Phi_n$})\ \cdot\ (H\ \cdot\ \text{The leading coefficient of $g$}) \\
    =&\ (\text{The leading coefficient of $x^n-1$}) \\=&\ 1
\end{align*}

The only factorization of $1$ in $\mathbb{Z}$ is $1 \cdot 1$ and $(-1) \cdot (-1)$, which means $(G \cdot$ The leading coefficient of $\Phi_n$) $\in \{\pm 1\}$ and $(H \cdot \text{The leading coefficient of $g$}) \in \{\pm 1\}$. But by construction, the the leading coefficient of $\Phi_n$ and $g$ are both $1$, which gives us $G,H \in \{\pm 1\}$ and hence $\Phi_n \in \mathbb{Z}[x]$. \\ \\

\textbf{Theorem 26. }The $\Phi_n$ defined above is a monic, irreducible polynomial in $\mathbb{Z}[x]$. \\

pf. Here, we only need to show $\Phi_n$ is irreducible in $\mathbb{Z}[x]$. WLOG, $n \ge 3$ since $\Phi_1=x-1$ and $\Phi_2=x+1$ are irreducible in $\mathbb{Z}[x]$. Suppose $\Phi_n=fg$ for some $f,g \in \mathbb{Z}[x]$, $f$ is irreducible in $\mathbb{Z}[x]$. Since the roots of $f$ are also roots of $\Phi_n$, we can choose a primitive $n$-th root of unity, $\zeta$, s.t. $f(\zeta)=0$. Consider a prime $p$ with $p \mid n$, then $\zeta^p$ is still primitive $n$-th root of unity. Thus, $\Phi_n(\zeta^p)=0$, i.e. $\zeta^p$ is either a root of $f$ or a root of $g$. \\

Before we discuss whether $\zeta^p$ is a root of $f$ or $g$, we first consider the homomorphism $h \mapsto \overline{h}$ that reduces the coefficients of a polynomial by modulo $p$, i.e. $a_mx^m+...+a_1x+a_0 \mapsto \overline{a_m}x^m+...+\overline{a_1}x+\overline{a_0}$, where $\overline{a_i} \equiv a_i$ (mod $p$). Then, $\overline{\Phi_{n}}=\overline{f} \cdot \overline{g}$ in $(\mathbb{Z}/p\mathbb{Z})[x]$ since it is a homomorphism. On the other hand, we know that $(a+b)^p=a^p+b^p$ in $\mathbb{Z}/p\mathbb{Z}$, so inductively, if $\overline{g}(x)=b_mx^m+...+b_1x+b_0 \in (\mathbb{Z}/p\mathbb{Z})[x]$, then $\overline{g}(x)^p=(b_mx^m+...+b_1x+b_0)^p=b_m^px^{mp}+...+b_{1}^px^p+b_0^p$. But in $(\mathbb{Z}/p\mathbb{Z})=\mathbb{F}_p$, all elements $\alpha$ satisfies $\alpha^p=\alpha$. Thus, $\overline{g}(x)^p=b_nx^{mp}+...+b_{1}x^p+b_0=\overline{g}(x^p)$. \\

Now, suppose $g(\zeta^p)=0$, then $\overline{g}(\zeta)^p=\overline{g}(\zeta^p)=0$, i.e. $\overline{g}(\zeta)=0$. As $f(\zeta)=0$ by our choice of $\zeta$, we have $\overline{f}(\zeta)=0$. Therefore, $\overline{f}$ and $\overline{g}$ have a common root $\zeta$. But we know that $\overline{\Phi_{n}}$ is separable in $(\mathbb{Z}/p\mathbb{Z})[x]$ as $x^n-1$ is separable in $(\mathbb{Z}/p\mathbb{Z})[x]$, it leads to a contradiction. Thus, $g(\zeta^p) \ne 0$, which means $\zeta^p$ can only be a root of $f$. \\

So far, we have shown that given any prime $p$ with $p \nmid n$, $\zeta^p$ is a root of $f(x)$. Now, consider the set $S$:

\begin{align*}
    S&=\{\text{Primitive $n$-th roots of unity}\} \\
    &=\{\zeta^i:(\zeta^i)^n=1\text{ and $n$ is the smallest positive integer to make this true}\} \\
    &=\{\zeta^i:\ord(i)=n\text{ in $\mathbb{Z}/n\mathbb{Z}$}\} \\
    &=\{\zeta^i:\mathbb{Z}/n\mathbb{Z}\text{ is generated by $i$}\} \\
    &=\{\zeta^i:\exists m\text{ s.t. }mi=1\text{ (mod $n$)}\} \\
    &=\{\zeta^i:i \in (\mathbb{Z}/n\mathbb{Z})^\times\} \\
    &=\{\zeta^i:\gcd(i,n)=1\}
\end{align*}

Given any $\zeta^i \in S$, we write $i=\prod_{j=1}^e p_j$ for some prime $p_j$, then $\gcd(p_j,n)=1$, for all $j$. From the discussion above, $\zeta^{p_j}$ is a root of $f(x)$, for all $j$. We prove by induction on $e$ to show that $\zeta^i$ is a root of $f(x)$ (by separability of $\Phi_n$, it implies that it is not a root of $g(x)$). The case of $e=1$ has been proved. Suppose that it holds for $e-1 \ge 1$, then for $e \ge 2$, suppose $g(\zeta^i)=g(\zeta^{p_1p_2...p_e})=0$, similarly, we consider the map $h \mapsto \overline{h}$, $a_mx^m+...+a_1x+a_0 \mapsto \overline{a_m}x^m+...+\overline{a_1}x+\overline{a_0}$, $\overline{a_i} \equiv a_i$ (mod $p_1$). Then, again, $\overline{g}(x)^{p_1}=\overline{g}(x^{p_1})$ and hence $\overline{g}(\zeta^{p_2p_3...p_e})^{p_1}=\overline{g}(\zeta^{p_1p_2...p_e})=\overline{g}(\zeta^i)=0$, which means $\overline{g}(\zeta^{p_2p_3...p_e})=0$. Induction hypothesis tells us $\zeta^{p_2...p_e}$ is a root of $f(x)$, i.e. $f(\zeta^{p_2...p_m})=0$, which also gives us $\overline{f}(\zeta^{p_2...p_m})=0$. Then, it again contradicts to the separability of $\overline{\Phi_n} \in (\mathbb{Z}/p_2...p_m\mathbb{Z})[x]$, so $\zeta^i$ is not a root of $g(x)$, it can only be a root of $f(x)$. Since $S$ is exactly the set of roots of $\Phi_n$, all roots of $\Phi_n$ are roots of the irreducible polynomial $f(x)$, which means $\Phi_n$ is irreducible. \\ \\

\textbf{Recall. }The \textbf{Euler function} $\varphi:\mathbb{Z}_{>0} \to \mathbb{Z}_{\ge 0}$ is defined to be $\varphi(d)=\#\{i \le d:\gcd(i,d)=1\}=\#(\mathbb{Z}/d\mathbb{Z})^\times$. \\ \\

\textbf{Remark. }In the proof of theorem 26, the $S=\{\text{Primitive $n$-th roots of unity}\}=\{\zeta^i:\gcd(i,n)=1\}$, so $\#S=\#(\mathbb{Z}/n\mathbb{Z})^\times=\varphi(n)$, i.e. the number of generators of $\mu_n$ is $\varphi(n)$. \\ \\

\textbf{Corollary 27. }Let $F=\mathbb{Q}$, then:

\begin{itemize}
    \item[] (i) The minimal polynomial for any primitive $n$-th root of unity is the same.

    \item[] (ii) For any primitive $n$-th root of unity $\zeta_n$, $[\mathbb{Q}(\zeta_n):\mathbb{Q}]=\varphi(n)=\#(\mathbb{Z}/n\mathbb{Z})^\times$.
\end{itemize}

pf-1. Let $S=\{\text{Primitive $n$-th roots of unity}\}$. Given any $\zeta_n \in S$, then $\zeta_n$ is a root of $\Phi_n(x)=\prod_{\zeta \in S}(x-\zeta)$. By theorem 26, $\Phi_n$ is irreducible and monic, so the minimal polynomial is exactly $\Phi_n$. \\

pf-2. By (i), $[\mathbb{Q}(\zeta_n):\mathbb{Q}]=\deg \Phi_{\mathbb{Q},\zeta_n}=\deg\Phi_n=\#S$. By remark, $\#S=\varphi(n)=\#(\mathbb{Z}/n\mathbb{Z})^\times$, which proves the claim. \\ \\

\textbf{Def. }Let $F$ be a field with $\chr F=0$. A \textbf{cyclotomic extension} of $F$ is the splitting field $K$ for $x^n-1$ over $F$ for some $n \ge 1$. If $F$ has been fixed, then we say $K$ is a \textbf{cyclotomic field}. \\ \\

\textbf{Theorem 28. (Cyclotomic field) }Let $F$ be a field with $\chr F=0$, and $K/F$ be a cyclotomic extension, i.e. $K$ is a splitting field for $x^n-1$ over $F$ for some $n \ge 1$. Then:

\begin{itemize}
    \item[] (i) $K/F$ is Galois.

    \item[] (ii) The field $K$ contains a primitive $n$-th root of unity.

    \item[] (iii) $\forall \sigma \in \Gal(K/F)$, $\exists !\ \kappa(\sigma) \in (\mathbb{Z}/n\mathbb{Z})^\times$ s.t. $\sigma(\zeta_n)=\zeta_n^{\kappa(\sigma)}$, where $\zeta_n$ is a primitive $n$-th root of unity.

    \item[] (iv) Let $\sigma,\tau \in \Gal(K/F)$, then:
    
    \begin{itemize}
        \item[] (a) $\kappa(\sigma) \in (\mathbb{Z}/n\mathbb{Z})^\times$
        \item[] (b) $\kappa(\sigma)=\kappa(\tau) \Leftrightarrow \sigma=\tau$
        \item[] (c) $\kappa(\sigma\tau)=\kappa(\sigma)\kappa(\tau)$
    \end{itemize}

    \item[] (v) If $F=\mathbb{Q}$, then $\kappa$ is a group isomorphism.
\end{itemize}

pf-1. Since $\gcd(x^n-1,nx^{n-1})=1$, $x^n-1$ is separable. By definition, $K/F$ is Galois. \\

pf-2. Since $K$ is a splitting field for $x^n-1$, $\mu_n=\{\alpha \in K:\alpha^n-1\} \subseteq K^\times$. It's clear that $\mu_n$ is a group, so $\mu_n$ is a multiplicative subgroup of $K^\times$, which implies that $\mu_n$ is cyclic. Thus, $\exists \theta \in K^\times$ s.t. $\mu_n=\langle \theta \rangle$, i.e. $\theta \in K$ is a primitive $n$-th root of unity. \\

pf-3. Since $\theta$ generates the roots of $x^n-1$, $K=F(\theta)$. Given $\sigma \in \Gal(K/F)$, since $\sigma$ is an isomorphism, $\ord(\sigma(\theta))$ is also $n$, then $\sigma(\theta)$ is a primitive $n$-th root of unity, which means $\langle \sigma(\theta) \rangle=\mu_n=\langle \theta \rangle$. Therefore, $\sigma(\theta)=\theta^i$ for some unique $i \in (\mathbb{Z}/n\mathbb{Z})^\times$. Then, we define $\kappa(\sigma)=i$. \\

pf-4. In (iii), we define a map:

\begin{align*}
    \kappa:\Gal(K/F) &\to (\mathbb{Z}/n\mathbb{Z})^\times \\
    \sigma &\mapsto i
\end{align*}

The $i$ is determined by $\sigma(\theta)=\theta^i$. For (a), by construction, $i \in (\mathbb{Z}/n\mathbb{Z})^\times$. For (b), $\kappa(\sigma)=\kappa(\tau) \Leftrightarrow \sigma(\theta)=\theta^{\kappa(\sigma)}=\theta^{\kappa(\tau)}=\tau(\theta) \Leftrightarrow \sigma=\tau$ (since $K=F(\theta)$). For (c), first, $(\sigma\tau)(\theta)=\theta^{\kappa(\sigma\tau)}$. On the other hand, $(\sigma\tau)(\theta)=\sigma(\tau(\theta))=\sigma(\theta^{\kappa(\tau)})=\sigma(\theta)^{\kappa(\tau)}=\theta^{\kappa(\sigma)\kappa(\tau)}$. Thus, $\theta^{\kappa(\sigma\tau)}=\theta^{\kappa(\sigma)\kappa(\tau)}$, which gives us $\kappa(\sigma\tau)=\kappa(\sigma)\kappa(\tau)$ in $(\mathbb{Z}/n\mathbb{Z})^\times$. \\

pf-5. Now, $F=\mathbb{Q}$, we claim: $\kappa$ is surjective. For all $i \in (\mathbb{Z}/n\mathbb{Z})^\times$, $\theta^i$ is also a primitive $n$-th root of unity. By corollary 28, the minimal polynomial of $\theta$ and the minimal polynomial of $\theta^i$ are both $\Phi_n(x)$. Since $\Phi_n$ is irreducible by theorem 26 and $\theta,\theta^i$ are both roots of $\Phi_n$, by lemma 1, there exists an isomorphism $\sigma:F(\theta) \xrightarrow{\sim} F(\theta^i)$ s.t. $\sigma(\theta)=\theta^i$ and $\sigma(c)=c$, $\forall c \in F$. But both $\theta$ and $\theta^i$ are primitive $n$-th root of unity, $F(\theta)=K=F(\theta^i)$. Hence, $\sigma \in \Gal(K/F)$, which means there exists $\sigma \in \Gal(K/F)$ s.t. $\kappa(\sigma)=i$, for all $i \in (\mathbb{Z}/n\mathbb{Z})^\times$. So, $\kappa$ is surjective and thus an isomorphism along with (iv). \\ \\

\textbf{Remark. }By theorem 28-(v), if $F=\mathbb{Q}$, then $\Gal(K/F) \simeq (\mathbb{Z}/n\mathbb{Z})^\times$. For example, for $n=4$, $\Gal(K/F) \simeq (\mathbb{Z}/4\mathbb{Z})^\times=\{1,3,4\}$. Another example is that if $n$ is a prime $p$, then $\Gal(K/F)$ is cyclic since $(\mathbb{Z}/p\mathbb{Z})^\times$ is cyclic. \\ \\

\textbf{Fact. (Kronecker-Weber theorem) }Let $K/\mathbb{Q}$ be a Galois extension whose Galois group $\Gal(K/\mathbb{Q})$ is abelian, then $K$ is a subfield of a cyclotomic field, i.e. $\exists n,\zeta_n\text{ s.t. }K \subseteq \mathbb{Q}(\zeta_n)$. \\ \\

\textbf{Def. }Let $F$ be a field with $\chr F=0$, $A \in F^\times$. If $K$ is a splitting field for $x^n-A$ over $F$, for some $n \in \mathbb{Z}_{\ge 2}$, then we say $K$ is a \textbf{Kummer field}. \\ \\

\textbf{Theorem 29. (Kummer field) }Let $F$ be a field of $\chr F=0$, $n \in \mathbb{Z}_{\ge 2}$, $A \in F^\times$, and $K$ be the splitting field for $x^n-A$ over $F$. Let $\zeta$ be a primitive $n$-th root of unity, then:

\begin{itemize}
    \item[] (i) The Kummer field $K=F(\alpha,\zeta)$, where $\alpha$ is a root of $x^n-A$.

    \item[] (ii) $\forall \sigma \in \Gal(K/F)$, there is a unique $\lambda(\sigma) \in \mathbb{Z}/n\mathbb{Z}$ s.t. $\sigma(\alpha)=\alpha\zeta^{\lambda(\sigma)}$. 

    \item[] (iii) $\forall \sigma,\tau \in \Gal(K/F)$, set $\overline{\sigma}$ be $\sigma \vert_{F(\zeta)}$, and let $\kappa(\overline{\sigma})$ be defined as theorem 28, then the following holds:

    \begin{itemize}
        \item[] (a) $(\lambda(\sigma)=\lambda(\tau)$ and $\kappa(\overline{\sigma})=\kappa(\overline{\tau})) \Leftrightarrow \sigma=\tau$
        \item[] (b) $\lambda(\sigma\tau)=\lambda(\sigma)+\kappa(\overline{\sigma})\lambda(\tau)$ in $\mathbb{Z}/n\mathbb{Z}$
    \end{itemize}
\end{itemize}

pf-1. As $K$ is a splitting field of $x^n-A$, $\alpha \in K$. Since $\alpha^n=A$, we have $(\alpha\zeta)^n=\alpha^n\zeta^n=\alpha^n=A$, which means $\alpha\zeta$ is also a root of $x^n-A$. Hence, $\alpha\zeta \in K$, this tells us $\zeta=(\alpha\zeta)/\alpha \in K$ and $F(\alpha,\zeta) \subseteq K$. Conversely, as $\zeta$ is a primitive $n$-th root of unity, $\alpha,\alpha\zeta,\alpha\zeta^2,...,\alpha\zeta^{n-1}$ are distinct roots of $x^n-A$. Since $x^n-A$ has exactly $n$ roots ($\deg(x^n-A)=n$ and it is separable), $F(\alpha,\zeta)$ contains all roots of $x^n-A$, i.e. $F(\alpha,\zeta) \supseteq K$. In conclusion, $F(\alpha,\zeta)=K$. \\

pf-2. Given $\sigma \in \Gal(K/F)$, then $\sigma(\alpha)$ is also a root of $x^n-A$, i.e. $\sigma(\alpha) \in \{\alpha\zeta^i:0 \le i \le n-1\}$. So, $\sigma(\alpha)=\alpha\zeta^i$ for some unique $i \in \mathbb{Z}/n\mathbb{Z}$, then we can define $\lambda(\sigma)=i$. \\

pf-3. First, $F \subseteq F(\zeta) \subseteq F(\alpha,\zeta)=K$. Since $F(\zeta)$ is the splitting field for $x^n-1$ over $F$, by theorem 28, $F(\zeta)/F$ is Galois. Given any $\sigma \in \Gal(K/F)$, if $\overline{\sigma}=\sigma\vert_{F(\zeta)}$, then by theorem 20-(iv), $\overline{\sigma}(F(\zeta))=F(\zeta)$, which means $\overline{\sigma} \in \Gal(F(\zeta)/F)$ since $\sigma$ itself is an automorphism in $K$ that fixes $F$. So, given $\sigma,\tau \in \Gal(K/F)$, we have the equivalent statements:

\begin{align*}
    \sigma=\tau &\Leftrightarrow\sigma(\alpha)=\tau(\alpha)\text{ and }\sigma(\zeta)=\tau(\zeta)\text{ ($\Leftarrow$ is because that $K=F(\alpha,\zeta)$)} \\
    &\Leftrightarrow \alpha\zeta^{\lambda(\sigma)}=\alpha\zeta^{\lambda(\tau)}\text{ and }\zeta^{\kappa(\overline{\sigma})}=\zeta^{\kappa(\overline{\tau})} \\
    &\Leftrightarrow \lambda(\sigma)=\lambda(\tau)\text{ and }\kappa(\overline{\sigma})=\kappa(\overline{\tau})
\end{align*}

Moreover, we can compute that $\sigma\tau(\alpha)=\sigma(\alpha\zeta^{\lambda(\tau)})=\sigma(\alpha)\sigma(\zeta)^{\lambda(\tau)}=\alpha\zeta^{\lambda(\sigma)+\kappa(\overline{\sigma})\lambda(\tau)}$. By definition, $\sigma\tau(\alpha)=\alpha\zeta^{\lambda(\sigma\tau)}$. Thus, $\lambda(\sigma\tau)=\lambda(\sigma)+\kappa(\overline{\sigma})\lambda(\tau)$ in $\mathbb{Z}/n\mathbb{Z}$. \\ \\

\textbf{Remark. }Theorem 29 tells us $\lambda:\Gal(K/F) \to \mathbb{Z}/n\mathbb{Z}$ is NOT a homomorphism. \\ \\


\textbf{Recall. (semidirect product) }Let $G_1,G_2$ be groups, and consider the group homomorphism $\phi:G_2 \to \Aut(G_1)$, $g \mapsto [\phi_{g}:G_1 \to G_1]$ (i.e. $G_2 \acts G_1$), we defined the \textbf{semidirect product} of $G_1$ and $G_2$ is $G_1 \rtimes_\phi G_2=\{(g_1,g_2):g_1 \in G_1,\ g_2 \in G_2\}$ with group law $(a_1,a_2) \star (b_1,b_2)=(a_1\phi_{a_2}(b_1),a_2b_2)$. \\ \\

\textbf{Remark. }Consider $G_1=(\mathbb{Z}/n\mathbb{Z},+)$, $G_2=((\mathbb{Z}/n\mathbb{Z})^\times,\times)$, define $G_2 \acts G_1$ by $y \cdot x=yx$, then $1 \cdot x=x$ and $(y_1y_2) \cdot x=(y_1y_2)x=y_1(y_2x)=y_1 \cdot (y_2 \cdot x)$, it is well-defined. So, $G_1 \rtimes_\phi G_2$ is a well-defined semidirect product. \\ \\

\textbf{Remark. }Consider the map:

\begin{align*}
    \iota:\Gal(K/F) &\to (\mathbb{Z}/n\mathbb{Z}) \rtimes_\phi (\mathbb{Z}/n\mathbb{Z})^\times \\
    \sigma &\mapsto (\lambda(\sigma),\kappa(\overline{\sigma}))
\end{align*}

Claim: $\iota$ is an injective group homomorphism. \\

pf. Given $\sigma,\tau \in \Gal(K/F)$, then:

\begin{align*}
    \iota(\sigma\tau)&=(\lambda(\sigma\tau),\kappa(\overline{\sigma\tau})) \\
    &=(\lambda(\sigma)+\kappa(\overline{\sigma})\lambda(\tau),\kappa(\overline{\sigma})\kappa(\overline{\tau}))
\end{align*}

And:

\begin{align*}
    \iota(\sigma) \star \iota(\tau) &= (\lambda(\sigma),\kappa(\overline{\sigma})) \star (\lambda(\tau),\kappa(\overline{\tau})) \\
    &=(\lambda(\sigma)+\phi_{\kappa(\overline{\sigma})}(\lambda(\tau)),\kappa(\overline{\sigma})\kappa(\overline{\tau})) \text{ (Note that the group law of $(\mathbb{Z}/n\mathbb{Z})$ is $+$)}\\
    &=(\lambda(\sigma)+\kappa(\overline{\sigma})\lambda(\tau),\kappa(\overline{\sigma})\kappa(\overline{\tau}))
\end{align*}

So, $\iota(\sigma\tau)=\iota(\sigma)\iota(\tau)$, i.e. $\iota$ is a group homomorphism. \\

Moreover, if $\iota(\sigma)=\iota(\tau)$, then $(\lambda(\sigma),\kappa(\overline{\sigma}))=(\lambda(\tau),\kappa(\overline{\tau}))$. Theorem 28-(iv)-(b) and theorem 29-(iii)-(a) gives us $\sigma=\tau$.

\subsection*{9.10 Application on regular 17-gon}
\indent

\textbf{Question. }Let $\zeta$ be a primitive $17$-th root of unity. Is $\zeta$ constructible? Answer: Yes. \\ \\

\textbf{Steps of the proof. }First, we show that $[\mathbb{Q}(\zeta):\mathbb{Q}]=16$ and there exists intermediate fields $\mathbb{Q}=F_0 \subsetneq F_1 \subsetneq F_2\subsetneq F_3 \subsetneq F_4=\mathbb{Q}(\zeta)$ s.t. $[F_{i+1}:F_i]=2$ for all $i=0,1,2,3$. This proves that $\zeta$ is constructible since $[F_{i+1}:F]=2$ means that we can construct any number in $F_{i+1}$ through square root. Finally, we find an explicit expression of $F_1,F_2,F_3$. \\ \\

\textbf{Step 1. }First, $\mathbb{Q}(\zeta)$ is the splitting field of $x^{17}-1$ over $\mathbb{Q}$, so the minimal polynomial of $\zeta$ over $\mathbb{Q}$ is $\Phi_{17}=\frac{x^{17}-1}{\Phi_1}=x^{16}+x^{15}+...+x+1$. Thus, $[\mathbb{Q}(\zeta):\mathbb{Q}]=\deg\Phi_{17}=16=2^4$, which is a necessary condition of $\zeta$ being construtible. \\ \\

\textbf{Step 2. }By theorem 28, $\Gal(\mathbb{Q}(\zeta)/\mathbb{Q}) \xrightarrow[]{\sim} (\mathbb{Z}/17\mathbb{Z})^\times$ with the isomorphism $\sigma \mapsto \kappa(\sigma)$, where $\kappa(\sigma)$ satisfies $\sigma(\zeta)=\zeta^{\kappa(\sigma)}$. So, $\Gal(\mathbb{Q}(\zeta)/\mathbb{Q})$ is cyclic, and to find a generator of it is equivalent to find a generator of $(\mathbb{Z}/17\mathbb{Z})^\times$. \\

One can guess that $3$ is a generator of $(\mathbb{Z}/17\mathbb{Z})^\times$. Check: $3^2=9$, $3^4=9^2=81=4$, $3^8=4^2=16=-1$. So $\ord(3)=16$, $3$ is a generator. This tells $\sigma:\zeta \mapsto \zeta^3$ is a generator of $\Gal(\mathbb{Q}(\zeta)/\mathbb{Q})$. Then, $(\mathbb{Z}/17\mathbb{Z})^\times=\langle \sigma \rangle \supsetneq \langle \sigma^2 \rangle \supsetneq \langle \sigma^4 \rangle \supsetneq \langle \sigma^8 \rangle \supsetneq \langle \sigma^{16} \rangle=\{\id\}$. By Galois theorem, there exists $F_0 \subsetneq F_1 \subsetneq F_3 \subsetneq F_3 \subsetneq F_4$ s.t.:

\begin{gather*}
    F_0=\mathbb{Q}(\zeta)^{\langle \sigma \rangle}=\mathbb{Q} \\
    F_1=\mathbb{Q}(\zeta)^{\langle \sigma^2 \rangle} \\
    F_2=\mathbb{Q}(\zeta)^{\langle \sigma^4 \rangle} \\
    F_3=\mathbb{Q}(\zeta)^{\langle \sigma^8 \rangle} \\
    F_4=\mathbb{Q}(\zeta)^{\{\id\}}=\mathbb{Q}(\zeta)
\end{gather*}

So, the intermediate fields $F_0 \subsetneq F_1 \subsetneq F_2 \subsetneq F_3 \subsetneq F_4$ exist, i.e. $\zeta$ is constructible. \\ \\

\textbf{Step 3. }Finally, we compute $\mathbb{Q}^{\langle \sigma^2 \rangle}$, $\mathbb{Q}^{\langle \sigma^4 \rangle}$, $\mathbb{Q}^{\langle \sigma^8 \rangle}$. \\

First, by $\Phi_{17}=x^{16}+x^{15}+...+x+1$, $\mathcal{B}:=\{\zeta,\zeta^2,...,\zeta^{16}\}$ is linearly independent (otherwise, there exists $f$ with $\deg f<\deg \Phi_{17}$ s.t. $f(\zeta)=0$). Since $[\mathbb{Q}(\zeta):\mathbb{Q}]=16=\lvert {B} \rvert$, $\mathcal{B}$ is a basis for $\mathbb{Q}(\zeta)$ over $\mathbb{Q}$. \\

Apply $\sigma$ to $\zeta$, we will get a sequence:

\[
    \zeta,\ \zeta^3,\ \zeta^9,\ \zeta^{10},\ \zeta^{13},\ \zeta^{5},\ \zeta^{-2},\ \zeta^{11},\ \zeta^{-1},\ \zeta^{14},\ \zeta^8,\ \zeta^7,\ \zeta^4,\ \zeta^{12},\ \zeta^2,\ \zeta^6
\]

Let:

\begin{gather*}
    \alpha_1=\zeta+\zeta^9+\zeta^{13}+\zeta^{-2}+\zeta^{-1}+\zeta^{8}+\zeta^4+\zeta^2 \\
    \alpha_2=\zeta^3+\zeta^{10}+\zeta^{5}+\zeta^{11}+\zeta^{14}+\zeta^7+\zeta^{12}+\zeta^6
\end{gather*}

Then $\{\alpha_1,\alpha_2\}$ is linearly independent and thus a basis for $\mathbb{Q}^{\langle \sigma^2 \rangle}$ over $\mathbb{Q}$ since $\sigma^2(\alpha_1)=\alpha_1$, $\sigma^2(\alpha_2)=\alpha_2$, and $[\mathbb{Q}^{\langle \sigma^2 \rangle}:\mathbb{Q}]=2$. So, $\mathbb{Q}^{\langle \sigma^2 \rangle}=\mathbb{Q}(\alpha_1,\alpha_2)$. How to compute $\alpha_1$ and $\alpha_2$? An useful observation is that $\alpha_1$ and $\alpha_2$ are roots of $(x-\alpha_1)(x-\alpha_2)$, so we can compute $\alpha_1+\alpha_2$ and $\alpha_1\alpha_2$ instead and solve the equation $x^2-(\alpha_1+\alpha_2)x+\alpha_1\alpha_2$. By $1+\zeta+\zeta^2+...+\zeta^{16}=0$, we have $\alpha_1+\alpha_2=-1$. On the other hand, $\sigma(\alpha_1)=\alpha_2$ and $\sigma(\alpha_2)=\alpha_1$, so $\sigma(\alpha_1\alpha_2)=\alpha_1\alpha_2$, i.e. $\sigma$ fixes $\alpha_1\alpha_2$. Write $\alpha_1\alpha_2=\sum_{i=1}^{16}a_i\zeta^i$, then $\sum_{i=1}^{16}a_i\zeta^i=\alpha_1\alpha_2=\sigma(\alpha_1\alpha_2)=\sigma(\sum_{i=1}^{16}a_i\zeta^i)=\sum_{i=1}^{16}a_i\zeta^{3i}$. But $i \mapsto 3i$ is bijective in $\mathbb{Z}/17\mathbb{Z}$, we have $a_{3i}=a_i$, for all $i$, i.e. $a_1=a_2=...=a_{16}$. Since there are $64$ terms if we expand $\alpha_1\alpha_2$ and there is no "$1$" term ($\because$ both $\zeta^j$ and $\zeta^{-j}$ are in $\alpha_1$ (or $\alpha_2$) simultaneously), $a_1=a_2=...=a_{16}$ can only be $64/16=4$. Thus, $\alpha_1\alpha_2=4\sum_{i=1}^{16}\zeta^i=-4$. In conclusion, $\alpha_1$ and $\alpha_2$ are roots of $x^2+x-4$, i.e. $\alpha_{1,2}=\frac{-1\pm\sqrt{17}}{2}$, which gives us $\mathbb{Q}^{\langle \sigma^2 \rangle}=\mathbb{Q}(\alpha_1,\alpha_2)=\mathbb{Q}(\sqrt{17})$. \\

Next, we let:

\begin{gather*}
    \beta_1=\zeta+\zeta^{13}+\zeta^{-1}+\zeta^4 \\
    \beta_2=\zeta^9+\zeta^{-2}+\zeta^{8}+\zeta^2 \\
    \beta_3=\zeta^3+\zeta^{5}+\zeta^{14}+\zeta^{12} \\
    \beta_4=\zeta^{10}+\zeta^{11}+\zeta^7+\zeta^6
\end{gather*}

Then $\{\beta_1,\beta_2,\beta_3,\beta_4\}$ is a basis for $\mathbb{Q}^{\langle \sigma^4 \rangle}$ over $\mathbb{Q}$. Similarly, we compute $\beta_1+\beta_2$ and $\beta_1\beta_2$. First, $\beta_1+\beta_2=\alpha_1$, $\sigma^2(\beta_1)=\beta_2$, $\sigma^2(\beta_2)=\beta_1$, so $\sigma^2(\beta_1\beta_2)=\beta_1\beta_2$. If $\beta_1\beta_2=\sum_{i=1}^{16}b_1\zeta^i$, then $a_i=a_{9i}$. Again, $i \mapsto 9i$ is an isomorphism, we have $b_1=b_2=...=b_{16}=16/16=1$. Therefore, $\beta_1\beta_2=\sum_{i=1}^{16}\zeta^i=-1$, $\beta_1$ and $\beta_2$ are roots of $x^2-\alpha_1x+1$, which is $\frac{\alpha_1\pm\sqrt{\alpha_1^2-4}}{2}$. Similarly, $\beta_{3,4}=\frac{\alpha_2\pm\sqrt{\alpha_2^2-4}}{2}$. \\

Finally, we let:

\begin{gather*}
    \gamma_1=\zeta+\zeta^{-1} \\
    \gamma_2=\zeta^{13}+\zeta^4 \\
    \gamma_3=\zeta^9+\zeta^{8} \\
    \gamma_4=\zeta^{-2}+\zeta^2 \\
    \gamma_5=\zeta^3+\zeta^{14} \\
    \gamma_6=\zeta^{5}+\zeta^{12} \\
    \gamma_7=\zeta^{10}+\zeta^7 \\
    \gamma_8=\zeta^{11}+\zeta^6
\end{gather*}

It's easy to see that $\gamma_i$ is in the form $\zeta^j+\zeta^{-j}$, which is not a coincidence since $\sigma^8(\zeta^j)=\zeta^{3^8j}=\zeta^{-j}$. Use the same process, we can find that $\gamma_{1},\gamma_{2}$ are roots of $x^2-\beta_1x+\beta_{3}$, which is $\frac{\beta_1\pm \sqrt{\beta_1^2-4\beta_3}}{2}$.

\subsection*{9.11 Insolubility of polynomial equations by radicals}
\indent

\textbf{Notation. }In this section, $\chr F=0$. \\ \\

\textbf{Def. }A field extension $L/F$ is a \textbf{radical extension} if there exists a sequence of fields $F=E_0 \subsetneq E_1 \subsetneq \cdots \subsetneq E_{r-1} \subsetneq E_r=L$ s.t. for each $0 \le i < r$, there exists is an integer $n_i \ge 2$ and $A_i \in E_i$ s.t. $E_{i+1}$ is the splitting field for $x^{n_i}-A_i$ over $E_i$. Note that whether it is a strict inclusion is not important since we can always remove an equality case. \\ \\

\textbf{Prop 30. }Let $L/F$ be an extension of field. If $L/F$ is radical, then there exists an extension $L'/L$ s.t. $L'/F$ is radical and Galois. \\

pf. First, $L/F$ is radical, so there exists a sequence $L=E_r \supsetneq E_{r-1} \supsetneq \cdots \supsetneq E_1 \supsetneq E_0=F$ s.t. $E_{i+1}$ is the splitting field for $x^{n_i}-A_i$ for some $n_i \ge 2$, $A_i \in E_i$, $i=0,1,...,r-1$. We prove by induction on $r$. When $r=1$, $L/F=E_1/E_0$ is radical by assumption, and it is Galois since $L$ is a Kummer field. Suppose it holds until $r-1 \ge 1$, then for $r \ge 2$, by induction hypothesis, there exists $E_{r-1}'/E_{r-1}$ s.t. $E_{r-1}'/F$ is Galois and radical. Let $n$ denote $n_{r-1}$ Consider $f(x)=\prod_{\sigma \in \Gal(E'_{r-1}/F)}(x^{n}-\sigma(A_{r-1}))$, by HW5, 9.23-(c), $f(x)=(\Phi_{F,A_{r-1}}(x^{n}))^d \in F[x]$ for some $d \ge 1$. Since $E_{r-1}'/F$ is Galois, there exists a separable polynomial $g(x) \in F[x]$ s.t. $E_{r-1}'$ is the splitting field for $g(x)$ over $F$. Now, let $L'$ be the splitting field for $f(x)g(x)(x^n-1)$ over $F$, then $L'$ is Galois since we assume $\chr F=0$ (by proposition 11). \\

Claim: $L'/F$ is radical. First, $E_r=E_{r-1}(\sqrt[n]{A_{r-1}},\zeta_n)$. Since $E_{r-1} \subseteq E_{r-1}' \subseteq L'$ and both $\sqrt[n]{A_{r-1}}$ (a root of $f(x)$) and $\zeta_n$ (a root of $(x^n-1)$) are in $L'$, we have $E_r \subseteq L'$. Now, it suffices to show that $L'/E_{r-1}'$ is radical since $E_{r-1}'/F$ is radical. List the elements of the set $\{\sigma(A_{r-1}):\sigma \in \Gal(E_{r-1}'/F)\}$ as $\alpha_1,\alpha_2,...,\alpha_k$, then we can write $L'/E_{r-1}'$ as a sequence of field extensions $F \subsetneq E_{r-1}' \subsetneq K_1 \subseteq K_2 \subseteq ... \subseteq K_{k} \subseteq L'$, where $K_1=E_{r-1}'(\zeta_n,\sqrt[n]{\alpha_1})$, $K_{i}=K_{i-1}(\sqrt[n]{\alpha_{i}})$, $i=2,3,...,k$. Each field extension $K_{i}/K_{i-1}$ is the splitting field $x^n-\alpha_i$, and $K_k$ is exactly $L'$. Thus, $L'/E_{r-1}'$ is radical, which proves the claim. \\ \\

\textbf{Remark. }Consider the field extension:

\[
    \mathbb{Q}-\mathbb{Q}(\sqrt{2})-\mathbb{Q}(\sqrt[4]{2})
\]

Then $\mathbb{Q}(\sqrt{2})/\mathbb{Q}$ and $\mathbb{Q}(\sqrt[4]{2})/\mathbb{Q}(\sqrt{2})$ are both Galois, but $\mathbb{Q}(\sqrt[4]{2})/\mathbb{Q}$ is only radical but not Galois. Follow the proof of proposition 30, since $\mathbb{Q}(\sqrt[4]{2})$ is the splitting field for $x^2-\sqrt{2}$ over $\mathbb{Q}(\sqrt{2})$, we can consider $f(x)=\prod_{\sigma \in \Gal(\mathbb{Q}(\sqrt{2})/\mathbb{Q})}(x^2-\sigma(\sqrt{2}))=(x^2-\sqrt{2})(x^2+\sqrt{2})=x^4-2$, $g(x)=x^2-2$, and let $L'$ be the splitting field for $f(x)g(x)=(x^4-2)(x^2-2)(x^2-1)$ over $F$. Then, $L'/F$ is Galois since $L$ is also the splitting field for $f(x)g(x)$ (since $x^2-1$ has roots in $\mathbb{Q}$) and $f(x)g(x)$ is separable, and we can decompose $L'/\mathbb{Q}(\sqrt{2})$ into $\mathbb{Q}(\sqrt{2}) \subseteq K_1 \subseteq K_2=L'$, where $K_1=\mathbb{Q}(\sqrt{2})(\sqrt[4]{2})=\mathbb{Q(\sqrt[4]{2})}$, $K_2=\mathbb{Q}(\sqrt[4]{2})(\sqrt{-\sqrt{2}})=\mathbb{Q}(\sqrt[4]{2})(\sqrt[4]{2}i)=\mathbb{Q}(\sqrt[4]{2},i)$. Thus, $L'=\mathbb{Q}(\sqrt[4]{2},i)$ is Galois and radical over $F$. \\ \\


\textbf{Def. }Let $f(x) \in F[x] \setminus \{0\}$. We say $f(x)$ can be \textbf{solved using radicals} if there exists a radical extension $L$ s.t. $f(x)$ splits completely in $L$. That is, $\exists L/K/F$ s.t. $L/F$ is radical and $K$ it the splitting field for $f(x)$ over $F$. \\ \\

\textbf{Remark. }By proposition 30, we can assume that $L/F$ is Galois. \\ \\

\textbf{Recall. }The Kummer extension $F(\alpha,\zeta_n)/F$ can be decomposed into $F(\alpha,\zeta_n)/F(\zeta_n)$ and $F(\zeta_n)/F$, where $\Gal(F(\alpha,\zeta_n)/F(\zeta_n))$ is a subgroup of $\mathbb{Z}/n\mathbb{Z}$ and $\Gal(F(\zeta_n)/F)$ is a subgroup of $(\mathbb{Z}/n\mathbb{Z})^\times$. It implies that both of the Galois groups are abelian. So, a radical extension can be decomposed into a sequence of extensions, and the Galois group for each extension is abelian. \\ \\

\textbf{Def. }A group $G$ is a \textbf{solvable group} if there is a sequence of subgroups $\{e\}=H_0 \subsetneq H_1 \subsetneq \cdots \subsetneq H_{k-1} \subsetneq H_k=G$ s.t. for all $j=0,1,...,k-1$, we have:

\begin{itemize}
    \item[] (i) $H_j \trianglelefteq H_{j+1}$
    \item[] (ii) $H_{j+1}/H_j$ is abelian
\end{itemize}

Note that $\Gal(L/F)$ is solvable if $L/F$ is radical and Galois. (It needs to be Galois to apply Galois correspondence.) \\ \\

\textbf{Remark. }We can modify the definition to (i) $H_j \trianglelefteq H_{j+1}$, (ii) $H_{j+1}/H_j$ is cyclic. \\

pf. If $H_{j+1}/H_j$ is cyclic, then it must be abelian. If $H_{j+1}/H_j$ is abelian, then by the fundamental theorem of finite abelian group, we can write $H_{j+1}/H_j=\mathcal{C}_{i_1} \times \mathcal{C}_{i_2} \times \cdots \times \mathcal{C}_{i_n}$ for some $n$. There exists a surjective homomorphism $H_{j+1} \surj H_{j+1}/H_j \surj \mathcal{C}_{i_1} \times \cdots \times \mathcal{C}_{i_{n-1}}$ (by projecting $\mathcal{C}_{i_1} \times \cdots \times \mathcal{C}_{i_n}$ onto $\mathcal{C}_{i_2} \times \cdots \times \mathcal{C}_{i_n}$). So, if the kernel of this map is $N_{n-1}$, then $H_{j+1}/N_{n-1} \simeq \mathcal{C}_{i_1} \times \cdots \times \mathcal{C}_{i_{n-1}}$. Note that $H_j$ is in the kernel of the first surjection, so $H_j \subseteq N_n$. Recursively, we have:

\[
    \{e\}=H_{j+1}/H_{j+1} \subsetneq H_{j+1}/N_1 \subsetneq \cdots \subsetneq H_{j+1}/N_n \subsetneq H_{j+1}/H_j,
\]

where $H_{j+1}/N_m \simeq \mathcal{C}_{i_1} \times \cdots \times \mathcal{C}_{i_{m}}$ for $m \ge 1$, and $H_j \subsetneq N_n \subsetneq N_{n-1} \subsetneq \cdots \subseteq N_1 \subsetneq H_{j+1}$. This sequence satisfies the two conditions, so we're done. \\ \\

\textbf{Lemma 31. }Let $G$ be a solvable group, then:

\begin{itemize}
    \item[] (i) If $H \subseteq G$ is a subgroup of $G$, then $H$ is solvable. 
    \item[] (ii) If $N \trianglelefteq G$ is a normal subgroup of $G$, then $G/N$ is solvable.
\end{itemize}

pf-1. Since $G$ is solvable, there exists $\{e\}=G_0 \subsetneq G_1 \subsetneq \cdots \subsetneq G_{k-1} \subsetneq G_k=G$ satisfy the condition of a solvable group. Consider $\{e\}=G_0 \cap H \subseteq G_1 \cap H \subseteq \cdots \subseteq G_{k-1} \cap H \subseteq G_k \cap H=H$. For any $j=0,1,...,k-1$, we need to show that $G_j \cap H \trianglelefteq G_{j+1} \cap H$ and $(G_{j+1} \cap H)/(G_j \cap H)$ is abelian. \\

Given any $g_1 \in G_{j+1} \cap H$ and $g_2 \in G_j \cap H$, we have $g_1g_2g_1^{-1} \in G_j$ since $G_j \trianglelefteq G_{j+1}$. Also, $g_1g_2g_1^{-1} \in H$ by closure. Thus, $g_1g_2g_1^{-1} \in G_j \cap H$, $G_j \cap H \trianglelefteq G_{j+1} \cap H$. Moreover, we claim the map $\iota$:

\begin{align*}
    \iota:(G_{j+1} \cap H)/(G_j \cap H) &\to G_{j+1}/G_j \\
    \iota:g(G_j \cap H) &\mapsto gG_j
\end{align*}

is a well-defined injective homomorphism. First, for any $g_1,g_2$ s.t. $g_1(G_j \cap H)=g_2(G_j \cap H)$, then $g_2^{-1}g_1 \in G_j \cap H \subseteq G_j$, which means $\iota(g_1(G_j \cap H))=g_1G_j=g_2G_j=\iota(g_2(G_j \cap H))$. In addition, if there is some $g \in G_{j+1} \cap H$ s.t. $\iota(g(G_j \cap H))=gG_j=G_j$ for some $g$, then $g \in G_j$, which means $g \in G_j \cap H$, i.e. $g(G_j \cap H)=G_j \cap H$. Thus, $\ker \iota$ is trivial, it is injective, which means $(G_{j+1} \cap H)/(G_j \cap H)$ is abelian since $G_{j+1}/G_j$ is abelian by assumption. \\


pf-2. Similarly, there exists $\{e\}=H_0 \subsetneq H_1 \subsetneq \cdots \subsetneq H_{k-1} \subsetneq H_k=G$ satisfy the condition of a solvable group. Now we claim that $\{e\}=H_0/N \cap H_0$, $H_1/N \cap H_1$,..., $H_{k-1}/N \cap H_{k-1}$, $H_k/N \cap H_k=G/N$ is a well-defined inclusion and satisfies the condition of a solvable group. For any $j=0,1,...,k-1$, consider the map $\iota$:

\begin{align*}
    \iota:H_j/N \cap H_j & \to H_{j+1}/N \cap H_{j+1} \\
    \iota:h(N \cap H_j) & \mapsto h(N \cap H_{j+1})
\end{align*}

It's clear that $\iota$ is a homomorphism. To check the "well-defined inclusion" part, it suffices to cheack $\iota$ is a well-defined injective map. Given $h_1,h_2 \in H_j$ s.t. $h_1(N \cap H_j)=h_2(N \cap H_j)$, then $h_2^{-1}h_1 \in N \cap H_j$. Since $N \cap H_j \subseteq N \cap H_{j+1}$, $h_2^{-1}h_1 \in N \cap H_{j+1}$ and thus $\iota(h_1(N \cap H_j))=h_1(N \cap H_{j+1})=h_2(N \cap H_{j+1})=\iota(h_2(N \cap H_j))$. So, $\iota$ is well-defined. Moreover, if $\iota(h_1(N \cap H_j))=\iota(h_2(N \cap H_j))$ for some $h_1,h_2 \in H_j$, then $h_1(N \cap H_{j+1})=h_2(N \cap H_{j+1})$, i.e. $h_2^{-1}h_1 \in N \cap H_{j+1} \subseteq N$. As $h_1,h_2 \in H_j$, we have $h_{2}^{-1}h_1(N \cap H_j)=N \cap H_j$, i.e. $h_1(N \cap H_j)=h_2(N \cap H_j)$. Thus, $\iota$ is injective. \\

For the "normal" part, given any $g \in H_{j+1}$ and $h \in H_j$ we compute:

\begin{align*}
    g(N \cap H_{j+1})[\iota(h(N \cap H_j))]g^{-1}(N \cap H_{j+1}) &= g(N \cap H_{j+1})[h(N \cap H_{j+1})]g^{-1}(N \cap H_{j+1}) \\
    &=(ghg^{-1})(N \cap H_{j+1})
\end{align*}

Since $ghg^{-1} \in H_j$ by $H_j \trianglelefteq H_{j+1}$, we have $(ghg^{-1})(N \cap H_{j+1})=\iota((ghg^{-1})(N \cap H_j)) \in \iota(H_j/N \cap H_j)$. Since $\iota$ is injective, this means that $(H_j/N \cap H_j) \trianglelefteq (H_{j+1} / N \cap H_{j+1})$\\

To ease notation, we now let $H_{j}'=H_j/N \cap H_j$, $H_{j+1}'=H_{j+1}/N \cap  H_{j+1}$. Clearly, by the common quotient map, $H_{j+1} \surj H_{j+1}'$ is surjective, and $H_{j+1}' \surj H_{j+1}'/H_j'$ is also surjective. Thus, there exists a surjective homomorphism $\nu:H_{j+1} \surj H_{j+1}'/H_{j}'$. If $h \in H_j \subseteq H_{j+1}$, then $h(N \cap H_{j+1})=N \cap H_{j+1}$, which means $h$ in the kernel of the first homomorphism, and thus in $\ker \nu$. Therefore, $H_j$ is contained in the kernel of $\nu$. So, by the first isomorphism theorem, $H_{j+1}/H_j \supseteq H_{j+1}/\ker\nu \simeq \Img \nu=H_{j+1}'/H_{j}'$. Since $H_{j+1}/H_j$ is abelian by assumption, $H_{j+1}'/H_j'$ is also abelian, which completes the proof. \\ \\

\textbf{Theorem 32. }Let $F$ be a field, $f(x) \in F[x]$, and $K$ be the splitting field for $f(x)$ over $F$. ($\chr F=0$ implies that $K/F$ is Galois) Then:

\[
    f(x) \text{ is solvable by radicals }\Rightarrow\text{ $\Gal(K/F)$ is solvable}
\]

pf. First, by definition, there exists a field extension $L-K-F$ s.t. $L/F$ is radical and Galois. Then, $\Gal(L/F)$ is solvable since a radical extension can be decomposed into Kummer extensions and a Kummer extension $E(\alpha,\zeta_n)/E$ can be decomposed into $E(\alpha,\zeta_n)-E(\zeta_n)-E$, where $\Gal(E(\alpha,\zeta_n)/E(\zeta_n))$ and $\Gal(E(\zeta_n)/E)$ are abelian. Since $K/F$ is Galois, $\Gal(K/F) \simeq \Gal(L/F)/\Gal(L/E)$. Since $\Gal(L/F)$ is solvable, by lemma 31-(ii), we have $\Gal(K/F)$ is also solvable. \\ \\

\textbf{Remark. }To prove that a general polynomial $f(x) \in F[x]$ has no general solution using radicals, by theorem 32, it suffices to prove that $\Gal(K/F)$ is not solvable, where $K$ is the splitting field for $f(x)$ over $F$. \\ \\

\textbf{Theorem 33. }For all $n \ge 5$, $\mathcal{S}_n$ is not a solvable group. \\

pf. Suppose that $\mathcal{S}_n$ is solvable, then there is a sequence of subgroups $\{e\}=H_0 \subsetneq H_1 \subsetneq \cdots \subsetneq H_{k-1} \subsetneq H_k=\mathcal{S}_n$ satisfy the condition of a solvable group. We know that all normal subgroups of $\mathcal{S}_n$ are $\{e\}$, $A_n$, $\mathcal{S}_n$, so $H_{k-1}$ can only be $\{e\}$ or $A_n$. But $\mathcal{S}_n/\{e\}=\mathcal{S}_n$ is not abelian, so $H_{k-1}=A_n$. However, $A_n$ is simple, which means the sequence is $\{e\} \subsetneq A_n \subsetneq \mathcal{S}_n$. This leads to a contradiction that $A_n/\{e\}=A_n$ is not abelian. Hence, $\mathcal{S}_n$ is not solvable. \\ \\

\textbf{Def. }Let $F$ be a ring and $F[X_1,...,X_n]$ be the multivariate polynomial ring. We define the \textbf{field of rational functions} to be the fractional field of $F[X_1,...,X_n]$, denoted as $F(X_1,...,X_n)$. \\ \\

\textbf{Def. }Let $\mathcal{S}_n \acts F[X_1,...,X_n]$ by $\sigma(X_i)=X_{\sigma(i)}$. A polynomial $p$ in $F[X_1,...,X_n]$ is said to be a \textbf{symmetric polynomial} if $\pi(p)=p$ for all $\pi \in \mathcal{S}_n$. \\ \\

\textbf{Def. }The \textbf{$k$-th elementary symmetric polynomial} in $F[X_1,...,X_n]$ is defined as $s_k(X_1,...,X_n)=\sum_{1 \le i_1<...<i_k \le n}X_{i_1}X_{i_2}\cdots X_{i_k}$. For instance, $n=3$, $s_1(X_1,X_2,X_3)=X_1+X_2+X_3$, $s_2(X_1,X_2,X_3)=X_1X_2+X_2X_3+X_1X_3$. \\ \\

\textbf{Theorem 34. }Let $F$ be a field and $\chr F=0$, say $F=\mathbb{Q}$. Let $F(t_1,...,t_n)$ be the field of rational functions in $n$ variables. Consider $\mathcal{S}_n \acts F(t_1,...,t_n)$ by $\sigma(t_i)=t_{\sigma(i)}$. Then:

\begin{itemize}
    \item[] (a) Let $s_1,...,s_n \in F[t_1,...,t_n]$ be the elementary symmetric polynomials. Let $f(x)=x^n-s_1x^{n-1}+s_2x^{n-2}-...+(-1)^{n-1}s_{n-1}x+(-1)^ns_n \in F(s_1,...,s_n)[x]$. Then, $F(t_1,...,t_n)$ is the splitting field for $f(x)$ over $F(s_1,...,s_n)$. 
    \item[] (b) $F(t_1,...,t_n)^{\mathcal{S}_n}=F(s_1,...,s_n)$
\end{itemize}

pf-1. Since $\prod_{i=1}^n(x-t_i)$ is exactly $x^n-s_1x^{n-1}+s_2x^{n-2}-...+(-1)^{n-1}s_{n-1}x+(-1)^ns_n$, it implies (a). \\

pf-2. First, given $\sigma \in \mathcal{S}_n$, by definition, $\sigma(s_i)=s_i$ for all $i$, i.e. $s_1,...,s_n \in F(t_1,...,t_n)^{\mathcal{S}_n}$. Thus, $F(s_1,...,s_n) \subseteq F(t_1,...,t_n)^{\mathcal{S}_n} \subseteq F(t_1,..,t_n)$. Moreover, by theorem 11 in 8.2, a splitting field for a polynomial of degree $n$ have dimension at most $n!$. By (a), $F(t_1,...,t_n)$ is a splitting field for $f(x)$ over $F(s_1,...,s_n)$, so:

\begin{align*}
    (\deg f)!&=n! \\
    &\ge [F(t_1,...,t_n):F(s_1,...,s_n)] \\
    &\ge [F(t_1,...,t_n):F(t_1,...,t_n)^{\mathcal{S}_n}] \\
    &=\#\Gal(F(t_1,...,t_n)/F(t_1,...,t_n)^{\mathcal{S}_n})\text{ (By lemma 19, it is Galois)} \\
    &=\#\mathcal{S}_n\text{ (By lemma 19, $\Gal(K/K^H)=H$)} \\
    &=n!
\end{align*}

Therefore, $[F(t_1,...,t_n)^{\mathcal{S}_n}:F(s_1,...,s_n)]=n!/n!=1$, i.e. $F(t_1,...,t_n)^{\mathcal{S}_n}=F(s_1,...,s_n)$, and $\Gal(F(t_1,...,t_n)/F(s_1,...,s_n))=\mathcal{S}_n$. \\ \\

\textbf{Corollary 35. }Let $f(x)=x^n+a_{1}x^{n-1}+a_{2}x^{n-2}+...+a_{n-1}x+a_n \in \mathbb{Q}(a_1,...,a_n)[x]$, and $K$ be the splitting field for $f(x)$ over $\mathbb{Q}(a_1,...,a_n)$. Then:

\begin{itemize}
    \item[] (a) $\Gal(K/\mathbb{Q}(a_1,...,a_n))=\mathcal{S}_n$
    \item[] (b) If $n \ge 5$, then $f(x)$ cannot be solved using radicals over $F$, which means $f(x)=0$ has no general formula using radicals to describe its solution.
\end{itemize}

pf-1. Let $t_1,...,t_n$ be roots of $f(x)$, then $K=\mathbb{Q}(t_1,...,t_n)$. Let $s_1,...,s_n$ be elementary symmetric polynomials in $\mathbb{Q}[t_1,...,t_n]$, then the relation between roots and coefficients tells us $a_i=(-1)^{i}s_i$. Thus, $\mathbb{Q}(a_1,...,a_n)=\mathbb{Q}(s_1,...,s_n)=\mathbb{Q}(t_1,...,t_1)^{\mathcal{S}_n}=K^{\mathcal{S}_n}$, which tells us $\mathcal{S}_n=\Gal(K/K^{\mathcal{S}_n})=\Gal(K/\mathbb{Q}(a_1,...,a_n))$. \\

pf-2. By theorem 33, $\mathcal{S}_n$ is not solvable when $n \ge 5$. Since $K$ is the splitting field for $f(x)$ over $\mathbb{Q}(a_1,...,a_n)$, by theorem 32, $f(x)$ is not solvable by radicals, which proves the claim. \\ \\

\textbf{Remark. }Corollary 35 is a "transcendental case", which sees $t_1,t_2,...,t_n$ as variables and thus can be permuted by $\Gal(K/\mathbb{Q}(a_1,...,a_n))$. But when we are given an actual polynomial $f(x) \in \mathbb{Q}[x]$, then $\mathbb{Q}(a_1,...,a_n)=\mathbb{Q}$, and $\Gal(G/\mathbb{Q})$ may fix some roots of $f(x)$, which reduces the possibilities of $\Gal(G/F)$. For instance, if $f(x)=(x-1)(x-2)(x-3)$, then $\Gal(K/\mathbb{Q})$ can only be $\{\id\}$ since other elements in $\mathcal{S}_3$ will not fix $1,2,3$. (That is, $\mathcal{S}_3$ is not a subgroup of $\Gal(K/\mathbb{Q})$, so $\mathcal{S}_3 \ne \Aut(K/K^{\mathcal{S}_3})$.) \\ \\

\textbf{Lemma 36. }Let $p$ be a prime and $G \subseteq \mathcal{S}_p$ be a subgroup. If there exist $\sigma,\tau \in G$ s.t. $\ord(\sigma)=p$ and $\tau$ is a transposition, then $G=\mathcal{S}_p$. \\

pf. If $p=2$, then $\mathcal{S}_p=\langle \tau \rangle$. If $p$ is an odd prime, we first relabel $\{1,2,...,p\}$ s.t. $\sigma=(1\ 2\ \cdots\ p)$. Suppose $\tau=(i\ j)$ with $1 \le i<j \le p$, then:

\[
    \sigma\tau\sigma^{-1}=(1\ 2\ \cdots\ p)(i\ j)(1\ 2\ \cdots\ p)^{-1}=(i+1\quad j+1)
\]

Note that we see the labels (like $i$ or $j+1$) as elements in the field $F=\{1,2,...,p\}$, where $\mathbb{F}_p \simeq F$ by $\phi:x \mapsto x+1$, in order that they won't exceed $p$. Thus, $(j\quad j+(j-i))=\sigma^{j-i}\tau(\sigma^{j-i})^{-1} \in G$. Then, since $j+(j-i)=2j-i=i+2(j-i)$, we have:

\[
    (j\quad j+(j-i))(i\quad j)(j\quad j+(j-i))^{-1}=(i\quad i+2(j-i))
\]

So, $\pi_k=(i\quad i+2k(j-i)) \in G$, for $k=0,1,2,...,p-1$ (we set $\pi_0=\id$). Since $p$ is an odd prime, we have $\gcd(2(j-i),p)=1$, which means for all $\ell \in F=\{1,2,...,p\}$, there exists $k$ s.t. $\pi_k=(i\quad \ell) \in G$. Therefore, for any $a,b \in F$, we have $(a\ b)=(i\ a)(i\ b)(a\ i)=(i\ a)(i\ b)(i\ a)^{-1} \in G$, which means $G$ contains all traspositions in $\mathcal{S}_p$. By theorem 5 in 12.1, every element in $\mathcal{S}_p$ can be written as a product of trasposition. Therefore, $G=\mathcal{S}_p$. \\ \\

\textbf{Theorem 37. }Let $p$ be a prime and $f(x) \in \mathbb{Q}[x]$ be an irreducible polynomial having $p-2$ real roots and $2$ non-real roots in $\mathbb{C}$. If $K$ is a splitting field for $f(x)$ over $\mathbb{Q}$, then $K=\mathcal{S}_p$ (in isomorphic sense). \\

pf. Since $K$ is a splitting field for $f$ over $\mathbb{Q}$ and $\chr\mathbb{Q}=0$, $K/F$ is Galois. We know that every element in $\Gal(K/F)$ should send root of $f(x)$ to root of $f(x)$, so $\Gal(K/F) \subseteq \mathcal{S}_p$. By lemma 36, it suffices to show that $\Gal(K/F)$ contains a $p$-cycle and a trasposition. \\

Let $\alpha_1,...,\alpha_p$ be distinct roots of $f$ with $\alpha_1,\alpha_2 \notin \mathbb{R}$, $\alpha_3,...,\alpha_p \in \mathbb{R}$. Then $K=\mathbb{Q}(\alpha_1,...,\alpha_n)$. Since $f$ is irreducible, $[\mathbb{Q}(\alpha_1):\mathbb{Q}]=p$, which says that $p \mid [K:\mathbb{Q}]=\#\Gal(K/\mathbb{Q})$. By Sylow's first theorem, there exists a subgroup of $\Gal(K/\mathbb{Q})$ of $p$-power order. In 9.7, we showed that a group of order $p^e$ contains a subgroup of order $p^i$, for all $0 \le i \le e$. Therefore, $\Gal(K/\mathbb{Q})$ contains a subgroup of order $p$, which is cyclic, say $\langle \sigma \rangle$ and $\ord(\sigma)=p$. Consider the cycle decomposition of $\sigma$, say it has cycle structure $\{m_1,...,m_k\}$, then $\sigma^{\text{lcm}(m_1,...,m_k)}=\id$, so $p \mid \text{lcm}(m_1,...,m_k)$. But $\ord(\sigma)=p$ is a prime and $m_1+...+m_k=p$, the only possible situation is $\sigma$ is a $p$-cycle. Therefore, $\Gal(K/\mathbb{Q})$ contains a $p$-cycle. \\

On the other hand, since $f \in \mathbb{Q}[x] \subset \mathbb{R}[x]$, for any $1 \le i \le p$, $\overline{\alpha_i}$ is also a root of $f$. Therefore, the automorphism $\tau=\overline{\cdot}$ is in $\Gal(K/\mathbb{Q})$, and $\tau$ is a transposition. In conclusion, $G$ contains $\sigma$ and $\tau$ with $\ord(\sigma)=p$, $\ord(\tau)=2$. By lemma 36, $\Gal(K/\mathbb{Q})=\mathcal{S}_p$.

\subsection*{9.12 Galois group of polynomials of degree 3}
\indent

\textbf{Note. }In this section, $\chr F=0$. \\ \\

\textbf{Question. }Given a separable polynomial $f \in F[x] \setminus \{0\}$, $\deg f=3$, $K$ be the splitting field for $f$ over $F$. We want to determine what is $K$. \\ \\

\textbf{Note. }Since $f(x)$ has $3$ roots (including multiple roots) in $K$, and any $\sigma \in \Gal(K/F)$ can only send root to root, $\Gal(K/F)$ can only be a subgroup of $\mathcal{S}_3$. Note that $\mathcal{S}_3$ is a solvable group since $\{e\} \subsetneq \mathcal{C}_3 \subsetneq \mathcal{S}_3$ and this sequence satisfies the condition of a solvable group. \\ \\

\textbf{Note. }Actually, $\mathcal{C}_3=\mathcal{A}_3=\{e,(1\ 2\ 3),(1\ 3\ 2)\}$. For the definition of alternating groups, see chapter 12. \\ \\

\textbf{Case 1. }If $f$ is reducible, then $f$ has exactly one root in $F$ or $f$ has three roots in $f$. \\

If $f$ has exactly one root in $F$, then any $\sigma \in \Gal(K/F)$ can only permute the remaining two roots. Thus, $\Gal(K/F)=\mathcal{S}_2$. \\

If $f$ has three roots in $F$, then all elements in $\Gal(K/F)$ fix the roots of $F$, i.e. $\Gal(K/F)=F$. \\ \\

\textbf{Case 2. }If $f$ is irreducible, then for any $\theta$, root of $f$, we have $F[x]/(f) \simeq F(\theta) \subseteq K$, which means $3=\deg f=[F(\theta):F] \mid [K:F]=\#\Gal(K/F)$. But $\Gal(K/F) \subseteq \mathcal{S}_3$, so $\#\Gal(K/F)=3$ or $6$ (Spoiler alert: this means $\Gal(K/F)=\mathcal{A}_3$ or $\mathcal{S}_3$). How to determine whether $\#\Gal(K/F)=3$ or $6$? \\

\textbf{Lemma 38. }Let $f(x)$ be a separable polynomial of degree $n$. In transcendental case, let $t_1,...,t_n$ be roots of $f(x)$ (which are regarded as variables), define $\delta:=\prod_{1 \le i<j \le n}(t_i-t_j) \in F[t_1,...,t_n]$. Let $K$ be the splitting field for $f(x)$ over $F(s_1,...,s_n)$, then $K=F(t_1,...,t_n)$, $K^{\mathcal{A}_n}=F(s_1,...,s_n,\delta)$, $K^{\mathcal{S}_n}=F(s_1,...,s_n)$. \\

pf. We only need to show that $K^{\mathcal{A}_n}=F(s_1,...,s_n,\delta)$ since the others are shown in theorem 34 and corollary 35-(a). Consider $\mathcal{S}_n \acts \{t_1,...,t_n\}$ by $\sigma(t_i)=t_{\sigma(i)}$. Let $D=\delta^2$, then $\sigma(D)=D$, for any $\sigma \in \mathcal{S}_n$. So, $\sigma(\delta)=\pm \delta$. Define the homomorphism $[\sign:\mathcal{S}_n \to (\{\pm 1\},\times)]$ by $\sign(\sigma)=\text{(The sign of $\sigma(\delta)$)}$, and $\mathcal{A}_n:=\ker(\sign)$. Clearly, 
$K^{\mathcal{A}_n} \supseteq F(s_1,...,s_n,\delta)$. Moreover, the minimal polynomial for $\delta$ is $x^2-D$, so $[F(s_1,...,s_n,\delta):F(s_1,...,s_n)]=2$, which means:

\begin{align*}
    [K^{\mathcal{A}_n}:F(s_1,...,s_n,\delta)]&=[K:F(s_1,...,s_n,\delta)]/[K:K^{\mathcal{A}_n}] \\
    &=[K:F(s_1,...,s_n)]/([K:K^{\mathcal{A}_n}][F(s_1,...,s_n,\delta):F(s_1,...,s_n)]) \\
    &=\#\mathcal{S}_n/(\#\mathcal{A}_n \cdot 2) \\
    &=1
\end{align*}

Therefore, $K^{\mathcal{A}_n}=F(s_1,...,s_n,\delta)$. \\ \\

\textbf{Back to case 2. }Now, back to our setting that $\deg f=3$ and $f$ is given. As we remark in 9.11, $\Gal(K/F)$ may not be $\mathcal{S}_3$ and thus $K^{\mathcal{S}_3}$ (or even $K^{\mathcal{A}_3}$) is not well-defined. So, the way we determine that whether $\Gal(K/F)=\mathcal{A}_3$ or $\mathcal{S}_n$ is to see that whether $\sigma$ fixes $\delta$, for all $\sigma \in \Gal(G/F)$, i.e. whether $F(\delta)=F$ or not. Note that $F(\delta)=F \Leftrightarrow D=\delta^2 \in (F^\times)^2$, so we compute $D$ explicitly. Since $f(x)=(x-t_1)(x-t_2)(x-t_3)$, we have $f'(x)=(x-t_1)(x-t_2)+(x-t_1)(x-t_3)+(x-t_2)(x-t_3)$. Then:

\begin{gather*}
    f'(t_1)=(t_1-t_2)(t_1-t_3) \\
    f'(t_2)=(t_2-t_1)(t_2-t_3)=-(t_1-t_2)(t_2-t_3) \\
    f'(t_3)=(t_3-t_1)(t_3-t_2)=(t_1-t_3)(t_2-t_3) \\
    D=\left(\prod_{1 \le i<j \le 3}(t_i-t_j)\right)^2=-f'(t_1)f'(t_2)f'(t_3)
\end{gather*}

WLOG, we suppose $f(x)=x^3+px+q$. For this $f$, we have $\left\{\begin{matrix}
    s_1&=&0 \\
    s_2&=&p \\
    s_3&=&q
\end{matrix}\right.$. We know the relation between roots and coefficients is $\left\{\begin{matrix}
    t_1+t_2+t_3&=&-s_1&=&0 \\
    t_1t_2+t_2t_3+t_3t_1&=&s_2&=&p \\
    t_1t_2t_3&=&-s_3&=&-q
\end{matrix}\right.$. So: \\

\[
    \left\{\begin{matrix}
        0&=&s_1^2&=&(t_1^2+t_2^2+t_3^2)+2(t_1t_2+t_2t_3+t_3t_1)=(t_1^2+t_2^2+t_3^2)+2p \\
        p^2&=&s_2^2&=&(t_1^2t_2^2+t_2^2t_3^2+t_3^2t_1^2)+2(t_1^2t_2t_3+t_1t_2^2t_3+t_1t_2t_3^2)=t_1^2t_2^2+t_2^2t_3^2+t_3^2t_1^2
    \end{matrix}\right.
\]

Then we have:

\begin{align*}
    D&=-f'(t_1)f'(t_2)f'(t_3) \\
    &=-(3t_1^2+p)(3t_2^2+p)(3t_3^2+p) \\
    &=-27(t_1t_2t_3)^2-9p(t_1^2t_2^2+t_2^2t_3^2+t_3^2t_1^2)-3p(t_1^2+t_2^2+t_3^2)-p^3 \\
    &=-27q^2-9p^3+6p^3-p^3 \\
    &=-4p^3-27q^2
\end{align*}

We say $D=-4p^3-27q^2$ is the \textbf{discriminant} of $f(x)$. In conclusion, if $D \in (F^\times)^2$, then $F(\delta)=F$, which means $\Gal(K/F)=\mathcal{A}_3$. If $D \notin (F^\times)^2$, then $[F(\delta):F]=[F(\sqrt{D}):F]=2$, so $2 \mid \#\Gal(K/F)=[K:F]$. Along with $3 \mid [K:F]$, we have $6 \mid [K:F]$ and thus $\Gal(K/F)=\mathcal{S}_3$.

\subsection*{9.13 Galois group of polynomials of degree 4}
\indent

\textbf{Note. }In this section, $\chr F=0$. \\ \\


\textbf{Question. }Given a separable polynomial $f \in F[x] \setminus \{0\}$, $\deg f=4$, $K$ be the splitting field for $f$ over $F$. We want to determine what is $K$. \\ \\

\textbf{Def. }Let $n \ge 1$ and $\sigma \in \mathcal{S}_n$. If the cycle decomposition of $\sigma$ is:

\[
    \sigma=(a_{1,1}\ a_{1,2}\ \cdots\ a_{1,m_1})(a_{2,1}\ a_{2,2}\ \cdots\ a_{2,m_2})\cdots(a_{k,1}\ a_{k,2}\ \cdots\ a_{k,m_k})
\]

Then we define the \textbf{cycle structure} of $\sigma$ to be the multiset $\{m_1,...,m_k\}$. For instance, $\sigma=(1\ 2)(3\ 4) \in \mathcal{S}_4$ has cycle structure $\{2,2\}$. We also say $\sigma$ has a $(2,2)$-type cycle structure. \\ \\

\textbf{Def. }In $\mathcal{S}_4$, the \textbf{Klein four-group} is the group $V=\{e,(1\ 2)(3\ 4),(1\ 3)(2\ 4),(1\ 4)(2\ 3)\}$, i.e. $V=\{e\} \cup \{\sigma \in \mathcal{S}_4:\text{$\sigma$ has a $(2,2)$-type cycle structure}\}$. \\ \\

\textbf{Theorem 39. }Given $n \ge 1$, then two elements in $\mathcal{S}_n$ have the same cycle structure if and only if they are conjugated. (A helful note: conjugation is purely relabeling.) \\

pf ($\Rightarrow$). Given $\pi_1,\pi_2 \in \mathcal{S}_n$ s.t. they share the same cycle structure, say:

\begin{gather*}
    \pi_1=(a_{1,1}\ a_{1,2}\ \cdots\ a_{1,m_1})(a_{2,1}\ a_{2,2}\ \cdots\ a_{2,m_2})\cdots(a_{k,1}\ a_{k,2}\ \cdots\ a_{k,m_k}) \\
    \pi_2=(b_{1,1}\ b_{1,2}\ \cdots\ b_{1,m_1})(b_{2,1}\ b_{2,2}\ \cdots\ b_{2,m_2})\cdots(b_{k,1}\ b_{k,2}\ \cdots\ b_{k,m_k})
\end{gather*}

Define $\sigma:\{1,2,...,n\} \to \{1,2,...,n\}$ by $\sigma(a_{i,j})=b_{i,j}$, then it is clearly injective since all cycles are disjoint. By counting elements, $\sigma$ is bijective and thus $\sigma \in \mathcal{S}_n$. Now, for all $i,j$, we check:

\[
    \left\{\begin{matrix}
        \sigma\pi_1\sigma^{-1}(b_{i,j})=\sigma\pi_1(a_{i,j})=\sigma(a_{i,{j+1}})=b_{i,j+1}=\pi_2(b_{i,j})&,&\text{if $j<m_i$} \\
        \sigma\pi_1\sigma^{-1}(b_{i,j})=\sigma\pi_1(a_{i,j})=\sigma(a_{i,{1}})=b_{i,1}=\pi_2(b_{i,j})&,&\text{if $j=m_i$}
    \end{matrix}\right.
\]

Therefore, $\pi_2=\sigma\pi_1\sigma^{-1}$, i.e. they are conjugated. \\

pf ($\Leftarrow$). Given $\pi \in \mathcal{S}_n$, suppose the cycle decomposition of $\pi$ is $\pi_1\pi_2...\pi_r$, i.e. $\pi_i$ and $\pi_j$ are disjoint if $i \ne j$. Given any $\sigma \in \mathcal{S}_n$, we know that $\sigma$ has a transposition decomposition, i.e. $\sigma$ can be written as a product of transposition. So, it suffices to show that for every transposition $\tau \in \mathcal{S}_n$, $\tau\pi\tau^{-1}=\tau\pi\tau$ has the same cycle structure as $\pi$. After relabeling, $\tau=(1\ 2)$. WLOG, we only need to discuss the case that $\pi_1$ contains $1$ and $2$ and the case that $\pi_1$ contains $1$, $\pi_2$ contains $2$. ($\because$ Disjoint cycles commute.) \\

Case 1. $\pi_1$ contains $1,2$. Suppose $\pi_1=(1\ x_1\ \cdots\ x_i\ 2\ y_1\ \cdots\ y_j)$ for some $i,j \ge 0$ and distinct $1,2,x_1,...,x_i,y_1,...,y_j$, then:

\[
    \tau\pi_1\tau=(1\ 2)(1\ x_1\ \cdots\ x_i\ 2\ y_1\ \cdots\ y_j)(1\ 2)=(1\ y_1\ \cdots\ y_j\ 2\ x_1\ \cdots\ x_i)
\]

So, $\pi_1$ and $\tau\pi_1\tau$ are both $(i+j+2)$-cycle. \\

Case 2. $\pi_1$ contains $1$ and $\pi_2$ contains $2$. Suppose $\pi_1=(1\ x_1\ \cdots\ x_i)$, $\pi_2=(2\ y_1\ \cdots\ y_j)$ for some $i,j \ge 0$ and distinct $1,2,x_1,...,x_i,y_1,...,y_j$, then:

\[
    \tau\pi_1\pi_2\tau=(1\ 2)(1\ x_1\ \cdots\ x_i)(2\ y_1\ \cdots\ y_j)(1\ 2)=(1\ y_1\ \cdots\ y_j)(2\ x_1\ \cdots\ x_i)
\]

So, $\pi_1\pi$ and $\tau\pi_1\pi_2\tau$ are both $(i+1,j+1)$-type. In conclusion, $\tau\pi\tau$ and $\pi$ share the same cycle structure, which gives us $\sigma\pi\sigma^{-1}$ and $\pi$ share the same cycle structure. \\ \\

\textbf{Remark. }$V$ is the only Sylow 2-subgroup of $\mathcal{A}_4$ and $V \trianglelefteq \mathcal{A}_4$, $V \trianglelefteq \mathcal{S}_4$. \\

pf. We show that $V \trianglelefteq \mathcal{S}_4$. (Since $V \subseteq \mathcal{A}_4$, it implies $V \trianglelefteq \mathcal{A}_4$.) By Sylow's second theorem, it also means that $V$ is the only Sylow 2-subgroup of $\mathcal{A}_4$. Given $\pi \neq e$ in $V$ and $\sigma \in \mathcal{S}_4$, then by theorem 39, $\sigma\pi\sigma^{-1}$ is still a $(2,2)$-type element. Therefore, $\sigma\pi\sigma^{-1} \in V$, which means $V \trianglelefteq \mathcal{S}_4$. \\ \\

\textbf{Start the discussion. }Similarly, $\Gal(K/F)$ can only be a subgroup of $\mathcal{S}_4$, which is a solvable group since $\{e\} \subsetneq V \subsetneq \mathcal{A}_4 \subsetneq \mathcal{S}_4$ and this sequence satisfies the condition of a solvable group. \\ \\

\textbf{Case 1. }If $f$ is reducible, then there are two possible situations: \\

If $f(x)=(x-\theta)g(x)$ with $\theta \in F$ and $g$ is an irreducible polynomial of degree $3$, then any $\sigma \in \Gal(K/F)$ can only permutes roots of $g(x)$, i.e. $\Gal(K/F) \subseteq \mathcal{S}_3$. It reduces to the degree $3$ case. \\

If $f(x)=g(x)h(x)$ for some $g,h$ of degrees $2$. The $K/F$ is a biquadratic extension (if $g,h$ irreducible), or a quadratic extension (only one of them is irreducible), or $K=F$ (if $g,h$ reducible, i.e. $f$ splits). \\ \\

\textbf{Case 2. }If $f$ is irreducible, then for any $\theta$, root of $f$, we have $F[x]/(f) \simeq F(\theta) \subseteq K$, which means $4=\deg f=[F(\theta):F] \mid [K:F]=\#\Gal(K/F)$. So, $\#\Gal(K/F)=4,8,12,24$. In order to reduce the case to degree 3 case, we first introduce lemma 40 and proposition 41. \\ \\

\textbf{Lemma 40. }Let $K$ be the splitting field for an irreducible, separable polynomial $f(x)=x^4+a_1x^3+a_2x^2+a_3x+a_4$ over $F$, then $\Gal(K/F)$ acts transitively on the roots of $f$. \\

pf. Let $S=\{\theta_1,\theta_2,\theta_3,\theta_4\}$ be roots of $f$. Since $f$ is irreducible, by lemma 1, there exists $\sigma:F(\theta_i) \to F(\theta_j)$ s.t. $\sigma$ fixes the elements in $F$ and $\sigma(\theta_i)=\theta_j$, for any $i,j$. Also, by lemma 10, $\sigma$ can be extended to an isomorphism on $K$, i.e. $\sigma \in \Gal(K/F)$. Therefore, $\Gal(K/F)$ acts transitively on $S$. \\ \\

\textbf{Proposition 41. }Let $\mathcal{S}_n$ be the symmetric group of order $n!$ and $V$ be the Klein four-group, then $\mathcal{S}_4/V \simeq \mathcal{S}_3$. \\

pf. Let $T=\{t_1t_2+t_3t_4,\ t_1t_3+t_2t_4,\ t_1t_4+t_2t_3\}$, it's easy to check that $\mathcal{S}_4 \acts T$ by $\sigma(t_i)=t_{\sigma(i)}$ and $V$ fixes any elements in $T$. Therefore, this group action defined a homomorphism $\phi:\mathcal{S}_4 \to \mathcal{S}_3$ with kernel being $V$, which means $\mathcal{S}_4/V \simeq \Img\phi$. Our goal is to show that $\phi$ is injective, then by $\#(\mathcal{S}_4/V)=24/4=6=\#\mathcal{S}_3$, it is also bijective. \\

Given $\sigma \in \ker\phi$, then $\sigma(t_1t_2+t_3t_4)=t_1t_2+t_3t_4$. Since $\sigma(t_1t_2+t_3t_4)=t_{\sigma(1)}t_{\sigma(2)}+t_{\sigma(3)}t_{\sigma(4)}$ by definition, we have $\sigma(t_1t_2)=t_1t_2$ or $\sigma(t_1t_2)=t_3t_4$. Therefore, $\sigma \in \langle (1\ 2),(3\ 4), V_4 \rangle$. But by $\sigma(t_1t_3+t_2t_4)=t_1t_3+t_2t_4$ and $\sigma(t_1t_4+t_2t_3)=t_1t_4+t_2t_3$, we can deduce that $\sigma \in V$, i.e. $\ker\phi$ is trivial and thus $\phi$ is injective. In conclusion, $\mathcal{S}_4/V \simeq \mathcal{S}_3$ by the isomorphism $\phi$. \\ \\

\textbf{Lemma 42. }Let $f(x)$ be a separable polynomial of degree $4$. In transcendental case, let $t_1,t_2,t_3,t_4$ be roots of $f(x)$ (which are regarded as variables). Let $K$ be the splitting field for $f(x)$ over $F(s_1,s_2,s_3,s_4)$. Denote $r_1=t_1t_2+t_3t_4,r_2=t_1t_3+t_2t_4,r_3=t_1t_4+t_2t_3$, then $K=F(t_1,...,t_n)$, $K^V=F(r_1,r_2,r_3,s_1,s_2,s_3,s_4)$, $K^{\mathcal{A}_4}=F(s_1,s_2,s_3,s_4,\delta)$, $K^{\mathcal{S}_n}=F(s_1,...,s_n)$. \\

pf. We only need to show that $K^V=F(r_1,r_2,r_3,s_1,s_2,s_3,s_4)$ since the others are shown in theorem 34, corollary 35-(a), and lemma 38. We consider the \textbf{resolvent cubic} $H(x)=(x-r_1)(x-r_2)(x-r_3)$, then $\forall \sigma \in \mathcal{S}_4$, $\sigma(H)=H$. Therefore, $H \in F(s_1,s_2,s_3,s_4)[x]$. Since we are in transcendental case, if $\overline{\delta}=\prod_{1 \le i<j \le 3}(r_i-r_j)$, then $\overline{\delta} \notin F(s_1,s_2,s_3)$, which gives us the splitting field for $H$ over $F(s_1,s_2,s_3)$, $F(r_1,r_2,r_3,s_1,s_2,s_3,s_4)$, has dimension $6$ by the discussion in 9.12. By the proof in proposition 41, $K^V \supseteq F(r_1,r_2,r_3,s_1,s_2,s_3,s_4)$, then we have:

\begin{align*}
    [K^V:F(r_1,r_2,r_3,s_1,s_2,s_3,s_4)]&=[K:F(r_1,r_2,r_3,s_1,s_2,s_3,s_4)]/[K:K^V] \\
    &=\frac{[K:F(s_1,s_2,s_3,s_4)]}{[K:K^V][F(r_1,r_2,r_3,s_1,s_2,s_3,s_4):F(s_1,s_2,s_3,s_4)]} \\
    &=\#\mathcal{S}_4/(\#V \cdot 6) \\
    &=1
\end{align*}

Therefore, $K^V=F(r_1,r_2,r_3,s_1,s_2,s_3,s_4)$. \\ \\

\textbf{Back to case 2. }Now, we start classifying $\Gal(K/F)$. By lemma 40, we know that $\Gal(K/F) \ne \{e,(1\ 2),(3\ 4),(1\ 2)(3\ 4)\}$. In transcendental case, if $t_i$ are roots of $f(x)$ and $s_i$ are elementary symmetric polynomials, by lemma 42, the Galois correspondence of $\{e\} \subsetneq V \subsetneq \mathcal{A}_4 \subsetneq \mathcal{S}_4$ is:

\[
    F(t_1,t_2,t_3,t_4) \supsetneq F(r_1,r_2,r_3,s_1,s_2,s_3,s_4) \supsetneq F(s_1,s_2,s_3,s_4,\delta) \supsetneq F(s_1,s_2,s_3,s_4),
\]

where $r_1=t_1t_2+t_3t_4,\ r_2=t_1t_3+t_2t_4,\ t_1t_4+t_2t_3,\ \delta=\prod_{1 \le i<j \le 4}(t_i-t_j)$. \\

Now, suppose $f(x)=x^4+cx^2+dx+e \in F[x]$, i.e. $s_1=0,\ s_2=c,\ s_3=d,\ s_4=e$ are all in $F$. Let the roots of $f$ be $\theta_1,\theta_2,\theta_3,\theta_4$, then the Galois correspondence becomes $F(\theta_1,\theta_2,\theta_3,\theta_4) \supseteq F(r_1,r_2,r_3) \supseteq F(\delta) \supseteq F$, where $r_1=\theta_1\theta_2+\theta_3\theta_4,\ r_2=\theta_1\theta_3+\theta_2\theta_4,\ \theta_1\theta_4+\theta_2\theta_3,\ \delta=\prod_{1 \le i < j \le 4}(\theta_i-\theta_j)$. By $V \trianglelefteq \mathcal{S}_4$, $F(r_1,r_2,r_3)/F$ is Galois. \\

Let $K=F(\theta_1,\theta_2,\theta_3,\theta_4)$ be the splitting field for $f$ over $F$. By lemma 42, $F(r_1,r_2,r_3)$ is the splitting field for the resolvent cubic $h(x)=(x-r_1)(x-r_2)(x-r_3)$ over $F$. Then, we can apply the case of polynomials of degree $3$ to $F(r_1,r_2,r_3)/F$. Note that we can compute to get $h(x)=x^3-cx^2-4ex-(d^2-4ce)$. \\

Compute the discriminant $\disc(f)=(\prod_{1 \le i<j \le 4}(\theta_i-\theta_j))^2$ and $\disc(h)=(\prod_{1 \le i<j \le 3}(r_i-r_j))^2$. Surprisingly:

\begin{gather*}
    r_1-r_2=(\theta_1-\theta_4)(\theta_2-\theta_3) \\
    r_1-r_3=(\theta_1-\theta_3)(\theta_2-\theta_4) \\
    r_2-r_3=(\theta_1-\theta_2)(\theta_3-\theta_4)
\end{gather*}

This tells us $h$ is separable if $f$ is separable and $\disc(f)=\disc(h)$, which allows us to apply the result of 9.12 to $h$. \\ \\

\textbf{Case 2-1. }If $\disc(f)=\delta^2 \in (F^\times)^2$, then, $F(\delta)=F$, which means $\Gal(K/F) \subseteq \mathcal{A}_4$. There are two subcases: \\

\textbf{Case 2-1-a. }If $h$ is irreducible, then $3=\deg h \mid [F(r_1,r_2,r_3):F] \mid [K:F]$. Along with $4 \mid [K:F]$, we have $12 \mid [K:F]$. Therefore, $\Gal(K/F)=\mathcal{A}_4$. \\

\textbf{Case 2-1-b. }If $h$ is reducible, then we only have $4 \mid [K:F]=\#\Gal(K/F)$. But this tells us that $\Gal(K/F)$ contains a Sylow 2-subgroup of $\mathcal{A}_4$, which is exactly $V$. Then $\Gal(K/F)$ can be $\mathcal{A}_4$ or $V$. Claim: $\Gal(K/F) \ne \mathcal{A}_4$. Given $\sigma=(1\ 2\ 3) \in \mathcal{A}_4 \setminus V$, then $\sigma(r_1)=\sigma(t_1t_2+t_3t_4)=t_2t_3+t_1t_4=r_3$, $\sigma(r_3)=\sigma(t_2t_3+t_1t_4)=t_3t_1+t_2t_4=r_2$, which means $\sigma$ acts transitively on $T=\{r_1,r_2,r_3\}$. But $h$ is reducible, at least one root of $h$ is in $F$ and thus any elements in $\Gal(K/F)$ cannot acts transitively on $T$. Therefore, $\sigma \notin \Gal(K/F)$, then $\Gal(K/F) \ne \mathcal{A}_4$. In conclusion, $\Gal(K/F)=V$. \\ \\

\textbf{Case 2-2. }If $\disc(f) \notin (F^\times)^2$, then, $\Gal(K/F) \nsubseteq \mathcal{A}_4$ since if $\Gal(K/F) \subseteq \mathcal{A}_4$, we will have $\delta=\sqrt{\disc(f)} \in K^{\Gal(K/F)}=F$. There are two subcases: \\

\textbf{Case 2-2-a. }If $h$ is irreducible, then similarly, $12 \mid [K:F]$. Claim: $\Gal(K/F)=\mathcal{S}_4$. Suppose NOT, i.e. $\#\Gal(K/F)=12$. Since $[\mathcal{S}_4:\Gal(K/F)]=\#\mathcal{S}_4/\#\Gal(K/F)=2$, $\Gal(K/F) \trianglelefteq \mathcal{S}_4$. Now, we see the members of $\mathcal{S}_4$:

\begin{center}
    \begin{tabular}{c|c|c}
         & $\#$ & Note \\
       \hline
       $e$  & $1$ & \text{Trivial} \\
       \hline
       $2$-cycle  & $6$ & $\binom{4}{2}$ \\
       \hline
       $3$-cycle  & $8$ & $\binom{4}{3} \cdot 2$ \\
       \hline
       $4$-cycle  & $6$ & Circular permutation $3!=6$ \\
       \hline
       $(2,2)$-type  & $3$ & All in $V$
    \end{tabular}
\end{center} 

If $\Gal(K/F)$ contains an element of cycle structure $S$, then it contains all elements of cycle structure $S$ since $\Gal(K/F)$ is normal and these elements are conjugated by theorem 39. Therefore, the only choice for $\Gal(K/F)$ s.t. $\#\Gal(K/F)=12$ is that $1+8+3=12$, i.e. $\Gal(K/F)=\mathcal{A}_4$, which is a contradiction. Thus, $\#\Gal(K/F)=24$, i.e. $\Gal(K/F)=\mathcal{S}_4$. \\

\textbf{Case 2-2-b. }If $h$ is reducible, then we only have $4 \mid [K:F]$. As the case in 9.12, since $(1\ 2\ 3) \in \mathcal{S}_3$ permutes the three roots of $h$, it is not allowed. Therefore, $(1\ 2\ 3) \notin \Gal(K/F)$ and thus $\Gal(K/F) \notin \mathcal{S}_4$. We also know that $\Gal(K/F) \nsubseteq \mathcal{A}_4$, the order of $\Gal(K/F)$ can only be $4$ and $8$. \\

If $\#\Gal(K/F)=4$, by Lagrange theorem, it cannot have a three cycle, and if it had a four cycle $\sigma$, then $\Gal(K/F)=\langle \sigma \rangle \simeq \mathcal{C}_4$. Suppose $\#\Gal(K/F)$ only contains $e$, $2$-cycles, and $(2,2)$-type. By lemma 40, it cannot be in the form $\{e,(a\ b),(c\ d),(a\ b)(c\ d)\}$ since it has to act transitively on the roots of $f(x)$. Therefore, $\Gal(K/F)$ can only be $V=\{e,(1\ 2)(3\ 4),(1\ 3)(2\ 4),(1\ 4)(2\ 3)\}$. But it leads to a contradiction since in this case, $\disc(f) \notin (F^\times)^2$  and thus $\#\Gal(K/F) \nsubseteq \mathcal{A}_4$. Therefore, the only choice is $\mathcal{C}_4$. \\

If $\#\Gal(K/F)=8$, then it is a Sylow $2$-subgroup of $\mathcal{S}_4$. By Sylow's second theorem, all Sylow $2$-subgroups are isomorphic (via conjugation map). Also, $\mathcal{D}_4 \subset \mathcal{S}_4$ is a Sylow $2$-subgroup. Therefore, $\Gal(K/F)=\mathcal{D}_4$. \\ \\

\textbf{Case 2-2-b, I. ($\Gal(K/F) \simeq \mathcal{C}_4$) }By lemma 31-(a), $\mathcal{C}_4$ is solvable by the sequence $\{e\} \cap \mathcal{C}_4 \subseteq V  \cap \mathcal{C}_4 \subseteq \mathcal{A}_4  \cap \mathcal{C}_4 \subseteq \mathcal{S}_4  \cap \mathcal{C}_4$. WLOG, suppose $\mathcal{C}_4=\langle (1\ 2\ 3\ 4)\rangle$, then $\mathcal{A}_4 \cap \mathcal{C}_4=\{e,(1\ 3)(2\ 4)\}$. So, the sequence is: \\

\[
    \{e\} \subsetneq  \{e,(1\ 3)(2\ 4)\} \subsetneq \mathcal{C}_4
\]

The corresponding fixed fields are $K \supsetneq K^{\{e,(1\ 3)(2\ 4)\}} \supsetneq K^{\mathcal{C}_4}$. Since $(1\ 2\ 3\ 4) \in \mathcal{C}_4$, $K^{\mathcal{C}_4}=F$. We know that $\{e,(1\ 3)(2\ 4)\} \subset \mathcal{A}_4$ fixes elements in $F(\sqrt{D})$, so $K^{\{e,(1\ 3)(2\ 4)\}} \supseteq F(\sqrt{D})$. In this case, $D:=\disc(f)$ is not a square, which means $[F(\sqrt{D}):F]=2$. Also, $[K:K^{\{e,(1\ 3)(2\ 4)\}}]=2$. Therefore, $[K^{\{e,(1\ 3)(2\ 4)\}}:F(\sqrt{D})]=[K:F]/([K:K^{\{e,(1\ 3)(2\ 4)\}}][F(\sqrt{D}):F])=\#\mathcal{C}_4/(2 \cdot 2)=1$.  By dimension, we have $K^{\{e,(1\ 3)(2\ 4)\}}=F(\sqrt{D})$. This tells us $f(x)$ is reducible over $F(\sqrt{D})$ since $(1\ 2\ 3\ 4)$ is no longer in $\Gal(K/F(\sqrt{D}))=\{e,(1\ 3)(2\ 4)\}$. (If it is irreducible over $F(\sqrt{D})$, then by lemma 1 and lemma 10, $(1\ 2\ 3\ 4)$ remains in $\Gal(K/F(\sqrt{D}))$). \\ \\


\textbf{Case 2-2-b, II. ($\Gal(K/F) \simeq \mathcal{D}_4$) }By lemma 31-(a), $\mathcal{D}_4$ is solvable by the sequence $\{e\} \cap \mathcal{D}_4 \subseteq V  \cap \mathcal{D}_4 \subseteq \mathcal{A}_4  \cap \mathcal{D}_4 \subseteq \mathcal{S}_4  \cap \mathcal{D}_4$. Since $\mathcal{A}_4 \cap \mathcal{D}_4=V$, the sequence is:

\[
    \{e\} \subsetneq V \subsetneq \mathcal{D}_4
\]

The corresponding fixed fields are $K \supsetneq K^V \supsetneq K^{\mathcal{D}_4}$. Since $(1\ 2\ 3\ 4) \in \mathcal{D}_4$, $K^{\mathcal{D}_4}=F$. We know that $V$ fixes elements in $F(\sqrt{D})$, so $K^{V} \supseteq F(\sqrt{D})$. In this case, $D:=\disc(f)$ is not a square, which means $[F(\sqrt{D}):F]=2$. Also, $[K:K^{V}]=4$. Therefore, $[K^{V}:F(\sqrt{D})]=[K:F]/([K:K^{V}][F(\sqrt{D}):F])=\#\mathcal{D}_4/(4 \cdot 2)=1$. By dimension, we have $K^{V}=F(\sqrt{D})$. This tells us $f(x)$ is irreducible over $F(\sqrt{D})$. (If it is reducible over $F(\sqrt{D})$, say $f=g_1g_2$ for some $g_1$ and irreducible $g_2$, and $\theta_1$ is a root of $g_1$, $\theta_2$ is a root of $g_2$. Then, any $\sigma \in V=\Gal(K/F(\sqrt{D}))$ cannot send $\theta_2$ to $\theta_1$, which contradicts to $(1\ 2)(3\ 4) \in V$. Therefore, $f$ is irreducible) \\ \\

\textbf{Summary for case 2-2-b. }Suppose $f$ is irreducible, $D:=\disc(f) \notin (F^\times)^2$, $h$ is reducible. Then, $\Gal(K/F) \simeq \mathcal{C}_4$ or $\Gal(K/F) \simeq \mathcal{D}_4$. A method to identify them (which may not be easy) is to check that whether $f(x)$ is reducible over $F(\sqrt{D})$. If so, then $\Gal(K/F) \simeq \mathcal{C}_4$. If not, then $\Gal(K/F) \simeq \mathcal{D}_4$. \\ \\

\textbf{Theorem 43. }Let $F=\mathbb{Q}$ and $f \in \mathbb{Q}[x]$ be an irreducible quartic polynomial (i.e. $\deg f=4$). If $\Gal(K/F) \simeq \mathcal{C}_4$, then $D:=\disc(f)=(\prod_{1 \le i<j \le 4}(\theta_i-\theta_j))^2>0$, where $\theta_1,\theta_2,\theta_3,\theta_4$ are distinct roots of $f$. (It is irreducible and $\chr \mathbb{Q}=0$, so it is separable.) In particular, if $\Gal(K/F) \simeq \mathcal{C}_4$ or $\mathcal{D}_4$ and $D<0$, then $\Gal(K/F) \simeq \mathcal{D}_4$. \\ \\

pf. Suppose $\Gal(K/F) \simeq \mathcal{C}_4$, then $[K:F]=\#\mathcal{C}_4=4$. Since $f$ is irreducible, $[F(\theta_1):F]=4$, which means $K=F(\theta_1)$. Similarly, $K=F(\theta_2)=F(\theta_3)=F(\theta_4)$. This implies that $f$ either has no real roots or has four real roots. If it has four real roots, then $D>0$ is trivial. (Note that $D \ne 0$ since $f$ is separable.) If it has no real roots, since $f \in \mathbb{Q}[x]$, both $\theta_1$ and $\overline{\theta}_1$ are roots of $f$. WLOG, let $\theta_3=\overline{\theta}_1$. Similarly, $\theta_2$, $\overline{\theta}_2=:\theta_4$ are roots of $f$. Therefore:

\begin{align*}
    \delta&:=\prod_{1 \le i < j \le 4}(\theta_i-\theta_j)  \\
    &=(\theta_1-\overline{\theta}_1)(\theta_1-\theta_2)(\theta_1-\overline{\theta_2})(\overline{\theta}_1-\theta_2)(\overline{\theta}_1-\overline{\theta_2})(\theta_2-\overline{\theta}_2) \\
    &=[(\theta_1-\overline{\theta}_1)(\theta_2-\overline{\theta}_2)] \cdot [(\theta_1-\theta_2)(\overline{\theta}_1-\overline{\theta_2})] \cdot[(\theta_1-\overline{\theta_2})(\overline{\theta}_1-\theta_2)]
\end{align*}

Since $\theta_i-\overline{\theta}_i$ is purely imaginary for $i=1,2$, so $(\theta_1-\overline{\theta}_1)(\theta_2-\overline{\theta}_2)$ is real. Since $\overline{(\theta_1-\theta_2)(\overline{\theta}_1-\overline{\theta_2})}=(\theta_1-\theta_2)(\overline{\theta}_1-\overline{\theta_2})$ and $\overline{(\theta_1-\overline{\theta_2})(\overline{\theta}_1-\theta_2)}=(\theta_1-\overline{\theta_2})(\overline{\theta}_1-\theta_2)$, so $(\theta_1-\theta_2)(\overline{\theta}_1-\overline{\theta_2})$ and $(\theta_1-\overline{\theta_2})(\overline{\theta}_1-\theta_2)$ are real. Therefore, $\delta \in \mathbb{R}^\times$, which gives us $D=\delta^2>0$. \\ \\

%\textbf{Remark. }$\Gal(K/F) \simeq \mathcal{C}_4$, then apart from $D>0$, we also have $D=a^2+b^2$ for some $a,b \in \mathbb{Q}$. \\

%pf. ??? \\ \\

\textbf{Example-1. }Let $f(x)=x^4-x-1 \in \mathbb{Q}[x]$, then what is $\Gal(K/F)$ if $F=\mathbb{Q}$ and $K$ is the splitting field for $f(x)$ over $\mathbb{Q}$? \\

Sol. Step 1. We first check that if $f$ is irreducible in $\mathbb{Q}[x]$ or not. By Gauss lemma and $C(f)=1$, it is equivalent to check that if $f$ is irreducible in $\mathbb{Z}[x]$ or not. Consider $\overline{f}_2(x)=x^4-x-1$, if $\overline{f}_2$ is reducible, then it must have an irreducible factor of degree $1$ or $2$. But all irreducible factors of degree $1$ and $2$ in $\mathbb{F}_2$ divide $x^4-x=x^{2^2}-x$, it means that $\overline{f}_2(x)=x^4-x-1$ is irreducible in $\mathbb{F}_2$ and thus $f$ is irreducible in $\mathbb{Z}[x]$. This gives us $4 \mid [K:F]$\\

Step 2. Now, suppose $h(x)$ is the resolvent cubic of $f$, then $\disc(h)=\disc(f)$. By the formula of resolvent cubic ($x^3-cx^2-4ex-(d^2-4ce)$), we have $h(x)=x^3+4x-1$. Then, $\disc(f)=\disc(h)=-4 \cdot 4^3-27 \cdot(-1)^2=-283 \notin (\mathbb{Q}^\times)^2$. Therefore, $\Gal(K/F) \nsubseteq \mathcal{A}_4$. \\

Step 3. Finally, $h(1) \ne 0$ and $h(-1) \ne 0$, which means $h(x)$ has no integer roots. Again by Gauss lemma, $h(x)$ is irreducible in $\mathbb{Q}[x]$. This gives us $3 \mid [K:F]$. Therefore, $4 \mid [K:F]$ and $3 \mid [K:F]$ but $\Gal(K/F) \nsubseteq \mathcal{A}_4$, we have $\Gal(K/F)=\mathcal{S}_4$. \\ \\

\textbf{Example-2. }Let $f(x)=x^4+8x+12 \in \mathbb{Q}[x]$, then what is $\Gal(K/F)$ if $F=\mathbb{Q}$ and $K$ is the splitting field for $f(x)$ over $\mathbb{Q}$? \\

Sol. Step 1. Again. we check that if $f$ is irreducible in $\mathbb{Z}[x]$ or not. Here, we first use simple calculus to deduce that $f(x)$ has no linear factor in $\mathbb{Q}[x]$. Set $f'(x)=4x^3+8=0$ to get $x=-\sqrt[3]{2}$. Since $f''(-\sqrt[3]{2})=12(-\sqrt[3]{2})^2>0$ and $f(-\sqrt[3]{2})=(-\sqrt[3]{2})^4+8(-\sqrt[3]{2})+12=\sqrt[3]{16}-8\sqrt[3]{2}+12 \ge 12-8\sqrt{2}>0$, we have $f(x)>0$ for all real $x$. Therefore, $f$ has no linear factor in $\mathbb{Q}[x]$; if it is reducible, then it must be factored into two irreducible polynomials. \\

Next, consider $\overline{f}_{5}(x)=x^4-2x+2$, we compute that $\overline{f}_{5}(0)=2 \ne 0$, $\overline{f}_{5}(1)=1 \ne 0$, $\overline{f}_{5}(-1)=5=0$, $\overline{f}_{5}(2)=14 \ne 0$, $\overline{f}_{5}(-2)=22 \ne 0$. Therefore, $\overline{f}_{5}(x)=(x+1)g(x)$ for some irreducible cubic polynomial $g(x)$. It tells us $f$ cannot be factored into two irreducible polynomials. Therefore $f$ is irreducible in $\mathbb{Z}[x]$. This gives us $4 \mid [K:F]$. \\

Step 2. Now, suppose $h(x)$ is the resolvent cubic of $f$, then $\disc(h)=\disc(f)$. By the formula of resolvent cubic ($x^3-cx^2-4ex-(d^2-4ce)$), we have $h(x)=x^3-48x-64$. Then, $\disc(f)=\disc(h)=-4 \cdot (-48)^3-27 \cdot(-64)^2=4 \cdot 3^3 \cdot 16^3-3^3 \cdot 16^3=3^4 \cdot 16^3 \in (\mathbb{Q}^\times)^2$. Therefore, $\Gal(K/F) \subseteq \mathcal{A}_4$. \\

Step 3. Finally, we consider $\overline{h}_5(x)=x^3+2x+1$ and compute that $\overline{h}_{5}(n) \ne 0$ for $n=0,\pm1,\pm2$, which means $h(x)$ is irreducible in $\mathbb{Q}[x]$. This gives us $3 \mid [K:F]$. Therefore, $4 \mid [K:F]$, $3 \mid [K:F]$, and $\Gal(K/F) \subseteq \mathcal{A}_4$, we have $\Gal(K/F)=\mathcal{A}_4$.

\subsection*{9.14 Cyclic extensions and Lagrange resolvent}
\indent

\textbf{Def. }Let $K/F$ be an Galois extension, it is said to be a \textbf{cyclic extension} if $\Gal(K/F)$ is cyclic. \\ \\

\textbf{Def. }Let $K/F$ be a cyclic extension of degree $n$ (i.e. $[K:F]=\#\Gal(K/F)=n$). Suppose $F$ contains a primitive $n$-th root of unity $\zeta$. Given $\alpha \in K$ and let $\sigma$ be a generator of $\Gal(K/F)$, we define the \textbf{Lagrange resolvent}, denoted by $(\alpha,\zeta) \in K$, as:

\[
    (\alpha,\zeta):=\alpha+\zeta\sigma(\alpha)+\zeta^2\sigma^2(\alpha)+\cdots+\zeta^{n-1}\sigma^{n-1}(\alpha)
\]\\

\textbf{Prop 44. }Use the settings above, then:

\begin{itemize}
    \item[] (i) $\sigma(\alpha,\zeta)=\zeta^{-1}(\alpha,\zeta)$

    \item[] (ii) $(\alpha,\zeta)^n \in F$, and if $(\alpha,\zeta) \ne 0$, then $(\alpha,\zeta),(\alpha,\zeta)^2,...,(\alpha,\zeta)^{n-1} \notin F$

    \item[] (iii) $(\alpha,\zeta) \notin F'$, for any $F' \subsetneq K$
\end{itemize}

pf-1. Since $\zeta \in F$ by assumption and $\sigma^n=\id$, we can compute that:

\begin{align*}
    \sigma(\alpha,\zeta)&=\sigma(\alpha)+\zeta\sigma^2(\alpha)+\cdots+\zeta^{n-1}\sigma^{n}(\alpha) \\
    &=\zeta^{-1}\alpha+\sigma(\alpha)+\cdots+\zeta^{n-2}\sigma^{n-1}(\alpha) \\
    &=\zeta^{-1}(\alpha,\zeta)
\end{align*}

pf-2. Since $\zeta$ is a primitive $n$-th root of unity, $\sigma^n(\alpha,\zeta)=\zeta^{-n}(\alpha,\zeta)=(\alpha,\zeta)$ and $\sigma^i(\alpha,\zeta) =\zeta^{-1}(\alpha,\zeta)\ne (\alpha,\zeta)$ for all $1 \le i \le n-1$, if $(\alpha,\zeta) \ne 0$. Therefore, $\sigma[(\alpha,\zeta)^n]=[\sigma(\alpha,\zeta)]^n=(\alpha,\zeta)^n$, i.e. $(\alpha,\zeta)^n$ is fixed by $\sigma$, and $(\alpha,\zeta)^i$ is not fixed by $\sigma$ for all $1 \le i \le n-1$. It implies that if $(\alpha,\zeta)\ne 0$, then $(\alpha,\zeta),(\alpha,\zeta)^2,...,(\alpha,\zeta)^{n-1} \notin F$ and $(\alpha,\zeta)^n \in F$. \\

pf-3. Suppose $(\alpha,\zeta) \in F' \subsetneq K$, then $H:=\Gal(K/F') \supsetneq \Gal(K/K)=\{\id\}$. Since $\Gal(K/F)=\langle \sigma \rangle $ and $H$ is its subgroup, $H=\langle \sigma^i \rangle$ for some $i$. But $\sigma^i(\alpha,\zeta) \ne (\alpha,\zeta)$ if $i<n$ by (ii), this tells us $i=n$ and thus $H=\langle \sigma^n \rangle=\langle \id \rangle=\{\id\}$, which is a contradiction. \\ \\

\textbf{Prop 45. }Any cyclic extension $K$ of degree $n$ over $F$($\chr F \nmid n$) which contains a primitive $n$-th root of unity is of the form $F(\sqrt[n]{a})$, for some $a \in F^\times$. \\

pf. Given $\alpha \in K$ and $\sigma$ be a generator of $\Gal(K/F)$. Pick $a=(\alpha,\zeta)^n$ s.t. $a \ne 0$, then $K \supseteq F(\sqrt[n]{a}) \supseteq F$ by the definition of $\sqrt[n]{a}=(\alpha,\zeta)$. But by proposition 44-(iii), $F(\sqrt[n]{\alpha})$ can only be $K$. Therefore, $K=F(\sqrt[n]{a})$. \\ \\

\textbf{Theorem 46. }Let $F$ be a field of characteristic $0$, $f \in F[x] \setminus \{0\}$, $K$ be the splitting field for $f(x)$ over $F$. Then:

\[
    f(x)\text{ is solvable by radicals} \Leftrightarrow \Gal(K/F) \text{ is solvable}
\]

pf. The $(\Rightarrow)$ direction is exactly theorem 32, we now show $(\Leftarrow)$, our goal is to \textbf{find a radical extension} $L/F$ s.t. $K \subseteq L$. Denote $G:=\Gal(K/F)$, then since $G$ is solvable, there exists a sequence:

\[
    \{e\} = G_0 \subsetneq G_1 \subsetneq \cdots \subsetneq G_{r-1} \subsetneq G_r=G
\]

s.t. $G_i \trianglelefteq G_{i+1}$ and $G_{i+1}/G_i$ is cyclic, for $i=0,1,...,r-1$. Suppose the Galois correspondence is:

\[
    K=F_0 \supsetneq F_1 \supsetneq \cdots \supsetneq F_{r-1} \supsetneq F_r=F
\]

That is:

\[
    \begin{matrix}
        \{e\} &=& G_0 &\subsetneq& G_1 &\subsetneq& \cdots &\subsetneq& G_{r-1} &\subsetneq& G_r&=&G \\
        \updownarrow && \updownarrow && \updownarrow &&  && \updownarrow && \updownarrow && \updownarrow &&\\
        K&=&F_0 &\supsetneq& F_1 &\supsetneq& \cdots &\supsetneq& F_{r-1} &\supsetneq& F_r&=&F
    \end{matrix}
\]

For any $i=1,2,...,r$, since $\Gal(K/F_{i-1})=G_{i-1} \trianglelefteq G_i=\Gal(K/F_i)$, we have $\Gal(F_{i-1}/F_i)=G_i/G_{i-1}$, which is cyclic by assumption, say $\#\Gal(F_{i-1}/F_i)=d_{i-1}$. Now, let $F'/F$ be the cyclotomic field contains all primitive all $d_{i-1}$-th root of unity, $i=1,2,...,r$. (Say, $F'$ is the splitting field for $x^{d_0d_1...d_{r-1}}-1$ over $F$.) In particular, $F'/F$ is Galois since $\chr F=0$ and $F'$ is a splitting field. \\

Let $KF'=\{\sum_{i=1}^n\alpha_i\beta_i:n \in \mathbb{Z}_{>0},\ \alpha_i \in K,\ \beta_i \in F'\}$, it's clear that $K \subseteq KF'$ and $F' \subseteq KF'$. \textbf{Goal: $KF'/F$ is radical}, then since $K \subseteq KF'$, we have $f$ can be solved by radicals. We take the intersection of the sequence of fields above and $F'$, we have:

\[
    KF'=F_0F' \supseteq F_1F' \supseteq \cdots \supseteq F_{r-1}F' \supseteq F_rF'=F'
\]

Note that the inclusion might not be strict; for example, $K=\mathbb{Q}(\sqrt{2},i) \supsetneq \mathbb{Q}(\sqrt{2})=F$ and $F'=F(i)$. The first thing we want to show is that \textbf{each $F_{i-1}F'/F_iF'$ is still cyclic}. Consider the map $res$ defined by:

\[
    \begin{matrix}
        res: \Gal(F_{i-1}F'/F_{i}F') &\to& \Gal(F_{i-1}/F_i) \\
    \sigma &\mapsto& \sigma\vert_{F_{i-1}}
    \end{matrix}
\]

We should justify that this map is well-defined. Given any $\sigma \in \Gal(F_{i-1}F'/F_iF')$. First, $\sigma$ fixes elements in $F_iF' \supseteq F_i$, so $\sigma$. fixes elements in $F_{i}$. Moreover, if we see $\sigma$ as a homomorphism from $F_{i-1}$ to $F_{i-1}F'$, since $F_{i-1} \subseteq F_{i-1}F'$ and $\sigma$ fixes elements in $F_i$, by theorem 20-(vi), we have $\sigma(F_{i-1})=F_{i-1}$. Therefore, $\sigma\vert_{F_{i-1}}$ is well-defined. Since $\Gal(F_{i-1}/F_i)$ is cyclic by assumption, if we can show that $res$ is injective, then $\Gal(F_{i-1}F'/F_iF')$ is also cyclic. \\

Since $F_{i-1}/F_i$ is Galois, we let $h_i(x) \in F_i[x]$ be a separable polynomial s.t. $F_{i-1}$ is the splitting field for $h_i(x)$ over $F_i$. Let the roots of $h_i$ be $\gamma_1,...,\gamma_{\ell_i}$, then $F_{i-1}=F_i(\gamma_1,...,\gamma_{\ell_i})$. This gives us $F_{i-1}F'=F_{i}F'(\gamma_1,...,\gamma_{\ell_i})$. Given $\sigma \in \ker(res)$, i.e. $\sigma(x)=x$ for all $x \in F_{i-1}$. This gives us $\sigma(\gamma_{j})=\gamma_j$ for all $j=1,2,...,\ell_i$. Along with the fact that $\sigma$ fixes the elements of $F_iF'$, it tells us $\sigma$ fixes all elements in $F_iF'(\gamma_1,...,\gamma_{\ell_i})=F_{i-1}F'$, i.e. $\sigma$ is an identity map on $F_{i-1}$. Hence, $F_{i-1}F'/F_iF'$ is cyclic since it can be regarded as a subgroup of $F_{i-1}/F_i$. \\

So far, we have proved that $F_{i-1}F'/F_iF'$ is a cyclic extension for $i=1,2,...,r$. Since we have assumed that $F'$ contains a primitive $d_{i-1}$-th root of unity, by proposition 45, we have $F_{i-1}F'=F_iF'(\sqrt[n_i]{a})$ for some $a_i \in (F_{i+1}F')^\times$, where $n_i=d_{i-1}$. Therefore, $F_{i-1}F'/F_iF'$ is Kummer for $i=1,2,...,r$. It remains to show that $F'/F$ can also be decomposed into a sequence of Kummer extensions. But this is trivial since $F'/F$ is a cyclotomic extension, so it is also a Kummer extension by definition. In conclusion, $KF'=F_0F' \supseteq F_1F' \supseteq \cdots \supseteq F_{r-1}F' \supseteq F_rF'=F' \supseteq F$ is a sequence that satisfies the condition of radical extensions. Since $K \subseteq KF'$ and $K$ is a splitting field for $f$ over $F$, we conclude that $f$ can be solved by radicals. \\ \\

\textbf{Example. }Let $f(x)=x^3+px+q \in F[x]$, $\chr F=0$. We know that $\mathcal{S}_3$ and its subgroups are solvable, so by theorem 46, $f(x)$ can be solved by radicals. \\

WLOG, $f$ is irreducible, so $\Gal(K/F)=\mathcal{A}_3$ or $\mathcal{S}_3$. Let $K$ be a splitting field for $f$ over $F$ and $\theta_1,\theta_2,\theta_3$ be roots of $f$. Let $\sigma=(1\ 2\ 3)$, then it is a generator of $\Gal(K/F(\sqrt{D}))=\mathcal{A}_3$, where $D=(\prod_{i=1}^3(\theta_i-\theta_j))^2=-4p^3-27q^2$. If $\zeta$ is a primitive $3$-rd roots of unity, then $\zeta^2+\zeta+1=0$, and we can compute the Lagrange resolvent:

\[
    (\theta_1,\zeta)=\theta_1+\zeta\sigma(\theta_1)+\zeta^2\sigma^2(\theta_1)=\theta_1+\zeta\theta_2+\zeta^2\theta_3
\]

Then, the $K=F(\sqrt[3]{a})$, where $a=(\theta_1,\zeta)^3$. By complicated calculation, we have $a=-\frac{27}{2}+\frac{3}{2}\sqrt{-3D}$.

\begin{comment}
    Now, we use:

\[
    \left\{\begin{matrix}
        \theta_1+\theta_2+\theta_3&=&0 \\
        \theta_1\theta_2+\theta_1\theta_3+\theta_2\theta_3&=&p \\
        \theta_1\theta_2\theta_3&=&-q \\
        (\theta_1-\theta_2)(\theta_1-\theta_3)(\theta_2-\theta_3)&=&\sqrt{D}
    \end{matrix}\right.
\]

to compute $a=(\theta_1,\zeta)^3=(\theta_1+\zeta\theta_2+\zeta^2\theta_3)^3$. Compute:

\begin{align*}
    (\theta_1+\zeta\theta_2+\zeta^2\theta_3)^3&=(\theta_1^3+\theta^3_2+\theta^3_3)+3[\theta_1^2(\zeta\theta_2 +\zeta^2\theta_3)+\zeta^2\theta_2^2(\theta_1+\zeta^2\theta_3)+\zeta\theta^2_3(\theta_1+\zeta\theta_2)]+6(\theta_1 \cdot \zeta\theta_2\cdot\zeta^2\theta_3) \\
    &=(\theta_1^3+\theta^3_2+\theta^3_3)+3[\zeta(\theta_1^2\theta_2+\theta_2^2\theta_3+\theta_1\theta_3^2)+\zeta^2(\theta_1^2\theta_3+\theta_1\theta_2^2+\theta_2\theta_3^2)]+6\theta_1\theta_2\theta_3
\end{align*}



For $(\theta_1^3+\theta_2^3+\theta_3^3)$, we compute:

\begin{align*}
    \theta_1^3+\theta_2^3+\theta_3^3&=(\theta_1+\theta_2+\theta_3)^3-3\theta_1\theta_2\theta_3(\theta_1+\theta_2+\theta_3)+
\end{align*}
\end{comment}


\end{document}