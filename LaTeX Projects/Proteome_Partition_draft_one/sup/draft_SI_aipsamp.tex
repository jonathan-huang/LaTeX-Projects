% ****** Start of file aipsamp.tex ******
%
%   This file is part of the AIP files in the AIP distribution for REVTeX 4.
%   Version 4.2a of REVTeX, December 2014
%
%   Copyright (c) 2014 American Institute of Physics.
%
%   See the AIP README file for restrictions and more information.
%
% TeX'ing this file requires that you have AMS-LaTeX 2.0 installed
% as well as the rest of the prerequisites for REVTeX 4.2
%
% It also requires running BibTeX. The commands are as follows:
%
%  1)  latex  aipsamp
%  2)  bibtex aipsamp
%  3)  latex  aipsamp
%  4)  latex  aipsamp
%
% Use this file as a source of example code for your aip document.
% Use the file aiptemplate.tex as a template for your document.
\documentclass[%
 aip,
 jmp,%
 amsmath,amssymb,
%preprint,%
 reprint,%
%author-year,%
%author-numerical,%
]{revtex4-2}

\usepackage{graphicx}% Include figure files
\usepackage{dcolumn}% Align table columns on decimal point
\usepackage{bm}% bold math
\usepackage{tabularx}
\usepackage{booktabs}
\usepackage{amsthm}
\usepackage[shortlabels]{enumitem}
%\usepackage[mathlines]{lineno}% Enable numbering of text and display math
%\linenumbers\relax % Commence numbering lines

\theoremstyle{plain}
\newtheorem{theorem}{Theorem}[section]
\newtheorem{lemma}[theorem]{Lemma}
\newtheorem{proposition}[theorem]{Proposition}
\newtheorem{corollary}[theorem]{Corollary}

\usepackage{thmtools}
\usepackage{mdframed}
\usepackage[dvipsnames]{xcolor} % different colors

\declaretheoremstyle[
    headfont=\sffamily\bfseries\color{black}, % Sans-serif, bold, blue title
    bodyfont=\normalfont,
    headpunct={.}, % Adds a period after the theorem heading
    postheadspace=1em,
]{mydefstyle}

\declaretheorem[
    style=mydefstyle,
    name=Definition,
    numberwithin=section % <-- THIS makes it Definition 1.1, 1.2, etc.
]{definition}

\declaretheorem[
    style=mydefstyle,
    name=Example,
    numberwithin=section % <-- THIS makes it Definition 1.1, 1.2, etc.
]{example}

\theoremstyle{remark}
\newtheorem*{remark}{Remark}

\newcommand{\bvec}[1]{\mathbf{#1}} % vector
\newcommand{\mycomment}[1]{} % block comments

\begin{document}

\preprint{AIP/123-QED}

\title[Sample title]{TBD Supplementary Information}
\thanks{Footnote to title of article.}

\author{Shao-Kai Jonathan Huang}
%  \altaffiliation[Also at ]{Physics Department, XYZ University.}
\email{huang20041014@gmail.com}
\affiliation{ 
Institute of Molecular Biology, Academia Sinica, Taipei 115, Taiwan
}%
\affiliation{ 
Department of Physics, National Taiwan University, Taipei 106, Taiwan
}%

\author{Wei-Hsiang Lin}
\affiliation{ 
Institute of Molecular Biology, Academia Sinica, Taipei 115, Taiwan
}%

\date{\today}

\begin{abstract}
    This is the supplementary information for TBD.
\end{abstract}

\keywords{Suggested keywords}%Use showkeys class option if keyword
                              %display desired
\maketitle
\tableofcontents

\newpage

% Growth Laws
\section{Growth Laws and Bacterial Physiology}

% Model
\section{\label{sec:model}Model}
\subsection{Theory of Reaction Networks}

Reaction-network theory provides a natural and rigorous language for describing cellular growth by treating the cell as a collection of interacting biochemical processes. By representing distinct intermediates as nodes in a network and their biomass as values associated with the nodes, and their interconversions as fluxes with well-defined stoichiometry and kinetics, we can connect coarse-grained quantities such as growth rate and directly to the underlying structure of the system. This framework makes it possible to ask questions such as when exponential growth emerge, how they depend on network topology and kinetic parameters, and which features of “growth laws” are generic consequences of certain fluxes. 

We now introduce the standard notation used in the article. Let $ \mathcal{S} := \{1,2,\dots,m\} $ denote the index set of chemical species, and $ \mathcal{R} := \{1,2,\dots,r\} $ denote the index set of reactions. We write $X = (X_1,\dots,X_m)^\top \in \mathbb{R}_{\ge 0}^m$ for the vector of species concentrations (or, more generally in our context, biomass contents), where $X_i$ is the amount of species $i \in \mathcal{S}$.

Each reaction $j \in \mathcal{R}$ is specified by a pair of \emph{complexes}
\[
    y^{(j)} \longrightarrow y'^{(j)},
\]
where $y^{(j)}, y'^{(j)} \in \mathbb{R}_{\ge 0}^m$ encode the stoichiometric coefficients of reactants and products, respectively. The \emph{reaction vector} (or \emph{stoichiometric vector}) of reaction $j$ is defined by
\[
    s^{(j)} := y'^{(j)} - y^{(j)} \in \mathbb{R}^m.
\]
The s\emph{stoichiometric matrix} $S \in \mathbb{R}^{m \times r}$ can then be written column-wise as
\[
    S = \bigl(s^{(1)} \; s^{(2)} \; \cdots \; s^{(r)}\bigr),
\]
so that the $j$-th column $S_{\cdot j}$ is precisely the reaction vector $s^{(j)}$. Each entry $ S_{ij} $ represents the net change in the amount of species $ i $ due to reaction $ j $. The dynamics can be described by the system of ordinary differential equations (ODEs):
\begin{equation}
    \label{eq:reaction_network_ode}
    \frac{\mathrm{d} X}{\mathrm{d} t} = S J(X),
\end{equation}
where $ X \in \mathbb{R}^m $ is the vector of species concentrations, and $ J(X) \in \mathbb{R}^r $ is the vector of \emph{reaction fluxes} or \emph{reaction rates} (sometimes also denoted $ v $), which depend on the concentrations $ X $. Since biomass cannot admit negative values, the \emph{state space} of the reaction network is the nonnegative orthant $\mathbb{R}_{\ge 0}^m$. In the state space, the system of ODEs \ref{eq:reaction_network_ode} defines a vector field $F : \mathbb{R}_{\ge 0}^m \to \mathbb{R}^m$ given by 
\begin{equation}
    F(X) := S J(X).
\end{equation}
The components $J_j(X)$ of the flux vector $J(X) = (J_1(X),\dots,J_r(X))^\top$ are nonnegative functions
\[
    J_j : \mathbb{R}_{\ge 0}^m \to \mathbb{R}_{\ge 0}, \qquad j \in \mathcal{R},
\]
whose specific form is determined by the choice of kinetics. In our study, we consider the two most biologically relevant fluxes: Michaelis-Menten type flux and mass action flux. Under \emph{mass-action kinetics}, the rate of reaction $j$ is given by
\[
    J_j(X) = k_j X^{y^{(j)}} := k_j \prod_{i=1}^m X_i^{y^{(j)}_i},
\]
where $k_j > 0$ is the rate constant and $y^{(j)}_i$ is the stoichiometric coefficient of species $i$ in the reactant complex $y^{(j)}$. Under \emph{Michaelis--Menten kinetics} or other saturating kinetics, $J_j(X)$ typically has the form 
\begin{equation}
    J_j(X) = \frac{V_{\max,j} X^{y^{(j)}}}{K_{M,j} + X^{y^{(j)}}},
\end{equation}
where $V_{\max,j} > 0$ is the maximum reaction rate and $K_{M,j} > 0$ is the Michaelis constant. In all cases the notation $J(X)$ denotes the vector of reaction rates associated with the reactions indexed by $\mathcal{R}$.

When the fluxes $J_j(X)$ are linear functions of $X$, the system is called a \emph{linear reaction network (LRN)}. A particularly important case is when there exists a constant matrix $M \in \mathbb{R}^{m \times m}$ such that $ \dot X = M X $, in which case $M$ is sometimes called the \emph{effective stoichiometric matrix} and the long-time behavior of $X(t)$ is determined by the spectral properties of $M$. This is the \emph{Euclidean view} for LRNs, where by the Perron-Frobenius theorem, the growth rate of the system may be read off directly as the top eigenvalue of $M$ when $M$ is principally nonnegative.

However, most biological networks are highly nonlinear, so LRNs are only a first approximation. Following \cite{Lin2020}, we say that a reaction network is a \emph{scalable reaction network (SRN)}, if the vector field $F(X) = S J(X)$ is homogeneous of degree one, i.e.
\begin{equation}
    F(\alpha X) = \alpha F(X) \quad \text{for all } \alpha > 0, \; X \in \mathbb{R}_{\ge 0}^m.
\end{equation}

As opposed to the Eulerian view for reaction networks, where dynamics are described in terms of concentrations $X$, we may also adopt a \emph{Lagrangian view}, where the system is described in terms of total biomass $N$ and biomass fractions $Y$. This is particularly useful for SRNs, where the dynamics of $N$ and $Y$ can be decoupled. In the Lagrangian view, machinery from the theory of ordinary differential equations, such as waiting time analysis, will be used to analyze the system dynamics.

\subsection{Reaction Network Model of Proteome Partition}

\begin{figure}[htbp]
    \centering
    \begin{minipage}{0.58\textwidth}
        \centering
        \includegraphics[width=\textwidth]{Images/three_sector_model.png}
    \end{minipage}
    \hfill
    \begin{minipage}{0.38\textwidth}
        \centering
        \includegraphics[width=\textwidth]{Images/partition_pie_chart.png}
    \end{minipage}
    \caption{(a) Schematic diagram of the three-sector proteome partition model. The proteome sectors $ P, Q, R $ are the transporters, house-keeping enzymes for metabolic conversion, and ribosomal proteins, respectively. (b) An illustration of proteome partitioning into three sectors $ P, Q, R $ with respective translation fractions $ \theta_1, \theta_2, \theta_3 $.}
    \label{fig:proteome_partition_model}
\end{figure}

The proteome sectors $P$, $Q$, $R$ are the transporters, metabolic enzymes, and ribosomal proteins, respectively. We label the biomass nodes associated with $ S, U, P, Q, R $ as $ x_1, x_2, x_3, x_4. x_5 $, respectively, and represent their respective biomass content with capital letters $ X_1, X_2, X_3, X_4, X_5 \in \mathbb{R}_{\geq 0}$. The total biomass is the positive quantity $ N = X_1 + X_2 + X_3 + X_4 + X_5 $.

The fluxes $ J_k = J[\phi_k] $ associated with each reaction pathway $ \phi_k $ are given by 
\begin{subequations}
    \begin{align}
        J_1(X) &= J[\phi_1] = b X_3, \\
        J_2(X) &= J[\phi_2] = \left(\frac{a_1 X_4}{ k_1 N + X_1}\right) X_1 \equiv R_1 (X) X_1, \\
        J_3(X) &= J[\phi_3] = \theta_1 J_{\text{syn}}, \\
        J_4(X) &= J[\phi_4] = \theta_2 J_{\text{syn}}, \\
        J_5(X) &= J[\phi_5] = \theta_3 J_{\text{syn}}, \\
        J_{\text{syn}} &= \left(\frac{a_2 X_5}{k_2 N + X_2}\right) X_2 \equiv R_2(X) X_2.
    \end{align}
\end{subequations} 

Or (putting the translation fraction $ \theta_i $ into the stoichiometric matrix instead):
\begin{subequations}
    \begin{align}
        J_1(X) &= J[\phi_1] = b X_3, \\
        J_2(X) &= J[\phi_2] = \left(\frac{a_1 X_4}{ k_1 N + X_1}\right) X_1 \equiv R_1 (X) X_1, \\
        J_k (X) &= J[\phi_k] = \theta_{k-2} \left(\frac{a_2 X_5}{k_2 N + X_2}\right) X_2 \equiv \theta_{k-2} R_2(X) X_2, 
    \end{align}
\end{subequations} 
where $ k=3,4,5 $. Here, we have defined the nonlinear rate $ R_j(X) $ based on \emph{Michaelis-Menten kinetics}, the standard model for enzyme kinetics. The quantities $ P_1 $, $ P_2 $, and $ P_3 $, defined by  
\begin{equation}
    P_1 = \frac{X_3}{X_3 + X_4 + X_5}, \quad P_2 = \frac{X_4}{X_3 + X_4 + X_5}, \quad P_3 = \frac{X_5}{X_3 + X_4 + X_5},
\end{equation}
are the proteome fractions. Each $ \theta_i $ is called the \emph{translation fraction}, since it determines what fraction of metabolites are translated into each proteome sector. Since the system has \emph{no degradation}, the proteome fractions are exactly equal to the translation fractions, and we may use both terms interchangeably. Hence, we call the unit vector $ \theta = (\theta_1, \theta_2, \theta_3) \in \Delta^2 $ the \emph{proteome partition}, which is in fact equivalent to $ (P_1, P_2, P_3) $, as defined earlier. The stoichiometric matrix of the system is given by
\begin{equation}
    S = 
    \begin{pmatrix}
        +1 & -1 & 0 & 0 & 0 \\
        0 & +1 & -1 & 0 & 0 \\
        0 & 0 & +1 & 0 & 0 \\
        0 & 0 & 0 & +1 & 0 \\
        0 & 0 & 0 & 0 & +1 
    \end{pmatrix}.
\end{equation}

The system of ordinary differential equations governing the dynamical system is therefore 
\begin{subequations}
    \begin{align}
        \frac{\mathrm{d} X_1}{\mathrm{d} t} &= b X_3 - \frac{a_1 X_1}{ k_1 N + X_1 X_4},\\
        \frac{\mathrm{d} X_2}{\mathrm{d} t} &= \frac{a_1 X_1 X_4}{ k_1 N + X_1} - \frac{a_2 X_2 X_5}{ k_2 N + X_2},\\
        \frac{\mathrm{d} X_3}{\mathrm{d} t} &= \theta_1 \left(\frac{ a_2 X_2 X_5}{ k_2 N + X_2}\right), \\
        \frac{\mathrm{d} X_4}{\mathrm{d} t} &= \theta_2 \left(\frac{ a_2 X_2 X_5}{ k_2 N + X_2}\right), \\
        \frac{\mathrm{d} X_5}{\mathrm{d} t} &= \theta_3 \left(\frac{ a_2 X_2 X_5}{ k_2 N + X_2}\right),
    \end{align}
\end{subequations}
which may be compactly written by defining the biomass vector $ X = (X_1, X_2, X_3, X_4, X_5) $ and a function $ F : \mathbb{R}^5_{\geq 0} \to \mathbb{R}^5 $ such that 
\begin{equation}
    \frac{\mathrm{d} X}{\mathrm{d} t} \equiv F(X),
\end{equation}

Since a bacteria is in a steady state of concentration, or, biomass fraction, during exponential growth, it is convenient to work in the $ (N,Y) $ coordinates. Define \emph{biomass fraction} to be $ Y = X/N $, and define $ \mu: \mathbb{R}^5_{\geq 0} \to \mathbb{R} $ to be the \emph{instantaneous growth rate}, i.e., 
\begin{equation}
    \frac{\mathrm{d} N}{\mathrm{d} t} = \mu (Y) N, 
\end{equation}
Then, since our system is scalable, the original ODE system may be rewritten as
\begin{subequations}
    \begin{align}
        \frac{\mathrm{d} Y}{\mathrm{d} t} &= \frac{1}{N} \frac{\mathrm{d} X}{\mathrm{d} t} - \frac{X}{N^2} \frac{\mathrm{d} N}{\mathrm{d} t} = F(Y) - \mu (Y) Y, \\
        \mu(Y) &= \sum^{5}_{k=1} F_k(Y) = b Y_3,
    \end{align}
\end{subequations}

Then the ODE of interest becomes 
\begin{subequations}
    \label{equ:Y}
    \begin{align}
        \frac{\mathrm{d} Y_1}{\mathrm{d} t} &= b Y_3 - \frac{a_1 Y_1 Y_4}{ k_1 + Y_1} - b Y_3 Y_1 ,\\
        \frac{\mathrm{d} Y_2}{\mathrm{d} t} &= \frac{a_1 Y_1 Y_4}{ k_1 + Y_1} - \frac{a_2 Y_2 Y_5}{ k_2 + Y_2} - b Y_3 Y_2 ,\\
        \frac{\mathrm{d} Y_3}{\mathrm{d} t} &= \theta_1 \left(\frac{ a_2 Y_2 Y_5}{ k_2 + Y_2}\right) - b Y_3^2, \\
        \frac{\mathrm{d} Y_4}{\mathrm{d} t} &= \theta_2 \left(\frac{ a_2 Y_2 Y_5}{ k_2 + Y_2}\right) - b Y_3 Y_4, \\
        \frac{\mathrm{d} Y_5}{\mathrm{d} t} &= \theta_3 \left(\frac{ a_2 Y_2 Y_5}{ k_2 + Y_2}\right) - b Y_3 Y_5,
    \end{align}
\end{subequations}

For the above ODE system, steady state occurs when the bacterium goes into exponential growth phase, in which case $ N(t) \sim e^{\lambda t} $ for some real number $ \lambda > 0 $ after a long time. More generally, the exponent may be negative, zero, positive, or complex, corresponding to exponential decay, stationary phase, exponential growth, and oscillatory behavior, respectively. Therefore, we can define the \emph{growth rate} of the system to be the quantity 
\begin{equation}
    \lambda = \lim_{t \to \infty} \frac{1}{t} \log \frac{N(t)}{N(0)}, 
\end{equation}
where the base $ N(0) $ is usually assumed to be $ 1 $. It is established in \cite{Lin2020} that an SRN has a unique, well-defined growth rate $ \lambda $. Enzymes inside cells are evolved to operate under Michaelis-Menten kinetics under normal nutrient concentrations, i.e. equipped with low values of $ k_i $. However, in cases when protein concentration is extremely scarce, such that $ Y_i \ll k_i $, the kinetics may be approximated by \emph{mass action kinetics}. In this case, the fluxes becom 
\begin{equation}
    R_1 (Y) = \frac{a_1 Y_4}{k_1} \equiv r_1 Y_4, \quad R_2 (Y) = \frac{a_2 Y_5}{k_2} \equiv r_2 Y_5. 
\end{equation}
This scenario will also be analyzed in later sections, where sometimes the mass-action approximation yields cleaner analytic solutions while nevertheless retaining similar qualitative behavior.

\subsection{Generalized Model}
In later sections, we will consider a natural generalization of the above model to include more proteome sectors and biomass nodes, as well as different metabolism kinetics governing the flow of biomass. Figure~\ref{fig:general_proteome_partition_model} shows a schematic diagram of the generalized $ n $-sector proteome partition model, which has in total $ 2n-1 $ nodes, including $ 1 $ transporter node, $ 1 $ ribosomal node, and $ n-2 $ metabolic nodes, each associated with a proteome sector, and a proteome partition $ \theta = (\theta_1, \theta_2, \dots, \theta_n) $. The metabolic nodes $ x_j $ with $ 1 \leq j \leq n-1 $ are called \emph{non-terminal nodes}, while the protein nodes $ x_j $ with $ n \leq j \leq 2n-1 $ are called the \emph{terminal nodes}.

\begin{figure}
    \includegraphics[width=0.9\textwidth]{Images/general_model.png}%
    \caption{Schematic diagram of the generalized $ n $-sector proteome partition model. The biomass nodes $ x_i $ for $ n \leq i \leq 2n-1 $ are proteome sectors that correspond to the transporter proteins (in red, $ i=n $), house-keeping enzymes for metabolism (in orange, $ n+1 \leq i \leq 2n-2 $), and ribosomal proteins (in yellow, $ i = 2n-1 $). The right panel shows a typical distribution of biomass fractions when nutrient availability is low.}
    \label{fig:general_proteome_partition_model}
\end{figure}

% Two-Sector Model
\section{Two-Sector Model with Mass Action Kinetics}

The two-sector model under mass action kinetics has an analytic solution.

\subsection{Optimal Partition Strength}
The two-sector model with MA kinetics has a unique steady state solution given by 
\begin{subequations}
    \begin{align}
        Y_1 &= \left(1 + \frac{r \theta_2}{b \theta_1}\right)^{-1}, \\
        Y_2 &= \frac{r \theta_2}{b} Y_1 = \frac{r \theta_2}{b} \left(1 + \frac{r \theta_2}{b \theta_1}\right)^{-1}, \\
        Y_3 &= \frac{r \theta_2^2}{b \theta_1} Y_1 = \frac{r \theta_2^2}{b \theta_1} \left(1 + \frac{r \theta_2}{b \theta_1}\right)^{-1}.
    \end{align}
\end{subequations}

The Lagrangian as a function of $ \theta = (\theta_1, \theta_2) $ and $ \mu  $ has a simple form given by 
\begin{equation}
    L(\theta, \mu) = \frac{r \theta_2}{1 + \dfrac{r \theta_2}{b \theta_1}} - \mu (\theta_1 + \theta_2 - 1).
\end{equation}

If we carry out the method of Lagrange multipliers as before, we obtain the optimality condition
\begin{subequations}
    \begin{align}
        \frac{\partial L}{\partial \theta_1} = \frac{1}{b} \left(\frac{r \theta_2}{\theta_1}\right)^2 \left(1 + \frac{r \theta_2}{b \theta_1}\right)^{-2} - \mu &= 0, \\
        \frac{\partial L}{\partial \theta_2} = r \left(1 + \frac{r \theta_2}{b \theta_1}\right)^{-2} - \mu &= 0.
    \end{align}
\end{subequations}

Solving the equations simultaneously, we obtain the optimal solution
\begin{equation}
    \frac{1}{b} \left(\frac{r \theta_2}{\theta_1}\right)^2 \left(1 + \frac{r \theta_2}{b \theta_1}\right)^{-2} = r \left(1 + \frac{r \theta_2}{b \theta_1}\right)^{-2} \;\Longrightarrow\; \frac{\theta_2}{\theta_1} = \sqrt{\frac{b}{r}} \quad \text{for all } b > 0.
\end{equation}
Therefore, $ \theta_1 = (1 + \theta_2 / \theta_1)^{-1} $ and $ \theta_2 = 1 - \theta_1 $ gives 
\begin{equation}
    \theta_1^* = \frac{1}{1 + \sqrt{b/r}}, \quad \theta_2^* = \frac{\sqrt{b/r}}{1 + \sqrt{b/r}}, \quad \text{for all } b > 0.
\end{equation}

\subsection{Biomass Fractions Under Optimal Condition}
Using the optimal partition strengths derived above, the biomass fractions at optimality have a closed form given by 
\begin{subequations}
    \begin{align}
        Y_1 &= \frac{\sqrt{b/r}}{1 + \sqrt{b/r}}, \quad Y_2 = \frac{1}{\left(1 + \sqrt{b/r}\right)^2}, \quad Y_3 = \frac{\sqrt{b/r}}{\left(1 + \sqrt{b/r}\right)^2}, \\
        \text{Small} &= Y_1 = \frac{\sqrt{b/r}}{1 + \sqrt{b/r}}, \quad \text{Large} = Y_2 + Y_3 = \frac{1}{1 + \sqrt{b/r}}.
    \end{align}
\end{subequations}

The relationship between growth rate $ \lambda $ and optimal "ribosome biomass fraction" $ Y_3^\ast $ is then found to be exactly
\begin{subequations}
    \begin{align}
        \lambda &= b Y_2 = \frac{b}{\left(1+\sqrt{b/r}\right)^2}, \\
        Y_3^* &= Y_3^* (\lambda) = \sqrt{\frac{\lambda}{r}} - \frac{\lambda}{r}.
    \end{align}
\end{subequations}
Here "ribosomal protein biomass fraction" $ Y_3^* $ has a nice closed-form relationship with $ \lambda $, but not the other way round since $ \lambda $ is not a function of $ Y_3^* $, as shown in Figure~\ref{fig:MA2_lambda_Y3}. Notice that there is clearly a maximum ribosomal biomass fraction which does not correspond to the maximum growth rate. In Figure~\ref{fig:MA2_lambda_b}, we show the relationships between growth rate and biomass fractions for a large range of nutrient quality $ b $.

\begin{figure}[htbp]
    \centering
    \begin{minipage}{0.48\textwidth}
        \centering
        \includegraphics[width=\textwidth]{Images/two_MA_lambda_b.png}
        \caption{Growth rate $ \lambda $ as a function of nutrient quality $ b $.}
        \label{fig:MA2_lambda_b}
    \end{minipage}
    \hfill
    \begin{minipage}{0.48\textwidth}
        \centering
        \includegraphics[width=\textwidth]{Images/two_MA_lambda_Y3.png}
        \caption{Optimal "ribosomal protein biomass fraction" $ Y_3^* $ as a function of growth rate $ \lambda $.}
        \label{fig:MA2_lambda_Y3}
    \end{minipage}
\end{figure}

In the case of low nutrient quality $ b \to 0 $, there is a linear law between growth rate and ribosomal biomass fraction:

\subsection{No Meaningful Closed Form for Michaelis-Menten Kinetics}
Following the calculations for Michaelis-Menten kinetics, we try to calculate the derivative of $ L $ after substituting $ \theta_2 = 1 - \theta_1 $. This gives 
\begin{equation}
    L(\theta_1, \mu) = b Y_2 = a (1-\theta_1) \left[ \dfrac{- \left(\frac{a}{b}\frac{1-\theta_1}{\theta_1} + k - 1\right) + \sqrt{\left(\frac{a}{b}\frac{1-\theta_1}{\theta_1} + k - 1\right)^2 + 4k}}{-\left(\frac{a}{b}\frac{1-\theta_1}{\theta_1} - k - 1\right) + \sqrt{\left(\frac{a}{b}\frac{1-\theta_1}{\theta_1} + k - 1\right)^2 + 4k} } \right].
\end{equation}

Carrying out the differentiation using WolframAlpha differentiator and doing some simplification, we get 
\begin{equation}
    \begin{split}
        \frac{2ak}{b}\frac{1-\theta_1}{\theta_1^2} &= \left(- \frac{a}{b} \frac{1-\theta_1}{\theta_1} + k + 1\right) \sqrt{\left(\frac{a}{b} \frac{1-\theta_1}{\theta_1} + k - 1\right)^2 + 4k} \\
        &+ \left(\frac{a}{b} \frac{1-\theta_1}{\theta_1} + k - 1\right)^2 + 4k .
    \end{split}
\end{equation}
This is a high-order polynomial equation in $ \theta_1 $, which does not admit a simple solution. So, while mass action is a good toy example, using a series expansion in powers of $ b/a $ is a better way to approach the Michaelis-Menten example, which our general theory in later sections also accounts for. In the following section, we will dedicate some time to analyze the Michaelis-Menten two-sector model, before proceeding with the classic three sector model. 

\section{Two-Sector Proteome Partition Model Under Michaelis-Menten Kinetics}

Consider again a simplified model with only two sectors: the translational sector and the ribosomal sector. In the starvation case, our methods will easily generalize to any number of sectors, while the overabundance case is only easily solvable for two and no more sectors.

In the $ (Y,N) $ coordinates, we can write:  
\begin{subequations}
    \begin{align}
        \frac{\mathrm{d} Y_1}{\mathrm{d} t} &= b Y_2 - \frac{a Y_1 Y_3}{ k + Y_1} - b Y_2 Y_1 ,\\
        \frac{\mathrm{d} Y_2}{\mathrm{d} t} &= \theta_1 \left(\frac{a Y_1}{ k + Y_1}\right) Y_3 - b Y_2^2 ,\\
        \frac{\mathrm{d} Y_3}{\mathrm{d} t} &= \theta_2 \left(\frac{ a Y_1}{ k + Y_1}\right) Y_3 - b Y_2 Y_3
    \end{align}
\end{subequations}

The unique nonnegative steady states for this system are given by
\begin{subequations}
    \begin{align}
        Y_1 &= \frac{1}{2} \left[ \sqrt{\left(\frac{a \theta_2}{b \theta_1} + k - 1\right)^{2} + 4k} - \left(\frac{a \theta_2}{b \theta_1} + k - 1\right) \right], \\
        Y_2 &= \left(\frac{a \theta_2}{b}\right) \frac{Y_1}{k + Y_1} \\
        Y_3 &= \left(\frac{a \theta_2^2}{b \theta_1}\right) \frac{Y_1}{k + Y_1}
    \end{align}
\end{subequations}
as in the general $ n $-sector case which we will see later. 

\subsection{Starvation and Overabundance Limits}
In the $ b \to 0 $ limit, apply Lagrangian multipliers on $ L(\theta , \mu) = b Y_2 (\theta) - \mu (\theta_1 + \theta_2 - 1) $ gives 
\begin{subequations}
    \begin{align}
        Y_1 &= k \left(\frac{ b \theta_{1}}{a_{1} \theta_{2}}\right) + k (1 - k) \left(\frac{ b \theta_{1}}{a_{1} \theta_{2}}\right)^2 + O(b^3),\\[4pt]
        Y_2 &= \theta_{1} \left[1 - k \left(\frac{ b \theta_{1}}{a_{1} \theta_{2}}\right) + k (1- k) \left(\frac{ b \theta_{1}}{a_{1} \theta_{2}}\right)^2\right] + O(b^3), \\[4pt]
        Y_3 &= \theta_{2} \left[1 - k \left(\frac{ b \theta_{1}}{a_{1} \theta_{2}}\right) + k (1- k) \left(\frac{ b \theta_{1}}{a_{1} \theta_{2}}\right)^2\right] + O(b^3). 
    \end{align}
\end{subequations}
and
\begin{subequations}
    \begin{align}
        \frac{\partial L}{\partial \theta_1} &= b - 2b^2\theta_1 \left(\frac{k_1}{a_1\theta_2} + \frac{k_2}{a_2\theta_3}\right) - \mu = 0, \label{equ:two_sector_lagrange1} \\
        \frac{\partial L}{\partial \theta_3} &= \left(\frac{k_2\theta_1^2}{a_2\theta_3^2}\right) b^2 - \mu = 0, \label{equ:two_sector_lagrange2}
    \end{align}
\end{subequations}
and hence 

\begin{subequations}
    \begin{align}
        \theta_1 &= 1 - \sqrt{\frac{k}{a}} \sqrt{b} + \frac{1}{2} \left(\frac{k}{a}\right)^{3/2} b^{3/2} + O(b^{5/2}), \\
        \theta_2 &= \sqrt{\frac{k}{a}} \sqrt{b} - \frac{1}{2} \left(\frac{k}{a}\right) b^{3/2} + O(b^{5/2}), \\
        \lambda &= b - 2 \sqrt{\frac{k}{a}} b^{3/2} - \left(\frac{1 - 2k}{a}\right) b^2 + O(b^{5/2}).  
    \end{align}
\end{subequations}

\begin{figure}[htbp]
    \centering
    \begin{minipage}{0.48\textwidth}
        \centering
        \includegraphics[width=\textwidth]{Images/two_theta1_small.png}
        \caption{$ \theta_1 $ in the starvation limit.}
        \label{fig:MM2_theta1_low_b}
    \end{minipage}
    \hfill
    \begin{minipage}{0.48\textwidth}
        \centering
        \includegraphics[width=\textwidth]{Images/two_theta2_small.png}
        \caption{$ \theta_2 $ in the starvation limit.}
        \label{fig:MM2_theta2_low_b}
    \end{minipage}
\end{figure}

\begin{figure}
    \centering
    \includegraphics[width=0.75\textwidth]{Images/two_lambda_b_small.png}
    \caption{Growth rate $ \lambda $ versus $ b $ in the starvation limit.}
    \label{fig:MM2_lambda_low_b}
\end{figure}

On the other hand, in the $ b \to \infty $ limit we have  
\begin{subequations}
    \begin{align}
        Y_1 &= 1 - \frac{1}{1 + k} \left(\frac{a \theta_{2}}{b \theta_{1}}\right) + \frac{k}{(1 + k)^3} \left(\frac{a \theta_{2}}{b \theta_{1}}\right)^2 + O\!\left(b^{-3}\right),\\[4pt]
        Y_2 &= \theta_1 \left[\frac{1}{1 + k} \left(\frac{a \theta_{2}}{b \theta_{1}}\right) - \frac{k}{(1 + k)^3} \left(\frac{a \theta_{2}}{b \theta_{1}}\right)^2\right] + O\!\left(b^{-3}\right), \\[4pt]
        Y_3 &= \theta_2 \left[\frac{1}{1 + k} \left(\frac{a \theta_{2}}{b \theta_{1}}\right) - \frac{k}{(1 + k)^3} \left(\frac{a \theta_{2}}{b \theta_{1}}\right)^2\right] + O\!\left(b^{-3}\right).
    \end{align}
\end{subequations}

Then apply Lagrangian multipliers on $ L $ up to order $ b^{-2} $, where 
\begin{equation}
    L(\theta, \mu) = b \theta_1 \left[\frac{1}{1 + k} \left(\frac{a \theta_{2}}{b \theta_{1}}\right) - \frac{k}{(1 + k)^3} \left(\frac{a \theta_{2}}{b \theta_{1}}\right)^2\right] - \mu (\theta_1 + \theta_2 - 1).
\end{equation}
Then we have
\begin{subequations}
    \begin{align}
        \frac{\partial L}{\partial \theta_1} &= \frac{k a^{2}}{(1+k)^{3}} \left(\frac{\theta_2}{\theta_1}\right)^2 \frac{1}{b} - \mu = 0, \\
        \frac{\partial L}{\partial \theta_2} &= \frac{a}{1+k} - \frac{k}{(1+k)^3} \frac{2 a_1^{2} \theta_2}{b \theta_1} - \mu = 0.
    \end{align}
\end{subequations}

Solving for $ \theta_2 / \theta_1 $ gives
\begin{equation}
    \frac{a}{1+k} - \frac{2ka^2}{(1+k)^3}\left(\frac{\theta_2}{\theta_1}\right) \frac{1}{b} = \mu = \frac{k a^2}{(1+k)^2} \left(\frac{\theta_2}{\theta_1}\right)^2 \frac{1}{b}
\end{equation}
\begin{equation}
    \Longrightarrow\; \frac{\theta_2}{\theta_1} = -1 + \sqrt{1 + \frac{(1+k)^2}{ka} b}, \quad \text{ up to order $ O(b) $.}
\end{equation}
We have \emph{$ \theta_1 = O(1 / \sqrt{b}) $, $ \theta_2 = O(1) $}. Unlike the $ b \to 0 $ limit, the $ b \to \infty $ limit is different from the $ n \geq 3 $-sector models in that we can solve for $ \theta_2 / \theta_1 $ nicely without further approximations. More precisely, we have 

\begin{subequations}
    \begin{align}
        \theta_1 &= \frac{\sqrt{k}}{1+k} \sqrt{\frac{a}{b}} - \frac{k}{2(1+k)^2} \left(\frac{a}{b}\right)^{3/2} + O\left(\left(a/b\right)^{5/2}\right), \\
        \theta_2 &= 1 - \frac{\sqrt{k}}{1+k} \sqrt{\frac{a}{b}} + \frac{k}{2(1+k)^2} \left(\frac{a}{b}\right)^{3/2} + O\left(\left(a/b\right)^{5/2}\right), \\
        \lambda &= \frac{a}{1+k} - \frac{2a\sqrt{k}}{(1+k)^{2}}\,\sqrt{\frac{a}{b}} + \frac{2k}{(1+k)^{3}}\frac{a^{2}}{b} + O\!\left( b^{-2}\right).
    \end{align}
\end{subequations}

\begin{figure}[htbp]
    \centering
    \begin{minipage}{0.48\textwidth}
        \centering
        \includegraphics[width=\textwidth]{Images/two_theta1_large.png}
        \caption{$ \theta_1 $ in the overabundance limit.}
        \label{fig:MM2_theta1_large_b}
    \end{minipage}
    \hfill
    \begin{minipage}{0.48\textwidth}
        \centering
        \includegraphics[width=\textwidth]{Images/two_theta2_large.png}
        \caption{$ \theta_2 $ in the overabundance limit.}
        \label{fig:MM2_theta2_large_b}
    \end{minipage}
\end{figure}

\begin{figure}
    \centering
    \includegraphics[width=0.75\textwidth]{Images/two_lambda_b_large.png}
    \caption{Growth rate $ \lambda $ versus $ b $ in the overabundance limit.}
    \label{fig:MM2_lambda_large_b}
\end{figure}

\subsection{Bottleneck Limit}
In the $ a \to 0 $ limit, the expansion is identical to the $ b \to \infty $ limit for the two-sector model, since \emph{it is only meaninful to consider the dimensionless parameter $ a / b $ in the system}, instead of $ a $ and $ b $ as absolute quantities. Applying Lagrange multipliers on $ L $ gives, as before, 
\begin{equation}
    w\; \frac{\theta_2}{\theta_1} = -1 + \sqrt{1 + \frac{(1+k)^2}{ka} b}, \quad \text{ up to order $ O(b) $.}
\end{equation}
We have \emph{$ \theta_1 = O(\sqrt{a}) $, $ \theta_2 = O(1) $}, and $ \theta_1, \theta_2 $ are as given above.

% Main results: this section is the most important
\section{The Three-Sector Proteom Partition Model of Scott}

The model described in the main text corresponds to the three-sector proteome partition model of Scott et al. \cite{Scott2010}, which has been widely used to describe bacterial growth physiology. Apart from the aforementioend reference, it has also been elaborated in (*). In this section, we analyze the steady states of the system, and derive analytic solutions in the starvation and overabundance limits.

\subsection{Steady States of Concentrations and Biomass Fractions}

As time tends to infinity, the biomass fractions for this system tend to a steady limit, i.e. no oscillation is observed. So we have the steady-state approximation $ \mathrm{d} Y / \mathrm{d} t \to 0 $ when $ t > T $, and $ Y(t) \approx Y^* $ is the steady-state, which for simplicity we denote by $ Y $ . Then equations (\ref{equ:Y}) become 

% Main ODE equations 
\begin{subequations}
    \label{equ:Y_steady}
    \begin{align}
        b Y_3 - \frac{a_1 Y_1 Y_4}{ k_1 + Y_1} - b Y_3 Y_1 &= 0, \label{equ:Y1_steady} \\
        \frac{a_1 Y_1 Y_4}{ k_1 + Y_1} - \frac{a_2 Y_2 Y_5}{ k_2 + Y^*_2} - b Y_3 Y_2 &= 0, \label{equ:Y2_steady} \\
        \theta_1 \left(\frac{ a_2 Y_2 Y_5}{ k_1 + Y_2}\right) - b (Y_3)^2 &= 0, \label{equ:Y3_steady} \\
        \theta_2 \left(\frac{ a_2 Y_2 Y_5}{ k_1 + Y_2}\right) - b Y_3 Y_4 &= 0, \label{equ:Y4_steady} \\
        \theta_3 \left(\frac{ a_2 Y_2 Y_5}{ k_1 + Y_2}\right) - b Y_3 Y_5 &= 0. \label{equ:Y5_steady} \\
    \end{align}
\end{subequations}
Since $ Y_3, Y_4, Y_5 \neq 0 $, the equations (\ref{equ:Y3_steady}), (\ref{equ:Y4_steady}), and (\ref{equ:Y5_steady}) show that 
\[
    Y_3 \colon Y_4 \colon Y_5 = \theta_1 \colon \theta_2 \colon \theta_3.
\]
Work with $ Y_3 $, so $ Y_4 = (\theta_2 / \theta_1) Y_3 $ and $ Y_5 = (\theta_3 / \theta_1) Y_3 $. Then the first two equations become 
\begin{subequations}
    \begin{align}
        (Y_1)^{2} + \left(\frac{a_1 \theta_2}{b \theta_1} + k_1 - 1 \right) Y_1 - k_1 &= 0, \\
        (Y_2)^{2} + \left(\frac{a_1 \theta_2}{b \theta_1} + k_1 - (1 - Y_1) \right) Y_2 - (1 - Y_1 ) k_{2} &= 0. \\ 
    \end{align}
\end{subequations} 

Equation (\ref{equ:Y5_steady}) gives
\begin{equation}
    Y_3 = \left(\frac{a_2 \theta_3}{b}\right) \left(\frac{Y_2}{k_2 + Y_2}\right).
\end{equation}
That is, given $ ( \theta_1 , \theta_2 , \theta_3 ) $, we have an analytic solution of the steady state $ Y = Y[\theta_1, \theta_2, \theta_3] $, given by equation (\ref{equ:Y_steady_sol}): 

\begin{equation*}
    1 - Y_1 = \frac{1}{2} \left[ -\sqrt{\left(\frac{a_1 \theta_2}{b \theta_1}+ k_1 - 1\right)^{2} + 4k_1} + \left(\frac{a_1 \theta_2}{b \theta_1} + k_1 + 1\right) \right]. 
\end{equation*}

% \notag disables numbering
% This is the analytic solution of the system
\begin{subequations}
    \label{equ:Y_steady_sol}
    \begin{align}
        %
        Y_1 &= \frac{1}{2} \left[ \sqrt{\left(\frac{a_1 \theta_2}{b \theta_1} + k_1 - 1\right)^{2} + 4k_1} - \left(\frac{a_1 \theta_2}{b \theta_1} + k_1 - 1\right) \right], \\
        %
        Y_2 &= \frac{1}{2} \left[ \sqrt{\left(\frac{a_2 \theta_3}{b \theta_1} + k_2 - (1 - Y_1 )\right)^{2} + 4 (1 - Y_1) k_2} - \left(\frac{a_1 \theta_2}{b \theta_1} + k_2 - (1 - Y_1 ) \right) \right] \notag \\ 
        &= \frac{1}{2} \sqrt{
            \begin{aligned}
                & \left\{ \frac{a_2 \theta_3}{b \theta_1} + k_2 - \frac{1}{2} \left[ -\sqrt{\left(\frac{a_1 \theta_2}{b \theta_1}+ k_1 - 1\right)^{2} + 4k_1} + \left(\frac{a_1 \theta_2}{b \theta_1} + k_1 + 1\right) \right] \right\}^{2} + \\
                &4k_1 \left\{ \frac{1}{2} \left[ -\sqrt{\left(\frac{a_1 \theta_2}{b \theta_1}+ k_1 - 1\right)^{2} + 4k_1} + \left(\frac{a_1 \theta_2}{b \theta_1} + k_1 + 1\right) \right] \right\} \\
            \end{aligned}
        } \notag \\ 
        & - \frac{1}{2} \left\{ \frac{a_2 \theta_3}{b \theta_1} + k_2 - \frac{1}{2} \left[ - \sqrt{ \left(\frac{a_1 \theta_2}{b \theta_1} + k_1 -1\right)^{2} + 4 k_1} + \left(\frac{a_1 \theta_2}{b \theta_1} + k_1 + 1\right) \right] \right\}, \\
        %
        Y_3 &= \left( \frac{a_2 \theta_3}{b}\right) \left(\frac{Y_2}{k_2 + Y_2}\right), \\
        %
        Y_4 &= \left( \frac{a_2 \theta_2 \theta_3}{b \theta_1}\right) \left(\frac{Y_2}{k_2 + Y_2}\right), \\
        %
        Y_5 &= \left( \frac{a_2 \theta_3^{2}}{b \theta_1}\right) \left(\frac{Y_2}{k_2 + Y_2}\right).
    \end{align}
\end{subequations}
This is an expression of $Y = Y(\theta_1, \theta_2, \theta_3)$, where $\theta$ is undetermined. The main result expresses $\theta$ in terms of system parameters $a_1, a_2, b, k_1, k_2$, and hence $Y$ in terms of $a_1, a_2, b, k_1, k_2$. In the above solution, the minus sign solution of the quadratic equations 
\begin{align}
    Y_1^2 + \left(\frac{a_1\theta_2}{b\theta_1} + k_1 - 1\right)Y_1 - k_1 &= 0 , \\
    Y_2^2 + \left[\frac{a_2\theta_3}{b\theta_1} + k_2 - \left(1 - Y_1\right)\right]Y_2 - k_2\left(1 - Y_1\right) &= 0
\end{align}
is discarded, since biomass fraction must be positive.

\subsection{Derivation of the Partition Strength in the Low Nutrient Limit}
The partition strength is given here up to order $O(b^2)$:
\begin{align}
    \theta_1 &= 1 - A\sqrt{b} + \frac{1}{2}A^3 b^{3/2} + O\! \left(b^{5/2}\right),\\
    \theta_2 &= \sqrt{\frac{k_1}{a_1}} \left[\sqrt{b} - \frac{1}{2}A^2 b^{3/2} \right] + O(b^{5/2}), \\
    \theta_3 &= \sqrt{\frac{k_2}{a_2}} \left[\sqrt{b} - \frac{1}{2}A^2 b^{3/2} \right] + O(b^{5/2}),
\end{align}

where 
\begin{equation}
    A = \sqrt{\frac{k_1}{a_1}} + \sqrt{\frac{k_2}{a_2}}, \quad 
    B = \frac{k_1}{a_1} + \frac{k_2}{a_2}, \quad 
    C = \sqrt{\frac{k_1k_2}{a_1 a_2}}, \quad 
    D = \frac{1}{a_1}+\frac{1}{a_2}.
\end{equation}
are symmetric in $(k_1, a_1), (k_2, a_2)$. To derive this, consider the series expansion of $Y_1, Y_2, Y_3$ in the limit $b \to 0$. 

\begin{align}
    Y_1 &= \left(\frac{k_1\theta_1}{a_1\theta_2}\right)b + \left[\frac{k_1 \theta_1^2 (1-k_1)}{a_1^2 \theta_2^2}\right]b^2 + O(b^3), \\
    Y_2 &= \left(\frac{k_2\theta_1}{a_2\theta_3}\right)b - \left[\frac{k_2\theta_1^2\left(k_1a_2\theta_3 + k_2a_1\theta_2 - a_1\theta_2\right)}{a_1 a_2^2 \theta_2 \theta_3^2}\right]b^2 + O(b^3), \\
    Y_3 &= \theta_1 - b\theta_1^2\left(\frac{k_1}{a_1\theta_2} + \frac{k_2}{a_2\theta_3}\right) + b^2\theta_1^3 \left[\frac{k_1 k_2}{a_1 a_2\theta_2\theta_3} - \frac{k_1(1-k_1)}{a_1^2\theta_2^2} - \frac{k_2(1-k_2)}{a_2^2\theta_3^2}\right].
\end{align}

Therefore 
\begin{equation}
    \lambda = b\theta_1 - b^2\theta_1^2\left(\frac{k_1}{a_1\theta_2} + \frac{k_2}{a_2\theta_3}\right) + b^3\theta_1^3 \left[\frac{k_1 k_2}{a_1 a_2\theta_2\theta_3} - \frac{k_1(1-k_1)}{a_1^2\theta_2^2} - \frac{k_2(1-k_2)}{a_2^2\theta_3^2}\right].
\end{equation}

By the method of Lagrange multipliers, set the Lagrangian to 
\begin{equation}
    L(\theta) = \lambda(\theta) - \mu \left(\theta_1 + \theta_2 + \theta_3 - 1\right)
\end{equation}
with Lagrange multiplier $\mu$. Then 
\begin{align}
    \frac{\partial L}{\partial \theta_1} &= b - 2b^2\theta_1 \left(\frac{k_1}{a_1\theta_2} + \frac{k_2}{a_2\theta_3}\right) + 3b^3 \theta_1^2\left[\frac{k_1 k_2}{a_1 a_2\theta_2\theta_3} - \frac{k_1(1-k_1)}{a_1^2\theta_2^2} - \frac{k_2(1-k_2)}{a_2^2\theta_3^2}\right] - \mu, \label{equ:lagrange1}\\
    \frac{\partial L}{\partial \theta_2} &= \left(\frac{k_1\theta_1^2}{a_1\theta_2^2}\right) b^2 - \theta_1^3\left[\frac{k_1k_2}{a_1a_2\theta_2\theta_3} - \frac{2k_1(1-k_1)}{a_1^2\theta_2^3}\right]b^3 - \mu, \label{equ:lagrange2} \\
    \frac{\partial L}{\partial \theta_3} &= \left(\frac{k_2\theta_1^2}{a_2\theta_3^2}\right) b^2 - \theta_1^3\left[\frac{k_1k_2}{a_1a_2\theta_2\theta_3} - \frac{2k_2(1-k_2)}{a_2^2\theta_3^3}\right]b^3 - \mu. \label{equ:lagrange3} \\
\end{align}
Keep only terms of order up to $O(b^2)$ and solve equations (\ref{equ:lagrange2}, \ref{equ:lagrange3}), we get an important relationship that generalizes to the $n$-sector model,
\begin{equation}
    \frac{\theta_2}{\theta_3} = \sqrt{\frac{k_1 a_2}{k_2 a_1}}.
\end{equation}
Then equation (\ref{equ:lagrange1}) becomes a quadratic in $\theta_1/\theta_2$:
\begin{equation}
    \left(\frac{bk_1}{a_1}\right)\left(\frac{\theta_1}{\theta_2}\right)^2 + 2b\sqrt{\frac{k_1}{a_1}}A\left(\frac{\theta_1}{\theta_2}\right) - 1 = 0,
\end{equation}
where $A = \sqrt{\frac{k_1}{a_1}} + \sqrt{\frac{k_2}{a_2}}$. The solution admits an asymptotic expansion in the form of a Puiseux series:
\begin{equation}
    r \equiv \frac{\theta_1}{\theta_2} = \sqrt{\frac{a_1}{k_1}}\left(\frac{1}{\sqrt{b}} - A + \frac{1}{2}A^2\sqrt{b}\right) + O(b),
\end{equation}
and
\begin{equation}
    s \equiv \frac{\theta_1}{\theta_3} = \sqrt{\frac{a_2}{k_3}}\left(\frac{1}{\sqrt{b}} - A + \frac{1}{2}A^2\sqrt{b}\right) + O(b).
\end{equation}
The symmetry in the equations hints that we should solve for $\theta_1$ first, then find $\theta_2 = \theta_1/r, \theta_3 = \theta_1/s$. We expand up to order $O(b^{3/2})$ and solve for $ \theta_1 $. 
\begin{equation}
    \theta_1 = (1 + 1/r + 1/s)^{-1},
\end{equation} 
then 
\begin{subequations}
    \begin{align}
        \theta_1 &= 1 - A\sqrt{b} - \frac{1}{2} A^3 b^{3/2} + O(b^2), \\ 
        \theta_2 &= \sqrt{\frac{k_1}{a_1}}\left(\sqrt{b} - \frac{1}{2}A^{2} b^{3/2} \right) + O(b^{2}), \\
        \theta_3 &= \sqrt{\frac{k_2}{a_2}}\left(\sqrt{b} - \frac{1}{2}A^{2} b^{3/2} \right) + O(b^{2}). 
    \end{align}
\end{subequations}

\subsection{Derivation of Protein Fraction and Growth Rate During Starvation}
\subsubsection{Trajectories and Growth Rate}
The growth rate $\lambda$ is given as a perturbative series in $\sqrt{b}$:
\begin{equation}
    \lambda = b - 2Ab^{3/2} + \left( 2A^2 - B \right) b^2 + O(b^{5/2}),
    \label{equ:growth_rate}
\end{equation}
which is easily derived by plugging in equations for $\theta$. This formula makes intuitively sense, since it is invariant under exchange of indices (i.e. $(a_1, k_1) \to (a_2, k_2)$), as will be observed for many quantities below. Substitute in expressions for $\theta_{\ast} $ to get
\begin{align}
    Y_{1*} &= \sqrt{\frac{k_1}{a_1}}\sqrt{b} + \left(\frac{1}{a_1}(1-k_1) - A\sqrt{\frac{k_1}{a_1}}\right)b + \sqrt{\frac{k_1}{a_1}}A^2 b^{3/2} + O(b^2), \\
    Y_{2*} &= \sqrt{\frac{k_2}{a_2}}\sqrt{b} + \left(\sqrt{\frac{k_2}{a_2}}A + C - \frac{(1-k_2)}{a_2}\right)b + \left[\frac{1}{2}\sqrt{\frac{k_2}{a_2}}A^2 + 2AC - 2A\frac{(1-k_2)}{a_2}\right]b^{3/2} + O(b^2), \\
    Y_{3*} &= 1 - 2A\sqrt{b} + ( 2A^2 - B ) b + O(b^{3/2}), \\
    Y_{4*} &= \sqrt{\frac{k_1}{a_1}}\left[\sqrt{b} - Ab + \left(\frac{1}{2}A^2 + B + C - 1 \right)b^{3/2}\right] + O(b^2), \\
    Y_{5*} &= \sqrt{\frac{k_2}{a_2}}\left[\sqrt{b} - Ab + \left(\frac{1}{2}A^2 + B + C - 1 \right)b^{3/2} \right] + O(b^2). \label{equ:Y5}
\end{align}

We also have the fraction of small molecules (intermediate metabolites, amino acids, and peptides) and of large molecules (transport/metabolic/ribosomal and translational proteins), denoted $\text{Small}$ and $\text{Large}$ respectively.
\begin{align}
    \text{Small} &= A\sqrt{b} - \left(A^2+B+C-D\right)b + O(b^{3/2}), \\
    \text{Large} &= 1 - \text{Small} = 1 - A\sqrt{b} + \left(A^2+B+C-D\right)b + O(b^{3/2}).
\end{align}
These are also invariant under the exchange of indices, so the rate of nutrient processing does not affect the overall fraction of products.

\subsubsection{Growth Rate and Ribosomal Protein Fraction}
Equations (\ref{equ:growth_rate}) and (\ref{equ:Y5}) express $\lambda$ and $Y_5$ in terms of nutrient level $b$. Recall the theory of \textbf{series reversion}: Suppose \( y = f(x) = a_1 x + a_2 x^2 + a_3 x^3 + a_4 x^4 + a_5 x^5 + \cdots \). Then the inverse series \( x = f^{-1}(y) \) up to the $5$th order is:
\begin{equation*}
    \begin{aligned}
    x &= \frac{1}{a_1} y 
        - \frac{a_2}{a_1^3} y^2 
        + \frac{2a_2^2 - a_1 a_3}{a_1^5} y^3 \\
      &\quad + \frac{5a_1 a_2 a_3 - 5a_2^3 - a_1^2 a_4}{a_1^7} y^4 \\
      &\quad + \frac{-6a_2^4 + 9a_1 a_2^2 a_3 - 3a_1^2 a_3^2 - 4a_1^2 a_2 a_4 + a_1^3 a_5}{a_1^9} y^5 + \cdots,
    \end{aligned}
\end{equation*}

and ribosomal protein fraction $R$ can be written in terms of $ \lambda $ if we plug in $\sqrt{b} = \sqrt{b}(\lambda) $:
\begin{equation}
    R = Y_{5*} = R (\lambda) = \sqrt{\frac{k_2}{a_2}}\;\sqrt{\lambda}\left[ 1 + \frac{1}{2}\bigl(A^2 - C - D\bigr)\lambda \right].
\end{equation}


% Special Function Analysis
\section{Michaelis-Menten in the Language of Special Functions}

The quadratic equations that arise in the steady state analysis of our proteome partition model is ubiquitous in similar analyses. Therefore, it is extremely helpful to define a relevant special function, which would aid our analysis of the fixed points of the system.

\subsection{Michaelis-Menten Calculations}
The equation 
\begin{equation}
    1 - mx = \frac{cx}{k+x} \label{equ:MM_special}
\end{equation}
is ubiquitous in the analysis of Michaelis-Menten fluxes. It admits the nonnegative solution given by
\begin{equation}
    p_{\text{MM}} (c, k, m) = \frac{1}{2m}\left[(1 - c - mk) + \sqrt{(1 - c- mk)^2 + 4mk} \right]. 
\end{equation}
The quantity $ p_{\text{MM}}(c,k,m) $ always lies between $ 0 $ and $ 1/m $, is an increasing function of $ c $, $ m $, and a decreasing function of $ k $. For our system, we can express the fixed point solution as

\begin{subequations}
    \begin{align}
        Y^\ast_1 &= p_{\text{MM}}\left(\frac{a_1 \theta_2}{b \theta_1}, k_1, 1\right) \equiv p_{\text{MM}}(c_1, k_1, m_1), \\
        Y^\ast_2 &= p_{\text{MM}}\left(\frac{a_2 \theta_3}{a_1 \theta_2} \frac{k_1 + Y^\ast_1}{Y^\ast_1}, k_2, \frac{b \theta_1}{a_1 \theta_2} \frac{k_1 + Y^\ast_1}{Y^\ast_1}\right) \equiv p_{\text{MM}}(c_2, k_2, m_2), \\
        Y^\ast_{k+2} &= \theta_k \left(1 - Y^\ast_1 - Y^\ast_2\right), \quad k = 1,2,3.
    \end{align}
\end{subequations}

We can \emph{express the general solution to the $ n $-partition model using the $ p_{\text{MM}} $ function}. This will help us in showing the system has a unique nonnegative fixed point.

\begin{subequations}
    \begin{align}
        Y_1^\ast &= p_{\text{MM}}\left(\frac{a_1 \theta_2}{b \theta_1}, k_1, 1\right), \\
        Y_i^\ast &= p_{\text{MM}}\left(\frac{a_i \theta_{i+1}}{a_{i-1} \theta_i} \frac{k_{i-1} + Y_{i-1}^\ast}{Y_{i-1}^\ast}, k_i, \frac{b \theta_1}{a_{i-1} \theta_i} \frac{k_{i-1} + Y_{i-1}^\ast}{Y_{i-1}^\ast}\right), \quad i = 2, \ldots, n-1, \\
        Y_{k+n-1}^\ast &= \theta_k \left(1 - \sum_{i=1}^{k-1} Y_k^\ast\right), \quad k = 1, \dots , n.
    \end{align}
\end{subequations} 

Since $ p_\text{MM}(c,k,m) $ is the intersection of the line $ 1 - mx = 0 $ and the Michaelis-Menten curve $ (cx)/(k+x) $, it is clear that for any $ c, k, m > 0 $, there is a unique nonnegative solution to equation (\ref{equ:MM_special}). By iterating $ p_\text{MM} $ and applying the Intermediate Value Theorem, we immediately see that the $ n $-partition model has a unique nonnegative fixed point for any choice of positive parameters.

Let's consider the relevant limits of $ p_{\text{MM}}(c,k,m) $ for the $ b \to 0 $ and $ b \to \infty $ limits. When $ b \to 0 $, we have $ c_1 \to \infty $ and $ c_2 \to \infty $. Here, even though it seems like $ m_2 \to 0 $, we simultaneously have $ Y_1^\ast \to 0 $, so in fact $ m_2 = O(1) $. When $ b \to \infty $, we have $ c_1 \to 0 $, $ Y_1^\ast \to 1 $, and hence $ m_2 \to \infty $.

\begin{enumerate}[(1)]
    \item $ c \to 0 $: By direct expansion of $ p(c,k,m) $ about $ c=0 $, we have 
    \begin{equation}
        p(c,k,m) = \frac{1}{m} - \frac{c}{m\,(k m + 1)} + \frac{k}{(k m + 1)^3}\,c^{2} + \frac{k\,(1 - k m)}{(k m + 1)^5}\,c^{3} + O(c^{4}).
    \end{equation}
    \item $ c \to \infty $: By direct expansion of $ p(c,k,m) $ about $ c = \infty $, we have 
    \begin{equation}
        p(c,k,m) = \frac{k}{c} + \frac{k - k^{2} m}{c^{2}} + \frac{k\,(k^{2} m^{2} - 3 k m + 1)}{c^{3}} + O(c^{-4}).
    \end{equation}
    \item $ m \to \infty $: By direct expansion of $ p(c,k,m) $ about $ m = \infty $, we have  
    \begin{equation}
        p(c,k,m) = \frac{1}{m} - \frac{c}{k\,m^{2}} + \frac{c(1 + c)}{k^{2}\,m^{3}} + O(m^{-4}).
    \end{equation}
\end{enumerate}

\subsubsection{Starvation Limit Calculations}
We use the special function expansions to illustrate the \emph{leading-order behavior} of the system in the starvation ($ b \to 0 $) and overabundance ($ b \to \infty $) limits.

When $ b \to 0 $, we have $ c_1 \to \infty $, $ c_2 \to \infty $, and $ m_2 = O(1) $. Therefore, using the $ c \to \infty $ expansion on $ c_1 $, we have
\begin{equation}
    Y_1 = k_1 \left(\frac{b \theta_1}{a_1 \theta_2}\right) + k_1 (1 - k_1) \left(\frac{b \theta_1}{a_1 \theta_2}\right)^2 + O \left( n_1^3 \right), 
\end{equation}
and
\begin{align}
    \frac{Y_1^\ast}{k_1 + Y_1^\ast} &= \left(\frac{b \theta_1}{a_1 \theta_2}\right) - k_1 \left(\frac{b \theta_1}{a_1 \theta_2}\right)^2 + O\left(n_1^3\right), \\
    \frac{k_1 + Y_1^\ast}{Y_1^\ast} &= \left(\frac{a_1 \theta_2}{b \theta_1}\right) + k_1 + O\left(n_1^3\right).
\end{align}
Then, using the same expansion on $ c_2 $, we have 
\begin{equation}
    \frac{1}{c_2} = \left(\frac{a_1 \theta_2}{a_2 \theta_3}\right) \left(\frac{Y_1^\ast}{k_1 + Y_1^\ast}\right) = \frac{b \theta_1}{a_2 \theta_3} - k_1 \left(\frac{b \theta_1}{a_2 \theta_3}\right) \left(\frac{b \theta_1}{a_1 \theta_2}\right) + O\left(n_2^3\right), 
\end{equation}
\begin{equation}
    m_2 = \frac{b \theta_1}{a_1 \theta_2} \frac{k_1 + Y_1^\ast}{Y_1^\ast} = 1 + k_1 \left(\frac{b \theta_1}{a_1 \theta_2}\right) + O\left(n_1^2\right).
\end{equation}
Applying the $ c \to \infty $ expansion on $ Y_2^\ast $, we have
\begin{equation}
    Y_2 = k_2 \left(\frac{b \theta_1}{a_2 \theta_3}\right) - \frac{k_2 (a_1 k_2 \theta_2 - a_1 \theta_2 + a_2 k_1 \theta_3)}{a_1 \theta_2 } \left(\frac{b \theta_1}{a_2 \theta_3}\right)^2 + O\left(n_2^3\right),
\end{equation}

so $ Y_3^\ast, Y_4^\ast, Y_5^\ast $ can be computed accordingly. \emph{This agrees with the full calculation.} In effect, we are computing the series expansion with respect to the dimensionless quantity $ n_1 = b \theta_1 / a_1 \theta_2 $ and $ n_2 = b \theta_1 / a_2 \theta_3 $. Similar analysis for the overabundance limit $ b \to \infty $ using the $ c \to 0 $ expansion also agrees with the full calculation.

\begin{align}
    Y_1 &= k_1 \left(\frac{b \theta_1}{a_1 \theta_2}\right) + \frac{k_1 \theta_2 (1 - k_1)}{\theta_2} \left(\frac{b \theta_1}{a_1 \theta_2}\right)^2 + O \left( n_1^3 \right), \\
    Y_2 &= k_2 \left(\frac{b \theta_1}{a_2 \theta_3}\right) - \frac{k_2 (a_1 k_2 \theta_2 - a_1 \theta_2 + a_2 k_1 \theta_3)}{a_1 \theta_2 } \left(\frac{b \theta_1}{a_2 \theta_3}\right)^2 + O\left(n_2^3\right).
\end{align}

\subsection{Mass Action Calculations}
Take the limit of the Michaelis-Menten equation when $ x \ll k $, and rename $ \frac{c}{k} \to w $. The equation becomes
\begin{equation}
    1 - mx = w x
\end{equation}
with $ m, w, x \geq 0$, $ x < 1 $. This equation admits the nonnegative solution
\begin{equation}
    p_{\text{MA}}(w, m) = \frac{1}{m + w}.
\end{equation}
The quantity $ p_{\text{MA}}(w, m) $ always lies between $ 0 $ and $ 1/m $, and is a decreasing function of $ w $ and $ m $. For the three-sector model, we can express the fixed point solution as

\begin{subequations}
    \begin{align}
        Y^\ast_1 &= p_{\text{MA}}\left(1, \frac{r_1 \theta_2}{b \theta_1} \right) \equiv p_{\text{MA}}(w_1, m_1), \\
        Y^\ast_2 &= p_{\text{MA}}\left(\frac{b \theta_1}{r_1 \theta_2} \frac{1}{Y_1^\ast}, \frac{r_2 \theta_3}{r_1 \theta_2} \frac{1}{Y^\ast_1}\right) \equiv p_{\text{MA}}(w_2, m_2), \\
        Y^\ast_{k+2} &= \theta_k W^\ast, \quad W^\ast \equiv 1 - Y^\ast_1 - Y^\ast_2, \quad k = 1,2,3.
    \end{align}
\end{subequations}

\begin{proposition}
    \label{prop:MA_MM_relation}
    ~ 

    \begin{enumerate}[(1)]
        \item The function $ p_\text{MM}(c,k,m) $ satisfies the rescaling relation 
        \begin{equation}
            p_{\text{MM}} (c,k,m) = \frac{1}{m} p_\text{MM} (c, km, 1) \equiv \frac{1}{m} \Phi (c, km),
        \end{equation}
        \item The function $ p_\text{MM}(c,k,m) $ satisfies the rescaling relation 
        \begin{equation}
            p_{\text{MM}} (c,k,m) = k p_\text{MM} (c, 1, km) \equiv \frac{1}{m} \Psi (c, km),
        \end{equation}
        \item Rescaling $ p_\text{MM} $ by $ k $ and taking the limit $ x \to 0 $ gives $ p_\text{MA} $.
    \end{enumerate}
\end{proposition}

\begin{proof}
    ~

    \begin{enumerate}[(1)]
        \item Let $ \xi = mx $, then 
        \[
            1 - \xi = \frac{c \, \xi}{km + \xi} \;\Longrightarrow\; x = \frac{\xi}{m} = \frac{1}{m} p_\text{MM}(c, km, 1).
        \]
        \item Let $ \xi = x / k $, then
        \[
            1 - km \xi = \frac{c \, \xi}{1 + \xi} \;\Longrightarrow\; x = k \, \xi = k \, p_\text{MM}(c, 1, km).
        \]
        \item Let $ \xi = x / k $, then
        \[
            1 - km \xi = \frac{c \, \xi}{1 + \xi} \to c \, \xi .
        \]
        Hence, we have $ p_{\text{MA}} (m, w) \equiv k \, p_{\text{MM}} (c, 1, km) $ when $ x \ll 1 $.
    \end{enumerate}
\end{proof}

As shown in proposition \ref{prop:MA_MM_relation}, we can express the general solution to the $ n $-partition model using the $ p_{\text{MA}} $ function. This will again help us in showing the system has a unique nonnegative fixed point.

\begin{subequations}
    \begin{align}
        Y_1^\ast &= p_{\text{MA}}\left(1, \frac{r_1 \theta_2}{b \theta_1} \right), \\
        Y_i^\ast &= p_{\text{MA}}\left(\frac{b \theta_1}{r_{i-1} \theta_i} \frac{1}{Y_{i-1}^\ast}, \frac{r_i \theta_{i+1}}{r_{i-1} \theta_i} \frac{1}{Y_{i-1}^\ast}\right), \quad i = 2, \ldots, n-1, \\
        Y_{k+n-1}^\ast &= \theta_k \left(1 - \sum_{i=1}^{k-1} Y_k^\ast\right), \quad k = 1, \dots , n.
    \end{align}
\end{subequations}

Before diving into explicit calculations, let us first discuss qualitatively the behavior of the system under Michaelis-Menten kinetics by defining some useful special functions.

\subsection{Plot}
A log-scale plot of the Michaelis-Menten special function $ p_{\text{MM}}(c,k,m) $ is shown in Figure \ref{fig:special_function_plot} below, with $ m = \frac{1}{2} $. We can see that as $ c $ increases, $ p_{\text{MM}}(c,k,m) $ increases from $ 0 $ to $ \frac{1}{m} = 2 $. As $ k $ increases, $ p_{\text{MM}}(c,k,m) $ decreases for fixed $ c $. This behavior is consistent with our earlier analysis.

\begin{figure}[h]
    \centering
    \includegraphics[width=0.8\textwidth]{Images/special.png}
    \caption{Plotted with $ m = \frac{1}{2} $.}
    \label{fig:special_function_plot}
\end{figure}



% General Model - Approximate Theory
\section{General Model in the Starvation Limit}
In this section, we derive the explicit formula for \emph{optimal partition fractions} and \emph{optimal biomass fractions} for the arbitrarily-long metabolic chain model in the starvation limit.

\subsection{Example Calculation for Five Partitions}
Following the spirit of last weeks' calculation, the steady state equations for the five-sector model are
\begin{subequations}
    \begin{align}
        Y_1 &= \frac{k_1\,\theta_1}{a_1\,\theta_2}\,b + \frac{k_1\,\theta_1^2(1-k_1)}{a_1^2\,\theta_2^2}\,b^2,\\
        Y_2 &= \frac{k_2\,\theta_1}{a_2\,\theta_3}\,b - \,k_2\,\theta_1^2\!\left[\frac{k_2-1}{a_2^2\,\theta_3^2}+\frac{k_1}{a_1 a_2\,\theta_2\theta_3}\right] b^2,\\
        Y_3 &= \frac{k_3\,\theta_1}{a_3\,\theta_4}\,b - \,k_3\,\theta_1^2\!\left[\frac{k_3-1}{a_3^2\,\theta_4^2}+\frac{k_2}{a_2 a_3\,\theta_3\theta_4}+\frac{k_1}{a_1 a_3\,\theta_2\theta_4}\right] b^2,\\
        Y_4 &= \frac{k_4\,\theta_1}{a_4\,\theta_5}\,b - \,k_4\,\theta_1^2\!\left(\frac{k_4-1}{a_4^2\,\theta_5^2}+\frac{k_3}{a_3 a_4\,\theta_4\theta_5}+\frac{k_2}{a_2 a_4\,\theta_3\theta_5}+\frac{k_1}{a_1 a_4\,\theta_2\theta_5}\right) b^2,\\
        Y_5 &= \theta_1 \left[ 1 - b \theta_1 \!\left(\sum_{i=1}^4 \frac{k_i}{a_i\theta_{i+1}} \right) + b^2 \theta_1^2 \!\left(\sum_{i=1}^{4}\frac{k_i^2-k_i}{a_i^2\,\theta_{i+1}^2}+\sum_{1\le i<j\le 4}\frac{k_i k_j}{a_i a_j\,\theta_{i+1}\theta_{j+1}}\right) \right],\\
        Y_6 &= \frac{\theta_2}{\theta_1} Y_5, \quad Y_7 = \frac{\theta_3}{\theta_1} Y_5, \quad Y_8 = \frac{\theta_4}{\theta_1} Y_5, \quad Y_9 = \frac{\theta_5}{\theta_1} Y_5.
    \end{align}
\end{subequations}

The form of the partition strength in the five-partition model during starvation (small $b$) has been found in the result from 20251029 - 20251102. Let $ c_j = \sqrt{k_j / a_j} $, $ 1 \leq j\leq 4 $, $ A = c_1 + c_2 + c_3 + c_3 $, and
\begin{equation}
    B = \frac{1-k_1}{a_1} + \frac{1-k_2}{a_2} + \frac{1-k_3}{a_3} + \frac{1-k_4}{a_4} - (c_1 c_2 + c_1 c_3 + c_1 c_4 + c_2 c_3 + c_2 c_4 + c_3 c_4).
\end{equation}
Both $ A $ and $ B $ are invariant under exchange of indices.

% Leave one page to display the equation
\newpage
The optimal biomass fractions up to order $ b^2 $ are listed below:
\begin{subequations}
    \begin{align}
        Y_1^* &= c_1 \sqrt{b} + \left(\frac{1-k_1}{a_1} - A c_1\right) b + \left[\frac{1}{2}A^2 c_1 - 2A \left(\frac{1-k_1}{a_1}\right)\right] b^{3/2}, \\
        Y_2^* &= c_2 \sqrt{b}  + \left[\frac{1-k_2}{a_2} - \left(A + c_1\right)c_2\right] b + \left[\frac{1}{2} A^2 c_2 + 2A \left(\frac{1-k_2}{a_2} + c_1 c_2\right)\right] b^{3/2}, \\
        Y_3^* &= c_3 \sqrt{b} + \left[\frac{1-k_3}{a_3} - \left(A + c_1 + c_2\right)c_3\right] b \notag \\
        &\quad + \left[\frac{1}{2} A^2 c_3 + 2A \left(\frac{1-k_3}{a_3} + (c_1 + c_2) c_3\right)\right] b^{3/2}, \\
        Y_4^* &= c_4 \sqrt{b} + \left[\frac{1-k_4}{a_4} - \left(A + c_1 + c_2 + c_3\right)c_4\right] b \notag \\
        &\quad + \left[\frac{1}{2} A^2 c_4 + 2A \left(\frac{1-k_4}{a_4} + (c_1 + c_2 + c_3) c_4\right)\right] b^{3/2}, \\ 
        Y_5^* &= 1 - 2A \sqrt{b} + (2A^2 - B) b - (A^3 + 3AB) b^{3/2}, \\
        Y_6^* &= c_1 \left\{ \sqrt{b} - Ab + \left[\frac{1}{2} A^2 - (1-c_1^2) + \sum_{i\neq 1} c_i^2 + c_1 \sum_{i\neq 1} c_i + \sum_{i<j, i \neq 1} c_i c_j \right] b^{3/2} \right\}, \\
        Y_7^* &= c_2 \left\{ \sqrt{b} - Ab + \left[\frac{1}{2} A^2 - (1-c_2^2) + \sum_{i\neq 2} c_i^2 + c_2 \sum_{i\neq 2} c_i + \sum_{i<j, i, j \neq 2} c_i c_j \right] b^{3/2} \right\}, \\
        Y_8^* &= c_3 \left\{ \sqrt{b} - Ab + \left[\frac{1}{2} A^2 - (1-c_3^2) + \sum_{i\neq 3} c_i^2 + c_3 \sum_{i\neq 3} c_i + \sum_{i<j, i, j \neq 3} c_i c_j \right] b^{3/2} \right\}, \\
        Y_9^* &= c_4 \left\{ \sqrt{b} - Ab + \left[\frac{1}{2} A^2 - (1-c_4^2) + \sum_{i\neq 4} c_i^2 + c_4 \sum_{i\neq 4} c_i + \sum_{i<j, i, j \neq 4} c_i c_j \right] b^{3/2} \right\}.
    \end{align} 
\end{subequations}

\subsection{Proof of Propositions}
The form of the partition strength in the general $n$-sector partition model during starvation (small $b$) has been shown to be 
\begin{subequations}
    \begin{align}
        \theta_1 &= 1 - A \sqrt{b} + \frac{1}{2} A^3 b^{3/2} + O(b^{5/2}), \\
        \theta_j &= \sqrt{\frac{k_{j-1}}{a_{j-1}}}\left[\sqrt{b} - \frac{1}{2} A^2 b^{3/2} + O(b^{5/2})\right], \quad (2 \le j \le n),
    \end{align}
\end{subequations}
where
\begin{equation}
    A = \sqrt{\frac{k_1}{a_1}} + \sqrt{\frac{k_2}{a_2}} + \cdots + \sqrt{\frac{k_n}{a_n}}.
\end{equation}

As suggested by the above calculation, we make the following claims about the form of the biomass fraction in the general $n$-sector partition model during starvation (small $b$) to one more order. 

\begin{proposition}[Expansion in the Starvation Limit]
    In the $ b \to 0 $ limit, the biomass fractions of the nonterminal nodes ($1 \leq j \leq n-1$) and the terminal nodes ($ n \leq j \leq 2n-1 $) are given by
    \begin{subequations}
        \begin{align}
            Y_j &= \frac{b\,\theta_1\,k_j}{a_j\theta_{j+1}} - \frac{b^2\,\theta_1^{2}\,k_j\,(k_j-1)}{a_j^{2}\theta_{j+1}^{2}}, \quad 1 \le j \le n-1, \\
            Y_j &= \theta_{j-n+1} \left[ 1 - b\theta_1\sum_{i=1}^{n-1}\frac{k_i}{a_i\theta_{i+1}} + b^{2}\,\theta_1^{2}\sum_{i=1}^{n-1}\frac{k_i(k_i-1)}{a_i^{2}\theta_{i+1}^{2}} \right], \quad n \le j \le 2n-1,
        \end{align}
    \end{subequations}
    up to second order in $ b $.
\end{proposition}

\begin{proof}
    The proof is a generalization of the one provided in the 20251029 - 20251102 summary. First, recall the expansion 
    \begin{equation}
        \frac{1}{2} \left[\sqrt{(z+u)^2 + 4k} - (z+u) \right] = \frac{k}{z} - \frac{ku}{z^2} + O(z^{-3}),
    \end{equation}
    when $ z $ is very large. In the $ b \to 0 $ limit, notice that the quantity $ \alpha_i = \frac{a_i \theta_{i+1}}{b \theta_1} \to \infty $.

    Again, since the $ n $-sector system consists of both \emph{terminal nodes} and \emph{nonterminal nodes}, whose biomass fraction expression have distinct forms, we will discuss them separately.
    \begin{enumerate}[(1)]
        \item Nonterminal nodes: These are the nodes with index $ 1 \leq i \leq n-1 $. Let 
        \begin{equation}
            \alpha_i = \frac{a_i \theta_{i+1}}{b \theta_1}, \quad u_i = k_i - 1 - \sum_{r=1}^{i-1} Y_r , 
        \end{equation}
        and write the nonterminal node biomass fraction as
        \begin{equation}
            \begin{split}
                Y_1 &= \frac{1}{2} \left[\sqrt{(\alpha_1 + k_1 - 1)^2 + 4k_1} - \left(\alpha_1 + k_1 - 1\right) \right] \\
                &= \frac{k_1}{\alpha_1} + \frac{k_1 (k_1 - 1)}{\alpha_1^2} + O(b^{2}) = \frac{b k_1 \theta_1}{a_1 \theta_2} + \frac{b^2 \theta_1^2 k_1 (k_1 - 1)}{a_1^2 } + O(b^{3}),
            \end{split}
        \end{equation}
        and
        \begin{equation}
            \begin{split}
                Y_i &= \frac{1}{2} \left[\sqrt{\left(\alpha_i + k_i - \left(1 - \sum_{r=1}^{i-1} Y_r\right)\right)^2 + 4 k_i} - \left(\alpha_i + k_i - \left(1 - \sum_{r=1}^{i-1} Y_r \right) \right)\right] \\
                &= \frac{k_i}{\alpha_i} + \frac{k_i u_i}{\alpha_i^2} + O(b^{2}) = \frac{b k_i \theta_{1}}{a_i \theta_{i+1}} + \frac{b^2 \theta_1^2 k_i (k_i - 1)}{a_i^2 \theta_{i+1}^2} + O(b^{3}),
            \end{split}
        \end{equation}
        for $ 2 \leq i \leq n-1 $. We used the fact that $ u_i = k_i - 1 - \sum_{r=1}^{i-1} Y_r = k_i - 1 + O(b) $.

        \item Terminal nodes: These are the nodes with index $ n \leq i \leq 2n-1 $.
        \begin{equation}
            Y_{n+1} = \frac{\theta_2}{\theta_1} Y_n, \quad Y_{n+2} = \frac{\theta_3}{\theta_1} Y_n, \quad \ldots, \quad Y_{2n-1} = \frac{\theta_n}{\theta_1} Y_n,
        \end{equation}
        subject to the condition $ \sum_{i=1}^n \theta_i = 1 $. Using $ \sum_{i=1}^{2n-1} Y_i = 1 $, we have  
        \begin{equation}
            Y_n + Y_{n+1} + \ldots + Y_{2n-1} = \left(1 + \frac{\theta_2}{\theta_1} + \ldots + \frac{\theta_n}{\theta_1}\right) Y_n = 1 - \sum_{i=1}^{n-1} Y_i, 
        \end{equation} 
        and hence
        \begin{equation}
            Y_n = \frac{\theta_1}{\theta_1 + \cdots + \theta_n} \left(1 - \sum_{i=1}^{n-1} Y_i\right) = \theta_1 \left(1 - \sum_{i=1}^{n-1} Y_i\right).
        \end{equation}
        Up to order $ O(b^2) $, plug in results for $ Y_i $, $ 1 \leq i \leq n-1 $, we have 
        \begin{equation}
            \begin{split}
                Y_n = \theta_1 \left[ 1 - b\theta_1\sum_{i=1}^{n-1}\frac{k_i}{a_i\theta_{i+1}} + b^{2}\,\theta_1^{2}\sum_{i=1}^{n-1}\frac{k_i(k_i-1)}{a_i^{2}\theta_{i+1}^{2}} \right] + O(b^3), \quad Y_j = \frac{\theta_{j-n+1}}{\theta_1} Y_n. 
            \end{split}
        \end{equation}
    \end{enumerate}
\end{proof}

\begin{proposition}[Optimized Biomass Fractions in the Starvation Limit]
    In the $ b \to 0 $ limit, the optimized biomass fractions of the nonterminal nodes ($1 \leq j \leq n-1$) and the terminal nodes ($ n \leq j \leq 2n-1 $) are given by
    \begin{subequations}
        \begin{align}
            Y_{1*} &= c_1 \sqrt{b} + \left(\frac{1-k_1}{a_1} - A c_1\right) b + \left[\frac{1}{2}A^{2} c_1 - 2A \left(\frac{1-k_1}{a_1}\right) \right] b^{3/2}, \\
            Y_{j*} &= c_j \sqrt{b} + \left[\frac{1-k_j}{a_j} - \left(A + \sum_{i=1}^{j-1} c_i \right) c_j \right]b + \left[\frac{1}{2}A^2 c_j + 2A \left(\frac{1-k_j}{a_j} + c_j \sum_{i=1}^{j-1} c_i \right)\right] b^{3/2}, \notag \\
            &\quad (2 \le j \le n-1), \\
            Y_{n*} &= 1 - 2A \sqrt{b} + \left(2A^{2}-B \right) b - (A^3 + 3AB) b^{3/2}, \\
            Y_{j+n *} &= c_j \left\{ \sqrt{b} - Ab + \left[\frac{1}{2} A^2 - (1- c_j^2) + \sum_{i\neq j}c_i^2 + c_j \sum_{i\neq j} c_i + \sum_{\substack{i,l \neq j, i<l}} c_i c_l \right] b^{3/2} \right\}, \notag\\
            &\quad (1 \le j \le n-1),   
        \end{align}
    \end{subequations}
    up to second order in $ b $. Here, $ c_j \equiv \sqrt{k_j / a_j} $ for $ 1 \leq j \leq n-1 $, and 
    \begin{equation}
        A \equiv \sum_{i=1}^{n-1} c_i, \quad B \equiv \sum_{i=1}^{n-1} \left(\frac{1-k_i}{a_i}\right) - \sum_{1 \leq i < j \leq n-1} c_i c_j.
    \end{equation}
\end{proposition}

\begin{proof}
It is convenient to introduce the notation $ s \equiv \sqrt{b} $ and to work in powers of $ s $ instead of $ b $, as then we are working with Taylor series only.

\begin{enumerate}[label=\textbf{\arabic*.}]
    \item Optimal partition: The optimal solution in the starvation limit satisfies
    \begin{equation}
        \theta_1^* = 1 - A s + \frac{1}{2} A^3 s^3 + O(s^5), \quad \theta_{2 \leq j \leq n}^* = c_{j-1} \left(s - \frac{1}{2} A^2 s^3 + O(s^5)\right).
    \end{equation}

    From these we obtain the expansions of the reciprocals:
    \begin{subequations}
        \begin{align}
            \frac{1}{\theta_{j+1}^*}
            &= \frac{1}{c_j}\,\left(s^{-1} + \frac{1}{2}A^2 s\right) + O(s^3), \\
            \frac{1}{(\theta_{j+1}^*)^2}
            &= \frac{1}{c_j^2}\,\left(s^{-2} + A^2  \right) + O(s^2),
        \end{align}
    \end{subequations}
    and likewise
    \begin{equation}
        \theta_1^* = 1 - A s + O(s^3), \qquad
        (\theta_1^*)^2 = 1 - 2A s + A^2 s^2 + O(s^3).
    \end{equation}

    \item Nonterminal nodes: For $1\le j\le n-1$, Proposition~1 gives
    \begin{equation}
        Y_j = \frac{b\,\theta_1 k_j}{a_j\theta_{j+1}}
        - \frac{b^2\,\theta_1^{2}k_j(k_j-1)}{a_j^{2}\theta_{j+1}^{2}}
        + O(b^3) 
        = \frac{\theta_1 k_j}{a_j\theta_{j+1}} s^2
        - \frac{(\theta_1)^2 k_j(k_j-1)}{a_j^{2}\theta_{j+1}^{2}}s^4
        + O(s^6).
    \end{equation}

    For the case $j=1$, we have $\theta_2^* = c_1(s - \frac{1}{2}A^2 s^3 + O(s^5))$, $ c_1^2 = k_1/a_1 $, and 
    \begin{equation}
        \frac{k_1(k_1-1)}{a_1^2} = c_1^2 \left(\frac{k_1-1}{a_1}\right).
    \end{equation}
    Using the expansions in \textbf{1.}, we obtain
    \begin{equation}
        \frac{b\,\theta_1^* k_1}{a_1\theta_2^*} 
        = c_1 \theta_1^*\,s + \frac{1}{2}A^2 c_1 \theta_1^* \, s^3 + O(s^4),
    \end{equation}
    so that, expanding $\theta_1^*$ to the relevant order,
    \[
        \frac{b\,\theta_1^* k_1}{a_1\theta_2^*}
        = c_1 s
        + \left(\frac{1-k_1}{a_1} - A c_1\right)s^2
        + \left(\frac{1}{2}A^2 c_1\right)s^3
        + O(s^4).
    \]
    The second term contributes at order $s^3$:
    \[
        \frac{b^2 (\theta_1^*)^2 k_1(k_1-1)}{a_1^2 (\theta_2^*)^2}
        = \frac{k_1(k_1-1)}{a_1^2 c_1^2} (\theta_1^*)^2 s^2 + O(s^4).
    \]
    Using $k_1/a_1 = c_1^2$, we have
    \(
        \frac{k_1(k_1-1)}{a_1^2c_1^2} = \frac{k_1-1}{a_1},
    \)
    and the $s^3$ term from $(\theta_1^*)^2$ gives
    \[
        -\,\frac{b^2 (\theta_1^*)^2 k_1(k_1-1)}{a_1^2 (\theta_2^*)^2}
        = -\,2A\,\frac{1-k_1}{a_1}\,s^3 + O(s^4).
    \]
    Adding the $O(s)$, $O(s^2)$, and $O(s^3)$ contributions together yields
    \[
        Y_1^* = c_1 s
        + \left(\frac{1-k_1}{a_1} - A c_1\right)s^2
        + \left[\frac{1}{2}A^{2} c_1
                - 2A\left(\frac{1-k_1}{a_1}\right)\right] s^3
        + O(s^4),
    \]
    which is exactly the claimed expansion for $Y_1^*$ once we revert to $s=\sqrt{b}$.

    For the remaining nonterminal nodes $2\le j\le n-1$, we have
    \[
        \theta_{j+1}^* = c_j\left(s - \tfrac{1}{2}A^2 s^3\right) + O(s^5),\quad
        \frac{1}{\theta_{j+1}^*}
        = \frac{1}{c_j}s^{-1} + \frac{1}{2}A^2 c_j^{-1} s + O(s^3),
    \]
    and using $ u_j = k_j-1-\sum_{r=1}^{j-1}Y_r $, $ Y_r = O(s) $, we have
    \[
        \frac{b\,\theta_1^* k_j}{a_j\theta_{j+1}^*}
        = c_j s
        + \left[\frac{1-k_j}{a_j}
            - \bigg(A + \sum_{i=1}^{j-1}c_i\bigg)c_j\right]s^2
        + \left(\frac{1}{2}A^2 c_j + 2A c_j\sum_{i=1}^{j-1}c_i\right) \, s^3
        + O(s^4),
    \]
    while the $b^2$--term contributes again an order--$s^3$ correction
    proportional to $\frac{k_j(k_j-1)}{a_j^2}$, which combines with
    the above to give
    \begin{equation}
        \begin{split}
            Y_j^* &= c_j s
            + \left[\frac{1-k_j}{a_j} - \bigg(A + \sum_{i=1}^{j-1}c_i\bigg)c_j\right]s^2 \\
            &+ \left[\frac{1}{2}A^2 c_j + 2A\left(\frac{1-k_j}{a_j} + c_j\sum_{i=1}^{j-1}c_i\right)\right]s^3 + O(s^4).
        \end{split}
    \end{equation}
    Rewriting in powers of $b$ gives the desired formula for $Y_j^*$.
    
    \item Terminal node $Y_n^*$: 
        By Proposition~1 we have, for the terminal root node,
    \[
        Y_n = \theta_1\left[
            1 - b\theta_1\sum_{i=1}^{n-1}\frac{k_i}{a_i\theta_{i+1}}
            + b^{2}\theta_1^{2}\sum_{i=1}^{n-1}
                \frac{k_i(k_i-1)}{a_i^{2}\theta_{i+1}^{2}}
        \right] + O(b^3).
    \]
    Define the two sums
    \[
        S_1(b) \equiv \sum_{i=1}^{n-1}\frac{k_i}{a_i\theta_{i+1}},\qquad
        S_2(b) \equiv \sum_{i=1}^{n-1}
                \frac{k_i(k_i-1)}{a_i^{2}\theta_{i+1}^{2}}.
    \]
    Using the expansions from Step~1, we find
    \begin{align*}
        \frac{k_i}{a_i\theta_{i+1}^*}
        &= \frac{c_i^2}{c_i s}\left(1+\frac{1}{2}A^2 s^2\right)
        + O(s^3)
        = c_i s^{-1} + \frac{1}{2}A^2 c_i s + O(s^3),\\
        \frac{k_i(k_i-1)}{a_i^{2}(\theta_{i+1}^*)^{2}}
        &= \frac{k_i(k_i-1)}{a_i^{2}}
        \left(\frac{1}{c_i^2}s^{-2} + A^2 c_i^{-2} + O(s^2)\right)\\
        &= \frac{k_i-1}{a_i}s^{-2}
        + A^2\frac{k_i-1}{a_i} + O(s^2),
    \end{align*}
    so that
    \begin{align*}
        S_1(b)
        &= A\,s^{-1} + \frac{1}{2}A^3 s + O(s^3),\\
        S_2(b)
        &= \left(\sum_{i=1}^{n-1}\frac{k_i-1}{a_i}\right)s^{-2}
        + A^2\sum_{i=1}^{n-1}\frac{k_i-1}{a_i} + O(s^2).
    \end{align*}
    A short but straightforward calculation using
    $\theta_1^* = 1 - A s + \frac{1}{2} A^3 s^3 + O(s^5)$ then yields
    \[
        Y_n^*
        = 1 - 2A s + (2A^2 - B) s^2
        - (A^3 + 3AB) s^3 + O(s^4),
    \]
    where the combination $B$ is precisely
    \[
        B = \sum_{i=1}^{n-1}\frac{1-k_i}{a_i}
            - \sum_{1\le i<j\le n-1} c_i c_j,
    \]
    arising from the $s^2$ and $s^3$ coefficients in the product
    $\theta_1^*[1 - s^2\theta_1^* S_1(b) + s^4(\theta_1^*)^2 S_2(b)]$.
    Rewriting again in powers of $b$ gives the stated form of $Y_n^*$.

    \item Remaining terminal nodes $Y_{n+j}^*$: 
    For $1\le j\le n-1$, Proposition~1 gives
    \[
        Y_{n+j} = \frac{\theta_{j+1}}{\theta_1}\,Y_n.
    \]
    Hence, under the optimal partition,
    \[
        Y_{n+j}^*
        = \frac{\theta_{j+1}^*}{\theta_1^*}\,Y_n^*.
    \]
    Using the expansions from Step~1,
    \[
        \frac{\theta_{j+1}^*}{\theta_1^*}
        = c_j\left\{
            s - A s^2
            + \left[\tfrac{1}{2}A^2
                    - (1-c_j^2)
                    + \sum_{i\ne j} c_i^2
                    + c_j\sum_{i\ne j}c_i
                    + \!\!\sum_{\substack{i,\ell\ne j\\ i<\ell}}\!c_i c_\ell\right]s^3
            + O(s^4)
        \right\},
    \]
    and multiplying by the expression for $Y_n^*$ from Step~3, we obtain
    \[
        Y_{n+j}^*
        = c_j\left\{
            s - A s^2
            + \left[\tfrac{1}{2}A^2
                    - (1-c_j^2)
                    + \sum_{i\ne j} c_i^2
                    + c_j\sum_{i\ne j}c_i
                    + \!\!\sum_{\substack{i,\ell\ne j\\ i<\ell}}\!c_i c_\ell\right]s^3
            \right\}
        + O(s^4),
    \]
    which, upon replacing $s$ by $\sqrt{b}$, is exactly the form claimed for
    $Y_{j+n}^*$ in the proposition.
\end{enumerate}

Collecting the results of \textbf{2.} - \textbf{4.}, we prove the formulas of the proposition.
\end{proof}

\subsection{Waiting Time Analysis}
It is insightful to analyze the waiting time at each sector in the starvation limit. The analysis will provide a biological interpretation of the optimized proteome partitioning strategy. In the picture of ODEs, the optimality condition can be viewed as the solution that minimizes the waiting time of the system. In the starvation limit, we can write down the leading order waiting time explicitly as an expression of the model parameters.

For the $ i $-th non-terminal node, the waiting time is 
\begin{equation}
    \frac{1}{\tau_i} = \theta_{i+1} \frac{a_i Y_i}{k_i + Y_i}, \quad 1 \leq i \leq n-1. 
\end{equation}
In the starvation limit, $ Y_i \ll k_i $, and we have
\begin{equation}
    \frac{1}{\tau_i} = \theta_{i+1} \frac{a_i Y_i}{k_i}, \quad 1 \leq i \leq n-1,
\end{equation}
and thus the total waiting time $ T $ as $ b \to 0 $ is
\begin{equation}
    T = \frac{1}{\theta_2} \frac{k_1}{a_1 Y_1} + \frac{1}{\theta_3} \frac{k_2}{a_2 Y_2} + \cdots + \frac{1}{\theta_n} \frac{k_{n-1}}{a_{n-1} Y_{n-1}}.
\end{equation}

To further analyze this equation, let's invoke the following assumption: \emph{The product $ Y_i \theta_{i+1} $ is independent of $ Y_j \theta_{j+1} $ for all $ j \neq i $.} The assumption can be justified by the fact that, during starvation, the flux through each node is approximately balanced, so exchanging any two nodes will not change the growth rate, and hence the waiting time. Under this assumption, since each term in the total waiting time is positive, the arithmetic-geometric inequality tells us that the total waiting time is minimized when each term is equal, i.e.,
\begin{equation}
    \theta_2 \frac{a_1 Y_1}{k_1} = \theta_3 \frac{a_2 Y_2}{k_2} = \cdots = \theta_n \frac{a_{n-1} Y_{n-1}}{k_{n-1}}.
\end{equation}

The minimum total waiting time is then
\begin{equation}
    T = \sum_{i=1}^{n-1} \frac{1}{\theta_{i+1}} \frac{k_i}{a_i Y_i} \geq (n-1) \left( \prod_{i=1}^{n-1} \frac{1}{\theta_{i+1}} \frac{k_i}{a_i Y_i} \right)^{\frac{1}{n-1}} = (n-1) \frac{1}{\theta_2} \frac{a_1 Y_1}{k_1}.
\end{equation}  
The optimized total waiting time should satisfy $ T^* = (n-1) / \theta_1 b $ (?), so 
\begin{equation}
    T \geq (n-1) \frac{1}{\theta_2} \frac{a_1 Y_1}{k_1} = T^* = \frac{n-1}{\theta_1 b },
\end{equation}
and rearranging gives
\begin{equation}
    Y_i = \left(\frac{k_i \theta_1}{a_i \theta_{i+1}}\right) b, \quad 1 \leq i \leq n-1,
\end{equation}
which agrees with the rigorous derivation shown earlier.

On the other hand, the results for the nutrient overabundance regime is less straightforward. The waiting time expression is still valid, but the assumption that $ Y_i \theta_{i+1} $ are independent no longer holds. Since the fluxes are no longer balanced, exchanging two nodes will change the growth rate, and hence the waiting time, as our theory shows. 


\section{System Under Overabundance of Nutrients}
The overabundance limit corresponds to the case where $ b \to \infty $, or, more precisely, $ b / a_i \to \infty $ for all $ i $. In this limit, the optimized proteome partition strengths have been derived for the first time, and verified with numerical simulations. For completeness, numerical check for the starvation limit ($ b \to 0 $) is also provided.

\subsection{Perturbation Analysis}
We follow the previous calculations and carry out the asymptotic expansion in $ \frac{1}{b} $. The equations () are given in the large $ b $ limit by 

\subsubsection{Trajectories and Growth Rate}
Large supply or demand limits yield corner solutions. In optimal resource allocation (e.g. flux-balance analysis), an unconstrained input leads to one pathway dominating flux. in a high-$b$ regime the metabolic sector $Y_4$ can hardly operate, while the ribosomal/translational sector dominates. Biologically, this implies nearly all proteome is allocated for ribosomes and translational molecules, shutting down metabolic enzyme production. 

The partition strength is given here up to order $O(b^{-3})$:
\begin{subequations}
    \begin{align}
        \theta_1 &= \frac{1}{(2\alpha + \beta\alpha^2)b} + \frac{1}{(2\alpha + \beta\alpha^2)^2b^2} + O(b^{-3}),\\
        \theta_2 &= \frac{\alpha}{2\alpha + \beta\alpha^2} - \frac{\alpha}{(2\alpha + \beta\alpha^2)^2b} + \frac{\alpha}{(2\alpha + \beta\alpha^2)^3 b^2} + O(b^{-3}),\\
        \theta_3 &= \frac{\alpha + \beta\alpha^2}{2\alpha + \beta\alpha^2} - \frac{\alpha + \beta\alpha^2}{(2\alpha + \beta\alpha^2)^2b} + \frac{\alpha + \beta \alpha^2}{(2\alpha + \beta\alpha^2)^3 b^2} + O(b^{-3}),\\
        \alpha &= \frac{k_2^2(1+k_1)^4a_2}{(1+k_1+k_1k_2)^2a_1^2 - k_2(1+k_1)^2a_1a_2 + (1+k_1)^4(1+k_2)a_2^2}, \\
        \beta &= \frac{(1+k_1+k_1k_2)a_1 + (1+k_1)^2a_2}{(1+k_1)^2k_2}.
    \end{align}
\end{subequations}

Note that $\theta_1 + \theta_2 + \theta_3 = 1$ up to order $ O(b^{-2}) $. Furthermore, we see that $ \theta_1 = O(b^{-1}) $, thus making $ b \theta_1 = O(1) $. So, in the high nutrient level limit, the influx is kept constant, while constant resources are allocated to metabolism and ribosome production. To derive this, consider the asymptotic expansion of $Y_1, Y_2, Y_3$ in the limit $b \to \infty$.

\begin{subequations}
    \begin{align}
        Y_1 &= 1 - \frac{1}{1+k_1}\left(\frac{a_1\theta_2}{b\theta_1}\right) + \frac{k_1}{(1+k_1)^3}\left(\frac{a_1\theta_2}{b\theta_1}\right)^2, \\
        Y_2 &= \frac{1}{1+k_1}\left(\frac{a_1\theta_2}{b\theta_1}\right) - \frac{1}{(1+k_1)^3}\left[k_1 + \frac{(1+k_1)^2a_2\theta_3}{k_2a_1\theta_2}\right]\left(\frac{a_1\theta_2}{b\theta_1}\right)^2 \\
        &+ \Bigg[\frac{a_1^{3}\theta_2^{3}\,k_1\,(k_1-1)}{\theta_1^{3}(k_1+1)^5}+\frac{a_1^{2}a_2\theta_2^{2}\theta_3\,k_1}{\theta_1^{3}k_2\,(k_1+1)^3}+\frac{a_1^{2}a_2\theta_2^{2}\theta_3}{\theta_1^{3}k_2^{2}(k_1+1)^2}+\frac{a_1a_2^{2}\theta_2\theta_3^{2}}{\theta_1^{3}k_2^{2}(k_1+1)}\Bigg]\frac{1}{b^{3}}, \\
        Y_3 &= \frac{a_1 a_2}{(1+k_1)k_2 b^2}\left(\frac{\theta_2\theta_3}{\theta_1}\right) - \frac{a_1 a_2}{(1+k_1)^3 k_2^2}\left(\frac{\theta_2\theta_3}{\theta_1^2}\right) \left[(1+k_1)^2a_2\theta_3 + (1+k_1+k_1k_2)a_1\theta_2\right]\frac{1}{b^3}, \\
        Y_4 &= \frac{a_1 a_2}{(1+k_1)k_2 b^2}\left(\frac{\theta_2^2 \theta_3}{\theta_1^2}\right) - \frac{a_1 a_2}{(1+k_1)^3 k_2^2}\left(\frac{\theta_2^2 \theta_3}{\theta_1^3}\right) \left[(1+k_1)^2a_2\theta_3 + (1+k_1+k_1k_2)a_1\theta_2\right]\frac{1}{b^3}, \\
        Y_5 &= \frac{a_1 a_2}{(1+k_1)k_2 b^2}\left(\frac{\theta_2\theta_3^2}{\theta_1^2}\right) - \frac{a_1 a_2}{(1+k_1)^3 k_2^2}\left(\frac{\theta_2\theta_3^2}{\theta_1^3}\right) \left[(1+k_1)^2a_2\theta_3 + (1+k_1+k_1k_2)a_1\theta_2\right]\frac{1}{b^3}.
    \end{align}
\end{subequations}

The growth rate is
\begin{equation}
    \lambda = \frac{a_1 a_2}{(1+k_1)k_2 b}\left(\frac{\theta_2\theta_3}{\theta_1}\right) - \frac{a_1 a_2}{(1+k_1)^3k_2^2b^2}\left(\frac{\theta_2\theta_3}{\theta_1^2}\right) \left[(1+k_1)^2a_2\theta_3 + (1+k_1+k_1k_2)a_1\theta_2\right].
\end{equation}

Again by the method of Lagrange multipliers, set the Lagrangian to $L(\theta) = \lambda(\theta) - \mu(\theta_1 + \theta_2 + \theta_3 - 1)$ with Lagrange multiplier $\mu$. Then 
\begin{subequations}
    \begin{align}
        \frac{\partial L}{\partial \theta_1} &= -\frac{a_1a_2\theta_2\theta_3}{(1+k_1)k_2\theta_1^2b} + \frac{2a_1a_2\theta_2\theta_3\left[(1+k_1+k_1k_2)a_1\theta_2 + (1+k_1)^2a_2\theta_3\right]}{(1+k_1)^3k_2^2\theta_1^3b^2} - \mu, \label{equ:lagrange_high1}\\ 
        \frac{\partial L}{\partial\theta_2} &= \frac{a_1a_2\theta_3}{(1+k_1)k_2\theta_1 b} - \frac{a_1a_2\theta_3\left[2(1+k_1+k_1k_2)a_1\theta_2 + (1+k_1)^2a_2\theta_3\right]}{(1+k_1)^3k_2^2\theta_1^2b^2} - \mu, \label{equ:lagrange_high2}\\
        \frac{\partial L}{\partial\theta_3} &= \frac{a_1a_2\theta_2}{(1+k_1)k_2\theta_1 b} - \frac{a_1a_2\theta_2\left[(1+k_1+k_1k_2)a_1\theta_2 + 2(1+k_1)^2a_2\theta_3\right]}{(1+k_1)^3k_2^2\theta_1^2b^2} - \mu. \label{equ:lagrange_high3}
    \end{align}
\end{subequations}

Let $p = \theta_3/\theta_1$, $q = \theta_2/\theta_1$, $P \equiv r/b, Q = q/b$. From equations (\ref{equ:lagrange_high2}) and (\ref{equ:lagrange_high3}), up to order $O(b^{-2})$ we have 
\begin{align}
    &\frac{a_1 a_2^2}{(1+k_1)k_2^2}P^2 + \left[\frac{a_1a_2}{(1+k_1)k_2} - \frac{2a_1a_2\left((1+k_1+k_1k_2)a_1 + (1+k_1)^2a_2\right)}{(1+k_1)^3k_2^2}Q\right]P \\
    &\quad + \left[\frac{a_1^2a_2(1+k_1+k_1k_2)}{(1+k_1)^3k_2^2}Q^2 - \frac{a_1a_2}{(1+k_1)k_2}Q\right] = 0.
\end{align}
Solve for $P$ and keep terms up to $O(b^{-2})$, i.e. $Q^2$.
\begin{align}
    P &= Q + \frac{(1+k_1+k_1k_2)a_1 + (1+k_1)^2a_2}{(1+k_1)^2k_2}Q^2 + O(Q^3) \notag \\
    &\equiv Q + \beta Q^2 + O(Q^3).\label{equ:lagrange_high4}
\end{align}
Notice that when $Q$ is small, this reduces to the previous (unphysical) result. However, the coefficient of $Q^2$ is approximately $\sim 40000$ for $a_1 = 23.8, k_1 = 0.1, a_2 = 1.42, k_2 = 0.003$, while $Q \sim 0.01/b$, so the expression is reasonable. Plug equation (\ref{equ:lagrange_high4}) into equation (\ref{equ:lagrange_high1}) and use equation (\ref{equ:lagrange_high2}), we have
\begin{align}
    &\left[\frac{a_1a_2}{(1+k_1)k_2} - \frac{2a_1^2a_2(1+k_1+k_1k_2)}{(1+k_1+k_1k_2)^3k_2^2}  - \frac{a_1a_2^2}{(1+k_1)k_2} \right] \left[Q + \frac{(1+k_1+k_1k_2)a_1 + (1+k_1)^2 a_2}{(1+k_1)^2k_2}Q^2\right] \\
    &= \frac{a_1a_2}{(1+k_1)k_2}Q - \frac{a_1^2 a_2(1+k_1+k_1k_2)}{(1+k_1)^3k_2}Q^2 - \frac{2a_1a_2^2}{(1+k_1)k_2}Q\left[Q + \frac{(1+k_1+k_1k_2)a_1 + (1+k_1)^2a_2}{(1+k_1)^2k_2}Q^2\right].
\end{align}
Leading factor $O(Q)$ cancels out, and dividing by $Q$ gives
\begin{equation}
    Q = \frac{k_2^2(1+k_1)^4a_2}{(1+k_1+k_1k_2)^2a_1^2 - k_2(1+k_1)^2a_1a_2 + (1+k_1)^4(1+k_2)a_2^2} \equiv \alpha. \label{equ:Q}
\end{equation}
By equation (\ref{equ:lagrange_high4}), 
\begin{equation}
    P = \alpha + \beta\alpha^2 = O(1), \label{equ:P}
\end{equation}
so both $\theta_2/\theta_1$ and $\theta_3/\theta_1$ are $O(b)$. To determine the specific dependencies, solve for $\theta_1$ with
\begin{equation}
    \theta_1 = \left(1 + \frac{\theta_2}{\theta_1} + \frac{\theta_3}{\theta_1}\right)^{-1} = \frac{1}{(2\alpha + \beta\alpha^2)b} - \frac{1}{(2\alpha + \beta\alpha^2)^2b^2} + O(b^{-3}).
\end{equation}
up to leading order. Then $\theta_2$ and $\theta_3$ can be derived from equations (\ref{equ:P}) and (\ref{equ:Q}).
*

\subsection{Mass Action Kinetics and the Low Metabolite Limit}
A clean solution may be obtained up to two leading terms for mass action kinetics. We will start with the steady state equations, which are given by 
\begin{subequations}
    \begin{align}
        Y_1 &= \left( 1 + \frac{r_1 \theta_{2}}{b \theta_{1}} \right)^{-1}, \\
        Y_2 &= \left(\frac{r_1 \theta_{2}}{b \theta_{1}}\right) \left[ \left(1 + \frac{r_2 \theta_{3}}{b \theta_{1}} \right)\left( 1 + \frac{r_1 \theta_{2}}{b \theta_{1}} \right) \right] ^{-1}, \\
        Y_3 &= \left(\frac{r_{2}\theta_{3}}{b}Y_2 \right), \quad Y_4 = \left(\frac{r_2 \theta_2 \theta_3 }{b \theta_1} \right) Y_2, \quad Y_5 = \left(\frac{r_2 \theta_3^2 }{b \theta_1} \right) Y_2.
    \end{align}
\end{subequations}

In the $ b \to \infty $ limit, we can carry out the asymptotic expansion in $ \frac{1}{b} $ and obtain
\begin{subequations}
    \begin{align}
        Y_1 &= 1 - \frac{r_1 \theta_2}{b \theta_1} + \left(\frac{r_1 \theta_2}{b \theta_1}\right)^2 + O\left(b^{-3}\right), \\
        Y_2 &= \frac{r_1 \theta_2}{b \theta_1} - \left(\frac{r_1 \theta_2}{b \theta_1}\right)^2 \left(1 + \frac{r_2 \theta_3}{b \theta_1}\right) + O\left(b^{-3}\right), \\
        Y_3 &= \frac{r_1 r_2 \theta_2 \theta_3}{b^2 \theta_1^2} + O\left(b^{-3}\right), \quad Y_4 = \frac{\theta_2}{\theta_1} Y_3, \quad Y_5 = \frac{\theta_3}{\theta_1} Y_3.
    \end{align}
\end{subequations}

Let the objective function in this case be given by $ L(\theta, \mu) = bY_3 - \mu (\theta_1 + \theta_2 + \theta_3 - 1) $. The optimality conditions $ \frac{\partial L}{\partial \theta_i} = 0 $ lead to the solution
\begin{subequations}
    \begin{align}
        \frac{\partial L}{\partial \theta_1} &= - \frac{r_1 r_2 \theta_2 \theta_3}{\theta_1^2 b} + \frac{2r_1 r_2 \theta_2 \theta_3 (r_1 l_2 + r_2 \theta_3)}{\theta_1^2 b^2} - \mu = 0, \\
        \frac{\partial L}{\partial \theta_2} &= \frac{r_1 r_2 \theta_3}{\theta_1 b} - \frac{r_1 r_2 \theta_3 (2r_1 \theta_2 + r_2 \theta_3)}{\theta_1^2 b^2} - \mu = 0, \\
        \frac{\partial L}{\partial \theta_3} &= \frac{r_1 r_2 \theta_2}{\theta_1 b} - \frac{r_1 r_2 \theta_2 (r_1 \theta_2 + 2 r_2 \theta_3)}{\theta_1^2 b^2} - \mu = 0.
    \end{align}
\end{subequations}

As in the Michaelis-Menten case, keeping the leading order leads to unphysical results where $ \theta_2 = \theta_3 \to \frac{1}{2} $. Let's analyze the solution by including the next order terms. Define 
\begin{equation}
    p = \frac{\theta_2}{b \theta_1}, \qquad 
    q = \frac{\theta_3}{b \theta_1}, \qquad 
    P = r_1 p, \qquad 
    Q = r_2 q, \qquad
    \mu^{\prime} = \frac{\mu}{r_1 r_2},
\end{equation}
Then, the optimality conditions become
\begin{subequations}
    \begin{align}
        - bPQ \left[1 - 2 (r_1 P + r_2 Q)\right] &= \mu^{\prime}, \\
        Q \left[1 - (2r_1 P + r_2 Q)\right] &= \mu^{\prime} , \\
        P \left[1 - (r_1 P + r_2 Q)\right] &= \mu^{\prime}.
    \end{align}
\end{subequations}

When $ b \to \infty $, the quantities $ P, Q \to 0 $. In fact, we can further define $ U = r_1 P $, $ V = r_2 Q $, $ S = U + V $, and find that $ U, V \to 0 $ as $ b \to \infty $ as well. The optimality conditions become
\begin{equation}
    \label{eq:opt_cond_overabundance}
    \frac{\mu}{r_1 r_2} = - b \frac{UV}{r_1 r_2} (1 - 2S) = \frac{V}{r_2} (1 - 2U - V) = \frac{U}{r_1} (1 - U - 2V). 
\end{equation}

\begin{lemma}
    The Lagrange multiplier $ \mu^{\prime} = O(b^0) $.
\end{lemma}
\begin{proof}
    From the third equality, we have $ \mu^{\prime} = \frac{U}{r_1}(1-S) \geq 0 $. But also from this equality, we have 
    \begin{equation}
        \mu^{\prime} = \frac{U}{r_1} (1-S) \leq \frac{1}{r_1}S(1-S) \leq \frac{1}{4r_1},
    \end{equation}
    hence $ 0 \leq \mu^{\prime} (b) \leq \frac{1}{4 r_1} $. Taking the limit $ b\to \infty $ proves the claim. 
\end{proof}

By our claim and the first equality, $ bUV(1-2S) = O(b^0) $. If $ U, V = O\left(b^{m > 0}\right) $, then $ S = O\left(b^{m > 0}\right) $ also, contradicting the fact that $ \mu^\prime = O(b^0) $. Therefore, $ S = O(b^{m \leq 0}) $. It must be that $ U, V, S = O(b^0) $, and $ 1-2S = O(b^0) $. Hence, the leading order terms $ U_0, V_0, S_0 $ satisfy 
\begin{equation}
    S_0 = \frac{1}{2} = U_0 + V_0.
\end{equation}

\begin{enumerate}[(1)]
    \item $ b^0 $: By the second equality of equation~(\ref{eq:opt_cond_overabundance}), we have 
    \begin{equation}
        \frac{U_0}{r_1} (1 - 2U_0 - V_0) = \frac{V_0}{r_2} (1 - U_0 - V_0) \implies 2 r_1 V_0^2 = r_2 U_0.
    \end{equation}
    Along with the condition $ U_0 + V_0 = \frac{1}{2} $, we can solve for $ U_0 $ and $ V_0 $ as 
    \begin{equation}
        U_0 = \frac{2r_1 + r_2 - \sqrt{r_2^2 + 4r_1 r_2}}{4r_1} \quad V_0 = \frac{-r_2 + \sqrt{r_2^2 + 4r_1 r_2}}{4 r_1}.
    \end{equation} 

    \item $ b^1 $: To find the next order terms $ U_1, V_1 $, we can write $ S = S_0 + \varepsilon $, where $ \varepsilon = \varepsilon (b) = O\left(b^{m<0}\right) $ is a small pertubation. Then, we have 
    \begin{equation}
        U = U_0 + U_1 \varepsilon , \quad V = V_0 + V_1 \varepsilon,
    \end{equation}
    such that $ U_0 + V_0 = S_0 = \frac{1}{2} $ and $ U_1 + V_1 = S_1 $, and 
    \begin{equation}
        \frac{\mathrm{d}U}{\mathrm{d}S} = U_1, \quad \frac{\mathrm{d}V}{\mathrm{d}S} = V_1 \implies U_1 + V_1 = 1.
    \end{equation}
    
    Substitute $ U $ and $ V $ into equation~(\ref{eq:opt_cond_overabundance}), we have
    \begin{equation}
        \frac{U}{2 r_1} = \frac{2b}{r_1 r_2} UV \varepsilon \implies \varepsilon = \frac{r_2}{4 V_0 b} = O(b^{-1}).
    \end{equation}

    Next, we use equation~(\ref{eq:opt_cond_overabundance}) to define the function $ f: \mathbb{R}^2 \to \mathbb{R} $, given by  
    \begin{equation}
        f(U,V) = r_1 V (1 - 2U - V) - r_2 U (1 - U - 2V).
    \end{equation}
    Consider the set $ \mathcal{S} = \{ U, V \in \mathbb{R} \mid f(U,V) = 0, U+V=S \} $. We will differentiate the system at $ S=\frac{1}{2} $. Then we have
    \begin{subequations}
        \begin{align}
            U_1 &= \left.\frac{\partial_V f}{\partial_V f - \partial_U f}\right|_{\varepsilon=0} = \frac{r_2 U_0}{r_2 U_0 + (2r_1 + r_2)V_0}, \\
            V_1 &= 1 - U_1 = \frac{(2r_1 + r_2)V_0}{r_2 U_0 + (2r_1 + r_2)V_0}.
        \end{align}
    \end{subequations}

    To simplify the expressions, further define 
    \begin{equation}
        D_0 = \frac{U_0}{r_1} + \frac{V_0}{r_2}, \quad D_1 = \frac{U_1}{r_1} + \frac{V_1}{r_2}, \quad \kappa = \frac{r_2}{4 V_0}
    \end{equation}
    Then, the partition strengths up to $ O(b^{-1}) $ are symbolically given by 
    \begin{equation}
        \theta_1 = \dfrac{1}{1 + b \left(\frac{U}{r_1} + \frac{V}{r_2}\right)}, \quad \theta_2 = \dfrac{\frac{b U}{r_1}}{1 + b \left(\frac{U}{r_1} + \frac{V}{r_2}\right)}, \quad \theta_3 = \dfrac{\frac{b V}{r_2}}{1 + b \left(\frac{U}{r_1} + \frac{V}{r_2}\right)}.
    \end{equation}
    Expand up to $ O(\varepsilon) $ gives
    \begin{subequations}
        \begin{align}
            \theta_1 &= \frac{1}{b D_0} - \frac{1 + \kappa D_1}{b^2 D_0^2} + O\left(b^{-3}\right), \\
            \theta_2 &= \frac{U_0}{r_1 D_0} + \left[\frac{\kappa U_1}{r_1 D_0} - \frac{U_0 (1 + \kappa D_1)}{r_1 D_0^2}\right]\frac{1}{b} + O\left(b^{-2}\right), \\
            \theta_3 &= \frac{V_0}{r_2 D_0} + \left[\frac{\kappa V_1}{r_2 D_0} - \frac{V_0 (1 + \kappa D_1)}{r_2 D_0^2}\right]\frac{1}{b} + O\left(b^{-2}\right).
        \end{align}
    \end{subequations}
\end{enumerate}

Up to leading order, the transporter fraction $ \theta_1 \to 0 $, while the enzyme and R-protein fractions approach constants determined by the reaction rates $ r_1, r_2 $:
\begin{equation}
    \frac{\theta_3}{\theta_2} = \frac{V_0 r_1}{U_0 r_2} = \frac{r_1}{r_2} \left(\frac{2r_1 + r_2 - \sqrt{r_2^2 + 4r_1 r_2}}{-r_2 + \sqrt{r_2^2 + 4r_1 r_2}}\right). 
\end{equation}

\begin{example}[Two metabolic steps have the same reaction rate]
    To illustrate the solution, let's consider a simpler case where $ r_1 = r_2 = r $. Then, the leading order terms become
    \begin{equation}
        U_0 = \frac{3 - \sqrt{5}}{4} = \frac{1}{2 \varphi^2}, \quad V_0 = \frac{-1 + \sqrt{5}}{4} = \frac{1}{2 \varphi},
    \end{equation}
    where $ \varphi = \frac{1 + \sqrt{5}}{2} $ is the golden ratio. Here we recall a few properties of the golden ratio:
    \begin{itemize}
        \item $ \varphi \approx 1.61803398875 $ and $ - \frac{1}{\varphi} = 1 - \varphi \approx -0.61803398875 $ are the roots of $ x^2 - x - 1 $. 
        \item Reduction of order: $ \varphi^2 = \varphi + 1 $. 
        \item Reciprocal: $ \frac{1}{\varphi} = \varphi - 1 $.
    \end{itemize}

    Next, we compute 
    \begin{equation}
        D_0 = \frac{1}{r} (U_0 + V_0) = \frac{1}{2r}, \quad D_1 = \frac{1}{r} (U_1 + V_1) = \frac{1}{r}, \quad \kappa = \frac{r}{4 V_0} = \frac{r \varphi}{2}.
    \end{equation}
    Therefore, after some simplification and using the above properties, the optimal partition fractions are found to be
    \begin{subequations}
        \begin{align}
            \theta_1 &= \frac{2r}{b} + \frac{2r^2 (2 + \varphi)}{b^2} + O\left(b^{-3}\right), \\
            \theta_2 &= \frac{1}{\varphi^2} - \left(\frac{4 \varphi}{1+ 3 \varphi}\right) \frac{r}{b} + O\left(b^{-2}\right), \\
            \theta_3 &= \frac{1}{\varphi} - \left(\frac{2 + 2\varphi}{1 + 3 \varphi}\right) \frac{r}{b} + O\left(b^{-2}\right).
        \end{align}
    \end{subequations}

    To leading order, the transporter partition fraction $ \theta_1 \to 0 $, while the enzyme and R-protein fractions approach constants determined by the golden ratio:
    \begin{equation}
        \theta_2 \to \frac{1}{\varphi^2} \approx 0.381966, \quad \theta_3 \to \frac{1}{\varphi} \approx 0.618034,
    \end{equation}
    while $ \theta_3/\theta _2 \to \varphi = 1.1.61803 $ approaches the golden ratio. 
\end{example}

\medskip

\begin{example}[Bottleneck reaction in the overabundance limit]
    We can analyze the effect of a bottleneck reaction in the overabundance limit. If the first step is extremely slow, i.e $ r_1 \ll r_2 $, then, from the leading order solution, we have
    \begin{equation}
        U_0 \approx \frac{1}{2}, \quad V_0 \approx \frac{r_1}{4 r_2} \ll U_0.
    \end{equation}
    Therefore, the optimal partition fractions become
    \begin{equation}
        \theta_1 \approx \frac{2 r_1}{b}, \quad \theta_2 \approx 1 - \frac{r_1}{2 r_2}, \quad \theta_3 \approx \frac{r_1}{2 r_2}.
    \end{equation}
    In this case, most of the proteome is allocated to the enzyme of the bottleneck reaction (first step), while very little is allocated to the downstream enzyme and transporter, as we expect. If the second step is the bottleneck instead, i.e., $ r_2 \ll r_1 $, then
    \begin{equation}
        U_0 \approx \frac{r_2}{4 r_1} \ll V_0 \approx \frac{1}{2},
    \end{equation}
    and the optimal partition fractions become
    \begin{equation}
        \theta_1 \approx \frac{2 r_2}{b}, \quad \theta_2 \approx \frac{r_2}{2 r_1}, \quad \theta_3 \approx 1 - \frac{r_2}{2 r_1}.
    \end{equation}
    In this case, most of the proteome is allocated to the enzyme of the bottleneck reaction (second step), while very little is allocated to the upstream enzyme and transporter. This is because the bottleneck reaction limits the overall flux, so it is more economic to allocatie more resources to the bottleneck step in order to process the accumulated biomass. 
\end{example}

\section{Arbitrarily Long Metabolic Chain Under Mass Action Kinetics}
For completeness, we consider an arbitrarily long metabolic chain of $ n $ steps under mass action kinetics, where the network topology is the same as the one given in Figure~\ref{fig:general_proteome_partition_model}. The steady state equations are given by


% Numerical Methods
\section{Numerical Methods}
\subsection{Simulation Parameters}
Here is a detailed estimate of the average number of total molecules involved in the synthesis of a protein in an \textit{E. coli} cell. The molecules will include both small molecules such as metabolic intermediates and amino acids, and large molecules such as proteins and other polymeric compounds. 

For a ribosome to synthesize a protein, it requires the following ingredients: amino acids ($\approx 350$), tRNA ($\approx 350$), ATP/GTP $\approx 1050$, ribosomes ($\approx 350$). The average protein length in E. coli is 350 amino acids. Although ribosomes are not consumed, each protein needs one ribosome to synthesize. The above is summarized in table \ref{tab:protein_makeup}, and we find that on average about $2100$ molecules are involved (though not necessarily \textit{consumed}).

\begin{table}[ht]
\centering
\caption{Estimated counts of total molecules involved in the synthesis of a protein on average in a typical \textit{E. coli} Cell}
\begin{tabularx}{\textwidth}{
    >{\raggedright\arraybackslash}p{0.25\textwidth}
    >{\raggedright\arraybackslash}p{0.15\textwidth}
    >{\raggedright\arraybackslash}X
    }
    \hline
    \textbf{Molecule type} & \textbf{Count} & \textbf{Notes} \\
    \hline
    Amino acids ($ N_{\text{AA}} $) & $ \approx 350 $ & The average protein length is $ 350 $ amino acids, while the longest is a helicase encoded by the \textit{lhr} gene, comprising $1,538$ amino acids. \\
    \hline
    tRNA molecules ($ N_{\text{tRNA}} $)  & $\approx 350 $ & Each amino acid is delivered to the ribosome by a specific tRNA molecule. \\
    \hline
    ATP/GTP ($ N_{\text{ATP/GPT}} $) & $ \approx 1050 $  & \textbf{Amino acid activation}: Each amino acid is activated by an aminoacyl-tRNA synthetase, consuming 1 ATP per amino acid. \textbf{Translation steps}: Each elongation cycle consumes 2 GTP, respectively for the binding of aminoacyl-tRNA to the A site and the translocation step [\ref{ref:Jakubowsky}]. \\
    \hline
    Ribosomes ($ N_{\text{rib}} $) & $ \approx 350 $ & Ribosomes are not  consumed, but each protein needs one ribosome to synthesise. \\
    \hline 
    Total molecule count ($ N_{\text{mol}}) $ & $ \approx 2100 $ & \\ 
    \hline
    \bottomrule
\end{tabularx}
\label{tab:protein_makeup}
\end{table}

\subsection{Optimization}
We can solve for the $\theta$ value corresponding to optimal growth rate by numerical optimization using $4$th-order Runge-Kutta or with the analytic solution in a simplex. To speed up the parameter sweep, we use a softmax transformation of the following form to formulate it as an unconstrained problem,
\begin{equation}
    \theta_1 \to \frac{e^u}{1+e^u + e^v}, \; \theta_2 \to \frac{1}{1+e^u + e^v}, \; \theta_3 \to \frac{e^v}{1+e^u + e^v},
\end{equation}
and optimize over the region $(u, v)\in (-\infty, 0]^2$. The simulation speed is greatly enhanced with the python optimization package \verb|scipy.optimize import minimize|, compared to traditional two-stage mninimization. Parallel processing is used for different $b$ values by calling \verb|from joblib import Parallel, delayed|.

The difference between different methods is shown in figure \ref{fig:optimize}.
\begin{figure}
    \centering
    \begin{minipage}[b]{0.32\linewidth}
        \centering
        \includegraphics[width=\linewidth]{Images/heatmap.png}
        \par (a)
    \end{minipage}
    \hfill
    \begin{minipage}[b]{0.32\linewidth}
        \centering
        \includegraphics[width=\linewidth]{Images/l_b10.png}
        \par (b)
    \end{minipage}
    \hfill
    \begin{minipage}[b]{0.32\linewidth}
        \centering
        \includegraphics[width=\linewidth]{Images/l_b10_ODE.png}
        \par (c)
    \end{minipage}
    \caption{(a) Unconstrained optimization in $(-\infty, 0]^2$ after SoftMax transformation. (b) Constrained optimization in simplex using the analytical solution, as in \cite{lin2025}. (c) Numerical solution of the system of coupled ODEs using the python optimization package.}
    \label{fig:optimize}
\end{figure}


\begin{acknowledgments}
We wish to acknowledge the support of the author community in using
REV\TeX{}, offering suggestions and encouragement, testing new versions,
\dots.
\end{acknowledgments}


\appendix

\section{Appendixes}

\section{A little more on appendixes}


\nocite{*}
\bibliography{draft_SI_aipsamp}% Produces the bibliography via BibTeX.


\end{document}