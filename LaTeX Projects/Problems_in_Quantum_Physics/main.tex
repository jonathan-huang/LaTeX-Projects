\documentclass[12pt, a4paper, twoside]{article}
\usepackage{scrextend}
    %% font & format %%
\usepackage[margin=3cm]{geometry}
\usepackage{type1cm, titlesec, fancyhdr, titling}
    %% Math, Logos & symbols %%
\usepackage{amsmath,amsthm,amssymb, mathtools}
\usepackage{yhmath, faktor, dsfont}

\usepackage{array} % tables
\usepackage[shortlabels]{enumitem}
\usepackage[normalem]{ulem}
\usepackage{mathrsfs}
\usepackage{indentfirst}
\usepackage{pifont}
\usepackage{fancyhdr}   
\usepackage{gensymb}
\usepackage{amssymb}
\usepackage{pgfplots}
\pgfplotsset{compat=1.15}

% Mandarin
\usepackage{CJKutf8}
% bkai = 標楷體
% bsmi = 新細明體

%% Enhancement %%
\usepackage{graphicx, tabularx}

% paragraph
\usepackage{parskip}
\setlength{\parindent}{2em}

\setlength{\headheight}{15pt}
\setlength{\droptitle}{-1.5cm}
\parindent=24pt

\newtheoremstyle{mystyle}
  {6pt}{15pt}
  {}%
  {}%
  {\bf}% 
  {.}%
  {1em}%
  {}% Theorem head spec (can be left empty, meaning 'normal')

\theoremstyle{mystyle}	
\newtheorem{theorem}{Theorem}
\newtheorem*{definition}{Definition}
\newtheorem{example}[theorem]{Example}
\newtheorem{exercise}{Exercise}
\newtheorem{solution}{Solution}
\newtheorem{corollary}[theorem]{Corollary}
\newtheorem{property}[theorem]{Property}
\newtheorem{proposition}[theorem]{Proposition}
\newtheorem{lemma}[theorem]{Lemma}
\newtheorem{problem}[theorem]{Problem}
\newtheorem*{remark}{Remark}
\newtheorem*{claim}{Claim}

\title{Problems in Quantum Physics}
\author{Jonathan Huang (Giant Water Bird)}
\date{\today}
\cfoot{\thepage}
% header and footer
\pagestyle{fancy}
\rhead{Jonathan Huang}
\lhead{Problems in quantum physics}

\begin{document}

\maketitle
\tableofcontents

\section{Introduction}
    \begin{figure}[h!]%
        \centering
        \subfloat{\includegraphics[width=8cm]{giant water bird.jpg}}
        \caption{}
    \end{figure}
This is a collection of interesting problems in quantum physics.

\newpage
\section{Problems in Quantum Chemistry}
\begin{problem}[Particle in a ring]
    Solve Schr\"{o}dinger's equation for a particle of mass $m$, constrained to move in a ring of radius $R$. Use this result to explain H\"{u}ckel's rule, e.g. the most stable number of $\pi$-electrons in a planar ring molecule is $4n+2$.
\end{problem}
\begin{solution}
    In polar coordinates, the Laplacian operator is
    \begin{equation}
        \label{polar Lap}
        \nabla^2 = \frac{1}{r}\frac{\Par}{\Par r}\left(r\frac{\Par}{\Par r}\right) + \frac{1}{r^2}\frac{\Par^2}{\Par\theta^2}.
    \end{equation}
    Since the particle is constrained to have $r=R$, this is just
    \begin{equation*}
        \nabla^2 = \frac{1}{R^2}\frac{\Par^2}{\Par\theta^2}.
    \end{equation*}
    The time-independent Schr\"{o}dinger's equation
    \begin{equation}
        \label{Schrodinger}
        \left[-\frac{\hbar^2}{2m}\nabla^2 + V(\mathbf{r})\right]\Psi(\mathbf{r}) = E\Psi(\mathbf{r})
    \end{equation}
    in this case becomes
    \begin{equation*}
        -\frac{\hbar^2}{2mR^2}\frac{\D^2\Psi(\theta)}{\D\theta^2} = E\Psi(\theta).
    \end{equation*}
    Solving the ODE gives the general solution
    \begin{equation*}
        \Psi = A_{+}e^{i\frac{R}{\hbar}\sqrt{2mE}\theta} + A_{-}e^{-i\frac{R}{\hbar}\sqrt{2mE}\theta}.
    \end{equation*}
    Since the wavefunction should be single valued at each point in space, we enforce the boundary condition 
    \begin{equation*}
        \Psi(\theta) = \Psi(\theta+2\pi),
    \end{equation*}
    Giving the quantisation condition
    \begin{equation*}
        \frac{R}{\hbar}\sqrt{2mE} = n, \text{ where }n=0,\;\pm 1,\;\pm 2,\;...\;.
    \end{equation*}
    Therefore the energy eigenvalues are
    \begin{equation}
        \label{PIR energy}
        \boxed{E_n = \frac{n^2\hbar^2}{2mR^2}, \text{ where }n=0,\;\pm 1,\;\pm 2,\;...\;.}
    \end{equation}
    The normalisation condition is
    \begin{equation}
        \label{normalise}
        \int\D\mathbf{r}\,\Psi(\mathbf{r})^*\Psi(\mathbf{r}) = 1,
    \end{equation}
    plugging in $\Psi(\theta)$ and integrating over one revolution gives
    \begin{equation*}
        \boxed{A_{+}+A_{-} = \frac{1}{2\pi}.}
    \end{equation*}
    \begin{remark}
        Fourier theorem: any wavefunction for a particle on a ring can be written as a superposition of these eigenfunctions.
    \end{remark}
    From equ. (\ref{PIR energy}) we know that to fill all energy levels up to the $n$-th one we need $(2n+1)\times 2 = 4n+2$ particles (electrons), since each electron (a spin-$\frac{1}{2}$ particle) has two spin states. 
\end{solution}

\begin{CJK}{UTF8}{bkai}
\begin{problembk}[Resonance energy 共振能的計算] % problem environment with line break

    \begin{enumerate}[(a)]
        \item 
        一個簡單分子的分子軌域可以近似寫成其組成原子的原子軌域的線性疊加,這稱為「原子軌域的線性疊加」(LCAO, linear combination of atomic orbitals)。假設一個分子由$N$個原子組成,這些原子有原子軌域$\chi_1$, $\chi_2$, ..., $\chi_N$,則分子軌域可以寫成
        \begin{equation}
            \label{LCAO}
            \Psi = c_1\chi_1 + c_2\chi_2 + \cdots + c_N\chi_N,
        \end{equation}
        其中$c_i$是實數。我們將系統能量寫成系統哈密頓量對$\Psi$取期望值
        \begin{equation*}
            E \equiv \frac{\Braket{\Psi|H|\Psi}}{\Braket{\Psi|\Psi}} ,
        \end{equation*}
        $\Braket{\cdot|\cdot|\cdot}$和$\Braket{\cdot|\cdot}$是狄拉克記號,定義為
        \begin{equation*}
            \Braket{\Psi | H | \Psi} \equiv \int_{\bR^3}\mathrm{d}^3r\,\Psi^* H \Psi,
        \end{equation*}
        \begin{equation*}
            \Braket{\Psi | \Psi} \equiv \int_{\bR^3}\mathrm{d}^3r\,\Psi^*\Psi.
        \end{equation*}
        為求計算方便,做出下列定義和近似:
        \begin{enumerate}[(i)]
            \item 庫倫位能(Coulomb energy)
            \begin{equation*}
                H_{ii} \equiv \Braket{\chi_i|H|\chi_i}.
            \end{equation*}
            這是電子分布在自己原子核中時獲得的位能。
            \item 相互作用能(interaction energy)
            \begin{equation*}
                H_{ij} \equiv \Braket{\chi_i|H|\chi_j}, \quad H_{ij}=H_{ji}.
            \end{equation*}
            這是電子跟其他原子核交互作用獲得的位能。
            \item 重疊積分(overlap integral)
            \begin{equation*}
                S_{ij} \equiv \Braket{\chi_i|\chi_j}, \quad S_{ij}=S_{ji}.
            \end{equation*}
        \end{enumerate}
        試用以上符號表示出$E$($H$, $\Psi$, $\chi_1$, $\chi_2$,..., $\chi_N$不可以出現)。
        
        \item 
        $c_i$必須讓系統的能量有最小值,試用以上符號寫出極值有解的條件(目前要求出$c_i$還過於複雜,因此不需要寫出$c_i$的解)。本題求出的方程稱為「永年方程式」(secular equation),這條方程式可以用來直接計算分子軌域的各個能階。

        \item 
        為了計算方便,我們可以做出以下近似:
        \begin{enumerate}[(i)]
            \item 原子軌域互相正交:
            \begin{equation*}
                S_{ij} = 0.
            \end{equation*}
            \item 每個原子的環境相似,因此相互作用的結果是一樣的:
            \begin{equation}
                H_{ii}\equiv \alpha, \quad H_{ij}\equiv\beta.
            \end{equation}
        \end{enumerate}
        注意$\alpha$, $\beta <0$。請寫出一個由$N$個原子組成,只有相鄰原子有相互作用的線性分子的永年方程式,你可以將$(\alpha-E)/\beta$以$x$表示。
        
        \item 
        我們可以用以上的結果來計算丙烯自由基的共振能。共振能定義為一個分子因為「電子離域化」造成的能量下降,因此共振能的計算需要一個沒有$\pi$軌域離域的參考分子。請解出丙烯自由基和其參考分子的永年方程式,以$\alpha$和$\beta$表示其能階,並計算丙烯的共振能。
        
        提示:丙烯自由基分子的去離域化版本可以視為是乙烯加上一個單獨$p$軌域。
    \end{enumerate}
\end{problembk}
\end{CJK}

\begin{CJK}{UTF8}{bkai}
\begin{solutionbk}
    \begin{enumerate}[(a)]
        \item 
        
    \end{enumerate}
\end{solutionbk}
\end{CJK}

\begin{CJK}{UTF8}{bkai}
\begin{problembk}[Aromatic compounds 芳香性化合物的探討]
    \begin{enumerate}[(a)]
        \item
        參考前一題的計算方法,寫下苯的永年方程式。
        \item 
        這個
        \item 
    \end{enumerate}
\end{problembk}
\end{CJK}

\begin{CJK}{UTF8}{bkai}
\begin{problem}[Spherical aromaticity 球芳香性]
    
\end{problem}
\end{CJK}

\section{Problems in }
\begin{problem}[Uncertainty principle and pencil]
    A thin rigid rod of mass $m$ and length $l$ is balanced upright. The gravitational acceleration is $g$, calculate the time for the rod to fall down.
\end{problem}
\begin{solution}
    Consider
\end{solution}

\end{document}