\documentclass[a4paper]{article}
\usepackage{scrextend}
    %% font & format %%
\usepackage[margin=3cm]{geometry}
\usepackage{type1cm, titlesec, fancyhdr, titling}
    %% Math, Logos & symbols %%
\usepackage{amsmath,amsthm,amssymb, mathtools}
\usepackage{yhmath, faktor, dsfont}

\usepackage{array} % tables
\usepackage[shortlabels]{enumitem}
\usepackage[normalem]{ulem}
\usepackage{mathrsfs}
\usepackage{indentfirst}
\usepackage{pifont}
\usepackage{fancyhdr}   
\usepackage{gensymb}
\usepackage{amssymb}
\usepackage{pgfplots}
\pgfplotsset{compat=1.15}

% Mandarin
\usepackage{CJKutf8}
% bkai = 標楷體
% bsmi = 新細明體

%% Enhancement %%
\usepackage{graphicx, tabularx}

% paragraph
\usepackage{parskip}
\setlength{\parindent}{2em}

\setlength{\headheight}{15pt}
\setlength{\droptitle}{-1.5cm}
\parindent=24pt

\newtheoremstyle{mystyle}
  {6pt}{15pt}
  {}%
  {}%
  {\bf}% 
  {.}%
  {1em}%
  {}% Theorem head spec (can be left empty, meaning 'normal')

\theoremstyle{mystyle}	
\newtheorem{theorem}{Theorem}
\newtheorem*{definition}{Definition}
\newtheorem{example}[theorem]{Example}
\newtheorem{exercise}{Exercise}
\newtheorem{solution}{Solution}
\newtheorem{corollary}[theorem]{Corollary}
\newtheorem{property}[theorem]{Property}
\newtheorem{proposition}[theorem]{Proposition}
\newtheorem{lemma}[theorem]{Lemma}
\newtheorem{problem}[theorem]{Problem}
\newtheorem*{remark}{Remark}
\newtheorem*{claim}{Claim}

\begin{document}

\begin{CJK}{UTF8}{bkai}
\title{
    \textbf{Counterexamples in Algebra} \\
    \vspace{0.5cm}
    \large{Introduction to Algebra, Fall 2024}
}
\author{
編者:黃紹凱、有寫的人自己填名字
}
\maketitle
\end{CJK}

\section{Linear Algebra}

\section{Abstract Algebra}

\subsection{Groups}

\begin{example}[Noncyclic group whose proper subgroups are all cyclic]
    ~
    
    The Klein-four group $V_4 \cong \mathbb{Z}/2\mathbb{Z} \times \mathbb{Z}/2\mathbb{Z}$. The result follows from the fact that it is the smallest noncyclic group.
\end{example}

\begin{example}[Nonisomorphic groups with the same lattice]
    
\end{example}

\begin{example}[The converse of Lagrange's Theorem is not true]
    ~
    
    $A_4$ does not have a subgroup of index 2.
\end{example}

\subsection{Rings}

The term \textbf{ring} is reserved for a ring with identity, while a ring without identity shall be called a \textbf{rng} in the following article.

\begin{example}[Are associates always unit multiples? Part I]
    ~

    Following the notation of Anderson, D.D et al., we adopt the following definitions and notations. Let $R$ be a commutative ring, and let $a, b \in R$. We say $a$ and $b$ are \textit{associates}, denoted $a \sim b$, if $(a) = (b)$. On the other hand, if $a = ub$ for some $u \in U(R)$ the unit group of $R$, we say $a$ and $b$ are \textit{strong associates} and write $a \approx b$.

    Now take the ring $R$ as the set of continuous functions on the closed interval $[0,3]$, i.e. $R = \{ f \in \mathcal{C}([0,3]) \}$. It is straightforward to show that $R$ is a ring with respect to the usual addition and multiplication in $\mathbb{R}$. 

    Now consider the piecewise functions
    $$
     f(x) =
    \begin{dcases}
        1 - x, & 0 \leq x \leq 1, \\
        0, & 1 \leq x \leq 2, \\
        x-2, & 2 \leq x \leq 3. \\
    \end{dcases} $$ 
    $$ g(x) =
    \begin{dcases}
        1 - x, & 0 \leq x \leq 1, \\
        0, & 1 \leq x \leq 2, \\
        -(x-2), & 2 \leq x \leq 3. \\
    \end{dcases}\;
    $$
    \begin{claim}
        $f(x) \mid g(x)$ and $g(x) \mid f(x)$. Or, equivalently, $f$ and $g$ are associates: $f \sim g$.
    \end{claim}
    \begin{proof}
        Let 
        $$ h(x) =
        \begin{dcases}
            1, & 0 \leq x \leq 1, \\
            3 - 2x, & 1 \leq x \leq 2, \\
            -1, & 2 \leq x \leq 3. \\
        \end{dcases}\;
        $$
        Then we can check that $h(x)$ is continuous, so $h \in R$, and $f(x) = h(x)g(x)$, $g(x) = h(x)f(x)$. Therefore $f \in (g)$ and $g \in (f)$, which implies $f \sim g$.
    \end{proof}
    Now suppose $h(x)$ is a unit in $R$, then there exists $k \in R$ such that $h(x)k(x) = 1$. But such a $k(x)$ cannot be continuous, since it has to satisfy $(3-2x)k(x) = 1$ on $[1,2]$, contradiction. Therefore $h \not\in U(R)$ is not a unit.
\end{example}
\begin{example}[Are associates always unit multiples? Part II]
    ~

    Here we give another counterexample of the above claim. Consider the ring 
\end{example}
\begin{example}[E.D.s, P.I.D.s, and U.F.D.s]
~

    \begin{enumerate}
        \item A \textit{P.I.D.} that is not a \textit{E.D.}: 
        \item A \textit{U.F.D.} that is not a \textit{P.I.D.}: 
        \item Rings that are not a \textit{U.F.D.}s:
    \end{enumerate}
\end{example}
\begin{example}[Ring which is isomorphic to its own square]
    ~

    Let $R = \prod^\infty_{i=1} \mathbb{Z}$. Then $R \cong R \times R$ by the isomorphism $$\phi : R \to R\times R, \;\phi((x_1, x_2, \dots)) = ((x_1, x_3, \dots), (x_2, x_4, \dots)). $$
\end{example}

\subsection{Fields}

\end{document}