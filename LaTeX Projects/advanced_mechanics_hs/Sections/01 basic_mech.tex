\subsection{牛頓力學}
牛頓力學是描述宏觀物體運動的基本理論,基於牛頓的三大運動定律。這些定律描述了物體在外力作用下的運動行為,並且在日常生活中得到了廣泛的應用。

\subsection{因次分析}

\begin{example}[斯托克斯定理]
    斯托克斯定理告訴我們球體在水中受到的阻力與其速度、半徑和水的黏度有關。假設球體的半徑為 $R$,速度為 $v$,水的黏度為 $\eta$,則阻力 $F$ 的因次可以表示為:
\end{example}

\begin{problem}[原子彈釋放出的能量]
    1945 年美國政府的曼哈頓計畫在新墨西哥州洛斯阿拉莫斯進行了原子彈的測試,被稱為三位一體(Trinity Test)。試爆的畫面(圖\ref{fig:trinity})由高速照相機拍下,雖然原子彈釋放的能量是機密而沒有被公開,我們可以從釋出的照片進行估算。假設原子彈釋放出的能量為 $E$。

    \begin{figure}
        \centering
        \includegraphics[width=0.8\textwidth]{Figures/trinity.jpg}
        \caption{三位一體原子彈試爆後 \(15\) 毫秒的照片,照片下方的座標尺是100公尺。}
        \label{fig:trinity}
    \end{figure}
    \begin{enumerate}
        \item 除了爆炸處未受波及的空氣密度 \( \rho \) 之外,再寫出兩個跟原子彈爆炸能量有關的物理量,幫助我們從圖片進行估算。
        \item 寫下這些物理量的因次。
        \item 根據因次分析,寫出一個公式來表達原子彈釋放的能量 $E$,並假設前面的正比常數是\(1\)。
        \item 估算三位一體原子彈釋放的能量,並將結果用「千噸 TNT」($\text{kT} = 4.184 \times 10^{12} \text{J}$)表示。
    \end{enumerate}
\end{problem}

假設我們用的單位制有 \(m\) 個基本量,這些基本量的因次可以用 \(M_1, M_2, \dots , M_m\) 來表示。則對於一個物理量 \(P\),我們可以寫出它的因次表示為
\begin{equation}
    [P] = M_1^{\alpha_1} M_2^{\alpha_2} \cdots M_m^{\alpha_m}.
\end{equation}
取對數可以得到
\begin{equation}
    \ln [P] = \alpha_1 \ln M_1 + \alpha_2 \ln M_2 + \cdots + \alpha_m \ln M_m.
\end{equation}

我們可以將 \( \{ \ln M_1, \ln M_2, \dots , \ln M_m \} \) 看成 \(m\) 維空間的基底,並將 \(\alpha_1, \alpha_2, \dots , \alpha_m\) 看成向量 \( [P] \) 在這個空間中的向量表示。我們可以寫 
\begin{equation}
    [P] = (\alpha_1, \alpha_2, \dots , \alpha_m).
\end{equation}

\begin{theorem}[白金漢 \(\pi \) 定理]
    假設一個物理問題涉及 \(n\) 個變量 \( P_1, P_2, \dots , P_n \),這些變量的因次可以用 \(m\leq n\) 個基本因次(單位)表示。則這個物理問題可以用 \(n-m\) 個無因次的變量 \( \Pi_1, \Pi_2, \dots , \Pi_{n-m} \) 來描述。

    因此如果先不考慮 \( n=m \) 的情況,我們可以把原本涉及 \(n\) 個變量的問題
    \begin{equation}
        f(P_1, P_2, \dots , P_n) = 0
    \end{equation}
    轉化成包含 \(n-m\) 個無因次變量的問題
    \begin{equation}
        F(\Pi_1, \Pi_2, \dots , \Pi_{n-m}) = 0.
    \end{equation}
    我們也可以用以上方程式解出其中一個無因次變量:
    \begin{equation}
        \Pi_1 = \Phi (\Pi_{2}, \dots , \Pi_{n-m}).
    \end{equation}
\end{theorem}