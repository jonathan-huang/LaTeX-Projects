\subsection{斜拋運動}
斜拋運動是指物體以一定的初速度和角度向上拋出後,在重力作用下的運動。這種運動可以分解為水平方向和豎直方向的運動,並且遵循獨立的運動規律。

%
\begin{problem}[高處的斜拋]
    ~
    假設一個物體被從離地面高 \( H \) 的懸崖上以初速度 \(v_0\) 和拋射角度 \(\theta\) 斜向上拋出。
    \begin{enumerate}
        \item 當拋射角度分別是 \(\theta = 45^{\circ}\) 和 \(\theta = 43^{\circ}\) ,\(H = 508\)公尺、\(v_0 = 100\)公尺/秒時,求物體的射程。你可能會需要用計算機幫助估算。 
        \item 你覺得從高處拋射時最遠射程的角度是大於、等於、還是小於 \(45^{\circ}\)?
        \item 求使物體有最大射程的拋射角度 \( \theta \)。
    \end{enumerate}
\end{problem}
\begin{solution}
    ~
    \begin{enumerate}
        \item 略。
        \item 小於 \(45^{\circ}\)。
        \item 根據能量守恆原理,動能變化量等於位能變化量:
        \begin{equation}
            \frac{1}{2} m v_{0}^{2} + mgH = \frac{1}{2} m v^{2},
        \end{equation}
        因此物理落地時的速度大小為
        \begin{equation}
            v = \sqrt{v_0^2 + 2gH}.
        \end{equation}

        重力加速度方向恆向下,因此初速度 \( \bvec{v_{0}} \)、末速度 \( \bvec{v} \)、和重力加速度造成的速度變化量 \( \bvec{g}t \) 形成一個一邊鉛直的三角形。注意到射程是
        \begin{equation}
            R = ( v_{0} \cos \theta ) T = \left(\frac{2}{g}\right) \left[ \frac{1}{2}(v_{0} \cos \theta ) (g T) \right].
        \end{equation}
        因此最大射程的 \(\theta\) 即為讓三角形有最大面積的 \( \theta\)。兩短邊長度固定,以發射點為旋轉點旋轉可知 \( \bvec{v_{0}} \) 和 \( \bvec{v} \) 夾角為 \( 90^{\circ} \) 時,三角形面積最大。由相似形得 
        \begin{equation}
            \tan \theta = \frac{v_0}{\sqrt{v_{0}^{2} + 2gH } }.
        \end{equation}
    \end{enumerate}
    
\end{solution}