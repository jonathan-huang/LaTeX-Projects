\subsection{Bucket--Brigade QRAM}
\label{ssec:bbqram}

Let each routing node be a qutrit
\(\{\ket{0},\ket{1},\ket{\bullet}\}\).
During one query exactly \(n\) nodes are driven by the address photons.
Following the approach in Arunachalam \emph{et al.}~\cite{Arunachalam2015}, we assume each active routing node is affected by a local, independent CPTP noise channel, which we approximate as a depolarizing channel for analytical tractability.
Three logical outcomes may occur:

\begin{enumerate}
\item \textbf{Right path}.  
  All \(n\) nodes switch correctly.  
  Probability
  \[
     P_{\mathrm{right}}=(1-p)^{n}.
  \]

\item \textbf{Wrong path.}  
  Exactly one active node flips to the wrong
  branch while the others work.  
  The bus qubit still reaches a leaf but
  addresses a wrong memory cell, corrupting the data oracle.
  To first order in \(p\)
  \[
     P_{\mathrm{wrong}}\approx n\,p\,(1-p)^{n-1}\le np .
  \]

\item \textbf{No path.}  
  Any combination of two or more failed routers
  disconnects the tree, leaving the bus qubit in a dangling wave-guide.
  This event dominates the residual probability:
  \(P_{\mathrm{nop}}\approx 1-P_{\mathrm{right}}-P_{\mathrm{wrong}}\).
\end{enumerate}

A single call therefore has fidelity
\(F=P_{\mathrm{right}}\simeq 1-np\).
Grover search makes
\(K=\Theta(2^{n/2})\) calls, so the overall success probability is
\(F^{K}\simeq\exp(-np\,2^{n/2})\).
Keeping that constant demands
\begin{equation}
   p = \mathcal{O}\!\bigl(2^{-n/2}\bigr).
   \label{eq:bbBound}
\end{equation}
Correlated dephasing or photon-loss leakage changes only the constant
prefactor~\cite{Robust2024}.  
Crucially, \eqref{eq:bbBound} shrinks \emph{exponentially} with the
address size, so bb-QRAM is unlikely to scale beyond \(n\approx10\)
without full quantum error correction.


\subsection{Fan--Out QRAM}
\label{ssec:fanout}

A fan-out tree entangles the address register with a single bus photon
that propagates through an \((2^{n}-1)\)-element GHZ state of routers
\cite{Giovannetti2008}.  
Modeling every router with an open-system master equation is intractable
because the Lindblad operators act \emph{non-locally} on the collective
GHZ mode.
The accepted workaround is a union-bound argument:

\begin{itemize}
\item If an independent Pauli error hits \emph{any} router with
      probability \(p\), coherence between two branches of the bus
      photon is destroyed.
\item The probability that \emph{no} router fails is
      \(P_{\mathrm{right}}=(1-p)^{2^{n}-1}\).
\end{itemize}

Demanding \(P_{\mathrm{right}}\ge 1-\varepsilon\) yields
\begin{equation}
   p \le \frac{\varepsilon}{2^{n}-1}
       = \mathcal{O}(2^{-n}).
   \label{eq:fanoutSingle}
\end{equation}
Because the failure of \emph{one} router already reveals which-path
information, \eqref{eq:fanoutSingle} is strictly tighter than the
bucket-brigade bound~\eqref{eq:bbBound}.
Inside Grover the requirement tightens to
\(p=\mathcal{O}(2^{-3n/2})\).
Hence fan-out QRAM is even more noise-intolerant than bb-QRAM, a
conclusion echoed by recent architectural surveys
\cite{Phalak2023,Morales2023}.


\subsection{Flip--Flop QRAM}
\label{ssec:ff}

Flip-flop (FF) QRAM is implemented by a \emph{circuit} that writes
\(M\) classical \(n\)-bit words into amplitude encoding using
\(\Theta(M\,2^{n})\) elementary gates
\cite{Park2019}.
Park, Petruccione and Rhee assume that after every logical time step
each qubit is depolarized with probability \(\varepsilon\).  
Writing one data set therefore succeeds with probability
\begin{equation}
   P_{\mathrm{succ}}
      =(1-\varepsilon)^{D},
      \quad
      D=\Theta(M\,n\log n),
   \label{eq:ffDepth}
\end{equation}
where the \(\log n\) term is the T-count overhead of decomposing an
\(n\)-control rotation.  
Solving \eqref{eq:ffDepth} for \(\varepsilon\) and expanding gives
\[
   \varepsilon
      = \mathcal{O}\!\bigl(1/(M n\log n)\bigr).
\]
Thus FF-QRAM tolerates \emph{inverse-polynomial} gate noise when the
state is prepared only once.  
If that same circuit is queried \(K=\Theta(2^{n/2})\) times,
one multiplies the error budget by \(K\), reproducing an exponential
constraint akin to \eqref{eq:bbBound}.  
Because the register width is merely \(n+m+1\) qubits, however, the
entire FF-QRAM can be encoded in a surface code, making the design far
more fault-tolerance-friendly than bb- or fan-out QRAM.


\subsection{Parameterized-Circuit QRAM (PQC-QRAM)}
\label{ssec:pqc}

A newer line of work replaces large, deterministic loaders by shallow
\emph{variational} state-preparation circuits
\cite{Benedetti2019,Du2022}.  
Given classical amplitudes \(\mathbf{v}\in\mathbb{R}^{N}\) one optimizes
a logarithmic-depth unitary
\(U(\boldsymbol{\theta})\) such that
\(U(\boldsymbol{\theta})\ket{0}^{\otimes n}\approx\ket{v}\).

\paragraph{Error sources.}
Two contributions add in series:

\begin{enumerate}
\item \textbf{Training error}  
      \(
  \varepsilon_{\text{train}}
    = 1 \;-\; \left|\langle v \mid U(\theta)\mid 0^{\otimes n}\rangle\right|^{2}
\).
      This is a \emph{classical} approximation error that can be made
      arbitrarily small given enough optimizer iterations
      \cite{Du2022}.

\item \textbf{Hardware noise}  
      Each of the \(D=\Theta(poly(n))\) gates undergoes a Pauli
      channel with probability \(p_{g}\).
      The state fidelity becomes
      \(F\approx (1-p_{g})^{D}\).
\end{enumerate}

For an algorithm that calls the loader \(K\) times we demand  
\(F^{K}(1-\varepsilon_{\text{train}})\ge 1-\varepsilon\), giving
\begin{equation}
   p_{g}
     \le \frac{\varepsilon-\varepsilon_{\text{train}}}{K\,D}
     = \mathcal{O}\!\Bigl(\tfrac{1}{poly(n)}\Bigr),
   \label{eq:pqcBound}
\end{equation}
because both \(K\) and \(D\) are polynomial in \(n\)
for all known PQC-QRAM proposals.
Equation~\eqref{eq:pqcBound} is \emph{polynomially} less stringent than
the exponential bounds for router-based QRAMs, explaining why
PQC-based loaders are currently regarded as the most NISQ-friendly
memory interface~\cite{Gilyen2023}.
They sacrifice the strict \(O(1)\) query depth of true QRAM for
trainability and noise resilience, a trade-off acceptable in many
variational or sampling-based workloads.