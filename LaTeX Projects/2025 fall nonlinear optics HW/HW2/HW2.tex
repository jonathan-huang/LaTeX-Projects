\documentclass[a4paper]{article}
%% Formatting %%
\usepackage[margin=3cm]{geometry}
\usepackage{type1cm, titlesec, fancyhdr, titling}
\usepackage{multicol}
\usepackage[dvipsnames]{xcolor}
\usepackage{ulem}
\usepackage{parskip}
\setlength{\parindent}{2em}
\setlength{\headheight}{15pt}
\setlength{\droptitle}{-1.5cm}
\parindent=24pt
%% Math and Symbols %%
\usepackage{amsmath,amsthm,amssymb, mathtools}
\usepackage{yhmath, faktor, dsfont}
\usepackage{academicons, wasysym, marvosym}
\usepackage[scr]{rsfso} 
\usepackage{latexsym, amsmath, amscd, amsmath, amsthm}
\usepackage{amssymb,amsmath,amsthm,graphicx,dsfont}
\usepackage{hyperref}

%% Enhancement %%
\usepackage{graphicx, tabularx}
\usepackage[shortlabels,inline]{enumitem}
%% TikZ %%
\usepackage{tikz-cd}
\usepackage[breakable]{tcolorbox}
\usetikzlibrary{decorations.pathmorphing}
\usetikzlibrary{calc, arrows,matrix}

%% Other packages %%
\usepackage{amsopn}

%% Traditional Chinese %%
\usepackage{CJKutf8}

%% Math environments %%
\newtheoremstyle{mystyle}
  {6pt}{15pt}% 上下間距
  {}%          內文字體
  {}%              縮排
  {\bf}%       標頭字體
  {.}%       標頭後標點
  {1em}% 內文與標頭距離
  {}% Theorem head spec (can be left empty, meaning 'normal')
\theoremstyle{mystyle}	
\newtheorem{theorem}{Theorem}
\newtheorem*{definition}{Definition}
\newtheorem{example}[theorem]{Example}
\newtheorem{exercise}{Exercise}
\newtheorem{solution}{Solution}
\newtheorem{corollary}[theorem]{Corollary}
\newtheorem{property}[theorem]{Property}
\newtheorem{proposition}[theorem]{Proposition}
\newtheorem{lemma}[theorem]{Lemma}
\newtheorem{problem}[theorem]{Problem}
\newtheorem{answer}{Answer}[section]
\newtheorem{fact}[theorem]{fact}
\newtheorem*{remark}{Remark}
\newtheorem*{claim}{Claim}
\newtheorem*{observation}{Observation}

\newcommand\bvec[1]{\mathbf{#1}}

\begin{document}
\begin{CJK}{UTF8}{bkai}

\title{%
  \textbf{2025 Fall Nonlinear Optics} \\
  \vspace{0.5cm}
  \large 
  Graduate Institute of Photonics and Optoelectronics, \\
  National Taiwan University\\
  Homework \#2 (Due Sep 18, 2025)\\
}
\author{物理三 黃紹凱 B12202004}
\date{\today}

\maketitle

% Problem 1
\begin{problem}[Two-photon absorption coefficient]
    Two-photon absorption coefficient can be considered as a nonlinear absorption coefficient $\beta$, where absorption constant $\alpha$ will be modified by the light intensity $I$ as
    \[
    \alpha = \alpha_0 + \beta I.
    \]
    Derive the relationship between $\beta$ and $\chi^{(3)}(\omega_1; \omega_1, -\omega_1, \omega_1)$ of degenerate four wave mixing.
\end{problem}

\begin{solution}
    The third-order nonlinear polarization for a monochromatic field at frequency $\omega_1$ can be expressed as
    \[
        P^{(3)}(\omega) = \frac{3}{4} \varepsilon_0 \chi^{(3)}(\omega_1; \omega_1, -\omega_1, \omega_1) |E(\omega_1)|^2 E(\omega_1),
    \]
    where the factor is permutation degeneracy. To derive the expression for $ \beta $, consider the intensity-dependent refractive index $ n = n_0 + n_2 I $, where $ n_2 $ is the nonlinear refractive index related to $ \chi^{(3)} $ by the relation $ D = \varepsilon_0 E + P = \varepsilon_0 \left(1 + \chi^{(1)} + 3 \chi^{(3)} \vert E \vert^2  \right) E $, and $ I = \frac{1}{2} n_0 \varepsilon_0 c \vert E \vert^2 $. Since $ D \equiv \varepsilon_0 n^2 E $, we have
    \[
        n^2 = n_0^2 + \frac{3}{4} \chi^{(3)} \vert E \vert^2, \quad n_0^2 = 1 + \chi^{(1)}.
    \] 
    For weak nonlinearities, we can linearize $ n(\omega) $ as 
    \[
        n(\omega) \approx n_0 + \frac{3}{8 n_0} \chi^{(3)} \vert E \vert^2 = \frac{3}{4 \varepsilon_0 c n_0^2} \chi^{(3)} I.
    \]
    Hence, the nonlinear refractive index is given by
    \[
        n_2 = \frac{3}{4 \varepsilon_0 c n_0^2} \chi^{(3)},
    \]
    Given the complex refractive index $ n(\omega) $, the absorption coefficient $ \alpha $ is related to the imaginary part of the refractive index by $ \alpha = \frac{2\omega}{c} \operatorname{Im}(n) $, so $ \beta = \frac{2\omega}{c} \operatorname{Im} (n_2) $. Plugging in the expression for $ n_2 $, we obtain
    \[
        \beta (\omega) = \frac{3 \omega}{2 \varepsilon_0 c^2 n_0^2} \operatorname{Im} \left( \chi^{(3)}(\omega_1; \omega_1, -\omega_1, \omega_1) \right).
    \] 
\end{solution}

\newpage 

% Problem 2
\begin{problem}[Third harmonic generation microscopy]
    Third harmonic generation microscopy (THGM) is an innovative optical imaging modality featuring high resolution through stainless process. In fact, THGM utilizes the endogenous optical nonlinearity of a sample to generate contrasts. THGM induces harmonic generation process by focusing a Gaussian beam through objectives on samples and achieves high resolution beyond diffraction limits.

    \begin{enumerate}[(a)]
        \item Now consider a third harmonic generation process induced with a Gaussian beam. Please use the wave propagation equation with $ P^{\text{NL}} $ (assume steady state and lossless medium) to derive the following equation in the handout,
        \[
            \frac{\partial A_{3\omega}(z)}{\partial z} = \frac{i3\omega}{2^n c n_{3\omega}} \chi^{(3)}(z) \frac{A_{\omega}^3}{\left(1 + i\frac{z}{b}\right)^2} e^{i\Delta k z}.
        \]

        \item Assume the resolution of conventional optical microscopy is the minimum waist of Gaussian beam, $w_0$, which could be approximated by the Rayleigh criterion,
        \[
            w_0 = \frac{0.61 \lambda}{n \text{NA}},
        \]
        where $n$ is the refractive index, and NA is the numerical aperture of the objective. Please calculate the THG microscopy resolution in water given $\text{NA} = 1.2$ (water immersion), and $1260\,\text{nm}$ excitation laser wavelength.
    \end{enumerate}
\end{problem}

\begin{solution}
    ~

    \begin{enumerate}[(a)]
        \item In the THG field, we start with the wave equation in a lossless nonlinear medium: 
        \[
            \nabla^2 \bvec{E} - \frac{n^2}{c^2} \frac{\partial^2 \bvec{E}}{\partial t^2} = \frac{1}{\varepsilon_0 c^2} \frac{\partial^2 \bvec{P}^{\text{NL}}}{\partial t^2}.
        \]  
        In the \emph{slowly-varying envelope approximation}, we express the electric field at angular frequency $ \omega $ and $ 3\omega $ as 
        \begin{align*}
            E_\omega (\bvec{r}, z, t) = \frac{1}{2} A_z (\omega) u_\omega (\bvec{r}, z) e^{i(k_\omega z - \omega t)} + c.c., \\
            E_{3\omega} (\bvec{r}, z, t) = \frac{1}{2} A_z (3\omega) u_{3\omega} (\bvec{r}, z) e^{i(k_{3\omega} z - 3\omega t)} + c.c.,
        \end{align*}
        where $ u_i (\bvec{r}, z) $ is the transverse mode profile of the Gaussian beam at frequency $ i $. For a THG from a fundamental Gaussian beam, we take 
        \[
            u_{\omega} (0, z) = \frac{1}{1 + \frac{iz}{z_0}}, \quad z_0 = \frac{\pi n w_0^2}{\lambda},
        \]
        and assume $ u_{3 \omega} (0, z) \approx u_{\omega} (0,z) $ on the axis. For a degenerate process, the nonlinear polarization at $ 3\omega $ is given by
        \[
            P^{\text{NL}} (3\omega) = 3 \varepsilon_0 \chi^{(3)} (3\omega; \omega, \omega, \omega) E_\omega^3. 
        \]
        Let's substitute the Gaussian-beam terms and keep only the terms oscillating at $ 3\omega $, then we have
        \[
            P^{\text{NL}} = 3 \varepsilon_0 \chi^{(3)} (3\omega; \omega, \omega, \omega) \left( A_\omega \frac{1}{1 + \frac{iz}{z_0}} e^{i(k_\omega z - \omega t)} + c.c. \right)^3.
        \]
        Substituting $ P^{\text{NL}} $ and our ansatz for $ E_{3\omega} $ into the wave equation and applying the \emph{slowly-varying envelope approximation}, we match the coefficient of $ e^{i(k_{3\omega} z - 3\omega t)} $ on both sides to obtain

        \item For THG, since the frequency is tripled, the wavelength is reduced to $ \lambda_{3\omega} = \frac{\lambda}{3} = 420\,\text{nm} $. Water has a refractive index of about $ 1.33 $, and the new NA is $ 1.2 $. The resolution of THG microscopy can be calculated using the Rayleigh criterion:
        \[
            \omega_{0, \text{THG}} = \frac{0.61 \times 420}{1.33 \times 1.2} = 160.5 \,\text{nm},  
        \]
        which is an improvement over the conventional microscopy resolution by roughly $ 3 $ times.
    \end{enumerate}                                         
\end{solution}

\newpage 

% Problem 3
\begin{problem}[Instantaneous polarization and $\chi^{(3)}$ symmetry for third harmonic generation]
    In an isotropic medium with a cubic nonlinearity, considering only instantaneous response, derive the $\chi^{(3)}$ tensor symmetry for the third harmonic generation process.

    Following the similar approach in our lecture, derive also the $\chi^{(3)}$ with circular polarization. If we used a clockwise circular polarized light as our excitation source, do we expect to see any THG in isotropic media?

    What will be the difference in the expected THGM images by using either linear polarized excitation light or circular polarized excitation light?
\end{problem}

\begin{solution}
    
\end{solution}

\end{CJK}
\end{document}
