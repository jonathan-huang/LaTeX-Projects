\documentclass[12pt]{article}
\usepackage{geometry}
\geometry{a4paper, margin=1in}
\usepackage{setspace}
\usepackage{enumitem}
\setlist{nosep}
\usepackage{titlesec}
\titleformat{\section}{\large\bfseries}{\thesection}{1em}{}

% Mandarin
\usepackage{CJKutf8}
% bkai = 標楷體
% bsmi = 新細明體

\begin{document}
\begin{CJK}{UTF8}{bkai}

\title{物理家教簡介}
\author{黃紹凱 Jonathan Huang}
\date{\today}
\maketitle

\section*{姓名}
黃紹凱

\section*{學歷}
\begin{itemize}
    \item 建國中學38屆數資班
    \item 臺灣大學物理學系大三
    \item 臺灣大學數學系(雙主修)
    \item 臺灣大學電機系(輔系)
    \item 大二 GPA(滿分為 4.30): 4.20 / 4.14
\end{itemize}

\section*{大學申請結果}
\begin{itemize}
    \item 學測國、英、數A、自皆15級分
    \item 臺大物理學系(正取1、就讀中))
    \item 臺大電機系(正取)
    \item 臺大化學系(正取)
    \item 清大物理學系(正取1)
    \item 師大物理學系(正取1)
\end{itemize}

\section*{非競賽經歷}
\begin{itemize}
    \item 物理相關的研究所課程修課表現:量子場論一(A+)、密度泛涵理論(A+)、量子力學一(A)、相對論(A+)、量子光學導論(A+)、量子資訊與計算(A+)、訊號處理與機器學習的數學基礎(A)
    \item 2025 IELTS 成績:8.5/9.0
    \item 臺大普通物理設計實驗競賽(特優)
    \item 大一通過臺大微積分一、微積分二、普通物理、普通化學與英文免修
    \item 建中數資班第一類市長獎(總成績前1\%)
    \item 台大物理人才培育計畫(成績優良畢業)
    \item 高中曾通過生物、化學、數學、物理、英文五科免修
    \item 2022 ~ 2024 於台北市補習班擔任化學科輔導老師兩年
\end{itemize}

\section*{競賽經歷}
\begin{itemize}
    \item 2023 全國物理能競一等獎第1名(保送物奧選訓營,未參加)
    \item 2023 台北市物理能競二等獎
    \item 2023 建中校內物理能競一等獎(校隊正取1)
    \item 2023 建中校內化學能競一等獎(放棄校隊)
    \item 2022 ARML美國高中數學聯賽團體賽全國銀牌、IRML國際組團體賽全球第4名、全國第1名
    \item 2022 清華盃個人銅牌
    \item \item 2021 建中校內科展物理與天文學科特優
\end{itemize}

\section*{教學經歷}
\begin{itemize}
    \item 2022-2024 台北市補習班化學科輔導老師2年經驗
    \item 台中女中高一(物理競賽)
    \item 高一(物理競賽)
    \item 高二(物理競賽、奧林匹亞)
    \item 雙語實驗小學小六(自然科)
\end{itemize}

\section*{教學項目與期望待遇(可議)}
\begin{itemize}
    \item 普通物理、其他大學物理內容(視自己能力決定能不能勝任)超修(\$1200+ / hr)
    \item 高中物理競賽(\$1200以上/小時)
\end{itemize}

\section*{教學時間與地點}
\begin{itemize}
    \item 每次上課時間為2至3小時,平日晚間或假日。
    \item 實體(雙北交通便利地點為佳)或線上,線上以 Google Meet 進行
    \item 可直接私訊討論
\end{itemize}

\section*{教學理念}
在臺大物理系遇到很多講話時眼睛發光、對自己專業可以侃侃而談的同學,從討論中總是可以延伸出有趣的想法。因此我希望學生不僅有可以請教課業、競賽問題的老師,也有可以一起深入討論的對象,並從討論中激發出新的想法。

\end{CJK}
\end{document}