\documentclass[a4paper]{article}
%% Formatting %%
\usepackage[margin=3cm]{geometry}
\usepackage{type1cm, titlesec, fancyhdr, titling}
\usepackage{multicol}
\usepackage[dvipsnames]{xcolor}
\usepackage{ulem}
\usepackage{parskip}
\setlength{\parindent}{2em}
\setlength{\headheight}{15pt}
\setlength{\droptitle}{-1.5cm}
\parindent=24pt
%% Math and Symbols %%
\usepackage{amsmath,amsthm,amssymb, mathtools}
\usepackage{yhmath, faktor, dsfont}
\usepackage{academicons, wasysym, marvosym}
\usepackage[scr]{rsfso} 
\usepackage{latexsym, amsmath, amscd, amsmath, amsthm}
\usepackage{amssymb,amsmath,amsthm,graphicx,dsfont}
\usepackage{hyperref}

%% Enhancement %%
\usepackage{graphicx, tabularx}
\usepackage[shortlabels,inline]{enumitem}
%% TikZ %%
\usepackage{tikz-cd}
\usepackage[breakable]{tcolorbox}
\usetikzlibrary{decorations.pathmorphing}
\usetikzlibrary{calc, arrows,matrix}

%% Other packages %%
\usepackage{amsopn}

%% Traditional Chinese %%
\usepackage{CJKutf8}

%% Math environments %%
\newtheoremstyle{mystyle}
  {6pt}{15pt}% 上下間距
  {}%          內文字體
  {}%              縮排
  {\bf}%       標頭字體
  {.}%       標頭後標點
  {1em}% 內文與標頭距離
  {}% Theorem head spec (can be left empty, meaning 'normal')
\theoremstyle{mystyle}	
\newtheorem{theorem}{Theorem}
\newtheorem*{definition}{Definition}
\newtheorem{example}[theorem]{Example}
\newtheorem{exercise}{Exercise}
\newtheorem{solution}{Solution}
\newtheorem{corollary}[theorem]{Corollary}
\newtheorem{property}[theorem]{Property}
\newtheorem{proposition}[theorem]{Proposition}
\newtheorem{lemma}[theorem]{Lemma}
\newtheorem{problem}[theorem]{Problem}
\newtheorem{answer}{Answer}[section]
\newtheorem{fact}[theorem]{fact}
\newtheorem*{remark}{Remark}
\newtheorem*{claim}{Claim}
\newtheorem*{observation}{Observation}

% code environment
\usepackage{listings}
\usepackage{xcolor}
\lstset{
    language=MATLAB,
    basicstyle=\ttfamily,
    keywordstyle=\color{blue},
    commentstyle=\color{green},
    stringstyle=\color{cyan},
    showstringspaces=false
}

\begin{document}
\begin{CJK}{UTF8}{bkai}

\title{%
  \textbf{2025 Fall Introduction to ODE} \\
  \vspace{0.5cm}
  \large 
  Homework 3 (Due Sep 22 12:00, 2025)\\
}
\author{物理/數學三 黃紹凱 B12202004}
\date{\today}

\maketitle

\begin{problem}
    Give the solutions, where possible in terms of the Bessel
    functions, of the differential equations

    \begin{enumerate}
        \item [(a)]\(x\frac{d^2y}{dx^2}+(x+1)^2y=0\),
        \item [(b)]\((1-x^2)\frac{d^2y}{dx^2}-2x\frac{dy}{dx}
    +n(n+1)y=0\)
    \end{enumerate}
\end{problem}
\begin{solution}
    Bessel's equation can be written in the form 
    \begin{equation}
        x^{2} \frac{\mathrm{d}^2 y}{\mathrm{d} x^{2}} + x \frac{\mathrm{d} y}{\mathrm{d} x} + (x^{2} - \nu^{2}) y = 0, 
    \end{equation}
    where $ \nu $ is real and positive. 

    \begin{enumerate}[(a)]
        \item We first solve the equation with the method of Frobenius. Multiply both sides by $ x $ to obtain
        \[
            x^2 \frac{d^2y}{dx^2}+x\,(x+1)^2\,y=0.
        \]
        Notice that $ x(x+1)^2 $ is analytic and $ x=0 $ is a singular point, so we assume a solution of the form
        \[
            y(x) = \sum_{n=0}^{\infty} a_n x^{n+r}, \quad a_0 \neq 0.
        \]
        Substituting into the differential equation, we find the equation 
        \begin{equation}
            \begin{split}
                & r(r-1) a_0 x^r + \left[r(r+1)a_1 + a_0\right] x^{r+1} + \left[(r+1)(r+2)a_2 + a_1 + 2a_0\right] x^{r+2} \\
                &+ \sum_{n=3}^{\infty} \left[(n+r)(n+r-1)a_n + a_{n-1} + 2 a_{n-2} + a_{n-3} \right]x^{n+r} = 0.
            \end{split}
        \end{equation}
        The $ x^r $ terms gives $ r = 0 $ or $ 1 $, but the $ x^{r+1} $ terms gives $ a_1 = -\frac{a_0}{r(r+1)} $, which is undefined for $ r=0 $, so we must have $ r=1 $. Solving for the coefficients of $ a_0 $, $ a_1 $, and $ a_2 $, we find 
        \begin{equation}
            a_1 = -\frac{1}{2}a_0, \quad a_2 = -\frac{1}{4} a_0.
        \end{equation}  
        Then from the recurrence relation
        \[
            \sum_{n=3}^{\infty} \left[(n+r)(n+r-1)a_n + a_{n-1} + 2 a_{n-2} + a_{n-3} \right]x^{n+r} = 0,
        \]
        thus 
        \begin{equation}
            a_n = -\frac{1}{n(n+1)} \left( a_{n-1} + 2 a_{n-2} + a_{n-3} \right), \quad n \geq 3,
        \end{equation}
        we can recursively solve for $ a_n $ in terms of $ a_0 $, giving the series solution 
        \begin{equation}
            \begin{split}
                y(x) &= a_0 \left[ x - \frac{1}{2} x^2 - \frac{1}{4} x^3 + \frac{1}{48} x^4 + \frac{47}{960} x^5 + \frac{17}{3200} x^6 + \frac{397}{134400} x^7 \right. \\
                &\left. - \frac{2537}{2508800} x^8 + \frac{12091}{541900800} x^9 + \frac{2684597}{48771072000} x^{10} + \frac{44458303}{5364817920000} x^{11} + \cdots \right] .
            \end{split}
        \end{equation}
        The other linearly independent solution may be found using the reduction of order method. The above coefficients were verified with the following MATLAB code. 
        \begin{lstlisting}
function a = gen_coeffs_rational(N)
    a=sym('a', [N+1, 1]); 
    a(1:3)=[1;-1/2;-1/4]; % initial values
    for k=4:(N+1)
        n=sym(k-1); 
        a(k)=simplify(-(a(k-1)+2*a(k-2)+a(k-3))/(n*(n+1)));
    end
end

N=10; a=gen_coeffs_rational(N); 
fprintf('a_0 = %s\n', char(a(1)));
arrayfun(@(i) fprintf('a_%d = %s\n', i-1, char(a(i))), 2:N+1);
        \end{lstlisting}

        \item This is Legendre's equation. The general solution is given by
        \[
            y(x) = A P_n(x) + B Q_n(x),
        \]
        where \( P_n(x) \) and \( Q_n(x) \) are the Legendre functions of the first and second kind, respectively, and \( A \) and \( B \) are constants. I will give a series expansion of $ P_n(x) $  in terms of a Fourier-Bessel series on $ [0,1] $. First write
        \begin{equation}
            P_n(x) = \sum_{m=1}^{\infty} a_m J_{l}(\alpha_m x),
        \end{equation} 
        where $ \alpha_1 < \alpha_2 < \alpha_3 < \ldots $ are roots of $ J_{l}(x) $. Since $ P_n(x) $ and $ P_n^{\prime} (x) $ are piecewise continuous on $ [0,1] $, the Fourier-Bessel series converges. Let $ n, l \in \mathbb{Z} $, then the coefficients $ a_m $ are given by 
        \begin{equation}
            \begin{split}
                a_m &= \frac{1}{2\left[J_{l}^{\prime} (\alpha_{m})\right]^{2}} \int^1_{-1} \mathrm{d}x\, x P_n(x) J_{l}(\alpha_m x) \\
                &= \frac{2^{n-1}}{\left[J_{l}^{\prime} (\alpha_{m})\right]^{2}}\sum^n_{k=1} {n \choose k} {\frac{n+k-1}{2} \choose n} \left[\int^1_0 \mathrm{d}x\, x^{k+1} J_{l}(\alpha_m x) + \int^0_{-1} \mathrm{d}x\, x^{k+1} J_{l}(\alpha_m x) \right] \\
                &= \frac{2^{n-1}}{\left[J_{l}^{\prime} (\alpha_{m})\right]^{2}}\sum^n_{k=1} {n \choose k} {\frac{n+k-1}{2} \choose n} \left[\int^1_0 \mathrm{d}x\, x^{k+1} J_{l}(\alpha_m x) + (-1)^k \int^0_{1} \mathrm{d}x\, x^{k+1} J_{l}(-\alpha_m x) \right] \\
                &= \frac{2^{n-1}}{\left[J_{l}^{\prime} (\alpha_{m})\right]^{2}}\sum^n_{k=1} {n \choose k} {\frac{n+k-1}{2} \choose n} \\
                &\times \left( \left\{ \frac{2^{k+1} \Gamma \left(\frac{l+k+2}{2}\right)}{\alpha_m^{k+2} \Gamma \left(\frac{l-k}{2}\right)} + \alpha_m^{-(k+1)}\left[(k+l)J_{l}(\alpha_m)S_{k,l-1}(\alpha_m) - J_{l-1}(\alpha_m) S_{k,l-1}(\alpha_m)\right]\right\} \right. \\
                &+ \left. \left\{ \frac{2^{k+1} \Gamma \left(\frac{l+k+2}{2}\right)}{\alpha_m^{k+2} \Gamma \left(\frac{l-k}{2}\right)} + (-\alpha_m)^{-(k+1)} \left[(l+k)J_{l}(-\alpha_m) S_{k, l-1}(-\alpha_m) - J_{l-1}(-\alpha_m) S_{k+1, l}(-\alpha_m)\right] \right\} \right) \\
                &= \frac{2^{n-1}}{\left[J_{l}^{\prime} (\alpha_{m})\right]^{2}}\sum^n_{k=1} {n \choose k} {\frac{n+k-1}{2} \choose n} \frac{2^{k+2} \Gamma \left(\frac{l+k+2}{2}\right)}{\alpha_m^{k+2} \Gamma \left(\frac{l-k}{2}\right)} \\
                &+ \frac{1}{\alpha_m^{k+1}}\left\{ (k+l)\left[J_{l}(\alpha_m) S_{k,l-1}(\alpha_m) + (-1)^{l+k}J_{-l}(\alpha_m) S_{k,-l+1}(\alpha_m)\right] \right. \\
                &- \left. \left[J_{l-1}(\alpha_m) S_{k+1,l}(\alpha_m) + (-1)^{l+k}J_{-l+1}(\alpha_m) S_{k+1, -l}(\alpha_m)\right] \right\}, \\
            \end{split}
        \end{equation}
        where we have used the series expansion for $ P_n(x) $
        \begin{equation}
            P_n(x) = 2^{n-1} \sum^n_{k=1} {n \choose k} {\frac{n+k-1}{2} \choose n} x^k,
        \end{equation}
        the symmetry condition for $ J_{l}(x) $
        \begin{equation}
            J_{l}(-x) = (-1)^l J_{-l}(x),
        \end{equation}
        and the Bessel function integral identity from [Gradshteyn \& Ryzhik] \textit{Table of integrals, series, and products} 6.561-13. 
        \begin{equation}
            \int^1_0 \mathrm{d}x\, x^{\mu} J_{\nu}(ax) = \frac{2^{\mu} \Gamma \left(\frac{\nu+\mu+2}{2}\right)}{a^{\mu+1} \Gamma \left(\frac{\nu-\mu}{2}\right)} + \frac{1}{a^{\mu}} \left[(\mu+\nu)J_{\nu}(a) S_{\mu, \nu-1}(a) - J_{\nu-1}(a) S_{\mu+1, \nu}(a)\right].
        \end{equation}
        Here we have defined the Lommel function $ S_{\mu, l} $ to be 
        \begin{equation}
            \begin{split}
                &S_{\mu, l}(x) = \frac{\pi}{2} \left[ Y_{l}(x) \int^x_0 \mathrm{d}u\, u^{\mu} J_{l}(u) - J_{l}(x) \int^x_0 \mathrm{d}u\, u^{\mu} Y_{l}(u) \right] \\
                &+ 2^{\mu - 1} \Gamma \left(\frac{\mu - l + 1}{2}\right) \Gamma \left(\frac{\mu + l+1}{2}\right)\left[\sin \left(\frac{1}{2}(\mu - l)\pi \right) J_{l}(x) - \cos \left(\frac{1}{2}(\mu -l)\pi \right)Y_{l}(x)\right] .
            \end{split}
        \end{equation} 
        Note that this function is also known as $ s^{(2)}_{\mu, l} $, while the first term inside square brackets is known as $ s^{(1)}_{\mu, l} $.

        For $ l\in \mathbb{Z} $, the Lommel function $ S_{\mu, l}(x) $ satisfies the symmetry identity 
        \begin{equation}
            S_{\mu , l}(-x) = (-1)^{\mu +1} S_{\mu, -l}(x).
        \end{equation}
        Finally, using the identity 
        \begin{equation}
            \frac{\mathrm{d}}{\mathrm{d}x}J_{l}(x) = \frac{1}{2} \left[J_{l-1}(x) - J_{l+1}(x)\right],
        \end{equation}
        we have 
        \begin{equation}
            \begin{split}
                a_m &= \frac{2^{n+1}}{J_{l-1}(2\alpha_m) - J_{l+1}(2\alpha_m)} \sum^\infty_{k=1} {n \choose k} {\frac{n+k-1}{2} \choose n} \frac{2^{k+2} \Gamma \left(\frac{l+k+2}{2}\right)}{\alpha_m^{k+2} \Gamma \left(\frac{l-k}{2}\right)} \\
                &+ \frac{1}{\alpha_m^{k+1}}\left\{ (k+l)\left[J_{l}(\alpha_m) S_{k,l-1}(\alpha_m) + (-1)^{l+k}J_{-l}(\alpha_m) S_{k,-l+1}(\alpha_m)\right] \right. \\
                &- \left. \left[J_{l-1}(\alpha_m) S_{k+1,l}(\alpha_m) + (-1)^{l+k}J_{-l+1}(\alpha_m) S_{k+1, -l}(\alpha_m)\right] \right\}.
            \end{split}
        \end{equation}
        Therefore, one solution is
        \begin{equation}
            \begin{split}
                P_n(x) &= 2^{n+1} \sum_{m=1}^{\infty} \frac{1}{J_{l-1}(\alpha_m) - J_{l+1}(\alpha_m)} \sum^\infty_{k=1} {n \choose k} {\frac{n+k-1}{2} \choose n} \frac{2^{k+2} \Gamma \left(\frac{l+k+2}{2}\right)}{\alpha_m^{k+2} \Gamma \left(\frac{l-k}{2}\right)} \\
                &+ \frac{1}{\alpha_m^{k+1}}\left\{ (k+l)\left[J_{l}(\alpha_m) S_{k,l-1}(\alpha_m) + (-1)^{l+k}J_{-l}(\alpha_m) S_{k,-l+1}(\alpha_m)\right] \right. \\
                &- \left. \left[J_{l-1}(\alpha_m) S_{k+1,l}(\alpha_m) + (-1)^{l+k}J_{-l+1}(\alpha_m) S_{k+1, -l}(\alpha_m)\right] \right\} J_{l}(\alpha_m x). 
            \end{split}
        \end{equation}
    \end{enumerate}
    The other solution $ Q_n(x) $ may be expressed in terms of $ P_n(x)$, $ \ln (x) $, and the standard recurrence relations for Legendre functions.
\end{solution}

\medskip

\begin{problem}
    Determine the coefficients of the Fourier-Bessel series for the function
    \begin{align*}
        f(x)=
        \begin{cases}
            1 \quad \text{ for } 0 \leq x< 1,\\
            -1 \quad \text{ for } 1 \leq x \leq 2,
        \end{cases}
    \end{align*}
    in terms of the Bessel function \(J_0(x)\).\\
\end{problem}
\begin{solution}
    Suppose the Fourier-Bessel series of \(f(x)\) is given by
    \begin{equation}
        f(x) = \sum_{n=1}^{\infty} a_n J_0(\alpha_n x),  
    \end{equation}
    where $ \alpha_1 < \alpha_2 < \alpha_3 < \ldots $ are roots of $ J_0(2 x) $. Since $ f(x) $ and $ f^{\prime} (x) = 0 $ (except at $ x=0 $) are piecewise continuous on $ [0,2] $, the Fourier-Bessel series converges. First we compute two integrals using the identity $ J_{0}^{\prime} (2 \alpha_{n}) = \frac{1}{2} J_{1}(2 \alpha_{n}) $ and $ \frac{\mathrm{d}}{\mathrm{d}u} \left( u J_{1}(u) \right) = u J_{0}(u) $:
    \begin{equation}
        \begin{split}
            \int^1_0 \mathrm{d}x\, x J_0(\alpha_{n}x) &= \frac{1}{\alpha_{n}^{2}} \int^{\alpha_{n}}_0 \mathrm{d}u\, u J_0(u) \\
            &= \frac{1}{\alpha_{n}^{2}} \left[ u J_1(u) \right]^{\alpha_{n}}_0 \\
            &= \frac{1}{\alpha_{n}} J_1(\alpha_{n}),
        \end{split}
    \end{equation}
    and
    \begin{equation}
        \begin{split}
            \int^2_1 \mathrm{d}x\, x J_0(\alpha_{n}x) &= \frac{1}{\alpha_{n}^{2}} \int^{2 \alpha_{n}}_{\alpha_{n}} \mathrm{d}u\, u J_0(u) \\
            &= \frac{1}{\alpha_{n}^{2}} \left[ u J_1(u) \right]^{2 \alpha_{n}}_{\alpha_{n}} \\
            &= \frac{1}{\alpha_{n}^{2}} \left( 2 \alpha_{n} J_1(2 \alpha_{n}) - \alpha_{n} J_1(\alpha_{n}) \right) \\
            &= \frac{1}{\alpha_{n}} \left( 2 J_1(2 \alpha_{n}) - J_1(\alpha_{n}) \right),
        \end{split}
    \end{equation}
    where we have use a substitution $ u = \alpha_n x $. Then, the coefficients $ a_n $ are given by
    \begin{equation}
        \begin{split}
            a_n &= \frac{1}{\left[ 2 J_{1}(2 \alpha_{n}) \right]^{2} } \int_0^2 \mathrm{d}x\, x f(x) J_0(\alpha_n x) \\
            &= \frac{1}{\left[ 2 J_{1}(2 \alpha_{n}) \right]^{2} } \left( - \int^2_1 \mathrm{d}x\, x J_{0}(\alpha_{n}x) + \int^1_0 \mathrm{d}x\, x J_{0}(\alpha_{n}x) \right) \\
            &= \frac{1}{ 2 \alpha_n \left[ J_{1}(2 \alpha_{n}) \right]^{2} } \left( - 2 J_1 (2 \alpha_n) + J_1 (\alpha_n)  + J_1 (\alpha_n) \right), \\
            &= \frac{1}{\alpha_n \left[ J_{1}(2 \alpha_{n}) \right]^{2} } \left( J_1 (\alpha_n) - J_1 (2 \alpha_n) \right).
        \end{split}
    \end{equation}
    Therefore, 
    \begin{equation}
        \begin{split}
            f(x) &= \sum_{n=1}^{\infty} \frac{1}{\alpha_n \left[ J_{1}( \alpha_{n}) \right]^{2} } \left( J_1 (\alpha_n) - J_1 (2 \alpha_n) \right) J_0(\alpha_n x).
        \end{split}
    \end{equation}
\end{solution}

\end{CJK}
\end{document}