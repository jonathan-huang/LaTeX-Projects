\documentclass[a4paper]{article}
%% Formatting %%
\usepackage[margin=3cm]{geometry}
\usepackage{type1cm, titlesec, fancyhdr, titling}
\usepackage{multicol}
\usepackage[dvipsnames]{xcolor}
\usepackage{ulem}
\usepackage{parskip}
\setlength{\parindent}{2em}
\setlength{\headheight}{15pt}
\setlength{\droptitle}{-1.5cm}
\parindent=24pt
%% Math and Symbols %%
\usepackage{amsmath,amsthm,amssymb, mathtools}
\usepackage{yhmath, faktor, dsfont}
\usepackage{academicons, wasysym, marvosym}
\usepackage[scr]{rsfso} 
\usepackage{latexsym, amsmath, amscd, amsmath, amsthm}
\usepackage{amssymb,amsmath,amsthm,graphicx,dsfont}
\usepackage{hyperref}

%% Enhancement %%
\usepackage{graphicx, tabularx}
\usepackage[shortlabels,inline]{enumitem}
%% TikZ %%
\usepackage{tikz-cd}
\usepackage[breakable]{tcolorbox}
\usetikzlibrary{decorations.pathmorphing}
\usetikzlibrary{calc, arrows,matrix}

%% Other packages %%
\usepackage{amsopn}

%% Traditional Chinese %%
\usepackage{CJKutf8}

%% Math environments %%
\newtheoremstyle{mystyle}
  {6pt}{15pt}% 上下間距
  {}%          內文字體
  {}%              縮排
  {\bf}%       標頭字體
  {.}%       標頭後標點
  {1em}% 內文與標頭距離
  {}% Theorem head spec (can be left empty, meaning 'normal')
\theoremstyle{mystyle}	
\newtheorem{theorem}{Theorem}
\newtheorem*{definition}{Definition}
\newtheorem{example}[theorem]{Example}
\newtheorem{exercise}{Exercise}
\newtheorem{solution}{Solution}
\newtheorem{corollary}[theorem]{Corollary}
\newtheorem{property}[theorem]{Property}
\newtheorem{proposition}[theorem]{Proposition}
\newtheorem{lemma}[theorem]{Lemma}
\newtheorem{problem}[theorem]{Problem}
\newtheorem{answer}{Answer}[section]
\newtheorem{fact}[theorem]{fact}
\newtheorem*{remark}{Remark}
\newtheorem*{claim}{Claim}
\newtheorem*{observation}{Observation}

% code environment
\usepackage{listings}
\usepackage{xcolor}
\lstset{
    language=MATLAB,
    basicstyle=\ttfamily,
    keywordstyle=\color{blue},
    commentstyle=\color{green},
    stringstyle=\color{cyan},
    showstringspaces=false
}

\begin{document}
\begin{CJK}{UTF8}{bkai}

\title{%
  \textbf{2025 Fall Introduction to ODE} \\
  \vspace{0.5cm}
  \large 
  Homework 11 (Due December 1 12:00, 2025)\\
}
\author{物理三 黃紹凱 B12202004}
\date{\today}

\maketitle

% Exercise 1
\begin{exercise}
    Consider the second-order, autonomous ordinary differential equation
    \begin{align*}
        \ddot x=3x^2-1,
    \end{align*}
    where a dot represents \(\frac{d}{dt}\). By integrating this equation once, obtain a relation between \(\dot x\) and \(x\). Sketch the phase portrait in the \((x,\dot x)\)-phase plane. Determine the coordinates of the two equilibrium points.
\end{exercise}

\begin{solution}
    ~

    \underline{Steps:}
    \begin{enumerate}
        \item Integrate once to obtain a relation between \(\dot x\) and \(x\) by multiplying both sides by \(\dot x\).
        \item Sketch the phase portrait in the \((x,\dot x)\)-phase plane.
    \end{enumerate}
    
    \underline{Method:}
    ~ 

    \begin{enumerate}
        \item Multiply both sides by \(\dot x\), and notice that $ \frac{\mathrm{d}}{\mathrm{d}t} \left( \frac{1}{2} \dot x^2 \right) = \dot x \ddot x $ by the chain rule: 
        \begin{align*}
            \frac{1}{2} \frac{\mathrm{d}}{\mathrm{d}t} \dot x^2 = \ddot x \dot x = (3x^2 - 1) \dot x.
        \end{align*}
        Then, integrate both sides with respect to \(t\):
        \begin{align*}
            \frac{1}{2} \dot x^2 = \int \mathrm{d}t\, (3x^2 - 1) \dot x = \int \mathrm{d}x\, (3x^2 - 1) = x^3 - x + C,
        \end{align*}
        where \(C\) is the constant of integration. Thus, we have the relation:
        \begin{align*}
            \frac{1}{2} \dot x^2 = x^3 - x + C.
        \end{align*}
        \item We plot the phase portrait diagram using Python. The result is given in Figure \ref{fig:ex1}.
        \begin{figure}
            \centering
            \includegraphics[width=0.7\textwidth]{ex1.png}
            \caption{Phase portrait in the \((x,\dot x)\)-phase plane for various values of \(C\).}
            \label{fig:ex1}
        \end{figure}
        
        The equilibrium points occur where \(\dot x = 0\) and \(\ddot x = 0\). Setting \(3x^2 - 1 = 0\) gives \(x = \pm \frac{1}{\sqrt{3}}\). Thus, the coordinates of the two equilibrium points are:
        \begin{align*}
            \left( -\frac{1}{\sqrt{3}}, 0 \right) \quad \text{and} \quad \left( \frac{1}{\sqrt{3}}, 0 \right).
        \end{align*}
        This corresponds to the two extrema in the phase portrait diagram.
    \end{enumerate}
\end{solution}

\newpage 

% Exercise 2
\begin{exercise}
    By sketching the curve \(\dot x=X(x)\), determine the equilibrim points and corresponding domains of attraction when
    \begin{enumerate}
        \item \(X(x)=x^2-x-2\)
        \item \(X(x)=e^{-x}-1\)
        \item \(X(x)=\sin{x}\).
    \end{enumerate}
    Now, check that this qualitative analysis is correct by actually solving each equation with initial conditions \(x=x_0\) when \(t=0\). Which method do you think is easier to use, qualitative or quantitative?
\end{exercise}

\begin{solution}
    ~

    \underline{Steps:}
    \begin{enumerate}
        \item Sketch the curves \(\dot x=X(x)\) for each case using a plotting code.
        \item Determine the equilibrim points and corresponding domains of attraction by looking at the slopes of the curves at the equilibrium points.
        \item Solve each differential equation explicitly. 
    \end{enumerate}
    
    \underline{Method:}
    ~ 

    \begin{enumerate}
        \item The phase portraits for each case are plotted using Python. The results are given in Figures \ref{fig:ex2-1}, \ref{fig:ex2-2}, and \ref{fig:ex2-3}.
        \begin{figure}
            \centering
            \includegraphics[width=0.7\textwidth]{ex2-1.png}
            \caption{Phase portrait in the \((x,\dot x)\)-phase plane for $ \dot x = x^2 - x - 2 $.}
            \label{fig:ex2-1}
        \end{figure}
        \begin{figure}
            \centering
            \includegraphics[width=0.7\textwidth]{ex2-2.png}
            \caption{Phase portrait in the \((x,\dot x)\)-phase plane for $ \dot x = e^{-x} - 1 $.}
            \label{fig:ex2-2}
        \end{figure}
        \begin{figure}
            \centering
            \includegraphics[width=0.7\textwidth]{ex2-3.png}
            \caption{Phase portrait in the \((x,\dot x)\)-phase plane for $ \dot x = \sin{x} $.}
            \label{fig:ex2-3}
        \end{figure}
        \item The equilibrim points are points where $ \dot x = 0 $. For 1., $ x^2 - x - 2 = (x-2)(x+1) $, so the equilibrim points are \(x = -1\) and \(x = 2\). The domain of attraction is given by points where $ X(x) < 0 $ to the right of the equilibrim point and $ X(x) > 0 $ to the left. Thus, the domain of attraction is $ (-\infty, 2) $. For 2., \(e^{-x} - 1 = 0\) gives \(x = 0\) as the only solution. The domain of attraction for \(x = 0\) is \((-\infty, \infty)\). For 3., \(\sin{x} = 0\) gives \(x = n\pi\) for \(n \in \mathbb{Z}\). The domain of attraction is \(( (2n)\pi, (2n+2)\pi )\) for $ n \in \mathbb{Z} $. 
        
        \item We solves each differential equation explicitly below: 
        
        \textbf{(1)} For $ X(x) = x^2 - x - 2 $,  
        \[
            \frac{\mathrm{d}x}{\mathrm{d}t} = X(x) = x^2 - x - 2.
        \]
        Separation of variables and partial fraction decomposition give 
        \[
            \int \frac{\mathrm{d}x}{x^2 - x - 2} = \int \mathrm{d}x\, \frac{1}{3} \left(\frac{1}{x-2} - \frac{1}{x+1}\right) = \int \mathrm{d}t
        \]
        Then,
        \[
            \frac{1}{3} \ln{\left| \frac{x-2}{x+1} \right|} = t + C_1
        \] 
        The initial condition $ x(0) = x_0 $ gives 
        \[
            C = \ln{\left| \frac{x_0 - 2}{x_0 + 1} \right|} \implies x = \frac{2 + C^{3t}}{1 - C e^{3t}}. 
        \] 
        Differentiating explicitly, we get
        \[
            \dot x = X(x) = x^2 - x - 2 \implies x_{\text{equil}} = -1, 2.
        \]
        When $ x_0 < 2 $, we have $ \lim_{t \to \infty} x(t) = -1 $, and when $ x_0 > 2 $, we have $ \lim_{t \to \infty} x(t) = \infty $ since $ 0<C<1 $, and the denominator approaches $ 0 $ in finite time. Therefore, the domain of attraction is $ (-\infty, 2) $.  

        \textbf{(2)} For $ X(x) = e^{-x} - 1 $,  
        \[
            \frac{\mathrm{d}x}{\mathrm{d}t} = X(x) = e^{-x} - 1 \implies \int \frac{\mathrm{d}x}{e^{-x} - 1} = \int \mathrm{d}t.
        \]
        Then,
        \[
            -\ln{\left| 1 - e^{x} \right|} = t + C_3 \implies x(t) = \ln{\left( 1 - C_4 e^{-t} \right)}, \quad C_4 = e^{-C_3}.
        \]
        The initial condition $ x(0) = x_0 $ gives 
        \[
            t = \ln{\left\vert \frac{1 - e^x}{1 - e^{x_0}} \right\vert } \implies x = \ln{\left( 1 - (1 - e^{x_0}) e^{-t} \right)}.
        \]
        Differentiating explicitly, we get
        \[
            \dot x = X(x) = e^{-x} - 1 \implies x_{\text{equil}} = 0.
        \]
        When $ x_0<0 $, $ e^{-x_0}-1<0 $, so $ \dot x < 0 $ and $ x $ is decreasing and bounded below by $ 0 $. When $ x_0>0 $, $ e^{-x_0}-1>0 $, so $ \dot x > 0 $ and $ x $ is increasing and bounded above by $ 0 $. Solving $ e^{-L}-1=0 $ gives $ L=0 $, so the domain of attraction is $ (-\infty, \infty) $. 

        \textbf{(3)} For $ X(x) = \sin x $,  
        \[
            \frac{\mathrm{d}x}{\mathrm{d}t} = X(x) = \sin{x} \implies \int \frac{\mathrm{d}x}{\sin{x}} = \int \mathrm{d}t.
        \]
        Then,
        \[
            \ln{\left| \tan{\frac{x}{2}} \right|} = t + C_5 \implies x(t) = 2 \arctan{\left( C_6 e^{t} \right)}, \quad C_6 = e^{C_5}.
        \]
        The initial condition $ x(0) = x_0 $ gives
        \[
            x_0 = 2 \arctan{C_6} \implies C_6 = \tan{\frac{x_0}{2}}, 
        \]
        and 
        \[
            x(t) = 2 \arctan{\left( \tan{\frac{x_0}{2}} e^{t} \right)}.
        \]
        Differentiating explicitly, we get $ \dot x = X(x) = \sin x $, so the equilibrim points are $ x_{\text{equil}} = n\pi, n \in \mathbb{Z} $. When $ (2n)\pi < x_0 < (2n+2)\pi $, $ n \pi < \frac{x_0}{2} < (n+1)\pi $, and 
        \[
            0 < \tan \frac{x_0}{2} < \infty.
        \]
        we have $ \lim_{t \to \infty} x(t) = 2\cdot \frac{\pi}{2} + 2n\pi = (2n+1)\pi $, where we are taking the principal value of the arctan function. Therefore, the domain of attraction is $ ( (2n)\pi, (2n+2)\pi ) $ for $ n \in \mathbb{Z} $.

        \item The qualitative method agrees with the quantitatvie method, and provides a quick overview of the system's behavior without explicit solutions, which is useful when the ODE does not admit simple closed-form solutions. Therefore, this method is easier to use in general.
    \end{enumerate}
\end{solution}

\end{CJK}
\end{document}