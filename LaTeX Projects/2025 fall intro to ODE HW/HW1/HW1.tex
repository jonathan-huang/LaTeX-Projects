\documentclass[a4paper]{article}
%% Formatting %%
\usepackage[margin=3cm]{geometry}
\usepackage{type1cm, titlesec, fancyhdr, titling}
\usepackage{multicol}
\usepackage[dvipsnames]{xcolor}
\usepackage{ulem}
\usepackage{parskip}
\setlength{\parindent}{2em}
\setlength{\headheight}{15pt}
\setlength{\droptitle}{-1.5cm}
\parindent=24pt
%% Math and Symbols %%
\usepackage{amsmath,amsthm,amssymb, mathtools}
\usepackage{yhmath, faktor, dsfont}
\usepackage{academicons, wasysym, marvosym}
\usepackage[scr]{rsfso} 
\usepackage{latexsym, amsmath, amscd, amsmath, amsthm}
\usepackage{amssymb,amsmath,amsthm,graphicx,dsfont}
\usepackage{hyperref}

%% Enhancement %%
\usepackage{graphicx, tabularx}
\usepackage[shortlabels,inline]{enumitem}
%% TikZ %%
\usepackage{tikz-cd}
\usepackage[breakable]{tcolorbox}
\usetikzlibrary{decorations.pathmorphing}
\usetikzlibrary{calc, arrows,matrix}

%% Other packages %%
\usepackage{amsopn}

%% Traditional Chinese %%
\usepackage{CJKutf8}

%% Math environments %%
\newtheoremstyle{mystyle}
  {6pt}{15pt}% 上下間距
  {}%          內文字體
  {}%              縮排
  {\bf}%       標頭字體
  {.}%       標頭後標點
  {1em}% 內文與標頭距離
  {}% Theorem head spec (can be left empty, meaning 'normal')
\theoremstyle{mystyle}	
\newtheorem{theorem}{Theorem}
\newtheorem*{definition}{Definition}
\newtheorem{example}[theorem]{Example}
\newtheorem{exercise}{Exercise}
\newtheorem{solution}{Solution}
\newtheorem{corollary}[theorem]{Corollary}
\newtheorem{property}[theorem]{Property}
\newtheorem{proposition}[theorem]{Proposition}
\newtheorem{lemma}[theorem]{Lemma}
\newtheorem{problem}[theorem]{Problem}
\newtheorem{answer}{Answer}[section]
\newtheorem{fact}[theorem]{fact}
\newtheorem*{remark}{Remark}
\newtheorem*{claim}{Claim}
\newtheorem*{observation}{Observation}

\begin{document}
\begin{CJK}{UTF8}{bkai}

\title{%
  \textbf{2025 Fall Introduction to ODE} \\
  \vspace{0.5cm}
  \large 
  Homework 1 (Due Sep 8, 2025)\\
}
\author{物理/數學三 黃紹凱 B12202004}
\date{\today}

\maketitle

\begin{problem}
  Find the two linearly independent solutions of each of the differential equations
  \begin{enumerate}[(a)]
    \item $ x^{2}\,\frac{d^{2}y}{dx^{2}} + x\!\left(x-\tfrac{1}{2}\right)\frac{dy}{dx} + \tfrac{1}{2}y = 0, $ 
    \item $ x^{2}\,\frac{d^{2}y}{dx^{2}} + x(x+1)\frac{dy}{dx} - y = 0, $ 
  \end{enumerate}
  using the method of Frobenius
\end{problem}

\begin{solution}
  Given an ordinary differential equation of the form
  \begin{equation}
    x^{2}\,\frac{d^{2}y}{dx^{2}} + x\,p(x)\frac{dy}{dx} + q(x)y = 0,
  \end{equation}
  where $ p(x) $ and $ q(x) $ are analytic at $ x = 0 $, the existence of two linearly independent solutions as a Frobenius series
  \begin{equation}
    y(x) = x^{r}\sum_{n=0}^{\infty} a_{n}x^{n}
  \end{equation}
  with $ a_{0} \neq 0 $ around the regular singular point $ x = 0 $ is guaranteed by Fuchs Theorem. It is obvious that $ x = 0 $ is not a solution to (a) nor (b), and the corresponding $ p(x) $ and $ q(x) $ are analytic at $ x = 0 $. Thus, we can apply the method of Frobenius. 
  \begin{enumerate}[(a)]
    \item Assume the solution is of the form $ y(x) = x^{r} \sum_{n=1}^{\infty} a_{n}x^{n}$. 
    \item Assume the solution is of the form $ y(x) = x^{r} \sum_{n=1}^{\infty} a_{n}x^{n}$. 
  \end{enumerate}
\end{solution}

\end{CJK}
\end{document}