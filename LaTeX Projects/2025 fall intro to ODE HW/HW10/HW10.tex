\documentclass[a4paper]{article}
%% Formatting %%
\usepackage[margin=3cm]{geometry}
\usepackage{type1cm, titlesec, fancyhdr, titling}
\usepackage{multicol}
\usepackage[dvipsnames]{xcolor}
\usepackage{ulem}
\usepackage{parskip}
\setlength{\parindent}{2em}
\setlength{\headheight}{15pt}
\setlength{\droptitle}{-1.5cm}
\parindent=24pt
%% Math and Symbols %%
\usepackage{amsmath,amsthm,amssymb, mathtools}
\usepackage{yhmath, faktor, dsfont}
\usepackage{academicons, wasysym, marvosym}
\usepackage[scr]{rsfso} 
\usepackage{latexsym, amsmath, amscd, amsmath, amsthm}
\usepackage{amssymb,amsmath,amsthm,graphicx,dsfont}
\usepackage{hyperref}

%% Enhancement %%
\usepackage{graphicx, tabularx}
\usepackage[shortlabels,inline]{enumitem}
%% TikZ %%
\usepackage{tikz-cd}
\usepackage[breakable]{tcolorbox}
\usetikzlibrary{decorations.pathmorphing}
\usetikzlibrary{calc, arrows,matrix}

%% Other packages %%
\usepackage{amsopn}

%% Traditional Chinese %%
\usepackage{CJKutf8}

%% Math environments %%
\newtheoremstyle{mystyle}
  {6pt}{15pt}% 上下間距
  {}%          內文字體
  {}%              縮排
  {\bf}%       標頭字體
  {.}%       標頭後標點
  {1em}% 內文與標頭距離
  {}% Theorem head spec (can be left empty, meaning 'normal')
\theoremstyle{mystyle}	
\newtheorem{theorem}{Theorem}
\newtheorem*{definition}{Definition}
\newtheorem{example}[theorem]{Example}
\newtheorem{exercise}{Exercise}
\newtheorem{solution}{Solution}
\newtheorem{corollary}[theorem]{Corollary}
\newtheorem{property}[theorem]{Property}
\newtheorem{proposition}[theorem]{Proposition}
\newtheorem{lemma}[theorem]{Lemma}
\newtheorem{problem}[theorem]{Problem}
\newtheorem{answer}{Answer}[section]
\newtheorem{fact}[theorem]{fact}
\newtheorem*{remark}{Remark}
\newtheorem*{claim}{Claim}
\newtheorem*{observation}{Observation}

% code environment
\usepackage{listings}
\usepackage{xcolor}
\lstset{
    language=MATLAB,
    basicstyle=\ttfamily,
    keywordstyle=\color{blue},
    commentstyle=\color{green},
    stringstyle=\color{cyan},
    showstringspaces=false
}

\begin{document}
\begin{CJK}{UTF8}{bkai}

\title{%
  \textbf{2025 Fall Introduction to ODE} \\
  \vspace{0.5cm}
  \large 
  Homework 10 (Due November 24 12:00, 2025)\\
}
\author{物理三 黃紹凱 B12202004}
\date{\today}

\maketitle

% Exercise 1
\begin{exercise}
    Let \(f(y,t)\) and \(g(y,t)\) be continuous and satisfy a Lipschitz
    condition with respect to \(y\) in a region \(D\). Suppose \(|f(y,t)-
    g(y,t)|<\varepsilon\) in \(D\) for some \(\varepsilon>0\). If \(y_1(t)\) is a solution of \(y'=f(y,t)\) and \(y_2(t)\) is a solution of \(y'=g(y,t)\), such that \(|y_2(t_0)-y_1(t_0)|<\delta\) for some \(t_0\) and \(\ delta>0\). Show that for all \(t\) for which \(y_1(t)\) and \(y_2(t)\) both exist, we have the inequality
    \begin{align*}
        |y_2(t)-y_1(t)| \leq \delta \exp{(K|t-t_0|)}+\frac{\varepsilon}{K}\{\exp{(K|t-t_0|)-1}\},
    \end{align*}
    where \(K\) is the Lipschitz constant. Hint: Use the Gronwall
    inequality.
\end{exercise}

\begin{solution}
    ~

    \underline{Steps:}
    \begin{enumerate}
        \item Bound $ \vert y_2^{\prime} (t) - y_1^{\prime} (t) \vert $ using the Lipschitz condition and the given inequality.
        \item Apply Gronwall's inequality to obtain the desired bound.
    \end{enumerate}
    
    \underline{Method:}
    ~ 

    \begin{enumerate}
        \item Let $ K $ be the Lipschitz constant of both $ f $ and $ g $ with respect to $ y $. Consider the difference
        \begin{align*}
            |y_2^{\prime} (t) - y_1^{\prime} (t)| &= \left\vert f(y_1 (t), t) - g(y_2(t), t) \right\vert \\
            &\leq \left\vert f(y_1 (t), t) - f(y_2(t), t) \right\vert + \left\vert f(y_2 (t), t) - g(y_2(t), t) \right\vert \\
            &\leq K \left\vert y_1 (t) - y_2(t) \right\vert + \varepsilon.
        \end{align*}
        Let $ u(t) = \vert y_2 (t) - y_1 (t) \vert  $, then we have 
        \begin{align*}
            u^{\prime} (t) &= \frac{y_2 (t) - y_1 (t)}{\vert y_2 (t) - y_1 (t) \vert} \left(y_2^{\prime} (t) - y_1^{\prime} (t)\right) \leq \left\vert y_2^{\prime} (t) - y_1^{\prime} (t) \right\vert \leq K u(t) + \varepsilon.
        \end{align*}
        
        \item By assumption, we have $ u(t_0) = \vert y_2 (t_0) - y_1 (t_0) \vert < \delta $. Let $ v(t) = u(t) + \frac{\varepsilon}{K} $, then 
        \begin{equation*}
            v^{\prime} (t) \leq K v(t), \quad v(t_0) < \delta + \frac{\varepsilon}{K}.
        \end{equation*}
        By Gronwall's inequality, we have
        \begin{align*}
            v(t) &\leq v(t_0) \exp{(K|t-t_0|)} < \left(\delta + \frac{\varepsilon}{K}\right) \exp{(K|t-t_0|)}.
        \end{align*}
        Since $ u(t) = \vert y_2 (t) - y_1 (t) \vert $, for all $ t \geq t_0 $, we have 
        \begin{align*}
            u(t) &= v(t) - \frac{\varepsilon}{K} < \left(\delta + \frac{\varepsilon}{K}\right) \exp{(K|t-t_0|)} - \frac{\varepsilon}{K} \\
            &= \delta \exp{(K|t-t_0|)} + \frac{\varepsilon}{K} \left(\exp{(K|t-t_0|)} - 1\right).
        \end{align*}
        Similarly, for all $ t < t_0 $, apply the same reasoning to the interval $ [t, t_0] $ gives 
        \begin{align*}
            u(t) &= v(t) - \frac{\varepsilon}{K} < \left(\delta + \frac{\varepsilon}{K}\right) \exp{(K|t-t_0|)} - \frac{\varepsilon}{K} \\
            &= \delta \exp{(K|t-t_0|)} + \frac{\varepsilon}{K} \left(\exp{(K|t-t_0|)} - 1\right).
        \end{align*}
    \end{enumerate}
\end{solution}

% Exercise 2
\newpage
\begin{exercise}
    Let \(\sigma(t) \in C^1([a,a+\varepsilon])\), \(\sigma(t)>0\), and \(0<\sigma(a) \leq 1\). Suppose \(\sigma(t)\) satisfies the differential
    inequality \(\sigma' \leq K \sigma \log\sigma\), show the inequality
    \begin{align*}
        \sigma(t) \leq \sigma(a)e^{K(t-a)}, \quad \text{for } t \in [a,a+\varepsilon].
    \end{align*}
\end{exercise}

\begin{solution}
    ~

    \underline{Steps:}
    \begin{enumerate}
        \item Consider a change of variables to simplify the differential inequality.
        \item Apply Gronwall's inequality to derive the desired bound.
    \end{enumerate}
    
    \underline{Method:}
    ~ 

    \begin{enumerate}
        \item Consider the function $ \phi (t) = \log \sigma (t) $. Since $ \sigma (t) > 0 $, $ \phi (t) $ is well-defined and differentiable on $[a,a+\varepsilon]$. We have $ \phi (a) = \log \sigma (a) < 0 $ and  
        \begin{align*}
            \sigma (t) \phi^{\prime} (t) = \sigma^{\prime} (t) \leq K \sigma (t) \log \sigma (t). 
        \end{align*}
        Since $ \sigma (t) > 0 $, divide both sides by $ \sigma (t) $ to obtain
        \begin{align*}
            \phi^{\prime} (t) \leq K \phi (t).
        \end{align*} 
        \item Apply Gronwall's inequality to $ \phi (t) $ on the interval $ [a,t] $ for any $ t \in [a,a+\varepsilon] $, we have
        \begin{align*}
            \phi (t) &\leq \phi (a) e^{K(t-a)} = \log \sigma (a) e^{K(t-a)}.
        \end{align*}
        Exponentiating both sides now gives
        \begin{align*}
            \sigma (t) = e^{\phi (t)} \leq e^{\log \sigma (a) e^{K(t-a)}} = \sigma (a)^{e^{K(t-a)}}. 
        \end{align*}
        To obtain the desired inequality, note that since $ 0 < \sigma (a) < 1 $, we have $ \phi (a) < 0 $ and $ K(t-a) > 0 $ for $ t \in [a,a+\varepsilon] $. Then  
        \begin{align*}
            \phi (t) \leq \phi (a) e^{K(t-a)} \leq \phi (a) \left(1 + K(t-a)\right) \leq \phi (a) + K(t-a),
        \end{align*}
        which implies by exponentiation that
        \begin{align*}
            \sigma (t) = e^{\phi (t)} \leq e^{\phi (a) + K(t-a)} = \sigma (a) e^{K(t-a)}, \quad t \in [a,a+\varepsilon].
        \end{align*}
    \end{enumerate}
\end{solution}

% Exercise 3
\newpage
\begin{exercise}
    For each fixed \(x\), let \(F(x,y)\) be a nonincreasing function of \(y\). Show that, if \(f(x)\) and \(g(x)\) are two solutions of \(y'=F(x,y)\) and \(b>a\), then
    \begin{align*}
        |f(b)-g(b)| \leq |f(a)-g(a)|.
    \end{align*}
    Hence, deduce a result concerning the uniqueness of solutions. This is known as the \textbf{Peano uniqueness theorem}.\\
\end{exercise}
\begin{solution}
    ~

    \underline{Steps:}
    \begin{enumerate}
        \item Show that $ u(x) = \vert f(x) - g(x) \vert $ is a nonincreasing function of $ x $.
        \item Deduce the Peano uniqueness theorem as a consequence.
    \end{enumerate}
    
    \underline{Method:}
    ~ 

    \begin{enumerate}
        \item Consider the difference $ u(x) = \vert f (x) - g (x) \vert  $. Then, 
        \begin{align*}
            u^{\prime} (x) = \frac{f(x) - g(x)}{u(x)} \left(f^{\prime} (x) - g^{\prime} (x)\right) = \frac{f(x) - g(x)}{u(x)} \left(F(x, f(x)) - F(x, g(x))\right). 
        \end{align*}
        Suppose $ f(x) > g(x) $, then $ u(x) = f(x) - g(x) > 0 $. However, since $ F(x,y) $ is nonincreasing in $ y $, we have $ F(x, f(x)) - F(x, g(x)) \leq 0 $, and hence $ u^{\prime} (x) \leq 0 $. On the other hand, if $ f(x) < g(x) $, then $ u(x) > 0 $ and $ u^{\prime} (x) \leq 0 $ again. Therefore, we have
        \begin{align*}
            u(x) u^{\prime} (x) \leq 0 \implies \frac{d}{dx} (u(x))^2 = 2 u(x) u^{\prime} (x) \leq 0,
        \end{align*}
        while the case $ f(x) = g(x) $ is trivial. Therefore, $ \left(u(x)\right)^2 $ is a nonincreasing function for all $ x \in \mathbb{R} $. Since $ u(x)\geq 0 $, $ u(x) $ is also nonincreasing on $ \mathbb{R} $, and we have 
        \begin{align*}
            |f(b) - g(b)| = |u(b)| \leq |u(a)| = |f(a) - g(a)|, \quad \text{for all } b > a. 
        \end{align*}
        \item As a direct consequence, suppose $ f $ and $ g $ are two solutions to the initial value problem $ y^{\prime} (x) = F(x, y(x)) $, subject to the same initial condition $ f(x_0) = g(x_0) $ for some $ x_0 \in \mathbb{R} $, and $ F(x,y) $ is a nondecreasing function in $ y $. Then $ f(x) = g(x) $ for all $ x > x_0 $. Thus, the solution to the initial value problem $ y' = F(x,y) $, $ y(x_0) = y_0 $ is unique. 
        
        This result may be what is referred to as the Peano uniqueness theorem in the problem statement.
    \end{enumerate}
\end{solution}

\end{CJK}
\end{document}