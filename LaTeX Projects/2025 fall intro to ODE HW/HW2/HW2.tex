\documentclass[a4paper]{article}
%% Formatting %%
\usepackage[margin=3cm]{geometry}
\usepackage{type1cm, titlesec, fancyhdr, titling}
\usepackage{multicol}
\usepackage[dvipsnames]{xcolor}
\usepackage{ulem}
\usepackage{parskip}
\setlength{\parindent}{2em}
\setlength{\headheight}{15pt}
\setlength{\droptitle}{-1.5cm}
\parindent=24pt
%% Math and Symbols %%
\usepackage{amsmath,amsthm,amssymb, mathtools}
\usepackage{yhmath, faktor, dsfont}
\usepackage{academicons, wasysym, marvosym}
\usepackage[scr]{rsfso} 
\usepackage{latexsym, amsmath, amscd, amsmath, amsthm}
\usepackage{amssymb,amsmath,amsthm,graphicx,dsfont}
\usepackage{hyperref}

%% Enhancement %%
\usepackage{graphicx, tabularx}
\usepackage[shortlabels,inline]{enumitem}
%% TikZ %%
\usepackage{tikz-cd}
\usepackage[breakable]{tcolorbox}
\usetikzlibrary{decorations.pathmorphing}
\usetikzlibrary{calc, arrows,matrix}

%% Other packages %%
\usepackage{amsopn}

%% Traditional Chinese %%
\usepackage{CJKutf8}

%% Math environments %%
\newtheoremstyle{mystyle}
  {6pt}{15pt}% 上下間距
  {}%          內文字體
  {}%              縮排
  {\bf}%       標頭字體
  {.}%       標頭後標點
  {1em}% 內文與標頭距離
  {}% Theorem head spec (can be left empty, meaning 'normal')
\theoremstyle{mystyle}	
\newtheorem{theorem}{Theorem}
\newtheorem*{definition}{Definition}
\newtheorem{example}[theorem]{Example}
\newtheorem{exercise}{Exercise}
\newtheorem{solution}{Solution}
\newtheorem{corollary}[theorem]{Corollary}
\newtheorem{property}[theorem]{Property}
\newtheorem{proposition}[theorem]{Proposition}
\newtheorem{lemma}[theorem]{Lemma}
\newtheorem{problem}[theorem]{Problem}
\newtheorem{answer}{Answer}[section]
\newtheorem{fact}[theorem]{fact}
\newtheorem*{remark}{Remark}
\newtheorem*{claim}{Claim}
\newtheorem*{observation}{Observation}

\begin{document}
\begin{CJK}{UTF8}{bkai}

\title{%
  \textbf{2025 Fall Introduction to ODE} \\
  \vspace{0.5cm}
  \large 
  Homework 2 (Due Sep 15, 2025)\\
}
\author{物理/數學三 黃紹凱 B12202004}
\date{\today}

\maketitle

\begin{problem}[Legendre Polynomials]
    Let $ P_{n} $ be the Legendre polynomial of degree $ n $. Prove that $ \left\vert P^{\prime}_{n} (x) \right\vert < n^{2} $ and $ \left\vert P^{\prime \prime}_{n} (x) \right\vert < n^{4} $ for $ -1 < x < 1 $.
\end{problem}

\begin{solution}
    The Legendre polynomial $ P_{n}(x) $ satisfies the Legendre differential equation
    \begin{equation}
        \label{equ:diff}
        (1 - x^{2}) y^{\prime\prime} - 2xy^{\prime} + n(n+1)y = 0.
    \end{equation}

    We cite the paper \textit{"On a question by D. I. Mendeleev". Zap. Imp. Akad. Nauk. St. Petersburg. 62: 1-24.} for the result referred to as \textbf{Markov brothers' inequality}: Let $ P(x) \in \mathbb{R}[x] $ be a polynomial of degree $ n $, then for each integer $ k \geq 1 $, we have
    \begin{equation}
        \max_{-1\le x \le 1}\!\left\vert P^{(k)}(x) \right\vert
        \;\le\; \frac{n^{2}\,\bigl(n^{2}-1^{2}\bigr)\,\bigl(n^{2}-2^{2}\bigr)\cdots\bigl(n^{2}-(k-1)^{2}\bigr)}{1\cdot 3\cdot 5\cdots (2k-1)}\,\max_{-1\le x \le 1}\! \left\vert P(x) \right\vert.
    \end{equation}

    Furthermore, we claim that for $ x \in [-1, 1] $,
    \begin{equation}
        \label{equ:max}
        \left\vert P(x) \right\vert \leq 1
    \end{equation}
    \begin{proof}
    From \textit{George Arfken et al. Mathematical Methods For Physicists, Second Edition, Academic Press (1970)} , we have the \textbf{Schläfli integral representation} of the Legendre polynomial: 
    \begin{equation}
        P_n(z) = \frac{1}{2\pi i}\int_C \mathrm{d}\zeta \, \frac{(\zeta^2-1)^n}{2^n(\zeta - z)^{n+1}} \, ,
    \end{equation}
    where $ z \in \mathbb{C} $ and $ C $ surrounds $ z $. With the substitution $ \zeta = z + \sqrt{z^{2} - 1}e^{i \theta} $, we have
    \begin{equation}
        P_n(z)=\frac{1}{\pi}\int_{0}^{\pi} \mathrm{d}\theta \, \!\left(z+\sqrt{z^2-1}\,\cos\theta\right)^{n} \, .
    \end{equation}
    This is the Laplace integral representation of the Legendre polynomial. For $ x \in [-1, 1] $, let $ x = \cos \phi $, then
    \begin{equation}
        \begin{split}
            \left\vert P_n(\cos \phi) \right\vert &= \left\vert \frac{1}{\pi}\int_{0}^{\pi} \mathrm{d}\theta \, \!\left(\cos \phi + i \sin \phi \cos \theta\right)^{n} \right\vert \\
            &\leq \frac{1}{\pi}\int_{0}^{\pi} \mathrm{d}\theta \, \left\vert \cos\phi +i \sin\phi \cos\theta \right\vert^{n} \\
            &= \frac{1}{\pi}\int_{0}^{\pi} \mathrm{d}\theta \, \left(\cos^{2}\phi + \sin^{2}\phi \cos^{2}\theta\right)^{\frac{n}{2}} \\
            &\leq \frac{1}{\pi}\int_{0}^{\pi} \mathrm{d}\theta = 1.
        \end{split}
    \end{equation}
    \end{proof}

    By Markov brothers' inequality and equation (\ref{equ:max}), we have
    \begin{equation}
        \begin{split}
            P_{n}^{\prime} (x) &\le \max_{-1\le x \le 1}\!\left\vert P^{(k)}(x) \right\vert \le n^{2} \max_{-1\le x \le 1}\! \left\vert P(x) \right\vert = n^{2}, \\
            P_{n}^{\prime\prime} (x) &\le \max_{-1\le x \le 1}\!\left\vert P^{(k)}(x) \right\vert \le \frac{n^{2}(n^{2}-1)}{3} \max_{-1\le x \le 1}\! \left\vert P(x) \right\vert = \frac{n^{2}(n^{2}-1)}{3} \le n^{4}.
        \end{split}
    \end{equation}
\end{solution}

\end{CJK}
\end{document}