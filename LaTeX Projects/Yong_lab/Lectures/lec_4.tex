\chapter{Ultrashort Pulse Characterization}
% \lecture{4}{9 Sep. 08:00}{Fourth Lecture}

\section{Frequency-Resolved Optical Grating}
\textbf{Frequency-resolved optical gating (FROG)} is a technique for the complete characterization of ultrashort pulses. It measures not only pulse parameters such as the pulse energy or pulse duration, but also the full time-dependent electric field, including the optical spectrum\cite{Paschotta}. This method allows us to measure an ultrashort laser pulse without a shorter reference pulse \cite{}.

A sophisticated iterative phase retrieval algorithm, implemented with a computer program, can then be used for reconstructing the pulse shape from the FROG trace.

\section{Cross-Correlation Frequency Optical Grating}
really. It uses the spectrally resolved cross-correlation signal of the weak pulse with a fully characterized reference pulse to generate a spectrogram, which is analyzed by an iterative algorithm. No spectral overlap between unknown and reference pulse is needed, which makes this method very flexible.