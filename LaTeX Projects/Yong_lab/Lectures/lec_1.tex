\chapter{Nonlinear Optics}
% \lecture{1}{13 Oct. 08:00}{Ultra-High-Energy Cosmic Rays}

\section{Classical Optics}
Maxwell's equations are fundamentally linear. In standard units, they are given by four partial differential equations 
\begin{equation}
  \begin{split}
    \nabla \cdot \mathbf{E} &= \frac{\rho}{\epsilon_0} \\
    \nabla \cdot \mathbf{B} &= 0 \\
    \nabla \times \mathbf{E} &= -\frac{\partial \mathbf{B}}{\partial t} \\
    \nabla \times \mathbf{B} &= \mu_0 \mathbf{J} + \mu_0 \epsilon_0 \frac{\partial \mathbf{E}}{\partial t}. 
  \end{split}
\end{equation}

We have linear constitutive relations 
\begin{equation}
  \begin{split}
    \bvec{D} &= \epsilon_0 \bvec{E} + \bvec{P} \\
    \bvec{P} &= \chi \epsilon_0 \bvec{E}.
  \end{split}
\end{equation}

However, in the presence of matter, nonlinear effects arise and must be accounted for. We can write 
\begin{equation}
  \bvec{P} = \epsilon_0 \chi^{(1)} \bvec{E} + \bvec{P}_{\text{NL}} = \epsilon_0 \chi^{(1)} \bvec{E} + \chi^{(2)} \bvec{E}^2 + \chi^{(3)} \bvec{E}^3 + \ldots .
\end{equation}

\subsection{Anharmonic Oscillator}

The electric field of monochromatic light can be written as 
\begin{equation}
  \bvec{E}(t) = \frac{1}{2} \left[ \bvec{E}(\omega) e^{ -i \omega t} + \bvec{E}^*(-\omega) e^{i \omega t} \right],
\end{equation}
where \( \bvec{E}( \pm \omega) \) is the amplitude, \(\omega \) is the carrier frequency, and \( \bvec{E}(-\omega) \) is the negative-frequency component. By reality of \(\bvec{E}\), we have \( \bvec{E}^*(-\omega) = \bvec{E}^*(\omega) \). Therefore, we simply redefine \(\bvec{E}(\omega)\) and use 
\begin{equation}
  \bvec{E}(t) = \bvec{E}(\omega) e^{-i \omega t}.
\end{equation}

A simple model for light-matter interaction is the \textbf{Lorentz oscillator model}, which treats the electrons as damped classical oscillators and assumes harmonic fields. The equation of motion for a bound electron is given by 
\begin{equation}
  \frac{\mathrm{d}^2 \bvec{r}}{\mathrm{d}t^2} + \frac{1}{\tau} \frac{\mathrm{d}\bvec{r}}{\mathrm{d}t} + \omega_0^2 \bvec{r} = -\frac{e}{m} E(t).
\end{equation}
Here $ \bvec{r} $ is the displacement of the electron from rest position, $ \tau $ is the relaxation time, and $ \omega_0 $ is the resonance frequency of the oscillator. For time-harmonic fields, assume $ \bvec{E}(t) = \bvec{E}_0 e^{-i \omega t} $, giving 
\begin{equation}
  \bvec{r}(\omega) = - \frac{e / m}{\omega_0^{2} - \omega^{2} - i \omega / \tau} \bvec{E}(\omega).
\end{equation}

The polarization has the following relationship with the electric susceptibility (which we will simply denote by $ \chi $, since magnetic susceptibility is not of our concern here):
\begin{equation}
  \bvec{P} = Ne \bvec{r} = \epsilon_0 \chi \bvec{E},
\end{equation}
where \( N \) is the number density of oscillators. Then the susceptibility is found to be 
\begin{equation}
  \begin{split}
    \chi &= \frac{N e^{2}}{\epsilon_0 m} 1/ (\omega_0^{2} - \omega^{2} - i \omega / \tau) \\
    &\equiv \chi_{1} + i \chi_{2}.
  \end{split}
\end{equation}
The real and imaginary (electric) susceptibilities are
\begin{equation}
  \begin{split}
    \chi_1 &= \frac{N e^{2}}{\epsilon_0 m} \frac{\omega_0^{2} - \omega^{2}}{(\omega_0^{2} - \omega^{2})^{2} + (\omega / \tau)^{2}}, \\
    \chi_{2} &= \frac{N e^{2}}{\epsilon_0 m} \frac{\omega / \tau}{(\omega_0^{2} - \omega^{2})^{2} + (\omega / \tau)^{2}}.
  \end{split}
\end{equation}

\begin{eg}[Brendel-Bormann oscillator model]
  There is another more sophisticated model for modelling the electric susceptibility inside matter called the \textbf{Brendel-Bormann oscillator model}. This model takes into account the effects of the local field around the oscillators and provides a more accurate description of the nonlinear optical response of materials.
\end{eg}

This model does not give rise to any nonlinear effects or deviations. In the first days of the advent of laser, researchers observed unexpected phenomena that could not be explained by the Lorentz model, namely that unknown frequencies were shooting out of crystal samples.

However, when the electric field becomes strong enough, the assumption that fields obey harmonic relations is no longer valid. The nonlinear effects can be accounted by introducing higher order terms in the potential, which we will demonstrate with a $ x^{2} $ term in the potential of a 1D system. Then 
\begin{equation}
  \frac{\mathrm{d}^{2}x}{\mathrm{d}t^{2}} + \frac{1}{\tau} \frac{\mathrm{d}x}{\mathrm{d}t} + \omega_0^{2} x + \alpha x^{2} = -\frac{1}{m} F(t).
\end{equation}

We shall assume for now that the electric field is strong enough for nonlinear effects to emerge, but not too strong such that it still lies in the \textbf{perturbative regime}. 

\begin{note}
  When are nonlinear (anharmonic) effects important? This question can be answered with a simple estimation. We will assume $ \omega_0 \gg \omega_1, \omega_2 $, so that 
  \begin{equation}
    x \approx \frac{\omega_{0^{2}}}{\alpha} 
  \end{equation} 
  and the atomic binding electric field is 
  \begin{equation}
    E_{\text{atom}} = \frac{m \omega_0^{4}}{\alpha}.
  \end{equation}
\end{note}

\section{Second-Order Nonlinear Effects}

When the intensity of electromagnetic waves, or light beams, exceed a certain threshold, effects not describable by the linear Maxwell equations come into play.

We introduced the nonlinear correction term $ P_{\text{NL}} $ previously, now we will express the nonlinear contribution to polarization as a series of contracted tensors. 
\begin{equation}
  \begin{split}
    \bvec{P}(t) &= \epsilon_0 \chi^{(1)} \bvec{E}(t) + \bvec{P}_{\text{NL}}(t) \\ 
    &\equiv \epsilon_0 \chi^{(1)} \bvec{E}(t) + \bvec{P}^{(1)} + \bvec{P}^{(2)} + \cdots 
  \end{split}
\end{equation}
Each polarization can be expressed in terms of the electric field using the appropriate susceptibility tensor:

\begin{equation}
  \begin{split}
    \bvec{P}^{(1)} &= \epsilon_0 \chi^{(1)} \bvec{E}(t), \\
    \bvec{P}^{(2)} &= \epsilon_0 \chi^{(2)} \bvec{E}(t) \bvec{E}(t), \\
    \bvec{P}^{(3)} &= \epsilon_0 \chi^{(3)} \bvec{E}(t) \bvec{E}(t) \bvec{E}(t), \\
    &\vdots
  \end{split}
\end{equation}

\subsection{Sum- and Difference-Frequency Generation}
\begin{figure}[H]
    \centering
    \begin{subfigure}[b]{0.45\linewidth}
        \includegraphics[width=\linewidth]{Figures/SFG.png}
        \caption{Schematic diagram of sum-frequency generation (SFG).}
        \label{fig:sfg}
    \end{subfigure}
    \hfill
    \begin{subfigure}[b]{0.45\linewidth}
        \includegraphics[width=\linewidth]{Figures/DFG.png}
        \caption{Schematic diagram of difference-frequency generation (DFG).}
        \label{fig:dfg}
    \end{subfigure}
    \caption{}
    \label{fig:sfg_dfg}
\end{figure}

\subsection{Rectification}

\begin{note}[Creation of Tera-Hertz EM Waves]
  As an example of application, rectification can be used to generate Tera-Hertz electromagnetic waves.
\end{note}

\subsection{Second Harmonic Generation}

\subsection{Spontaneous Parametric Down-Conversion}
\textbf{Spontaneous parametric downconversion (SPDC)} is a second-order optical process in which a single high energy photon (a pump photon) splits into two daughter photons of lower energies (conventionally known as the signal and idler photons) \cite{boyd_nonlinear_2020}. It is a purely quantum process, so a complete treatment would require a full quantum description of the electromagnetic field.

\begin{figure}
    \centering
    \includegraphics[width=0.7\linewidth]{Figures/spdc.png}
    \caption{Illustration of the SPDC process.}
    \label{fig:spdc_schematics}
\end{figure}

\section{Third-Order Nonlinear Effects}
For noncentro-symmetric media, potential walls have both even and odd powers of position $ x $. Therefore, third-order effects are ubiquitous in materials, while second-order effects are only present in special materials where the symmetry of the potential allows for an even-powered correction. Here we shall list a few important examples of third-order effects.

\subsection{DC Kerr Effect}
\begin{equation}
  \chi^{(3)} = \chi^{(3)}(\omega_1 : \omega_1, 0, 0).
\end{equation}

\subsection{Self-Phase Modulation}
This effect describes the interaction of a light wave with its own electric field, leading to a change in the wave's phase and frequency. It is a key mechanism in the generation of supercontinuum light, which we will mention later.

\section{Symmetry and the Epsilon Tensor}
Date: 20250911
The epsilon tensor of lossless media is \textbf{hermitian} .

\section{Supercontinuum Generation}

Spectral broadening and the generation of new frequency components are inherent features of nonlinear optics, and have been studied intensively since the early 1960s. One particular case is supercontinuum generation (SC), which occurs when narrow-band incident pulses undergo extreme nonlinear spectral broadening to yield a broadband, often a spectrally continuous white light.

\subsection{Nonlinear Schrödinger Equation}
The nonlinear Schrödinger equation (NLSE) is a fundamental equation in nonlinear optics that describes the evolution of slowly varying envelopes of optical pulses in a nonlinear medium. It takes the form:

\begin{equation}
    i \frac{\partial A}{\partial z} + \frac{1}{2k_0} \nabla^2 A + \frac{n_2}{c} |A|^2 A = 0
\end{equation}

where \( A \) is the envelope of the optical field, \( k_0 \) is the linear wavevector, \( n_2 \) is the nonlinear refractive index, and \( c \) is the speed of light.

For light propagating in an optical fiber, we can include significant dispersion effects in the generalized nonlinear Schrödinger equation (GNLSE):

\begin{equation}
    \label{equ:gnlse}
    \frac{\partial A}{\partial z} + \frac{\alpha}{2}A + \sum_{m\ge 2} \frac{i^{\,m+1}\beta_m}{m!}\,\frac{\partial^{\,m}A}{\partial T^{\,m}} = i\gamma\!\left(1+\frac{i}{\omega_0}\frac{\partial}{\partial T}\right) \left[ A(T)\int_{-\infty}^{\infty} R(T')\,\lvert A(T-T')\rvert^{2}\, dT' \right].
\end{equation}

This equation takes into accounts the main effects taking place in a fiber during supercontinuum generation: Raman scattering, self-phase modulation, cross-phase modulation, four wave mixing, self-steepening, and dispersion. Raman scattering is considered a \textbf{delayed effect} , while the rest are \textbf{instantaneous \( \chi^({(3)}) \)-effects} .

\subsection{Self-Phase Modulation}

\subsection{Cross-Phase Modulation}

\subsection{Four-Wave Mixing}
Four-wave mixing between continuous waves (CW) waves is one of the most fundamental processes in nonlinear optics. In the absence of any initial seeding (what does this mean?), four-wave mixing corresponds to an instability of the propagating CW pump, and growth from noise of sidebands symmetric in frequency about the pump can be observed.

Four-wave mixing and modulation instability are fequency- and time-domain description of identical physics. 

\subsection{Self-Steepening}

\subsection{Raman Scattering}

Soliton fission

\subsection{Dispersion}

\subsection{Simulation}
\lstset{language=Matlab}
When making simulations on the SCG effect, just as in the case of any quantitative simulations, the correct colormap should be chosen. In particular, the \lstinline|jet| colormap is not perceptually uniform, i.e. the color transitions are not evenly spaced from a human perception perspective, and may lead to false features in the plot.

The code is based on code given in \cite{scg_in_photonics_2006}. First a separate function is defined to propagate the light through the fiber with the GNLSE, given by equation (\ref{equ:gnlse}).

\begin{lstlisting}[language=Matlab]
function [Z, AT, AW, W] = gnlse(T, A, w0, gamma, betas, ...
                             loss, fr, RT, flength, nsaves)

n = length(T); dT = T(2)-T(1); % grid parameters
V = 2*pi*(-n/2:n/2-1)'/(n*dT); % frequency grid
alpha = log(10.^(loss/10));    % attenuation coefficient

B = 0;
for i = 1:length(betas)        % Taylor expansion of betas
  B = B + betas(i)/factorial(i+1).*V.^(i+1);
end
L = 1i*B - alpha/2;            % linear operator

if abs(w0) > eps               % if w0>0 then include shock
    gamma = gamma/w0;    
    W = V + w0;                % for shock W is true freq
else
    W = 1;                     % set W to 1 when no shock
end

RW = n*ifft(fftshift(RT.'));   % frequency domain Raman
L = fftshift(L); W = fftshift(W); % shift to fft space

% === define function to return the RHS of Eq. (3.13)
function R = rhs(z, AW)
  AT = fft(AW.*exp(L*z));         % time domain field
  IT = abs(AT).^2;                % time domain intensity
  if (isscalar(RT)) || (abs(fr) < eps) % no Raman case
    M = ifft(AT.*IT);             % response function
  else
    RS = dT*fr*fft(ifft(IT).*RW); % Raman convolution
    M = ifft(AT.*((1-fr).*IT + RS));% response function
  end
  R = 1i*gamma*W.*M.*exp(-L*z);   % full RHS of Eq. (3.13)
end

% === define function to print ODE integrator status
function status = report(z, ~, flag) % 
  status = 0;
  if isempty(flag)
    fprintf('%05.1f %% complete\n', z/flength*100);
  end
end

% === setup and run the ODE integrator
Z = linspace(0, flength, nsaves);  % select output z points
% === set error control options
options = odeset('RelTol', 1e-5, 'AbsTol', 1e-12, ...
                 'NormControl', 'on', ...
                 'OutputFcn', @report);
[Z, AW] = ode45(@rhs, Z, ifft(A), options); % run integrator

% === process output of integrator
AT = zeros(size(AW(1,:)));
for i = 1:length(AW(:,1))
  AW(i,:) = AW(i,:).*exp(L.'*Z(i)); % change variables
  AT(i,:) = fft(AW(i,:));           % time domain output
  AW(i,:) = fftshift(AW(i,:)).*dT*n;  % scale
end

W = V + w0; % the absolute frequency grid
end
\end{lstlisting}

Then the desired plots are generated with the appropriate parameters. Dispersion coefficients up to 9th order are included in the simulation, with the 9th-order term taking on a value as small as \( -1.7140e-144 \).

\begin{lstlisting}
% === numerical grid ===
n = 2^13;                   % number of grid points
twidth = 12.5e-12;          % width of time window [s]
c = 299792458;              % speed of light [m/s]
wavelength = 835e-9;        % reference wavelength [m]
w0 = (2*pi*c)/wavelength;   % reference frequency [Hz]
dt = twidth/n;
T = (-n/2:n/2 - 1).*dt; % time grid

% === input pulse ===
power = 10000;              % peak power of input [W]
t0 = 28.4e-15;              % duration of input [s]
A = sqrt(power)*sech(T/t0); % input field [W^(1/2)]

% === fibre parameters ===
flength = 0.15;             % fibre length [m]
% betas = [beta2, beta3, ...] in units [s^2/m, s^3/m ...]
betas = [-1.1830e-026, 8.1038e-041, -9.5205e-056,  2.0737e-070, ...
         -5.3943e-085,  1.3486e-099, -2.5495e-114,  3.0524e-129, ...
         -1.7140e-144];
gamma = 0.11;               % nonlinear coefficient [1/W/m]
loss = 0;                   % loss [dB/m]

% === Raman response ===
fr = 0.18;                  % fractional Raman contribution
tau1 = 0.0122e-12; tau2 = 0.032e-12;
RT = (tau1^2+tau2^2)/tau1/tau2^2*exp(-T/tau2).*sin(T/tau1);
RT(T<0) = 0;                % heaviside step function

% === simulation parameters ===
nsaves = 200;     % number of length steps to save field at

% propagate field
[Z, AT, AW, W] = gnlse(T, A, w0, gamma, betas, loss, ...
                       fr, RT, flength, nsaves);
                   
% === plot output ===
figure();
WL = 2*pi*c./W; iis = (WL>450e-9 & WL<1350e-9); % wavelength grid
lIW = 10*log10(abs(AW).^2 .* 2*pi*c./WL'.^2); % log scale spectral intensity
mlIW = max(max(lIW));       % max value, for scaling plot

ax1 = subplot(1,2,1);
C = lIW(:,iis);
C = C - max(C, [], 'all');                 % now max(C) = 0 dB
pcolor(ax1, WL(iis).*1e9, Z, C); 
shading(ax1,'interp');
clim(ax1, [-40, 0]); xlim([450,1350]);
xlabel(ax1, 'Wavelength / nm'); ylabel(ax1, 'Distance / m');
colormap(ax1, turbo);
cb = colorbar(ax1,'eastoutside'); 
cb.Label.String = 'Spectral intensity (dB rel. max)';
set(gca,'YDir','normal','TickDir','out'); axis tight
cb.Ticks = -40:10:0;

lIT = 10*log10(abs(AT).^2); % log scale temporal intensity
mlIT = max(max(lIT));     
ax2 = subplot(1,2,2);
pcolor(ax2, T.*1e12, Z, lIT);    % plot as pseudocolor map
clim(ax2, [mlIT-40.0, mlIT]);  
xlim(ax2, [-0.5,5]); 
shading interp;
xlabel('Delay / ps'); ylabel('Distance / m');
colormap(ax2, turbo);
cb = colorbar(ax2,'eastoutside'); 
cb.Label.String = 'Spectral intensity (dB rel. max)';
cb.Ticks = -40:10:0;
\end{lstlisting}

\section{Nonlinear Interaction Reduce Noise}
\begin{eg}[Three modes with filter]
  \begin{equation}
    \begin{split}
      \text{Noise} &= 
      \begin{pmatrix}
        \alpha_1^* & \alpha_2^* & \alpha_3^* & \alpha_1 & \alpha_2 & \alpha_3 \\
      \end{pmatrix}
      \begin{pmatrix}
        0 & 0 & 0 & c_1 + 1 & c_2 & c_3 \\
        0 & 0 & 0 & c_4 & c_5 + 1 & c_6 \\
        0 & 0 & 0 & c_7 & c_8 & c_9 + 1 \\
        c_1 & c_2 & c_3 & 0 & 0 & 0 \\
        c_4 & c_5 & c_6 & 0 & 0 & 0 \\
        c_7 & c_8 & c_9 & 0 & 0 & 0 \\
      \end{pmatrix}
      \begin{pmatrix}
        \alpha_1^* \\ \alpha_2^* \\ \alpha_3^* \\ \alpha_1 \\ \alpha_2 \\ \alpha_3 \\
      \end{pmatrix} \\
      &= (2c_1 + 1)\vert \alpha_1 \vert^2 + (2c_5 + 1)\vert \alpha_2 \vert^2 + (2c_9 + 1)\vert \alpha_3 \vert^2 \\
      &+ 2 \mathrm{Re} \left[ c_2 \alpha_1^* \alpha_2 + c_3 \alpha_1^* \alpha_3 + c_4 \alpha_2^* \alpha_1 + c_6 \alpha_3^* \alpha_2 + c_7 \alpha_3^* \alpha_1 + c_8 \alpha_3^* \alpha_2 \right]. 
    \end{split}
  \end{equation}
  The terms with moduli squared are the vacuum fluctuations, and the cross terms are the correlations between different modes. If we apply a filter that blocks out mode $ \alpha_3 $, then the noise becomes
  \begin{equation}
    \text{Noise}^{\prime} = (2c_1 + 1)\vert \alpha_1 \vert^{2} + (2c_5 + 1)\vert \alpha_2 \vert^{2} + 2 \mathrm{Re} \left[ c_2 \alpha_1^* \alpha_2 + c_4 \alpha_2^* \alpha_1 \right].
  \end{equation}
\end{eg}