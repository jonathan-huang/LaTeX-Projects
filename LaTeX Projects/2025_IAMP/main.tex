\documentclass[12pt, a4paper, twoside]{article}
\usepackage{scrextend}
    %% font & format %%
\usepackage[margin=3cm]{geometry}
\usepackage{type1cm, titlesec, fancyhdr, titling}
    %% Math, Logos & symbols %%
\usepackage{amsmath,amsthm,amssymb, mathtools}
\usepackage{yhmath, faktor, dsfont}

\usepackage{array} % tables
\usepackage[shortlabels]{enumitem}
\usepackage[normalem]{ulem}
\usepackage{mathrsfs}
\usepackage{indentfirst}
\usepackage{pifont}
\usepackage{fancyhdr}   
\usepackage{gensymb}
\usepackage{amssymb}
\usepackage{pgfplots}
\pgfplotsset{compat=1.15}

% Mandarin
\usepackage{CJKutf8}
% bkai = 標楷體
% bsmi = 新細明體

%% Enhancement %%
\usepackage{graphicx, tabularx}

% paragraph
\usepackage{parskip}
\setlength{\parindent}{2em}

\setlength{\headheight}{15pt}
\setlength{\droptitle}{-1.5cm}
\parindent=24pt

\newtheoremstyle{mystyle}
  {6pt}{15pt}
  {}%
  {}%
  {\bf}% 
  {.}%
  {1em}%
  {}% Theorem head spec (can be left empty, meaning 'normal')

\theoremstyle{mystyle}	
\newtheorem{theorem}{Theorem}
\newtheorem*{definition}{Definition}
\newtheorem{example}[theorem]{Example}
\newtheorem{exercise}{Exercise}
\newtheorem{solution}{Solution}
\newtheorem{corollary}[theorem]{Corollary}
\newtheorem{property}[theorem]{Property}
\newtheorem{proposition}[theorem]{Proposition}
\newtheorem{lemma}[theorem]{Lemma}
\newtheorem{problem}[theorem]{Problem}
\newtheorem*{remark}{Remark}
\newtheorem*{claim}{Claim}

\begin{document}
\begin{CJK}{UTF8}{bkai}

\title{原子與分子科學研究所暑期實習申請自傳}
\author{台大物理系大二 \; 黃紹凱}
\date{\today}
% header and footer
\pagestyle{fancy}
\rhead{黃紹凱}
\lhead{原子與分子科學研究所暑期實習}
\cfoot{\thepage}

\maketitle

\begin{itemize}
    \item 姓名:黃紹凱
    \item 就讀學校/系所/年級:臺灣大學物理學系二年級
    
    \begin{figure}[htbp]
        \centering
        \includegraphics[width=6cm]{photo 01.jpg}
        \caption{我(2024)}
    \end{figure}
\end{itemize}

\section{個人特質、個性闡述}

個性的部分我是一個滿慢熟的人,不過從別人反饋中得知自己在談話上還算熱絡,也會主動爭取自己想要的東西。在臺大物理系學習兩年來,發現系上充滿形形色色的人,也開始習慣和喜歡這樣有趣的生態。我現在是台大單車社的活躍社員,同時修習物理系和數學系(準備未來申請雙主修)的課業之餘,跟著社團騎腳踏車上山下海,是目前非學術生活的重心。學術生活的重心則包含物理所的課程,物理系系學會學術部以及今年加入的中研院分生所生物物理實驗室。雖然如此多面向的生活有些時候讓我喘不過氣,但更多時候這些挑戰讓我對生活充滿鬥氣和信心。

\section{能力簡述}
我的研究經驗包含了今年剛加入的分生所實驗室,除了有經驗以外也是我的興趣所在,因此本次實習申請的實驗室大多是生物物理研究相關。我在分生所老師指導下進行生物物理相關的理論專題,目前每個禮拜會播出一天的時間實際到中研院和老師進行討論,也會在實驗室 meeting 中進行論文報告,累積了不少閱讀、整理文獻的經驗。

學校課業的部分,我高中時即參加 UC Berkley 的線上遠距 REYES 課程,大一以第一名正取台大物理系。大二時由於我已經修完大部分物理系必修課,我將學習重心放在研讀物理所課程以及數學系必修,期望未來可以成功申請雙主修。我在物理所修過量子力學及密度泛函導論,身為班上唯一的大二生,我最後在期末口頭報告拿到不錯的成績,成功將成績拉到 A+。本學期我在系上修習德研究所課程包含量子光學導論、廣義相對論及量子場論,雖然跟生物物理較沒有直接關係,但我認為這是對自己學習能力及時間管理能力很好的挑戰,場論的背景更是能夠應用在物理學的各個領域。

\section{申請動機}
我希望申請此計畫的原因主要有兩個,首先是臺大物理系中生物物理相關的師資相對其他物理領域較少,中研院的相關資源則豐富很多(這也是當時決定進入分生所做專題的原因),而中研院原分所也是國內物理研究的頂尖機構,包含我很有興趣的細胞生物物理及 AI 在生物學研究上的應用,同時也有機會接觸到各式各樣的研究主題。最後,另外一個使我考慮原分所的原因是通勤便利性,因為原分所位置就在臺大校本部內,通勤上與中研院南港院區相對來說方便許多。

\section{研究興趣 \& 學習計畫}
我希望暑期可以在原分所進行生物物理相關的專題研究,同時學習數學及程式模擬等相關的實用技能。大一時我曾修過數學所的機器學習課程,雖然相關的經驗比較少,但我也因此開始對機器學習及 AI 在物理研究上的應用感興趣,而這應該也是近期非常熱門的研究主題之一。

在申請確定之前我會繼續做相關的準備,在分生所繼續原來的專題,累積自己的研究及報告經驗。在學校我也會專注在課業上,妥善利用時間讓這學期 GPA 維持在 4.10/4.30 以上。獲選後我則會主動聯絡指導老師,與老師擬定適合的時間規劃,又因為原分所位置就在台大校內,我也會積極爭去是否可以提早進入實驗室了解環境。

\end{CJK}
\end{document}