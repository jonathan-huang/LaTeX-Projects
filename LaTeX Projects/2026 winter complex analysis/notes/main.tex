\documentclass[12pt, twoside]{article}
%\documentclass[a4paper]{article}

\usepackage{import}
\usepackage{xifthen}
\usepackage{pdfpages}
\usepackage{transparent}

\newcommand{\incfig}[1]{%
    \def\svgwidth{\columnwidth}
    \import{./Figures/}{#1.pdf_tex}
}

\usepackage{scrextend}
\usepackage{geometry}
    \geometry{
        a4paper,
        total={170mm,257mm},
        left=20mm,
        top=20mm,
    }

% table settings
\usepackage{multirow} 
\def\arraystretch{2} % 1 is the default, change whatever you need

\usepackage[version=4,arrows=pgf-filled,
textfontname=sffamily,
mathfontname=mathsf]{mhchem} % chemistry
\usepackage{amsmath}
\usepackage[shortlabels]{enumitem} % enumerate with letters
\usepackage[normalem]{ulem}
\usepackage{indentfirst}
\usepackage{pifont}
\usepackage{fancyhdr}   % 頁首頁尾
\usepackage{amssymb}

% Mandarin
\usepackage{CJKutf8}
% bkai = 標楷體
% bsmi = 新細明體

\usepackage{graphicx} % Required for inserting images

%
\usepackage{lmodern,bm}                
\usepackage[T1]{sansmath} 
\SetMathAlphabet{\mathsfbf}{sans}{\sansmathencoding}{\sfdefault}{bx}{sl}
\usepackage{etoolbox}
\AtBeginEnvironment{sansmath}{\let\bm\mathsfbf}{}{}
\usepackage{mdframed}
\usepackage{braket} % braket notation
\usepackage{amsmath, nccmath}
\usepackage{mathtools} % for text above and under arrows

\usepackage{bbm}
\usepackage{dsfont} % bold numerals
\usepackage{bbold} % blackboard bold font
\usepackage{mathrsfs} % script

\usepackage{url}

% set to equal
\newcommand{\defeq}{\vcentcolon=}
\newcommand{\eqdef}{=\vcentcolon}

% circled number
\newcommand*\circlednum[1]{\raisebox{.5pt}{\textcircled{\raisebox{-.9pt} {#1}}}}
\newcommand*\circled[1]{\tikz[baseline=(char.base)]{
            \node[shape=circle,draw,inner sep=1pt] (char) {#1};}}

% question mark styling
\newcommand{\?}{\stackrel{?}{=}}

\usepackage{relsize} % math symbol size
\usepackage{amsthm}

\newcommand{\daggerequ}{\mathrel{\overset{\makebox[0pt]{\mbox{\normalfont\tiny\sffamily $\dagger$}}}{=}}} % equal with dagger on top
\newcommand{\ddaggerequ}{\mathrel{\overset{\makebox[0pt]{\mbox{\normalfont\tiny\sffamily $\ddagger$}}}{=}}} % equal with ddagger on top

\newtheorem{theorem}{Theorem}[section]
\newtheorem{definition}{Definition}[section]
\newtheorem{lemma}[theorem]{Lemma}
\newtheorem{proposition}[theorem]{Proposition}
\newtheorem{corollary}[theorem]{Corollary}
%\newtheorem{property}{Property}

\newtheoremstyle{problemstyle}
        {5pt} % <space above>
        {15pt} % <space below>
        {\normalfont} % <body font>
        {} % <indent amount}
        {\bfseries} % <theorem head font>
        {\normalfont\bfseries.} % <punctuation after theorem head>
        {.5em} % <space after theorem head>
        {} % <theorem head spec (can be left empty, meaning `normal')>
\theoremstyle{problemstyle}
\newtheorem{problem}{Problem}[section]
\newtheorem{solution}{Solution}[section]
\newtheorem{example}{Example}[section]
\newtheorem{exercise}{Exercise}[section]

\theoremstyle{remark}
\newtheorem*{remark}{Remark}

\usepackage{mdframed}
\newenvironment{fthm}
    {\begin{mdframed}\begin{thm}}
    {\end{thm}\end{mdframed}}

\newcommand{\bvec}[1]{\mathbf{#1}} % vector

% figures
\usepackage{wrapfig} 
\usepackage{subfig}

% ckickable table of contents
\usepackage{hyperref}
\hypersetup{
    colorlinks=true,
    citecolor=black,
    filecolor=black,
    linkcolor=cyan,
    urlcolor=blue,
    pdftitle={notes on classical gravitation}, % opened pdf settings
    pdfpagemode=FullScreen,
}


\begin{document}
\begin{CJK}{UTF8}{bkai}

\title{
    UWaterloo 2026 Winter PMATH352 - Complex Analysis
    \\ 
    \vspace{1cm}
    \large Instructor: Professor \textbf{Jason P. Bell} 
    \\ 
    \large Student: \textbf{Shao-Kai Jonathan Huang}    
}
% \author{Shao-Kai Jonathan Huang}
\date{\today}
\cfoot{\thepage}
% header and footer
\pagestyle{fancy}
\setlength{\headheight}{15pt}
\rhead{Shao-Kai Jonathan Huang}
\lhead{PMATH352 Complex Analysis}

\maketitle
\tableofcontents

\newpage
\section{Complex Differentiation}
Date: 12/1/2026

\begin{theorem}[Cauchy-Riemann]
    Let $ f $ be a differentiable function from an open set $ U \subseteq \mathbb{C} $ to $ \mathbb{C} $. Write $ f(z) = u(x,y) + i v(x,y) $, where $ z = x + iy $, $ u,v : U \to \mathbb{R} $. Then the partial derivatives of $ u $ and $ v $ satisfy the \emph{Cauchy-Riemann equations} given by 
    \begin{equation}
        \frac{\partial u}{\partial x} = \frac{\partial v}{\partial y}, \quad \frac{\partial u}{\partial y} = - \frac{\partial v}{\partial x}.
    \end{equation}
\end{theorem}

This is a necessary condition for complex differentiability. To see why it is not sufficient, we can consider the following example.

\begin{example}[Non-differentiability]
    Let $ f(z) $ be defined by $ z^5 / |z|^4 $ when $ z \neq 0 $ 
\end{example}

\newpage 
Date: 14/1/2026

\begin{theorem}
    If $ f(z) = \sum a_n z^n \in \mathbb{C}[[z]] $ has radius of convergence $ R>0 $ and if $ g(z) = \sum+{n} n a_n z^{n-1} $. Then $ g(z) $ converges in $ B(0,R) $ and $ f^{\prime} (z) = g(z) $ for all $ z \in B(0,R) $.   
\end{theorem}
\begin{proof}
    We know from previous result that $ g(z) $ has radius of convergence $ R $. Now fix $ z_0 \in B(0,R) $. There exists $ \delta >0 $ such that $ B(z_0 , \delta) \subseteq B(0,R) $, so for all $ h $ such that $ |h| < \delta $, $ z_0 + h \in B(0,R) $. Let $ h \in \mathbb{C} $, $ 0<|h|<\delta $, consider 
    \begin{equation*}
        \begin{split}
            \left\vert \frac{f(z_0 + h) - f(z_0)}{h} - g(z_0) \right\vert &= \frac{\sum_{n=0}^{\infty} a_n (z_0 + h)^n - \sum_{n=0}^{\infty} a_n z_0^n}{h} - \sum_{n=1}^{\infty} n a_n z_0^{n-1} \\
            &= \left\vert \sum_{n=1}^{\infty} a_n \frac{(z_0 + h)^n - z_0^n}{h} - \sum_{n=1}^{\infty} n a_n z_0^{n-1} \right\vert \\
            &= \left\vert \sum_{n=1}^{\infty} a_n \left( \frac{(z_0 + h)^n - z_0^{n}}{h} - n z_0^{n-1} \right) \right\vert \\
            &= \left\vert \sum_{n=0}^{\infty} a_n \left[ \sum_{j=2}^{n} \binom{n}{j} z_0^{n-j} h^{j-1} \right] \right\vert \\
            &\leq \sum_{n=0}^{\infty} |a_n| \sum_{j=2}^{n} \binom{n}{j} |z_0|^{n-j} |h|^{j-1} \\
            &\leq |h| \sum_{n=0}^{\infty} |a_n| \sum_{l=0}^{n-2} \binom{n}{2} \binom{n-2}{l} |z_0|^{n-2-l} |h|^l, \quad l = j-2 \\
            &= |h| \sum_{n=2}^{\infty} |a_n| \binom{n}{2} (|z_0| + |h|)^{n-2} \\
            &\leq \frac{|h|}{2} \sum_{n=2}^{\infty} |a_n| n (n-1) \rho^{n-2} = C |h| \to 0, \quad \text{as } h \to 0. 
        \end{split}
    \end{equation*}
    In the above derivation, we used the fact that the series $ \sum_{n=0}^{\infty} $ is absolutely convergent, $ |z_0| + |h| < |z_0| + \delta \leq \rho < R $. Therefore, the limit
    \begin{equation*}
        \lim_{h \to 0} \left\vert \frac{f(z_0 + h) - f(z_0)}{h} \right\vert
    \end{equation*}
    existss and is equal to $ g(z_0) $. Hence, for all $ z \in B(0,R) $, we have 
    \begin{equation*}
        f^{\prime} (z) = \sum_{n=1}^{\infty} n a_n z^{n-1}. 
    \end{equation*}  
\end{proof}

\begin{corollary}
    If $ f(z) = \sum_{n=0}^{\infty} a_n z^n $ is convergent in $ B(0,R) $, then $ f^{(j)} (z) $ exists and 
    \begin{equation*}
        \frac{f^{(j)}}{j!} = \sum_{n=j}^{\infty} \binom{n}{j} a_n z^{n-j}, \quad \text{for all } z \in B(0,R).
    \end{equation*}  
\end{corollary}

\begin{definition}
    We define 
    \begin{equation}
        \exp z = \sum_{n=0}^{\infty} \frac{z^n}{n!}, \quad \text{for all } z \in \mathbb{C}. 
    \end{equation}
    This power series has $ R = \infty $.
\end{definition}

\begin{theorem}
    The power series $ \exp z $ has the following properties:
    \begin{enumerate}[(i)]
        \item $ \exp x = e^x > 0 $ for all $ x \in \mathbb{R} $.
        \item $ \exp \overline{z} = \overline{\exp z} $ for all $ z \in \mathbb{C} $.
        \item $ \exp (z + w) = \exp z \cdot \exp w $ for all $ z,w \in \mathbb{C} $.
        \item $ \frac{d}{dz} \exp z = \exp z $ for all $ z \in \mathbb{C} $.
        \item $ \left\vert \exp z \right\vert = e^{\operatorname{Re}(z)} $ for all $ z \in \mathbb{C} $.  
    \end{enumerate}
\end{theorem}
\begin{proof}
    ~

    \begin{enumerate}[(i)]
        \item For $ x \in \mathbb{R} $, we have $ \exp x = \sum_{n=0}^{\infty} \frac{x^n}{n!} = e^x > 0 $, by definition.
        \item If $ S_N = \sum_{n=0}^N a_n \to S $, then $ \overline{S_N} = \overline{\sum_{n=0}^N a_n} = \sum_{n=0}^N \overline{a_n} \to \overline{S} $. Therefore, $ \exp \overline{z} = \overline{\exp z} $.
        \item Let $ c \in \mathbb{C} $ and $ f(z) = \exp (c-z) \exp z $. Assuming (iV) to be true, we have 
        \begin{equation*}
            \frac{\mathrm{d}}{\mathrm{d}z} f(z) = \exp (c-z) \exp z - \exp (c-z) \exp z = 0.
        \end{equation*}
        Therefore, $ f(z) $ is constant. Evaluating at $ z=0 $, we have $ f(z) = f(0) = \exp c $. Let $ c=z+w $, we get the desired result. 
        \item By the corollary above, we have
        \begin{equation*}
            \frac{\mathrm{d}}{\mathrm{d}z} \exp z = \sum_{n=1}^{\infty} \frac{n}{n!} z^{n-1} = \sum_{n=0}^{\infty} \frac{z^n}{n!} = \exp z.
        \end{equation*}
        \item we have 
        \begin{equation*}
            \begin{split}
                \left\vert \exp (z) \right\vert^2 &= \exp z \overline{ \exp z} = \exp z \exp \overline{z} = \exp (z + \overline{z}) = \exp (2 \operatorname{Re}(z)) = e^{2 \operatorname{Re}(z)}.
            \end{split}
        \end{equation*}
        Therefore, $ \left\vert \exp z \right\vert = e^{\operatorname{Re}(z)} $.
    \end{enumerate}
\end{proof}

\begin{definition}[Trigonometric functions]
    We also define 
    \begin{equation}
        \sin z = \sum_{n=0}^{\infty} (-1)^n \frac{z^{2n+1}}{(2n+1)!}, \quad \cos z = \sum_{n=0}^{\infty} (-1)^n \frac{z^{2n}}{(2n)!}, \quad \text{for all } z \in \mathbb{C}.
    \end{equation}
    These power series have $ R = \infty $.
\end{definition}

\begin{corollary}
    For all $ z \in \mathbb{C} $, we have 
    \begin{equation}
        \frac{\mathrm{d}}{\mathrm{d}z} \sin z = \cos z, \quad \frac{\mathrm{d}}{\mathrm{d}z} \cos z = - \sin z.
    \end{equation} 
\end{corollary}

\begin{corollary}[Euler's identity]
    For all $ z \in \mathbb{C} $, we have 
    \begin{equation}
        \exp (iz) = \cos z + i \sin z.
    \end{equation}
\end{corollary}
\begin{proof}
    Notice that $ \exp (iz) = e^{iz} = 1 + iz + \frac{(iz)^2}{2!} + \frac{(iz)^3}{3!} + \cdots $ absolutely converges, so we may rearrange the terms as we wish.
\end{proof}

\end{CJK}
\end{document}