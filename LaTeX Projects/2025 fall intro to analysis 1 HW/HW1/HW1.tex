\documentclass[a4paper]{article}
%% Formatting %%
\usepackage[margin=3cm]{geometry}
\usepackage{type1cm, titlesec, fancyhdr, titling}
\usepackage{multicol}
\usepackage[dvipsnames]{xcolor}
\usepackage{ulem}
\usepackage{parskip}
\setlength{\parindent}{2em}
\setlength{\headheight}{15pt}
\setlength{\droptitle}{-1.5cm}
\parindent=24pt
%% Math and Symbols %%
\usepackage{amsmath,amsthm,amssymb, mathtools}
\usepackage{yhmath, faktor, dsfont}
\usepackage{academicons, wasysym, marvosym}
\usepackage[scr]{rsfso} 
\usepackage{latexsym, amsmath, amscd, amsmath, amsthm}
\usepackage{amssymb,amsmath,amsthm,graphicx,dsfont}
\usepackage{hyperref}

%% Enhancement %%
\usepackage{graphicx, tabularx}
\usepackage[shortlabels,inline]{enumitem}
%% TikZ %%
\usepackage{tikz-cd}
\usepackage[breakable]{tcolorbox}
\usetikzlibrary{decorations.pathmorphing}
\usetikzlibrary{calc, arrows,matrix}

%% Other packages %%
\usepackage{amsopn}

%% Traditional Chinese %%
\usepackage{CJKutf8}

%% Math environments %%
\newtheoremstyle{mystyle}
  {6pt}{15pt}% 上下間距
  {}%          內文字體
  {}%              縮排
  {\bf}%       標頭字體
  {.}%       標頭後標點
  {1em}% 內文與標頭距離
  {}% Theorem head spec (can be left empty, meaning 'normal')
\theoremstyle{mystyle}	
\newtheorem{theorem}{Theorem}
\newtheorem*{definition}{Definition}
\newtheorem{example}[theorem]{Example}
\newtheorem{exercise}{Exercise}
\newtheorem{solution}{Solution}
\newtheorem{corollary}[theorem]{Corollary}
\newtheorem{property}[theorem]{Property}
\newtheorem{proposition}[theorem]{Proposition}
\newtheorem{lemma}[theorem]{Lemma}
\newtheorem{problem}[theorem]{Problem}
\newtheorem{answer}{Answer}[section]
\newtheorem{fact}[theorem]{fact}
\newtheorem*{recall}{Recall}
\newtheorem*{remark}{Remark}
\newtheorem*{claim}{Claim}
\newtheorem*{observation}{Observation}

\begin{document}
\begin{CJK}{UTF8}{bkai}

\title{%
  \textbf{Math 2213 Introduction to Analysis} \\
  \vspace{0.5cm}
  \large Homework 1 Due September 10 (Thursday), 2025
}
\author{物理、數學三 黃紹凱 B12202004}
\date{\today}

\maketitle

Below is the definition of a metric from the lecture notes.
\begin{definition}[metric]
  A function $d : X \times X \to [0,\infty)$ is called a \emph{metric} on $X$ if, for all $x,y,z \in X$, the following properties hold:
  \begin{enumerate}[(i)]
    \item For any $ x \in X $, we have $d(x,x) = 0$. 
    \item (Positivity) For any distinct $ x, y \in X$, we have $d(x,y) > 0$.
    \item (Symmetry) For any $ x, y \in X$, we have $d(x,y) = d(y,x)$.
    \item (Triangle Inequality) For any $ x, y, z \in X$, we have $d(x,z) \leq d(x,y) + d(y,z)$.
  \end{enumerate}
\end{definition}

% Problem 1
\begin{problem}[(10 pts) Dyadic density via the Archimedean property]
  ~

  \label{ex:dyadic-density}
  Let $a<b$ be real numbers. Prove that there exists a \emph{dyadic rational}
  \[
  q=\frac{k}{2^n}\in\mathbb{Q}\qquad(k \in \mathbb{Z},n\in\mathbb{N})
  \]
  such that $a<q<b$.
  Further show that there are \emph{infinitely many} such dyadic rationals in $(a,b)$.
\end{problem}

\smallskip

\begin{solution}
  Let $ L = b - a > 0$. Notice that $ 2^n > n $ for all natural numbers $ n \geq 1 $, which derives from induction as follows: $ 2^1 > 1 $ and $ 2^{n+1} = 2 \cdot 2^n > 2 \cdot n = n + n > n + 1 $ for all $ n \geq 1 $. By the Archimedean property, there exists a natural number $n \geq 1$ such that $ 2^n L > nL > 1 $, hence $ \frac{1}{2^n} < L $.

  Let $ S_n = \{ m \in \mathbb{Z} \mid m > 2^n a\} $. Since $ S_n $ is a nonempty set of integers bounded from below, it has a minimal element, say $ k = \operatorname{inf}(S_n) \in \mathbb{Z}$. Then we have $ k > 2^n a $, $ k - 1 \leq 2^n a $, $ 2^n a + 1 < 2^n b $, so 
  \[
    2^n a < k \leq 2^n a + 1 < 2^n b.
  \]
  Dividing by $ 2^n $ gives the desired dyadic rational. 

  To show that there are infinitely many such dyadic rationals in $(a,b)$, we note that we can take any natural number $ n^{\prime} \geq n $, where $ n $ is some natural number satisfying $ 1 / 2^n < L $ found above. Then by the same argument, we can find a dyadic rational $ q' = k' / 2^{n'} \in (a,b) $, where $ k' = \operatorname{inf}(\{ m \in \mathbb{Z} \mid m > 2^{n'} a\}) $. Since there are infinitely many natural numbers, hence infinitely many choices of $ n' $, there are infinitely many such dyadic rationals in $(a,b)$.  
\end{solution}

\medskip

% Problem 2
\begin{problem}[A tour of the $p$-adic world]
  ~

  The field $\mathbb{Q}$ inherits the Euclidean metric from $\mathbb{R}$, but it also carries a very different metric: the \emph{$p$-adic metric}.  Given a prime number $p$ and an integer $n$, the $p$-adic norm of $n$ is defined as
  \[
  |n|_p = \frac{1}{p^k},
  \]
  where $p^k$ is the largest power of $p$ dividing $n$.  
  (We define $|0|_p := 0$.)  
  The more factors of $p$ appear in $n$, the smaller the $p$-adic norm becomes. 
  
  For a rational number $x=\tfrac{a}{b}$ with $a,b\in\mathbb{Z}$, we may factor $x$ as
  \[
  x = p^k \cdot \frac{r}{s},
  \]
  where $k \in \mathbb{Z}$ and $p$ divides neither $r$ nor $s$. We then define
  \[
  |x|_p = p^{-k}.
  \]
  The $p$-adic metric on $\mathbb{Q}$ is given by
  \[
  d_p(x,y) := |x-y|_p.
  \]

  \begin{enumerate}

  \item To compute the $5$-adic norm $|x|_5$ of a rational number $x$, 
  we examine how many factors of $5$ occur in $x$ (in either numerator or denominator). If $x=5^k \cdot \tfrac{a}{b}$ with $a,b$ not divisible by $5$ and $k\in \mathbb{Z}$, 
  then the $5$-adic norm is
  \[
  |x|_5 = 5^{-k}.
  \]
  For example:
  \begin{enumerate}
    \item $30 = 2 \cdot 3 \cdot 5$. There is exactly one factor of $5$, so
    \[|30|_5 = 5^{-1} = \tfrac{1}{5}.\]

    \item $32 = 2^5$. There is no factor of $5$, so
    \[|32|_5 = 5^{0} = 1.\]

    \item Compute $\left|\tfrac{1}{250}\right|_5$.
    \[ 250 = 2 \cdot 5^3. \]

    So
    \[ \frac{1}{250} = \frac{1}{2 \cdot 5^3}  = 5^{-3} \cdot \frac{1}{2},
    \]
    where $\tfrac{1}{2}$ has no factor of $5$ in numerator or denominator. Therefore,
    \[\left|\tfrac{1}{250}\right|_5 = 5^{-(-3)} = 5^3 = 125.\]

    Hence,
    \[\boxed{\;\;\left|\tfrac{1}{250}\right|_5 = 125.\;\;} \]
  \end{enumerate}
  
  Now practice computing the following $5$-adic norms:
  (6 pts) 
  \begin{enumerate}
    \item $|75|_5$
    \item $\left|\tfrac{10}{9}\right|_5$
    \item $\left|-\tfrac{20}{375}\right|_5$
    \end{enumerate}

    \medskip

    \item  (9 pts)  Further properties of the $5$-adic norm.
    \begin{enumerate}
    \item Determine all rational numbers $x$ satisfying $|x|_5\le 1$. 
    \item Which rational numbers $x$ satisfy $|x|_5=1$?
    \item What is $\lim_{n \to \infty} 5^n$ in $(\mathbb{Q}, d_5)$ (the $5$-adic metric)? \\
    \emph{Hint:} Compute $d_5(5^n,0)$.
    \end{enumerate}

    \medskip

    \item (15 pts) \textbf{Non-Archimedean absolute value and metric.}  
    ~
    Prove that $|\cdot|_p$ satisfies
    \[
    |xy|_p=|x|_p|y|_p,\qquad |x+y|_p\le \max\{|x|_p,|y|_p\},
    \]
    and show that $d_p$ is a metric on $\mathbb{Q}$.
  \end{enumerate}
\end{problem}

\smallskip

\begin{solution}
  ~
  \begin{enumerate}
    \item
    \begin{enumerate}
      \item $|75|_5 = 5^{-2} = \tfrac{1}{25}.$
      \item $\left|\tfrac{10}{9}\right|_5 = 5^{-1} = \tfrac{1}{5}.$
      \item $\left|-\tfrac{20}{375}\right|_5 = 5^{1} = 25.$
    \end{enumerate}
    \item
    \begin{enumerate}
      \item Suppose $ \vert x \vert_5 \leq 1 = 5^0 $. Since 
      \[ 
        \vert x \vert_5 = 5^{- \#(\text{factors of 5 in reduced form})}, 
      \] 
      there must be no factors of $ 5 $ in the denominator of $ x $ when written as a reduced fraction. Thus, $ x = 5^l p / q $, where $ l \geq 0 $ and $ 5 $ does not divide either $ p $ or $ q $. 
      \item This is the $ l=0 $ case from above. So $ x = p / q $, where $ 5 $ does not divide either $ p $ or $ q $. 
      \item Notice that $ 5^n > n $ for all $ n \geq 1 $ by mathematical induction, since $ 5^1 \geq 1 $ and $ 5^{n+1} = 5 \times 5^n \geq 5n \geq n + 1 $. So for all $ \epsilon > 0 $, choose $ N = 1 / \epsilon $, then 
      \[
        d_5(5^n, 0) = \vert 5^n \vert_5 = \frac{1}{5^n} < \frac{1}{n} < \epsilon
      \]
      whenever $ n > N $. Thus, $ \lim_{n \to \infty} 5^n = 0 $ in $ (\mathbb{Q}, d_5) $.
    \end{enumerate}
    \item Suppose $ x $ and $ y $ can be expressed as $ x = p^k \cdot \frac{m}{n} $ and $ y = p^l \cdot \frac{u}{v} $, where $ k, l \in \mathbb{Z} $, and $ m, n, u, v \in \mathbb{Z}$ are not divisible by $ p $. Then 
    \[
      \vert x \vert_p = p^{-k}, \quad \vert y \vert_p = p^{-l},
    \]
    \[
      xy = p^{k+l} \cdot \frac{mu}{nv},
    \]
    and 
    \[
      \vert xy \vert_p = p^{-(k+l)} = p^{-k} \cdot p^{-l} = \vert x \vert_p \cdot \vert y \vert_p.
    \]
    
    Without loss of generality, assume $ k \leq l $. Then
    \[
      x + y = p^k \cdot \frac{m}{n} + p^l \cdot \frac{u}{v} = p^k \left( \frac{m}{n} + p^{l-k} \cdot \frac{u}{v} \right).
    \]
    Since $ \frac{m}{n} + p^{l-k} \cdot \frac{u}{v} $ is not divisible by $ p $, we have 
    \[
      \vert x + y \vert_p = p^{-k} = \max\{p^{-k}, p^{-l}\} = \max\{\vert x \vert_p, \vert y \vert_p\}.
    \]

    Finally, we verify that $ d_p $ is a metric on $ \mathbb{Q} $ by checking the four properties of a metric:
    \begin{enumerate}[(i)]
      \item For $ x \in \mathbb{Q} $, we have $ d_p(x, x) = \vert x - x \vert_p = \vert 0 \vert \equiv 0. $
      \item For $ x, y \in \mathbb{Q} $ and $ x \neq y $, we have $ d_p(x, y) = \vert x - y \vert_p = a / b$ for some $ a, b \in \mathbb{Z}, a \neq 0 $.
      \item For $ x, y \in \mathbb{Q} $, we have $ d_p(x, y) = d_p(y, x) $.
      \item For $ x, y, z \in \mathbb{Q} $, we have
      \begin{equation}
        \begin{split}
          d_p(x, z) &= \vert x - z \vert_p = \vert (x - y) + (y - z) \vert_p \\
          &\leq \max\{\vert x - y \vert_p, \vert y - z \vert_p\} \\
          &\leq \vert x - y \vert_p + \vert y - z \vert_p = d_p(x,y) + d_p(y,z),
        \end{split}
      \end{equation}
      since $ \max\{a, b\} \leq a + b $ for all $ a, b \geq 0 $.
    \end{enumerate}
  Thus, $ d_p $ is a metric on $ \mathbb{Q} $. Furthermore, it is a non-Archimedean metric since it satisfies the \textbf{strong triangle inequality} $ d_p(x, z) \leq \max\{d_p(x, y), d_p(y, z)\} $.
  \end{enumerate}
\end{solution}

% 3
\begin{problem}[Exercise 1.1.3 (20 pts)]
    Let $X$ be a set, and let $d : X \times X \to [0,\infty)$ be a function. 

  \begin{enumerate}[(a)]
  \item Give an example of a pair $(X,d)$ which obeys axioms (bcd) of Definition~1.1.2, but not (a). 
  \hfill (Hint: modify the discrete metric.)
  \item Give an example of a pair $(X,d)$ which obeys axioms (acd) of Definition~1.1.2, but not (b).
  \item Give an example of a pair $(X,d)$ which obeys axioms (abd) of Definition~1.1.2, but not (c).
  \item Give an example of a pair $(X,d)$ which obeys axioms (abc) of Definition~1.1.2, but not (d). 
  \hfill (Hint: try examples where $X$ is a finite set.)
  \end{enumerate}
\end{problem}

\smallskip

\begin{solution}
  ~
  Recall the definition of a metric. We shall give examples for each case below.
  \begin{enumerate}[(a)]
    \item Let $ X = \mathbb{R} $ and define $ d $ such that $ d(x,y) = 0.5 $ if $ x \neq y $ and $ d(x,x) = 1 $. By construction (a) is not satisfied. Furthermore, $ d(x,y) = d(y,x) = 0.5 $ for all distinct $ x, y \in X $, so (b) and (c) are satisfied. Finally, for distinct $ x, y, z \in X $, we have $ d(x,z) = 0.5 \leq 0.5 + 0.5 = d(x,y) + d(y,z) $; for $ x = z $, we have $ d(x,z) = 1 \leq 0.5 + 0.5 = d(x,y) + d(y,z) $; for $ x=y \neq z $, we have $ d(x,z) = 0.5 \leq 1 + 0.5 = d(x,y) + d(y,z)$, so (d) is satisfied.
    \item Let $ X = \mathbb{R} $ and $ d(x,y) = 0 $ for all $ x, y \in X $. By construction (b) is not satisfied. Furthermore, $ d(x,x) = 0 $ for all $ x \in X $, $ d(x,y) = d(y,x) = 0 $ for all $ x, y \in X $, and $ d(x,z) = 0 \leq 0 + 0 = d(x,y) + d(y,z)$, for any $ x,y,z \in X $, so (a), (c) and (d) are satisfied.
    \item Let $ X = S^1 $ the unit circle, and $ d $ the shortest clockwise distance between two points on the circle. Then $ d(x,x) = 0 $ for all $ x \in X $ and $ d(x,y) > 0 $ for all distinct $ x, y \in X $. Furthermore, for any $ x,y,z \in X $, if $ y $ lies between $ x $ and $ z $, then $ d(x,z) = d(x,y) + d(y,z)$, and $ d(x,z) < d(x,y) + d(y,z) $ otherwise. Thus (a), (b) and (d) are satisfied. However, unless $ x, y $ lie on the antipodal points of the circle, $ d(x,y) \neq d(y,x) $, so (c) is not satisfied.
    \item Let $ X = \mathbb{R} $ and $ d(x,y) = (x-y)^2 $. Then $ d(x,x) = 0 $ for all $ x \in X $, $ d(x,y) > 0 $ for all distinct $ x, y \in X $, and $ d(x,y) = d(y,x) $ for all $ x, y \in X $, so (a), (b) and (c) are satisfied. However, for $ x=0, y=1, z=2 $, we have $ d(x,z) = 4 \nleq 1 + 1 = d(x,y) + d(y,z)$, so (d) is not satisfied.
  \end{enumerate}
\end{solution}

% Problem 4
\begin{problem}[20 pts]
  Let $x=(x_1,\dots,x_n)$ and $y=(y_1,\dots,y_n)$ be vectors in $\mathbb{R}^n$.
  \begin{enumerate}[(a)]
  \item The $\ell^1$ metric is defined by
  \[
  d_1(x,y) := \sum_{i=1}^n |x_i - y_i|.
  \]
  Show that $d_1$ is a metric on $\mathbb{R}^n$

  \item The $\ell^\infty$ metric is defined by
  \[
  d_\infty(x,y) := \max_{1 \leq i \leq n} |x_i - y_i|.
  \]
  Show that $d_{\infty}$ is a metric on $\mathbb{R}^n$
  \end{enumerate}
\end{problem}

\medskip

\begin{solution}
  ~
  \begin{enumerate}[(a)]
    \item
    We verify the four properties of a metric:
    \begin{enumerate}[(i)]
      \item $ d(x,x) = 0 $
      \item Each absolute value in the sum is non-negative. Moreover, if $ x \neq y $, there must exist some $ i $ such that $ x_i \neq y_i $, hence $ d_1(x,y) > 0 $.
      \item $ d_1(x,y) = \sum_{i=1}^n |x_i - y_i| = \sum_{i=1}^n |y_i - x_i| = d_1(y,x) $
      \item By the triangle inequality of real numbers, we have
      \begin{equation}
        \begin{split}
          d_1(x,z) &= \sum_{i=1}^n |x_i - z_i| \\
          &\leq \sum_{i=1}^n (|x_i - y_i| + |y_i - z_i|) \\
          &= \sum_{i=1}^n |x_i - y_i| + \sum_{i=1}^n |y_i - z_i| \\
          &= d_1(x,y) + d_1(y,z).
        \end{split}
      \end{equation}
    \end{enumerate}
    Hence $ d_1 $ is a metric on $ \mathbb{R}^n $.
    \item 
    We verify the four properties of a metric:
    \begin{enumerate}[(i)]
      \item $ d_\infty(x,x) = 0 $
      \item Each absolute value in the maximum is non-negative. Moreover, if $ x \neq y $, there must exist some $ i $ such that $ x_i \neq y_i $, hence $ d_\infty(x,y) > 0 $.
      \item $ d_\infty(x,y) = \max_{1 \leq i \leq n} |x_i - y_i| = \max_{1 \leq i \leq n} |y_i - x_i| = d_\infty(y,x) $
      \item By the triangle inequality of real numbers, we have
      \begin{equation}
        \begin{split}
          d_\infty(x,z) &= \max_{1 \leq i \leq n} |x_i - z_i| \\
          &\leq \max_{1 \leq i \leq n} (|x_i - y_i| + |y_i - z_i|) \\
          &\leq \max_{1 \leq i \leq n} |x_i - y_i| + \max_{1 \leq i \leq n} |y_i - z_i| \\
          &= d_\infty(x,y) + d_\infty(y,z).
        \end{split}
      \end{equation}
    \end{enumerate}
    Hence $ d_\infty $ is a metric on $ \mathbb{R}^n $.
  \end{enumerate}
\end{solution}

% Problem 5
\begin{problem}[10 pts]
  A \emph{vector space} $V$ over $\mathbb{R}$ s a set 
  equipped with two operations:
  \begin{enumerate}
    \item \textbf{Vector addition:} $+: V \times V \to V$, written $(u,v) \mapsto u+v$.
    \item \textbf{Scalar multiplication:} $\cdot : \mathbb{R} \times V \to V$, written $(\alpha,v) \mapsto \alpha v$,
  \end{enumerate}
  such that the following properties hold for all $u,v,w \in V$ and $\alpha,\beta \in \mathbb{R}$:
  \begin{enumerate}
    \item[(VS1)] $(u+v)+w = u+(v+w)$ \hfill (associativity of addition)
    \item[(VS2)] $u+v = v+u$ \hfill (commutativity of addition)
    \item[(VS3)] There exists $0 \in V$ such that $u+0=u$ \hfill (additive identity)
    \item[(VS4)] For each $u \in V$, there exists $-u \in V$ such that $u+(-u)=0$ \hfill (additive inverse)
    \item[(VS5)] $\alpha(u+v) = \alpha u + \alpha v$ \hfill (distributivity I)
    \item[(VS6)] $(\alpha+\beta)u = \alpha u + \beta u$ \hfill (distributivity II)
    \item[(VS7)] $(\alpha\beta)u = \alpha(\ beta u)$ \hfill (compatibility of scalar multiplication)
    \item[(VS8)] $1 \cdot u = u$ \hfill (identity element of scalar multiplication)
  \end{enumerate}

  A function $\|\cdot\| : V \to [0,\infty)$ is called a \emph{norm} on $V$ if, 
  for all $u,v \in V$ and $\alpha \in \mathbb{R}$, the following properties hold:
  \begin{enumerate}
    \item[(N1)] $\|v\| \geq 0$, and $\|v\| = 0$ if and only if $v=0$. \hfill (positivity)
    \item[(N2)] $\|\alpha v\| = |\alpha| \cdot \|v\|$. \hfill (homogeneity)
    \item[(N3)] $\|u+v\| \leq \|u\| + \|v\|$. \hfill (triangle inequality)
  \end{enumerate}

  Given a norm $\|\cdot\|$ on $V$, define $d : V \times V \to [0,\infty)$ by
  \[
  d(u,v) = \|u-v\|.
  \]

  Prove that $d$ is a \emph{metric} on $V$, that is, for all $x,y,z \in V$ show that:
    \begin{enumerate}[(i)]
      \item $d(x,y) \geq 0$ and $d(x,y)=0$ if and only if $x=y$.
      \item $d(x,y) = d(y,x)$.
      \item $d(x,z) \leq d(x,y) + d(y,z)$.
    \end{enumerate}
  (Thus we conclude that every normed vector space $(V,\|\cdot\|)$ is also a metric space with metric $d(u,v)=\|u-v\|$.
  )
\end{problem}

\smallskip

\begin{solution}
  We will show that the three properties of a metric are satisfied.
  \begin{enumerate}[(i)]
    \item The conditions that $ d(x,y) = \lVert x - y \rVert \geq 0 $ and $ d(x,y) = \lVert x - y \rVert = 0 $ if and only if $ x = y $ are equivalent to (N1). 
    \item $ d(x,y) = \lVert x - y \rVert = \lVert -(y - x) \rVert = \lVert (-1)(y - x) \rVert = \vert -1 \vert \cdot \lVert y - x \rVert = d(y,x) $ by (N2).
    \item $ d(x,z) = \lVert x - z \rVert = \lVert (x - y) + (y - z) \rVert \leq \lVert x - y \rVert + \lVert y - z \rVert = d(x,y) + d(y,z) $ by (N3).
  \end{enumerate}
  Thus every normed vector space $(V,\|\cdot\|)$ is also a metric space with metric $d(u,v)= \lVert u-v \rVert$.
\end{solution}

% Problem 6
\begin{problem}
    Let $S$ be a bounded nonempty set of real numbers, and let $a$ and $b$
  be fixed nonzero real numbers. Define $T=\{as+b| s\in S\}$ Find formulas
  for
  $\sup T$ and $\inf T$ in terms of $\sup S$ and $\inf S$. Prove your
  formulas.
\end{problem}

\smallskip

\begin{solution}
  \begin{claim}
    The supremum and infimum of $ T $ are given by 
    \begin{equation}
      \sup T = a \sup S + b, \quad \inf T = a \inf S + b.
    \end{equation}
  \end{claim}
  \begin{proof}
    Since $ S $ is a bounded nonempty set of real numbers, both $ \sup S $ and $ \inf S $ exist. We consider two cases based on the sign of $ a $.
    \begin{enumerate}[(a)]
      \item If $ a > 0 $, then for all $ s \in S $, we have $ as + b \leq a \cdot \sup S + b $, so $ a \cdot \sup S + b $ is an upper bound of $ T $. By the definition of supremum, for any $ \epsilon > 0 $ there exists some $ s' \in S $ such that $ \sup S - \epsilon \leq s^{\prime} \leq \sup S $. Multiplying by $ a > 0 $ and adding $ b $ gives
      \[
        a \sup S + b - a \epsilon \leq  as' + b \leq a \sup S + b.
      \]
      Hence $ \sup T = a \sup S + b $.
      \item If $ a < 0 $, similarly for all $ s \in S $, we have $ as + b \leq a \inf S + b $, so $ a \inf S + b $ is an upper bound of $ T $.
      By definition of the supremum, for any $ \epsilon > 0 $ there exists some $ s' \in S $ such that $ \inf S \leq s' < \inf S + \epsilon $. Multiplying by $ a < 0 $ and adding $ b $ gives
      \[
        a \inf S + b \leq as' + b < a \inf S + b - a \epsilon.
      \]
      Hence $ \sup T = a \cdot \inf S + b $. 
    \end{enumerate}
    Since $ a, b $ are nonzero, we are done. 
  \end{proof}
\end{solution}

\end{CJK}
\end{document}    