\documentclass[a4paper]{article}
%% Formatting %%
\usepackage[margin=3cm]{geometry}
\usepackage{type1cm, titlesec, fancyhdr, titling}
\usepackage{multicol}
\usepackage[dvipsnames]{xcolor}
\usepackage{ulem}
\usepackage{parskip}
\setlength{\parindent}{2em}
\setlength{\headheight}{15pt}
\setlength{\droptitle}{-1.5cm}
\parindent=24pt
%% Math and Symbols %%
\usepackage{amsmath,amsthm,amssymb, mathtools}
\usepackage{yhmath, faktor, dsfont}
\usepackage{academicons, wasysym, marvosym}
\usepackage[scr]{rsfso} 
\usepackage{latexsym, amsmath, amscd, amsmath, amsthm}
\usepackage{amssymb,amsmath,amsthm,graphicx,dsfont}
\usepackage{hyperref}

%% Enhancement %%
\usepackage{graphicx, tabularx}
\usepackage[shortlabels,inline]{enumitem}
%% TikZ %%
\usepackage{tikz-cd}
\usepackage[breakable]{tcolorbox}
\usetikzlibrary{decorations.pathmorphing}
\usetikzlibrary{calc, arrows,matrix}

%% Other packages %%
\usepackage{amsopn}

%% Traditional Chinese %%
\usepackage{CJKutf8}

%% Math environments %%
\newtheoremstyle{mystyle}
  {6pt}{15pt}% 上下間距
  {}%          內文字體
  {}%              縮排
  {\bf}%       標頭字體
  {.}%       標頭後標點
  {1em}% 內文與標頭距離
  {}% Theorem head spec (can be left empty, meaning 'normal')
\theoremstyle{mystyle}	
\newtheorem{theorem}{Theorem}
\newtheorem{definition}{Definition}
\newtheorem{example}[theorem]{Example}
\newtheorem{exercise}{Exercise}
\newtheorem{solution}{Solution}
\newtheorem{corollary}[theorem]{Corollary}
\newtheorem{property}[theorem]{Property}
\newtheorem{proposition}[theorem]{Proposition}
\newtheorem{lemma}[theorem]{Lemma}
\newtheorem{problem}{Problem}
\newtheorem{answer}{Answer}[section]
\newtheorem{fact}[theorem]{Fact}
\newtheorem*{recall}{Recall}
\newtheorem*{remark}{Remark}
\newtheorem*{claim}{Claim}
\newtheorem*{observation}{Observation}

\begin{document}
\begin{CJK}{UTF8}{bkai}

    \title{%
  \textbf{Math 2213 Introduction to Analysis I} \\
  \vspace{0.5cm}
  \large Homework 7 Due November 7 (Friday), 2025
}
\author{物理、數學三 黃紹凱 B12202004}
\date{\today}

\maketitle

% Exercise 1
\begin{exercise}[15 pts]
    Assume that $(S,d)$ is a metric space, and let $f_n, f : S \to \mathbb{R}$ be real-valued functions. Suppose that $f_n \to f$ uniformly on $S$, and there exists a constant $M > 0$ such that
    \[
    |f_n(x)| \le M \quad \text{for all } x \in S \text{ and all } n.
    \]
    Let $g : \overline{B(0;M)} \to \mathbb{R}$ be continuous, where
    \[
    B(0;M) = \{ y \in \mathbb{R} : |y| < M \}.
    \]
    Define
    \[
    h_n(x) = g(f_n(x)), \qquad h(x) = g(f(x)), \quad x \in S.
    \]
    Prove that $h_n \to h$ uniformly on $S$.
\end{exercise}

\begin{solution}
    Since $ g $ is continuous on the closed interval $ \overline{B(0;M)} $, by previous homework it is uniformly continuous on this interval. Therefore, for the given $ \varepsilon > 0 $, there exists a $ \delta > 0 $ such that whenever $ |y_1 - y_2| < \delta $ for any $ y_1, y_2 \in \overline{B(0;M)} $, we have $ |g(y_1) - g(y_2)| < \varepsilon $. Since $ f_n \rightrightarrows f $ uniformly on $ S $, for any $ \delta > 0 $, there exists $ N \in \mathbb{N} $ such that for all $ n \geq N $ and all $ x \in S $, we have $ |f_n(x) - f(x)| < \delta $. Hence, for all $ n \geq N $ and all $ x \in S $, we have
    \[
        |h_n(x) - h(x)| = |g(f_n(x)) - g(f(x))| < \varepsilon,
    \]
    since $ |f_n(x) - f(x)| < \delta $. Therefore, $ h_n \rightrightarrows h $ on $ S $.
\end{solution}
 
% Exercise 2
\begin{exercise}[15 pts]
    Let $f_n(x) = x^n$. The sequence $\{f_n\}$ converges pointwise but not uniformly on $[0,1]$. Let $g$ be continuous on $[0,1]$ with $g(1) = 0$. Prove that the sequence $\{ g(x) x^n \}$ converges uniformly on $[0,1]$.
\end{exercise}

\begin{solution}
    The sequence $ \{f_n\} $ converges to the function 
    \[
        f(x) =
        \begin{dcases}
            0, \quad 0\leq x < 1, \\
            1, \quad x = 1.
        \end{dcases}
    \]
    We claim that $ g(x) x^n \rightrightarrows 0 $, and by continuity of $ g $ at $ 1 $, for $ \varepsilon > 0 $ there exists $ \delta > 0 $ such that $ \vert g(x) - g(1) \vert < \varepsilon $ whenever $ \vert x-1 \vert < \delta $. We have
    \[
        \vert g(x) x^n - 0 \vert = \vert g(x)x^n \vert \leq \vert g(x) \vert = \vert g(x) - g(1) \vert < \varepsilon
    \]
    whenever $ \vert x-1 \vert \leq \delta $. Next, consider the case when $ x \in [0, 1-\delta] $. Since $ g $ is continuous on $ [0, 1] $, it is bounded by $ M = \max \{g\} > 0 $ on $ [0, 1] $. Thus, for $ x \in [0, 1-\delta] $, we have
    \[
        \vert g(x)x^n - 0 \vert = \vert g(x)x^n \vert \leq M (1-\delta)^n.
    \]
    Since $ 0 < 1-\delta < 1 $, we can choose $ N \in \mathbb{N} $ such that $ M(1-\delta)^N < \varepsilon $. Therefore, for all $ n \geq N $ and all $ x \in [0, 1-\delta] $, we have $ \vert g(x)x^n - 0 \vert < \varepsilon $. Combining both cases proves uniform convergence of the sequence $\{ g(x)x^n \}$. 
\end{solution}
 
% Exercise 3
\begin{exercise}[15 pts] 
    Assume that $g_{n+1}(x) \le g_n(x)$ for each $x$ in $T$ and each $n = 1, 2, \ldots$, and suppose that $g_n \to 0$ uniformly on $T$. Prove that 
    \[
    \sum (-1)^{n+1} g_n(x)
    \]
    converges uniformly on $T$.
\end{exercise}

\begin{solution}
    Since $ g_n \rightrightarrows 0 $ and $ g_{n+1} (x) \leq g_n (x) $, for all $ n $, we have $ g_n (x) \geq 0 $ for all $ x \in T $. Fix $ x \in T $, the $ n $-th partial sum $ S_n = \sum_{k=1}^n g_k(x) $ satisfies the following inequalities:
    \[
        S_{2m+1} (x) \leq S_{2m+3} (x) \leq S_{2m+2} (x) , \quad S_{2m} (x) \leq S_{2m+2} (x) \leq S_{2m+1} (x). 
    \] 
    Hence, every later partial sum $ S_{m \geq n+1} $ lies in the interval $ [S_{n+1} (x), S_n (x)] $ or $ [S_n (x), S_{n+1} (x)] $. Therefore, for all $ m > n $, we have 
    \[
        \left\vert S_m (x) - S_n (x) \right\vert \leq \left\vert S_{n+1} (x) - S_n (x) \right\vert = g_{n+1} (x).
    \]
    Since $ g_n \rightrightarrows 0 $ on $ T $, for any $ \varepsilon > 0 $, there exists $ N \in \mathbb{N} $ such that for all $ n \geq N $ and all $ x \in T $, we have $ g_{n+1} (x) < \varepsilon $. Therefore, for all $ m > n \geq N $ and all $ x \in T $, we have $ \left\vert S_m (x) - S_n (x) \right\vert < \varepsilon $. Hence $ (S_n) $ is Cauchy on $ T $, and the pointwise limit $ S(x) = \lim_{n \to \infty} S_n(x) $ exists for each $ x \in T $. Then 
    \[
        \left\vert S(x) - S_n(x) \right\vert = \lim_{m \to \infty} \left\vert S_m(x) - S_n(x) \right\vert \leq g_{n+1} (x) < \varepsilon, \quad \text{as } m \to \infty,
    \] 
    for all $ n \geq N $ and all $ x \in T $. Therefore, $ S_n \rightrightarrows S $ on $ T $.
\end{solution}

% Exercise 4
\begin{exercise}[15 pts]
    Let
    \[
    f_n(x) = \frac{x}{1 + n x^2}, \quad x \in \mathbb{R}, \; n = 1, 2, \ldots
    \]
    Find the limit function $f$ of the sequence $\{ f_n \}$ and the limit function $g$ of the sequence $\{ f_n' \}$.
    \begin{enumerate}[(a)]
        \item Prove that $f'(x)$ exists for every $x$ but that $f'(0) \ne g(0)$. For what values of $x$ is $f'(x) = g(x)$?
        \item In what subintervals of $\mathbb{R}$ does $f_n \to f$ uniformly?
        \item In what subintervals of $\mathbb{R}$ does $f_n' \to g$ uniformly?
    \end{enumerate}
\end{exercise}

\begin{solution}
    Since $ f_n (0) = 0 $ for all $ n $, suppose $ x \ne 0 $, so 
    \[
        0 \leq \vert f_n (x) \vert = \left\vert \frac{x}{1+ nx^2} \right\vert \leq \left\vert \frac{1}{nx} \right\vert \to 0 \quad \text{as } n \to \infty.
    \]
    By the Squeeze Theorem, the sequence $ \{f_n\} $ converges to $ f = 0 $. On the other hand, we have
    \[
        f^{\prime}_n (x) = \frac{1 - n x^2}{(1 + n x^2)^2}.
    \]
    Since $ f^{\prime}_n (0) = 1 $, suppose $ x \ne 0 $, then
    \[
        \vert f^{\prime}_n (x) \vert = \left\vert \frac{1 - n x^2}{(1 + n x^2)^2} \right\vert \leq \left\vert \frac{nx^2}{n^2 x^4} \right\vert = \left\vert \frac{1}{n x^2} \right\vert \to 0 \quad \text{as } n \to \infty. 
    \]
    By the Squeeze Theorem, the sequence $ \{ f_n^{\prime} \} $ converges to $ g(x) = 0 $ for $ x \ne 0 $ and $ g(0) = 1 $.
    \begin{enumerate}[(a)]
        \item By the above calculation, since $ 1 + nx^2 > 0 $ for all $ x\in \mathbb{R} $, $ f^{\prime} (x) $ exists for every $ x $. However, $ f^{\prime} (0) = 0 \ne g(0) = 1 $. For $ x \ne 0 $, we have $ f^{\prime} (x) = g(x) = 0 $.
        \item We have
        \[
            \vert f_n (x) - f(x) \vert = \left\vert \frac{x}{1 + n x^2} \right\vert \leq \sup_{x \in \mathbb{R}} \left\vert \frac{x}{1 + n x^2} \right\vert \leq \left.\frac{\vert x \vert}{1 + n x^2}\right|_{x = n^{-1 / 2}} = \frac{1}{2\sqrt{n}},
        \]
        Given $ \varepsilon > 0 $, choose $ N = \frac{1}{4 \varepsilon^2} $, then for all $ n > N $ we have $ \frac{1}{2 \sqrt{n}} < \varepsilon $. Therefore, $ f_n \rightrightarrows f $ on $ \mathbb{R} $.
        \item For any interval $ [a, b] $ not containing the origin, where without loss of generality we set $ 0<a<b $. For all $ \varepsilon > 0 $, let $ N = \frac{1}{\varepsilon a^2} $, then for all $ n \geq N $ and all $ x \in [a, b] $, we have
        \[
            \vert f_n^{\prime} (x) - g(x) \vert = \left\vert \frac{1 - n x^2}{(1 + n x^2)^2} \right\vert \leq \frac{1}{n x^2} \leq \frac{1}{na^2} < \varepsilon.
        \]
        Therefore, $ f_n^{\prime} \rightrightarrows g $ on any interval not containing $ 0 $. Next, consider an open interval with $ 0 $ as an end point. Without loss of generality, let it be $ (0,b) $, $ b>0 $. Since $ \lim_{x \to 0^+} f_n^{\prime} (x) = 1 $ for all $ n $, for every $ \varepsilon>0 $ there exists $ \delta > 0 $ such that for all $ x \in (0, \delta) $, we have 
        \[
            \left\vert \frac{1-nx}{(1+nx^2)^2} \right\vert > 1 - \varepsilon \;\Longrightarrow\; \sup_{x \in (0,b)} \vert f^{\prime}_n (x) - g(x) \vert = 1
        \]
        for all $ n $. Hence, convergence is not uniform on $ (0,b) $. Therefore, $ f^{\prime}_n \rightrightarrows g $ exactly on the intervals $ I \subseteq \mathbb{R} $ where $ \inf_{x \in I} \vert x \vert > 0 $.
    \end{enumerate}
\end{solution}

% Exercise 5
\begin{exercise}[15 pts]
    Prove that 
    \[
    \sum x^n (1 - x)
    \]
    converges pointwise but not uniformly on $[0,1]$, whereas 
    \[
    \sum (-1)^n x^n (1 - x)
    \]
    converges uniformly on $[0,1]$. This illustrates that uniform convergence of $\sum f_n(x)$ along with pointwise convergence of $\sum |f_n(x)|$ does not necessarily imply uniform convergence of $\sum |f_n(x)|$.
\end{exercise}

\begin{solution}
    ~

    \begin{enumerate}
        \item $ \sum x^n (1 - x) $: If $ x=0 $ or $ x=1 $, then $ \sum x^n (1 - x) = 0 $ for all $ n $. Suppose $ x \in (0,1) $, let
        \[
            f_n = \sum_{k=1}^n x^k (1-x) = \frac{x(1-x)(1 - x^n)}{1 - x} = x(1 - x^n) \to x \quad \text{as } n \to \infty.
        \]
        be the $ n $-th partial sum. Then, for some $ x \in \mathbb{R} $ and $ \varepsilon>0 $, let $ N_x = \frac{\log \varepsilon}{\log x} - 1 $, we have 
        \[
            \left\vert f_n (x) - x \right\vert = \vert x^{n+1} \vert < \varepsilon , \quad \text{whenever } n > N_x.
        \]
        Since $ N_x $ is unbounded for $ x \in \mathbb{R} $, the convergence is not uniform on $ [0,1] $.
        \item $ \sum (-1)^n x^n (1 - x) $: If $ x=0 $ or $ x=1 $, then $ \sum x^n (1 - x) = 0 $ for all $ n $. Suppose $ x \in (0,1) $, let 
        \[
            g_n (x) = \sum_{k=1}^n (-1)^k x^k (1-x) = x(1-x) \frac{1 - (-x)^{n+1}}{1 + x} \to \frac{x(1-x)}{1+x} \quad \text{as } n \to \infty.
        \]
        Then, for all $ \varepsilon > 0 $, let $ N = 1 / (e \varepsilon) - 2 $, then whenever $ n>N $, we have
        \[
            \left\vert g_n (x) - \frac{x(1-x)}{1+x} \right\vert = \left\vert \frac{x(1-x)}{1+x} x^{n+1} \right\vert < (1-x)x^{n+1} < \frac{1}{(n+2)\left(1 + \frac{1}{n+1}\right)^{n+1}} < \varepsilon. 
        \]
        Here we used 
        \[
            \frac{\mathrm{d}}{\mathrm{d}x} \left( (1-x)x^{n+1} \right) = x^n \left( n+1 - (n+2)x \right) = 0 \;\Longrightarrow\; x = \frac{n+1}{n+2}.
        \]
    \end{enumerate}
\end{solution}
 
% Exercise 6
\begin{exercise}[15 pts] 
    Let 
    \[
    f_n(x) = \frac{1}{n} e^{-n^2 x^2}, \quad x \in \mathbb{R}, \; n = 1, 2, \ldots
    \]
    Prove that $f_n \to 0$ uniformly on $\mathbb{R}$, that $f_n' \to 0$ pointwise on $\mathbb{R}$, but that the convergence of $\{ f_n' \}$ is not uniform on any interval containing the origin.
\end{exercise}

\begin{solution}
    ~ 

    \begin{enumerate}
        \item $ f_n \rightrightarrows 0 $ on $ \mathbb{R} $: For any $ x \in \mathbb{R} $ and $ \varepsilon > 0 $, let $ N = \varepsilon^{-1} $, then for all $ n > N $ we have 
        \[
            \vert f_n (x) - 0 \vert = \left\vert \frac{1}{n} e^{-n^2 x^2} \right\vert \leq \frac{1}{n} < \varepsilon .
        \] 
        \item $ f^{\prime} _n \to 0 $ on $ \mathbb{R} $: For any $ x \in \mathbb{R} $, we have $ f^{\prime} _n (x) = -2 x e^{-n^2 x^2} $. When $ x=0 $, $ f_n = 0 $. So consider $ x \in \mathbb{R} \setminus \{0\} $, let $ N_x = \frac{1}{x} \sqrt{\log (2\vert x \vert / \varepsilon)} $ if $ \vert x \vert > \frac{\varepsilon}{2} $ and $ N_x = \frac{1}{\vert x \vert} $ otherwise. Then for all $ n > N_x $ we have
        \[
            \left\vert f^{\prime}_n (x) - 0 \right\vert = \left\vert 2x e^{- n^2 x^2} \right\vert < \varepsilon .
        \] 
        Hence $ f^{\prime}_n \to 0 $ pointwise. However, if zero is contained in the interval, $ \lim_{x \to 0} N_x = \lim_{x \to 0} \frac{1}{x^2} $ does not exist, so convergence is not uniform.
    \end{enumerate}
\end{solution}
 
% Exercise 7 
\begin{exercise}[10 pts] 
    Let $\{ f_n \}$ be a sequence of real-valued continuous functions defined on $[0,1]$ and assume that $f_n \to f$ uniformly on $[0,1]$.  Prove or disprove
    \[
    \lim_{n \to \infty} \int_{0}^{1 - 1/n} f_n(x) \, dx = \int_{0}^{1} f(x) \, dx.
    \]
\end{exercise}

\begin{solution}
    For each $ n $, notice that 
    \[
        \int_0^{1- 1/n} f_n = \int_0^{1 - 1/n} f + \int_0^{1 - 1/n} (f_n - f).
    \] 
    Hence, 
    \[
        \left\vert \int_0^{1 - 1/n} f_n - \int_0^1 f \right\vert \leq \left\vert \int_0^{1 - 1/n} (f_n - f) \right\vert + \left\vert \int_{1 - 1/n}^1 f \right\vert.
    \]
    Since $ f_n \rightrightarrows f $ on $ [0,1] $, for any $ \varepsilon > 0 $, there exists $ N_1 \in \mathbb{N} $ such that for all $ n \geq N_1 $ and all $ x \in [0,1] $, we have $ \vert f_n (x) - f(x) \vert < \varepsilon / 2 $. Therefore, for all $ n \geq N_1 $, we have
    \[
        \left\vert \int_0^{1 - 1/n} (f_n - f) \right\vert \leq \int_0^{1 - 1/n} \vert f_n - f \vert < \frac{\varepsilon}{2}.
    \]
    On the other hand, since $ f $ is continuous on $ [0,1] $, it is integrable on $ [0,1] $. Thus, there exists $ N_2 \in \mathbb{N} $ such that for all $ n \geq N_2 $, we have
    \[
        \left\vert \int_{1 - 1/n}^1 f \right\vert < \frac{\varepsilon}{2}.
    \]
    Therefore, for all $ n \geq \max \{N_1, N_2\} $, we have
    \[
        \left\vert \int_0^{1 - 1/n} f_n - \int_0^1 f \right\vert < \varepsilon \;\Longrightarrow\; \lim_{n \to \infty} \int_0^{1 - 1/n} f_n = \int_0^1 f.
    \]
\end{solution}

\newpage

\begin{center}
    You can do the following problems to practice. \\
    You don't have to submit the following problems.
\end{center}

% Optional 1
\begin{exercise}[Optional]
    Prove that the series 
    \[
    \zeta(s) = \sum_{n=1}^{\infty} n^{-s}
    \]
    converges uniformly on every half-infinite interval 
    \[
    1 + h \le s < +\infty,
    \]
    where $h > 0$. Show that the equation
    \[
    \zeta'(s) = - \sum_{n=1}^{\infty} \frac{\log n}{n^s}
    \]
    is valid for each $s > 1$, and obtain a similar formula for the $k$th derivative $\zeta^{(k)}(s)$.
\end{exercise}

\begin{solution}

\end{solution}

% Optional 2
\begin{exercise}[Optional]
    If $r$ is the radius of convergence of 
    \[
    \sum a_n (x - x_0)^n,
    \]
    where each $a_n \ne 0$, show that
    \[
    \liminf_{n \to \infty} \left| \frac{a_n}{a_{n+1}} \right|
    \le r \le 
    \limsup_{n \to \infty} \left| \frac{a_n}{a_{n+1}} \right|.
    \]
\end{exercise}

\begin{solution}

\end{solution}

% Optional 3
\begin{exercise}[Optional]
    Prove that the series
    \[
    \sum_{n=0}^{\infty} \left( \frac{x^{2n+1}}{2n+1} - \frac{x^{n+1}}{2n+2} \right)
    \]
    converges pointwise but not uniformly on $[0,1]$.
\end{exercise}

\begin{solution}

\end{solution}

% Optional 4
\begin{exercise}[Optional]
    Prove that 
    \[
    \sum_{n=1}^{\infty} a_n \sin nx 
    \quad \text{and} \quad
    \sum_{n=1}^{\infty} a_n \cos nx
    \]
    are uniformly convergent on $\mathbb{R}$ if 
    \[
    \sum_{n=1}^{\infty} |a_n|
    \]
    converges.
\end{exercise}

\begin{solution}
\end{solution}

% Optional 5
\begin{exercise}[Optional]
    Let $\{a_n\}$ be a decreasing sequence of positive terms. 
    Prove that the series 
    \[
    \sum a_n \sin nx
    \]
    converges uniformly on $\mathbb{R}$ if and only if $n a_n \to 0$ as $n \to \infty$.
\end{exercise}

\begin{solution}

\end{solution}

\end{CJK}
\end{document}