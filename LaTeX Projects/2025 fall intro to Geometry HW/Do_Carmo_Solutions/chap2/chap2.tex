\documentclass[a4paper]{article}
%% Formatting %%
\usepackage[margin=3cm]{geometry}
\usepackage{type1cm, titlesec, fancyhdr, titling}
\usepackage{multicol}
\usepackage[dvipsnames]{xcolor}
\usepackage{ulem}
\usepackage{parskip}
\setlength{\parindent}{2em}
\setlength{\headheight}{15pt}
\setlength{\droptitle}{-1.5cm}
\parindent=24pt
%% Math and Symbols %%
\usepackage{amsmath,amsthm,amssymb, mathtools}
\usepackage{yhmath, faktor, dsfont}
\usepackage{academicons, wasysym, marvosym}
\usepackage[scr]{rsfso} 
\usepackage{latexsym, amsmath, amscd, amsmath, amsthm}
\usepackage{amssymb,amsmath,amsthm,graphicx,dsfont}
\usepackage{hyperref}

%% Enhancement %%
\usepackage{graphicx, tabularx}
\usepackage[shortlabels,inline]{enumitem}
%% TikZ %%
\usepackage{tikz-cd}
\usepackage[breakable]{tcolorbox}
\usetikzlibrary{decorations.pathmorphing}
\usetikzlibrary{calc, arrows,matrix}

%% Other packages %%
\usepackage{amsopn}

%% Traditional Chinese %%
\usepackage{CJKutf8}

%% Math environments %%
\newtheoremstyle{mystyle}
  {6pt}{15pt}% 上下間距
  {}%          內文字體
  {}%              縮排
  {\bf}%       標頭字體
  {.}%       標頭後標點
  {1em}% 內文與標頭距離
  {}% Theorem head spec (can be left empty, meaning 'normal')
\theoremstyle{mystyle}	
\newtheorem{theorem}{Theorem}
\newtheorem{definition}{Definition}
\newtheorem{example}[theorem]{Example}
\newtheorem{exercise}{Exercise}
\newtheorem{solution}{Solution}
\newtheorem{corollary}[theorem]{Corollary}
\newtheorem{property}[theorem]{Property}
\newtheorem{proposition}[theorem]{Proposition}
\newtheorem{lemma}{Lemma}
\newtheorem{problem}[theorem]{Problem}
\newtheorem{answer}{Answer}[section]
\newtheorem{fact}[theorem]{fact}
\newtheorem*{claim}{Claim}
\newtheorem*{observation}{Observation}

\newenvironment{exerciseManual}[1]{%
  \renewcommand{\theexercise}{#1}%
  \begin{exercise}%
  \addtocounter{exercise}{-1}%
}{%
  \end{exercise}%
}

\newenvironment{solutionManual}[1]{%
  \renewcommand{\thesolution}{#1}%
  \begin{solution}%
  \addtocounter{solution}{-1}%
}{%
  \end{solution}%
}

\theoremstyle{remark}
\newtheorem*{remark}{Remark}

\newcommand{\bvec}[1]{\mathbf{#1}} % vector

\begin{document}
\begin{CJK}{UTF8}{bkai}

\title{%
  \textbf{2025 Fall Introduction to Geometry} \\
  \vspace{0.5cm}
  \Large Solutions to Exercises in Do Carmo \\
}
\author{黃紹凱 B12202004}
\date{\today}

\maketitle


\section{Chapter 2}
\subsection{Chapter 2.1}
\subsection{Chapter 2.2}

\begin{definition}[regular surface] \label{def:regular_surface}
    A subset $ S \subseteq \mathbb{R}^3 $ is a \emph{regular surface} if, for each $ p \in S $, there exists a neighborhood $ V \subseteq \mathbb{R}^3 $ and a map $ \bvec{x}: U \to V \cap S $ of an open set $ U \subseteq \mathbb{R}^2 $ onto $ V \cap S \subseteq \mathbb{R}^3 $ such that 
    \begin{enumerate}[(i)]
        \item $ \bvec{x} $ is (infinitely) differentiable. 
        \item $ \bvec{x} $ is a homeomorphism, i.e. $ \bvec{x} $ is a bijection, and both $ \bvec{x} $ and $ \bvec{x}^{-1} $ are continuous. 
        \item For each $ q \in U $, the differential $ \mathrm{d}\bvec{x}_q : \mathbb{R}^2 \to \mathbb{R}^3 $ is one-to-one (the regularity condition).
    \end{enumerate}
    The mapping $ \bvec{x} $ is called a \emph{parametrization} of the surface $ S $ or a system of \emph{local coordinates} around the point $ p $. The neighborhood $ V \cap S $ of $ p $ is called a \emph{coordinate neighborhood}.
\end{definition}

\begin{figure}
    \centering
    \includegraphics[width=0.8\textwidth]{2-1.png}
    \label{fig:2-1}
\end{figure}

\begin{definition}[regular and critical value]
    Given a differential map $ F: U \subseteq \mathbb{R}^n \to \mathbb{R}^m $ defined on an open set $ U $, we say that $ p \in U $ is a \emph{critical value} of $ F $ is the differential $ \mathrm{d}F_p : \mathbb{R}^n \to \mathbb{R}^m $ is not surjective. Otherwise, $ p $ is called a \emph{regular value} of $ F $.
\end{definition}

\begin{proposition}
    \label{prop:2.2.1}
    If $ f: U \to R $ is a differentiable function in an open set $ U $ of $ \mathbb{R}^2 $, then the graph of $ f $, that is, the subset of $ \mathbb{R}^3 $ given by $ \left(x, y, f(x,y)\right) $ for $ (x,y) \in U $, is a regular surface. 
\end{proposition}

\begin{proposition}
    \label{prop:2.2.2}
    If $ f: U \subseteq \mathbb{R}^2 \to \mathbb{R}^3 $ is a differentiable function and $ a \in f(U) $ is a regular value of $ f $, then $ f^{-1} (a) $ is a regular surface in $ \mathbb{R}^3 $.
\end{proposition}

\begin{proposition}
    \label{prop:2.2.3}
    Let $ S \subseteq \mathbb{R}^3 $ be a regular surface and $ p \in S $. Then there exists a neighborhood $ V $ of $ p $ in $ S $ such that $ V $ is the graph of a differentiable function which has one of the following three forms: $ z = f(x,y) $, $ y = g(x,z) $, $ x = h(y,z) $.
\end{proposition}

\begin{proposition}
    \label{prop:2.2.4}
    Let $ p\in S $ be a point of a regular surface $ S $ and let $ \bvec{x}: U \subseteq \mathbb{R}^{2} \to \mathbb{R}^3 $ be a map with $ p \in \bvec{x}(U) \subseteq S $ such that the conditions 1 and 3 of Def. 1 (for a regular surface) hold. Assume that $ \bvec{x} $ is one-to-one. Then $ \bvec{x}^{-1} $ is continuous. 
\end{proposition}

% 2.2.1 
\begin{exercise}
    Show that the cylinder \( \{(x, y, z) \in \mathbb{R}^3, \ x^2 + y^2 = 1\} \) is a regular surface, and find parametrizations whose coordinate neighborhoods cover it.
\end{exercise}

\begin{solution}
    
\end{solution}

% 2.2.2 
\begin{exercise}
    Is the set \( \{(x, y, z) \in \mathbb{R}^3, \ z = 0 \text{ and } x^2 + y^2 \leq 1\} \) a regular surface? Is the set \( \{(x, y, z) \in \mathbb{R}^3, \ z = 0 \text{ and } x^2 + y^2 < 1\} \) a regular surface?
\end{exercise}

\begin{solution}
    
\end{solution}

% 2.2.3
\begin{exercise}
    Show that the two-sheeted cone, with its vertex at the origin, that is, the set \( \{(x, y, z) \in \mathbb{R}^3, \ x^2 + y^2 - z^2 = 0\} \), is not a regular surface.
\end{exercise}

\begin{solution}
    
\end{solution}

% 2.2.4
\begin{exercise}
    Let \( f(x, y, z) = z^2 \). Prove that \( 0 \) is not a regular value of \( f \) and yet that \( f^{-1}(0) \) is a regular surface.
\end{exercise}

\begin{solution}
    
\end{solution}

% 2.2.5
\begin{exercise}[\textbf{*}]
    Let \( P = \{(x, y, z) \in \mathbb{R}^3, \ x = y\} \) (a plane) and let \( x: U \subset \mathbb{R}^2 \to \mathbb{R}^3 \) be given by
    \[
    x(u, v) = (u + v, u + v, uv),
    \]
    where \( U = \{(u, v) \in \mathbb{R}^2; \ u > v\} \). Clearly, \( x(U) \subset P \). Is \( x \) a parametrization of \( P \)?
\end{exercise}

\begin{solution}
    
\end{solution}

% 2.2.6
\begin{exercise}
    Give another proof of Proposition 1 by applying Proposition 2 to \( h(x, y, z) = f(x, y) - z \).
\end{exercise}

\begin{solution}
    
\end{solution}

% 2.2.7
\begin{exercise}
    Let \( f(x, y, z) = (x + y + z - 1)^2 \).
    \begin{itemize}
        \item[a.] Locate the critical points and critical values of \( f \).
        \item[b.] For what values of \( c \) is the set \( f(x, y, z) = c \) a regular surface?
        \item[c.] Answer the questions of parts a and b for the function \( f(x, y, z) = xy z^2 \).
    \end{itemize}
\end{exercise}

\begin{solution}
    
\end{solution}

% 2.2.8
\begin{exercise}
    Let \( x(u, v) \) be as in Definition 1. Verify that \( dx_q: \mathbb{R}^2 \to \mathbb{R}^3 \) is one-to-one if and only if
    \[
    \frac{\partial x}{\partial u} \wedge \frac{\partial x}{\partial v} \neq 0.
    \]
\end{exercise}

\begin{solution}
    
\end{solution}

% 2.2.9
\begin{exercise}
    Let \( V \) be an open set in the \( xy \)-plane. Show that the set
    \[
    \{(x, y, z) \in \mathbb{R}^3; \ z = 0 \text{ and } (x, y) \in V \}
    \]
    is a regular surface.
\end{exercise}

\begin{solution}
    
\end{solution}

% 2.2.10
\begin{exercise}
    Let \( C \) be a figure ``8'' in the \( xy \)-plane and let \( S \) be the cylindrical surface over \( C \) (Fig. 2-11); that is,
    \[
    S = \{(x, y, z) \in \mathbb{R}^3; \ (x, y) \in C \}.
    \]
    Is the set \( S \) a regular surface?
\end{exercise}

\begin{solution}
    
\end{solution}

% 2.2.11 
\begin{exercise}
    Show that the set
    \[
        S=\{(x,y,z)\in\mathbb{R}^3\; ;\; z=x^2-y^2\}
    \]
    is a regular surface and check that parts (a) and (b) are parametrizations for $S$:
    \begin{enumerate}[(a)]
        \item $\mathbf{x}(u,v)=(u+v,\;u-v,\;4uv), \quad (u,v)\in\mathbb{R}^2.$
        \item $\mathbf{x}(u,v)=(u\cosh v,\;u\sinh v,\;u^2), \quad (u,v)\in\mathbb{R}^2,\; u\neq 0.$
    \end{enumerate}
    Which parts of $S$ do these parametrizations cover?
\end{exercise}

\begin{solution}
    ~

    Notice that $ z(x,y) = x^2 - y^2 $ is a differentiable function from the open set $ U = \mathbb{R}^2 $ to $ \mathbb{R} $, so by Proposition 2.2.1 in Do Carmo, $ S $ is a regular surface. Recall that a map $ \bvec{x}: U \to V \cap S $ if $ \bvec{x} $ is differentiable, a homeomorphism, and $ \mathrm{d}\bvec{x}_p $ is one-to-one for all $ p \in U $.
    \begin{enumerate}[(a)]
        \item The map $ \bvec{x} $ is a polyminial in $ u $ and $ v $, so it is differentiable.  By explicit calculation,
        \begin{equation*}
            \mathrm{d}\bvec{x}_q = 
            \begin{pmatrix}
                1 & 1 \\
                1 & -1 \\
                4v & 4u
            \end{pmatrix}
        \end{equation*}
        in the canonical basis, so $ \left\vert \partial (x,y) / \partial (u,v) \right\vert = 2 $ and $ \mathrm{d}\bvec{x} $ is one-to-one. To show that $ \bvec{x} $ is a homeomorphism, observe that for any $ (x,y,z) \in S $, we have $ z = x^2 - y^2 $, so $ z = (u+v)^2 - (u-v)^2 = 4uv $, and
        \begin{equation*}
            u = \frac{x+y}{2}, \quad v = \frac{x-y}{2}
        \end{equation*}
        from the remaining equations. This determines a unique $ (u,v) $ for each $ (x,y,z) \in S $, and we can conclude that the inverse map $ \bvec{x}^{-1} $ exists and is continuous. 
        \item The map $ \bvec{x} $ is a composition of polynomials and exponential functions, so it is differentiable. By explicit calculation,
        \begin{equation*}
            \mathrm{d}\bvec{x}_q = 
            \begin{pmatrix}
                \cosh v & u\sinh v \\
                \sinh v & u\cosh v \\
                2u & 0
            \end{pmatrix}
        \end{equation*}
        in the canonical basis, so $ \left\vert \partial (x,y) / \partial (u,v) \right\vert = u $, and $ \mathrm{d}\bvec{x} $ is one-to-one for $ u \neq 0 $. To show that $ \bvec{x} $ is a homeomorphism, observe that for any $ (x,y,z) \in S $ with $ x^2 - y^2 > 0 $, we have $ z = x^2 - y^2 $, so $ z = u^2(\cosh^2 v - \sinh^2 v) = u^2 $, and 
        \begin{equation*}
            u = \pm \sqrt{x^{2} - y^{2}}, \quad v = \tanh^{-1} \frac{y}{x}
        \end{equation*}
        from the remaining equations. This determines a unique $ (u,v) $ for each $ (x,y,z) \in S $ with $ x^2 - y^2 > 0 $, and we can conclude that the inverse map $ \bvec{x}^{-1} $ exists and is continuous.
    \end{enumerate}
    Parametrization (a) covers the whole surface $ S $, while parametrization (b) only covers the parts of $ S $ where $ \vert x \vert > \vert y \vert $.
    \begin{remark}
        The graph of $ z = f(x,y) = x^2 - y^2 $ is a hyperbolic paraboloid, also known as saddle. 
    \end{remark}
\end{solution}

% 2.2.12
\begin{exercise}
    Show that \( x: U \subset \mathbb{R}^2 \to \mathbb{R}^3 \) given by
    \[
    x(u, v) = (a \sin u \cos v, b \sin u \sin v, c \cos u), \quad a, b, c \neq 0, \quad 0 < u < \pi, \ 0 < v < 2\pi,
    \]
    is a parametrization for the ellipsoid
    \[
    \frac{x^2}{a^2} + \frac{y^2}{b^2} + \frac{z^2}{c^2} = 1.
    \]
    Describe geometrically the curves \( u = \text{const.} \) on the ellipsoid.
\end{exercise}

\begin{solution}
    
\end{solution}

% 2.2.13
\begin{exercise}[\textbf{*}]
    Find a parametrization for the hyperboloid of two sheets
    \[\{(x,y,z)\in\mathbb{R}^{3}\;:\;-x^{2}-y^{2}+z^{2}=1\}.\]
\end{exercise}

\begin{solution}
    
\end{solution}

% 2.2.14
\begin{exercise}
    A half-line \([0,\infty)\) is perpendicular to a line \(E\) and rotates about \(E\) from a given initial position while its origin \(0\) moves along \(E\). The movement is such that when \([0,\infty)\) has rotated through an angle \(\theta\), the origin is at a distance \(d=\sin^{2}(\theta/2)\) from its initial position on \(E\). Verify that by removing the line \(E\) from the image of the rotating line, we obtain a regular surface. If the movement were such that \(d=\sin(\theta/2)\), what else would need to be excluded to have a regular surface?
\end{exercise}

\begin{solution}
\end{solution}

% 2.2.15
\begin{exercise}[\textbf{*}]
    Let two points \(p(t)\) and \(q(t)\) move with the same speed, \(p\) starting from \((0,0,0)\) and moving along the \(z\)-axis, and \(q\) starting at \((a,0,0)\), \(a\neq0\), and moving parallel to the \(y\)-axis. Show that the line through \(p(t)\) and \(q(t)\) describes a set in \(\mathbb{R}^{3}\) given by
    \[ y(x-a)+zx=0. \]
    Is this a regular surface?
\end{exercise}

\begin{solution}
\end{solution}

% 2.2.16
\begin{exercise}
    One way to define a system of coordinates for the sphere $S^2$, given by
    \[
        x^2+y^2+(z-1)^2=1,
    \]
    is to consider the so-called stereographic projection
    \[
        \pi:S^2\setminus\{N\}\longrightarrow\mathbb{R}^2
    \]
    which carries a point $p=(x,y,z)$ of the sphere $S^2$ minus the north pole $N=(0,0,2)$ onto the intersection of the $xy$-plane with the straight line which connects $N$ to $p$ (Fig.\ 2--12). Let $(u,v)=\pi(x,y,z)$, where $(x,y,z)\in S^2\setminus\{N\}$ and $(u,v)$ lies in the $xy$-plane.
    \begin{enumerate}[label=\textbf{\alph*.}]
        \item Show that $\pi^{-1}:\mathbb{R}^2\to S^2$ is given by
        \[
            x=\frac{4u}{u^2+v^2+4},\qquad
            y=\frac{4v}{u^2+v^2+4},\qquad
            z=\frac{2(u^2+v^2)}{u^2+v^2+4}.
        \]
        \item Show that it is possible, using stereographic projection, to cover the sphere with two coordinate neighborhoods.
    \end{enumerate}
\end{exercise}

\begin{solution}
    ~
    \begin{enumerate}[label=\textbf{\alph*.}]
        \item Let's construct the map $ \pi : S^2 \to \mathbb{R}^2 $ explicitly. For a point $ p = (x,y,z) \in S^2 \setminus \{N\} $, the line connecting $ N $ and $ p $ can be parametrized as 
        \begin{equation}
            \label{eq:line}
            L(t) = N + t(p - N) = (0,0,2) + t(x,y,z-2) = (tx, ty, 2 + t(z-2))
        \end{equation}
        The intersection of this line with the $ xy $-plane occurs when $ z = 0 $, so $ t = 2 / (2-z) $. Substituting this back to equation (\ref{eq:line}) gives 
        \begin{equation*}
            \pi(p) = (u,v) = \left( \frac{2x}{2-z}, \frac{2y}{2-z} \right).
        \end{equation*}
        Solving for $ (x,y) $ gives
        \begin{equation*}
            (x,y) = \left( \frac{u(2-z)}{2}, \frac{v(2-z)}{2} \right).
        \end{equation*}
        From the equation for the sphere, we have 
        \begin{equation*}
            \left(\frac{u(2-z)}{2}\right)^2 + \left(\frac{v(2-z)}{2}\right)^2 + (z-1)^2 = 1 \; \Longrightarrow \; z = \frac{2(u^2 + v^2)}{u^2 + v^2 + 4},
        \end{equation*}
        hence
        \begin{equation*}
            x = \frac{4u}{u^2 + v^2 + 4}, \quad y = \frac{4v}{u^2 + v^2 + 4}, \quad z = \frac{2(u^2 + v^2)}{u^2 + v^2 + 4}.
        \end{equation*}
        \item Using the inverse stereographic projection $ \pi^{-1} $, we can cover the whole sphere except the north pole $ N $. To cover the north pole, use another stereographic projection from the south pole $ S = (0,0,0) $ to the $ xy $-plane, with the inverse map given by
        \begin{equation*}
            x = \frac{4u}{u^2 + v^2 + 4}, \quad y = \frac{4v}{u^2 + v^2 + 4}, \quad z = \frac{8}{u^2 + v^2 + 4}.
        \end{equation*}
    \end{enumerate}
\end{solution}

% 2.2.17
\begin{exercise}
    Define a regular curve in analogy with a regular surface. Prove that
    \begin{enumerate}[label=\textbf{\alph*.}]
        \item The inverse image of a regular value of a differentiable function \(f:U\subset\mathbb{R}^{2}\to\mathbb{R}\) is a regular plane curve. Give an example of such a curve which is not connected.
        \item The inverse image of a regular value of a differentiable map \(F:U\subset\mathbb{R}^{3}\to\mathbb{R}^{2}\) is a regular curve in \(\mathbb{R}^{3}\). Show the relationship between this proposition and the classical way of defining a curve in \(\mathbb{R}^{3}\) as the intersection of two surfaces.
        \item[\textbf{*c.}] The set \(C=\{(x,y)\in\mathbb{R}^{2}\;:\;x^{2}=y^{3}\}\) is not a regular curve. 
    \end{enumerate}
\end{exercise}

\begin{solution}
\end{solution}

% 2.2.18
\begin{exercise}[\textbf{*}]
    Suppose that
    \[
    f(x,y,z)=u=\text{const.},\qquad
    g(x,y,z)=v=\text{const.},\qquad
    h(x,y,z)=w=\text{const.},
    \]
    describe three families of regular surfaces and assume that at \((x_{0},y_{0},z_{0})\) the Jacobian
    \[ \frac{\partial(f,g,h)}{\partial(x,y,z)}\neq0. \]
    Prove that in a neighborhood of \((x_{0},y_{0},z_{0})\) the three families will be described by a mapping \(F(u,v,w)=(x,y,z)\) of an open set of \(\mathbb{R}^{3}\) into \(\mathbb{R}^{3}\), where a local parametrization for the surface of the family \(f(x,y,z)=u\), for example, is obtained by setting \(u=\text{const.}\) in this mapping.

    Determine \(F\) for the case where the three families of surfaces are:
    \[
    \begin{aligned}
    f(x,y,z)&=x^{2}+y^{2}+z^{2}=u=\text{const.} &\text{(spheres with center }(0,0,0));\\
    g(x,y,z)&=\frac{y}{x}=v=\text{const.} &\text{(planes through the }z\text{-axis});\\
    h(x,y,z)&=\frac{x^{2}+y^{2}}{z^{2}}=w=\text{const.} &\text{(cones with vertex at }(0,0,0)).
    \end{aligned}
    \]
\end{exercise}

\begin{solution}
\end{solution}

% 2.2.19
\begin{exercise}[\textbf{*}]
    ~

    Let $\alpha:(-3,0)\to\mathbb{R}^2$ be defined by (Fig.\ 2--13)
    \[
        \alpha(t)=
        \begin{cases}
        (0,\;-(t+2)), & t\in(-3,-1),\\[2pt]
        \text{a regular parametrized curve joining } p=(0,-1)\text{ to } q=\left(\dfrac{1}{\pi},0\right), & t\in(-1,-\tfrac{1}{\pi}),\\[8pt]
        (-t,\; \sin \tfrac{1}{t}), & t\in\!\left(-\tfrac{1}{\pi},\,0\right).
        \end{cases}
    \]

    It is possible to define the curve joining $p$ to $q$ so that all the derivatives of $\alpha$ are continuous at the corresponding points and $\alpha$ has no self-intersections. Let $C$ be the trace of $\alpha$.
    \begin{enumerate}[label=\textbf{\alph*.}]
        \item Is $C$ a regular curve?
        \item Let a normal line to the plane $\mathbb{R}^2$ run through $C$ so that it describes a ``cylinder'' $S$. Is $S$ a regular surface?
    \end{enumerate}
\end{exercise}

\begin{solution}
    ~
    \begin{enumerate}[label=\textbf{\alph*.}]
        \item Let $ C $ be the trace of $ \alpha $, $ \alpha $ is said to be regular if at every point $ p \in C $, $ C $ is the graph of a $ C^1 $ function $ y = f(x) $ or $ x = g(y) $ in a neighborhood of $ p $. Notice that the origin $ (0,0) $ belongs to the trace of $ \alpha $ since $ \alpha (-2) = (0,0) $. Consider the sequence $ t_n = -\frac{1}{2n \pi} $, which satisfies $ t_n \in (-\frac{1}{\pi},0) $ for all $ n \in \mathbb{N} $. Therefore, in any neighborhood of $ (0,0) $, we can find some $ n \in \mathbb{N} $ such that $ \alpha (t_n) \in U $, so $ C $ cannot be the graph of $ x = f(y) $ locally. Similarly, $ C $ cannot be the graph of $ y = g(x) $ on the line segment $ \{0\}\times (-1,1) \subseteq \mathbb{R}^{2} $. Hence, $ C $ is not a regular curve.
        \item If the surface $ S $ were regular, then by Do Carmo Proposition 2.2.3, there exists a neighborhood $ V $ of any $ p \in S $ such that $ V $ is the graph of a differentiable function $ z = f(x,y) $ or $ x = g(y,z) $ or $ y = h(x,z) $. However, consider a point $ p \in (-\frac{1}{\pi}, 0, z) $ on the side boundary of $ S $. In (a) we concluded that locally around $ (0,0,z) $, the curve (translated by some $ z $ along the $ z $ axis) is not the graph of a $ C^1 $ function $ x = g(y,z) $ or $ y = h(x,z) $, while $ z $ cannot be a function of $ x $, $ y $. Therefore, $ S $ is not a regular surface.
    \end{enumerate}
\end{solution}

\newpage

\subsection{Chapter 2.3}

\begin{definition}[differentiability on a surface]
    \label{def:differentiability}
    Let $ f: V \subseteq S \to \mathbb{R} $ be a function defined on an open set $ V $ of a regular surface $ S $. Then $ f $ is said to be \emph{differentiable at} $ p \in V $ if, for some parametrization $ \bvec{x}: U \subseteq \mathbb{R}^2 \to S $ with $ p \in \bvec{x}(U) \subseteq V $, the composition $ f \circ \bvec{x} $ is differentiable at $ \bvec{x}^{-1}(p) \in U $. $ f $ is \emph{differentiable} in $ V $ if it is differentiable at every point $ p \in V $.
\end{definition}

\begin{definition}[diffeomorphism]
    \label{def:diffeomorphism}
    Two regular surfaces $ S_1 $ and $ S_2 $ are said to be \emph{diffeomorphic} if there exists a differentiable map $ \phi : S_1 \to S_2 $ with a differentiable inverse $ \phi^{-1} : S_2 \to S_1 $. Such a map $ \phi $ is called a \emph{diffeomorphism} from $ S_2 $ to $ S_1 $.
\end{definition}

\begin{remark}
    The natural notion of equivalence associated with differentiability is the notion of diffeomorphism. From the point of view of differentiability, two diffeomorphic surfaces are indistinguishable.
\end{remark}

\begin{proposition}[change of parameters]
    \label{prop:change_of_parameters}
    Let $ p \in S $ be a point of a regular surface, and let $ \bvec{x}: U \subseteq \mathbb{R}^2 \to S $, $ \bvec{y}: V \subseteq \mathbb{R}^2 \to S $ be two parametrizations with $ p \in \bvec{x}(U) \cap \bvec{y}(V) \equiv W $. Then the \emph{change of coordinates map} $ h = \bvec{x}^{-1} \circ \bvec{y}: \bvec{y}^{-1} (W) \to \bvec{x}^{-1} (W) $ is a diffeomorphism. 
\end{proposition}

% 2.3.1
\begin{exercise}[\textbf{*}]
    Let \(S^{2}=\{(x,y,z)\in\mathbb{R}^{3}\;:\;x^{2}+y^{2}+z^{2}=1\}\) be the unit sphere and let
    \(A:S^{2}\to S^{2}\) be the (antipodal) map
    \[
    A(x,y,z)=(-x,-y,-z).
    \]
    Prove that \(A\) is a diffeomorphism.
\end{exercise}

\begin{solution}
    To show that $ A $ is differentiable, we need to show that for an atlas (a collection of charts that cover the surface) $ \left(\phi_\alpha : U_\alpha \to V_\alpha \right)_{\alpha \in J} $ of $ S^2 $, the maps 
    \begin{equation*}
        \phi^{-1}_\beta \circ A \circ \phi_\alpha : \phi^{-1}_\alpha (U_\alpha \cap U_\beta \cap A^{-1}(U_\alpha \cap U_\beta)) \to \phi^{-1}_\beta (U_\alpha \cap U_\beta \cap A(U_\alpha \cap U_\beta))
    \end{equation*}
    are differentiable. Recall that $ S^2 $ has an atlas consisting of the stereographic projections from the north and south poles:
    \begin{equation*}
        \left\{ \left(\phi_S: U_S \to V_S\right), \; \left(\phi_N: U_N \to V_N\right) \right\}.
    \end{equation*}
    Let the sphere be centered about \( (0,0,0) \) with the north and south poles at \( (0,0,1) \) and \( (0,0,-1) \), respectively. Then
    \begin{equation*}
        \phi_S(x,y,z) = \left( \frac{x}{1+z}, \frac{y}{1+z} \right), \quad
        \phi_N(x,y,z) = \left( \frac{x}{1-z}, \frac{y}{1-z} \right).
    \end{equation*}  
    It is trivial to check that the maps 
    \begin{equation*}
        \begin{split}
            \phi^{-1}_S \circ A \circ \phi_N (x,y) &= (-x,-y) = \phi^{-1}_N \circ A \circ \phi_S (x,y), \\
            \phi^{-1}_N \circ A \circ \phi_N &= \left(- \frac{x^2}{x^2 + y^2}, - \frac{y^2}{x^2 + y^2}\right) = \phi^{-1}_S \circ A \circ \phi_S (x,y)
        \end{split}
    \end{equation*}
    are differentiable for all $ (x,y,z) \in \mathbb{R}^3 \setminus \{(0,0,0)\} $. Since $ A^{-1} = A $, the same analysis applies to $ A^{-1} $, and thus $ A $ is a diffeomorphism. 
\end{solution}

% 2.3.2
\begin{exercise}
    Let \(S\subset\mathbb{R}^{3}\) be a regular surface and let
    \(\pi:S\to\mathbb{R}^{2}\) be the map which takes each \(p\in S\) to its
    orthogonal projection onto
    \[ \mathbb{R}^{2}=\{(x,y,z)\in\mathbb{R}^{3}\;:\;z=0\}. \]
    Is \(\pi\) differentiable?
\end{exercise}

\begin{solution}
    The map $ \pi $ is differentiable if for each $ p \in S $, there exists a parametrization $ \bvec{x} : U \subseteq \mathbb{R}^2 \to S $ with $ p \in \bvec{x}(U) $ such that the composition $ \pi \circ \bvec{x} : U \to \mathbb{R}^2 $ is differentiable. 

    Given the standard basis $ \{e_j\} $ of $ \mathbb{R}^3 $, we may assume that $ N(p) = e_3 $, where $ N(p) $ is the normal vector of $ S $ at $ p $. There exists a neighborhood $ V $ of $ p $ in $ S $ such that $ V $ is the graph of a differentiable function of the form $ z = f(x,y) $. Thus, we can choose a parametrization $ \bvec{x} : U \subseteq \mathbb{R}^2 \to S $ defined by $ \bvec{x}(u,v) = u e_1 + v e_2 + f(u,v) e_3 $. By applying translations in $ \mathbb{R}^3 $ and $ U $, we can ensure that $ p=0 $ and $ \bvec{x}(0,0) = 0 $, and $ T_p (S) = \mathbb{R} e_1 + \mathbb{R} e_2 $. Then 
    \[
        \pi \circ \bvec{x} : U \to \mathbb{R}^2, \quad (u,v) \mapsto u e_1 + v e_2
    \]
    is differentiable for all $ p \in \mathbb{R}^3 $. Also, note that $ \mathrm{d} \pi $ is injective, since in this coordinate we have 
    \[
        \mathrm{d} \pi = \begin{pmatrix} 1 & 0 \\ 0 & 1 \end{pmatrix}.
    \]
\end{solution}

% 2.3.3
\begin{exercise}
    Show that the paraboloid \(z=x^{2}+y^{2}\) is diffeomorphic to a plane.
\end{exercise}

\begin{solution}
    Consider the map $ \phi : \mathbb{R}^2 \to S $ defined by $ \phi (u,v) = (u, v, u^2 + v^2). $ It is easy to see that $ \phi $ is differentiable and one-to-one. The inverse map $ \phi^{-1} : S \to \mathbb{R}^2 $ is given by $ \phi^{-1} (x,y,z) = (x,y) $, which is also differentiable. Thus, the paraboloid is diffeomorphic to the plane.
\end{solution}

% 2.3.4
\begin{exercise}
    Construct a diffeomorphism between the ellipsoid
    \[
    \frac{x^{2}}{a^{2}}+\frac{y^{2}}{b^{2}}+\frac{z^{2}}{c^{2}}=1
    \]
    and the sphere
    \[
    x^{2}+y^{2}+z^{2}=1.
    \]
\end{exercise}

\begin{solution}
    Let $ \phi : \mathbb{R}^3 \to \mathbb{R}^3 $ be defined by 
    \[
        \phi (x,y,z) = \left( \frac{x}{a}, \frac{y}{b}, \frac{z}{c} \right).
    \]
    It is easy to see that $ \phi $ is differentiable and one-to-one. The inverse map $ \phi^{-1} : \mathbb{R}^3 \to \mathbb{R}^3 $ is given by
    \[
        \phi^{-1} (x,y,z) = (ax, by, cz),
    \] 
    which is also differentiable. Thus, the ellipsoid is diffeomorphic to the sphere.
\end{solution}

% 2.3.5
\begin{exercise}[\textbf{*}]
    *Let \(S\subset\mathbb{R}^{3}\) be a regular surface, and define \(d:S\to\mathbb{R}\) by
    \[
    d(p)=\lvert p-p_{0}\rvert,
    \]
    where \(p\in S\), \(p_{0}\in\mathbb{R}^{3}\), and \(p_{0}\notin S\).
    That is, \(d\) is the distance from \(p\) to a fixed point \(p_{0}\) not in \(S\).
    Prove that \(d\) is differentiable.
\end{exercise}

\begin{solution}
    By definition \ref{def:differentiability}, it suffices to show that for any parametrization $ \bvec{x}: U \subset \mathbb{R}^2 \to S $, the composition $ d \circ \bvec{x} : U \to \mathbb{R} $ is differentiable. Since $ S $ is a regular surface, for any point $ p \in S $, there exists a neighborhood $ V \subseteq \mathbb{R}^3 $ of $ p $ such that $ V \cap S $ is the graph of a differentiable function $ z(x,y) $ or $ x(y,z) $ or $ y(x,z) $. Assume that $ V \cap S $ is the graph of a differentiable function $ z(x,y) $, then define a parametrization
    \begin{equation*}
        \bvec{x}(u,v) = (u,v,z(u,v)), \quad (u,v) \in U \subseteq \mathbb{R}^2,
    \end{equation*}
    where $ U $ is open in $ \mathbb{R}^2 $. The composition $ d \circ \bvec{x} : U \to \mathbb{R} $ is given by
    \begin{equation*}
        \begin{split}
            (d \circ \bvec{x})(u,v) &= d(\bvec{x}(u,v)) = \sqrt{ \langle \bvec{x}(u,v) - p_0,\, \bvec{x}(u,v) - p_0 \rangle } \\
            &= \sqrt{ (u - x_0)^2 + (v - y_0)^2 + (z(u,v) - z_0)^2 }. 
        \end{split}
    \end{equation*}
    Since 
    \begin{equation*}
        \begin{split}
            \left. \frac{\partial}{\partial u} (d \circ \bvec{x})(u,v) \right|_{(u,v)} &= \frac{(u - x_0 + (z(u,v) - z_0) z_u(u,v))}{\sqrt{(u - x_0)^2 + (v - y_0)^2 + (z(u,v) - z_0)^2}}, \\
            \left. \frac{\partial}{\partial v} (d \circ \bvec{x})(u,v) \right|_{(u,v)} &= \frac{(v - y_0 + (z(u,v) - z_0) z_v(u,v))}{\sqrt{(u - x_0)^2 + (v - y_0)^2 + (z(u,v) - z_0)^2}},
        \end{split}
    \end{equation*}
    and $ z(u,v) $ is differentiable, we conclude that $ d \circ \bvec{x} $ is differentiable except when $ (u,v) = (x_0,y_0) = \bvec{x}^{-1}(p_0) $. Since the choice of $ p \in S $ is arbitrary, we conclude that $ d $ is differentiable on $ S \backslash \{p_0\} $.
\end{solution}

% 2.3.6
\begin{exercise}
    Prove that the definition of a differentiable map between surfaces does not depend on the parametrizations chosen.
\end{exercise}

\begin{solution}
    \begin{definition}[differentiable map between surfaces] 
        A map $ \phi : S_1 \to S_2 $ between two regular surfaces is said to be \emph{differentiable} at $ p \in S_1 $ if for some parametrizations $ \bvec{x}_1 : U_1 \subseteq \mathbb{R}^2 \to S_1 $ and $ \bvec{x}_2 : U_2 \subseteq \mathbb{R}^2 \to S_2 $ with $ p \in \bvec{x}_1(U_1) $ and $ \phi(p) \in \bvec{x}_2(U_2) $, the composition map $ \bvec{x}_2^{-1} \circ \phi \circ \bvec{x}_1 $ is differentiable at $ \bvec{x}_1^{-1}(p) $.
    \end{definition}
    Suppose that $ \phi : S_1 \to S_2 $ is differentiable at $ p \in S_1 $ with respect to parametrizations $ \bvec{x}_1 : U_1 \subseteq \mathbb{R}^2 \to S_1 $ and $ \bvec{x}_2 : U_2 \subseteq \mathbb{R}^2 \to S_2 $. Then $ \bvec{x}_2^{-1} \circ \phi \circ \bvec{x}_1 $ is differentiable at $ q = \bvec{x}_1^{-1}(p) $. Let $ \bvec{y}_1 : V_1 \subseteq \mathbb{R}^2 \to S_1 $ and $ \bvec{y}_2 : V_2 \subseteq \mathbb{R}^2 \to S_2 $ be another pair of parametrizations with $ p \in \bvec{y}_1(V_1) $ and $ \phi(p) \in \bvec{y}_2(V_2) $. Then the map 
    \[
        \bvec{y}_2^{-1} \circ \phi \circ \bvec{y}_1 = (\bvec{y}_2^{-1} \circ \bvec{x}_2) \circ (\bvec{x}_2^{-1} \circ \phi \circ \bvec{x}_1) \circ (\bvec{x}_1^{-1} \circ \bvec{y}_1)
    \]
    is differentiable at $ q $ since $ \bvec{y}_2^{-1} \circ \bvec{x}_2 $ and $ \bvec{x}_1^{-1} \circ \bvec{y}_1 $ are change of coordinates maps, which are diffeomorphisms by proposition \ref{prop:change_of_parameters}. Conversely, suppose $ \bvec{y}_2^{-1} \circ \phi \circ \bvec{y}_1 $ is differentiable at $ q $, then by the same argument $ \bvec{x}_2^{-1} \circ \phi \circ \bvec{x}_1 $ is also differentiable at $ q $. Thus, the definition of differentiability of a map between surfaces does not depend on the choice of parametrizations.
\end{solution}

% 2.3.7
\begin{exercise}
    Prove that the relation “\(S_{1}\) is diffeomorphic to \(S_{2}\)” is an equivalence relation in the set of regular surfaces.
\end{exercise}

\begin{solution}
    We verify the three properties of an equivalence relation. When $ S_1 $ is diffeomorphic $ S_2 $, we write $ S_1 \cong S_2 $.
    \begin{enumerate}[(i)]
        \item \textbf{Reflexivity:} For any regular surface $ S $, the identity map $ \mathrm{id}_S : S \to S $ is a diffeomorphism since it is differentiable and its inverse (itself) is also differentiable. Thus, $ S \cong S $.
        \item \textbf{Symmetry:} Suppose $ S_1 $ and $ S_2 $ are regular surfaces such that $ S_1 \cong S_2 $. Then there exists a diffeomorphism $ \phi : S_1 \to S_2 $ with differentiable inverse $ \phi^{-1} : S_2 \to S_1 $. Therefore, $ S_2 \cong S_1 $.
        \item \textbf{Transitivity:} Suppose $ S_1 $, $ S_2 $, and $ S_3 $ are regular surfaces such that $ S_1 \cong S_2 $ and $ S_2 \cong S_3 $. Then $ S_1 \cong S_3 $ since composition of differentiable maps are differentiable, and composition of bijections are bijective.
    \end{enumerate}
\end{solution}

% 2.3.8
\begin{exercise}[\textbf{*}]
    Let \(S^{2}=\{(x,y,z)\in\mathbb{R}^{3}\;:\;x^{2}+y^{2}+z^{2}=1\}\) and
    \(H=\{(x,y,z)\in\mathbb{R}^{3}\;:\;x^{2}+y^{2}-z^{2}=1\}\).
    Denote by \(N=(0,0,1)\) and \(S=(0,0,-1)\) the north and south poles of \(S^{2}\), respectively,
    and let \(F:S^{2}-\{N\}\cup\{S\}\to H\) be defined as follows:

    For each \(p\in S^{2}-\{N\}\cup\{S\}\), let the perpendicular from \(p\) to the \(z\)-axis meet \(Oz\) at \(q\).
    Consider the half-line \(l\) starting at \(q\) and containing \(p\).
    Then \(F(p)=l\cap H\) (see Fig.~2--20).

    Prove that \(F\) is differentiable.

    \begin{figure}
        \centering
        \includegraphics[width=0.6\textwidth]{2-20.png}
        \caption{}
    \end{figure}
\end{exercise}

\begin{solution}
    The map $ F $ is a projection of the sphere onto a one-sheeted hyperboloid along lines parallel to the $ Oxy $ plane. Consider (1) the parametrization of the sphere 
    \[
        \bvec{x}(\theta, \phi) = (\cos \theta \sin \phi, \sin \theta \sin \phi, \cos \phi)    
    \] 
    for $ \theta \in [0, 2\pi) $, $ \phi \in (0, \pi) $, and (2) the parametrization for the hyperboloid
    \[
        \bvec{y}(u,v) = (\sqrt{1+v^2} \cos u, \sqrt{1+v^2} \sin u, v)
    \]
    Then we have 
    \[
        (\bvec{y}^{-1} \circ F \circ \bvec{x})(\theta , \phi) = \bvec{y}^{-1} \left(\sqrt{1+\cos^2 \phi} \cos \theta ,\, \sqrt{1+\cos^2 \phi} \sin \theta , \cos \phi \right) = (u,v).
    \]
    Since $ \bvec{y}^{-1} \circ F \circ \bvec{x} $ is differentiable for the parametrization $ \bvec{x} $, $ \bvec{y} $, $ F $ is differentiable.
\end{solution}

% 2.3.9 
\begin{exercise}
    ~
    \begin{enumerate}[label=\textbf{\alph*.}]
    \item Define the notion of differentiable function on a regular curve.
    What does one need to prove for the definition to make sense?
    Do not prove it now. If you have not omitted the proofs in this section,
    you will be asked to do it in Exercise 15.

    \item Show that the map \(E:\mathbb{R}\to S^{1}=\{(x,y)\in\mathbb{R}^{2}\;:\;x^{2}+y^{2}=1\}\)
    given by
    \[
    E(t)=(\cos t,\sin t),\quad t\in\mathbb{R},
    \]
    is differentiable (geometrically, \(E\) “wraps” \(\mathbb{R}\) around \(S^{1}\)).
    \end{enumerate}
\end{exercise}

\begin{solution}
    ~
    \begin{enumerate}[label=\textbf{\alph*.}]
        \item Suppose $ \alpha: I \to \mathbb{R}^2 $ is a regular curve with trace $ C \subseteq \mathbb{R}^2 $. We say that a function $ f: C \to \mathbb{R} $ is differentiable along $ C $ if the composition $ f \circ \alpha : I \to \mathbb{R} $ is differentiable. In other words, we need to check that the derivative $ (f \circ \alpha)'(t) $ exists for all $ t \in I $.
        \item *
    \end{enumerate}
\end{solution}

% 2.3.10
\begin{exercise}
    Let \(C\) be a plane regular curve which lies on one side of a straight line \(r\) of the plane and meets \(r\) at the points \(p,q\) (see Fig.~2--21). What conditions should \(C\) satisfy to ensure that the rotation of \(C\) about \(r\) generates an extended (regular) surface of revolution?
\end{exercise}

\begin{solution}
    We can analyze the point $ p \in C $ locally. Assume that $ r $ is the $ z $ axis, and $ C $ is the graph of a differentiable function $ y = f(x) $ in a neighborhood of $ p $, since $ C $ is a regular curve. Since $ S $ is the surface of revolution generated by rotating $ C $ about $ r $, we claim that there is a local chart at $ p \in S $ given by
    \begin{equation*}
        \bvec{x}: U \subseteq \mathbb{R}^{2} \to S, \quad (x,y) \mapsto (x, y, f(\sqrt{x^2 + y^2})),
    \end{equation*} 
    where $ U $ is an open set in $ \mathbb{R}^2 $. We will check each condition given in definition (\ref{def:regular_surface}) for $ S $.
    \begin{enumerate}[(i)]
        \item $ \bvec{x} $ is differentiable. We can calculate its differential at some $ (x,y) \in U $ as 
        \begin{equation}
            \label{eq:differential}
            \mathrm{d}\bvec{x}_{(x,y)} = 
            \begin{pmatrix}
                1 & 0 \\
                0 & 1 \\
                \cfrac{x}{\sqrt{x^2 + y^2}} f^{\prime} (\sqrt{x^2 + y^2}) & \cfrac{y}{\sqrt{x^2 + y^2}} f^{\prime} (\sqrt{x^2 + y^2})
            \end{pmatrix}. 
        \end{equation}
        Since $ f $ is differentiable, the partial derivatives of $ \bvec{x} $ exist whenever $ (x,y) \neq (0,0) $. By symmetry, $ f(w) = f(-w) $, so $ f^{\prime} (w) = -f^{\prime} (-w) $. When $ (x,y) = (0,0) $, we have $ f^{\prime} (0) = 0 $, and 
        \begin{equation*}
            \frac{x}{\sqrt{x^2 + y^2} }, \quad \frac{y}{\sqrt{x^2 + y^2} }
        \end{equation*}
        are bounded, so $ \mathrm{d}\bvec{x}_{(x,y)} $ exists at $ (0,0) $. To satisfy the symmetry condition, we require that $ f^{\prime} $ is odd, hence $ f $ is even, and all the odd-order derivatives of $ f $ vanish at $ 0 $. Similarly, the odd-order derivatives of $ g $ such that $ y = g(x) $ in a neighborhood of $ q $ must also vanish.
        \item $ \bvec{x} $ is a homeomorphism, since the graph of a continuous function is homeomorphic to its domain.
        \item From equation (\ref{eq:differential}), we have $ \left\vert \partial (x,y) / \partial (u,v) \right\vert = 1 $, so $ \mathrm{d}\bvec{x} $ is one-to-one. Hence $ \mathrm{d}\bvec{x}_{(x,y)} $ is one-to-one for all $ (x,y) \in U $. 
    \end{enumerate}
\end{solution}

% 2.3.11
\begin{exercise}
    Prove that the rotations of a surface of revolution \(S\) about its axis are diffeomorphisms of \(S\).
\end{exercise}

\begin{solution}
    Let $ S $ be a surface of revolution generated by rotating a regular curve $ C $ around an axis $ r \subseteq \mathbb{R}^3 $, without loss of generality let $ r $ be the $ z $-axis and let $ C $ lie on the $ Oxz $ plane. Since $ C $ is a regular curve, it can be parametrized as $ (f(t), 0, g(t)) $ for $ t \in (a,b) $. Then $ S $ has a parametrization $ \bvec{x}(u,v) = \left(f(v)\cos u, f(v)\sin u, v\right) $ for $ u \in [0, 2\pi) $ and $ v \in (0, \infty) $. A rotation of $ S $ about its axis by an angle $ \theta $ is given by the map
    \[
        R_\theta : S \to S, \quad (x,y,z) \mapsto (x \cos \theta - y \sin \theta, x \sin \theta + y \cos \theta, z).
    \]
    Consider the composition $ (\bvec{y}^{-1} \circ R_\theta \circ \bvec{x}) : U \to U $ for parametrizatoins $ \bvec{x} $, $ \bvec{y} $ of $ S $. Then 
    \[
        \left(\bvec{y}^{-1} \circ R_\theta \circ \bvec{x}\right) (u,v) = \bvec{y}^{-1} \left((f(v)\cos (u+\theta), f(v)\sin (u+\theta), v)\right) = (u + \theta ,v).
    \]
    is differentiable since $ \bvec{x} $, $ \bvec{y} $, $ R_\theta $ is differentiable. Similarly, the inverse map $ R_{\theta}^{-1} = R_{-\theta} $ is also differentiable. Thus, $ R_\theta $ is a diffeomorphism of $ S $.
\end{solution}

% 2.3.12
\begin{exercise}
    Parametrized surfaces are often useful to describe sets \(\Sigma\) which are regular surfaces except for a finite number of points and a finite number of lines. For instance, let \(C\) be the trace of a regular parametrized curve\(\alpha:(a,b)\to\mathbb{R}^{3}\) which does not pass through the origin \(O=(0,0,0)\). Let \(\Sigma\) be the set generated by the displacement of a straight line \(l\) passing through a moving point \(p\in C\) and the fixed point \(0\) (a cone with vertex \(0\); see Fig.~2--22). 

    \begin{enumerate}[\textbf{\alph*.}]
        \item Find a parametrized surface $\mathbf{x}$ whose trace is $\Sigma$.
        \item Find the points where $\mathbf{x}$ is not regular.
        \item What should be removed from $\Sigma$ so that the remaining set is a regular surface?
    \end{enumerate}
\end{exercise}

\begin{solution}
    
\end{solution}

% 2.3.13
\begin{exercise}[\textbf{*}]
    Show that the definition of differentiability of a function 
    $f: V \subset S \to \mathbb{R}$ given in the text (Def.~1) is equivalent to the following: \emph{$f$ is differentiable in $p \in V$ if it is the restriction to $V$ of a differentiable function defined in an open set of $\mathbb{R}^3$ containing $p$.}
\end{exercise}

\begin{solution}
\end{solution}

% 2.3.14
\begin{exercise}
    Let $A \subset S$ be a subset of a regular surface $S$. Prove that $A$ is itself a regular surface if and only if $A$ is open in $S$; that is, 
    \[
    A = U \cap S,
    \]
    where $U$ is an open set in $\mathbb{R}^3$.
\end{exercise}

\begin{solution}
    Suppose $ A \subset S $ is a regular surface. Then there are parametrizations $ \bvec{x}_A : U_A \to A $ and $ \bvec{x}_S : U_S \to S $ from open sets $ U_A $ and $ U_S $ in $ \mathbb{R}^2 $. Then the map $ \bvec{x}_S^{-1} \circ \bvec{x}_A : U_A \to U_S $ is an open map: for $ V $ open in $ U_A $, $ ( \bvec{x}_S^{-1} \circ \bvec{x}_A ) (V) $ is open in $ \operatorname{dom} (\bvec{x}_S) $ by the Inverse Function Theorem and $ \bvec{x}_S $ is a homeomorphism. Therefore, $ \bvec{x}_A (V) = \bvec{x}_S \circ (\bvec{x}_S^{-1} \circ \bvec{x}_A ) (V) $ is open in $ S $. Now let $ V = U_A $ be the whole domain, then $ \bvec{x}_A (U_A) = A $ is open in $ S $. 
    
    Conversely, suppose $ A $ is open in $ S $. Let $ p \in A $, *
\end{solution}

% 2.3.15
\begin{exercise}
    Let $C$ be a regular curve and let $\alpha: I \subset \mathbb{R} \to C$, $\beta: J \subset \mathbb{R} \to C$ be two parametrizations of $C$ in a neighborhood of $p \in \alpha(I) \cap \beta(J) = W$. Let
    \[
    h = \alpha^{-1} \circ \beta : \beta^{-1}(W) \to \alpha^{-1}(W)
    \]
    be the change of parameters. Prove that
    \begin{enumerate}[label=\textbf{\alph*.}]
        \item $h$ is a diffeomorphism.
        \item The absolute value of the arc length of $C$ in $W$ does not depend on which parametrization is chosen to define it, that is,
        \[
        \left|\int_{t_0}^t |\alpha' (t)|\, dt\right| 
        = 
        \left|\int_{\tau_0}^\tau |\beta' (\tau)|\, d\tau\right|,
        \qquad t = h(\tau), \; t \in I, \; \tau \in J.
        \]
    \end{enumerate}
\end{exercise}

\begin{solution}
    ~

    \begin{enumerate}[label=\textbf{\alph*.}]
        \item By the chain rule and inverse function rule, 
        \[
            h^{\prime} (t) = \left. \frac{1}{\alpha^{\prime} (\alpha ^{-1})} \right|_{\beta} \circ \beta^{\prime} (t) .
        \]
        Since $ \alpha $ and $ \beta $ are parametrizations for a regular curve, $ \vert \alpha \vert , \vert \beta \vert \neq 0 $. Then $ h^{\prime} $ always exists and $ h $ is differentiable on $ \beta^{-1}(W) $. Similarly, we have $ h^{-1} = \beta^{-1} \circ \alpha $, so by a similar calculation we know $ h^{-1} $ is differentiable on $ \alpha^{-1}(W) $. Therefore, $ h $ is a diffeomorphism. 
        \item For $ t \in I $, $ \tau \in J $, we have    
        \begin{equation*}
            \begin{split}
                \left\vert \int^t_{t_0} \mathrm{d}t\, \vert \alpha^{\prime} (t) \vert \right\vert &= \left\vert \int^\tau_{\tau_0} \mathrm{d}\tau \, h^{\prime} (\tau) \left\vert (\alpha^{\prime} \vert_{h} \circ h (\tau)) \right\vert \right\vert \\
                &= \left\vert \int^{\tau}_{\tau_0} \mathrm{d}\tau \, \left.\frac{1}{\alpha^{\prime} (\alpha^{-1})} \right|_{\beta} \circ \beta^{\prime} (\tau) \left\vert \alpha^{\prime} \circ (\alpha^{-1} \circ \beta) (\tau) \right\vert \right\vert \\
                &= \left|\int_{\tau_0}^\tau |\beta' (\tau)|\, d\tau\right|.
            \end{split}
        \end{equation*}
    \end{enumerate}
\end{solution}

% 2.3.16
\begin{exercise}[\textbf{*}]
    Let $R^{2} = \{(x, y, z) \in \mathbb{R}^{3} ; z = -1\}$ be identified with the complex plane $\mathbb{C}$ by setting $(x, y, -1) = x + iy = \zeta \in \mathbb{C}$. Let $P : \mathbb{C} \to \mathbb{C}$ be the complex polynomial
    \[
    P(\zeta) = a_{0}\zeta^{n} + a_{1}\zeta^{n-1} + \cdots + a_{n}, \quad a_{0} \neq 0, \, a_{i} \in \mathbb{C}, \, i = 0, \ldots, n.
    \]
    Denote by $\pi_{N}$ the stereographic projection of $S^{2} = \{(x, y, z) \in \mathbb{R}^{3}; x^{2} + y^{2} + z^{2} = 1\}$ from the north pole 
    $N = (0, 0, 1)$ onto $R^{2}$. Prove that the map $F : S^{2} \to S^{2}$ given by
    \[
    F(p) = 
    \begin{cases}
    \pi_{N}^{-1} \circ P \circ \pi_{N}(p), & \text{if } p \in S^{2} - \{N\}, \\
    N, & \text{if } p = N,
    \end{cases}
    \]
    is differentiable.
\end{exercise}

\begin{solution}
    Given a point $ p \in S^2 \backslash \{N\} $, write it as $ p = (x, y, z) $. Since the composition of differentiable functions is differentiable, we only need to show that $ \pi_N, \pi_N^{-1} $ and $ P $ are differentiable. The stereographic projection $ \pi_N : S^2 \backslash \{N\} \to \mathbb{R}^2 $ is given by
    \[
        \pi_N(x, y, z) = \left( \frac{x}{1-z}, \frac{y}{1-z} \right) .
    \]
    Since $ z \neq 1 $ for all $ p \in S^2 \backslash \{N\} $, $ \pi_N $ is differentiable. Similarly, note that the inverse stereographic projection $ \pi_N^{-1} : \mathbb{R}^2 \to S^2 \backslash \{N\} $ is given by
    \[
        \pi_N^{-1}(u, v) = \left( \frac{2u}{u^2 + v^2 + 1}, \frac{2v}{u^2 + v^2 + 1}, \frac{u^2 + v^2 - 1}{u^2 + v^2 + 1} \right) .
    \]
    Since $ u^2 + v^2 + 1 > 0 $ for all $ (u, v) \in \mathbb{R}^2 $, $ \pi_N^{-1} $ is differentiable. Moreover, polynomials are differentiable everywhere, so $ P $ is differentiable. Thus, $ F $ is differentiable on $ S^2 \backslash \{N\} $.
\end{solution}

\newpage

\subsection{Chapter 2.4}

% 2.4.1
\begin{exercise}
    Show that the equation of the tangent plane at $(x_0, y_0, z_0)$ of a regular surface given by $f(x, y, z) = 0$, where $0$ is a regular value of $f$, is
    \[
    f_x(x_0, y_0, z_0)(x - x_0) + f_y(x_0, y_0, z_0)(y - y_0) + f_z(x_0, y_0, z_0)(z - z_0) = 0.
    \]
\end{exercise}

\begin{solution}
    Suppose $ 0 $ is a regular value of $ f $, then by definition we have a regular surface defined implicitly by $ f(x,y,z) = 0 $. By proposition, a regular surface can be locally represented as the graph of a differentiable function. Without loss of generality, write $ z = g(x,y) = f^{-1}(0) $ in a neighborhood of $ p = (x_0, y_0, z_0) $. By the Inverse Function Theorem, we have
    \begin{equation}
        \frac{\partial g}{\partial x} (x_0, y_0) = -\frac{f_x(x_0, y_0, z_0)}{f_z(x_0, y_0, z_0)}, \quad
        \frac{\partial g}{\partial y} (x_0, y_0) = -\frac{f_y(x_0, y_0, z_0)}{f_z(x_0, y_0, z_0)}.
    \end{equation}
    
    Then the tangent at $ p $ is given by
    \begin{equation}
        \begin{split}
            z &= g(x_0, y_0) + g_x(x_0, y_0)(x - x_0) + g_y(x_0, y_0)(y - y_0) \\
            &= z_0 - \left(\frac{\partial f / \partial x}{\partial f / \partial z}\right)_{p} (x - x_0) - \left(\frac{\partial f / \partial y}{\partial f / \partial z}\right)_{p} (y - y_0),
        \end{split}
    \end{equation} 
    and 
    \begin{equation}
        (x- x_0) f_x (x_0, y_0, z_0) + (y - y_0) f_y (x_0, y_0, z_0) + (z - z_0) f_z (x_0, y_0, z_0) = 0.
    \end{equation}
    \begin{remark}
        We can express this more compactly with $ \bvec{x}_0 = (x_0, y_0, z_0) $, $ \bvec{x} = (x, y, z) $ and $ \nabla f = (f_x, f_y, f_z) $, then 
        \begin{equation}
            (\bvec{x} - \bvec{x}_0) \cdot \nabla f(\bvec{x}_0) = 0. 
        \end{equation}
    \end{remark}
\end{solution}

% 2.4.2
\begin{exercise}
    Determine the tangent planes of $x^2 + y^2 - z^2 = 1$ at the points $(x, y, 0)$ and show that they are all parallel to the $z$-axis.
\end{exercise}

\begin{solution}
    Use the result of Exercise 2.4.1, we have
    \begin{equation}
        2x_0 (x - x_0) + 2y_0 (y - y_0) - 2z_0 (z - z_0) = 0.
    \end{equation}
    At the point $ (x_0, y_0, 0) $, we have
    \begin{equation}
        2x_0 (x - x_0) + 2y_0 (y - y_0) = 0.
    \end{equation}
    The normal vector of the tangent plane is $ (2x_0, 2y_0, 0) $, which is perpendicular to the $ z $-axis. Thus the tangent planes are all parallel to the $ z $-axis.
\end{solution}

% 2.4.3
\begin{exercise}
    Show that the equation of the tangent plane of a surface which is the graph of a differentiable function $z = f(x, y)$, at the point $p_0 = (x_0, y_0)$, is given by
    \[
    z = f(x_0, y_0) + f_x(x_0, y_0)(x - x_0) + f_y(x_0, y_0)(y - y_0).
    \]
    Recall the definition of the differential $df$ of a function $f: \mathbb{R}^2 \to \mathbb{R}$ and show that the tangent plane is the graph of the differential $df_p$.
\end{exercise}

\begin{solution}
    Since $ z_0 = f(x_0, y_0) $,  the tangent plane at $ p_0 = (x_0, y_0, z_0) $ is given by
    \begin{equation}
        z = f(x_0, y_0) + f_x(x_0, y_0)(x - x_0) + f_y(x_0, y_0)(y - y_0).
    \end{equation}
    
    Recall the definition of the differential of a function $ f: \mathbb{R}^2 \to \mathbb{R} $, we have
    \begin{equation}
        df_{(x_0, y_0)}(h, k) = f_x(x_0, y_0) h + f_y(x_0, y_0) k.
    \end{equation}
    Let $ h = x - x_0 $, $ k = y - y_0 $, then
    \begin{equation}
        df_{(x_0, y_0)}(x - x_0, y - y_0) = f_x(x_0, y_0)(x - x_0) + f_y(x_0, y_0)(y - y_0).
    \end{equation}
    Thus the tangent plane can be expressed as
    \begin{equation}
        z = f(x_0, y_0) + df_{(x_0, y_0)}(x - x_0, y - y_0),
    \end{equation}
    which is the graph of the differential $ df_{(x_0, y_0)} $.
\end{solution}

% 2.4.4
\begin{exercise}
    Show that the tangent planes of a surface given by $z = x f(y/x)$, $x \neq 0$, where $f$ is a differentiable function, all pass through the origin $(0, 0, 0)$.
\end{exercise}

\begin{solution}
    Let $ g(x,y) = x f(y/x) $, then $ z = g(x,y) $ and 
    \[
        g_x (x,y) = f(y/x) - \frac{y}{x} f^{\prime} (y/x), \quad g_y (x,y) = f^{\prime} (y/x).
    \]
    Since $ z_0 = x_0 f(y_0/x_0) $, the tangent plane at $ (x_0, y_0, z_0) $ is given by
    \begin{equation*}
        \begin{split}
            z &= x_0 f(y_0 / x_0) + g_x(x_0, y_0)(x - x_0) + g_y(x_0, y_0)(y - y_0) \\
            &= x_0 f(y_0 / x_0) + \left( f(y_0/x_0) - \frac{y_0}{x_0} f^{\prime} (y_0/x_0) \right) (x - x_0) + f^{\prime} (y_0/x_0) (y - y_0).
        \end{split}
    \end{equation*}
    We can check that $ (0,0,0) $ is a solution, hence the desired result. 
\end{solution}

% 2.4.5
\begin{exercise}
    If a coordinate neighborhood of a regular surface can be parametrized in the form
    \[
    \mathbf{x}(u,v)=\alpha_1(u)+\alpha_2(v),
    \]
    where $\alpha_1$ and $\alpha_2$ are regular parametrized curves, show that the tangent
    planes along a fixed coordinate curve of this neighborhood are all parallel to a line.
\end{exercise}

\begin{solution}
    Here $ \alpha_1 (u), \alpha_2 (v) \in \mathbb{R}^3 $. The tangent plane at $ (u_0, v_0) $ is spanned by the vectors
    \[
        \bvec{x}_u (u_0, v_0) = \alpha_1^{\prime} (u_0), \quad \bvec{x}_v (u_0, v_0) = \alpha_2^{\prime} (v_0).
    \]
    Then the normal vector of the plane is given by 
    \[
        N(u_0, v_0) = \frac{\bvec{x}_u \wedge \bvec{x}_v}{\left\vert \bvec{x}_u \wedge \bvec{x}_v \right\vert} (u_0, v_0) = \frac{\alpha_1^{\prime} \wedge \alpha_2^{\prime} }{\left\vert \alpha_1^{\prime} \wedge \alpha_2^{\prime} \right\vert } (u_0, v_0) .
    \]
    *
\end{solution}

% 2.4.6
\begin{exercise}
    Let $\alpha:I\to\mathbb{R}^3$ be a regular parametrized curve with everywhere nonzero
    curvature. Consider the tangent surface of $\alpha$ (Example~5 of Sec.~2--3)
    \[
    \mathbf{x}(t,v)=\alpha(t)+v\,\alpha'(t), \qquad t\in I,\; v\neq 0.
    \]
    Show that the tangent planes along the curve $\mathbf{x}(\text{const.},v)$ are all equal.
\end{exercise}

\begin{solution}
\end{solution}

% 2.4.7
\begin{exercise}
    Let $f:S\to\mathbb{R}$ be given by $f(p)=\lvert p-p_0\rvert^2$, where $p\in S$ and $p_0$ is
    a fixed point of $\mathbb{R}^3$ (see Example~1 of Sec.~2--3). Show that
    \[
    (df)_p(w)=2\, w\cdot (p-p_0), \qquad w\in T_p(S).
    \]
\end{exercise}

\begin{solution}
\end{solution}

% 2.4.8
\begin{exercise}
    Prove that if $L:\mathbb{R}^3\to\mathbb{R}^3$ is a linear map and $S\subset\mathbb{R}^3$ is
    a regular surface invariant under $L$, i.e., $L(S)\subset S$, then the restriction
    $L\lvert_S$ is a differentiable map and
    \[
    dL_p(w)=L(w), \qquad p\in S,\; w\in T_p(S).
    \]
\end{exercise}

\begin{solution}
\end{solution}

% 2.4.9
\begin{exercise}
    Show that the parametrized surface
    \[
    \mathbf{x}(u,v) = (v\cos u,\, v\sin u,\, a u), \qquad a \neq 0,
    \]
    is regular. Compute its normal vector $N(u,v)$ and show that along the coordinate line $u = u_0$ the tangent plane of $\mathbf{x}$ rotates about this line in such a way that the tangent of its angle with the $z$ axis is proportional to the inverse of the distance $v(=\sqrt{x^2+y^2})$ of the point $\mathbf{x}(u_0,v)$ to the $z$ axis.
\end{exercise}

\begin{solution}
    
\end{solution}

% 2.4.10
\begin{exercise}[Tubular Surfaces]
    Let $\alpha:I\to\mathbb{R}^3$ be a regular parametrized curve with nonzero curvature everywhere and arc length as parameter. Let
    \[
    \mathbf{x}(s,v) = \alpha(s)+r\bigl(n(s)\cos v+b(s)\sin v\bigr), \qquad r=\text{const.}\neq 0,\; s\in I,
    \]
    be a parametrized surface (the tube of radius $r$ around $\alpha$), where $n$ is the normal vector and $b$ is the binormal vector of $\alpha$. Show that, when $\mathbf{x}$ is regular, its unit normal vector is
    \[
    N(s,v) = -\bigl(n(s)\cos v+b(s)\sin v\bigr).
    \]
\end{exercise}

\begin{solution}
    Let $ \bvec{x}: U \to \mathbb{R}^3 $ as defined in the problem statement be a regular Parametrization, where $ U $ is an open set in $ \mathbb{R}^2 $. The unit normal vector at each point $ q \in \bvec{x}(U) $ is defined as
    \[
        N(q) = \frac{\bvec{x}_s \wedge \bvec{x}_v}{|\bvec{x}_s \wedge \bvec{x}_v|} (q) .
    \]
    Let prime denote derivative with respect to $ s $. Then we have
    \[
        \bvec{x}_s = \alpha'(s) + r \big( n'(s)\cos v + b'(s)\sin v \big), \quad
        \bvec{x}_v = r \big( -n(s)\sin v + b(s)\cos v \big),
    \]
    and by the Frenet-Serret formulas,
    \[ \alpha^{\prime} (s) = t(s), \quad n'(s) = -\kappa(s)t(s) - \tau (s) b(s), \quad b'(s) = \tau(s)n(s), \]
    where $ t $ is the unit tangent, $ \kappa $ is the curvature, and $ \tau $ is the torsion of $ \alpha $. Thus, 
    \begin{align*}
        \bvec{x}_s &= t(s) + r \big( (-\kappa(s)t(s) - \tau (s) b(s)) \cos v + \tau(s)n(s)\sin v \big), \\
        \bvec{x}_v &= r \big( -n(s)\sin v + b(s)\cos v \big).
    \end{align*}
    Now suppress $ s $ and compute the wedge product in the Frenet frame $ \{t, n, b\} $:
    \begin{align*}
        \bvec{x}_s \wedge \bvec{x}_v &= \big( t + r \big( -\kappa t \cos v - \tau b \cos v + \tau n \sin v \big) \big) \wedge r \big( -n \sin v + b \cos v \big) \\
        &= -r (t \wedge n) \sin v + r (t \wedge b) \cos v - r^2 \kappa \sin v \cos  v (t \wedge n) - r^2 \kappa \cos^2 v (t \wedge b) \\
        &\quad + r^2 \tau \sin v \cos v (b \wedge n ) + r^2 \tau \sin v \cos v (n \wedge b) \\
        &= -r (1 - r \kappa \cos v) \left( \cos v n + \sin v b \right). 
    \end{align*}

    Dividing by the norm and noting that $ n $ and $ b $ are unit length and orthogonal, we have
    \[
        N(s, v) = -\big( n(s)\cos v + b(s)\sin v \big).
    \]
\end{solution}

% 2.4.11
\begin{exercise}
    Show that the normals to a parametrized surface given by
    \[
    \mathbf{x}(u,v) = \bigl(f(u)\cos v,\, f(u)\sin v,\, g(u)\bigr), \qquad f(u)\neq 0,\; g'(u)\neq 0,
    \]
    all pass through the $z$ axis.
\end{exercise}

\begin{solution}
\end{solution}

% 2.4.12
\begin{exercise}
Show that each of the equations $(a, b, c \neq 0)$
\[
x^2 + y^2 + z^2 = a x, \quad
x^2 + y^2 + z^2 = b y, \quad
x^2 + y^2 + z^2 = c z
\]
define a regular surface and that they all intersect orthogonally.
\end{exercise}

\begin{solution}
    Recall the following proposition: 
    \begin{proposition}
        \label{prop:implicit_reg_surface}
        A surface defined implicitly by $ f(x, y, z) = 0 $ is a regular surface if $ 0 $ is a regular value of $ f $, i.e., $ \nabla f \neq 0 $ on the surface.
    \end{proposition}
    Let $ f_1(x, y, z) = x^2 + y^2 + z^2 - a x $, $ f_2(x, y, z) = x^2 + y^2 + z^2 - b y $, $ f_3(x, y, z) = x^2 + y^2 + z^2 - c z $. Then we have
    \begin{equation}
        \nabla f_1 = (2x - a, 2y, 2z), \quad
        \nabla f_2 = (2x, 2y - b, 2z), \quad
        \nabla f_3 = (2x, 2y, 2z - c).
    \end{equation}
    Since $ a, b, c \neq 0 $, we have $ \nabla f_1 = 0 $ implies $ (x, y, z) = (a/2, 0, 0) $, which does not satisfy the equation of the surface. Similarly, we can show that $ \nabla f_2 \neq 0 $ and $ \nabla f_3 \neq 0 $ on the surfaces. Thus all three surfaces are regular surfaces. Moreover, the normal vectors of the tangent planes at $ (x, y, z) $ are given by $ \nabla f_1 $, $ \nabla f_2 $, and $ \nabla f_3 $, respectively. Then we have
    \begin{equation}
        \begin{split}
            \nabla f_1 \cdot \nabla f_2 &= 4x^2 + 4y^2 + 4z^2 - 2 a x - 2 b y = 0, \\
            \nabla f_2 \cdot \nabla f_3 &= 4x^2 + 4y^2 + 4z^2 - 2 b y - 2 c z = 0, \\
            \nabla f_1 \cdot \nabla f_3 &= 4x^2 + 4y^2 + 4z^2 - 2 a x - 2 c z = 0. 
        \end{split}
    \end{equation}
    Hence they all intersect orthogonally.
\end{solution}

% 2.4.13
\begin{exercise}
A \emph{critical point} of a differentiable function $f: S \to \mathbb{R}$ defined on a regular surface $S$ is a point $p \in S$ such that $d f_p = 0$.

\begin{enumerate}[label=\textbf{\alph*.}]
    \item Let $f: S \to \mathbb{R}$ be given by $f(p) = |p - p_0|$, $p \in S$, $p_0 \notin S$ (cf.\ Exercise 5, Sec.\ 2--3). Show that $p \in S$ is a critical point of $f$ if and only if the line joining $p$ to $p_0$ is normal to $S$ at $p$.
    \item Let $h: S \to \mathbb{R}$ be given by $h(p) = p \cdot v$, where $v \in \mathbb{R}^3$ is a unit vector (cf.\ Example 1, Sec.\ 2--3). Show that $p \in S$ is a critical point of $f$ if and only if $v$ is a normal vector of $S$ at $p$.
\end{enumerate}
\end{exercise}

\begin{solution}
\end{solution}

% 2.4.14
\begin{exercise}
    Let $Q$ be the union of the three coordinate planes $x = 0$, $y = 0$, $z = 0$.
    Let $p = (x, y, z) \in \mathbb{R}^3 - Q$.
    \begin{enumerate}[label=\textbf{\alph*.}]
        \item Show that the equation in $t$,
        \[
        \frac{x^2}{a - t} + \frac{y^2}{b - t} + \frac{z^2}{c - t}
        \equiv f(t) = 1, \qquad a > b > c > 0,
        \]
        has three distinct real roots $t_1, t_2, t_3$.
        \item Show that for each $p \in \mathbb{R}^3 - Q$, the sets given by
        \[
        f(t_1) - 1 = 0, \quad f(t_2) - 1 = 0, \quad f(t_3) - 1 = 0
        \]
        are regular surfaces passing through $p$ which are pairwise orthogonal.
    \end{enumerate}
\end{exercise}

\begin{solution}
    ~
    \begin{enumerate}[label=\textbf{\alph*.}]
        \item Consider the function $ F $ which implicitly defines the surfaces:
        \[
            F(t) =\frac{x^2}{a - t} + \frac{y^2}{b - t} + \frac{z^2}{c - t} - 1.
        \]
        It has three vertical asymptotes at $ t = a, b, c $. Moreover, it is continuous and monotone increasing in each of the open intervals $ t \in (-\infty, c) $, $ (c, b) $, $ (b, a) $, $ (a, \infty) $. Thus by the Intermediate Value Theorem, there exist exactly three distinct real roots $ t_1 < c < t_2 < b < t_3 < a $. 
        \item Given $ F(t_j) $ and $ F(t_k) $, their point of intersection $ p $ is 
        \begin{equation*}
            \begin{split}
                0 &= F(t_j) - F(t_k) \\
                &= \left( \frac{x^2}{a - t_j} + \frac{y^2}{b - t_j} + \frac{z^2}{c - t_j} - 1 \right) - \left( \frac{x^2}{a - t_k} + \frac{y^2}{b - t_k} + \frac{z^2}{c - t_k} - 1 \right) \\
                &= (t_j - t_k) \left( \frac{x^2}{(a - t_j)(a - t_k)} + \frac{y^2}{(b - t_j)(b - t_k)} + \frac{z^2}{(c - t_j)(c - t_k)} \right).
            \end{split}
        \end{equation*}
        Then, assuming $ t_j \neq t_k $, we have
        \begin{equation*}
            \begin{split}
                \nabla (F(t_j)) \cdot \nabla (F(t_k)) &= 4 \left(\frac{x}{(a-t_j)},  \frac{y}{(b-t_j)}, \frac{z}{(c-t_j)}\right) \cdot \left(\frac{x}{(a-t_k)},  \frac{y}{(b-t_k)}, \frac{z}{(c-t_k)}\right) \\
                &= 4 \left( \frac{x^2}{(a-t_j)(a-t_k)} + \frac{y^2}{(b-t_j)(b-t_k)} + \frac{z^2}{(c-t_j)(c-t_k)} \right) = 0.
            \end{split}
        \end{equation*}
        Therefore, the surfaces intersect orthogonally for all $ p \in \mathbb{R}^3 $, i.e. they are pairwise orthogonal. Furthermore, we see $ \nabla F \vert_p = 0 $ if and only if $ p = 0 \notin S $, so by proposition~\ref{prop:implicit_reg_surface}, they are regular surfaces.
    \end{enumerate}
    \begin{remark}
        The surfaces described above are called \emph{confocal quadrics}. Any point $ (x_0, y_0, z_0) \in \mathbb{R}^3 \setminus \{0\} $ lies on exactly one surface of each of the three types of confocal quadratics, and the three quadrics intersect orthogonally at that point.
    \end{remark}
\end{solution}

% 2.4.15
\begin{exercise}
    Show that if all normals to a connected surface pass through a fixed point, the surface is contained in a sphere.
\end{exercise}

\begin{solution}
\end{solution}

% 2.4.16
\begin{exercise}
    Let $w$ be a tangent vector to a regular surface $S$ at a point $p\in S$ and let $\mathbf{x}(u,v)$ and $\bar{\mathbf{x}}(\bar{u},\bar{v})$ be two parametrizations at $p$. Suppose that the expressions of $w$ in the bases associated to $\mathbf{x}(u,v)$ and $\bar{\mathbf{x}}(\bar{u},\bar{v})$ are
    \[
    w = \alpha_1 \mathbf{x}_u + \alpha_2 \mathbf{x}_v
    \]
    and
    \[
    w = \beta_1 \bar{\mathbf{x}}_{\bar{u}} + \beta_2 \bar{\mathbf{x}}_{\bar{v}}.
    \]
    Show that the coordinates of $w$ are related by
    \[
    \beta_1 = \alpha_1 \frac{\partial \bar{u}}{\partial u} + \alpha_2 \frac{\partial \bar{u}}{\partial v}, \qquad
    \beta_2 = \alpha_1 \frac{\partial \bar{v}}{\partial u} + \alpha_2 \frac{\partial \bar{v}}{\partial v},
    \]
    where $\bar{u} = \bar{u}(u,v)$ and $\bar{v} = \bar{v}(u,v)$ are the expressions of the change of coordinates.
\end{exercise}

\begin{solution}
    
\end{solution}

% 2.4.17
\begin{exercise}[\textbf{*}]
    Two regular surfaces $S_1$ and $S_2$ intersect \emph{transversally} if whenever $p\in S_1\cap S_2$ then $T_p(S_1)\neq T_p(S_2)$. Prove that if $S_1$ intersects $S_2$ transversally, then $S_1\cap S_2$ is a regular curve.
\end{exercise}

\begin{solution}
    Let $ S_1, S_2 $ be two regular surfaces that intersect transversally, and let $ p \in S_1 \cap S_2 $. Since $ S_1, S_2 $ are regular surfaces, there exists a differentiable function $ f : \mathbb{R}^3 \to \mathbb{R} $ and a neighborhood $ V_1 $ of $ p $ such that $ S_1 \cap V_1 = f^{-1}(0) \cap V_1 $. Similarly, there exists a differentiable function $ g : \mathbb{R}^3 \to \mathbb{R} $ and a neighborhood $ V_2 $ of $ p $ such that $ S_2 \cap V_2 = g^{-1}(0) \cap V_2 $. Define $ F : \mathbb{R}^3 \to \mathbb{R}^2 $ by $ F(q) = (f(q), g(q)) $. Then
    \[ 
    F^{-1}(0,0) = f^{-1}((0,0)) \cap g^{-1}((0,0)) \supseteq (V_1 \cap V_2) \cap (S_1 \cap S_2). 
    \]
    Let $ V = V_1 \cap V_2 $. In $ V $, we have $ S_1 \cap S_2 = F^{-1}(0,0) $. Since $ T_p(S_1) \neq T_p(S_2) $, we have $ N_{p_1}(0,0) \wedge N_{p_2}(0,0) \neq 0 $, where 
    \[ 
    N_{p_1} = \frac{(f_x, f_y, f_z)(p)}{\lVert (f_x, f_y, f_z)(p) \rVert }, \quad N_{p_2} = \frac{(g_x, g_y, g_z)(p)}{\lVert (g_x, g_y, g_z)(p) \rVert } . 
    \] 
    Hence 
    \[ 
    \mathrm{d}F_{(x,y,z)} = 
    \begin{pmatrix}
        f_x & f_y & f_z \\
        g_x & g_y & g_z
    \end{pmatrix}(x,y,z) \neq 0,
    \]
    and $ \mathrm{d}F $ is surjective. Therefore, $ (0,0) $ is a regular point of $ F $, and by [Do Carmo] Problem 2.2.17 (b) (The inverse image of a regular value of a differentiable map \(F:U\subset\mathbb{R}^{3}\to\mathbb{R}^{2}\) is a regular curve in \(\mathbb{R}^{3}\)), $ S_1 \cap S_2 $ is a regular curve.
\end{solution}

% 2.4.18
\begin{exercise}
    Prove that if a regular surface $S$ meets a plane $P$ in a single point $p$, then this plane coincides with the tangent plane of $S$ at $p$.
\end{exercise}

\begin{solution}

\end{solution}

% 2.4.19
\begin{exercise}
    Let $S\subset\mathbb{R}^3$ be a regular surface and $P\subset\mathbb{R}^3$ be a plane. If all points of $S$ are on the same side of $P$, prove that $P$ is tangent to $S$ at all points of $P\cap S$.
\end{exercise}

\begin{solution}
\end{solution}

% 2.4.20
\begin{exercise}[\textbf{*}]
    Show that the perpendicular projections of the center $(0,0,0)$ of the ellipsoid
    \[
    \frac{x^2}{a^2}+\frac{y^2}{b^2}+\frac{z^2}{c^2}=1
    \]
    onto its tangent planes constitute a regular surface given by
    \[
    \{(x,y,z)\in\mathbb{R}^3;\, (x^2+y^2+z^2)^2 = a^2 x^2 + b^2 y^2 + c^2 z^2\} - \{(0,0,0)\}.
    \]
\end{exercise}

\begin{solution}
\end{solution}

% 2.4.21
\begin{exercise}[\textbf{*}]
    Let $f:S\to\mathbb{R}$ be a differentiable function on a connected regular surface $S$. Assume that $df_p=0$ for all $p\in S$. Prove that $f$ is constant on $S$.
\end{exercise}

\begin{solution}
    Let $ p, q \in S $ be two arbitrary points. Since $ S $ is connected, there exists a piecewise regular curve $ \alpha : [0, 1] \to S $ such that $ \alpha(0) = p $, $ \alpha(1) = q $. Then we have
    \[
        (f \circ \alpha)'(t) = df_{\alpha(t)}(\alpha'(t)) = 0, \quad t \in [0, 1].
    \]
    Thus $ f \circ \alpha $ is constant on $ [0, 1] $, and in particular, we have
    \[
        f(p) = f(\alpha(0)) = f(\alpha(1)) = f(q).
    \]
    Since $ p, q $ are arbitrary points in $ S $, we conclude that $ f $ is constant on $ S $.
\end{solution}

% 2.4.22
\begin{exercise}
    Prove that if all normal lines to a connected regular surface $S$ meet a fixed straight line, then $S$ is a piece of a surface of revolution.
\end{exercise}

\begin{solution}
    
\end{solution}

% 2.4.23
\begin{exercise}
    Prove that the map $F : S^{2} \to S^{2}$ defined in Exercise 16 of Sec. 2-3 has only a finite number of critical points (see Exercise 13).
\end{exercise}

\begin{solution}
    From Problem 2.3.16, $F : S^{2} \to S^{2}$ is differentiable. Let $ p \in S^2 $ be a critical point of $ F $, then $ \mathrm{d}F_p = 0 $. Since $ F = \pi_N^{-1} \circ P \circ \pi_N $, by the chain rule, we have
    \[
        \mathrm{d}F_p = \mathrm{d}(\pi_N^{-1})_{P(\pi_N(p))} \circ \mathrm{d}P_{\pi_N(p)} \circ \mathrm{d}(\pi_N)_p.
    \]
    Note that $ \mathrm{d}(\pi_N)_p $ and $ \mathrm{d}(\pi_N^{-1})_{P(\pi_N(p))} $ are isomorphisms, so $ \mathrm{d} F_p = 0 $ if and only if $ \mathrm{d} P_{\pi_N(p)} = 0 $. Since $ P : \mathbb{C} \to \mathbb{C} $ is a polynomial of degree $ n $, $ \mathrm{d}P $ is a polynomial of degree $ n-1 $, and thus has $ n-1 $ roots by the Fundamental Theorem of Algebra. Therefore, the map $ F : S^2 \to S^2 $ has only a finite number of critical points.
\end{solution}

% 2.4.24
\begin{exercise}[Chain Rule]
    Show that if $\varphi:S_1\to S_2$ and $\psi:S_2\to S_3$ are differentiable maps and $p\in S_1$, then
    \[
    d(\psi\circ\varphi)_p = d\psi_{\varphi(p)}\circ d\varphi_p.
    \]
\end{exercise}

\begin{solution}
    
\end{solution}

% 2.4.25 
\begin{exercise}
    Prove that if two regular curves $C_1$ and $C_2$ of a regular surface $S$ are tangent at a point $p\in S$, and if $\varphi:S\to S$ is a diffeomorphism, then $\varphi(C_1)$ and $\varphi(C_2)$ are regular curves which are tangent at $\varphi(p)$.
\end{exercise}

\begin{solution}
\end{solution}

% 2.4.26
\begin{exercise}
    Show that if $p$ is a point of a regular surface $S$, it is possible, by a convenient choice of the $(x,y,z)$ coordinates, to represent a neighborhood of $p$ in $S$ in the form $z=f(x,y)$ so that
    \[
    f(0,0)=0,\quad f_x(0,0)=0,\quad f_y(0,0)=0.
    \]
    (This is equivalent to taking the tangent plane to $S$ at $p$ as the $xy$ plane.)
\end{exercise}

\begin{solution}
    
\end{solution}

% 2.4.27
\begin{exercise}[Theory of Contact]
    Two regular surfaces, $S$ and $\bar{S}$, in $\mathbb{R}^3$, which have a point $p$ in common, are said to have \emph{contact of order $\ge 1$ at $p$} if there exist parametrizations with the same domain $\mathbf{x}(u,v)$, $\bar{\mathbf{x}}(u,v)$ at $p$ of $S$ and $\bar{S}$, respectively, such that $\mathbf{x}_u=\bar{\mathbf{x}}_u$ and $\mathbf{x}_v=\bar{\mathbf{x}}_v$ at $p$. If, moreover, some of the second partial derivatives are different at $p$, the contact is said to be of \emph{order exactly equal to 1}. Prove that
    \begin{enumerate}[label=\textbf{\alph*.}]
        \item The tangent plane $T_p(S)$ of a regular surface $S$ at the point $p$ has contact of order $\ge 1$ with the surface at $p$.
        \item If a plane has contact of order $\ge 1$ with a surface $S$ at $p$, then this plane coincides with the tangent plane to $S$ at $p$.
        \item Two regular surfaces have contact of order $\ge 1$ if and only if they have a common tangent plane at $p$, i.e., they are tangent at $p$.
        \item If two regular surfaces $S$ and $\bar{S}$ of $\mathbb{R}^3$ have contact of order $\ge 1$ at $p$ and if $F:\mathbb{R}^3\to\mathbb{R}^3$ is a diffeomorphism of $\mathbb{R}^3$, then the images $F(S)$ and $F(\bar{S})$ are regular surfaces which have contact of order $\ge 1$ at $F(p)$ (that is, the notion of contact of order $\ge 1$ is invariant under diffeomorphisms).
        \item If two surfaces have contact of order $\ge 1$ at $p$, then $\displaystyle \lim_{r\to 0}\frac{d}{r}=0$, where $d$ is the length of the segment which is determined by the intersections with the surfaces of some parallel to the common normal, at a distance $r$ from this normal.
    \end{enumerate}
\end{exercise}

\begin{solution}
    
\end{solution}

% 2.4.28
\begin{exercise}[Do Carmo 2.4.28]
    ~
    \begin{enumerate}[label=\textbf{\alph*.}]
        \item Define regular value for a differentiable function $f : S \to \mathbb{R}$ on a regular surface $S$.
        \item Show that the inverse image of a regular value of a differentiable function on a regular surface $S$ is a regular curve on $S$.
    \end{enumerate}
\end{exercise}

\begin{solution}
    ~
    \begin{enumerate}[label=\textbf{\alph*.}]
        \item A \emph{regular value} of a differentiable function $f : S \to \mathbb{R}$ defined on a regular surface $S$ is a value $c \in \mathbb{R}$ such that for every point $p \in f^{-1}(c)$, the differential $\mathrm{d}f_{p} : T_{p}(S) \to \mathbb{R} $ is surjective (i.e., $\mathrm{d}f_{p} \neq 0$).
        \item Let $c$ be a regular value of a differentiable function $f : S \to \mathbb{R}$ and let $ p \in f^{-1}(c) $. Pick a local parametrization $ \bvec{x}: U \subseteq \mathbb{R}^2 \to S $ such that $ \bvec{x}((0,0)) = p $. Define $ g : U \to \mathbb{R} $ by $ g = f \circ \bvec{x} $, then $ g(0,0) = f(\bvec{x}(0,0)) = f(p) = c $. Since $ \mathrm{d}f_{p} \neq 0 $ and $ \mathrm{d}\bvec{x}_{(0,0)} $ is surjective onto $ T_p S $, we have $ \mathrm{d}g_{(0,0)} \neq 0 $. By the Implicit Function Theorem, there exists a neighborhood $ V \subseteq U $ of $ (0,0) $ such that $ g^{-1}(c) \cap V $ is the graph of a $ C^1 $ function, say $ v = \phi (u) $. Then we can define a local parametrization of the curve $ f^{-1}(c) $ on $ S $ by
        \[
            \alpha(u) = \bvec{x}(u, \phi(u)), \quad u \in I
        \] 
        where $ I $ is some neighborhood of $ u = 0 $. Suppose for some $ u^\ast $, we have $ \alpha^{\prime} (u^\ast) = 0 $, then 
        \[ \mathrm{d}\bvec{x}_{(u^\ast, \phi(u^\ast))} \left(1, \phi^{\prime}(u^\ast) \right) = 0. \]
        Since $ \mathrm{d}\bvec{x} $ is one-to-one, we must have $ (1, \phi^{\prime}(u^\ast)) = 0 $, contradiction. Thus, $ \alpha^{\prime} (u) \neq 0 $ for all $ u \in I $, and in a neighborhood of each $ p \in f^{-1}(c) $, $ f^{-1}(c) $ is the image of a regular curve $ \alpha $ on $ S $. Patching the local parametrizations together, we conclude that $ f^{-1}(c) $ is a regular curve on $ S $. 
    \end{enumerate}
\end{solution}

\newpage 

\subsection{Chapter 2.5}

\begin{definition}[first fundamental form]
    \label{def:first_fundamental_form}
    The quadratic form $ I_p : T_p(S) \to \mathbb{R} $ defined by $ I_p (w) = \langle w, w \rangle_p  = \vert w \vert^{2} $ is called the \emph{first fundamental form} of the surface $ S $ at the point $ p \in S $.
\end{definition}

\begin{definition}[area]
    \label{def:area}
    Let $ R \subseteq S $ be a bounded region of a regular surface $ S $ contained in the coordinate neighborhood of a parametrization $ \bvec{x} : U \subseteq \mathbb{R}^2 \to S $. The positive number 
    \begin{equation}
        A(R) = \iint_{\bvec{x}^{-1}(R)} \mathrm{d}u \, \mathrm{d}v\, \vert \bvec{x}_u \wedge \bvec{x}_v \vert = \iint_{\bvec{x}^{-1}(R)} \mathrm{d}u \, \mathrm{d}v\, \sqrt{EG - F^2} 
    \end{equation}
    is called the \emph{area} of the region $ R $.
\end{definition}

% 2.5.1
\begin{exercise}
    Compute the first fundamental forms of the following parametrized surfaces where they are regular:
    \begin{enumerate}[label=\textbf{\alph*.}]
    \item $\mathbf{x}(u,v) = (a \sin u \cos v, b \sin u \sin v, c \cos u)$; ellipsoid.
    \item $\mathbf{x}(u,v) = (a u \cos v, b u \sin v, u^2)$; elliptic paraboloid.
    \item $\mathbf{x}(u,v) = (a u \cosh v, b u \sinh v, u^2)$; hyperbolic paraboloid.
    \item $\mathbf{x}(u,v) = (a \sinh u \cos v, b \sinh u \sin v, c \cosh u)$; hyperboloid of two sheets.
    \end{enumerate}
\end{exercise}

\begin{solution}
    
\end{solution}

% 2.5.2
\begin{exercise}
    Let $\mathbf{x}(\varphi, \theta) = (\sin \theta \cos \varphi, \sin \theta \sin \varphi, \cos \theta)$ be a parametrization of the unit sphere $S^2$. Let $P$ be the plane $x = z \cot \alpha$, $0 < \alpha < \pi$, and $\beta$ be the acute angle which the curve $P \cap S^2$ makes with the semimeridian $\varphi = \varphi_0$. Compute $\cos \beta$.
\end{exercise}

\begin{solution}
    
\end{solution}

% 2.5.3
\begin{exercise}
    Obtain the first fundamental form of the sphere in the parametrization given by stereographic projection (cf. Exercise 16, Sec. 2-2).
\end{exercise}

\begin{solution}
    Refer to Exercise 2.2.16, let the sphere be $ S^2 = \{ (x,y,z) \in \mathbb{R}^3 : x^2 + y^2 + (z-1)^2 = 1 \} $. The stereographic projection from the north pole $ N = (0,0,2) $ to the $ xy $-plane is given by
    \begin{equation*}
        \mathbf{x}(u,v) = \left( \frac{4u}{u^2 + v^2 + 4}, \frac{4v}{u^2 + v^2 + 4}, \frac{2(u^2 + v^2)}{u^2 + v^2 + 4} \right).
    \end{equation*}
    We have
    \begin{align*}
        \mathbf{x}_u &= \left( \frac{4(-u^2 + v^2 + 4)}{(u^2 + v^2 + 4)^2}, \frac{-8uv}{(u^2 + v^2 + 4)^2}, \frac{16u}{(u^2 + v^2 + 4)^2} \right), \\
        \mathbf{x}_v &= \left( \frac{-8uv}{(u^2 + v^2 + 4)^2}, \frac{4(u^2 - v^2 + 4)}{(u^2 + v^2 + 4)^2}, \frac{16v}{(u^2 + v^2 + 4)^2} \right).
    \end{align*}
    Thus we have
    \begin{align*}
        E &= \langle \mathbf{x}_u, \mathbf{x}_u \rangle = \frac{16 (-u^2 + v^2 + 4)^2 + 64u^2 v^2 + 256 u^2}{(u^2 + v^2 + 4)^4} = \frac{16}{(u^2 + v^2 + 4)^2}, \\
        F &= \langle \mathbf{x}_u, \mathbf{x}_v \rangle = \frac{-32uv(-u^2 + v^2 + 4) - 32uv (u^2 - v^2 + 4) + 256uv}{(u^2 + v^2 + 4)^4} = 0, \\
        G &= \langle \mathbf{x}_v, \mathbf{x}_v \rangle = \frac{64 u^2 v^2 + 16 (u^2 - v^2 + 4)^2 + 256 v^2}{(u^2 + v^2 + 4)^4} = \frac{16}{(u^2 + v^2 + 4)^2}.
    \end{align*}
    Therefore, the first fundamental form is
    \begin{equation*}
        I_p = \frac{16}{(u^2 + v^2 + 4)^2} \left((u^{\prime})^2 + (v^{\prime})^2\right).
    \end{equation*}
\end{solution}

% 2.5.4
\begin{exercise}
    Given the parametrized surface
    \[
    \mathbf{x}(u,v) = (u \cos v, u \sin v, \log \cos v + u), \quad -\frac{\pi}{2} < v < \frac{\pi}{2},
    \]
    show that the two curves $\mathbf{x}(u,v_1)$, $\mathbf{x}(u,v_2)$ determine segments of equal lengths on all curves $\mathbf{x}(u,\text{const.})$.
\end{exercise}

\begin{solution}
    
\end{solution}

% 2.5.5
\begin{exercise}
    Show that the area $A$ of a bounded region $R$ of the surface $z = f(x,y)$ is
    \[
    A = \iint_Q \sqrt{1 + f_x^2 + f_y^2} \, dx\,dy,
    \]
    where $Q$ is the normal projection of $R$ onto the $xy$ plane.
\end{exercise}

\begin{solution}
    
\end{solution}

% 2.5.6
\begin{exercise}
    Show that 
    \[
    \mathbf{x}(u,v) = (u \sin \alpha \cos v, u \sin \alpha \sin v, u \cos \alpha),
    \quad 0 < u < \infty, \quad 0 < v < 2\pi, \quad \alpha = \text{const.},
    \]
    is a parametrization of the cone with $2\alpha$ as the angle of the vertex. In the corresponding coordinate neighborhood, prove that the curve
    \[
    \mathbf{x}(c e^{v \sin \alpha \cot \beta}, v), \quad c = \text{const.}, \ \beta = \text{const.},
    \]
    intersects the generators of the cone ($v = \text{const.}$) under the constant angle $\beta$.
\end{exercise}

\begin{solution}
    
\end{solution}

% 2.5.7
\begin{exercise}
    The coordinate curves of a parametrization $\mathbf{x}(u,v)$ constitute a \emph{Tchebyshef net} if the lengths of the opposite sides of any quadrilateral formed by them are equal. Show that a necessary and sufficient condition for this is
    \[
    \frac{\partial E}{\partial v} = \frac{\partial G}{\partial u} = 0.
    \]
\end{exercise}

\begin{solution}
    The \emph{coordinate curves} of a parametrization $ \mathbf{x}(u,v) $ are the curves obtained by fixing one of the parameters and varying the other. Suppose we have a quadrilateral formed by the coordinate curves at points $ (u_1, v_1), (u_2, v_1), (u_2, v_2), (u_1, v_2) $. Let $ s (\bvec{x}(u_1, v_1), \bvec{x}(u_2, v_2)) \equiv s ((u_1, v_1), (u_2, v_2)) $ denote the arc length between two points. Then the lengths of the opposite sides are equal if and only if
    \begin{align*}
        s ( (u_1, v_1), (u_2, v_1) ) = s ( (u_1, v_2), (u_2, v_2) ) \;\Longrightarrow\; \int _{u_1}^{u_2} \mathrm{d}u \,\sqrt{E(u,v_1)} &= \int _{u_1}^{u_2} \mathrm{d}u \,\sqrt{E(u,v_2)}, \\
        s ( (u_1, v_1), (u_1, v_2) ) = s ( (u_2, v_1), (u_2, v_2) ) \;\Longrightarrow\; \int _{v_1}^{v_2} \mathrm{d}v \,\sqrt{G(u_1,v)} &= \int _{v_1}^{v_2} \mathrm{d}v \,\sqrt{G(u_2,v)}.
    \end{align*}
    Since $ u_1, u_2, v_1, v_2 $ are arbitrary, we have
    \begin{equation*}
        \sqrt{E(u,v_1)} = \sqrt{E(u,v_2)}, \quad \sqrt{G(u_1,v)} = \sqrt{G(u_2,v)}.
    \end{equation*}
    Therefore, $ E $ is independent of $ v $ and $ G $ is independent of $ u $, giving the desired result:
    \begin{equation*}
        \frac{\partial E}{\partial v} = 0, \quad \frac{\partial G}{\partial u} = 0.
    \end{equation*}
\end{solution}

% 2.5.8
\begin{exercise}[\textbf{*}]
    Prove that whenever the coordinate curves constitute a Tchebyshef net (see Exercise 7) it is possible to reparametrize the coordinate neighborhood in such a way that the new coefficients of the first fundamental form are
    \[
    E = 1, \quad F = \cos \theta, \quad G = 1,
    \]
    where $\theta$ is the angle of the coordinate curves.
\end{exercise}

\begin{solution}
    Following the procedure in Exercise 2.5.7, since the coordinate curves constitute a Tchebyshef net, we have $ \frac{\partial E}{\partial v} = 0 $ and $ \frac{\partial G}{\partial u} = 0 $. Thus, $ E = E(u) $ and $ G = G(v) $. We can define an arc length parametrization $ \bvec{\bar{x}} (\bar{u}, \bar{v}) $ by
    \begin{equation*}
        \bar{u} = \int \sqrt{E(u)} \, \mathrm{d}u, \quad \bar{v} = \int \sqrt{G(v)} \, \mathrm{d}v,
    \end{equation*}
    Then we have
    \begin{equation*}
        \bvec{\bar{x}}_{\bar{u}} = \frac{\partial \bvec{x}}{\partial u} \frac{\partial u}{\partial \bar{u}} = \frac{\bvec{x}_u}{\sqrt{E(u)}}, \quad \bvec{\bar{x}}_{\bar{v}} = \frac{\partial \bvec{x}}{\partial v} \frac{\partial v}{\partial \bar{v}} = \frac{\bvec{x}_v}{\sqrt{G(v)}}.
    \end{equation*}
    Thus, the coefficients of the first fundamental form in the new parametrization are
    \begin{align*}
        \bar{E} &= \langle \bvec{\bar{x}}_{\bar{u}}, \bvec{\bar{x}}_{\bar{u}} \rangle = \left\langle \frac{\bvec{x}_u}{\sqrt{E(u)}}, \frac{\bvec{x}_u}{\sqrt{E(u)}} \right\rangle = \frac{E(u)}{E(u)} = 1, \\
        \bar{F} &= \langle \bvec{\bar{x}}_{\bar{u}}, \bvec{\bar{x}}_{\bar{v}} \rangle = \left\langle \frac{\bvec{x}_u}{\sqrt{E(u)}}, \frac{\bvec{x}_v}{\sqrt{G(v)}} \right\rangle = \frac{F(u,v)}{\sqrt{E(u) G(v)}} = \cos \theta, \\
        \bar{G} &= \langle \bvec{\bar{x}}_{\bar{v}}, \bvec{\bar{x}}_{\bar{v}} \rangle = \left\langle \frac{\bvec{x}_v}{\sqrt{G(v)}}, \frac{\bvec{x}_v}{\sqrt{G(v)}} \right\rangle = \frac{G(v)}{G(v)} = 1.
    \end{align*}
\end{solution}

% 2.5.9
\begin{exercise}[\textbf{*}]
    Show that a surface of revolution can always be parametrized so that
    \[
    E = E(v), \quad F = 0, \quad G = 1.
    \]
\end{exercise}

\begin{solution}
    \begin{definition}
        The coordinate curves of a parametrization are orthogonal if and only if $F(u, v) = 0$ for all $(u, v)$. Such a parametrization is called an \emph{orthogonal parametrization}.
    \end{definition}
    Without loss of generality, let the axis of revolution be the $z$-axis. A surface of revolution can be parametrized as
    \begin{equation*}
        \bvec{x}(u,v) = (f(v) \cos u, f(v) \sin u, g(v)), \quad u \in (0, 2\pi), \ v \in I,
    \end{equation*}
    where $ f(v) > 0 $ for all $ v \in I $. Then 
    \begin{align*}
        \bvec{x}_u &= (-f(v) \sin u, f(v) \cos u, 0), \\
        \bvec{x}_v &= (f'(v) \cos u, f'(v) \sin u, g^{\prime} (v)).
    \end{align*}
    The arc length parametrization of the $ v $-curves is given by 
    \[
       \bar{v} = \int \sqrt{[f^{\prime}(v)]^2 + [g^{\prime}(v)]^2} \, \mathrm{d}v \;\Longrightarrow\; \bvec{\bar{x}}(u, \bar{v}) = \left(f(\overline{v})\cos u, f(\overline{v})\sin u, g(\overline{v})\right). 
    \]
    Then, abbreviating $ v(\bar{v}) $ to $ \overline{v} $, we have
    \begin{align*}
        \bvec{\bar{x}}_u &= (-f(\bar{v}) \sin u, f(\bar{v}) \cos u, 0), \\
        \bvec{\bar{x}}_{\bar{v}} &= \left( f^{\prime}(\bar{v}) \cos u, f^{\prime}(\bar{v}) \sin u, g^{\prime}(\bar{v}) \right) \frac{\mathrm{d} v}{\mathrm{d} \bar{v}} = \frac{\left( f^{\prime}(\overline{v}) \cos u, f^{\prime}(\overline{v}) \sin u, g^{\prime}(\overline{v})\right)}{\sqrt{[f^{\prime}(\overline{v} )]^2 + [g^{\prime}(\overline{v})]^2}}.
    \end{align*}
    Then, the coefficients of the first fundamental form in the new parametrization are
    \begin{equation*}
        \bar{E} = \langle \bvec{\bar{x}}_u, \bvec{\bar{x}}_u \rangle = f^2 (\overline{v}), \quad \bar{F} = \langle \bvec{\bar{x}}_u, \bvec{\bar{x}}_{\bar{v}} \rangle = 0, \quad \bar{G} = \langle \bvec{\bar{x}}_{\bar{v}}, \bvec{\bar{x}}_{\bar{v}} \rangle = 1.
    \end{equation*}
    Note that if we force an arc length parametrization on the $ u $ curves instead, we would have $ \bar{E} = 1 $, $ \bar{F} = 0 $, and $ \bar{G} = G(u) $. However, it is not possible to do both and still have an orthogonal parametrization.
\end{solution}

% 2.5.10
\begin{exercise}
    Let $P = \{(x,y,z) \in \mathbb{R}^3; z = 0\}$ be the $xy$ plane and let $\mathbf{x}: U \to P$ be a parametrization of $P$ given by
    \[
    \mathbf{x}(\rho,\theta) = (\rho \cos \theta, \rho \sin \theta),
    \]
    where
    \[
    U = \{(\rho,\theta) \in \mathbb{R}^2; \rho > 0, 0 < \theta < 2\pi\}.
    \]
    Compute the coefficients of the first fundamental form of $P$ in this parametrization.
\end{exercise}

\begin{solution}
    We have
    \begin{align}
        \bvec{x}_{\rho} &= (\cos \theta, \sin \theta, 0), \\
        \bvec{x}_{\theta} &= (-\rho \sin \theta, \rho \cos \theta, 0).
    \end{align}
    Thus the coefficients of the first fundamental form are
    \begin{align*}
        E = \langle \bvec{x}_{\rho}, \bvec{x}_{\rho} \rangle &= \cos^2 \theta + \sin^2 \theta = 1, \\
        F = \langle \bvec{x}_{\rho}, \bvec{x}_{\theta} \rangle &= -\rho \cos \theta \sin \theta + \rho \sin \theta \cos \theta = 0, \\
        G = \langle \bvec{x}_{\theta}, \bvec{x}_{\theta} \rangle &= \rho^2 \sin^2 \theta + \rho^2 \cos^2 \theta = \rho^2.
    \end{align*}
\end{solution}

% 2.5.11
\begin{exercise}
    Let $S$ be a surface of revolution and $C$ its generating curve (cf. Example 4, Sec. 2-3). Let $s$ be the arc length of $C$ and denote by $\rho = \rho(s)$ the distance to the rotation axis of the point of $C$ corresponding to $s$.
    \begin{enumerate}[label=\alph*.]
    \item (\textit{Pappus' Theorem.}) Show that the area of $S$ is
    \[
    2\pi \int_0^l \rho(s)\,ds,
    \]
    where $l$ is the length of $C$.
    \item Apply part (a) to compute the area of a torus of revolution.
    \end{enumerate}
\end{exercise}

\begin{solution}

\end{solution}

% 2.5.12
\begin{exercise}
    Show that the area of a regular tube of radius $r$ around a curve $\alpha$ (cf. Exercise 10, Sec. 2-4) is $2\pi r$ times the length of $\alpha$.
\end{exercise}

\begin{solution}
    From Exercise 2.4.10, the parametrization of a tubular surface of radius $ r $ around a curve $ \alpha $ is given by 
    \[
        \bvec{x}(s, \theta) = \alpha (s) + r \left(n(s) \cos \theta + b(s) \sin \theta\right), \quad r \neq 0, s \in I .
    \]
    Then 
    \begin{align*}
        \bvec{x}_s &= \alpha^{\prime} (s) + r \left(n^{\prime}(s) \cos \theta + b^{\prime}(s) \sin \theta\right) \\
        &= t(s) + r \left(- k (s) t(s) \cos \theta - \tau (s) b(s) \cos \theta + \tau (s) n(s) \sin \theta\right) \\
        &= (1 - r k \cos \theta ) t + r \tau \sin \theta n - r \tau \cos \theta b, \\
        \bvec{x}_{\theta} &= r \left(-n(s) \sin \theta + b(s) \cos \theta\right),
    \end{align*}
    and 
    \begin{align*}
        \bvec{x}_s \wedge \bvec{x}_\theta &= \left( (1 - r k \cos \theta ) t + r \tau \sin \theta n - r \tau \cos \theta b \right) \wedge r \left(-n(s) \sin \theta + b(s) \cos \theta\right) \\
        &= - r (1- rk (s) \cos \theta) ( \cos \theta \, n (s) + \sin \theta \, b (s)). 
    \end{align*}
    By definition~\ref{def:area}, the area of the tube is given by
    \[
        A = \iint \mathrm{d}\theta \, \mathrm{d}s \, \vert \bvec{x}_s \wedge \bvec{x}_{\theta} \vert = \int_0^{2\pi} \mathrm{d}\theta \, \int_{s_1}^{s_2} \mathrm{d}s\, r \vert 1 - r k(s) \cos \theta \vert.
    \]
    Since $ k(s) r \leq 1 $ for all $ s \in I $, we have 
    \[
        A = \int^{2\pi}_0 \mathrm{d}\theta \, \int_{s_1}^{s_2} \mathrm{d}s\, r ( 1 - r k(s) \cos \theta ) = 2 \pi r \int_{s_1}^{s_2} \mathrm{d}s = 2 \pi r \, \ell (\alpha).
    \]
    \begin{remark}
        Let's formalize the problem in the following way: Let $\alpha : [0, \ell] \to \mathbb{R}^3$ be a curve parametrized by arc length with $ k(s) \neq 0 $. Suppose $\alpha$ has no self-intersections, $\alpha(0) = \alpha(\ell)$ and it induces a smooth map from $S^1$ to $\mathbb{R}^3$ (i.e. $\alpha$ is a \emph{smooth simple closed curve}). Let $r > 0$ and $\varphi : [0, \ell] \times [0, 2\pi] \to \mathbb{R}^3$ is given by:
        \[
        \varphi(s, v) = \alpha(s) + r (n(s)) \cos v + b(s) \sin v
        \]
        Then image $ T = \operatorname{Im} \varphi $ is called the tube of radius $r$ around $\alpha$. For $r$ sufficiently small, $T$ is a surface. Prove that $ A(T) = 2\pi r \ell $.
    \end{remark}
\end{solution}

% 2.5.13
\begin{exercise}
    (\textit{Generalized Helicoids.}) A natural generalization of both surfaces of revolution and helicoids is obtained as follows. Let a regular plane curve $C$, which does not meet an axis $E$ in the plane, be displaced in a rigid screw motion about $E$, that is, so that each point of $C$ describes a helix (or circle) with $E$ as axis. The set $S$ generated by the displacement of $C$ is called a \textit{generalized helicoid} with axis $E$ and generator $C$. If the screw motion is a pure rotation about $E$, $S$ is a surface of revolution; if $C$ is a straight line perpendicular to $E$, $S$ is (a piece of) the standard helicoid (cf. Example 3).  

    Choose the coordinate axes so that $E$ is the $z$ axis and $C$ lies in the $yz$ plane. Prove that
    \begin{enumerate}[label=\alph*.]
        \item If $(f(s), g(s))$ is a parametrization of $C$ by arc length $s$, $a < s < b$, $f(s) > 0$, then $\mathbf{x}: U \to S$, where
        \[
        U = \{ (s,u) \in \mathbb{R}^2; a < s < b, 0 < u < 2\pi \}
        \]
        and
        \[
        \mathbf{x}(s,u) = (f(s) \cos u, f(s) \sin u, g(s) + c u), \quad c = \text{const.},
        \]
        is a parametrization of $S$. Conclude that $S$ is a regular surface.
        \item The coordinate lines of the above parametrization are orthogonal (i.e., $F = 0$) if and only if $\mathbf{x}(U)$ is either a surface of revolution or (a piece of) the standard helicoid.
    \end{enumerate}
\end{exercise}

\begin{solution}
    
\end{solution}

% 2.5.14
\begin{exercise}[\textit{Gradient on Surfaces.}]
    The gradient of a differentiable function $f: S \to \mathbb{R}$ is a differentiable map $\operatorname{grad} f: S \to \mathbb{R}^3$ which assigns to each point $p \in S$ a vector $\operatorname{grad} f(p) \in T_p(S) \subset \mathbb{R}^3$ such that
    \[
    (\operatorname{grad} f(p), v)_p = d f_p(v) \quad \text{for all } v \in T_p(S).
    \]
    Show that
    \begin{enumerate}[label=\textbf{\alph*.}]
        \item If $E, F, G$ are the coefficients of the first fundamental form in a parametrization $\mathbf{x}: U \subset \mathbb{R}^2 \to S$, then $\operatorname{grad} f$ on $\mathbf{x}(U)$ is given by
        \[
        \operatorname{grad} f = \frac{f_u G - f_v F}{E G - F^2} \mathbf{x}_u + \frac{f_v E - f_u F}{E G - F^2} \mathbf{x}_v.
        \]
        In particular, if $S = \mathbb{R}^2$ with coordinates $x, y$,
        \[
        \operatorname{grad} f = f_x e_1 + f_y e_2,
        \]
        where $\{ e_1, e_2 \}$ is the canonical basis of $\mathbb{R}^2$ (thus, the definition agrees with the usual definition of gradient in the plane).
        \item If you let $p \in S$ be fixed and $v$ vary in the unit circle $|v| = 1$ in $T_p(S)$, then $d f_p(v)$ is maximum if and only if $v = \operatorname{grad} f / |\operatorname{grad} f|$ (thus, $\operatorname{grad} f(p)$ gives the direction of maximum variation of $f$ at $p$).
        \item If $\operatorname{grad} f \neq 0$ at all points of the level curve $C = \{ q \in S; f(q) = \text{const.} \}$, then $C$ is a regular curve on $S$ and $\operatorname{grad} f$ is normal to $C$ at all points of $C$.
    \end{enumerate}
\end{exercise}

\begin{solution}
    ~
    \begin{enumerate}[label=\textbf{\alph*.}]
        \item 
    \end{enumerate}
\end{solution}

% 2.5.15
\begin{exercise}[\textit{Orthogonal Families of Curves.}]
    ~  
    \begin{enumerate}[label=\textbf{\alph*.}]
        \item Let $E, F, G$ be the coefficients of the first fundamental form of a regular surface $S$ in the parametrization $\mathbf{x}: U \subset \mathbb{R}^2 \to S$. Let $\varphi(u,v) = \text{const.}$ and $\psi(u,v) = \text{const.}$ be two families of regular curves on $\mathbf{x}(U) \subset S$ (cf. Exercise 28, Sec. 2-4). Prove that these two families are orthogonal (i.e., whenever two curves of distinct families meet, their tangent lines are orthogonal) if and only if
        \[
        E \varphi_u \psi_v - F(\varphi_u \psi_u + \varphi_v \psi_v) + G \varphi_v \psi_u = 0.
        \]
        \item Apply part (a) to show that on the coordinate neighborhood $\mathbf{x}(U)$ of the helicoid of Example 3, the two families of regular curves
        \[
        v \cos u = \text{const.}, \quad v \neq 0,
        \]
        \[
        (v^2 + a^2) \sin^2 u = \text{const.}, \quad v \neq 0, \quad u \neq \pi,
        \]
        are orthogonal.
    \end{enumerate}
\end{exercise}

\begin{solution}
    ~
    \begin{enumerate}[label=\textbf{\alph*.}]
        \item Suppose $ \varphi(u,v) = c_1 $ and $ \psi(u,v) = c_2 $ are two families of regular curves on $ \mathbf{x}(U) \subset S $. The curves satisfy
        \begin{equation*}
            \phi_u u^{\prime} + \phi_v v^{\prime} = 0, \quad \psi_u u^{\prime} + \psi_v v^{\prime} = 0.
        \end{equation*}
        So we can choose the tangent vectors of the two families of curves to be
        \begin{equation*}
            t(\varphi) = - \varphi_v \bvec{x}_u + \varphi_u \bvec{x}_v, \quad t(\psi) = - \psi_v \bvec{x}_u + \psi_u \bvec{x}_v.
        \end{equation*}
        The two families are orthogonal if and only if $ \langle t(\varphi), t(\psi) \rangle = 0 $, which is equivalent to
        \begin{align*}
            \langle t(\varphi), t(\psi) \rangle &= \langle - \varphi_v \bvec{x}_u + \varphi_u \bvec{x}_v, - \psi_v \bvec{x}_u + \psi_u \bvec{x}_v \rangle \\
            &= \varphi_v \psi_v \langle \bvec{x}_u, \bvec{x}_u \rangle - \varphi_v \psi_u \langle \bvec{x}_u, \bvec{x}_v \rangle - \varphi_u \psi_v \langle \bvec{x}_v, \bvec{x}_u \rangle + \varphi_u \psi_u \langle \bvec{x}_v, \bvec{x}_v \rangle \\
            &= E \varphi_v \psi_v - F (\varphi_u \psi_v + \varphi_v \psi_u) + G \varphi_u \psi_u = 0.
        \end{align*}
        \item From Example 3, the helicoid is given by the parametrization $ \bvec{x}(u,v) = (v \cos u, v \sin u, a u) $, with the coefficients of the first fundamental form being
        \begin{equation*}
            E(u,v) = v^2 + a^2, \quad F(u,v) = 0, \quad G(u,v) = 1.
        \end{equation*}
        Let $ \phi (u,v) = v \cos u $, $ \psi (u,v) = (v^2 + a^2) \sin^2 u $. Then
        \begin{align*}
            \phi_u &= -v \sin u, & \phi_v &= \cos u, \\
            \psi_u &= 2 (v^2 + a^2) \sin u \cos u, & \psi_v &= 2v \sin^2 u.
        \end{align*}
        Substituting these into equation (\ref{eq:orthogonal_families}) in part (a), we have
        \begin{align*}
            &(v^2 + a^2) \cos u (2v \sin^2 u) - 0 + 1 (-v \sin u)(2 (v^2 + a^2) \sin u \cos u) = 0.
        \end{align*}
        Therefore, the two families of regular curves are orthogonal.
    \end{enumerate}
\end{solution}

\end{CJK}
\end{document}