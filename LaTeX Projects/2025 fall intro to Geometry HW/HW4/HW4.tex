\documentclass[a4paper]{article}
%% Formatting %%
\usepackage[margin=3cm]{geometry}
\usepackage{type1cm, titlesec, fancyhdr, titling}
\usepackage{multicol}
\usepackage[dvipsnames]{xcolor}
\usepackage{ulem}
\usepackage{parskip}
\setlength{\parindent}{2em}
\setlength{\headheight}{15pt}
\setlength{\droptitle}{-1.5cm}
\parindent=24pt
%% Math and Symbols %%
\usepackage{amsmath,amsthm,amssymb, mathtools}
\usepackage{yhmath, faktor, dsfont}
\usepackage{academicons, wasysym, marvosym}
\usepackage[scr]{rsfso} 
\usepackage{latexsym, amsmath, amscd, amsmath, amsthm}
\usepackage{amssymb,amsmath,amsthm,graphicx,dsfont}
\usepackage{hyperref}

%% Enhancement %%
\usepackage{graphicx, tabularx}
\usepackage[shortlabels,inline]{enumitem}
%% TikZ %%
\usepackage{tikz-cd}
\usepackage[breakable]{tcolorbox}
\usetikzlibrary{decorations.pathmorphing}
\usetikzlibrary{calc, arrows,matrix}

%% Other packages %%
\usepackage{amsopn}

%% Traditional Chinese %%
\usepackage{CJKutf8}

%% Math environments %%
\newtheoremstyle{mystyle}
  {6pt}{15pt}% 上下間距
  {}%          內文字體
  {}%              縮排
  {\bf}%       標頭字體
  {.}%       標頭後標點
  {1em}% 內文與標頭距離
  {}% Theorem head spec (can be left empty, meaning 'normal')
\theoremstyle{mystyle}	
\newtheorem{theorem}{Theorem}
\newtheorem{definition}{Definition}
\newtheorem{example}[theorem]{Example}
\newtheorem{exercise}{Exercise}
\newtheorem{solution}{Solution}
\newtheorem{corollary}[theorem]{Corollary}
\newtheorem{property}[theorem]{Property}
\newtheorem{proposition}[theorem]{Proposition}
\newtheorem{lemma}{Lemma}
\newtheorem{problem}[theorem]{Problem}
\newtheorem{answer}{Answer}[section]
\newtheorem{fact}[theorem]{fact}
\newtheorem*{remark}{Remark}
\newtheorem*{claim}{Claim}
\newtheorem*{observation}{Observation}

\newcommand{\bvec}[1]{\mathbf{#1}} % vector

\begin{document}
\begin{CJK}{UTF8}{bkai}

\title{%
  \textbf{2025 Fall Introduction to Geometry} \\
  \vspace{0.5cm}
  \large 
  Homework 4 (Due Oct 3, 2025)\\
}
\author{物理、數學三 黃紹凱 B12202004}
\date{\today}

\maketitle

\begin{definition}[Regular surface] \label{def:regular_surface}
    A subset $ S \subseteq \mathbb{R}^3 $ is a \emph{regular surface} if, for each $ p \in S $, there exists a neighborhood $ V \subseteq \mathbb{R}^3 $ and a map $ \bvec{x}: U \to V \cap S $ of an open set $ U \subseteq \mathbb{R}^2 $ onto $ V \cap S \subseteq \mathbb{R}^3 $ such that 
    \begin{enumerate}[(i)]
        \item $ \bvec{x} $ is (infinitely) differentiable. 
        \item $ \bvec{x} $ is a homeomorphism, i.e. $ \bvec{x} $ is a bijection, and both $ \bvec{x} $ and $ \bvec{x}^{-1} $ are continuous. 
        \item For each $ q \in U $, the differential $ \mathrm{d}\bvec{x}_q : \mathbb{R}^2 \to \mathbb{R}^3 $ is one-to-one (the \emph{regularity condition}).
    \end{enumerate}

    \begin{figure}[h]
        \centering
        \includegraphics[width=0.8\textwidth]{2-1.png}
        % \caption{Figure 2-1}
    \end{figure}
\end{definition}

\begin{definition}[Differentiability on regular surfaces] \label{def:differentiability}
    Let $f : V \subset S \to \mathbb{R}$ be a function defined in an open subset $V$ of a regular surface $S$. Then $f$ is said to be \emph{differentiable} at $p \in V$ if, for some parametrization $\mathbf{x} : U \subset \mathbb{R}^2 \to S$ with $p \in \mathbf{x}(U) \subset V$, the composition $f \circ \mathbf{x} : U \subset \mathbb{R}^2 \to \mathbb{R}$ is differentiable at $\mathbf{x}^{-1}(p)$. $f$ is differentiable in $V$ if it is differentiable at all points of $V$.
\end{definition}


% Problem 1
\begin{problem}[Do Carmo 2.2.11]
    Show that the set
    \[
        S=\{(x,y,z)\in\mathbb{R}^3\; ;\; z=x^2-y^2\}
    \]
    is a regular surface and check that parts (a) and (b) are parametrizations for $S$:
    \begin{enumerate}[(a)]
        \item $\mathbf{x}(u,v)=(u+v,\;u-v,\;4uv), \quad (u,v)\in\mathbb{R}^2.$
        \item $\mathbf{x}(u,v)=(u\cosh v,\;u\sinh v,\;u^2), \quad (u,v)\in\mathbb{R}^2,\; u\neq 0.$
    \end{enumerate}
    Which parts of $S$ do these parametrizations cover?
\end{problem}
\begin{solution}
    ~

    Notice that $ z(x,y) = x^2 - y^2 $ is a differentiable function from the open set $ U = \mathbb{R}^2 $ to $ \mathbb{R} $, so by Proposition 2.2.1 in Do Carmo, $ S $ is a regular surface. Recall that a map $ \bvec{x}: U \to V \cap S $ if $ \bvec{x} $ is differentiable, a homeomorphism, and $ \mathrm{d}\bvec{x}_p $ is one-to-one for all $ p \in U $.
    \begin{enumerate}[(a)]
        \item The map $ \bvec{x} $ is a polyminial in $ u $ and $ v $, so it is differentiable.  By explicit calculation,
        \begin{equation*}
            \mathrm{d}\bvec{x}_q = 
            \begin{pmatrix}
                1 & 1 \\
                1 & -1 \\
                4v & 4u
            \end{pmatrix}
        \end{equation*}
        in the canonical basis, so $ \left\vert \partial (x,y) / \partial (u,v) \right\vert = 2 $ and $ \mathrm{d}\bvec{x} $ is one-to-one. To show that $ \bvec{x} $ is a homeomorphism, observe that for any $ (x,y,z) \in S $, we have $ z = x^2 - y^2 $, so $ z = (u+v)^2 - (u-v)^2 = 4uv $, and
        \begin{equation*}
            u = \frac{x+y}{2}, \quad v = \frac{x-y}{2}
        \end{equation*}
        from the remaining equations. This determines a unique $ (u,v) $ for each $ (x,y,z) \in S $, and we can conclude that the inverse map $ \bvec{x}^{-1} $ exists and is continuous. 
        \item The map $ \bvec{x} $ is a composition of polynomials and exponential functions, so it is differentiable. By explicit calculation,
        \begin{equation*}
            \mathrm{d}\bvec{x}_q = 
            \begin{pmatrix}
                \cosh v & u\sinh v \\
                \sinh v & u\cosh v \\
                2u & 0
            \end{pmatrix}
        \end{equation*}
        in the canonical basis, so $ \left\vert \partial (x,y) / \partial (u,v) \right\vert = u $, and $ \mathrm{d}\bvec{x} $ is one-to-one for $ u \neq 0 $. To show that $ \bvec{x} $ is a homeomorphism, observe that for any $ (x,y,z) \in S $ with $ x^2 - y^2 > 0 $, we have $ z = x^2 - y^2 $, so $ z = u^2(\cosh^2 v - \sinh^2 v) = u^2 $, and 
        \begin{equation*}
            u = \pm \sqrt{x^{2} - y^{2}}, \quad v = \tanh^{-1} \frac{y}{x}
        \end{equation*}
        from the remaining equations. This determines a unique $ (u,v) $ for each $ (x,y,z) \in S $ with $ x^2 - y^2 > 0 $, and we can conclude that the inverse map $ \bvec{x}^{-1} $ exists and is continuous.
    \end{enumerate}
    Parametrization (a) covers the whole surface $ S $, while parametrization (b) only covers the parts of $ S $ where $ \vert x \vert > \vert y \vert $.
    \begin{remark}
        The graph of $ z = f(x,y) = x^2 - y^2 $ is a hyperbolic paraboloid, also known as saddle, shown in figure \ref{fig:saddle-1}, \ref{fig:saddle-2}.
    \end{remark}
        \begin{figure}[h]
        \centering
        \begin{minipage}{0.45\textwidth}
            \centering
            \includegraphics[width=\textwidth]{xz.png}
            \label{fig:saddle-1}
            \caption{XZ plane projection}
        \end{minipage}
        \hfill
        \begin{minipage}{0.45\textwidth}
            \centering
            \includegraphics[width=\textwidth]{yz.png}
            \label{fig:saddle-2}
            \caption{YZ plane projection}
        \end{minipage}
    \end{figure}
\end{solution}

% Problem 2
\begin{problem}[Do Carmo 2.2.16]
    One way to define a system of coordinates for the sphere $S^2$, given by
    \[
        x^2+y^2+(z-1)^2=1,
    \]
    is to consider the so-called stereographic projection
    \[
        \pi:S^2\setminus\{N\}\longrightarrow\mathbb{R}^2
    \]
    which carries a point $p=(x,y,z)$ of the sphere $S^2$ minus the north pole $N=(0,0,2)$ onto the intersection of the $xy$-plane with the straight line which connects $N$ to $p$ (Fig.\ 2--12). Let $(u,v)=\pi(x,y,z)$, where $(x,y,z)\in S^2\setminus\{N\}$ and $(u,v)$ lies in the $xy$-plane.
    \begin{enumerate}[label=\textbf{\alph*.}]
        \item Show that $\pi^{-1}:\mathbb{R}^2\to S^2$ is given by
        \[
            x=\frac{4u}{u^2+v^2+4},\qquad
            y=\frac{4v}{u^2+v^2+4},\qquad
            z=\frac{2(u^2+v^2)}{u^2+v^2+4}.
        \]
        \item Show that it is possible, using stereographic projection, to cover the sphere with two coordinate neighborhoods.
    \end{enumerate}
    \begin{figure}[h]
        \centering
        \includegraphics[width=0.6\textwidth]{2-12.png}
        % \caption{Figure 2-12}
    \end{figure}
\end{problem}

\begin{solution}
    ~
    \begin{enumerate}[label=\textbf{\alph*.}]
        \item Let's construct the map $ \pi : S^2 \to \mathbb{R}^2 $ explicitly. For a point $ p = (x,y,z) \in S^2 \setminus \{N\} $, the line connecting $ N $ and $ p $ can be parametrized as 
        \begin{equation}
            \label{eq:line}
            L(t) = N + t(p - N) = (0,0,2) + t(x,y,z-2) = (tx, ty, 2 + t(z-2))
        \end{equation}
        The intersection of this line with the $ xy $-plane occurs when $ z = 0 $, so $ t = 2 / (2-z) $. Substituting this back to equation (\ref{eq:line}) gives 
        \begin{equation*}
            \pi(p) = (u,v) = \left( \frac{2x}{2-z}, \frac{2y}{2-z} \right).
        \end{equation*}
        Solving for $ (x,y) $ gives
        \begin{equation*}
            (x,y) = \left( \frac{u(2-z)}{2}, \frac{v(2-z)}{2} \right).
        \end{equation*}
        From the equation for the sphere, we have 
        \begin{equation*}
            \left(\frac{u(2-z)}{2}\right)^2 + \left(\frac{v(2-z)}{2}\right)^2 + (z-1)^2 = 1 \; \Longrightarrow \; z = \frac{2(u^2 + v^2)}{u^2 + v^2 + 4},
        \end{equation*}
        hence
        \begin{equation*}
            x = \frac{4u}{u^2 + v^2 + 4}, \quad y = \frac{4v}{u^2 + v^2 + 4}, \quad z = \frac{2(u^2 + v^2)}{u^2 + v^2 + 4}.
        \end{equation*}
        \item Using the inverse stereographic projection $ \pi^{-1} $, we can cover the whole sphere except the north pole $ N $. To cover the north pole, use another stereographic projection from the south pole $ S = (0,0,0) $ to the $ xy $-plane, with the inverse map given by
        \begin{equation*}
            x = \frac{4u}{u^2 + v^2 + 4}, \quad y = \frac{4v}{u^2 + v^2 + 4}, \quad z = \frac{8}{u^2 + v^2 + 4}.
        \end{equation*}
    \end{enumerate}
\end{solution}

% Problem 3
\begin{problem}[Do Carmo 2.2.19*]
    ~

    Let $\alpha:(-3,0)\to\mathbb{R}^2$ be defined by (Fig.\ 2--13)
    \[
        \alpha(t)=
        \begin{cases}
        (0,\;-(t+2)), & t\in(-3,-1),\\[2pt]
        \text{a regular parametrized curve joining } p=(0,-1)\text{ to } q=\left(\dfrac{1}{\pi},0\right), & t\in(-1,-\tfrac{1}{\pi}),\\[8pt]
        (-t,\; \sin \tfrac{1}{t}), & t\in\!\left(-\tfrac{1}{\pi},\,0\right).
        \end{cases}
    \]

    \begin{figure}[h]
        \centering
        \includegraphics[width=0.6\textwidth]{2-13.png}
        % \caption{Figure 2-13}
    \end{figure}

    It is possible to define the curve joining $p$ to $q$ so that all the derivatives of $\alpha$ are continuous at the corresponding points and $\alpha$ has no self-intersections. Let $C$ be the trace of $\alpha$.
    \begin{enumerate}[label=\textbf{\alph*.}]
        \item Is $C$ a regular curve?
        \item Let a normal line to the plane $\mathbb{R}^2$ run through $C$ so that it describes a ``cylinder'' $S$. Is $S$ a regular surface?
    \end{enumerate}
\end{problem}

\begin{solution}
    ~
    \begin{enumerate}[label=\textbf{\alph*.}]
        \item Let $ C $ be the trace of $ \alpha $, $ \alpha $ is said to be regular if at every point $ p \in C $, $ C $ is the graph of a $ C^1 $ function $ y = f(x) $ or $ x = g(y) $ in a neighborhood of $ p $. Notice that the origin $ (0,0) $ belongs to the trace of $ \alpha $ since $ \alpha (-2) = (0,0) $. Consider the sequence $ t_n = -\frac{1}{2n \pi} $, which satisfies $ t_n \in (-\frac{1}{\pi},0) $ for all $ n \in \mathbb{N} $. Therefore, in any neighborhood of $ (0,0) $, we can find some $ n \in \mathbb{N} $ such that $ \alpha (t_n) \in U $, so $ C $ cannot be the graph of $ x = f(y) $ locally. Similarly, $ C $ cannot be the graph of $ y = g(x) $ on the line segment $ \{0\}\times (-1,1) \subseteq \mathbb{R}^{2} $. Hence, $ C $ is not a regular curve.
        \item If the surface $ S $ were regular, then by Do Carmo Proposition 2.2.3, there exists a neighborhood $ V $ of any $ p \in S $ such that $ V $ is the graph of a differentiable function $ z = f(x,y) $ or $ x = g(y,z) $ or $ y = h(x,z) $. However, consider a point $ p \in (-\frac{1}{\pi}, 0, z) $ on the side boundary of $ S $. In (a) we concluded that locally around $ (0,0,z) $, the curve (translated by some $ z $ along the $ z $ axis) is not the graph of a $ C^1 $ function $ x = g(y,z) $ or $ y = h(x,z) $, while $ z $ cannot be a function of $ x $, $ y $. Therefore, $ S $ is not a regular surface.
    \end{enumerate}
\end{solution}

% Problem 4
\begin{problem}[Do Carmo 2.3.5*]
    Let $S\subset\mathbb{R}^3$ be a regular surface, and let $d:S\to\mathbb{R}$ be given by
    \[
        d(p)=\lVert p-p_0\rVert,
    \]
    where $p\in S$, $p_0\in\mathbb{R}^3$, and $p_0\notin S$; that is, $d$ is the distance from $p$ to a fixed point $p_0$ not in $S$. Prove that $d$ is differentiable.
\end{problem}

\begin{solution}
    By definition \ref{def:differentiability}, it suffices to show that for any parametrization $ \bvec{x}: U \subset \mathbb{R}^2 \to S $, the composition $ d \circ \bvec{x} : U \to \mathbb{R} $ is differentiable. Since $ S $ is a regular surface, for any point $ p \in S $, there exists a neighborhood $ V \subseteq \mathbb{R}^3 $ of $ p $ such that $ V \cap S $ is the graph of a differentiable function $ z(x,y) $ or $ x(y,z) $ or $ y(x,z) $. Assume that $ V \cap S $ is the graph of a differentiable function $ z(x,y) $, then define a parametrization
    \begin{equation*}
        \bvec{x}(u,v) = (u,v,z(u,v)), \quad (u,v) \in U \subseteq \mathbb{R}^2,
    \end{equation*}
    where $ U $ is open in $ \mathbb{R}^2 $. The composition $ d \circ \bvec{x} : U \to \mathbb{R} $ is given by
    \begin{equation*}
        \begin{split}
            (d \circ \bvec{x})(u,v) &= d(\bvec{x}(u,v)) = \sqrt{ \langle \bvec{x}(u,v) - p_0,\, \bvec{x}(u,v) - p_0 \rangle } \\
            &= \sqrt{ (u - x_0)^2 + (v - y_0)^2 + (z(u,v) - z_0)^2 }. 
        \end{split}
    \end{equation*}
    Since 
    \begin{equation*}
        \begin{split}
            \left. \frac{\partial}{\partial u} (d \circ \bvec{x})(u,v) \right|_{(u,v)} &= \frac{(u - x_0 + (z(u,v) - z_0) z_u(u,v))}{\sqrt{(u - x_0)^2 + (v - y_0)^2 + (z(u,v) - z_0)^2}}, \\
            \left. \frac{\partial}{\partial v} (d \circ \bvec{x})(u,v) \right|_{(u,v)} &= \frac{(v - y_0 + (z(u,v) - z_0) z_v(u,v))}{\sqrt{(u - x_0)^2 + (v - y_0)^2 + (z(u,v) - z_0)^2}},
        \end{split}
    \end{equation*}
    and $ z(u,v) $ is differentiable, we conclude that $ d \circ \bvec{x} $ is differentiable except when $ (u,v) = (x_0,y_0) = \bvec{x}^{-1}(p_0) $. Since the choice of $ p \in S $ is arbitrary, we conclude that $ d $ is differentiable on $ S \backslash \{p_0\} $.
\end{solution}

% Problem 5
\begin{problem}[Do Carmo 2.3.10]
    Let $C$ be a plane regular curve which lies on one side of a straight line $r$ of the plane and meets $r$ at the points $p,q$ (Fig.\ 2--21). What conditions should $C$ satisfy to ensure that the rotation of $C$ about $r$ generates an extended (regular) surface of revolution?
    \begin{figure}[h]
        \centering
        \includegraphics[width=0.4\textwidth]{2-21.png}
        % \caption{Figure 2-21}
    \end{figure}
\end{problem} 

\begin{solution}
    We can analyze the point $ p \in C $ locally. Assume that $ r $ is the $ z $ axis, and $ C $ is the graph of a differentiable function $ y = f(x) $ in a neighborhood of $ p $, since $ C $ is a regular curve. Since $ S $ is the surface of revolution generated by rotating $ C $ about $ r $, we claim that there is a local chart at $ p \in S $ given by
    \begin{equation*}
        \bvec{x}: U \subseteq \mathbb{R}^{2} \to S, \quad (x,y) \mapsto (x, y, f(\sqrt{x^2 + y^2})),
    \end{equation*} 
    where $ U $ is an open set in $ \mathbb{R}^2 $. We will check each condition given in definition (\ref{def:regular_surface}) for $ S $.
    \begin{enumerate}[(i)]
        \item $ \bvec{x} $ is differentiable. We can calculate its differential at some $ (x,y) \in U $ as 
        \begin{equation}
            \label{eq:differential}
            \mathrm{d}\bvec{x}_{(x,y)} = 
            \begin{pmatrix}
                1 & 0 \\
                0 & 1 \\
                \cfrac{x}{\sqrt{x^2 + y^2}} f^{\prime} (\sqrt{x^2 + y^2}) & \cfrac{y}{\sqrt{x^2 + y^2}} f^{\prime} (\sqrt{x^2 + y^2})
            \end{pmatrix}. 
        \end{equation}
        Since $ f $ is differentiable, the partial derivatives of $ \bvec{x} $ exist whenever $ (x,y) \neq (0,0) $. By symmetry, $ f(w) = f(-w) $, so $ f^{\prime} (w) = -f^{\prime} (-w) $. When $ (x,y) = (0,0) $, we have $ f^{\prime} (0) = 0 $, and 
        \begin{equation*}
            \frac{x}{\sqrt{x^2 + y^2} }, \quad \frac{y}{\sqrt{x^2 + y^2} }
        \end{equation*}
        are bounded, so $ \mathrm{d}\bvec{x}_{(x,y)} $ exists at $ (0,0) $. To satisfy the symmetry condition, we require that $ f^{\prime} $ is odd, hence $ f $ is even, and all the odd-order derivatives of $ f $ vanish at $ 0 $. Similarly, the odd-order derivatives of $ g $ such that $ y = g(x) $ in a neighborhood of $ q $ must also vanish.
        \item $ \bvec{x} $ is a homeomorphism, since the graph of a continuous function is homeomorphic to its domain.
        \item From equation (\ref{eq:differential}), we have $ \left\vert \partial (x,y) / \partial (u,v) \right\vert = 1 $, so $ \mathrm{d}\bvec{x} $ is one-to-one. Hence $ \mathrm{d}\bvec{x}_{(x,y)} $ is one-to-one for all $ (x,y) \in U $. 
    \end{enumerate}
\end{solution}

\end{CJK}
\end{document}