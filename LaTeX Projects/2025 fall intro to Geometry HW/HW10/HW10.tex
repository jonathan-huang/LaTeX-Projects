\documentclass[a4paper]{article}
%% Formatting %%
\usepackage[margin=3cm]{geometry}
\usepackage{type1cm, titlesec, fancyhdr, titling}
\usepackage{multicol}
\usepackage[dvipsnames]{xcolor}
\usepackage{ulem}
\usepackage{parskip}
\setlength{\parindent}{2em}
\setlength{\headheight}{15pt}
\setlength{\droptitle}{-1.5cm}
\parindent=24pt
%% Math and Symbols %%
\usepackage{amsmath,amsthm,amssymb, mathtools}
\usepackage{yhmath, faktor, dsfont}
\usepackage{academicons, wasysym, marvosym}
\usepackage[scr]{rsfso} 
\usepackage{latexsym, amsmath, amscd, amsmath, amsthm}
\usepackage{amssymb,amsmath,amsthm,graphicx,dsfont}
\usepackage{hyperref}

%% Enhancement %%
\usepackage{graphicx, tabularx}
\usepackage[shortlabels,inline]{enumitem}
%% TikZ %%
\usepackage{tikz-cd}
\usepackage[breakable]{tcolorbox}
\usetikzlibrary{decorations.pathmorphing}
\usetikzlibrary{calc, arrows,matrix}

%% Other packages %%
\usepackage{amsopn}

%% Traditional Chinese %%
\usepackage{CJKutf8}

%% Math environments %%
\newtheoremstyle{mystyle}
  {6pt}{15pt}% 上下間距
  {}%          內文字體
  {}%              縮排
  {\bf}%       標頭字體
  {.}%       標頭後標點
  {1em}% 內文與標頭距離
  {}% Theorem head spec (can be left empty, meaning 'normal')
\theoremstyle{mystyle}	
\newtheorem{theorem}{Theorem}
\newtheorem{definition}{Definition}
\newtheorem{example}[theorem]{Example}
\newtheorem{exercise}{Exercise}
\newtheorem{solution}{Solution}
\newtheorem{corollary}[theorem]{Corollary}
\newtheorem{property}[theorem]{Property}
\newtheorem{proposition}[theorem]{Proposition}
\newtheorem{lemma}{Lemma}
\newtheorem{problem}[theorem]{Problem}
\newtheorem{answer}{Answer}[section]
\newtheorem{fact}[theorem]{fact}
\newtheorem*{claim}{Claim}
\newtheorem*{observation}{Observation}

\theoremstyle{remark}
\newtheorem*{remark}{Remark}

\newcommand{\bvec}[1]{\mathbf{#1}} % vector

\begin{document}
\begin{CJK}{UTF8}{bkai}

\title{%
  \textbf{2025 Fall Introduction to Geometry} \\
  \vspace{0.5cm}
  \large 
  Homework 10 (Due Nov 28, 2025)\\
}
\author{物理三 黃紹凱 B12202004}

\maketitle

\begin{definition}[isometry]
    A diffeomorphism $ \varphi: S \to \overline{S} $ is an \emph{isometry} if for all $ p \in S $ and all pairs $ w_1, w_2 \in T_p (S) $ we have 
    \[
        \langle w_1, w_2 \rangle_p = \langle \mathrm{d}\varphi_p (w_1), \mathrm{d}\varphi_p (w_2) \rangle_{\varphi(p)}.
    \] 
    The surfaces $ S $ and $ \overline{S} $ are then said to be \emph{isometric}.
\end{definition}
\begin{remark}
    An isometry is a diffeomorphism that preserves the first fundamental form.
\end{remark}

\begin{proposition}[Do Carmo Proposition 4.2.1]
    Assume the existence of parametrizations $ \bvec{x}: U \to S $ and $ \overline{\bvec{x}}: U \to \overline{S} $ such that $ E = \overline{E} $, $ F = \overline{F} $, $ G = \overline{G} $ in $ U $. Then $ \overline{\bvec{x}} \circ \bvec{x}^{-1}: \bvec{x}(U) \to \overline{S} $ is a local isometry.
\end{proposition}

% Exercise 1
\begin{exercise}[Do Carmo 4.2.5]
    Let $\alpha_1 : I \to \mathbb{R}^3$, $\alpha_2 : I \to \mathbb{R}^3$ be regular parametrized curves, where the parameter is the arc length. Assume that the curvatures $k_1$ of $\alpha_1$ and $k_2$ of $\alpha_2$ satisfy
    \[
    k_1(s) = k_2(s) \neq 0, \quad s \in I.
    \]
    Let
    \[
    \mathbf{x}_1(s,v) = \alpha_1(s) + v \alpha_1'(s), \qquad
    \mathbf{x}_2(s,v) = \alpha_2(s) + v \alpha_2'(s)
    \]
    be their (regular) tangent surfaces (cf.\ Example 5, Sec.\ 2-3) and let $V$ be a neighborhood of $(s_0,v_0)$ such that $\mathbf{x}_1(V) \subset \mathbb{R}^3$, $\mathbf{x}_2(V) \subset \mathbb{R}^3$ are regular surfaces (cf.\ Prop.\ 2, Sec.\ 2-3). Prove that
    \[
    \mathbf{x}_1 \circ \mathbf{x}_2^{-1} : \mathbf{x}_2(V) \longrightarrow \mathbf{x}_1(V)
    \]
    is an isometry.
\end{exercise}

\begin{solution}
    To show that $ \mathbf{x}_1 \circ \mathbf{x}_2^{-1} $ is an isometry, we need to show that it is a diffeomorphism and preserves the first fundamental form. From Example 2.3.5, the tangent surface of a regular curve $ \alpha $ is a regular surface, since for all $ (t,v) \subseteq U = \{(t,v) \in I \times \mathbb{R} \mid v \neq 0 \} $, we have
    \[
        k(s) = \frac{\vert \alpha^{\prime} (s) \wedge \alpha^{\prime\prime} (s) \vert}{\vert \alpha^{\prime}(s) \vert^3} \neq 0 \implies \frac{\partial \bvec{x}}{\partial s} \wedge \frac{\partial \bvec{x}}{\partial v} = v\alpha''(s) \wedge \alpha'(s) \neq 0.
    \]
    Thus, both $ \bvec{x}_1 $ and $ \bvec{x}_2 $ are regular parametrizations, and hence homeomorphisms on a small neighborhood $ V \subseteq \mathbb{R}^3 $. Since $ \bvec{x} $ is differentiable and $ \mathrm{d}\bvec{x}_i $ has full rank, $ \bvec{x}_i^{-1} $ is differentiable for $ i=1,2 $ by the Inverse Function Theorem. Therefore, $ \bvec{x}_1 \circ \bvec{x}_2^{-1} $ is a diffeomorphism. In the Frenet frames of $ \alpha_i $, $ i=1,2 $, we have $ \bvec{x}_i (s,v) = \alpha_i (s) + v \alpha^{\prime} (s) $, and  
    \[
        \bvec{x}_{i,s} = \alpha^{\prime} (s) + v \alpha^{\prime\prime} (s) = T_i (s) + v k_i (s) N_i (s), \quad \bvec{x}_{i,v} = \alpha^{\prime} (s) = T_i (s). 
    \] 
    The first fundamental form coefficients are computed to be 
    \[
        E_i = \langle \bvec{x}_{i,s}, \bvec{x}_{i,s} \rangle = 1 + v^2 k_i^2 (s), \quad F_i = \langle \bvec{x}_{i,s}, \bvec{x}_{i,v} \rangle = 1, \quad G_i = \langle \bvec{x}_{i,v}, \bvec{x}_{i,v} \rangle = 1.
    \]
    Since $ k_1 (s) = k_2 (s) $ for all $ s \in I $, we have $ E_1 = E_2 $, $ F_1 = F_2 $, $ G_1 = G_2 $. By Proposition 4.2.1, $ \bvec{x}_1 \circ \bvec{x}_2^{-1} $ is a local isometry. Since $ \bvec{x}_1 \circ \bvec{x}_2^{-1} $ is also a diffeomorphism, $ \bvec{x}_1 \circ \bvec{x}_2^{-1} $ is an isometry.
\end{solution}

% Exercise 2
\begin{exercise}[Do Carmo 4.2.6\textbf{*}]
Let $\alpha : I \to \mathbb{R}^3$ be a regular parametrized curve with $k(t) \neq 0$, $t \in I$. Let $\mathbf{x}(t,v)$ be its tangent surface. Prove that, for each $(t_0,v_0) \in I \times (\mathbb{R} - \{0\})$, there exists a neighborhood $V$ of $(t_0,v_0)$ such that $\mathbf{x}(V)$ is isometric to an open set of the plane (thus, tangent surfaces are locally isometric to planes).
\end{exercise}

\begin{solution}
    We will construct the desired local isometry. Fir reparametrize by arc length to get $ \alpha (s) $, and define $ \bvec{x} (s,v) = \alpha (s) + v \alpha^{\prime} (s) $. Let $ k(s) $ be the curvature of $ \alpha (s) $. As in a previous exercise, let 
    \[
        \theta (s) = \int^s_{s_0} \mathrm{d}u\, k(u), \quad s_0 \in I
    \]
    be the angle function, and define a plane curve $ \beta (s) $ by
    \[
        \beta (s) = \left( \int^s_{s_0} \mathrm{d}u\, \cos \theta (u), \int^s_{s_0} \mathrm{d}u\, \sin \theta (u), 0 \right), 
    \]
    \[
        \beta^{\prime} (s) = \left( \cos \theta (s), \sin \theta (s), 0 \right) \implies \vert \beta^{\prime} (s) \vert = 1,
    \]
    \[
        \beta^{\prime\prime} (s) = \theta^{\prime} (s) \left(-\sin \theta (s), \cos \theta (s), 0 \right) = k(s) \left( -\sin \theta (s), \cos \theta (s), 0 \right). 
    \]
    Then, the curvature of $ \beta (s) $ is exactly $ k(s) $, and hence $ \beta (s) $ is a unit-speed curve with the same curvature as $ \alpha $. Since both $ \beta $ and $ \beta^{\prime} $ lie in the plane $ z=0 $, the image of the tangent surface $ \overline{\bvec{x}} (s,v) = \beta (s) + v \beta^{\prime} (s) $ is an open subset of the $ xy $-plane. For $ \bvec{x} $ and $ \overline{\bvec{x}} $, we have 
    \[
        \bvec{x}_s = T(s) + v k(s) N(s), \quad \bvec{x}_v = T(s),
    \]
    \[
        \overline{\bvec{x}}_s = \overline{T}(s) + v k(s) \overline{N}(s), \quad \overline{\bvec{x}}_v = \overline{T}(s),
    \]
    where $ T, N, \overline{T}, \overline{N} $ are the tangent vector and normal vector of $ \bvec{x} $ and $ \overline{\bvec{x}} $, respectively. The first fundamental form coefficients of $ \bvec{x} $ and $ \overline{\bvec{x}} $ are, respectively,
    \[
        E = 1 + v^2 k^2 (s), \quad F = 1, \quad G = 1,
    \]
    \[
        \overline{E} = 1 + v^2 k^2 (s), \quad \overline{F} = 1, \quad \overline{G} = 1.
    \]
    Since the coefficients agree, by Proposition 4.2.1, the map $ \overline{\bvec{x}} \circ \bvec{x}^{-1} $ is a local isometry from $ \bvec{x} (V) $ to an open set of the plane for some neighborhood $ V $ of $ (s_0, v_0) $. Therefore, the tangent surface is locally isometric to an open set of the plane.
\end{solution}

% Exercise 3
\begin{exercise}[Do Carmo 4.2.7]
    Let $V$ and $W$ be $n$-dimensional vector spaces with inner products denoted by $\langle \,,\, \rangle$ and let $F : V \to W$ be a linear map. Prove that the following conditions are equivalent:
    \begin{enumerate}[label=\textbf{\alph*.}]
        \item $\langle F(v_1), F(v_2) \rangle = \langle v_1, v_2 \rangle$ for all $v_1, v_2 \in V$.
        \item $\lvert F(v) \rvert = \lvert v \rvert$ for all $v \in V$.
        \item If $\{v_1,\dots,v_n\}$ is an orthonormal basis in $V$, then $\{F(v_1),\dots,F(v_n)\}$ is an orthonormal basis in $W$.
        \item There exists an orthonormal basis $\{v_1,\dots,v_n\}$ in $V$ such that $\{F(v_1),\dots,F(v_n)\}$ is an orthonormal basis in $W$.
    \end{enumerate}
    If any of these conditions is satisfied, $F$ is called a \emph{linear isometry} of $V$ into $W$. (When $W = V$, a linear isometry is often called an \emph{orthogonal transformation}.)
\end{exercise}

\begin{solution}
    ~ 

    \begin{itemize}
        \item \textbf{a.}$ \implies $\textbf{b.} Suppose $ \langle F(v_1), F(v_2) \rangle = \langle v_1, v_2 \rangle $ for all $ v_1, v_2 \in V $. Then for all $ v \in V $,
        \[
            \vert v \vert = \sqrt{\langle v,v\rangle} = \sqrt{\langle F(v), F(v) \rangle} = \vert F(v) \vert.
        \]

        \item \textbf{b.}$ \implies $\textbf{c.} Suppose $ \vert F(v) \vert = \vert v \vert $ for all $ v \in V $. Let $ \{ v_1, \dots, v_n \} $ be an orthonormal basis of $ V $. Then, for all $ i, j = 1, \dots, n $, since the inner product is induced by a norm $ \vert \cdot \vert $, we have
        \begin{align*}
            \langle F(v_i), F(v_j) \rangle &= \frac{1}{2} \left( \vert F(v_i) + F(v_j) \vert^2 - \vert F(v_i) \vert^2 - \vert F(v_j) \vert^2 \right) \\
            &= \frac{1}{2} \left( \vert v_i + v_j \vert^2 - \vert v_i \vert^2 - \vert v_j \vert^2 \right) = \langle v_i, v_j \rangle = \delta_{ij}.
        \end{align*}
        Thus, $ \{ F(v_1), \dots, F(v_n) \} $ is an orthonormal set in $ W $. Since $ F $ is linear, $ \{ F(v_1), \dots, F(v_n) \} $ spans $ \operatorname{Im}(F) $. Since $ \dim(\operatorname{Im}(F)) \leq n $, we have $ \dim(\operatorname{Im}(F)) = n $, and hence $ \{ F(v_1), \dots, F(v_n) \} $ is an orthonormal basis of $ W $.

        \item \textbf{c.}$ \implies $\textbf{d.} Since $ V $ is finite-dimensional, just pick any orthonormal basis of $ V $.
        
        \item \textbf{d.}$ \implies $\textbf{a.} Suppose there exists an orthonormal basis $ \{ v_1, \dots, v_n \} $ of $ V $ such that $ \{ F(v_1), \dots, F(v_n) \} $ is an orthonormal basis of $ W $. For all $ v_1, v_2 \in V $, we can write 
        \[
            v_1 = \sum_{i=1}^n a_i v_i, \quad v_2 = \sum_{j=1}^n b_j v_j,
        \]
        where $ a_i, b_j \in \mathbb{R} $. Then,
        \begin{align*}
            \langle F(v_1), F(v_2) \rangle &= \left\langle F\left( \sum_{i=1}^n a_i v_i \right), F\left( \sum_{j=1}^n b_j v_j \right) \right\rangle \\
            &= \left\langle \sum_{i=1}^n a_i F(v_i), \sum_{j=1}^n b_j F(v_j) \right\rangle \\
            &= \sum_{i=1}^n \sum_{j=1}^n a_i b_j \langle F(v_i), F(v_j) \rangle \\
            &= \sum_{i=1}^n \sum_{j=1}^n a_i b_j \delta_{ij} = \sum_{i=1}^n a_i b_i = \left\langle \sum_{i=1}^n a_i v_i, \sum_{j=1}^n b_j v_j \right\rangle = \langle v_1, v_2 \rangle.
        \end{align*}
    \end{itemize}
\end{solution}

% Exercise 4
\begin{exercise}[Do Carmo 4.2.8\textbf{*}]
    Let $G : \mathbb{R}^3 \to \mathbb{R}^3$ be a map such that
    \[
    \lvert G(p) - G(q) \rvert = \lvert p - q \rvert \quad \text{for all } p,q \in \mathbb{R}^3
    \]
    (that is, $G$ is a distance-preserving map). Prove that there exists $p_0 \in \mathbb{R}^3$ and a linear isometry (cf.\ Exercise 7) $F$ of the vector space $\mathbb{R}^3$ such that
    \[
    G(p) = F(p) + p_0 \quad \text{for all } p \in \mathbb{R}^3.
    \]
\end{exercise}

\begin{solution}
    Let $ p_0 = G(0) $, and let $ F(p) = G(p) - p_0 $. Then, for all $ p, q \in \mathbb{R}^3 $, we have
    \[
        \vert F(p) - F(q) \vert = \vert G(p) - G(q) \vert = \vert p - q \vert, \quad F(0) = G(0) - p_0 = 0.
    \] 
    Hence $ F $ is a distance-preserving map that fixes the origin. Let $ \{e_1, e_2, e_3\} $ be the standard basis of $ \mathbb{R}^3 $, and $ v_i = F(e_i) $ for $ i=1,2,3 $. Since $ F $ is distance-preserving, we have
    \[
        \vert v_i \vert^2 = \vert F(e_i) - F(0) \vert^2 = \vert e_i - 0 \vert^2 = 1, \quad \vert v_i - v_j \vert^2 = \vert F(e_i) - F(e_j) \vert^2 = \vert e_i - e_j \vert^2 = 2,
    \]
    squaring both sides gives
    \[
        \langle v_i, v_j \rangle = 0 \text{ for } i \neq j \implies \{v_1, v_2, v_3\} \text{ is an orthonormal basis for } \mathbb{R}^3.
    \]
    Let $ L: \mathbb{R}^3 \to \mathbb{R}^3 $ be defined by $ L(e_i) = v_i $ for $ i=1,2,3 $. Then $ L $ is linear by construction, and $ L(e_i) = v_i = F(e_i) $, $ i=1,2,3 $. For any $ p \in \mathbb{R}^3 $, since $ L(0) = 0 $, by the distance-preserving property of $ F $, we have $ \vert F(p) \vert = \vert p \vert = \vert L(p) \vert $. Then, for all $ i=1,2,3 $, we have 
    \[
        \left\vert F(p) - F(e_i) \right\vert = \left\vert p - e_i \right\vert = \left\vert L(p) - L(e_i) \right\vert.
    \]
    Squaring both sides, then using $ \vert F(p) \vert = \vert L(p) \vert $ and $ F(e_i) = L(e_i) $, we have $ \langle F(p) - L(p), F(e_i) \rangle = 0 $. Hence, $ F = L $, and $ F $ is linear. By Exercise 4.3.7, $ F $ is a linear isometry. Therefore, there exists a linear isometry $ F $ such that $ G(p) = F(p) + p_0 $ for all $ p \in \mathbb{R}^3 $.
\end{solution}

% Exercise 5
\begin{exercise}[Do Carmo 4.2.9]
    Let $S_1$, $S_2$, and $S_3$ be regular surfaces. Prove that
    \begin{enumerate}[label=\textbf{\alph*.}]
        \item If $\varphi : S_1 \to S_2$ is an isometry, then $\varphi^{-1} : S_2 \to S_1$ is also an isometry.
        \item If $\varphi : S_1 \to S_2$, $\psi : S_2 \to S_3$ are isometries, then $\psi \circ \varphi : S_1 \to S_3$ is an isometry.
    \end{enumerate}
    This implies that the isometries of a regular surface $S$ constitute in a natural way a group, called the \emph{group of isometries} of $S$.
\end{exercise}

\begin{solution}
    ~

    \begin{enumerate}[label=\textbf{\alph*.}]
        \item Since $ \varphi $ is an isometry, for all $ p \in S_1 $ and all pairs $ w_1, w_2 \in T_p (S_1) $ we have 
        \[
            \langle w_1, w_2 \rangle_p = \langle \mathrm{d}\varphi_p (w_1), \mathrm{d}\varphi_p (w_2) \rangle_{\varphi(p)}.
        \] 
        Let $ q = \varphi(p) \in S_2 $ and $ u_1, u_2 \in T_q (S_2) $. Since $ \varphi $ is a diffeomorphism, $ \mathrm{d}\varphi $ is injective. Since the differential $ \mathrm{d}\varphi $ is a linear transformation between finite-dimensional spaces, it is also surjective. Thus, there exist $ w_1, w_2 \in T_p (S_1) $ such that $ \mathrm{d}\varphi_p (w_i) = u_i $ for $ i = 1, 2 $. Thus, 
        \[
            \langle \mathrm{d}\varphi_p^{-1} (u_1), \mathrm{d}\varphi_p^{-1} (u_2) \rangle_{q} = \langle w_1, w_2 \rangle_p = \langle u_1, u_2 \rangle_{\varphi(p)}.
        \]
        Therefore, $ \varphi^{-1} $ is an isometry.

        \item Suppose $ \varphi: S_1 \to S_2 $ and $ \psi: S_2 \to S_3 $ are isometries. Since diffeomorphism between regular surfaces is an equivalence relation (by previous exercise), $ \psi \circ \varphi $ is a diffeomorphism. For all $ p \in S_1 $ and all pairs $ w_1, w_2 \in T_p (S_1) $, we have
        \begin{align*}
            \langle w_1, w_2 \rangle_p 
            &= \langle \mathrm{d}\varphi_p (w_1), \mathrm{d}\varphi_p (w_2) \rangle_{\varphi(p)} \\
            &= \langle \mathrm{d}\psi_{\varphi(p)} (\mathrm{d}\varphi_p (w_1)), \mathrm{d}\psi_{\varphi(p)} (\mathrm{d}\varphi_p (w_2)) \rangle_{\psi(\varphi(p))} \\
            &= \langle \mathrm{d}(\psi \circ \varphi)_p (w_1), \mathrm{d}(\psi \circ \varphi)_p (w_2) \rangle_{(\psi \circ \varphi)(p)}, 
        \end{align*}
        where the chain rule is used in the last equality. Therefore, $ \psi \circ \varphi $ is an isometry.
    \end{enumerate}

    \begin{remark}
        Since function composition is associative and the identity map $ \operatorname{id}: S_1 \to S_1 $ is an isometry, by \textbf{a.} and \textbf{b.}, the set of isometries on $S$ forms a group.
    \end{remark}
\end{solution}

\end{CJK}
\end{document}