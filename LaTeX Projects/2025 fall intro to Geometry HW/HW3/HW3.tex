\documentclass[a4paper]{article}
%% Formatting %%
\usepackage[margin=3cm]{geometry}
\usepackage{type1cm, titlesec, fancyhdr, titling}
\usepackage{multicol}
\usepackage[dvipsnames]{xcolor}
\usepackage{ulem}
\usepackage{parskip}
\setlength{\parindent}{2em}
\setlength{\headheight}{15pt}
\setlength{\droptitle}{-1.5cm}
\parindent=24pt
%% Math and Symbols %%
\usepackage{amsmath,amsthm,amssymb, mathtools}
\usepackage{yhmath, faktor, dsfont}
\usepackage{academicons, wasysym, marvosym}
\usepackage[scr]{rsfso} 
\usepackage{latexsym, amsmath, amscd, amsmath, amsthm}
\usepackage{amssymb,amsmath,amsthm,graphicx,dsfont}
\usepackage{hyperref}

%% Enhancement %%
\usepackage{graphicx, tabularx}
\usepackage[shortlabels,inline]{enumitem}
%% TikZ %%
\usepackage{tikz-cd}
\usepackage[breakable]{tcolorbox}
\usetikzlibrary{decorations.pathmorphing}
\usetikzlibrary{calc, arrows,matrix}

%% Other packages %%
\usepackage{amsopn}

%% Traditional Chinese %%
\usepackage{CJKutf8}

%% Math environments %%
\newtheoremstyle{mystyle}
  {6pt}{15pt}% 上下間距
  {}%          內文字體
  {}%              縮排
  {\bf}%       標頭字體
  {.}%       標頭後標點
  {1em}% 內文與標頭距離
  {}% Theorem head spec (can be left empty, meaning 'normal')
\theoremstyle{mystyle}	
\newtheorem{theorem}{Theorem}
\newtheorem*{definition}{Definition}
\newtheorem{example}[theorem]{Example}
\newtheorem{exercise}{Exercise}
\newtheorem{solution}{Solution}
\newtheorem{corollary}[theorem]{Corollary}
\newtheorem{property}[theorem]{Property}
\newtheorem{proposition}[theorem]{Proposition}
\newtheorem{lemma}{Lemma}
\newtheorem{problem}[theorem]{Problem}
\newtheorem{answer}{Answer}[section]
\newtheorem{fact}[theorem]{fact}
\newtheorem*{remark}{Remark}
\newtheorem*{claim}{Claim}
\newtheorem*{observation}{Observation}

\begin{document}
\begin{CJK}{UTF8}{bkai}

\title{%
  \textbf{2025 Fall Introduction to Geometry} \\
  \vspace{0.5cm}
  \large 
  Homework 3 (Due Sep 26, 2025)\\
}
\author{物理、數學三 黃紹凱 B12202004}
\date{\today}

\maketitle

% Problem 1
\begin{problem}[Do Carmo 1.5.10]
    Consider the map
    \begin{equation}
        \alpha(t) = 
        \begin{cases} 
            (t, 0, e^{-1/t^{2}}), & t > 0 \\
            (t, e^{-1/t^{2}}, 0), & t < 0 \\
            (0, 0, 0), & t = 0
        \end{cases}
    \end{equation}
    \begin{enumerate}[label=\textbf{\alph*.}]
        \item Prove that $\alpha$ is a differentiable curve.
        \item Prove that $\alpha$ is regular for all $t$ and that the curvature $k(t) \neq 0$, for $t \neq 0$, $t \neq \pm\sqrt{2/3}$, and $k(0) = 0$.
        \item Show that the limit of the osculating planes as $t \to 0, t > 0$, is the plane $y = 0$ but that the limit of the osculating planes as $t \to 0, t < 0$, is the plane $z = 0$ (this implies that the normal vector is discontinuous at $t = 0$ and shows why we excluded points where $k = 0$).
        \item Show that $\tau$ can be defined so that $\tau \equiv 0$, even though $\alpha$ is not a plane curve.
    \end{enumerate}
\end{problem}
\begin{solution}
    ~
    \begin{enumerate}[(a)]
        \item The curve $ \alpha $ is differentiable if $ \alpha^{\prime} $ exists everywhere. For $ t>0 $ and $ t<0 $ it is made of elementary functions, so it is differentiable. At $ t=0 $, the x coordinate is differentiable, so consider the z coordinateo only. 
        \begin{lemma}
            \label{lem:differentiable}
            The map
            \begin{equation}
                f(x) = \begin{dcases}
                    e^{-1/x^{2}}, & x > 0;\\
                    0, & x \leq 0.
                \end{dcases}
            \end{equation} 
            is differentiable at $ x = 0 $ and $ f^{(n)}(0) = 0 $.
        \end{lemma}
        \begin{proof}
            Let $ f(x) = e^{-1/x^{2}} $, notice that 
            \begin{equation}
                f(x) \leq n! x^{2n} \quad \text{for all } n.
            \end{equation}
            Thus, for $ n=1 $ we have $ f^{\prime} (0) = \lim_{x \to 0} f(x)/x = 0 $ by the squeeze theorem. Assume that $ f^{(k)}(0) = 0 $ for all $ k < n $. By induction we know that $ f^{(k)} $ is of the form $ f^{(m)} (x) = f(x) \sum_{r=1}^{N}a_{r} x^{-r} $ for $ x>0 $, so choosing some $ n $ large enough such that 
            \begin{equation*}
                f^{(k+1)}(x) \leq n! x^{2n} \sum_{r=1}^{N}a_{r} x^{-r} \leq Cx^m
            \end{equation*} 
            for some constant $ C $, we have $ f $ is $ (k+1) $ times differentiable and $ f^{(k+1)}(0) = 0 $. By induction we are done. 
        \end{proof}
        By Lemma (\ref{lem:differentiable}), $ \alpha $ is differentiable. 
        \item The curve has derivative 
        \begin{equation*}
            \alpha^{\prime} = 
            \begin{dcases}
                \left(1, 0, \frac{2}{t^{3}} e^{-1/t^{2}}\right), & t > 0,\\
                \left(1, \frac{2}{t^{3}} e^{-1/t^{2}}, 0\right), & t < 0,\\
                (1, 0, 0), & t = 0.
            \end{dcases}
        \end{equation*}
        Since $ e^{-1/t^{2}} $ is always positive, $ \alpha^{\prime}(t) \neq 0 $ for all $ t $, so $ \alpha $ is regular. Next, we compute the curvature $ k(t) $. 
        \begin{lemma}
            \label{lem:curvature}
            For a regular curve $ \alpha(t) $, the curvature is given by
            \begin{equation}
                \label{eq:Tprime}
                k(t) = \frac{|\alpha^{\prime}(t) \wedge \alpha^{\prime\prime}(t)|}{|\alpha^{\prime}(t)|^3}.
            \end{equation}
        \end{lemma}
        \begin{proof}
            Let $ \alpha \colon I \to \mathbb{R}^3 $ be a regular curve. Then, we have $ T^{\prime} (t(s)) = k(t(s)) N(t(s)) $, where $ t(s) $ is the reparametrization by arc length. Then $ \lvert T^{\prime} (t(s)) \rvert = k(t(s)) \lvert N(t(s)) \rvert = k(t(s)) $. The left hand side is $ \mathrm{d}T / \mathrm{d}s = \left( \mathrm{d}T/\mathrm{d}t \right)\left(\mathrm{d}t / \mathrm{d}s\right) = \left( \mathrm{d}T/\mathrm{d}t \right) / \lvert \alpha^{\prime} (t) \rvert $. Moreover, 
            \begin{equation}
                \frac{\mathrm{d}T}{\mathrm{d}t} = \frac{\lvert \alpha^{\prime} \rvert^{2} \alpha^{\prime\prime} - \left(\alpha^{\prime} \cdot \alpha^{\prime\prime} \right)\alpha^{\prime} }{\lvert \alpha^{\prime} \rvert^{3} } = \frac{\alpha^{\prime} \wedge (\alpha^{\prime\prime} \wedge \alpha^{\prime} )}{\lvert \alpha^{\prime} \rvert^{3}}.  
            \end{equation} 
            Since $ \alpha^{\prime} \perp \alpha^{\prime\prime} \wedge \alpha^{\prime} $, 
            \[
                k(t(s)) = \lvert T^{\prime}(t(s)) \rvert = \frac{|\alpha^{\prime}(t) \wedge \alpha^{\prime\prime}(t)|}{|\alpha^{\prime}(t)|^3}.
            \]
        \end{proof}
        We have $ \alpha^{\prime} (t) $ given above, and 
        \begin{equation*}
            \begin{split}
                \alpha^{\prime\prime} &= 
                \begin{dcases}
                    \left(0, 0, \left(\frac{4}{t^{6}} - \frac{6}{t^{4}}\right) e^{-1/t^{2}}\right), & t > 0,\\
                    \left(0, \left(\frac{4}{t^{6}} - \frac{6}{t^{4}}\right) e^{-1/t^{2}}, 0\right), & t < 0,\\
                    (0, 0, 0), & t = 0.
                \end{dcases} \\
                \alpha^{\prime} \wedge \alpha^{\prime\prime} &= 
                \begin{dcases}
                    \left(0, - \left(\frac{4}{t^{6}} - \frac{6}{t^{4}}\right) e^{-1/t^{2}}, 0\right), & t > 0,\\
                    \left(0, 0, \left(\frac{4}{t^{6}} - \frac{6}{t^{4}}\right) e^{-1/t^{2}}\right), & t < 0,\\
                    (0, 0, 0), & t = 0.
                \end{dcases}
            \end{split}
        \end{equation*}
        Using Lemma \ref{lem:curvature}, we have
        \begin{equation}
            k(t) = 
            \begin{dcases}
                \left| \left(\frac{4}{t^{6}} - \frac{6}{t^{4}}\right) e^{-1/t^{2}} \right| \left.\middle/\right. \left(1 + \frac{4}{t^{6}} e^{-2/t^{2}}\right)^{3/2}, & t \neq 0,\\
                0, & t = 0.
            \end{dcases}
        \end{equation}
        From above we know $ k(t) = 0$ when and only when  $ t = 0 $ and $ t = \pm \sqrt{2/3} $.
        \item The osculating plane is determined by the normal vector $ N(t) $ and the tangent vector $ T(t) $. By equation (\ref{eq:Tprime}) and the definition $ \mathrm{d}T(t(s)) / \mathrm{d}s = k(t(s)) N(t(s))$, the normal vector is
        \begin{equation}
            \begin{split}
                N(t) &= \frac{1}{k(t)} \frac{\mathrm{d}T(t(s))}{\mathrm{d}s} \\
                &= \frac{\alpha^{\prime}(t) \wedge (\alpha^{\prime\prime} (t) \wedge \alpha^{\prime}(t) )}{\lvert \alpha^{\prime} (t) \rvert^{4}} \cdot \frac{|\alpha^{\prime}(t)|^3}{|\alpha^{\prime}(t) \wedge \alpha^{\prime\prime}(t)|} \\
                &= \frac{\alpha^{\prime}(t) \wedge (\alpha^{\prime\prime} (t) \wedge \alpha^{\prime}(t) )}{\lvert \alpha^{\prime}(t) \rvert |\alpha^{\prime}(t) \wedge \alpha^{\prime\prime}(t)|}.
            \end{split}
        \end{equation}
        % Then 
        % \begin{equation*}
        %     \begin{split}
        %         \alpha \wedge \alpha^{\prime} &= 
        %         \begin{dcases}
        %             \left( 0, \left(1 - \frac{2}{t^2}\right) e^{-1/t^{2}}, 0 \right), & t > 0,\\
        %             \left( 0, 0, -\left(1 - \frac{2}{t^2}\right) e^{-1/t^{2}} \right), & t < 0,\\
        %             (0, 0, 0), & t = 0.
        %         \end{dcases} \\
        %             \alpha \wedge \left(\alpha \wedge \alpha^{\prime}\right) &=
        %         \begin{dcases}
        %             \left(1-\frac{2}{t^2}\right)e^{-2/t^{2}} \left(-1,0,1\right), t>0, \\
        %             \left(1-\frac{2}{t^2}\right)e^{-2/t^{2}} \left(-1,1,0\right), & t < 0,\\
        %             (0, 0, 0), & t=0.
        %         \end{dcases}
        %     \end{split}
        % \end{equation*}
        For $ t > 0 $, we have
        \begin{equation*}
            N(t) = \left(1 + \frac{4}{t^6}e^{-1/t^{2}}\right)^{-1/2} \left(-\frac{2}{t^3}e^{-1/t^{2}}, 0, 1\right)
        \end{equation*}
        and
        \begin{equation*}
            T(t) = \left(1 + \frac{4}{t^6}e^{-1/t^{2}}\right)^{-1/2} \left(1, 0, \frac{2}{t^3}e^{-1/t^{2}}\right),
        \end{equation*} 
        hence $ N_P = \lim_{t \to 0^+} T(t)\wedge N(t) = (0, 0, 1) \wedge (1, 0, 0) = (0,1,0) $. Furthermore, $ \lim_{t \to 0^+} \alpha (t) = (0,0,0)$, so the osculating plane is $ y=0 $. 

        On the other hand, for $ t<0 $, we have 
        \begin{equation*}
            N(t) = \left(1 + \frac{4}{t^6}e^{-1/t^{2}}\right)^{-1/2} \left(-\frac{2}{t^3}e^{-1/t^{2}}, 1, 0\right)
        \end{equation*}
        and
        \begin{equation*}
            T(t) = \left(1 + \frac{4}{t^6}e^{-1/t^{2}}\right)^{-1/2} \left(1, \frac{2}{t^3}e^{-1/t^{2}}, 0\right),
        \end{equation*}
        hence $ N_P = \lim_{t \to 0^-} T(t) \wedge N(t) = (0, 1, 0) \wedge (1, 0, 0) = (0,0,-1) $. Furthermore, $ \lim_{t \to 0^-} \alpha (t) = (0,0,0)$, so the osculating plane is $ z=0 $. Notice that $ N(t) $ is discontinuous at $ t=0 $, thus undefined there. 
        \item Since $ k(0) = k(\pm \sqrt{2/3} ) = 0 $, $ N(0) $ and $ N(\pm\sqrt{2/3} ) $ are not well-defined. Therefore, we can define $ \tau $ to be zero at these points. For $ t \neq 0, \pm \sqrt{2/3}$, we have 
        \begin{equation*}
            B(t) = T(t) \wedge N(t) = 
            \begin{dcases}
                -(0,1,0), & t>0, \\
                (0,0,1), & t<0. \\
            \end{dcases}
        \end{equation*} 
        The binormal vector $ B(t) $ is constant on $ I \backslash \{0\} $, so $ B^{\prime} (s) = B^{\prime} (t) \cdot \vert \alpha^{\prime} (t) \vert^{-1} = 0 = \tau (t(s)) N(t(s)) $. Hence we can choose $ \tau (t) \equiv 0 $ for $ t \in I \backslash \{0, \pm \sqrt{2/3} \} $. This is an example of \textbf{a curve with identically zero torsion that is not a plane curve}.  
    \end{enumerate}
\end{solution}

% Problem 2
\begin{problem}[Do Carmo 1.5.17]
     In general, a curve $\alpha$ is called a \emph{helix} if the tangent lines of $\alpha$ make a constant angle with a fixed direction. Assume that $\tau(s) \neq 0$, $s \in I$, and prove that:
    \begin{enumerate}
        \item[*\textbf{a.}] $\alpha$ is a helix if and only if $\frac{k}{\tau} = \text{const}$.
        \item[*\textbf{b.}] $\alpha$ is a helix if and only if the lines containing $n(s)$ and passing through $\alpha(s)$ are parallel to a fixed plane.
        \item[*\textbf{c.}] $\alpha$ is a helix if and only if the lines containing $b(s)$ and passing through $\alpha(s)$ make a constant angle with a fixed direction.
        \item[\textbf{d.}] The curve
        \begin{equation}
            \alpha(s) = \left( \frac{a}{c} \int \sin \theta(s) \, ds, \frac{a}{c} \int \cos \theta(s) \, ds \right)
        \end{equation}
        where $c^2 = a^2 + b^2$, is a helix, and that $\frac{k}{\tau} = \frac{a}{b}$.
    \end{enumerate}
\end{problem}
\begin{solution}
    ~
    \begin{enumerate}[(a)]
        \item Suppose there exists a vector $ v \in \mathbb{R}^3 $ such that $ v \cdot t(s) = C $ for some constant $ C $. Then 
        \begin{equation*}
            \frac{\mathrm{d}t}{\mathrm{d}s} \cdot v = k(s) n(s) \cdot v = 0,
        \end{equation*}
        so $ n(s) \cdot v = 0 $. Differentiating again gives 
        \begin{equation*}
            \frac{\mathrm{d}n}{\mathrm{d}s} \cdot v = -k(s) t(s) \cdot v + \tau (s) b(s) \cdot v = -k(s) C + \tau (s) b(s) \cdot v = 0.
        \end{equation*}
        Since $ \tau (s) \neq 0 $, we have  
        \[ 
            C k(s) / \tau (s) = \left( b(s) \cdot v \right) = \left(t(s) \wedge n(s)\right) \cdot v = \left(v \wedge t(s)\right) \cdot n(s) .
        \]
        Since $ t(s), v \perp n(s) $, the triple product is equal to $ \vert n(s) \vert \vert t(s) \vert \vert v \vert \sin (C) = \vert v \vert \sin C$. Therefore, $ k(s) / \tau (s) $ is a constant. Conversely, if $ k(s) / \tau (s) = C^{\prime} $ for some constant $ C^{\prime} $, then we can take $ v = t(s) + C^{\prime} b(s) $, which is a constant vector since
        \begin{equation*}
            \frac{\mathrm{d}v}{\mathrm{d}s} = k(s) n(s) + C^{\prime} \left( -\tau (s) n(s) \right) = 0.
        \end{equation*} 
        Then
        \begin{equation*}
            \frac{\mathrm{d}t}{\mathrm{d}s}\cdot v = 0.
        \end{equation*} 
        \item Suppose $ \alpha (s) $ is a helix, then there exists a vector $ v \in \mathbb{R}^3 $ such that $ v \cdot t(s) = C $ for some constant $ C $. Let $ L $ be a line containing $ n(s) $ and passing through $ \alpha(s) $. Then $ n(s) \cdot v = 0 $ by result in part (a), so $ L \perp v $, hence parallel to the plane with normal vector $ v $. Conversely, for any point $ s\in I $, suppose the line $ L $ containing $ n(s) $ and passing $ \alpha (s) $ is parallel to the plane $ P $ with normal vector $ v \in \mathbb{R}^{3} $. Then $ n(s) \cdot v = 0 $, and 
        \begin{equation*}
            \frac{\mathrm{d}T}{\mathrm{d}s} \cdot v = k(s) n(s) \cdot v = 0.
        \end{equation*}
        Hence $ \mathrm{d}T / \mathrm{d}s = \mathrm{d} (T\cdot v) / \mathrm{d}s = 0 $, and $ T(s) \cdot v = C^{\prime} $ for some constant $ C^{\prime} $, and $ \alpha (s) $ is a helix.
        \item By definition of helix, there exists a vector $ v \in \mathbb{R}^3 $ such that $ v \cdot t(s) = C $ for some constant $ C $. By (b), all the lines containing $ n(s) $ and passing through $ \alpha (s) $ are parallel to the plane with some fixed normal vector $ u \in \mathbb{R}^{3} $, so $ n(s) \cdot u = 0 $. Consider $ b \cdot (u \wedge v) = (t(s) \wedge n(s)) \cdot (u \wedge v) = (t(s) \cdot u)(n(s) \cdot v) - (t(s) \cdot v)(n(s) \cdot u) = 0 $, since $ n(s) \cdot v = 0 $ from (a). Conversely, suppose there exists a vector $ v \in \mathbb{R}^3 $ such that $ b(s) \cdot v = C $ for some constant $ C $. Then $ \left(t(s) \wedge n(s)\right) \cdot v = C $,
        \begin{equation*}
            \frac{\mathrm{d}b}{\mathrm{d}s} \cdot v = -\tau (s) n(s) \cdot v = 0,
        \end{equation*} 
        and by $ \tau (s) \neq 0 $ we have $ n(s) \cdot v = 0 $. Finally, 
        \begin{equation*}
            \frac{\mathrm{d}}{\mathrm{d}s}\left(t(s) \cdot v\right) = k(s) n(s) \cdot v = 0,
        \end{equation*}
        therefore, $ \alpha (s) $ is a helix. 
        \item With $ s $ suppressed in the expressions, derivatives of $ \alpha $ are 
        \begin{equation*}
            \begin{split}
                \alpha^{\prime} &= \left( \frac{a}{c} \sin \theta(s), \frac{a}{c} \cos \theta(s), \frac{b}{c} \right),\\
                \alpha^{\prime\prime} &= \left( \frac{a}{c} \theta^{\prime}(s) \cos \theta(s), -\frac{a}{c} \theta^{\prime}(s) \sin \theta(s), 0 \right),\\
                \alpha^{\prime\prime\prime} &= \left( \frac{a}{c} \left( \theta^{\prime\prime}(s) \cos \theta(s) - (\theta^{\prime}(s))^2 \sin \theta(s) \right), -\frac{a}{c} \left( \theta^{\prime\prime}(s) \sin \theta(s) + (\theta^{\prime}(s))^2 \cos \theta(s) \right), 0 \right).
            \end{split}
        \end{equation*}
        The curvature is $ k(s) = \vert \alpha^{\prime} (s) \vert = \frac{a}{c} \theta^{\prime} $. The torsion is given by the formula 
        \begin{equation*}
            \tau (s) = - \frac{(\alpha^{\prime}(s) \wedge \alpha^{\prime\prime}(s)) \cdot \alpha^{\prime\prime\prime}(s)}{k(s)^2}
        \end{equation*}  
        by [Do Carmo] Exercise 1.5.2. Direct calculation gives 
        \begin{equation*}
            (\alpha^{\prime} \wedge \alpha^{\prime\prime}) \cdot \alpha^{\prime\prime\prime} = \left( \frac{ab}{c^2} \theta^{\prime}(s) \sin \theta(s), -\frac{ab}{c^2} \theta^{\prime}(s) \cos \theta(s), -\frac{a^2}{c^2} (\theta^{\prime}(s))^2 \right) = \frac{a^{2} b}{c^{3}}(\theta^{\prime})^{3} ,
        \end{equation*}
        so 
        \begin{equation*}
            \tau (s) = \frac{b}{c} \theta^{\prime}(s) = \frac{b}{a} k(s).
        \end{equation*}
    \end{enumerate}
\end{solution}

% Problem 3
\begin{problem}[Do Carmo 1.6.1]
    Let $\alpha: I \to \mathbb{R}^3$ be a curve parametrized by arc length with curvature $k(s) \neq 0$, $s \in I$. Let $P$ be a plane satisfying both of the following conditions:

    \begin{enumerate}
        \item $P$ contains the tangent line at $s$.
        \item Given any neighborhood $J \subset I$ of $s$, there exist points of $\alpha(J)$ in both sides of $P$.
    \end{enumerate}

    Prove that $P$ is the osculating plane of $\alpha$ at $s$.
\end{problem}
\begin{solution}
    Let $ n $ be the normal vector of plane $ P $, then condition 1 implies that $ n_P \perp t(s) $, as $ t(s) \in P $. To show the desired reuslt, we will show that $ n(s) \perp n_P $. Consider $ f(s) = t(s) \cdot n_P = 0 $, differentiating both sides gives $ f^{\prime}(s) = t(s) \cdot n_P^{\prime} = k(s) n(s) \cdot n_P = 0 $, so $ n(s) \perp n_P $. Thus, the binormal vector $ b(s) \parallel n_P$. Furthermore, by condition 2 we can take some interval $ J = \left(s-\frac{1}{m}, s+\frac{1}{m}\right) \subseteq I $, then there exists $ s_1^{(m)} \in \left(s-\frac{1}{m}, s\right) $ and $ s_2^{(m)} \in \left(s, s+\frac{1}{m}\right) $ such that $ \alpha(s_1^{(m)}) $ and $ \alpha(s_2^{(m)}) $ are in different sides of plane $ P $. This holds for all $ m \in \mathbb{N} $, so as $ m \to \infty $, $ p \equiv \alpha (s) = \lim_{m \to \infty} \alpha (s_1^{(m)}) $ lies on the left side of $ P $, and $ p \equiv \alpha (s) = \lim_{m \to \infty} \alpha (s_2^{(m)}) $ lies on the right side of $ P $, hence $ p = \alpha (s) \in P $. Since $ P $ contains $ \alpha (s) $ and has $ b(s) $ as a normal vector, $ P $ is the osculating plane of $ \alpha $ at $ s $. 
\end{solution}

% Problem 4
\begin{problem}[Do Carmo 1.6.2]
    Let $\alpha: I \to \mathbb{R}^3$ be a curve parametrized by arc length, with curvature $k(s) \neq 0$, $s \in I$. Show that

    \begin{enumerate}
        \item[*\textbf{a.}] The osculating plane at $s$ is the limit position of the plane passing through $\alpha(s)$, $\alpha(s + h_1)$, $\alpha(s + h_2)$ when $h_1, h_2 \to 0$.
        \item[\textbf{b.}] The limit position of the circle passing through $\alpha(s)$, $\alpha(s + h_1)$, $\alpha(s + h_2)$ when $h_1, h_2 \to 0$ is a circle in the osculating plane at $s$, the center of which is on the line that contains $n(s)$ and the radius of which is the radius of curvature $1/k(s)$; this circle is called the \emph{osculating circle} at $s$.
    \end{enumerate}
\end{problem}
\begin{solution}
    ~
    \begin{enumerate}[(a)]
        \item Since the plane, which we will call $ P $, by construction passes through $ \alpha (s) $, we are left to show that the normal vector $ n_P $ of $ P $ converges to $ b(s) $ in the limit $ h_1, h_2 \to 0 $. We have 
        \begin{equation*}
            \begin{split}
                n_P &= \frac{\left(\alpha (s + h_1) - \alpha (s) \right) \wedge \left(\alpha (s + h_2) - \alpha (s)\right)}{\left\vert \left(\alpha (s + h_1) - \alpha (s) \right) \wedge \left(\alpha (s + h_2) - \alpha (s)\right) \right\vert } \\
                &= \frac{\left( h_1 \alpha^{\prime} (s) + O(h_1^2) \right) \wedge \left( h_2 \alpha^{\prime} (s) + O(h_2^2) \right)}{\left\vert \left( h_1 \alpha^{\prime} (s) + O(h_1^2) \right) \wedge \left( h_2 \alpha^{\prime} (s) + O(h_2^2) \right) \right\vert } \\
                &= \left(\frac{ \alpha^{\prime} (s) \wedge \alpha^{\prime\prime} (s)}{\left\vert \alpha^{\prime} (s) \wedge \alpha^{\prime\prime} (s) \right\vert } + O(h_1) + O(h_2) \right),
            \end{split}
        \end{equation*}
        hence 
        \begin{equation*}
            \lim_{h_1, h_2 \to 0} n_P = \frac{ \alpha^{\prime} (s) \wedge \alpha^{\prime\prime} (s)}{\left\vert \alpha^{\prime} (s) \wedge \alpha^{\prime\prime} (s) \right\vert } .
        \end{equation*}
        Then the binormal vector is parallel to $ N_P $ since  
        \begin{equation*}
            b(s) = t(s) \wedge n(s) = \alpha^{\prime} (s) \wedge \alpha^{\prime\prime} (s) / \vert \alpha^{\prime\prime} (s) \vert \parallel n_P.
        \end{equation*}
        \item Without loss of generality, shift the origin to $ s $ so that $ \alpha (s), \alpha (s+h_1), \alpha (s+h_2) $ become $ \alpha (0), \alpha (h_1), \alpha (h_2) $, respectively. Let $ (x_0,y_0,z_0) $ be the center of the circle passing through $ \alpha (0) $, $ \alpha (h_1) $, and $ \alpha (h_2) $, then the equation of the circle can be written as $ F(s) = (x(s) - x_0)^{2} + (y(s)-y_0)^{2} + (z(s)-z_0)^{2} - r^{2} $. Calculate the derivatives to be 
        \begin{equation*}
            F^{\prime} (s) = 2(x(s)-x_0)x^{\prime}(s) + 2(y(s)-y_0)y^{\prime}(s) + 2(z(s)-z_0)z^{\prime}(s)
        \end{equation*} 
        and 
        \begin{equation*}
            \begin{split}
                F^{\prime\prime} (s) &= 2(x^{\prime}(s))^{2} + 2(y^{\prime}(s))^{2} + 2(z^{\prime}(s))^{2} \\
                &+ 2(x(s)-x_0)x^{\prime\prime}(s) + 2(y(s)-y_0)y^{\prime\prime}(s) + 2(z(s)-z_0)z^{\prime\prime}(s).
            \end{split}
        \end{equation*}
        Taking the limit as $ s \to 0 $ gives $ F^{\prime} (0) = -2 x_0 $ and $ F^{\prime\prime} (0) = 2-2k(0) y_0 $. Since the plane passes through $ \alpha (0), \alpha (h_1), \alpha (h_2) $, we have $ F(0) = F(h_1) = F(h_2) = 0 $. By the Mean Value Theorem, there exists some $ s_1 \in (0, h_1) $ such that $ F^{\prime} (s_1) = 0 $. As $ h_1 \to 0 $, we have $ s_1 \to 0 $, by continuity of $ F $ we have $ F^{\prime} (s_1) \to 0 $ as $ s_1 \to 0 $ as $ h_1, h_2 \to 0 $. Similarly, suppose $ h_1 < h_2 $, there exists some $ s_2 \in (h_1, h_2) $ such that $ F^{\prime} (s_2) = 0 $. By the Mean Value Theorem, there exists some $ s_3 \in (s_1, s_2) $ such that $ F^{\prime\prime} (s_3) = 0 $. As $ h_1, h_2 \to 0 $, we have $ s_1, s_2 \to 0 $, so by continuity of $ F^{\prime\prime} $, $ F^{\prime\prime} (s_3) \to 0 $ as $ s_3 \to 0 $. Therefore,
        \begin{equation*}
            \lim_{h_1, h_2 \to 0} F^{\prime} (s_1) = F^{\prime} (0) = -2 x_0 = 0 \implies x_0 = 0,  
        \end{equation*}
        and
        \begin{equation*}
            \lim_{h_1, h_2 \to 0} F^{\prime\prime} (s_2) = F^{\prime\prime} (0) = 2-2k(0)y_0 = 0 \implies y_0 = \frac{1}{k(0)}.
        \end{equation*}
        By $ (a) $ we know the circle lies on the osculating plane at $ \alpha (0) $ as $ h_1, h_2 \to 0 $, so $ c \to 0 $. Hence the center of the circle converges to $ (0, 1/k(0), 0) $, which lies on the line containing $ n(0) $, and the radius converges to $ 1/k(0) $.
    \end{enumerate}
\end{solution}

\end{CJK}
\end{document}