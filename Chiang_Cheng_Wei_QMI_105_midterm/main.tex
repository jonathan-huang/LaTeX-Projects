\documentclass[12pt, A4, twoside]{article}
\usepackage{scrextend}
\usepackage{geometry}
 \geometry{
	 a4paper,
 	total={170mm,257mm},
 	left=20mm,
	 top=20mm,
 }

\usepackage{amsmath}
\usepackage[shortlabels]{enumitem} %enumerate with letters
\usepackage[normalem]{ulem}
\usepackage{indentfirst}
\usepackage{pifont}
\usepackage{fancyhdr}   % 頁首頁尾
\usepackage{amssymb}
\usepackage{pgfplots}
\pgfplotsset{compat=1.15}
\usetikzlibrary{arrows}

\usepackage{hyperref}
\hypersetup{
    colorlinks,
    citecolor=black,
    filecolor=black,
    linkcolor=blue,
    urlcolor=url
}

% Mandarin
\usepackage{CJKutf8}
% bkai = 標楷體
% bsmi = 新細明體

\usepackage{graphicx} % Required for inserting images

% paragraph
%\setlength{\parskip}{\baselineskip}
\setlength{\parindent}{2em}
%
\usepackage{lmodern,bm}                
\usepackage[T1]{sansmath} 
\SetMathAlphabet{\mathsfbf}{sans}{\sansmathencoding}{\sfdefault}{bx}{sl}
\usepackage{etoolbox}
\AtBeginEnvironment{sansmath}{\let\bm\mathsfbf}{}{}
\usepackage{mdframed}
\usepackage{braket} % braket notation
\usepackage{amsmath, nccmath}

\usepackage{bbm}
\usepackage{dsfont} % bold numerals
\usepackage{bbold} % blackboard bold font
\usepackage{mathrsfs} % script

%% custom norm
\newcommand{\Norm}[1]{\left\lVert#1\right\rVert}
\newcommand{\Abs}[1]{\left\lvert#1\right\rvert}
%%

% circled number
\newcommand*\circlednum[1]{\raisebox{.5pt}{\textcircled{\raisebox{-.9pt} {#1}}}}

\usepackage{relsize} % math symbol size

\usepackage{amsthm}
\newtheorem{thm}{Theorem}[section]
\newtheorem{mydef}{Definition}[section]
\newtheorem{lemma}[thm]{Lemma}
\newtheorem{proposition}[thm]{Proposition}
\newtheorem{corollary}[thm]{Corollary}
\newtheorem{property}{Property}

\newtheoremstyle{problemstyle}
        {5pt} % <space above>
        {15pt} % <space below>
        {\normalfont} % <body font>
        {} % <indent amount}
        {\bfseries} % <theorem head font>
        {\normalfont\bfseries.} % <punctuation after theorem head>
        {.5em} % <space after theorem head>
        {} % <theorem head spec (can be left empty, meaning `normal')>
\theoremstyle{problemstyle}
\newtheorem{problem}{Problem}
\newtheorem{solution}{Solution}[section]
\newtheorem{example}{Example}[section]

\newenvironment{problembk}[1][]{%
  \begin{problem}[#1]$ $\par\nobreak\ignorespaces
}{%
  \end{problem}
}
\newenvironment{solutionbk}[1][]{%
  \begin{solution}[#1]$ $\par\nobreak\ignorespaces
}{%
  \end{solution}
}

\theoremstyle{remark}
\newtheorem*{remark}{Remark}

\usepackage{mdframed}
\newenvironment{fthm}
    {\begin{mdframed}\begin{thm}}
    {\end{thm}\end{mdframed}}

\newcommand{\bvec}[1]{\mathbf{#1}} % vector

% calculus
\newcommand{\D}{\mathrm{d}}
\newcommand{\Par}{\partial}
% Calligraphy (Euler script?)
\newcommand{\cA}{\mathcal{A}}
\newcommand{\cB}{\mathcal{B}}
\newcommand{\cC}{\mathcal{C}}
\newcommand{\cF}{\mathcal{F}}
\newcommand{\cH}{\mathcal{H}}
\newcommand{\cI}{\mathcal{I}}
\newcommand{\cL}{\mathcal{L}}
\newcommand{\cM}{\mathcal{M}}
\newcommand{\cP}{\mathcal{P}}
\newcommand{\cT}{\mathcal{T}}
% Calligraphy (blackboard bold)
\newcommand{\bC}{\mathbbm{C}}
\newcommand{\bD}{\mathbbm{D}}
\newcommand{\bF}{\mathbbm{F}}
\newcommand{\bH}{\mathbbm{H}}
\newcommand{\bI}{\mathbbm{I}}
\newcommand{\bK}{\mathbbm{K}}
\newcommand{\bN}{\mathbbm{N}}
\newcommand{\bO}{\mathbbm{O}}
\newcommand{\bP}{\mathbbm{P}}
\newcommand{\bQ}{\mathbbm{Q}}
\newcommand{\bR}{\mathbbm{R}}
\newcommand{\bT}{\mathbbm{T}}
\newcommand{\bZ}{\mathbbm{Z}}
% Calligraphy (fraktur font)
\newcommand{\fI}{\mathfrak{I}}
\newcommand{\fR}{\mathfrak{R}}
% lower indices (roman)
\newcommand{\rC}{\mathrm{C}} % Curie constant
\newcommand{\rf}{\mathrm{f}} % final
\newcommand{\ri}{\mathrm{i}} % initial
\newcommand{\rT}{\mathrm{T}} % isothermal

% figures
\usepackage{wrapfig} 
\usepackage{subfig}

% reference
\usepackage{hyperref}
\hypersetup{
    colorlinks=true,
    linkcolor=blue,
    filecolor=magenta,      
    urlcolor=blue,
    pdftitle={Overleaf Example},
    pdfpagemode=FullScreen,
}

\begin{document}
\begin{CJK}{UTF8}{bkai}

\title{蔣正偉 Quantum Mechanics I 105-1}
\author{Jonathan Huang (Giant Water Bird)}
\date{\today}
\cfoot{\thepage}
% header and footer
\pagestyle{fancy}
\rhead{蔣正偉}
\lhead{Quantum Mechanics I}

\maketitle

%\begin{figure}[h!]%
%    \centering
%    \subfloat{\includegraphics[width=8cm]{giant water bird.jpg}}
%    \caption{}
%\end{figure}

\begin{center}
    Instruction
\end{center}
\begin{enumerate}
    \item This is an open-book, 120-minute exam. You are only allowed to use the main textbook (by Sakurai and Napolitano), your own notes and homework assighments. You are allowed to refer to derived results in the textbook only. Please indicate both equation and page numbers when you do so. For example, Eq. (3.1) on p.90 of Sakurai and Napolitano.
    \item  Each sub-problem has 10 points. The total points is 100. Please arrange your time and do not waste too much time on a single problem.
    \item To avoid any misunderstanding, ask if you have any question about the problems or notations.
\end{enumerate}

Source: \url{https://www.ptt.cc/bbs/NTUcourse/M.1485016916.A.0DD.html}

\newpage

\section{Problems}
\begin{problem}
     Remember the operator identity for operators $A$ and $B$ and a real parameter $\lambda$:
     \begin{equation}
         e^{\lambda A}Be^{-\lambda A} = B + \lambda [A,B] + \frac{\lambda^2}{2!}[A,[A,B]] + \frac{\lambda^3}{3!}[A,[A,[A,B]]] + \cdots.
     \end{equation}
     \begin{enumerate}[(a)]
         \item Suppose $[A,[A,B]] = \beta B$ for some constant $\beta$, show that
         \begin{equation}
             e^{\lambda A}Be^{-\lambda A} = B\cosh{\lambda \sqrt{\beta}} + \frac{[A,B]}{\sqrt{\beta}}\sinh{\lambda\sqrt{B}}.
         \end{equation}
         \item If $A=A(t)$, show that
         \begin{equation}
             \frac{\D}{\D t} = e^A\left\{ A' - \frac{1}{2!}[A,A'] + \frac{1}{3!}[A,[A,A
             ]] - \cdots \right\}.
         \end{equation}
     \end{enumerate}
\end{problem}

\begin{problem}
    The wave function $\Psi(\bvec{r},t)$ expanded in terms of plane waves with definite momenta is
    \begin{equation}
        \Psi(\bvec{r},t) = \frac{1}{(2\pi\hbar)^{3/2}}\int \D^3 p\,\Phi(\bvec{p},t)\exp{\left[\frac{i}{\hbar}(\bvec{p}\cdot\bvec{r})\right]}.
    \end{equation}
    \begin{enumerate}[(a)]
        \item Derive the Schr\"{o}dinger equation for $\Phi(\bvec{p},t)$, state any assumption used in the derivation.
        \item Simplify the above result by further assuming that the potential $V(\bvec{r})$ is an analytic function of $\bvec{r}$. 
    \end{enumerate}
\end{problem}

\begin{problem}
    Suppose the Hamiltonian $H$ for a particular quantum system is a well-behaved (differentiable) function of some parameter $\lambda$. Denote $E_n$ and $\ket{\psi_n}$ as the eigenvalues and the corresponding orthonormal eigenkets of $H(\lambda)$. The variable $n$ may be a discrete or continuous set of indices. Assume either that $E_n$ is nondegenerate, or that the eigenkets are the "good" combinations.
    \begin{enumerate}[(a)]
        \item Prove the Feynman-Hellmann theorem:
        \begin{equation}
            \frac{\D E_n}{\D\lambda} = \Braket{\psi_n | \frac{\D H}{\D\lambda} | \psi_n}.
        \end{equation}
        \item  Apply the theorem to the one-dimensional simple harmonic oscillator, using $\lambda = m$, and explain what physics you find. Here you can directly make use the energy spectrum $E_n = (n+\frac{1}{2})\hbar\omega$.
    \end{enumerate}
\end{problem}

\begin{problem}
    Consider a one-dimensional simple harmonic oscillator of mass m and charge $q$. Suppose the system is placed in a static electric field of strength $E$. Therefore, the Hamiltonian of this oscillator is given by
    \begin{equation}
        \frac{p^2}{2m} + \frac{1}{2}m\omega^2x^2 - qEx.
    \end{equation}
    \begin{enumerate}[(a)]
        \item Suppose the electric field is a constant, i.e., $E = E_0$. Derive the energy level for all states.
        \item Write down the wave function $\psi_0(x)$ for the ground state in Part (a). Determine the most likely position of the oscillator in this ground state and give the physical interpretation for your result.
    \end{enumerate}
\end{problem}

\begin{problem}
    Work in Heisenberg picture.
    \begin{enumerate}[(a)]
        \item Derive the quantum mechanic version of the Lorentz force
        \begin{equation}
            m\frac{\D^2\bvec{r}}{\D t^2} = Qe\left[\bvec{E} + \frac{1}{2c}\left(\frac{\D\bvec{r}}{\D t}\times\bvec{B} - \frac{\D\bvec{r}}{\D t}\times\bvec{B} \right)\right].
        \end{equation}
        \item Explain whether the terms in the round parentheses are symmetric in the two operators?
    \end{enumerate}
\end{problem}

\end{CJK}
\end{document}