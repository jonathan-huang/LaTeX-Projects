\documentclass[11pt, twoside]{article}

\usepackage{geometry}
    \geometry{
        a4paper,
        total={170mm,257mm},
        left=15mm,
        top=20mm,
    }
\usepackage{type1cm, titlesec, fancyhdr, titling}
\usepackage{multicol}
\usepackage[dvipsnames]{xcolor}
\usepackage{ulem}
\usepackage{parskip}
\setlength{\parindent}{2em}
\setlength{\headheight}{15pt}

\pagestyle{fancy}
\lhead{Jonathan Huang}
\rhead{PHYS 359 Statistical Mechanics}

\setlength{\droptitle}{-1.5cm}
\parindent=24pt
%% Math and Symbols %%
\usepackage{amsmath,amsthm,amssymb, mathtools}
\usepackage{yhmath, faktor, dsfont}
\usepackage{academicons, wasysym, marvosym}
\usepackage[scr]{rsfso} 
\usepackage{latexsym, amsmath, amscd, amsmath, amsthm}
\usepackage{amssymb,amsmath,amsthm,graphicx,dsfont}
\usepackage{hyperref}

%% Enhancement %%
\usepackage{graphicx, tabularx}
\usepackage[shortlabels,inline]{enumitem}
%% TikZ %%
\usepackage{tikz-cd}
\usepackage[breakable]{tcolorbox}
\usetikzlibrary{decorations.pathmorphing}
\usetikzlibrary{calc, arrows,matrix}

%% Other packages %%
\usepackage{amsopn}

%% Traditional Chinese %%
\usepackage{CJKutf8}

%% Math environments %%
\newtheoremstyle{mystyle}
  {6pt}{15pt}% 上下間距
  {}%          內文字體
  {}%              縮排
  {\bf}%       標頭字體
  {.}%       標頭後標點
  {1em}% 內文與標頭距離
  {}% Theorem head spec (can be left empty, meaning 'normal')
\theoremstyle{mystyle}	
\newtheorem{theorem}{Theorem}
\newtheorem{definition}{Definition}
\newtheorem{example}[theorem]{Example}
\newtheorem{exercise}{Exercise}
\newtheorem{solution}{Solution}
\newtheorem{corollary}[theorem]{Corollary}
\newtheorem{property}[theorem]{Property}
\newtheorem{proposition}[theorem]{Proposition}
\newtheorem{lemma}{Lemma}
\newtheorem{problem}[theorem]{Problem}
\newtheorem{answer}{Answer}[section]
\newtheorem{fact}[theorem]{fact}
\newtheorem*{claim}{Claim}
\newtheorem*{observation}{Observation}

\newenvironment{exerciseManual}[1]{%
  \renewcommand{\theexercise}{#1}%
  \begin{exercise}%
  \addtocounter{exercise}{-1}%
}{%
  \end{exercise}%
}

\newenvironment{solutionManual}[1]{%
  \renewcommand{\thesolution}{#1}%
  \begin{solution}%
  \addtocounter{solution}{-1}%
}{%
  \end{solution}%
}

\theoremstyle{remark}
\newtheorem*{remark}{Remark}

\newcommand{\bvec}[1]{\mathbf{#1}} % vector

\begin{document}
\begin{CJK}{UTF8}{bkai}

\title{%
    \textbf{PHYS 359 Winter 2026 - Problem Set 1} \\
    \vspace{0.5cm}
    \large 
    Due Jan ?, 2026\\
    \vspace{0.1cm}
    Instructor: Professor Alan Jamison
}
\author{Jonathan Huang 21232848 MATH}

\maketitle

% 
\newpage
\begin{exercise}[Biophysics]
 
    We consider a simplified model of a polymer as a chain existing in two dimensions. Suppose one end of a long polymer (e.g., DNA) is fixed to the origin, $(x,y)=(0,0)$. The first monomer of the polymer points in one of four directions: up, down, left, or right. The second monomer attaches at the end of the first. The attachment point bends, such that the second monomer can also point in any of these four directions. This continues to the $N$th monomer, which is the end of the polymer.

    \begin{enumerate}[(a)]
        \item \textbf{(20)} Find a formula for the multiplicity $\Omega(x,y)$ of a polymer that ends at point $(x,y)$, where the $x$- and $y$-coordinates are measured in units of monomers (i.e., the first monomer can end at any point in the set $\{(0,1),(0,-1),(-1,0),(1,0)\}$).

        \item \textbf{(30)} Find the point $(x,y)_{\max}$ of maximum multiplicity.
    \end{enumerate}
\end{exercise}

\begin{solution}
    ~

    \begin{enumerate}[(a)]
        \item There are in total $ N \gg 1 $ monomers in the polymer chain. Let $ N_x^+ $ be the number of monomers pointing in the positive $ x $-direction, $ N_x^- $ be the number of monomers pointing in the negative $ x $-direction, and similarly $ N_y^+ $, $ N_y^- $ for the $ y $ direction. Then we have the following relations:
        \begin{equation}
            N = N_x^+ + N_x^- + N_y^+ + N_y^-, 
        \end{equation}
        \begin{equation}
            x = N_x^+ - N_x^-, \quad y = N_y^+ - N_y^-.
        \end{equation}
        The total number of steps in the $ x $-direction is $ N_x = N_x^+ + N_x^- $, and similarly for the $ y $-direction, $ N_y = N_y^+ + N_y^- $. Thus, we have
        \begin{equation}
            N = N_x + N_y.
        \end{equation}

        For a fixed number of steps $ N_x $ in the $ x $ direction, we have 
        \begin{equation}
            N_x^+ = \frac{N_x + x}{2}, \quad N_x^- = \frac{N_x - x}{2},
        \end{equation}
        and 
        \begin{equation}
            N_y^+ = \frac{N_y + y}{2} = \frac{N - N_x + y}{2}, \quad N_y^- = \frac{N - N_x - y}{2}.
        \end{equation}

        Hence, the multiplicity for a fixed $ N_x $ is given by
        \begin{equation}
            \Omega(N_x) = \frac{N!}{N_x^+! N_x^-! N_y^+! N_y^-!} = \frac{N!}{\left(\frac{N_x + x}{2}\right)! \left(\frac{N_x - x}{2}\right)! \left(\frac{N - N_x + y}{2}\right)! \left(\frac{N - N_x - y}{2}\right)!}.
        \end{equation}
        To get the total multiplicity $ \Omega(x,y) $, we sum over all possible values of $ N_x $:
        \begin{equation}
            \Omega(x,y) = \sum_{N_x = 0}^N \Omega(N_x) = \sum_{N_x = 0}^N \frac{N!}{\left(\frac{N_x + x}{2}\right)! \left(\frac{N_x - x}{2}\right)! \left(\frac{N - N_x + y}{2}\right)! \left(\frac{N - N_x - y}{2}\right)!}.
        \end{equation}
        This may be rearranged to give 
        \begin{equation}
            \Omega (x,y) = \sum_{N_x = 0}^N \binom{N}{N_x} \binom{N_x}{\frac{N_x + x}{2}} \binom{N - N_x}{\frac{N - N_x + y}{2}}, 
        \end{equation}
        with the additional convention that the binomial coefficients are zero if their arguments are not non-negative integers.

        \begin{remark}
            Notice that the final formula makes sense, since the number of ways to reach $ (x,y) $ can be counted by first choosing how many steps $ N_x $ to take in the $ x $-direction, then choosing how many of those steps point positively to get to $ x $, and finally choosing how many of the remaining steps point positively to get to $ y $.
        \end{remark}

        \item Intuitively, the point of maximum multiplicity should be at the origin $ (0,0) $, since there are more ways to return to the origin than to reach any other point. To show this more rigorously, let's consider again Stirling's formula in the limit of large $ N $ as we did in class. 
        \begin{equation}
            \begin{split}
                \Omega (x,y) &= \sum_{N_x = 0}^N \binom{N}{N_x} \binom{N_x}{\frac{N_x + x}{2}} \binom{N - N_x}{\frac{N - N_x + y}{2}} \\
                &\approx \sum_{N_x = 0}^N \frac{N^{N} e^{-N}}{N_x^{N_x} e^{-N_x} \left(N - N_x\right)^{N - N_x} e^{-(N - N_x)}} \\
                &\quad \times \frac{N_x^{N_x} e^{-N_x}}{\left(\frac{N_x + x}{2}\right)^{\frac{N_x + x}{2}} e^{-\frac{N_x + x}{2}} \left(\frac{N_x - x}{2}\right)^{\frac{N_x - x}{2}} e^{-\frac{N_x - x}{2}}} \\
                &\quad \times \frac{(N - N_x)^{N - N_x} e^{-(N - N_x)}}{\left(\frac{N - N_x + y}{2}\right)^{\frac{N - N_x + y}{2}} e^{-\frac{N - N_x + y}{2}} \left(\frac{N - N_x - y}{2}\right)^{\frac{N - N_x - y}{2}} e^{-\frac{N - N_x - y}{2}}} \\
                &= 2^N \sum_{N_x = 0}^N \frac{N^N}{(N_x + x)^{\frac{N_x + x}{2}} (N_x - x)^{\frac{N_x - x}{2}} (N - N_x + y)^{\frac{N - N_x + y}{2}} (N - N_x - y)^{\frac{N - N_x - y}{2}}}. 
            \end{split}
        \end{equation}
        Therefore, we have 
        \begin{equation}
            \Omega (x,y) = \sum_{N_x = 0}^N \left(\frac{2N}{N_x + x}\right)^{\frac{N_x + x}{2}} \left(\frac{2N}{N_x - x}\right)^{\frac{N_x - x}{2}} \left(\frac{2N}{N - N_x + y}\right)^{\frac{N - N_x + y}{2}} \left(\frac{2N}{N - N_x - y}\right)^{\frac{N - N_x - y}{2}}.
        \end{equation}
        Since $ \ln $ is monotonic and concave, we will maximize each term $ \Omega_m $ in $ \Omega (x,y) $ for fixed $ m = N_x $ with respect to $ x $ and $ y $, by maximizing $ \ln \Omega_m (x,y) $. We have
        \begin{equation}
          \begin{split}
              \ln \Omega_m (x,y) &= \left(\frac{m+x}{2}\right) \ln \left(\frac{2N}{m+x}\right) + \left(\frac{m - x}{2}\right)\ln \left(\frac{2N}{m - x}\right) \\
              &\quad + \left(\frac{N - m + y}{2}\right)\ln \left(\frac{2N}{N - m + y}\right) + \left(\frac{N - m - y}{2}\right)\ln \left(\frac{2N}{N - m - y}\right).
          \end{split}
        \end{equation}
        The partial derivatives are then 
        \begin{equation}
            \begin{split}
                \frac{\partial \ln \Omega_m}{\partial x} &= \frac{1}{2} \ln \left(\frac{2N}{m+x}\right) - \frac{1}{2} - \frac{1}{2} \ln \left(\frac{2N}{m - x}\right) + \frac{1}{2} \\
                &= \frac{1}{2} \ln \left(\frac{m - x}{m + x}\right), \\
                \frac{\partial \ln \Omega_m}{\partial y} &= \frac{1}{2} \ln \left(\frac{N - m - y}{N - m + y}\right).
            \end{split}
        \end{equation}
        Setting these partial derivatives to zero, we find that $ x = 0 $ and $ y = 0 $ maximize each term $ \Omega_m $ in the sum for $ \Omega (x,y) $, and hence maximize $ \Omega (x,y) $ itself. Thus, the point of maximum multiplicity for $ N \gg 1 $ and $ N $ an even number is
        \begin{equation}
            (x,y)_{\max} = (0,0).
        \end{equation}

        However, when $ N $ is odd, the polymer cannot end at the origin. Notice that $ \binom{n}{m} $ attains a maximum when $ m \approx \frac{n}{2} $, and decreases monotonically away from $ m = \lfloor\frac{n}{2}\rfloor $. Therefore,
        \begin{equation}
            \Omega_m (x,y) = \binom{N}{m} \binom{m}{\frac{m+x}{2}} \binom{N - m}{\frac{N - m + y}{2}}
        \end{equation}
        is maximized when 
        \begin{equation}
            \frac{m + x}{2} \approx \frac{m}{2} \implies x \approx 0, \quad \frac{N - m + y}{2} \approx \frac{N - m}{2} \implies y \approx 0.
        \end{equation}
        When $ N $ is odd, the maximum multiplicity then occurs at the four points closest to $ (0,0) $, which are $ (1,0), (-1,0), (0,1), (0,-1) $.
    \end{enumerate}
\end{solution}

%
\newpage
\begin{exercise}[Cosmology]
    The early universe, much less than a second after the big bang, had a temperature so high that
    \[
    kT > m_N c^2,
    \]
    with $m_N$ the mass of a nucleon. ``Nucleon'' is a generic term for both protons and neutrons. Protons and neutrons have very similar masses and can transform one into the other through the weak nuclear force.

    As the universe continued to expand and cool below $T = m_N c^2/k$, the number of nucleons became fixed. However, neutrons and protons could still transform into one another through thermal excitation. Consider the nucleon as a two-state particle whose ground state is the proton and whose excited state, $1.29\,\mathrm{MeV}$ higher in energy, is the neutron.

    \begin{enumerate}[(a)]
        \item \textbf{(10)} What temperature corresponds to $kT = 1.29\,\mathrm{MeV}$?

        \item \textbf{(10)} A major transition in the structure of the universe happens about $200\,\mathrm{s}$ after the big bang, when the universe has cooled to $kT = 80\,\mathrm{keV}$. At this temperature, deuterium (${}^2\mathrm{H}$) is stable enough to allow fusion of deuterium nuclei into heavier nuclei, leading eventually to ${}^4\mathrm{He}$, which is an extremely stable nucleus. Assuming the nucleons are at equilibrium at this temperature, just before fusion begins to take place rapidly, what is the expected ratio of neutrons to protons in the universe?

        \item \textbf{(10)} The nucleons are not actually at equilibrium at $200\,\mathrm{s}$. Abruptly, around $3\,\mathrm{s}$ after the big bang, the density of the universe drops low enough that the reactions converting protons to neutrons can no longer proceed. Therefore, the neutron-to-proton ratio is frozen at the equilibrium value found at $3\,\mathrm{s}$ after the big bang. The ratio of neutrons to protons at this time is $1/6$. What is the temperature $T_3$ at this freezing-out time?
    \end{enumerate}
\end{exercise}

\begin{solution}
    ~

    \begin{enumerate}[(a)]
        \item The temperature is 
        \begin{equation}
            T = \frac{1.29 \, \text{MeV}}{k} = \frac{1.29 \times 10^6 \times \left(1.602 \times 10^{-19} \right) \, \text{J}}{1.38 \times 10^{-23} \, \text{J/K}} \approx 1.50 \times 10^{10} \, \text{K}. 
        \end{equation}

        \item The nucleons constitute a two-level system, so we can use Boltzmann statistics to find the ratio of neutrons to protons at equilibrium. In the large number limit, the ratio is just the ratio of their occuring probabilities, and thus 
        \begin{equation}
            \frac{N_{\text{n}}}{N_{\text{p}}} = \frac{P(\text{n})}{P(\text{p})} = \frac{e^{-\beta E_{\text{n}}}}{e^{-\beta E_{\text{p}}}} = e^{-\beta (E_{\text{n}} - E_{\text{p}})} = e^{- 1.29 \, \text{MeV} / (80 \, \text{keV})} = e^{-16.125} \approx 9.93 \times 10^{-8}.
        \end{equation}

        \item We set the ratio equal to $1/6$ and solve for $T_3$:
        \begin{equation}
            \frac{P(\text{n})}{P(\text{p})} = \frac{1}{6} = e^{- 1.29 \, \text{MeV} / (k T_3)}.
        \end{equation}
        Taking the natural logarithm of both sides, we have
        \begin{equation}
            -\ln(6) = - \frac{1.29 \, \text{MeV}}{k T_3},
        \end{equation}
        so rearranging gives
        \begin{equation}
            T_3 = \frac{1.29 \, \text{MeV}}{k \ln(6)} = \frac{1.29 \times 10^6 \times \left(1.602 \times 10^{-19} \right)}{1.38 \times 10^{-23} \times \ln(6)} \, \text{K} \approx 8.35 \times 10^{9} \, \text{K}.
        \end{equation}
    \end{enumerate}
\end{solution}

% 
\newpage
\begin{exercise}[Thermodynamics \textbf{(20)}]
    In lecture we developed expressions for $\overline{E}$ and $\overline{E^2}$ in terms of $Z$ and $\beta$. Use these and the definition of heat capacity,
    \[
    C = \frac{\partial \overline{E}}{\partial T},
    \]
    to derive the following relation:
    \[
    \sigma_E = kT\sqrt{\frac{C}{k}}.
    \]
\end{exercise}

\begin{solution}
    Recall the definition of the partition function as 
    \begin{equation}
        Z = \sum_i e^{-\beta E_i},
    \end{equation}
    and hence 
    \begin{equation}
        \overline{E} = \sum_{i} E_i \left(\frac{e^{- \beta E_i}}{Z} \right) = - \frac{1}{Z} \frac{\partial Z}{\partial \beta}, \qquad \overline{E^2} = \sum_{i} E_i^2 \left(\frac{e^{- \beta E_i}}{Z} \right) = \frac{1}{Z} \frac{\partial^2 Z}{\partial \beta^2}.
    \end{equation}

    The variance of the energy is given by
    \begin{equation}
      \begin{split}
          \sigma_E^2 &= \overline{E^2} - \overline{E}^2 \\
          &= \frac{1}{Z} \frac{\partial^2 Z}{\partial \beta^2} - \left( - \frac{1}{Z} \frac{\partial Z}{\partial \beta} \right)^2 \\
          &= \frac{1}{Z} \frac{\partial^2 Z}{\partial \beta^2} - \frac{1}{Z^2} \left( \frac{\partial Z}{\partial \beta} \right)^2 \\
          &= \frac{\partial}{\partial \beta} \left(\frac{\partial \ln Z}{\partial \beta} \right) = - \frac{\partial \overline{E}}{\partial \beta}.
      \end{split}
    \end{equation}
    By the chain rule, 
    \begin{equation}
        \frac{\partial}{\partial \beta} = \frac{\mathrm{d}T}{\mathrm{d}\beta} \frac{\partial}{\partial T} = \left(\frac{\mathrm{d}\beta}{\mathrm{d}T}\right)^{-1} \frac{\partial}{\partial T} = - k T^2 \frac{\partial}{\partial T},
    \end{equation}
    \begin{equation}
        - \frac{\partial \overline{E}}{\partial \beta} = k T^2 \frac{\partial \overline{E}}{\partial T} = k T^2 C.
    \end{equation}
    Therefore, the standard deviation of energy is
    \begin{equation}
        \sigma_E = \sqrt{\sigma_E^2} = \sqrt{k T^2 C} = k T \sqrt{\frac{C}{k}}.
    \end{equation}
\end{solution}

\end{CJK}
\end{document}