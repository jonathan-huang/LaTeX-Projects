\documentclass[12pt, A4, twoside]{article}
%\documentclass[a4paper]{article}

\usepackage{import}
\usepackage{xifthen}
\usepackage{pdfpages}
\usepackage{transparent}

\newcommand{\incfig}[1]{%
    \def\svgwidth{\columnwidth}
    \import{./Figures/}{#1.pdf_tex}
}

\usepackage{scrextend}
\usepackage{geometry}
    \geometry{
        a4paper,
        total={170mm,257mm},
        left=20mm,
        top=20mm,
    }

% table settings
\usepackage{multirow} 
\def\arraystretch{2} % 1 is the default, change whatever you need

\usepackage[version=4,arrows=pgf-filled,
textfontname=sffamily,
mathfontname=mathsf]{mhchem} % chemistry
\usepackage{amsmath}
\usepackage[shortlabels]{enumitem} % enumerate with letters
\usepackage[normalem]{ulem}
\usepackage{indentfirst}
\usepackage{pifont}
\usepackage{fancyhdr}   % 頁首頁尾
\usepackage{amssymb}

% Mandarin
\usepackage{CJKutf8}
% bkai = 標楷體
% bsmi = 新細明體

\usepackage{graphicx} % Required for inserting images

%
\usepackage{lmodern,bm}                
\usepackage[T1]{sansmath} 
\SetMathAlphabet{\mathsfbf}{sans}{\sansmathencoding}{\sfdefault}{bx}{sl}
\usepackage{etoolbox}
\AtBeginEnvironment{sansmath}{\let\bm\mathsfbf}{}{}
\usepackage{mdframed}
\usepackage{braket} % braket notation
\usepackage{amsmath, nccmath}
\usepackage{mathtools} % for text above and under arrows

\usepackage{bbm}
\usepackage{dsfont} % bold numerals
\usepackage{bbold} % blackboard bold font
\usepackage{mathrsfs} % script

\usepackage{url}

% set to equal
\newcommand{\defeq}{\vcentcolon=}
\newcommand{\eqdef}{=\vcentcolon}

% circled number
\newcommand*\circlednum[1]{\raisebox{.5pt}{\textcircled{\raisebox{-.9pt} {#1}}}}
\newcommand*\circled[1]{\tikz[baseline=(char.base)]{
            \node[shape=circle,draw,inner sep=1pt] (char) {#1};}}

% question mark styling
\newcommand{\?}{\stackrel{?}{=}}

\usepackage{relsize} % math symbol size
\usepackage{amsthm}

\newcommand{\daggerequ}{\mathrel{\overset{\makebox[0pt]{\mbox{\normalfont\tiny\sffamily $\dagger$}}}{=}}} % equal with dagger on top
\newcommand{\ddaggerequ}{\mathrel{\overset{\makebox[0pt]{\mbox{\normalfont\tiny\sffamily $\ddagger$}}}{=}}} % equal with ddagger on top

\newtheorem{theorem}{Theorem}[section]
\newtheorem{definition}{Definition}[section]
\newtheorem{lemma}[theorem]{Lemma}
\newtheorem{proposition}[theorem]{Proposition}
\newtheorem{corollary}[theorem]{Corollary}
%\newtheorem{property}{Property}

\newtheoremstyle{problemstyle}
        {5pt} % <space above>
        {15pt} % <space below>
        {\normalfont} % <body font>
        {} % <indent amount}
        {\bfseries} % <theorem head font>
        {\normalfont\bfseries.} % <punctuation after theorem head>
        {.5em} % <space after theorem head>
        {} % <theorem head spec (can be left empty, meaning `normal')>
\theoremstyle{problemstyle}
\newtheorem{problem}{Problem}[section]
\newtheorem{solution}{Solution}[section]
\newtheorem{example}{Example}[section]
\newtheorem{exercise}{Exercise}[section]

\theoremstyle{remark}
\newtheorem*{remark}{Remark}

\usepackage{mdframed}
\newenvironment{fthm}
    {\begin{mdframed}\begin{thm}}
    {\end{thm}\end{mdframed}}

\newcommand{\bvec}[1]{\mathbf{#1}} % vector

% figures
\usepackage{wrapfig} 
\usepackage{subfig}

% ckickable table of contents
\usepackage{hyperref}
\hypersetup{
    colorlinks=true,
    citecolor=black,
    filecolor=black,
    linkcolor=cyan,
    urlcolor=blue,
    pdftitle={notes on classical gravitation}, % opened pdf settings
    pdfpagemode=FullScreen,
}


\begin{document}
\begin{CJK}{UTF8}{bkai}

\title{
    UWaterloo 2026 Winter PHYS359 - Statistical Mechanics
    \\ 
    \vspace{1cm}
    \large Instructor: Professor \textbf{Alan} 
    \\ 
    \large Student: \textbf{Shao-Kai Jonathan Huang}    
}
% \author{Shao-Kai Jonathan Huang}
\date{\today}
\cfoot{\thepage}
% header and footer
\pagestyle{fancy}
\setlength{\headheight}{15pt}
\rhead{Shao-Kai Jonathan Huang}
\lhead{PHYS359 Statistical Mechanics}

\maketitle
\tableofcontents

\newpage

\section{Einstein Solid and Oscillators}
Lecture on 5/1/2026 by ().

Consider 

\section{Stirling's Approximation}
Lecture on 7/1/2026 by ().

\begin{definition}[Gamma Function]
    The Gamma function is defined as
    \begin{equation}
        \Gamma (n) = \int_0^{\infty} \mathrm{d}t\, t^{n-1} e^{-t}
    \end{equation}
    for \( n > 0 \).    
\end{definition}

The gamma function is related to the factorial function via $ \Gamma (n+1) = n! $ for integers $ n \geq 0 $. Consider 
\begin{equation}
    n! = \Gamma (n+1) = \int_0^{\infty} \mathrm{d}t\, t^n e^{-t}  = \int_0^{\infty} \mathrm{d}t\, e^{n \ln t - t}.
\end{equation}

Since the complexity of calculating the factorial function grows rapidly with \( n \), we look for an approximation for large \( n \). The integrand is sharply peaked around \( t = n \) for large \( n \) because the exponent \( n \log t - t \) attains its maximum there. 

\begin{proposition}[Stirling]
    For large $ n $, we have 
    \begin{equation}
        n! = \sqrt{2 \pi n} \left( \frac{n}{e} \right)^n \left(1 + \frac{1}{12n} + O\!\left(\frac{1}{n^2}\right)\right).
    \end{equation}
\end{proposition}

In asymptotic notation, this is written as $ n! \sim \sqrt{2 \pi n} e^{-n} n^n $. The exponent $ n \log t - t $ may be Taylor expanded about $ t = n $ and write the remainder terms as $ \dots $:
\begin{equation}
    \begin{split}
        n \log t - t &= n \log \left(n \cdot \frac{t}{n}\right) - t \\
        &= n \log n + n \log \left(1 + \frac{t-n}{n}\right) - t \\
        &= n \log n + n \left( \frac{t-n}{n} - \frac{1}{2} \left(\frac{t-n}{n}\right)^2 + \frac{1}{3} \left(\frac{t-n}{n}\right)^3 - \frac{1}{4} \left(\frac{t-n}{n}\right)^4 + \cdots \right) - t \\
        &= n \log n - n - \frac{1}{2n} (t-n)^2 + \frac{1}{3n^2} (t-n)^3 - \frac{1}{4n^3} (t-n)^4 + \cdots .
    \end{split}
\end{equation}

Then, the integral becomes 
\[
    \begin{split}
        n! &= \int_0^{\infty} \mathrm{d}t\, e^{n \log n - n} e^{-\frac{1}{2n}(t-n)^2} e^{\frac{1}{3n^2}(t-n)^3} e^{-\frac{1}{4n^3}(t-n)^4} \cdots \\
        &= n^n e^{-n} \int_0^{\infty} \mathrm{d}t\, e^{-\frac{1}{2n}(t-n)^2} e^{\frac{1}{3n^2}(t-n)^3} e^{-\frac{1}{4n^3}(t-n)^4} \cdots \\
        &= n^n e^{-n} \int_0^{\infty} \mathrm{d}t\, e^{-\frac{1}{2n}(t-n)^2} \left( 1 + \frac{1}{3n^2}(t-n)^3 - \frac{1}{4n^3}(t-n)^4 + \cdots \right) \\
        &= n^n e^{-n} \left\{ \int_0^{\infty} \mathrm{d}t\, e^{-\frac{1}{2n}(t-n)^2} + \frac{1}{3n^2} \int_0^{\infty} \mathrm{d}t\, e^{-\frac{1}{2n}(t-n)^2}(t-n)^3 - \frac{1}{4n^3} \int_0^{\infty} \mathrm{d}t\, (t-n)^4 e^{-\frac{1}{2n}(t-n)^2} + \cdots \right\} \\
        &\equiv n^n e^{-n} \left(I_1 + I_2 + I_3 + \cdots\right) .
    \end{split}
\]
Since most of the Gaussian is centered about $ t=n>0 $, we may extend the lower limits of integration to $ -\infty $. Then, $ I_1 = \sqrt{2 \pi n} $, $ I_2 = 0 $ since the integrand is odd, and $ I_3 $ may be evaluated as follows: 


\section{Entropy}
Lecture 3 on 14/1/2026 by ().

Entropy is a physical measure of disorder in a system. The Boltzmann entropy formula is given by
\begin{equation}
    S = k_{\text{B}} \ln \Omega,
\end{equation}
where $ \Omega $ is the number of microstates corresponding to a macrostate, and $ k_{\text{B}} \approx 1.38 \times 10^{-23} \, \text{J/K} $ is the Boltzmann constant.

\begin{example}[Einstein solid]
    The number of microstates in an Einstein solid consisting of $ N $ oscillators and $ q $ energy quanta is given by
    \begin{equation}
        \Omega (N,q) = \left(\frac{qe}{N}\right)^N
    \end{equation}
    for large $ N $ and $ q $. The entropy of the Einstein solid is therefore
    \begin{equation}
        S = k_{\text{B}} \ln \Omega = k_{\text{B}} N \left( \ln q - \ln N + 1 \right).
    \end{equation}
    Since $ q = E / \hbar \omega $ and $ E / N \propto k_\text{B} T $, the term $ qe / N $ in the logarithm is proportional to $ k_\text{B} T / \hbar \omega $.
\end{example}

"There is no dumb questions, but there is a dumb time to ask a question."


\section{Canonical Ensemble}
Consider a system $ \mathcal{A} $ in thermal equilibrium with a heat reservoir $ \mathcal{R} $ at (constant) temperature $ T $. If the temperature of $ \mathcal{A} $ is not equal to $ T $, then energy will flow between $ \mathcal{A} $ and $ \mathcal{R} $ until thermal equilibrium is reached. The combined system $ \mathcal{A} + \mathcal{R} $ is isolated, so its total energy $ E_{\text{tot}} = E_{\mathcal{A}} + E_{\mathcal{R}} $ is constant, and hence the total system $ \mathcal{U} $ satisfies
\begin{equation}
    \Omega_{\mathcal{U}} = \Omega_{\mathcal{A}} \Omega_{\mathcal{R}}.    
\end{equation}

For a fixed temperature, or microstate, of $ \mathcal{A} $, we can count the number of microstates of $ \mathcal{R} $ consistent with that microstate of $ \mathcal{A} $. That is, for microstate $ \mathcal{A}_1 $ of $ \mathcal{A} $, $ \Omega (\mathcal{A}_1) = \Omega_{\mathcal{R}}(\mathcal{A}_1) $, $ \Omega (\mathcal{A}_2) = \Omega_{\mathcal{R}} (\mathcal{A}_2) $, and so on. 

By the Fundamental Assumption, the probability of finding the system $ \mathcal{A} $ in microstate $ \mathcal{A}_i $ is proportional to the number of microstates of the reservoir consistent with that microstate of $ \mathcal{A} $:
\begin{equation}
    \frac{P(\mathcal{A}_i)}{P(\mathcal{A}_j)} = \frac{\Omega_{\mathcal{R}} (\mathcal{A}_i)}{\Omega_{\mathcal{R}} (\mathcal{A}_j)}. 
\end{equation}
Therefore, 
\begin{equation}
    \frac{P(\mathcal{A}_i)}{P(\mathcal{A}_j)} = \exp \left[\frac{1}{k}\left( S_{\mathcal{R}} (\mathcal{A}_i)- S_{\mathcal{R}} (\mathcal{A}_j)\right)\right]. 
\end{equation}
Recall that 
\begin{equation}
    \mathrm{d}U = T \mathrm{d}S - P \mathrm{d}V + \mu \mathrm{d}N,
\end{equation}
so that at constant volume and particle number, we have
\begin{equation}
    \Delta S = \int \frac{\mathrm{d}U}{T} = \frac{\Delta U}{T}, 
\end{equation}
since temperature is constant for the reservoir. Thus,
\begin{equation}
    \begin{split}
        \frac{P(\mathcal{A}_i)}{P(\mathcal{A}_j)} &= e^{\frac{1}{kT} \left[U_{\mathcal{R}} (\mathcal{A}_i)- U_{\mathcal{R}} (\mathcal{A}_j)\right]} = e^{-\frac{1}{kT} \left[U_{\mathcal{A}} (\mathcal{A}_i)- U_{\mathcal{A}} (\mathcal{A}_j)\right]} = \frac{e^{- \frac{1}{kT} U_{\mathcal{A}} (\mathcal{A}_i)}}{e^{- \frac{1}{kT} U_{\mathcal{A}} (\mathcal{A}_j)}}. 
    \end{split}
\end{equation}
The probability of finding the system $ \mathcal{A} $ in microstate $ \mathcal{A}_i $ is therefore proportional to $ e^{-\frac{1}{kT} U_{\mathcal{A}} (\mathcal{A}_i)} $, which is known as the \emph{Boltzmann factor}. 

\begin{example}[thermal excitation]
    The energy of an hydrogen atom in the $ n $th energy level is given by
    \begin{equation}
        E_n = - \frac{13.6}{n^2} \, \text{eV}.
    \end{equation}
    For a hydrogen atom at $ T=300\, \text{K} $, we have 
    \begin{equation}
        \frac{P(2\text{s})}{P(1\text{s})} = \frac{e^{- \beta E_{1\text{s}}}}{e^{- \beta E_{2\text{s}}}} = \frac{e^{- \beta \left(-3.4 \,\text{eV}\right)}}{e^{- \beta \left(-13.6 \,\text{eV}\right)}} = e^{- \beta \left(10.2 \, \text{eV}\right)} \approx 10^{-171},
    \end{equation}
    where we have used the notation $ \beta = \left(kT\right)^{-1} $, known as the \emph{thermodynamic beta} or \emph{coldness}. It is helpful to remember that $ kT \approx \frac{1}{40} \, \text{eV} $ at room temperature. On the surface of the sun, $ T_{sun}=6000  \, \text{K} = 20 T_{\text{RT}} $, we have 
    \begin{equation}
        \frac{P(2\text{s})}{P(1\text{s})} = e^{- \beta \left(10.2 \, \text{eV}\right)} \approx 2.8 \times 10^{-9}.
    \end{equation}

    Now the probability is just one in a billion, and we expect hydrogen 1s $ \to  $ 2s transitions to occur often on the surface of the sun due to thermal excitation.
\end{example}

\begin{quote}
    "...and I can't do this anymore. I don't mean teaching of course."
\end{quote}

\begin{quote}
    "Thank goodness the number is only e to the power of -395 and not 10 to the power of -408."
\end{quote}

\begin{quote}
    "How many hydrogen atoms are there in the sun? I dunno, a bunch?"
\end{quote}

It turns out that the Boltzmann factor is not the actual probabilty, as it is not normalized. Defining the \emph{partition function} $ Z $ as
\begin{equation}
    Z = \sum_i e^{-\frac{1}{kT} E_{\mathcal{A}} (\mathcal{A}_i)},
\end{equation}
the probability of finding the system $ \mathcal{A} $ in microstate $ \mathcal{A}_i $ is given by
\begin{equation}
    P(\mathcal{A}_i) = \frac{1}{Z} \exp \left[-\frac{1}{kT} E_{\mathcal{A}} (\mathcal{A}_i) \right].
\end{equation}
The partition function $ Z $ is a constant, in that it does not depend on the state. However, it still depends on $ \beta $, and hence the name partition \textit{function}. Moreover, shifting $E$ by a constant does not change the probability.

\begin{quote}
    "At some point I tried becoming a zed player. It didn't work. So will just call it zee, but you will hear zed being thrown around randomly here and there throughout the course."
\end{quote}

\begin{remark}
    When we are thinking microscopically, we tend to use $ E $, and when we are thinking macroscopically we tend to use $ U $.
\end{remark}

The probability of a microstate of a system in contact with a heat reservoir depends on the energy of the microstate. The equal likelihood assumption only applies to an isolated system. 


\section{Random Walk}
TA Class on 12/1/2026 by (). 

\medskip

Consider the following problem: 

\begin{example}[1D random walk]
    \noindent \textbf{Question:} A particle is confined to move on a line starting at the origin. The experimenter flips a fair coin $ N $ times and moves the particle one unit to the right for heads and one unit to the left for tails. What is the formula for $ \Omega (x) $ of the number of ways to reach position $ x $ after $ N $ steps? 

    \medskip 

    Let $ N_\text{R} $ and $ N_\text{L} $ be the number of steps to the right and left, respectively. Then, we have
    \begin{equation}
        N = N_\text{R} + N_\text{L}, \quad x = N_\text{R} - N_\text{L}.
    \end{equation}
    We can then solve for $ N_\text{R} $, and 
    \begin{equation}
        \Omega (x) = \binom{N}{N_\text{R}} = \frac{N!}{N_\text{R}!N_\text{L}!},
    \end{equation} 
    and in fact 
    \begin{equation}
        \Omega (x) = \binom{N}{\frac{N+x}{2}} = \frac{N!}{\left(\frac{N+x}{2}\right)! \left(\frac{N-x}{2}\right)!}.
    \end{equation}

    Let's find the first derivative of $ \Omega (x) $ with respect to $ x $. Suppose $ N \gg x $, i.e. $ N, (N+x)/2, (N-x)/2 \gg 1 $, then we can apply Stirling's approximation without the prefactor: $ n! \sim n^n e^{-n} $. Then,  
    \begin{equation}
        \begin{split}
            \Omega (x) &= \frac{N!}{\left(\frac{N+x}{2}\right)! \left(\frac{N-x}{2}\right)!} \\
            &\sim \frac{\left(\frac{N}{e}\right)^N}{\left(\frac{N+x}{2e}\right)^{\frac{N+x}{2}} \left(\frac{N-x}{2e}\right)^{\frac{N-x}{2}}} \\
            &= \frac{(2N)^N}{(N+x)^{\frac{N+x}{2}} (N-x)^{\frac{N-x}{2}}} \\
            &= \left(\frac{2N}{N+x}\right)^{\frac{N+x}{2}} \left(\frac{2N}{N-x}\right)^{\frac{N-x}{2}}. 
        \end{split}
    \end{equation}
    Since $ \ln $ is a monotonic function, we can find the extrema of $ \Omega (x) $ by finding the extrema of $ \ln \Omega (x) $. We have
    \begin{equation}
        \begin{split}
            \ln \Omega (x) &\sim \left(\frac{N+x}{2}\right) \ln \left(\frac{2N}{N+x}\right) + \left(\frac{N-x}{2}\right) \ln \left(\frac{2N}{N-x}\right) \\
            &= \left(\frac{N}{2}\right) \ln \left(4N^2\right) - \left(\frac{N}{2}\right) \ln \left[(N+x)(N-x)\right] + \left(\frac{x}{2}\right) \ln \left(\frac{N-x}{N+x}\right).
        \end{split}
    \end{equation}
    Then, 
    \begin{equation}
        \begin{split}
            \frac{\partial \ln \Omega}{\partial x} &\sim - \frac{Nx}{(N+x)(N-x)} + \frac{1}{2} \ln \left(\frac{N-x}{N+x}\right) - \frac{Nx}{(N+x)(N-x)} \\
            &= - \frac{2Nx}{(N+x)(N-x)} + \frac{1}{2} \ln \left(\frac{N-x}{N+x}\right) \\
            &= \frac{1}{2} \ln \left(\frac{N-x}{N+x}\right) + O\! \left(\frac{1}{N}\right). 
        \end{split}
    \end{equation}
    Setting this to zero, we find that the maximum occurs at $ x=0 $. Therefore, the most probable position after $ N $ steps ($ N\gg 1 $) is the origin.

    Let's set $ N-x = \delta \ll 1 $. Then, we have 
    
\end{example}


\end{CJK}
\end{document}